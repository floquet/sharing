% /Users/dantopa/primary-repos/bitbucket/fortran-alpha/rcs/writers/section/subsection.f08

% 2020-05-17  6:31:56

% \input{./sections/ssec elev=60-nu=15}

% keep the images on common page
\clearpage{}
\break{}

\subsection{$\nu = 15$ MHz, degree of fit $d = 18$}

\begin{table}[h]
    \begin{center}
        \begin{tabular}{ccc}
                %
            Data and solution & \quad & Residual errors \\\hline
                %
            \includegraphics[ scale = 0.45 ]{./graphics/0p060-nu=15-d=18-A} &&
            \includegraphics[ scale = 0.45 ]{./graphics/0p060-nu=15-d=18-B} \\[15pt]
                %
            Amplitudes with error bars && Amplitude signal--to--noise \\\hline
                %
            \includegraphics[ scale = 0.45 ]{./graphics/0p060-nu=15-d=18-C} &&
            \includegraphics[ scale = 0.45 ]{./graphics/0p060-nu=15-d=18-D} \\[15pt]
                %
            Error spectrum $2-$norm && Error spectrum $\infty-$norm \\\hline
                %
            \includegraphics[ scale = 0.45 ]{./graphics/0p060-nu=15-d=18-E} &&
            \includegraphics[ scale = 0.45 ]{./graphics/0p060-nu=15-d=18-F} \\[15pt]
                %
        \end{tabular}
    \end{center}
\label{fig:elev=60, nu=15}
\end{table}

% grab equation with amplitudes respecting significant digits
\input{/Users/dantopa/primary-repos/bitbucket/fortran-alpha/rcs/writers/subsection/amplitudes/elev=60-nu=15-d=18-clean.txt}

\endinput
