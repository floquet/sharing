%  \input{\pGlobalSetup packages-global-bibliography-alpha.tex}

% Purpose: Centralized configuration for bibliography actions and settings
% Attribution: Designed collaboratively by Daniel and Achates (GPT-4), November 2024
% References:
%   - Biblatex documentation: https://ctan.org/pkg/biblatex
%   - Biber documentation: https://ctan.org/pkg/biber
% Notes:
%   - This configuration centralizes bibliography management, allowing flexibility
%     for citation styles, sorting, and formatting adjustments.

% Load the biblatex package with desired options
% Options explained:
%   - style=numeric: Produces numeric citations like [1].
%   - backend=biber: Uses Biber as the backend for advanced processing.
%   - sorting=nyt: Sorts entries by Name, Year, and Title.
%   - maxnames=3: Limits the number of authors displayed; "et al." is used for longer lists.
%   - minnames=1: Ensures at least one author is displayed in shortened lists.
\usepackage[
    style=numeric,          
    backend=biber,          
    sorting=nyt,            
    maxnames=3,             
    minnames=1              
]{biblatex}

% Define the bibliography resource(s)
% Add the main bibliography file (in this case, tle.bib).
% Multiple files can be added here if needed.
\addbibresource{tle.bib}

% Optional customizations for bibliography formatting:
% Adjustments to spacing, hanging indents, and paragraph formatting.
%   - \bibhang: Controls the hanging indentation for items.
%   - \bibitemsep: Adjusts the space between items.
%   - \bibparsep: Adjusts the space between paragraphs in an item.
\setlength{\bibhang}{2em}       
\setlength{\bibitemsep}{0.5em}  
\setlength{\bibparsep}{0.2em}   

% Define a custom bibliography environment:
%   - Uses \list to structure the bibliography entries.
%   - Sets lengths for indentation and spacing.
\defbibenvironment{bibliography}
  {\list{} % Begin a list for bibliography formatting
    {\setlength{\leftmargin}{\bibhang}%
     \setlength{\itemindent}{-\leftmargin}%
     \setlength{\itemsep}{\bibitemsep}%
     \setlength{\parsep}{\bibparsep}}}
  {\endlist} % End the list
  {\item} % Define an item in the list
