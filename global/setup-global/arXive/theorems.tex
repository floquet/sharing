% ===========================================================
% Theorem Environments (theorems.tex)
% This file defines theorem styles and environments for the document.
% Include it in your main file with:
% \input{\pGlobalSetup theorems.tex}
% ===========================================================

% Load AMS packages for mathematical environments
% Required for theorem environments
\usepackage{amsmath, amssymb, amsthm}

% Define theorem styles
\theoremstyle{plain} % Default style for theorems
\newtheorem{theorem}{Theorem}
\newtheorem{proposition}[theorem]{Proposition}
\newtheorem{lemma}[theorem]{Lemma}
\newtheorem{corollary}[theorem]{Corollary}

\theoremstyle{definition} % Style for definitions
\newtheorem{definition}[theorem]{Definition}
\newtheorem{example}[theorem]{Example}

\theoremstyle{remark} % Style for remarks
\newtheorem{remark}[theorem]{Remark}
\newtheorem{note}[theorem]{Note}

% Usage Examples:
% \begin{theorem}[Pythagoras' Theorem]
% For a right triangle, the square of the hypotenuse is equal to the sum of the squares of the other two sides.
% \end{theorem}
%
% \begin{definition}[Group]
% A group is a set G with a binary operation * satisfying associativity, identity, and invertibility.
% \end{definition}

\endinput  % Prevent unintended content from being included
