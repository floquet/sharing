% ===========================================================
% Enumeration Customization with enumitem
% The `enumitem` package allows precise control over enumeration styles.
% This configuration supports both numeric (1, 1.1) and alphabetic (A, B) lists.
% ===========================================================
\usepackage{enumitem}

% === Numeric List Customization ===
% Default styles for numeric lists:
% Level 1: 1, 2, 3...
% Level 2: 1.1, 1.2, 1.3...
\setlist[enumerate,1]{label=\arabic*.} % Top-level: 1, 2, 3...
\setlist[enumerate,2]{label=\arabic{enumi}.\arabic*, ref=\arabic{enumi}.\arabic*} % Sub-level: 1.1, 1.2...

% === Alphabetic List Customization ===
% Custom styles for alphabetic lists:
% Level 1: A, B, C...
% Level 2: a), b), c)...
\setlist[enumerate,1]{label=\Alph*.} % Top-level: A, B, C...
\setlist[enumerate,2]{label=\alph*)} % Sub-level: a), b), c)...

% === Counter Adjustment (if needed) ===
% Reset counters to customize starting values
\setcounter{enumi}{1}   % Start enumerate level 1 from 1
\setcounter{enumii}{1}  % Start enumerate level 2 from a)

% https://tex.stackexchange.com/questions/55000/continuing-enumerate-counters-in-beamer
\newcounter{saveenumi}
\newcommand{\seti}{\setcounter{saveenumi}{\value{enumi}}}
\newcommand{\conti}{\setcounter{enumi}{\value{saveenumi}}}

% === Usage Examples ===
% Example for numeric list:
% \begin{enumerate}
%   \item First item
%   \item Second item
%   \begin{enumerate}
%     \item Sub-item 1.1
%     \item Sub-item 1.2
%   \end{enumerate}
% \end{enumerate}
%
% Example for alphabetic list:
% \begin{enumerate}[label=\Alph*.]
%   \item First item
%   \item Second item
%   \begin{enumerate}[label=\alph*)]
%     \item Sub-item a)
%     \item Sub-item b)
%   \end{enumerate}
% \end{enumerate}

% === Notes ===
% Use `[label=...]` in `\begin{enumerate}` to override defaults if needed.
% Examples are provided above to demonstrate both numeric and alphabetic lists.
