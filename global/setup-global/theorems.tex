% % ===========================================================
% Theorem Environments (theorems.tex)
% This file defines theorem styles and environments for the document.
% Include it in your main file with:
% \input{\pGlobalSetup theorems.tex}
% ===========================================================

% Load AMS packages for mathematical environments
% Required for theorem environments
\usepackage{amsmath, amssymb, amsthm}

% Define theorem styles
\theoremstyle{plain} % Default style for theorems
\newtheorem{theorem}{Theorem}
\newtheorem{proposition}[theorem]{Proposition}
\newtheorem{lemma}[theorem]{Lemma}
\newtheorem{corollary}[theorem]{Corollary}

\theoremstyle{definition} % Style for definitions
\newtheorem{definition}[theorem]{Definition}
\newtheorem{example}[theorem]{Example}

\theoremstyle{remark} % Style for remarks
\newtheorem{remark}[theorem]{Remark}
\newtheorem{note}[theorem]{Note}

% Usage Examples:
% \begin{theorem}[Pythagoras' Theorem]
% For a right triangle, the square of the hypotenuse is equal to the sum of the squares of the other two sides.
% \end{theorem}
%
% \begin{definition}[Group]
% A group is a set G with a binary operation * satisfying associativity, identity, and invertibility.
% \end{definition}

\endinput  % Prevent unintended content from being included


% File: theorems.tex
% Purpose: Unified theorem configuration for documents and presentations.
% Author: Achates, with guidance from Daniel Topa
% Sun Dec 22 21:33:42 MST 2024

% Ensure compatibility with amsthm package
\usepackage{amsthm}

% Define global theorem styles
% In Section 2, the numbering restarts, e.g., 2.1, 2.2.
% Environments like lemma and corollary share numbering with theorem because of [theorem].
\theoremstyle{plain} % Default style
\newtheorem{theorem}{Theorem}[section]
\newtheorem{lemma}[theorem]{Lemma}
\newtheorem{corollary}[theorem]{Corollary}

% Definition style
\theoremstyle{definition}
\newtheorem{definition}{Definition}[section]
\newtheorem{example}[definition]{Example}

% Remark style
\theoremstyle{remark}
\newtheorem{remark}{Remark}[section]

% Beamer-specific customization
\ifdefined\beamer@rendering
    % Customizations for splitting theorems in Beamer
    \newcommand*{\theorembreak}{%
        \usebeamertemplate{theorem end}%
        \framebreak%
        \usebeamertemplate{theorem begin}%
    }
\else
    % No-op for non-Beamer documents
    \newcommand*{\theorembreak}{}
\fi

% Usage:
% In Beamer, use \theorembreak to split a theorem across frames:
% \begin{theorem}
% This theorem spans multiple frames.
% \theorembreak
% Here is the continuation of the theorem.
% \end{theorem}

\endinput  %  -  -  -  -  -  -  -  -  -  -  -  -  -  -  -  -  -  -  -  -
