% \input{\pGlobalSetup listings-global.tex}

\newcommand{\listingFontSize}{\footnotesize} % Default font size for listings
% \renewcommand{\listingFontSize}{\scriptsize} % Override for this document
% size spectrum:
% \tiny
% \scriptsize
% \footnotesize
% \small
% \normalsize
% \large
% \Large
% \huge
% \Huge

%%%%%%%%%%%%%%%%%%%%%%%%%%%%%%%%%%%%%%%%%%%%%%%%%%%%%%%%%%%%%%%%%%%%%%%%%%%%%%%
% Listings Configuration File (listings-global.tex)
% ---------------------------------------------------------------------------
% Purpose:
% This file defines global styles for code listings in LaTeX documents,
% providing a unified and maintainable way to handle syntax highlighting,
% formatting, and customization for different programming languages and contexts.
%
% Usage:
% 1. Include this file in your LaTeX document preamble with:
%    % \input{\pGlobalSetup listings-global.tex}

\newcommand{\listingFontSize}{\footnotesize} % Default font size for listings
% \renewcommand{\listingFontSize}{\scriptsize} % Override for this document
% size spectrum:
% \tiny
% \scriptsize
% \footnotesize
% \small
% \normalsize
% \large
% \Large
% \huge
% \Huge

%%%%%%%%%%%%%%%%%%%%%%%%%%%%%%%%%%%%%%%%%%%%%%%%%%%%%%%%%%%%%%%%%%%%%%%%%%%%%%%
% Listings Configuration File (listings-global.tex)
% ---------------------------------------------------------------------------
% Purpose:
% This file defines global styles for code listings in LaTeX documents,
% providing a unified and maintainable way to handle syntax highlighting,
% formatting, and customization for different programming languages and contexts.
%
% Usage:
% 1. Include this file in your LaTeX document preamble with:
%    % \input{\pGlobalSetup listings-global.tex}

\newcommand{\listingFontSize}{\footnotesize} % Default font size for listings
% \renewcommand{\listingFontSize}{\scriptsize} % Override for this document
% size spectrum:
% \tiny
% \scriptsize
% \footnotesize
% \small
% \normalsize
% \large
% \Large
% \huge
% \Huge

%%%%%%%%%%%%%%%%%%%%%%%%%%%%%%%%%%%%%%%%%%%%%%%%%%%%%%%%%%%%%%%%%%%%%%%%%%%%%%%
% Listings Configuration File (listings-global.tex)
% ---------------------------------------------------------------------------
% Purpose:
% This file defines global styles for code listings in LaTeX documents,
% providing a unified and maintainable way to handle syntax highlighting,
% formatting, and customization for different programming languages and contexts.
%
% Usage:
% 1. Include this file in your LaTeX document preamble with:
%    % \input{\pGlobalSetup listings-global.tex}

\newcommand{\listingFontSize}{\footnotesize} % Default font size for listings
% \renewcommand{\listingFontSize}{\scriptsize} % Override for this document
% size spectrum:
% \tiny
% \scriptsize
% \footnotesize
% \small
% \normalsize
% \large
% \Large
% \huge
% \Huge

%%%%%%%%%%%%%%%%%%%%%%%%%%%%%%%%%%%%%%%%%%%%%%%%%%%%%%%%%%%%%%%%%%%%%%%%%%%%%%%
% Listings Configuration File (listings-global.tex)
% ---------------------------------------------------------------------------
% Purpose:
% This file defines global styles for code listings in LaTeX documents,
% providing a unified and maintainable way to handle syntax highlighting,
% formatting, and customization for different programming languages and contexts.
%
% Usage:
% 1. Include this file in your LaTeX document preamble with:
%    \input{path/to/listings-global.tex}
% 2. Use the defined styles in your `lstlisting` environments. For example:
%    \begin{lstlisting}[style=python, caption={Python example.}]
%    import numpy as np
%    a = np.array([1, 2, 3])
%    print(a)
%    \end{lstlisting}
%
% Customization:
% To globally control the font size of all listings, redefine the command:
%    \renewcommand{\listingFontSize}{\scriptsize}
% To customize styles for specific languages, modify the corresponding 
% `lstdefinestyle` blocks in this file.
%
% Examples:
% 1. For a terminal command:
%    \begin{lstlisting}[style=terminal, caption={Terminal example.}]
%    $ gfortran -shared -fPIC -o libmyfortran.so myfortran.f90
%    \end{lstlisting}
%
% 2. For Python code:
%    \begin{lstlisting}[style=python, caption={Python example.}]
%    import numpy as np
%    result = np.add([1, 2, 3], [4, 5, 6])
%    print(result)
%    \end{lstlisting}
%
% 3. For Fortran code:
%    \begin{lstlisting}[style=fortran, caption={Fortran example.}]
%    module mymodule
%    contains
%        subroutine mysubroutine
%            print *, "Hello, Fortran!"
%        end subroutine mysubroutine
%    end module mymodule
%    \end{lstlisting}
%
% Notes:
% - Ensure that `\usepackage{listings}` and `\usepackage{xcolor}` are loaded
%   in your main document preamble.
% - Adjust the styles and configurations in this file to suit your needs.
%%%%%%%%%%%%%%%%%%%%%%%%%%%%%%%%%%%%%%%%%%%%%%%%%%%%%%%%%%%%%%%%%%%%%%%%%%%%%%%

% lststyles.tex - Collection of listings styles for LaTeX
% This file was created with guidance and collaboration from Achates (ChatGPT by OpenAI) on 2024-11-26.

% \lstinputlisting[ style = terminal ]{output.txt}
% \lstinputlisting[ style = fortran ]{mycode.f90}



% Ensure this file can only be used within another LaTeX document
\ProvidesFile{lststyles.tex}[2024/11/26 Custom Listings Styles]

% Terminal-style listings
%\lstdefinestyle{terminal}{
%    language={Bash},                       % Treat as plain text (disable language-specific formatting)
%    backgroundcolor=\color{white},    % White background
%    basicstyle=\ttfamily\listingFontSize,          % Monospaced font, tiny size
%    breaklines=true,                  % Enable line wrapping
%    frame=single,                     % Box around the listing
%    tabsize=4,                        % Set tab width
%    showstringspaces=false,           % Hide space markers
%    numbers=left,                     % Show line numbers on the left
%    numberstyle=\tiny,                % Line number font size
%    numbersep=5pt,                    % Space between line numbers and text
%    keywordstyle=\color{black},       % No color for keywords (disable syntax highlighting)
%    commentstyle=\color{black},       % No color for comments
%    stringstyle=\color{black},        % No color for strings
%}
% \lstdefinestyle{basic}{
%     basicstyle=\ttfamily\footnotesize,
%     frame=single
% }

\lstdefinestyle{basic}{
    basicstyle=\ttfamily\footnotesize, % Monospaced font, small size
    frame=single,                      % Box around the listing
    breaklines=true,                   % Enable line wrapping
    numbers=none                       % No line numbers
}

\lstdefinestyle{terminal}{
    backgroundcolor=\color{black}, % Black background
    basicstyle=\color{white}\ttfamily\footnotesize, % White text, monospaced font, small size
    frame=single,                  % Box around the listing
    breaklines=true,               % Enable line wrapping
    numbers=none,                  % No line numbers
    tabsize=4                      % Tab size
}

% Fortran-style listings
%\lstdefinestyle{fortran}{
%    language=[90]Fortran,
%    basicstyle=\ttfamily\tiny,        % Monospaced font, tiny size
%    keywordstyle=\color{blue}\bfseries, % Keywords in blue and bold
%    commentstyle=\color{gray},        % Comments in gray
%    stringstyle=\color{green!50!black}, % Strings in green
%    numbers=left,                     % Line numbers on the left
%    numberstyle=\tiny\color{gray},    % Line number font size and color
%    stepnumber=1,                     % Number every line
%    numbersep=5pt,                    % Space between line numbers and text
%    backgroundcolor=\color{white},    % White background
%    showspaces=false,                 % Do not show spaces
%    showstringspaces=false,           % Do not show string spaces
%    showtabs=false,                   % Do not show tabs
%    frame=single,                     % Box around the listing
%    rulecolor=\color{black},          % Frame color
%    breaklines=true,                  % Enable line wrapping
%    breakatwhitespace=true,           % Break at whitespace
%    captionpos=b,                     % Caption position at the bottom
%    tabsize=4,                        % Tab size
%    escapeinside={\%*}{*)},           % Escape sequences for LaTeX
%}

% Fortran-style listings for Fortran 2023
% Highlighting features from Fortran 2003, 2008, 2018, and 2023.
\lstdefinestyle{fortran}{
    language=[90]Fortran,                % Base language is Fortran 90 for compatibility
    basicstyle=\ttfamily\listingFontSize,          % Monospaced font, tiny size
    keywordstyle=\color{blue}\bfseries, % Keywords in blue and bold
    commentstyle=\color{gray},          % Comments in gray
    stringstyle=\color{green!50!black}, % Strings in green
    numbers=left,                       % Line numbers on the left
    numberstyle=\tiny\color{gray},      % Line number font size and color
    stepnumber=1,                       % Number every line
    numbersep=5pt,                      % Space between line numbers and text
    backgroundcolor=\color{white},      % White background
    showspaces=false,                   % Do not show spaces
    showstringspaces=false,             % Do not show string spaces
    showtabs=false,                     % Do not show tabs
    frame=single,                       % Box around the listing
    rulecolor=\color{black},            % Frame color
    breaklines=true,                    % Enable line wrapping
    breakatwhitespace=true,             % Break at whitespace
    captionpos=b,                       % Caption position at the bottom
    tabsize=4,                          % Tab size
    escapeinside={\%*}{*)},             % Escape sequences for LaTeX
    morekeywords={abstract,associate,class,extends,final,import,procedure,select,block,where}, % Fortran 2003+ keywords
    morekeywords={error,critical,team_type,event_post,event_wait,fail_image,stop},             % Fortran 2018 keywords
    morekeywords={alloc_error,do_concurrent,assume,enumerator,bit,coarray,get_environment_variable}, % Fortran 2023 keywords
}

% Style for C++ code
\lstdefinestyle{cpp}{
    language=C++,
    basicstyle=\ttfamily\listingFontSize,
    keywordstyle=\color{blue}\bfseries,
    commentstyle=\color{gray},
    stringstyle=\color{green!50!black},
    numbers=left,
    numberstyle=\tiny\color{gray},
    stepnumber=1,
    numbersep=5pt,
    backgroundcolor=\color{white},
    showspaces=false,
    showstringspaces=false,
    showtabs=false,
    frame=single,
    rulecolor=\color{black},
    breaklines=true,
    breakatwhitespace=true,
    captionpos=b,
    tabsize=4,
    escapeinside={\%*}{*)},
}

% Terminal-style listings
%\lstdefinestyle{terminal}{
%    backgroundcolor=\color{white},    % White background
%    basicstyle=\ttfamily\footnotesize, % Monospaced font
%    breaklines=true,                  % Enable line wrapping
%    frame=single,                     % Box around the listing
%    tabsize=4,                        % Set tab width
%    showstringspaces=false,           % Hide space markers
%    numbers=none,                     % No line numbers
%    keywordstyle=\color{black},       % No color for keywords
%    commentstyle=\color{black},       % No color for comments
%    stringstyle=\color{black},        % No color for strings
%    language={}                       % Treat as plain text
%}

\lstdefinestyle{terminalCompact}{
    backgroundcolor=\color{black},  % Black background for terminal appearance
    basicstyle=\color{white}\ttfamily\listingFontSize, % Monospaced white text
    frame=single,                    % Box around the listing
    rulecolor=\color{gray},         % Gray border
    breaklines=true,                 % Enable line wrapping
    tabsize=4,                       % Tab size
    showstringspaces=false,          % No space markers
    showtabs=false,                  % No tab markers
    numbers=none,                    % No line numbers
}

% Bash-style listings
\lstdefinestyle{bash}{
    language=bash,                    % Enable Bash syntax
    basicstyle=\ttfamily\listingFontSize, % Monospaced font
    keywordstyle=\color{blue}\bfseries, % Keywords in blue and bold
    commentstyle=\color{green!50!black}, % Comments in green
    stringstyle=\color{red},          % Strings in red
    numbers=left,                     % Line numbers on the left
    numberstyle=\tiny\color{gray},    % Line number font size and color
    frame=single,                     % Box around the listing
    backgroundcolor=\color{white},    % White background
    breaklines=true,                  % Enable line wrapping
    tabsize=4                         % Tab size
}

\lstdefinestyle{python}{
    language=Python,                  % Use Python language definition
    basicstyle=\ttfamily\listingFontSize, % Monospaced font, smaller size
    keywordstyle=\color{blue}\bfseries, % Keywords in blue and bold
    commentstyle=\color{green!50!black}, % Comments in green
    stringstyle=\color{red},          % Strings in red
    numbers=left,                     % Line numbers on the left
    numberstyle=\tiny\color{gray},    % Line number font size and color
    stepnumber=1,                     % Number every line
    numbersep=5pt,                    % Space between line numbers and text
    frame=single,                     % Box around the listing
    rulecolor=\color{black},          % Frame color
    breaklines=true,                  % Enable line wrapping
    breakatwhitespace=true,           % Break at whitespace
    captionpos=b,                     % Caption position at the bottom
    tabsize=4                         % Tab size
}
% \lstset{style=fortran}
% Default style to apply globally (if needed)
% \lstset{style=terminal}


\endinput  %  ==  ==  ==  ==  ==  ==  ==  ==  ==

% 2. Use the defined styles in your `lstlisting` environments. For example:
%    \begin{lstlisting}[style=python, caption={Python example.}]
%    import numpy as np
%    a = np.array([1, 2, 3])
%    print(a)
%    \end{lstlisting}
%
% Customization:
% To globally control the font size of all listings, redefine the command:
%    \renewcommand{\listingFontSize}{\scriptsize}
% To customize styles for specific languages, modify the corresponding 
% `lstdefinestyle` blocks in this file.
%
% Examples:
% 1. For a terminal command:
%    \begin{lstlisting}[style=terminal, caption={Terminal example.}]
%    $ gfortran -shared -fPIC -o libmyfortran.so myfortran.f90
%    \end{lstlisting}
%
% 2. For Python code:
%    \begin{lstlisting}[style=python, caption={Python example.}]
%    import numpy as np
%    result = np.add([1, 2, 3], [4, 5, 6])
%    print(result)
%    \end{lstlisting}
%
% 3. For Fortran code:
%    \begin{lstlisting}[style=fortran, caption={Fortran example.}]
%    module mymodule
%    contains
%        subroutine mysubroutine
%            print *, "Hello, Fortran!"
%        end subroutine mysubroutine
%    end module mymodule
%    \end{lstlisting}
%
% Notes:
% - Ensure that `\usepackage{listings}` and `\usepackage{xcolor}` are loaded
%   in your main document preamble.
% - Adjust the styles and configurations in this file to suit your needs.
%%%%%%%%%%%%%%%%%%%%%%%%%%%%%%%%%%%%%%%%%%%%%%%%%%%%%%%%%%%%%%%%%%%%%%%%%%%%%%%

% lststyles.tex - Collection of listings styles for LaTeX
% This file was created with guidance and collaboration from Achates (ChatGPT by OpenAI) on 2024-11-26.

% \lstinputlisting[ style = terminal ]{output.txt}
% \lstinputlisting[ style = fortran ]{mycode.f90}



% Ensure this file can only be used within another LaTeX document
\ProvidesFile{lststyles.tex}[2024/11/26 Custom Listings Styles]

% Terminal-style listings
%\lstdefinestyle{terminal}{
%    language={Bash},                       % Treat as plain text (disable language-specific formatting)
%    backgroundcolor=\color{white},    % White background
%    basicstyle=\ttfamily\listingFontSize,          % Monospaced font, tiny size
%    breaklines=true,                  % Enable line wrapping
%    frame=single,                     % Box around the listing
%    tabsize=4,                        % Set tab width
%    showstringspaces=false,           % Hide space markers
%    numbers=left,                     % Show line numbers on the left
%    numberstyle=\tiny,                % Line number font size
%    numbersep=5pt,                    % Space between line numbers and text
%    keywordstyle=\color{black},       % No color for keywords (disable syntax highlighting)
%    commentstyle=\color{black},       % No color for comments
%    stringstyle=\color{black},        % No color for strings
%}
% \lstdefinestyle{basic}{
%     basicstyle=\ttfamily\footnotesize,
%     frame=single
% }

\lstdefinestyle{basic}{
    basicstyle=\ttfamily\footnotesize, % Monospaced font, small size
    frame=single,                      % Box around the listing
    breaklines=true,                   % Enable line wrapping
    numbers=none                       % No line numbers
}

\lstdefinestyle{terminal}{
    backgroundcolor=\color{black}, % Black background
    basicstyle=\color{white}\ttfamily\footnotesize, % White text, monospaced font, small size
    frame=single,                  % Box around the listing
    breaklines=true,               % Enable line wrapping
    numbers=none,                  % No line numbers
    tabsize=4                      % Tab size
}

% Fortran-style listings
%\lstdefinestyle{fortran}{
%    language=[90]Fortran,
%    basicstyle=\ttfamily\tiny,        % Monospaced font, tiny size
%    keywordstyle=\color{blue}\bfseries, % Keywords in blue and bold
%    commentstyle=\color{gray},        % Comments in gray
%    stringstyle=\color{green!50!black}, % Strings in green
%    numbers=left,                     % Line numbers on the left
%    numberstyle=\tiny\color{gray},    % Line number font size and color
%    stepnumber=1,                     % Number every line
%    numbersep=5pt,                    % Space between line numbers and text
%    backgroundcolor=\color{white},    % White background
%    showspaces=false,                 % Do not show spaces
%    showstringspaces=false,           % Do not show string spaces
%    showtabs=false,                   % Do not show tabs
%    frame=single,                     % Box around the listing
%    rulecolor=\color{black},          % Frame color
%    breaklines=true,                  % Enable line wrapping
%    breakatwhitespace=true,           % Break at whitespace
%    captionpos=b,                     % Caption position at the bottom
%    tabsize=4,                        % Tab size
%    escapeinside={\%*}{*)},           % Escape sequences for LaTeX
%}

% Fortran-style listings for Fortran 2023
% Highlighting features from Fortran 2003, 2008, 2018, and 2023.
\lstdefinestyle{fortran}{
    language=[90]Fortran,                % Base language is Fortran 90 for compatibility
    basicstyle=\ttfamily\listingFontSize,          % Monospaced font, tiny size
    keywordstyle=\color{blue}\bfseries, % Keywords in blue and bold
    commentstyle=\color{gray},          % Comments in gray
    stringstyle=\color{green!50!black}, % Strings in green
    numbers=left,                       % Line numbers on the left
    numberstyle=\tiny\color{gray},      % Line number font size and color
    stepnumber=1,                       % Number every line
    numbersep=5pt,                      % Space between line numbers and text
    backgroundcolor=\color{white},      % White background
    showspaces=false,                   % Do not show spaces
    showstringspaces=false,             % Do not show string spaces
    showtabs=false,                     % Do not show tabs
    frame=single,                       % Box around the listing
    rulecolor=\color{black},            % Frame color
    breaklines=true,                    % Enable line wrapping
    breakatwhitespace=true,             % Break at whitespace
    captionpos=b,                       % Caption position at the bottom
    tabsize=4,                          % Tab size
    escapeinside={\%*}{*)},             % Escape sequences for LaTeX
    morekeywords={abstract,associate,class,extends,final,import,procedure,select,block,where}, % Fortran 2003+ keywords
    morekeywords={error,critical,team_type,event_post,event_wait,fail_image,stop},             % Fortran 2018 keywords
    morekeywords={alloc_error,do_concurrent,assume,enumerator,bit,coarray,get_environment_variable}, % Fortran 2023 keywords
}

% Style for C++ code
\lstdefinestyle{cpp}{
    language=C++,
    basicstyle=\ttfamily\listingFontSize,
    keywordstyle=\color{blue}\bfseries,
    commentstyle=\color{gray},
    stringstyle=\color{green!50!black},
    numbers=left,
    numberstyle=\tiny\color{gray},
    stepnumber=1,
    numbersep=5pt,
    backgroundcolor=\color{white},
    showspaces=false,
    showstringspaces=false,
    showtabs=false,
    frame=single,
    rulecolor=\color{black},
    breaklines=true,
    breakatwhitespace=true,
    captionpos=b,
    tabsize=4,
    escapeinside={\%*}{*)},
}

% Terminal-style listings
%\lstdefinestyle{terminal}{
%    backgroundcolor=\color{white},    % White background
%    basicstyle=\ttfamily\footnotesize, % Monospaced font
%    breaklines=true,                  % Enable line wrapping
%    frame=single,                     % Box around the listing
%    tabsize=4,                        % Set tab width
%    showstringspaces=false,           % Hide space markers
%    numbers=none,                     % No line numbers
%    keywordstyle=\color{black},       % No color for keywords
%    commentstyle=\color{black},       % No color for comments
%    stringstyle=\color{black},        % No color for strings
%    language={}                       % Treat as plain text
%}

\lstdefinestyle{terminalCompact}{
    backgroundcolor=\color{black},  % Black background for terminal appearance
    basicstyle=\color{white}\ttfamily\listingFontSize, % Monospaced white text
    frame=single,                    % Box around the listing
    rulecolor=\color{gray},         % Gray border
    breaklines=true,                 % Enable line wrapping
    tabsize=4,                       % Tab size
    showstringspaces=false,          % No space markers
    showtabs=false,                  % No tab markers
    numbers=none,                    % No line numbers
}

% Bash-style listings
\lstdefinestyle{bash}{
    language=bash,                    % Enable Bash syntax
    basicstyle=\ttfamily\listingFontSize, % Monospaced font
    keywordstyle=\color{blue}\bfseries, % Keywords in blue and bold
    commentstyle=\color{green!50!black}, % Comments in green
    stringstyle=\color{red},          % Strings in red
    numbers=left,                     % Line numbers on the left
    numberstyle=\tiny\color{gray},    % Line number font size and color
    frame=single,                     % Box around the listing
    backgroundcolor=\color{white},    % White background
    breaklines=true,                  % Enable line wrapping
    tabsize=4                         % Tab size
}

\lstdefinestyle{python}{
    language=Python,                  % Use Python language definition
    basicstyle=\ttfamily\listingFontSize, % Monospaced font, smaller size
    keywordstyle=\color{blue}\bfseries, % Keywords in blue and bold
    commentstyle=\color{green!50!black}, % Comments in green
    stringstyle=\color{red},          % Strings in red
    numbers=left,                     % Line numbers on the left
    numberstyle=\tiny\color{gray},    % Line number font size and color
    stepnumber=1,                     % Number every line
    numbersep=5pt,                    % Space between line numbers and text
    frame=single,                     % Box around the listing
    rulecolor=\color{black},          % Frame color
    breaklines=true,                  % Enable line wrapping
    breakatwhitespace=true,           % Break at whitespace
    captionpos=b,                     % Caption position at the bottom
    tabsize=4                         % Tab size
}
% \lstset{style=fortran}
% Default style to apply globally (if needed)
% \lstset{style=terminal}


\endinput  %  ==  ==  ==  ==  ==  ==  ==  ==  ==

% 2. Use the defined styles in your `lstlisting` environments. For example:
%    \begin{lstlisting}[style=python, caption={Python example.}]
%    import numpy as np
%    a = np.array([1, 2, 3])
%    print(a)
%    \end{lstlisting}
%
% Customization:
% To globally control the font size of all listings, redefine the command:
%    \renewcommand{\listingFontSize}{\scriptsize}
% To customize styles for specific languages, modify the corresponding 
% `lstdefinestyle` blocks in this file.
%
% Examples:
% 1. For a terminal command:
%    \begin{lstlisting}[style=terminal, caption={Terminal example.}]
%    $ gfortran -shared -fPIC -o libmyfortran.so myfortran.f90
%    \end{lstlisting}
%
% 2. For Python code:
%    \begin{lstlisting}[style=python, caption={Python example.}]
%    import numpy as np
%    result = np.add([1, 2, 3], [4, 5, 6])
%    print(result)
%    \end{lstlisting}
%
% 3. For Fortran code:
%    \begin{lstlisting}[style=fortran, caption={Fortran example.}]
%    module mymodule
%    contains
%        subroutine mysubroutine
%            print *, "Hello, Fortran!"
%        end subroutine mysubroutine
%    end module mymodule
%    \end{lstlisting}
%
% Notes:
% - Ensure that `\usepackage{listings}` and `\usepackage{xcolor}` are loaded
%   in your main document preamble.
% - Adjust the styles and configurations in this file to suit your needs.
%%%%%%%%%%%%%%%%%%%%%%%%%%%%%%%%%%%%%%%%%%%%%%%%%%%%%%%%%%%%%%%%%%%%%%%%%%%%%%%

% lststyles.tex - Collection of listings styles for LaTeX
% This file was created with guidance and collaboration from Achates (ChatGPT by OpenAI) on 2024-11-26.

% \lstinputlisting[ style = terminal ]{output.txt}
% \lstinputlisting[ style = fortran ]{mycode.f90}



% Ensure this file can only be used within another LaTeX document
\ProvidesFile{lststyles.tex}[2024/11/26 Custom Listings Styles]

% Terminal-style listings
%\lstdefinestyle{terminal}{
%    language={Bash},                       % Treat as plain text (disable language-specific formatting)
%    backgroundcolor=\color{white},    % White background
%    basicstyle=\ttfamily\listingFontSize,          % Monospaced font, tiny size
%    breaklines=true,                  % Enable line wrapping
%    frame=single,                     % Box around the listing
%    tabsize=4,                        % Set tab width
%    showstringspaces=false,           % Hide space markers
%    numbers=left,                     % Show line numbers on the left
%    numberstyle=\tiny,                % Line number font size
%    numbersep=5pt,                    % Space between line numbers and text
%    keywordstyle=\color{black},       % No color for keywords (disable syntax highlighting)
%    commentstyle=\color{black},       % No color for comments
%    stringstyle=\color{black},        % No color for strings
%}
% \lstdefinestyle{basic}{
%     basicstyle=\ttfamily\footnotesize,
%     frame=single
% }

\lstdefinestyle{basic}{
    basicstyle=\ttfamily\footnotesize, % Monospaced font, small size
    frame=single,                      % Box around the listing
    breaklines=true,                   % Enable line wrapping
    numbers=none                       % No line numbers
}

\lstdefinestyle{terminal}{
    backgroundcolor=\color{black}, % Black background
    basicstyle=\color{white}\ttfamily\footnotesize, % White text, monospaced font, small size
    frame=single,                  % Box around the listing
    breaklines=true,               % Enable line wrapping
    numbers=none,                  % No line numbers
    tabsize=4                      % Tab size
}

% Fortran-style listings
%\lstdefinestyle{fortran}{
%    language=[90]Fortran,
%    basicstyle=\ttfamily\tiny,        % Monospaced font, tiny size
%    keywordstyle=\color{blue}\bfseries, % Keywords in blue and bold
%    commentstyle=\color{gray},        % Comments in gray
%    stringstyle=\color{green!50!black}, % Strings in green
%    numbers=left,                     % Line numbers on the left
%    numberstyle=\tiny\color{gray},    % Line number font size and color
%    stepnumber=1,                     % Number every line
%    numbersep=5pt,                    % Space between line numbers and text
%    backgroundcolor=\color{white},    % White background
%    showspaces=false,                 % Do not show spaces
%    showstringspaces=false,           % Do not show string spaces
%    showtabs=false,                   % Do not show tabs
%    frame=single,                     % Box around the listing
%    rulecolor=\color{black},          % Frame color
%    breaklines=true,                  % Enable line wrapping
%    breakatwhitespace=true,           % Break at whitespace
%    captionpos=b,                     % Caption position at the bottom
%    tabsize=4,                        % Tab size
%    escapeinside={\%*}{*)},           % Escape sequences for LaTeX
%}

% Fortran-style listings for Fortran 2023
% Highlighting features from Fortran 2003, 2008, 2018, and 2023.
\lstdefinestyle{fortran}{
    language=[90]Fortran,                % Base language is Fortran 90 for compatibility
    basicstyle=\ttfamily\listingFontSize,          % Monospaced font, tiny size
    keywordstyle=\color{blue}\bfseries, % Keywords in blue and bold
    commentstyle=\color{gray},          % Comments in gray
    stringstyle=\color{green!50!black}, % Strings in green
    numbers=left,                       % Line numbers on the left
    numberstyle=\tiny\color{gray},      % Line number font size and color
    stepnumber=1,                       % Number every line
    numbersep=5pt,                      % Space between line numbers and text
    backgroundcolor=\color{white},      % White background
    showspaces=false,                   % Do not show spaces
    showstringspaces=false,             % Do not show string spaces
    showtabs=false,                     % Do not show tabs
    frame=single,                       % Box around the listing
    rulecolor=\color{black},            % Frame color
    breaklines=true,                    % Enable line wrapping
    breakatwhitespace=true,             % Break at whitespace
    captionpos=b,                       % Caption position at the bottom
    tabsize=4,                          % Tab size
    escapeinside={\%*}{*)},             % Escape sequences for LaTeX
    morekeywords={abstract,associate,class,extends,final,import,procedure,select,block,where}, % Fortran 2003+ keywords
    morekeywords={error,critical,team_type,event_post,event_wait,fail_image,stop},             % Fortran 2018 keywords
    morekeywords={alloc_error,do_concurrent,assume,enumerator,bit,coarray,get_environment_variable}, % Fortran 2023 keywords
}

% Style for C++ code
\lstdefinestyle{cpp}{
    language=C++,
    basicstyle=\ttfamily\listingFontSize,
    keywordstyle=\color{blue}\bfseries,
    commentstyle=\color{gray},
    stringstyle=\color{green!50!black},
    numbers=left,
    numberstyle=\tiny\color{gray},
    stepnumber=1,
    numbersep=5pt,
    backgroundcolor=\color{white},
    showspaces=false,
    showstringspaces=false,
    showtabs=false,
    frame=single,
    rulecolor=\color{black},
    breaklines=true,
    breakatwhitespace=true,
    captionpos=b,
    tabsize=4,
    escapeinside={\%*}{*)},
}

% Terminal-style listings
%\lstdefinestyle{terminal}{
%    backgroundcolor=\color{white},    % White background
%    basicstyle=\ttfamily\footnotesize, % Monospaced font
%    breaklines=true,                  % Enable line wrapping
%    frame=single,                     % Box around the listing
%    tabsize=4,                        % Set tab width
%    showstringspaces=false,           % Hide space markers
%    numbers=none,                     % No line numbers
%    keywordstyle=\color{black},       % No color for keywords
%    commentstyle=\color{black},       % No color for comments
%    stringstyle=\color{black},        % No color for strings
%    language={}                       % Treat as plain text
%}

\lstdefinestyle{terminalCompact}{
    backgroundcolor=\color{black},  % Black background for terminal appearance
    basicstyle=\color{white}\ttfamily\listingFontSize, % Monospaced white text
    frame=single,                    % Box around the listing
    rulecolor=\color{gray},         % Gray border
    breaklines=true,                 % Enable line wrapping
    tabsize=4,                       % Tab size
    showstringspaces=false,          % No space markers
    showtabs=false,                  % No tab markers
    numbers=none,                    % No line numbers
}

% Bash-style listings
\lstdefinestyle{bash}{
    language=bash,                    % Enable Bash syntax
    basicstyle=\ttfamily\listingFontSize, % Monospaced font
    keywordstyle=\color{blue}\bfseries, % Keywords in blue and bold
    commentstyle=\color{green!50!black}, % Comments in green
    stringstyle=\color{red},          % Strings in red
    numbers=left,                     % Line numbers on the left
    numberstyle=\tiny\color{gray},    % Line number font size and color
    frame=single,                     % Box around the listing
    backgroundcolor=\color{white},    % White background
    breaklines=true,                  % Enable line wrapping
    tabsize=4                         % Tab size
}

\lstdefinestyle{python}{
    language=Python,                  % Use Python language definition
    basicstyle=\ttfamily\listingFontSize, % Monospaced font, smaller size
    keywordstyle=\color{blue}\bfseries, % Keywords in blue and bold
    commentstyle=\color{green!50!black}, % Comments in green
    stringstyle=\color{red},          % Strings in red
    numbers=left,                     % Line numbers on the left
    numberstyle=\tiny\color{gray},    % Line number font size and color
    stepnumber=1,                     % Number every line
    numbersep=5pt,                    % Space between line numbers and text
    frame=single,                     % Box around the listing
    rulecolor=\color{black},          % Frame color
    breaklines=true,                  % Enable line wrapping
    breakatwhitespace=true,           % Break at whitespace
    captionpos=b,                     % Caption position at the bottom
    tabsize=4                         % Tab size
}
% \lstset{style=fortran}
% Default style to apply globally (if needed)
% \lstset{style=terminal}


\endinput  %  ==  ==  ==  ==  ==  ==  ==  ==  ==

% 2. Use the defined styles in your `lstlisting` environments. For example:
%    \begin{lstlisting}[style=python, caption={Python example.}]
%    import numpy as np
%    a = np.array([1, 2, 3])
%    print(a)
%    \end{lstlisting}
%
% Customization:
% To globally control the font size of all listings, redefine the command:
%    \renewcommand{\listingFontSize}{\scriptsize}
% To customize styles for specific languages, modify the corresponding 
% `lstdefinestyle` blocks in this file.
%
% Examples:
% 1. For a terminal command:
%    \begin{lstlisting}[style=terminal, caption={Terminal example.}]
%    $ gfortran -shared -fPIC -o libmyfortran.so myfortran.f90
%    \end{lstlisting}
%
% 2. For Python code:
%    \begin{lstlisting}[style=python, caption={Python example.}]
%    import numpy as np
%    result = np.add([1, 2, 3], [4, 5, 6])
%    print(result)
%    \end{lstlisting}
%
% 3. For Fortran code:
%    \begin{lstlisting}[style=fortran, caption={Fortran example.}]
%    module mymodule
%    contains
%        subroutine mysubroutine
%            print *, "Hello, Fortran!"
%        end subroutine mysubroutine
%    end module mymodule
%    \end{lstlisting}
%
% Notes:
% - Ensure that `\usepackage{listings}` and `\usepackage{xcolor}` are loaded
%   in your main document preamble.
% - Adjust the styles and configurations in this file to suit your needs.
%%%%%%%%%%%%%%%%%%%%%%%%%%%%%%%%%%%%%%%%%%%%%%%%%%%%%%%%%%%%%%%%%%%%%%%%%%%%%%%

% lststyles.tex - Collection of listings styles for LaTeX
% This file was created with guidance and collaboration from Achates (ChatGPT by OpenAI) on 2024-11-26.

% \lstinputlisting[ style = terminal ]{output.txt}
% \lstinputlisting[ style = fortran ]{mycode.f90}



% Ensure this file can only be used within another LaTeX document
\ProvidesFile{lststyles.tex}[2024/11/26 Custom Listings Styles]

% Terminal-style listings
%\lstdefinestyle{terminal}{
%    language={Bash},                       % Treat as plain text (disable language-specific formatting)
%    backgroundcolor=\color{white},    % White background
%    basicstyle=\ttfamily\listingFontSize,          % Monospaced font, tiny size
%    breaklines=true,                  % Enable line wrapping
%    frame=single,                     % Box around the listing
%    tabsize=4,                        % Set tab width
%    showstringspaces=false,           % Hide space markers
%    numbers=left,                     % Show line numbers on the left
%    numberstyle=\tiny,                % Line number font size
%    numbersep=5pt,                    % Space between line numbers and text
%    keywordstyle=\color{black},       % No color for keywords (disable syntax highlighting)
%    commentstyle=\color{black},       % No color for comments
%    stringstyle=\color{black},        % No color for strings
%}
% \lstdefinestyle{basic}{
%     basicstyle=\ttfamily\footnotesize,
%     frame=single
% }

\lstdefinestyle{basic}{
    basicstyle=\ttfamily\footnotesize, % Monospaced font, small size
    frame=single,                      % Box around the listing
    breaklines=true,                   % Enable line wrapping
    numbers=none                       % No line numbers
}

\lstdefinestyle{terminal}{
    backgroundcolor=\color{black}, % Black background
    basicstyle=\color{white}\ttfamily\footnotesize, % White text, monospaced font, small size
    frame=single,                  % Box around the listing
    breaklines=true,               % Enable line wrapping
    numbers=none,                  % No line numbers
    tabsize=4                      % Tab size
}

% Fortran-style listings
%\lstdefinestyle{fortran}{
%    language=[90]Fortran,
%    basicstyle=\ttfamily\tiny,        % Monospaced font, tiny size
%    keywordstyle=\color{blue}\bfseries, % Keywords in blue and bold
%    commentstyle=\color{gray},        % Comments in gray
%    stringstyle=\color{green!50!black}, % Strings in green
%    numbers=left,                     % Line numbers on the left
%    numberstyle=\tiny\color{gray},    % Line number font size and color
%    stepnumber=1,                     % Number every line
%    numbersep=5pt,                    % Space between line numbers and text
%    backgroundcolor=\color{white},    % White background
%    showspaces=false,                 % Do not show spaces
%    showstringspaces=false,           % Do not show string spaces
%    showtabs=false,                   % Do not show tabs
%    frame=single,                     % Box around the listing
%    rulecolor=\color{black},          % Frame color
%    breaklines=true,                  % Enable line wrapping
%    breakatwhitespace=true,           % Break at whitespace
%    captionpos=b,                     % Caption position at the bottom
%    tabsize=4,                        % Tab size
%    escapeinside={\%*}{*)},           % Escape sequences for LaTeX
%}

% Fortran-style listings for Fortran 2023
% Highlighting features from Fortran 2003, 2008, 2018, and 2023.
\lstdefinestyle{fortran}{
    language=[90]Fortran,                % Base language is Fortran 90 for compatibility
    basicstyle=\ttfamily\listingFontSize,          % Monospaced font, tiny size
    keywordstyle=\color{blue}\bfseries, % Keywords in blue and bold
    commentstyle=\color{gray},          % Comments in gray
    stringstyle=\color{green!50!black}, % Strings in green
    numbers=left,                       % Line numbers on the left
    numberstyle=\tiny\color{gray},      % Line number font size and color
    stepnumber=1,                       % Number every line
    numbersep=5pt,                      % Space between line numbers and text
    backgroundcolor=\color{white},      % White background
    showspaces=false,                   % Do not show spaces
    showstringspaces=false,             % Do not show string spaces
    showtabs=false,                     % Do not show tabs
    frame=single,                       % Box around the listing
    rulecolor=\color{black},            % Frame color
    breaklines=true,                    % Enable line wrapping
    breakatwhitespace=true,             % Break at whitespace
    captionpos=b,                       % Caption position at the bottom
    tabsize=4,                          % Tab size
    escapeinside={\%*}{*)},             % Escape sequences for LaTeX
    morekeywords={abstract,associate,class,extends,final,import,procedure,select,block,where}, % Fortran 2003+ keywords
    morekeywords={error,critical,team_type,event_post,event_wait,fail_image,stop},             % Fortran 2018 keywords
    morekeywords={alloc_error,do_concurrent,assume,enumerator,bit,coarray,get_environment_variable}, % Fortran 2023 keywords
}

% Style for C++ code
\lstdefinestyle{cpp}{
    language=C++,
    basicstyle=\ttfamily\listingFontSize,
    keywordstyle=\color{blue}\bfseries,
    commentstyle=\color{gray},
    stringstyle=\color{green!50!black},
    numbers=left,
    numberstyle=\tiny\color{gray},
    stepnumber=1,
    numbersep=5pt,
    backgroundcolor=\color{white},
    showspaces=false,
    showstringspaces=false,
    showtabs=false,
    frame=single,
    rulecolor=\color{black},
    breaklines=true,
    breakatwhitespace=true,
    captionpos=b,
    tabsize=4,
    escapeinside={\%*}{*)},
}

% Terminal-style listings
%\lstdefinestyle{terminal}{
%    backgroundcolor=\color{white},    % White background
%    basicstyle=\ttfamily\footnotesize, % Monospaced font
%    breaklines=true,                  % Enable line wrapping
%    frame=single,                     % Box around the listing
%    tabsize=4,                        % Set tab width
%    showstringspaces=false,           % Hide space markers
%    numbers=none,                     % No line numbers
%    keywordstyle=\color{black},       % No color for keywords
%    commentstyle=\color{black},       % No color for comments
%    stringstyle=\color{black},        % No color for strings
%    language={}                       % Treat as plain text
%}

\lstdefinestyle{terminalCompact}{
    backgroundcolor=\color{black},  % Black background for terminal appearance
    basicstyle=\color{white}\ttfamily\listingFontSize, % Monospaced white text
    frame=single,                    % Box around the listing
    rulecolor=\color{gray},         % Gray border
    breaklines=true,                 % Enable line wrapping
    tabsize=4,                       % Tab size
    showstringspaces=false,          % No space markers
    showtabs=false,                  % No tab markers
    numbers=none,                    % No line numbers
}

% Bash-style listings
\lstdefinestyle{bash}{
    language=bash,                    % Enable Bash syntax
    basicstyle=\ttfamily\listingFontSize, % Monospaced font
    keywordstyle=\color{blue}\bfseries, % Keywords in blue and bold
    commentstyle=\color{green!50!black}, % Comments in green
    stringstyle=\color{red},          % Strings in red
    numbers=left,                     % Line numbers on the left
    numberstyle=\tiny\color{gray},    % Line number font size and color
    frame=single,                     % Box around the listing
    backgroundcolor=\color{white},    % White background
    breaklines=true,                  % Enable line wrapping
    tabsize=4                         % Tab size
}

\lstdefinestyle{python}{
    language=Python,                  % Use Python language definition
    basicstyle=\ttfamily\listingFontSize, % Monospaced font, smaller size
    keywordstyle=\color{blue}\bfseries, % Keywords in blue and bold
    commentstyle=\color{green!50!black}, % Comments in green
    stringstyle=\color{red},          % Strings in red
    numbers=left,                     % Line numbers on the left
    numberstyle=\tiny\color{gray},    % Line number font size and color
    stepnumber=1,                     % Number every line
    numbersep=5pt,                    % Space between line numbers and text
    frame=single,                     % Box around the listing
    rulecolor=\color{black},          % Frame color
    breaklines=true,                  % Enable line wrapping
    breakatwhitespace=true,           % Break at whitespace
    captionpos=b,                     % Caption position at the bottom
    tabsize=4                         % Tab size
}
% \lstset{style=fortran}
% Default style to apply globally (if needed)
% \lstset{style=terminal}


\endinput  %  ==  ==  ==  ==  ==  ==  ==  ==  ==
