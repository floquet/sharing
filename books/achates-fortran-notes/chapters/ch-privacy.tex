% input{../chapters/privacy.tex}
\chapter{Privacy and Procedure Control in Fortran}

\section{Introduction to Privacy}
In Fortran, managing access to module entities, such as procedures and variables, is essential for creating clean, maintainable code. The `public` and `private` attributes control visibility, allowing module authors to expose only the necessary components while keeping implementation details hidden.

By default, procedures in a module are \textbf{public}. This means they can be accessed from outside the module unless explicitly marked as `private`. On the other hand, you can change the default behavior to `private` using a single statement at the start of the module.

\section{Declaring \texttt{private} and \texttt{public}}
Here's an example of controlling access to procedures and variables in a module:

\begin{lstlisting}
module example
    implicit none
    private           ! All module entities are private by default
    public :: my_function

    contains

    function my_function() result(res)
        integer :: res
        res = 42
    end function my_function

    subroutine hidden_sub()
        ! This subroutine remains private
    end subroutine hidden_sub
end module example
\end{lstlisting}

\section{Procedure Aliasing and Abstraction}
Procedure aliasing allows you to define user-friendly names for internal procedures. For instance:

\begin{lstlisting}
module allocator
    implicit none
    public :: allocate_rank_one
    private :: allocate_one_sub

    contains

    procedure, public :: allocate_rank_one => allocate_one_sub

    subroutine allocate_one_sub()
        ! Implementation for allocating a rank-one array
    end subroutine allocate_one_sub
end module allocator
\end{lstlisting}

\subsection{Benefits of Procedure Aliasing}
This design offers the following advantages:
\begin{itemize}
    \item Encapsulation: External users only see the public name, hiding implementation details.
    \item Clarity: Names like \texttt{allocate\_rank\_one} describe the procedure's purpose, while internal names remain short and specific.
    \item Flexibility: Swap implementations without affecting external code.
\end{itemize}

\section{Using Generic Interfaces}
Combining procedure aliasing with generic interfaces allows you to design polymorphic and user-friendly APIs. Here's an example:

\begin{lstlisting}
module allocator
    implicit none
    public :: allocate
    private :: allocate_one_sub, allocate_two_sub

    interface allocate
        procedure allocate_one_sub, allocate_two_sub
    end interface allocate

    contains

    subroutine allocate_one_sub()
        ! Allocate a rank-one array
    end subroutine allocate_one_sub

    subroutine allocate_two_sub()
        ! Allocate a rank-two array
    end subroutine allocate_two_sub
end module allocator
\end{lstlisting}

\section{Best Practices}
\begin{itemize}
    \item Always use \texttt{private} at the top of a module to enforce encapsulation by default.
    \item Leverage \texttt{procedure aliasing} for clean interfaces and flexibility.
    \item Use generic interfaces to simplify user interaction with your modules.
    \item Document the purpose of each public entity to maintain clarity.
\end{itemize}

\endinput  %  -  -  -  -  -  -  -  -  -  -  -  -  -  -  -  -  -  -  -  -
