% input{../chapters/coarrays.tex}
\chapter{Object-Oriented Programming in Fortran}

\section{Object-Oriented Programming in Fortran: Type-Bound Procedures and Arrays}

Object-oriented programming (OOP) in Fortran allows for encapsulation and abstraction using derived types and type-bound procedures. This section discusses the concept of type-bound procedures and their application, particularly when working with arrays of derived-type objects.

\subsection{Type-Bound Procedures: A Primer}

In Fortran, type-bound procedures are subroutines or functions that are logically associated with a derived type. They enable the encapsulation of operations within the type itself, leading to better organization and clearer code. Type-bound procedures are declared in the \texttt{CONTAINS} block of a type definition.

For example, a simple \texttt{satellite} type with type-bound procedures can be defined as:
\begin{lstlisting}[language=Fortran]
type :: satellite
    integer :: index
contains
    procedure, public :: update_parameters => update_parameters_sub
end type satellite
\end{lstlisting}

Here, the \texttt{update\_parameters\_sub} subroutine is bound to the \texttt{satellite} type. It operates on an instance of \texttt{satellite}, referred to as \texttt{self}.

\subsection{Extending Operations to Arrays of Objects}

Often, there is a need to perform operations on an array of objects. In such cases, the relationship between the type and the procedure can be maintained in two ways:
\begin{itemize}
    \item Using a type-bound procedure that accepts an array of objects.
    \item Defining a standalone module-level procedure for array operations.
\end{itemize}

\subsubsection{Using a Type-Bound Procedure}

A type-bound procedure can be defined to operate on an array of the associated type:
\begin{lstlisting}[language=Fortran]
type :: satellite
    integer :: index
contains
    procedure, public :: update_all => update_all_satellites_sub
end type satellite

subroutine update_all_satellites_sub(satArray)
    class(satellite), dimension(:), intent(inout) :: satArray
    integer :: i
    do i = 1, size(satArray)
        satArray(i) % index = satArray(i) % index + 1
    end do
end subroutine update_all_satellites_sub
\end{lstlisting}

This approach retains encapsulation by tying the array-level operation to the type. The routine can be invoked using a proxy object:
\begin{lstlisting}[language=Fortran]
type(satellite) :: proxy
type(satellite), allocatable :: satelliteArray(:)

allocate(satelliteArray(5))
satelliteArray(:) = proxy

call proxy % update_all(satelliteArray)
\end{lstlisting}

\subsubsection{Using a Standalone Module-Level Procedure}

For operations that are more logically tied to arrays than to individual objects, a standalone module-level procedure is more appropriate:
\begin{lstlisting}[language=Fortran]
module satellite_module
    type :: satellite
        integer :: index
    end type satellite
contains
    subroutine update_satellite_array(satArray)
        type(satellite), dimension(:), intent(inout) :: satArray
        integer :: i
        do i = 1, size(satArray)
            satArray(i) % index = satArray(i) % index + 1
        end do
    end subroutine update_satellite_array
end module satellite_module
\end{lstlisting}

This method is invoked directly on the array:
\begin{lstlisting}[language=Fortran]
type(satellite), allocatable :: satelliteArray(:)

allocate(satelliteArray(5))

call update_satellite_array(satelliteArray)
\end{lstlisting}

\subsection{Blending Approaches for Flexibility}

To maximize flexibility, you can blend these two approaches. Define a type-bound procedure as a wrapper that delegates the work to a module-level procedure:
\begin{lstlisting}[language=Fortran]
type :: satellite
    integer :: index
contains
    procedure, public :: update_all => update_all_satellites_sub
end type satellite

subroutine update_all_satellites_sub(self, satArray)
    class(satellite), intent(in) :: self
    type(satellite), dimension(:), intent(inout) :: satArray
    call update_satellite_array(satArray)
end subroutine update_all_satellites_sub

subroutine update_satellite_array(satArray)
    type(satellite), dimension(:), intent(inout) :: satArray
    integer :: i
    do i = 1, size(satArray)
        satArray(i) % index = satArray(i) % index + 1
    end do
end subroutine update_satellite_array
\end{lstlisting}

\subsection{Guidelines for Choosing an Approach}

\begin{itemize}
    \item Use type-bound procedures for operations that are conceptually part of the type’s behavior.
    \item Use standalone procedures for operations that are independent of specific instances or require global context.
    \item Blend approaches when you need the flexibility to operate both through type-bound methods and standalone interfaces.
\end{itemize}

This dual approach ensures both encapsulation and reusability while providing a clean and logical design for object-oriented programming in Fortran.


\section{type vs. class}

Fortran provides two mechanisms for defining user-defined data types: \texttt{type} and \texttt{class}. While both are used to create structured data and associated behaviors, their capabilities and intended uses differ significantly.

\subsection{type: Static, Non-Polymorphic}
A \texttt{type} variable is bound to a specific derived type. It does not support polymorphism or dynamic dispatch, which means the type and behavior are fixed at compile time. This results in simpler and more efficient code.

\textbf{Use \texttt{type} when:}
\begin{itemize}
    \item Polymorphism is not required.
    \item Performance and simplicity are priorities.
    \item Fixed-functionality components are sufficient.
\end{itemize}

\begin{lstlisting}[caption={Example of type Usage}, label={lst:type_example}]
type :: point
    real :: x, y, z
end type

type(point) :: p
p%x = 1.0
p%y = 2.0
p%z = 3.0
\end{lstlisting}

\subsection{class: Dynamic, Polymorphic}
A \texttt{class} variable can hold an instance of its declared type or any type that extends it. This feature enables polymorphism and dynamic dispatch, allowing behavior to vary based on the actual type of the object at runtime.

\textbf{Use \texttt{class} when:}
\begin{itemize}
    \item Polymorphism is needed.
    \item Inheritance and type extension are required.
    \item You need dynamic dispatch or heterogeneous collections.
\end{itemize}

\begin{lstlisting}[caption={Example of class Usage}, label={lst:class_example}]
type :: particle
    real :: mass
contains
    procedure :: move
end type

type, extends(particle) :: charged_particle
    real :: charge
end type

class(particle), allocatable :: p
allocate(charged_particle :: p)
call p%move()  ! Runtime dispatch to `move` of `charged_particle`.
\end{lstlisting}

\subsection{Comparison of type and class}
The following table summarizes the differences between \texttt{type} and \texttt{class}:

\begin{table}[htbp]
\centering
\begin{tabular}{l l l}
\hline
\textbf{Feature}       & \textbf{type}                              & \textbf{class}                            \\ \hline
Polymorphism           & Not supported                             & Supported                                 \\
Dynamic Dispatch       & Not supported                             & Supported                                 \\
Type Safety            & Static type checking at compile time      & Runtime type checking for extensions      \\
Inheritance            & Cannot store extended types               & Can store base and extended types         \\
Performance            & Faster, less overhead                     & Slight runtime overhead                   \\
\hline
\end{tabular}
\caption{Comparison of type and class in Fortran}
\label{tab:type_class_comparison}
\end{table}

\subsection{Best Practices}
\begin{itemize}
    \item Start with \texttt{type} for simple, static data structures.
    \item Use \texttt{class} when your design requires polymorphism, inheritance, or dynamic dispatch.
    \item Opt for \texttt{class} if you need flexibility and maintainability in an object-oriented program.
    \item Consider performance trade-offs: \texttt{class} adds a small runtime overhead compared to \texttt{type}.
\end{itemize}

\subsection{Rule of Thumb}
\textbf{If you don’t need polymorphism, stick to \texttt{type}. Switch to \texttt{class} only when object-oriented features such as inheritance and dynamic dispatch become essential.}

\begin{center}
\textit{Fortran's \texttt{type} and \texttt{class} empower you to choose between static simplicity and dynamic flexibility—decide based on your program's needs.}
\end{center}


\section{Dynamic Dispatch}

Dynamic dispatch is a fundamental mechanism in object-oriented programming that enables the selection of the appropriate implementation of a polymorphic procedure at runtime. Unlike static dispatch, where the procedure is determined at compile time based on the declared type of an object, dynamic dispatch resolves the procedure based on the object's actual (dynamic) type during execution.

\subsection{Definition and Key Concepts}

\textbf{Dynamic Dispatch:} The mechanism that selects the implementation of a type-bound procedure at runtime based on the dynamic type of a polymorphic object.

\textbf{Key Components:}
\begin{itemize}
    \item \textbf{Polymorphism:} Dynamic dispatch requires polymorphism, where a variable can hold objects of its declared type or any derived type.
    \item \textbf{Type Hierarchy:} Objects belong to a hierarchy of types, starting with a base type and extending to derived types that override base functionality.
    \item \textbf{Type-Bound Procedures:} Procedures associated with a type, resolved dynamically when called on polymorphic objects.
    \item \textbf{Dispatch Mechanism:} The compiler generates a virtual table (vtable) to map each object's type to its corresponding procedures. This table is used at runtime to resolve calls.
\end{itemize}

\subsection{Dynamic Dispatch in Fortran}

Fortran supports dynamic dispatch through the \texttt{class} keyword and type-bound procedures. Polymorphic objects declared with \texttt{class} can hold instances of their declared type or any of its extensions. When a type-bound procedure is invoked, the actual procedure executed depends on the dynamic type of the object.

\begin{lstlisting}[caption={Example of Dynamic Dispatch in Fortran}, label={lst:dynamic_dispatch}]
module mParticles
    implicit none

    ! Base type: particle
    type :: particle
        real :: mass
    contains
        procedure :: move => move_particle  ! Type-bound procedure
    end type particle

    ! Derived type: charged_particle
    type, extends(particle) :: charged_particle
        real :: charge
    contains
        procedure :: move => move_charged_particle  ! Override base procedure
    end type charged_particle

contains

    ! Procedure for base type
    subroutine move_particle(self)
        class(particle), intent(inout) :: self
        print *, "Moving a generic particle with mass", self%mass
    end subroutine move_particle

    ! Procedure for derived type
    subroutine move_charged_particle(self)
        class(charged_particle), intent(inout) :: self
        print *, "Moving a charged particle with mass", self%mass, "and charge", self%charge
    end subroutine move_charged_particle

end module mParticles

program test_dispatch
    use mParticles
    implicit none

    class(particle), allocatable :: p  ! Polymorphic object

    ! Allocate base type
    allocate(particle :: p)
    p%mass = 1.0
    call p%move()  ! Calls move_particle

    ! Allocate derived type
    allocate(charged_particle :: p)
    p%mass = 1.5
    call p%move()  ! Calls move_charged_particle
end program test_dispatch
\end{lstlisting}

\subsection{Advantages of Dynamic Dispatch}
\begin{itemize}
    \item \textbf{Extensibility:} New derived types can be added without modifying existing code.
    \item \textbf{Runtime Flexibility:} Behavior depends on the actual type of an object, enabling more general and reusable designs.
\end{itemize}

\subsection{Comparison with Static Dispatch}

\begin{table}[htbp]
\centering
\begin{tabular}{l l l}
\hline
\textbf{Feature}        & \textbf{Static Dispatch}                & \textbf{Dynamic Dispatch}                \\ \hline
Resolution Time         & Compile-time                           & Runtime                                  \\
Procedure Selection     & Based on declared (static) type        & Based on actual (dynamic) type           \\
Flexibility             & Limited to compile-time knowledge      & Supports runtime type-dependent behavior \\
Performance             & Faster, no runtime lookup              & Slightly slower due to runtime overhead  \\ \hline
\end{tabular}
\caption{Comparison of Static and Dynamic Dispatch}
\label{tab:dispatch_comparison}
\end{table}

\subsection{Key Takeaways}
\begin{itemize}
    \item Use \textbf{dynamic dispatch} when the behavior of a procedure depends on the runtime type of an object.
    \item Polymorphism and type-bound procedures make dynamic dispatch a powerful tool for object-oriented design.
    \item Be aware of the slight runtime overhead associated with dynamic dispatch due to vtable lookups.
\end{itemize}

\begin{center}
\textit{Dynamic dispatch is a cornerstone of object-oriented programming, enabling flexible and extensible designs by resolving procedure calls at runtime.}
\end{center}


\endinput  %  -  -  -  -  -  -  -  -  -  -  -  -  -  -  -  -  -  -  -  -
