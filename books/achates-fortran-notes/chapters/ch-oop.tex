% input{../chapters/coarrays.tex}
\chapter{Object-Oriented Programming in Fortran}

\section{Object-Oriented Programming in Fortran: Type-Bound Procedures and Arrays}

Object-oriented programming (OOP) in Fortran allows for encapsulation and abstraction using derived types and type-bound procedures. This section discusses the concept of type-bound procedures and their application, particularly when working with arrays of derived-type objects.

\subsection{Type-Bound Procedures: A Primer}

In Fortran, type-bound procedures are subroutines or functions that are logically associated with a derived type. They enable the encapsulation of operations within the type itself, leading to better organization and clearer code. Type-bound procedures are declared in the \texttt{CONTAINS} block of a type definition.

For example, a simple \texttt{satellite} type with type-bound procedures can be defined as:
\begin{lstlisting}[language=Fortran]
type :: satellite
    integer :: index
contains
    procedure, public :: update_parameters => update_parameters_sub
end type satellite
\end{lstlisting}

Here, the \texttt{update\_parameters\_sub} subroutine is bound to the \texttt{satellite} type. It operates on an instance of \texttt{satellite}, referred to as \texttt{self}.

\subsection{Extending Operations to Arrays of Objects}

Often, there is a need to perform operations on an array of objects. In such cases, the relationship between the type and the procedure can be maintained in two ways:
\begin{itemize}
    \item Using a type-bound procedure that accepts an array of objects.
    \item Defining a standalone module-level procedure for array operations.
\end{itemize}

\subsubsection{Using a Type-Bound Procedure}

A type-bound procedure can be defined to operate on an array of the associated type:
\begin{lstlisting}[language=Fortran]
type :: satellite
    integer :: index
contains
    procedure, public :: update_all => update_all_satellites_sub
end type satellite

subroutine update_all_satellites_sub(satArray)
    class(satellite), dimension(:), intent(inout) :: satArray
    integer :: i
    do i = 1, size(satArray)
        satArray(i) % index = satArray(i) % index + 1
    end do
end subroutine update_all_satellites_sub
\end{lstlisting}

This approach retains encapsulation by tying the array-level operation to the type. The routine can be invoked using a proxy object:
\begin{lstlisting}[language=Fortran]
type(satellite) :: proxy
type(satellite), allocatable :: satelliteArray(:)

allocate(satelliteArray(5))
satelliteArray(:) = proxy

call proxy % update_all(satelliteArray)
\end{lstlisting}

\subsubsection{Using a Standalone Module-Level Procedure}

For operations that are more logically tied to arrays than to individual objects, a standalone module-level procedure is more appropriate:
\begin{lstlisting}[language=Fortran]
module satellite_module
    type :: satellite
        integer :: index
    end type satellite
contains
    subroutine update_satellite_array(satArray)
        type(satellite), dimension(:), intent(inout) :: satArray
        integer :: i
        do i = 1, size(satArray)
            satArray(i) % index = satArray(i) % index + 1
        end do
    end subroutine update_satellite_array
end module satellite_module
\end{lstlisting}

This method is invoked directly on the array:
\begin{lstlisting}[language=Fortran]
type(satellite), allocatable :: satelliteArray(:)

allocate(satelliteArray(5))

call update_satellite_array(satelliteArray)
\end{lstlisting}

\subsection{Blending Approaches for Flexibility}

To maximize flexibility, you can blend these two approaches. Define a type-bound procedure as a wrapper that delegates the work to a module-level procedure:
\begin{lstlisting}[language=Fortran]
type :: satellite
    integer :: index
contains
    procedure, public :: update_all => update_all_satellites_sub
end type satellite

subroutine update_all_satellites_sub(self, satArray)
    class(satellite), intent(in) :: self
    type(satellite), dimension(:), intent(inout) :: satArray
    call update_satellite_array(satArray)
end subroutine update_all_satellites_sub

subroutine update_satellite_array(satArray)
    type(satellite), dimension(:), intent(inout) :: satArray
    integer :: i
    do i = 1, size(satArray)
        satArray(i) % index = satArray(i) % index + 1
    end do
end subroutine update_satellite_array
\end{lstlisting}

\subsection{Guidelines for Choosing an Approach}

\begin{itemize}
    \item Use type-bound procedures for operations that are conceptually part of the type’s behavior.
    \item Use standalone procedures for operations that are independent of specific instances or require global context.
    \item Blend approaches when you need the flexibility to operate both through type-bound methods and standalone interfaces.
\end{itemize}

This dual approach ensures both encapsulation and reusability while providing a clean and logical design for object-oriented programming in Fortran.

\endinput  %  -  -  -  -  -  -  -  -  -  -  -  -  -  -  -  -  -  -  -  -
