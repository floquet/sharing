% input{../chapters/coarrays.tex}
\chapter{Coarrays in Fortran}

\section{Introduction to Coarrays}

Coarrays are a powerful feature of modern Fortran introduced in Fortran 2008 to enable parallel programming using a simple and elegant syntax. They allow variables to be shared across multiple execution images, each with its own local memory, enabling distributed memory parallelism.

Coarrays are designed to simplify parallel programming by abstracting the complexity of traditional message-passing interfaces while still offering fine-grained control over data distribution and synchronization.

\section{Key Concepts of Coarrays}

\subsection{Execution Images}
An \emph{image} is an independent instance of a program running as part of a parallel execution. Each image has its own memory but can communicate with others via coarrays. Think of images as lightweight processes or threads:
\begin{itemize}
    \item Each image executes the same program.
    \item Images are identified by unique indices ranging from 1 to the total number of images.
    \item Communication between images is explicit and controlled.
\end{itemize}

\subsection{Declaring Coarrays}
Coarrays are declared using square brackets to specify the codimension. Here’s an example of a simple coarray declaration:
\begin{lstlisting}
real :: x[*]
\end{lstlisting}
This declares a scalar real coarray \texttt{x}, distributed across all images. The \texttt{[*]} codimension specifies that each image has a separate copy of \texttt{x}.

For multidimensional arrays, both normal dimensions and codimensions can be specified:
\begin{lstlisting}
real :: matrix(10,10)[*]
\end{lstlisting}

\subsection{Accessing Coarray Data}
To access data on another image, use the square bracket syntax to specify the image index. For example:
\begin{lstlisting}
x[2] = 3.14  ! Assign 3.14 to x on image 2
y = x[3]     ! Retrieve the value of x from image 3
\end{lstlisting}

If no image index is specified, the operation occurs on the local image.

\subsection{Synchronization}
Synchronization is crucial in parallel programming to ensure data consistency across images. Fortran provides the following intrinsic procedures for synchronization:
\begin{itemize}
    \item \texttt{sync all}: Synchronize all images.
    \item \texttt{sync images}: Synchronize specific images.
    \item \texttt{sync memory}: Ensure memory consistency across images.
\end{itemize}

Example:
\begin{lstlisting}
sync all  ! Wait for all images to reach this point
\end{lstlisting}

\subsection{Teams and Subgroups}
Fortran 2018 introduced teams, allowing images to be grouped for collective operations. Teams enable finer control over parallelism by creating subsets of images:
\begin{lstlisting}
form team(team_number)
change team(team_number)
    ! Code executed within the team
end team
\end{lstlisting}

---

\section{Examples of Coarray Usage}

\subsection{Hello, World with Coarrays}
Here’s a simple program demonstrating coarrays:
\begin{lstlisting}
program hello_coarrays
    implicit none
    integer :: me, num_images

    me = this_image()          ! Get the current image index
    num_images = num_images()  ! Get the total number of images

    print *, "Hello from image ", me, " of ", num_images
end program hello_coarrays
\end{lstlisting}

Run this program with multiple images using an MPI-compatible Fortran compiler:
\begin{verbatim}
mpirun -np 4 ./hello_coarrays
\end{verbatim}

---

\subsection{Data Sharing Across Images}
This example demonstrates sharing data between images:
\begin{lstlisting}
program data_sharing
    implicit none
    integer :: me
    real :: shared_value[*]

    me = this_image()

    if (me == 1) then
        shared_value = 42.0  ! Assign a value on image 1
    end if

    sync all  ! Ensure all images are synchronized

    print *, "Image ", me, " sees shared_value = ", shared_value[1]
end program data_sharing
\end{lstlisting}

---

\section{Best Practices and Tips}
\begin{itemize}
    \item Use \texttt{sync all} and \texttt{sync images} judiciously to avoid unnecessary synchronization overhead.
    \item Minimize direct communication between images to reduce potential bottlenecks.
    \item Test coarray code on multiple configurations to ensure scalability.
\end{itemize}

\section{Advanced Features}
Fortran coarrays also support asynchronous operations and collective procedures such as \texttt{co\_sum}, \texttt{co\_min}, and \texttt{co\_max}, which operate across images efficiently.

Example of a collective sum:
\begin{lstlisting}
real :: sum_value[*], total_sum

sum_value = this_image()
total_sum = co_sum(sum_value)  ! Sum values across all images
if (this_image() == 1) then
    print *, "Total sum: ", total_sum
end if
\end{lstlisting}

---

\section{Conclusion}
Fortran coarrays provide a high-level, intuitive framework for parallel programming that integrates seamlessly with Fortran’s core features. They simplify data sharing, synchronization, and team-based operations while retaining control and efficiency. By leveraging coarrays, you can write scalable parallel applications with minimal overhead.

---

\endinput  %  -  -  -  -  -  -  -  -  -  -  -  -  -  -  -  -  -  -  -  -
