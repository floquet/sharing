\documentclass[12pt]{article}
\usepackage{amsmath, amssymb}
\usepackage{graphicx}
\usepackage{geometry}
\geometry{a4paper, margin=1in}

\title{Stitching Multiple Camera Frames into a Single Picture}
\author{Daniel M. Topa \\ WaveFront Sciences Inc.}
\date{}

\begin{document}
\maketitle

\section{The Physical Problem}

Consider a lenslet array with $\xi'$ columns and $\eta'$ rows. Each lenslet samples the incident wavefront $\psi(x, y)$ over a distinct region of space and produces a well-defined focal spot on the CCD array. The average of the $x$ and $y$ slopes of the wavefront over the lenslet is:
\[
\theta_{\mu\nu} = 
\begin{bmatrix}
\theta_{\mu\nu,x} \\
\theta_{\mu\nu,y}
\end{bmatrix},
\]
where:
\[
\theta_{\mu\nu,x} = \int_{\mu-1}^{\mu} \int_{\nu-1}^{\nu} \frac{\partial \psi}{\partial x} \, dx \, dy, \quad
\theta_{\mu\nu,y} = \int_{\mu-1}^{\mu} \int_{\nu-1}^{\nu} \frac{\partial \psi}{\partial y} \, dx \, dy.
\]

A continuous wavefront is reduced to a discrete set of $N' = \xi' \eta'$ average slope measurements for each camera frame.

A simple case involving two frames is shown in Figure~\ref{fig:overlap}. The overlap region reveals the relative tilt difference between adjacent frames. The analysis assumes a perfect reference file.

\begin{figure}[h!]
    \centering
    % Placeholder for Figure 1
    \includegraphics[width=0.8\textwidth]{placeholder.png}
    \caption{Overlapping frames. The red and blue outlines represent consecutive frames with overlap regions shaded in gray.}
    \label{fig:overlap}
\end{figure}

\section{The Mathematical Problem}

The tilt of each frame is denoted as $c_{\mu\nu}$. For adjacent frames, the relative tilt difference is:
\[
\Delta v_{\mu\nu} = v_{\mu\nu} - v_{\mu+1,\nu}, \quad \Delta h_{\mu\nu} = h_{\mu\nu} - h_{\mu,\nu+1}.
\]

The least squares fit minimizes the sum of the squares of the differences:
\[
\chi^2 = \sum_{\mu=1}^\xi \sum_{\nu=1}^\eta 
\left( (\Delta v_{\mu\nu} - (c_{\mu\nu} - c_{\mu+1,\nu}))^2 + (\Delta h_{\mu\nu} - (c_{\mu\nu} - c_{\mu,\nu+1}))^2 \right).
\]

The system of equations derived from minimizing $\chi^2$ can be solved using Singular Value Decomposition (SVD). 

Define the interaction matrix $A$ and data matrix $D$ such that:
\[
A \cdot c = D.
\]

The matrix $A$ is singular and requires special handling. Using SVD, the inverse can be computed as:
\[
A^{-1} = V \Sigma^{-1} U^T,
\]
where $\Sigma$ contains the singular values, $U$ and $V$ are orthogonal matrices, and $A = U \Sigma V^T$.

\section{Example}

Consider a non-rectangular quilt (Figure~\ref{fig:nonrectangular}). The existence function $\epsilon_{\mu\nu}$ determines if data is present in a given panel:
\[
\epsilon_{\mu\nu} =
\begin{cases}
1 & \text{if data exists in panel } (\mu, \nu), \\
0 & \text{otherwise.}
\end{cases}
\]

\begin{figure}[h!]
    \centering
    % Placeholder for Figure 4
    \includegraphics[width=0.8\textwidth]{placeholder.png}
    \caption{Non-rectangular quilt example. The existence function $\epsilon_{\mu\nu}$ ensures valid data regions are identified.}
    \label{fig:nonrectangular}
\end{figure}

In a practical implementation, the solution vector $c$ must adjust for missing data. This ensures a robust solution to the stitching problem. For example, if a frame is missing data, the existence function accounts for it, and $A$ is adjusted accordingly.

\section{References}

\begin{enumerate}
    \item R.A. Horn and C.R. Johnson, \textit{Matrix Analysis}, Cambridge University Press, 1990.
    \item G.H. Golub and C.F. Van Loan, \textit{Matrix Computations}, Johns Hopkins University Press, 1996.
    \item W.H. Press et al., \textit{Numerical Recipes in FORTRAN}, Cambridge University Press, 1992.
    \item R.A. Horn and C.R. Johnson, \textit{Topics in Matrix Analysis}, Cambridge University Press, 1991.
\end{enumerate}

\end{document}
