\documentclass{article}
\usepackage[backend=biber,style=numeric]{biblatex} 
\addbibresource{/Users/dantopa/repos-xiuhcoatal/github/sharing/bibliographies/tle-01.bib}

\usepackage{lmodern}
\usepackage{hyperref}

\begin{document}

This is an example of citing a reference. For instance, we can refer to Myers' work \cite{Myers1973_ExecSummary}.

\section{More on Geosynchronous Orbits}
The \href{https://ntrs.nasa.gov/search?q=Geosynchronous\%20platform\%20definition\%20study}{Geosynchronous Platform Definition Study}, consisting of several volumes, is detailed in the following order:
\cite{Myers1973_ExecSummary, Myers1973_BaselineTraffic, Myers1973_StudySummary, 
      Myers1973_PlatformSynthesis, Myers1973_TransportationRequirements, 
      Myers1973_ProgramEvaluation}.

\printbibliography
\end{document}
