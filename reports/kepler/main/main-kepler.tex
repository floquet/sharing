% Compile Pdftex
% TeXShop > Settings... > Engine > BibTex engine > biber

\documentclass[10pt, oneside]{article}   	% use "amsart" instead of "article" for AMSLaTeX format
\usepackage{geometry}                		% See geometry.pdf to learn the layout options. There are lots.
\geometry{letterpaper}                   		% ... or a4paper or a5paper or ... 
%\geometry{landscape}                		% Activate for rotated page geometry
%\usepackage[parfill]{parskip}    		% Activate to begin paragraphs with an empty line rather than an indent
\usepackage{graphicx}				% Use pdf, png, jpg, or eps§ with pdflatex; use eps in DVI mode

% personalize environment
\newcommand{\pGlobal}[0]			{../../../global/}
\newcommand{\pGlobalSetup}[0]		{\pGlobal setup-global/}	

\input{\pGlobalSetup setup-global-reports}
\input{\pLocalSetup macros}

%\usepackage[ backend=biber, bibencoding=ascii, style=authoryear, sorting=none ]{biblatex}
\bibliography{\pGlobalBibliographies orbits.bib}
%
\usepackage{listings}
	\definecolor{textblue}{rgb}{.2,.2,.7}
	\definecolor{textred}{rgb}{0.54,0,0}
	\definecolor{textgreen}{rgb}{0,0.43,0}

\newcommand{\angdom}[0]			{\left[ 0, 2\pi \right.)}

\title{Kepler's Equation: Derivation, Implementation, Validation}
\author{Daniel Topa\\\href{mailto:daniel.topa@hii-tsd.com}{daniel.topa@hii-tsd.com}}
\affil{\href{https://hii.com/what-we-do/divisions/mission-technologies/}{Mission Technologies}
\\Huntington Ingalls Industries
\\Kirtland AFB, NM}

\begin{document}
\maketitle
\abstract{Kepler's law is a cornerstone of orbital mechanics.}
\tableofcontents

\section{Overview}
Kepler's law is a cornerstone of orbital mechanics.
Kepler's equation
		%
\begin{equation}
	M = E - e \sin E
\label{eq:kepler-law}
\end{equation}
		%
Kepler's Law III
		%
\begin{equation}
	n = \mu^{\tfrac{1}{2}} a^{\tfrac{-3}{2}}
\label{eq:kepler-law-III}
\end{equation}


		\input{\pSections sec-derivation}
		\input{\pSections sec-geometry}
		\input{\pSections sec-math}
		\appendix
		\input{\pSections sec-diagrams}
				
		\printbibliography
			
\end{document} 

\tiny
\scriptsize
\footnotesize
\small
\normalsize
\large
\Large
\huge
\Huge