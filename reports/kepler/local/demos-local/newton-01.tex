% https://tex.stackexchange.com/questions/551160/plot-that-demonstrate-newtons-method
% \documentclass[]{article}
\documentclass[margin=5pt, varwidth]{standalone}
\usepackage{amsmath}
\usepackage{pgfplots}
\usepackage{pgfplotstable}
\pgfplotsset{compat=newest}

\begin{document}
% Input 1/2 =====
\newcommand\fxshow{e^{0.9x}-x^2}
\pgfmathsetlengthmacro\mywidth{8.9cm}

\tikzset{trig format=rad, 
declare function={
% Input 2/2 =====
f(\x)=exp(0.9*\x) -\x*\x;  
xStart=2.6;
Steps=4;
% Calc ====
xNew(\x)=\x-f(\x)/df(\x);
dx=0.01;      
df(\x)=( f(\x+dx) -f(\x) )/dx;
},}

% Start row
\pgfmathsetmacro\xStart{xStart}
\pgfmathsetmacro\fxnStart{f(xStart)}
\pgfmathsetmacro\dfxnStart{df(xStart)}
\pgfmathsetmacro\xNewStart{xNew(xStart)}
\pgfplotstableread[header=false, col sep=comma,
]{
0, \xStart, \fxnStart, \dfxnStart,  \xNewStart
}\newtontable

% Further rows
\pgfmathsetmacro\Steps{Steps}
\pgfplotsforeachungrouped \n in {1,...,\Steps} {%%
\ifnum\n=1 \pgfplotstablegetelem{0}{[index]4}\of\newtontable \else
\pgfplotstablegetelem{0}{[index]4}\of\nextrow \fi
\pgfmathsetmacro\xOld{\pgfplotsretval}
%
\pgfmathsetmacro\fxn{f(\xOld)}
\pgfmathsetmacro\dfxn{df(\xOld)}
\pgfmathsetmacro\xNew{xNew(\xOld)}
%