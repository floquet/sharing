% \input{\pSections sec-math}

\section{Mathematics}
\label{sec:math}

\subsection{Definitions}
%%
\begin{definition}[The ellipse]
Given $\theta \in \angdom$, and parameters $a, b \in\mathbb{R}^{+}$ with $a>b$ the following parametric form defines an ellipse.
\begin{equation}
	\epsilon(\theta) = \paren{ a \cos \theta, b\sin \theta }
\label{eq:ellipse}
\end{equation}
\end{definition}
%%
\begin{definition}[Eccentricity of the ellipse]
The eccentricity is a scalar parameter $e \in \paren{0,1}$ and can be expressed in terms of fundamental parameters of the ellipse where $a > b$ as
\begin{equation}
	e = \frac{c}{a} = \sqrt{1 - \paren{\frac{b}{a}}^{2}}
\label{eq:eccentricity}
\end{equation}
\end{definition}
%%
\begin{definition}[Mean anomaly]
Kepler's Law\footnote{
\cite[eq 4.5]{bate2020fundamentals}
\cite[p.159]{moulton1970introduction}
\cite[\S2.2]{vallado2022}, \cite[3-19]{kaula2013theory}} 
defines the mean anomaly as the angular measure $M(e,E)\colon(0,1)\times \angdom \mapsto \angdom $ as
\begin{equation}
	M(e,E) = E - e \sin E
\label{eq:mean anomaly}
\end{equation}
\end{definition}

\begin{theorem}[Continuity of the mean anomaly]
The mean anomaly as defined in definition \ref{eq:mean anomaly} is a continuous function.
\begin{proof}
To prove continuity show that for any two points $p$ and $q$ in the domain there exists a majorization constant $K$ such that
\begin{equation}
	M(p) - M(q) \le K \abs{p-q}
\label{eq:continuous}
\end{equation}
Spoiler alert: the majorization constant is $2\pi$.
\end{proof}
\end{theorem}
Observation: given the continuity of the mean anomaly, one may use Newton's method \cite[\S4.6]{gautschi2011numerical} to solve the nonlinear equation.


\endinput  %  ==  ==  ==  ==  ==  ==  ==  ==  ==
