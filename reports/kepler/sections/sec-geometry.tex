% \input{\pSections sec-geometry}

\section{Geometry of Kepler's Law}
\label{sec:geo}
		 \input{\pSections fig-geometry}
%\begin{figure}[htbp] %  figure placement: here, top, bottom, or page
%\centering
%\begin{center}
%   \includegraphics[ width = 3.75in ]{\pLocalGraphics vallado-anomaly} 
%   \caption{Vallado's figure 2-2 showing $E$ and $\nu$.}
%\end{center}
%   \label{fig:anomaly-vallado}
%\end{figure}
%
%\begin{figure}[htbp] %  figure placement: here, top, bottom, or page
%\begin{center}
%   \includegraphics[ width = 3.5in ]{\pLocalGraphics Bate-etal-anomaly} 
%   \caption{Figure 4-4 in Bate \emph{et al.}  showing $E$ and $\nu$.}
%\end{center}
%   \label{fig:anomaly-bate}
%\end{figure}
%
%\begin{figure}[htbp] %  figure placement: here, top, bottom, or page
%   \centering
%   \includegraphics[ width = 4in ]{\pLocalGraphics moulton-anomaly} 
%   \caption{Moulton's figure 28 showing $E$ and $\nu$.}
%   \label{fig:anomaly-mouton}
%\end{figure}
%
%\begin{figure}[htbp] %  figure placement: here, top, bottom, or page
%\centering
%\begin{tikzpicture}
%	\def\major{4cm}
%	\def\minor{3cm}
%	\coordinate (origin) at (0, 0);
%	\coordinate (earth) at (2.234, 0);
%	\coordinate (aux) at (3.464, 2.);
%	\coordinate (satellite) at (3.464, 1.5);
%	\draw[ gray ] (origin) circle (\major);
%	\draw[ thick ] (0,0) ellipse (4cm and (3cm);
%	\draw[ -{Stealth[gray]} ] (-4, 0) -- (5, 0) node[anchor=west ] {$\xi$};
%	\draw[ -{Stealth[gray]} ] (0, -4) -- (0, 5) node[ anchor=south ] {$\eta$};
%	\node[ circle, inner sep=1pt, fill=gray, label=below right:{$O$}] at (origin) {};
%	\node[ circle, inner sep=1pt, fill=gray, label=below:{$f$}] at (earth) {};
%	\draw[ gray ] (origin) -- (aux);
%	\draw[ black ] (earth) -- (satellite);
%\end{tikzpicture}
%\caption{Seeing the anomalies. The auxiliary circle in gray and the satellite trajectory as the black ellipse.}
%\label{fig:ellipse-circle}
%\end{figure}
%
%\begin{tikzpicture}
%    % Draw the ellipse
%    \draw[blue, thick] (0, 0) ellipse (4cm and 3cm);
%
%    % Draw the circumscribing circle
%    \draw[red, thick] (0, 0) circle (4cm);
%
%    % Calculate the focus of the ellipse
%    \def\a{4}
%    \def\b{3}
%    \def\c{sqrt(\a^2 - \b^2)}  % c = sqrt(a^2 - b^2)
%    \coordinate (F) at (-\c, 0); % Left focus
%
%    % Calculate the point on the circle at angle pi/6
%    \coordinate (C) at ({4*cos(30)}, {4*sin(30)}); % 30 degrees = pi/6
%
%    % Calculate the corresponding point on the ellipse at angle pi/6
%    \coordinate (E) at ({4*cos(30)}, {3*sin(30)});
%
%    % Draw lines
%    \draw[green, thick] (0, 0) -- (C) node[midway, right] {Line to Circle};
%    \draw[orange, thick] (F) -- (E) node[midway, below] {Line to Ellipse};
%
%    % Add labels
%    \node at (4.5, 0) {Circle (radius = 4 cm)};
%    \node at (0, 3.5) {Ellipse (a = 4 cm, b = 3 cm)};
%    \node at (-\c, -0.5) {Focus};
%\end{tikzpicture}

%\begin{table}[htp]
%%\caption{default}
%\begin{center}
%\begin{tabular}{ccc}
%	%
%   \includegraphics[ width=2in ]{\pLocalGraphics vallado-anomaly} 
%	%
%   \includegraphics[ width=2in ]{\pLocalGraphics vallado-anomaly} 
%	%
%   \includegraphics[ width=2in ]{\pLocalGraphics vallado-anomaly} 
%	%
%\end{tabular}
%\end{center}
%\label{default}
%\end{table}%

Disagreement with this YouTuber \href{https://www.youtube.com/watch?v=cf9Jh44kL20}{True Anomaly vs. Mean Anomaly}

\endinput  %  ==  ==  ==  ==  ==  ==  ==  ==  ==
