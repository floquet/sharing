% \input{\pSections/Introduction}
%
\section{Introduction}
Trigonometric functions (like sine and cosine) are initially defined on the interval $\brac{-\pi,\pi}$.
Zernike polynomials extend these functions to the unit disk by adding radial components along with the angular components.
Spherical harmonics extend trigonometric functions to the unit sphere, involving both angular components (latitude and longitude) and radial components.

Fourier analysis, a powerful tool for decomposing functions into sinusoidal components, extends naturally from one-dimensional domains to higher dimensions. In the one-dimensional case, the Fourier series represents a periodic function \( f(\theta) \) defined on the domain \( -\pi \leq \theta \leq \pi \) as:

\[
f(\theta) = \sum_{k=-\infty}^{\infty} a_k e^{i k \theta},
\]
where \( a_k \) are the Fourier coefficients given by:
\[
a_k = \frac{1}{2\pi} \int_{-\pi}^\pi f(\theta) e^{-i k \theta} \, d\theta.
\]

\section{Extending Fourier Analysis to 2D and 3D}

Fourier analysis, which decomposes functions into sinusoidal components, extends naturally to higher-dimensional domains. In two dimensions, Zernike polynomials serve as the orthogonal basis for circular domains, while spherical harmonics provide an analogous basis for spherical domains in three dimensions. 

\subsection{Zernike Polynomials: Extending Fourier to 2D}

In two dimensions, the domain is often described in polar coordinates \((r, \theta)\) on the unit disk:
\[
\Omega = \{(r, \theta) : 0 \leq r \leq 1, -\pi \leq \theta \leq \pi\}.
\]
Zernike polynomials \( Z_n^m(r, \theta) \) provide a basis that extends the Fourier series to 2D by decomposing functions into radial and angular components:
\[
Z_n^m(r, \theta) = R_n^m(r) e^{i m \theta},
\]
where:
\begin{itemize}
    \item \( R_n^m(r) \) is the radial polynomial that encodes the radial dependence,
    \item \( e^{i m \theta} \) represents the angular dependence as in the standard Fourier series.
\end{itemize}

The Zernike polynomials are orthogonal over the unit disk:
\[
\int_0^{2\pi} \int_0^1 Z_n^m(r, \theta) Z_{n'}^{m'}(r, \theta)^* \, r \, dr \, d\theta \propto \delta_{nn'} \delta_{mm'}.
\]

A function \( f(r, \theta) \) defined on the disk can be expanded as:
\[
f(r, \theta) = \sum_{n=0}^\infty \sum_{m=-n}^n a_n^m Z_n^m(r, \theta),
\]
where \( a_n^m \) are the expansion coefficients. This decomposition is widely used in applications such as optics, image processing, and wavefront analysis.

\subsection{Spherical Harmonics: Extending Fourier to 3D}

In three dimensions, spherical harmonics extend Fourier analysis to spherical domains. The domain is described in spherical coordinates:
\[
\Omega = \{(r, \theta, \phi) : 0 \leq r \leq 1, 0 \leq \phi < 2\pi, 0 \leq \theta \leq \pi\},
\]
where \( r \) is the radial distance, \( \theta \) is the polar angle, and \( \phi \) is the azimuthal angle.

The spherical harmonics \( Y_\ell^m(\theta, \phi) \) are defined as:
\[
Y_\ell^m(\theta, \phi) = \sqrt{\frac{(2\ell+1)}{4\pi} \frac{(\ell-m)!}{(\ell+m)!}} P_\ell^m(\cos\theta) e^{i m \phi},
\]
where:
\begin{itemize}
    \item \( P_\ell^m(\cos\theta) \) are the associated Legendre polynomials, encoding the polar angle dependence,
    \item \( e^{i m \phi} \) represents the azimuthal angular dependence.
\end{itemize}

The spherical harmonics are orthogonal over the unit sphere:
\[
\int_0^{2\pi} \int_0^\pi Y_\ell^m(\theta, \phi) Y_{\ell'}^{m'}(\theta, \phi)^* \sin\theta \, d\theta \, d\phi = \delta_{\ell \ell'} \delta_{m m'}.
\]

Any function \( f(r, \theta, \phi) \) defined on the sphere can be expanded as:
\[
f(r, \theta, \phi) = \sum_{\ell=0}^\infty \sum_{m=-\ell}^\ell a_\ell^m(r) Y_\ell^m(\theta, \phi),
\]
where \( a_\ell^m(r) \) are the expansion coefficients. This decomposition is used in many fields, such as quantum mechanics, geophysics, and electromagnetism.

\subsection{Summary}

Zernike polynomials and spherical harmonics generalize Fourier analysis to higher dimensions:
\begin{itemize}
    \item \textbf{Zernike polynomials} extend Fourier analysis to 2D by introducing a radial component and decomposing functions on circular domains.
    \item \textbf{Spherical harmonics} generalize Fourier analysis to 3D by incorporating both polar and azimuthal angular components, suitable for spherical domains.
\end{itemize}
Both methods maintain the core concept of decomposing functions into orthogonal basis functions, adapted to the geometry of the domain.
%
\subsection{Spectral Decomposition Applications}
\begin{enumerate}
\item Image Processing: Representing 2D images in terms of spatial frequencies
\item Signal Processing: Analyzing and filtering multi-dimensional signals
\item Quantum Mechanics: Solving problems in periodic potentials, such as in crystallography
\item Fluid Dynamics: Studying turbulence in 3D domains
\end{enumerate}

By extending the Fourier series to higher dimensions, complex phenomena can be understood and manipulated in terms of their fundamental frequency components, their spectra.


\endinput  %  ==  ==  ==  ==  ==  ==  ==  ==  ==
