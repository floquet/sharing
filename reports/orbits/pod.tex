% Report Shell
% Generated by ChatGPT v4, 2024-12-02
\documentclass[a4paper,10pt]{article}

\usepackage[utf8]{inputenc}
\usepackage[T1]{fontenc}
\usepackage{geometry}
\usepackage{lmodern}
\usepackage{hyperref}
\usepackage{csquotes} % Recommended for BibLaTeX

% Page layout
\geometry{margin=1in}

% Bibliography settings
\usepackage[backend=biber,style=numeric]{biblatex}
\addbibresource{/Users/dantopa/repos-xiuhcoatal/github/sharing/bibliographies/precise.bib}

% Title and author
\title{A Quick Literature Survey: Precise Orbit Determination}
\author{Daniel Topa}
\date{\today}

\begin{document}

\maketitle

\begin{abstract}
What are the current limits on the precision in computing and measuring a satellite's location? A quick literature search outlined here shows sub-meter resolution, and in some cases, centimeter resolution. This report should be read as a gateway to the literature.
\end{abstract}
\tableofcontents
\section{Introduction}
The determination of a satellite's precise orbit is crucial for various applications, from Earth observation to deep-space exploration. Recent advances in orbit determination techniques are quickly surveyed and key findings are highlighted.

\section{Literature Review}
In this section, we summarize recent studies on precise orbit determination.

The results of our navigation experiment demonstrate
that RTK positioning accuracy is improved from meter to decimeter level with fixed ambiguity (horizontal < 2 cm, vertical < 18 cm).
Horizontal accuracy is improved by over 50\%, and the vertical accuracies of the results of the static and kinematic experiments are
increased by 47\% and 27\% respectively, compared with the results produced by the classical approach. Though as the baseline becomes
longer, the accuracy is weakened, our predictive algorithm is an improvement over existing approaches to overcome the issue of missing
data.


A novel predictive algorithm for double difference observations of obstructed BeiDou geostationary earth orbit (GEO) satellites




Reduced dynamic and kinematic precise orbit determination
for the Swarm mission from 4 years of GPS tracking
30\% improvement in the precision of the reduced dynamic orbits with resulting errors at the 0.5--1 cm level (1D RMS)



Precise relative positioning using real tracking data from COMPASS GEO and IGSO satellites
The precision of COMPASS only solutions is better than 2 cm for the North component and 4 cm for the vertical.


Precise Orbit Determination of the ZY3-03
Satellite Using the Yaw-Attitude Modeling
for Drift Angle CompensationThe orbit determination experiments have revealed that the zero-yaw assumption in the zero-attitude model would result in periodic orbit errors of up to ±86 mm in the normal direction, while
the proposed model can describe yaw angle variations accurately with errors of less than $\pm$0.01$^\circ$


Dynamic and Reduced-Dynamic Precise Orbit
Determination of Satellites in Low Earth Orbits
The precise positioning of satellites in Low Earth Orbits (LEO) has become a key technology
for advanced space missions. Dedicated satellite missions, such as CHAMP, GRACE and
GOCE, that aim to map the Earth’s gravity field and its variation over time with
unprecedented accuracy,

Orbital arcs over a whole day can be generated with an
accuracy of up to 4.5 cm RMS. 


Aiming at a 1-cm Orbit for Low Earth Orbiters: Reduced-Dynamic and Kinematic Precise Orbit Determination
Both techniques have reached a high level of maturity and have been successfully applied to missions in the past, for example to TOPEX/POSEIDON (T/P), leading to (sub-)decimeter orbit accuracy.

A novel method for improving LEO kinematic real-time precise orbit determination with neural networks
Benefiting from this method, a promising accuracy of 3.2 cm can be achieved in LEO KRTPOD


Precise orbit determination for low Earth orbit satellites using GNSS: Observations, models, and methods
Using a state-of-the-art combination of GNSS observations and satellite dynamics, the absolute orbit determination for a single satellite reached a precision of 1 cm.

\section{Some Orbit Propagators}

%%    %%    %%    %%    %%    %%    %%
\subsection{KASIOP}

%%    %%    %%    %%    %%    %%    %%
\subsection{NEOPROP}
The \href{https://neo.ssa.esa.int/}{European Space Agency} sponsors the \href{https://neo.ssa.esa.int/neo-propagator}{Asteroid and Comet Trajectory Propagator} NEOPROP\footnote{Splash page URL: https://neo.ssa.esa.int/neo-propagator} to model objects which may impact the Earth.
From the website:
\begin{quotation}
New orbital perturbations (e.g. Poynting-Robertson effect, solar radiation pressure, outgassing) to improve the propagator accuracy and to allow the identification and propagation of any celestial body (not only NEOs but also moons, comets, planets, etc.).
The pre-existing algorithms were further improved in order to increase the performance and reduce the need for human intervention. Robust and redundant preliminary orbit determination techniques were added in order to deal with very long and disrupted observational arcs, which usually would require a manual split of the observations. 
\end{quotation}

An \href{https://neo.ssa.esa.int/documents/20126/418165/Setup\_NEOPROP\_2.1.exe/efefe2bb-6ec4-1735-906c-3675a72cd2e1?t=1559724554583}{*.exe} file is available\footnote{Download URL: https://neo.ssa.esa.int/documents/20126/418165/Setup\_NEOPROP\_2.1.exe/efefe2bb-6ec4-1735-906c-3675a72cd2e1?t=1559724554583} for download.

The User's Manual focuses on running the software and has scant mathematical explanation.

% Table 17: Integrators implemented
% Source: Enhanced Orbit Propagator
% ESA Contract No. RFP/D/IPL-PTE/GLC/al/557.2014
% NEOPROP2 Software User Manual
% Reference: https://neo.ssa.esa.int/documents/20126/418165/propagator-manual.pdf/8e36ff2a-f499-a031-77bb-0bf917810d97?t=1559724493027 (page 42)

\begin{table}[h!]
\centering
\caption{Integrators Implemented}
\begin{tabular}{llll}
Integrator             & Step & Step-Size   & Integrator Identifier \\
Runge-Kutta 45         & single            & variable    & Runge\_Kutta\_45      \\
Dormand Prince 8       & single            & variable    & Dormand\_Prince\_8    \\
Runge-Kutta 853        & single            & variable    & Runge\_Kutta\_853     \\
Runge-Kutta 4          & single            & fixed       & Runge\_Kutta\_4       \\
Runge-Kutta 4 Adapted  & single            & fixed*      & Runge\_Kutta\_4\_Adapted \\
Gauss-Jackson 8        & multi             & fixed       & Gauss\_Jackson\_8     \\
Gauss-Jackson 8 Adapted & multi            & fixed*      & Gauss\_Jackson\_8\_Adapted \\
Gauss-Jackson 8 Self-Adapted & multi      & fixed*      & Gauss\_Jackson\_8\_Self\_Adapted \\
\end{tabular}

\begin{flushleft}
*The integration follows a fixed step-size scheme, but for some trajectory arcs (e.g., close to a celestial body), the step-size might be reduced by a factor of 10.

Source: \textit{Enhanced Orbit Propagator, ESA Contract No. RFP/D/IPL-PTE/GLC/al/557.2014. NEOPROP2 Software User Manual}.\\
For more details, see: \href{https://neo.ssa.esa.int/documents/20126/418165/propagator-manual.pdf/8e36ff2a-f499-a031-77bb-0bf917810d97?t=1559724493027#page=42}{ESA NEOPROP2 User Manual}.
\end{flushleft}
\end{table}

%%    %%    %%    %%    %%    %%    %%
\subsection{Orbit Determination Toolbox (ODTBX)}
\begin{quotation}
The \href{https://opensource.gsfc.nasa.gov/projects/ODTBX/}{Orbit Determination Toolbox} ODTBX is an orbit determination analysis tool based on Matlab and Java that provides a flexible way to do early mission analysis, especially for formation flying and exploration systems. ODTBX is composed of both Matlab and Java code.
\end{quotation}

Download\footnote{https://opensource.gsfc.nasa.gov/projects/ODTBX/ODTBX\_4\_0.jar} \href{https://opensource.gsfc.nasa.gov/projects/ODTBX/ODTBX_4_0.jar}{ODTBX\_4\_0.jar}

\begin{quotation}
The Java
Astrodynamics Toolbox is used as an engine for things that might be slow or inefficient in MATLAB, such as high-fidelity trajectory propagation, lunar and planetary ephemeris look-ups, precession, nutation, polar motion calculations, ephemeris file parsing, and the like.
\end{quotation}

%%    %%    %%    %%    %%    %%    %%
\subsection{polyastro: Astrodynamics in Python}
\begin{quotation}
poliastro is an open source (MIT) pure Python library for interactive Astrodynamics and Orbital Mechanics, with a focus on ease of use, speed, and quick visualization. It provides a simple and intuitive API, and handles physical quantities with units.

Some features include orbit propagation, solution of the Lambert's problem, conversion between position and velocity vectors and classical orbital elements and orbit plotting, among others. It focuses on interplanetary applications, but can also be used to analyze artificial satellites in Low-Earth Orbit (LEO).
\end{quotation}

The application \href{https://pypi.org/project/poliastro/}{polyastro} has a page the PyPI server\footnote{https://pypi.org/project/poliastro/} and adequate \href{https://docs.poliastro.space/en/stable/}{documentation}

\begin{itemize}
	\item Website: https://www.poliastro.space
	\item PyPi page: \href{https://pypi.org/project/poliastro/}{poliastro 0.17.0}
	\item Documentation: \href{https://docs.poliastro.space/en/stable/}{poliastro - Astrodynamics in Python}
\end{itemize}


%%    %%    %%    %%    %%    %%    %%
\subsection{OPI - Orbital Propagation Interface}
\begin{quotation}
\href{https://github.com/Space-Systems/OPI}{OPI} is an interface with the goal to facilitate the implementation of orbital propagators into different applications.

To calculate orbital motion, many different software programs exist emphasizing on different aspects such as execution speed or accuracy. They often require different input parameters and are written in different languages. This makes comparing or exchanging them a challenging task. OPI aims at simplifying this by providing a common way of handling propagation. Propagators using OPI are designed as plugins/shared libraries that can be loaded by a host program via the interface.
\end{quotation}

%%    %%    %%    %%    %%    %%    %%
\subsection{Orbit Predictor}
\begin{quotation}
\href{https://github.com/satellogic/orbit-predictor}{Orbit Predictor} is a Python library to propagate orbits of Earth-orbiting objects (satellites, ISS, Santa Claus, etc) using TLE (Two-Line Elements set). We can say Orbit predictor is kind of a "wrapper" for the python implementation of SGP4.
\end{quotation}

PyPi page: \href{https://pypi.org/project/orbit-predictor/}{orbit-predictor 1.15.0} \\
Download source: \href{https://files.pythonhosted.org/packages/c2/98/c1497925d73f522d17a2db3e704a85fdad17fcd191464b82fad82e73aedb/orbit-predictor-1.15.0.tar.gz}{orbit-predictor-1.15.0.tar.gz}


%%    %%    %%    %%    %%    %%    %%
\subsection{Orekit: an Open-source Library for Operational Flight Dynamics Applications}
\begin{quotation}
poliastro is an open source (MIT) pure Python library for interactive Astrodynamics and Orbital Mechanics, with a focus on ease of use, speed, and quick visualization. It provides a simple and intuitive API, and handles physical quantities with units.

Some features include orbit propagation, solution of the Lambert's problem, conversion between position and velocity vectors and classical orbital elements and orbit plotting, among others. It focuses on interplanetary applications, but can also be used to analyze artificial satellites in Low-Earth Orbit (LEO).
\end{quotation}

\section{Discussion}
The implications of achieving sub-meter or centimeter-level precision in satellite positioning are significant.
https://www.researchgate.net/profile/Luc-Maisonobe/publication/310250345_OREKIT_AN_OPEN_SOURCE_LIBRARY_FOR_OPERATIONAL_FLIGHT_DYNAMICS_APPLICATIONS/links/6034c01e299bf1cc26e4a550/OREKIT-AN-OPEN-SOURCE-LIBRARY-FOR-OPERATIONAL-FLIGHT-DYNAMICS-APPLICATIONS.pdf
% Add your discussion points here...

\section{Conclusion}
This quick survey outlines the state-of-the-art in precise orbit determination. Readers are encouraged to explore the referenced works for more in-depth information.

\newpage
%\section{References}
\nocite{*} % Include all entries from the bibliography file
\printbibliography

\end{document}
