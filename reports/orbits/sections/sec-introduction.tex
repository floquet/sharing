% \input{\pSections "sec-introduction.tex"}

\section{Introduction}
Advances in satellite orbit determination have achieved unprecedented precision, enabling centimeter-level accuracy crucial for modern scientific and operational missions. Since the launch of Sputnik in 1957, advances in tracking systems, computational models, and orbital mechanics have dramatically improved satellite positioning, achieving centimeter-level accuracy.

Table~\ref{tab:precision-limits} highlights representative studies showcasing the state of the art in Precision Orbit Determination (POD) as of 2019. These examples illustrate the range of methodologies and the precision achieved, serving as a gateway to deeper exploration of the field.

The availability of open-source tools and the integration of sophisticated perturbation models have democratized the field, enabling researchers and practitioners to address increasingly complex orbital challenges. As the capabilities of POD continue to grow, the inclusion of relativistic effects and advanced perturbation models ensures that the highest levels of accuracy can be achieved. These topics, along with emerging trends and applications, are explored in subsequent sections. 

\begin{table}[h!]
\centering
\begin{tabular}{lcccc}
 & Radial & Along-Track & Cross-Track & Residual \\
Satellite & Precision (cm) & Precision (cm) & Precision  (cm) &  RMS (mm/s) \\
\hline
TOPEX/POSEIDON & 1.2 & 2.3 & 2.8 & 0.406 \\
Jason-1        & 1.5 & 3.1 & 3.4 & 0.320 \\
Jason-2        & 1.3 & 2.7 & 3.0 & 0.360 \\
CryoSat-2      & 0.65 & 1.02 & 1.3 & 0.295 \\
Sentinel-3A    & 0.9 & 1.5 & 1.7 & 0.320 \\
\hline
\end{tabular}
\caption{Precision Orbit Determination Metrics for Various Satellites, taken from table 1 \cite{gaur2019precision}.}
\label{tab:precision-limits}
\end{table}



\endinput  %  ==  ==  ==  ==  ==  ==  ==  ==  ==
