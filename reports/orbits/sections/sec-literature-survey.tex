\section{Literature Survey}
In this section, we summarize recent studies on precise orbit determination.

\noindent Below is a summary of key insights from selected references quantifying the precision of orbit determination.

\begin{enumerate}[label=\textbf{\arabic{enumi}.}, leftmargin=0.5in]

\item \textbf{High-Precision GPS Orbit Determination by Integrating the Measurements from Regional Ground Stations and LEO Onboard Receivers} \cite{li2024high}
\begin{quote}
    \textit{The orbit and clock accuracies of GPS and LEO satellites are evaluated by comparison with precise products. The average Root Mean Square (RMS) of GPS orbit errors in the radial (R), along-track (T), and cross-track (N) directions are 2.27 cm, 3.45 cm, and 3.08 cm.}
\end{quote}

\item \textbf{Choices for Temporal Gravity Field Modeling for Precision Orbit Determination of CryoSat-2} \cite{schrama2024choices}
\begin{quote}
    \textit{The precision orbit determination (POD) of CryoSat-2 achieves 3 cm for the along-track component and 13 cm for the cross-track component. The laser residuals converge at approximately 1.02 cm, and Doppler residuals are at the 0.406 mm/s level. Additionally, the radial orbit difference relative to the CNES POE-F orbits narrows to 6.5 mm, demonstrating high accuracy in orbit determination.}
\end{quote}

\item \textbf{Impact of Pseudo-Stochastic Pulse and Phase Center Variation on Precision Orbit Determination of Haiyang-2A from Experimental HY2 Receiver GPS Data} \cite{wang2024impact} 
\begin{quote} 
	\textit{Validation using external precise science orbit (PSO) and satellite laser ranging (SLR) methods confirmed millimeter-level orbit precision.} 
\end{quote}

\item \textbf{A Novel Method for Improving LEO Kinematic Real-Time Precise Orbit Determination with Neural Networks} \cite{zhang2024novel}
\begin{quote}
    \textit{Benefiting from this method, a promising accuracy of 3.2 cm can be achieved in LEO KRTPOD.}
\end{quote}

\item \textbf{LEO Real-Time Ambiguity-Fixed Precise Orbit Determination with Onboard GPS/Galileo Observations} \cite{li2024leo}
\begin{quote}
    \textit{Using onboard GPS and Galileo observations, the 3D orbit accuracy of the ambiguity-fixed solution is significantly improved from 5.17 to 3.61 cm, by 30\%, compared to the ambiguity-float solution. Furthermore, the application of IAR also achieves a faster convergence to the centimeter-level orbit.}
\end{quote}

\item \textbf{Real-Time Precise Orbit Determination of Low Earth Orbit Satellites Based on GPS and BDS-3 PPP B2b Service} \cite{shi2024real}
\begin{quote}
    \textit{The RMS of the RT orbital errors in the radial, along, and cross directions is 0.10, 0.13, and 0.09 m, respectively, using BDS-3 and GPS PPP-B2b corrections.}
\end{quote}

\item \textbf{High-Precision Orbit Determination of the Small TJU-1 Satellite Using GPS, GLONASS, and BDS} \cite{10122508}
\begin{quote}
    \textit{According to the orbit differences between different solutions, the orbit accuracy of single-GNSS solutions is inferred to be at the level of about 4.7–6.2 cm. With the multi-GNSS fusion, the precision and inferred accuracy of orbits could be improved significantly to 1.7–2.1 cm and about 1.7–3.7 cm, respectively.}
\end{quote}    

\item \textbf{Precise Orbit Determination for Low Earth Orbit Satellites Using GNSS: Observations, Models, and Methods} \cite{mao2024precise}
\begin{quote}
    \textit{Using a state-of-the-art combination of GNSS observations and satellite dynamics, the absolute orbit determination for a single satellite reached a precision of 1 cm.}
\end{quote}

\item \textbf{Precise Orbit Determination of the ZY3-03 Satellite Using the Yaw-Attitude Modeling for Drift Angle Compensation} \cite{gong2024precise}
\begin{quote}
    \textit{The orbit determination experiments revealed that the zero-yaw assumption in the zero-attitude model would result in periodic orbit errors of up to $\pm$86 mm in the normal direction, while the proposed model describes yaw angle variations accurately with errors of less than $\pm$0.01$^\circ$.}
\end{quote}
    
\item \textbf{Long-Term Orbit Dynamics of Decommissioned Geostationary Satellites} \cite{PROIETTI2021559}
\begin{quote}
    \textit{Orbit propagations are performed using two algorithms based on different equations of motion and numerical integration methods. The numerical results exhibit excellent agreement over integration times of decades.}
\end{quote}
    
\item \textbf{Reduced Dynamic and Kinematic Precise Orbit Determination for the Swarm Mission from 4 Years of GPS Tracking} \cite{montenbruck2018reduced}
\begin{quote}
    \textit{30\% improvement in the precision of the reduced dynamic orbits with resulting errors at the 0.5--1 cm level (1D RMS).}
\end{quote}

\item \textbf{Precise Relative Positioning Using Real Tracking Data from COMPASS GEO and IGSO Satellites} \cite{shi2013precise}
\begin{quote}
    \textit{The precision of COMPASS-only solutions is better than 2 cm for the North component and 4 cm for the vertical.}
\end{quote}

\item \textbf{Dynamic and Reduced-Dynamic Precise Orbit Determination of Satellites in Low Earth Orbits} \cite{swatschina2012dynamic}
\begin{quote}
    \textit{Orbital arcs over a whole day can be generated with an accuracy of up to 4.5 cm RMS.}
\end{quote}
    
\item \textbf{Aiming at a 1-cm Orbit for Low Earth Orbiters: Reduced-Dynamic and Kinematic Precise Orbit Determination} \cite{visser2003aiming}
\begin{quote}
    \textit{Both techniques have reached a high level of maturity and have been successfully applied to missions in the past, such as TOPEX/POSEIDON (T/P), leading to (sub-) decimeter orbit accuracy.}
\end{quote}

\end{enumerate}


\endinput  %  ==  ==  ==  ==  ==  ==  ==  ==  ==
