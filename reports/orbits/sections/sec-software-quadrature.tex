% \input{\pSections "sec-software-quadrature.tex"}

\section{Algorithms: Solution by Quadrature}
\label{sec:quadrature}

A crucial component for orbit propagators is a numerical integration package as most orbit mechanics lacks analytic solution.

%%    %%    %%    %%    %%    %%    %%
\subsection{Dormand Prince}
From the Wikipedia page \href{https://en.wikipedia.org/wiki/Dormand%E2%80%93Prince_method}{Dormand–Prince method}
\begin{quote}
The Dormand–Prince (RKDP) method or DOPRI method, is an embedded method for solving ordinary differential equations (ODE). The method is a member of the Runge–Kutta family of ODE solvers. More specifically, it uses six function evaluations to calculate fourth- and fifth-order accurate solutions. The difference between these solutions is then taken to be the error of the (fourth-order) solution. This error estimate is very convenient for adaptive stepsize integration algorithms. Other similar integration methods are Fehlberg (RKF) and \href{https://en.wikipedia.org/wiki/Cash%E2%80%93Karp_method}{Cash–Karp} (RKCK). 
\end{quote}

\fullcite{dormand1980family}

%%    %%    %%    %%    %%    %%    %%
\section{Software Packages}
\subsection{Nonstiff Differential Equations}

For a comprehensive description, see \emph{Solving Ordinary Differential Equations: Nonstiff Problems} by Hairer, N\o rsett, and Wanner \cite{Hairer1993}.

The following methods and their associated drivers are designed to solve nonstiff ordinary differential equations of the form \( y' = f(x, y) \). More information and software implementations can be found on the \href{http://www.unige.ch/~hairer/software.html}{University of Geneva Software Page}.

\begin{itemize}
    \item \textbf{DOPRI5}: An explicit Runge-Kutta method of order 5(4) with dense output of order 4. 
    \begin{itemize}
        \item \textbf{DR\_DOPRI5}: Driver for the DOPRI5 method.
        \item \href{http://www.unige.ch/~hairer/prog/nonstiff/dopri5.f}{DOPRI5 Fortran Source Code}
    \end{itemize}
    
    \item \textbf{DOP853}: An explicit Runge-Kutta method of order 8(5,3) with dense output of order 7.
    \begin{itemize}
        \item \textbf{DR\_DOP853}: Driver for the DOP853 method.
        \item \href{http://www.unige.ch/~hairer/prog/nonstiff/dop853.f}{DOP853 Fortran Source Code}
    \end{itemize}
    
    \item \textbf{ODEX}: An extrapolation method (Gragg–Bulirsch–Stoer) for solving \( y' = f(x, y) \), with dense output.
    \begin{itemize}
        \item \textbf{DR\_ODEX}: Driver for the ODEX method.
        \item \href{http://www.unige.ch/~hairer/prog/nonstiff/odex.f}{ODEX Fortran Source Code}
    \end{itemize}
    
    \item \textbf{ODEX2}: An extrapolation method (Störmer's rule) for second-order differential equations of the form \( y'' = f(x, y) \), with dense output.
    \begin{itemize}
        \item \textbf{DR\_ODEX2}: Driver for the ODEX2 method.
        \item \href{http://www.unige.ch/~hairer/prog/nonstiff/odex2.f}{ODEX2 Fortran Source Code}
    \end{itemize}
\end{itemize}
\section{Dormand–Prince Method in Python Libraries}
The Dormand–Prince method has also been widely adopted in computational libraries, enabling its use in a variety of programming environments. In Python, the \texttt{RK45} integrator within the \texttt{scipy.integrate} module implements this method. Below is an excerpt from the official SciPy documentation:

\begin{quote}
\textbf{RK45: Explicit Runge-Kutta method of order 5(4)} \\
This uses the Dormand–Prince pair of formulas. The error is controlled assuming the accuracy of the fourth-order method, but steps are taken using the fifth-order accurate formula (local extrapolation). A quartic interpolation polynomial is used for dense output, making this method suitable for a variety of practical applications. Notably, it can also be applied in the complex domain \cite{scipy2024rk45}. (See \href{https://docs.scipy.org/doc/scipy/reference/generated/scipy.integrate.RK45.html}{SciPy Documentation})
\end{quote}

\subsection{Integrating \texttt{RK45} in Orbit Propagation}
The inclusion of \texttt{scipy.integrate.RK45} into Python-based astrodynamics tools simplifies the process of solving nonstiff ordinary differential equations such as \( y' = f(x, y) \). Leveraging this method ensures accurate and efficient numerical integration with adaptive step size, making it an ideal choice for orbit propagation tasks.

\begin{itemize}
    \item \textbf{Strengths}:
    \begin{itemize}
        \item Adaptive stepsize control based on the Dormand–Prince embedded method.
        \item Dense output suitable for event detection or interpolating orbital states.
        \item Native compatibility with Python, enhancing its usability in modern astrodynamics workflows.
    \end{itemize}
    \item \textbf{Limitations}:
    \begin{itemize}
        \item Primarily designed for nonstiff problems; not ideal for highly stiff systems.
        \item Additional overhead when compared to fixed-step integrators.
    \end{itemize}
\end{itemize}

\subsection{Example Usage in Python}
Below is an example of solving an orbital mechanics problem using \texttt{scipy.integrate.RK45}:

\begin{verbatim}
from scipy.integrate import RK45
import numpy as np

def orbit_dynamics(t, y):
    # Example: Simple 2D motion under gravity
    mu = 3.986e14  # Earth's standard gravitational parameter [m^3/s^2]
    r = np.linalg.norm(y[:2])
    dydt = np.zeros_like(y)
    dydt[:2] = y[2:]  # Velocity components
    dydt[2:] = -mu * y[:2] / r**3  # Acceleration components
    return dydt

# Initial conditions [x, y, vx, vy]
y0 = np.array([7000e3, 0, 0, 7.5e3])
t0, t_bound = 0, 3600  # Time bounds (seconds)

solver = RK45(orbit_dynamics, t0, y0, t_bound, max_step=60)
trajectory = []
while solver.status == 'running':
    solver.step()
    trajectory.append((solver.t, *solver.y))

trajectory = np.array(trajectory)
\end{verbatim}

This example showcases the versatility and power of SciPy's implementation of the Dormand–Prince method for solving differential equations in orbital mechanics.

\subsection{References}
\fullcite{scipy2024rk45}
	

\endinput  %  ==  ==  ==  ==  ==  ==  ==  ==  ==
