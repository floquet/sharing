% % % % \input{./components/man/man-ldd}
\subsection{\refLdd: Print Shared Object Dependencies}

{\tiny{
\begin{lstlisting}[language=bash]
NAME
       ldd - print shared object dependencies
SYNOPSIS
       ldd [option]... file...
DESCRIPTION
       ldd prints the shared objects (shared libraries) required by each
       program or shared object specified on the command line.  An
       example of its use and output is the following:

           $ ldd /bin/ls
               linux-vdso.so.1 (0x00007ffcc3563000)
               libselinux.so.1 => /lib64/libselinux.so.1 (0x00007f87e5459000)
               libcap.so.2 => /lib64/libcap.so.2 (0x00007f87e5254000)
               libc.so.6 => /lib64/libc.so.6 (0x00007f87e4e92000)
               libpcre.so.1 => /lib64/libpcre.so.1 (0x00007f87e4c22000)
               libdl.so.2 => /lib64/libdl.so.2 (0x00007f87e4a1e000)
               /lib64/ld-linux-x86-64.so.2 (0x00005574bf12e000)
               libattr.so.1 => /lib64/libattr.so.1 (0x00007f87e4817000)
               libpthread.so.0 => /lib64/libpthread.so.0 (0x00007f87e45fa000)

       In the usual case, ldd invokes the standard dynamic linker (see
       ld.so(8)) with the LD_TRACE_LOADED_OBJECTS environment variable
       set to 1.  This causes the dynamic linker to inspect the
       program's dynamic dependencies, and find (according to the rules
       described in ld.so(8)) and load the objects that satisfy those
       dependencies.  For each dependency, ldd displays the location of
       the matching object and the (hexadecimal) address at which it is
       loaded.  (The linux-vdso and ld-linux shared dependencies are
       special; see vdso(7) and ld.so(8).)

   Security
       Be aware that in some circumstances (e.g., where the program
       specifies an ELF interpreter other than ld-linux.so), some
       versions of ldd may attempt to obtain the dependency information
       by attempting to directly execute the program, which may lead to
       the execution of whatever code is defined in the program's ELF
       interpreter, and perhaps to execution of the program itself.
       (Before glibc 2.27, the upstream ldd implementation did this for
       example, although most distributions provided a modified version
       that did not.)

       Thus, you should never employ ldd on an untrusted executable,
       since this may result in the execution of arbitrary code.  A
       safer alternative when dealing with untrusted executables is:

           $ objdump -p /path/to/program | grep NEEDED

       Note, however, that this alternative shows only the direct
       dependencies of the executable, while ldd shows the entire
       dependency tree of the executable.
OPTIONS
       --version
              Print the version number of ldd.

       --verbose
       -v     Print all information, including, for example, symbol
              versioning information.

       --unused
       -u     Print unused direct dependencies.  (Since glibc 2.3.4.)

       --data-relocs
       -d     Perform relocations and report any missing objects (ELF
              only).

       --function-relocs
       -r     Perform relocations for both data objects and functions,
              and report any missing objects or functions (ELF only).

       --help Usage information.
BUGS
       ldd does not work on a.out shared libraries.

       ldd does not work with some extremely old a.out programs which
       were built before ldd support was added to the compiler releases.
       If you use ldd on one of these programs, the program will attempt
       to run with argc = 0 and the results will be unpredictable.
SEE ALSO
       pldd(1), sprof(1), ld.so(8), ldconfig(8)
COLOPHON
       This page is part of the man-pages (Linux kernel and C library
       user-space interface documentation) project.  Information about
       the project can be found at 
       https://www.kernel.org/doc/man-pages/.  If you have a bug report
       for this manual page, see
       https://git.kernel.org/pub/scm/docs/man-pages/man-pages.git/tree/CONTRIBUTING.
       This page was obtained from the tarball man-pages-6.9.1.tar.gz
       fetched from
       https://mirrors.edge.kernel.org/pub/linux/docs/man-pages/ on
       2024-06-26.  If you discover any rendering problems in this HTML
       version of the page, or you believe there is a better or more up-
       to-date source for the page, or you have corrections or
       improvements to the information in this COLOPHON (which is not
       part of the original manual page), send a mail to
       man-pages@man7.org

Linux man-pages 6.9.1          2024-05-02                         ldd(1)
\end{lstlisting}
}}
\endinput  %  ==  ==  ==  ==  ==  ==  ==  ==  ==

\subsection{\refLdd: Print Shared Object Dependencies}

{\tiny{
\begin{lstlisting}[language=bash]
NAME
       ldd - print shared object dependencies
SYNOPSIS
       ldd [option]... file...
DESCRIPTION
       ldd prints the shared objects (shared libraries) required by each
       program or shared object specified on the command line.  An
       example of its use and output is the following:

           $ ldd /bin/ls
               linux-vdso.so.1 (0x00007ffcc3563000)
               libselinux.so.1 => /lib64/libselinux.so.1 (0x00007f87e5459000)
               libcap.so.2 => /lib64/libcap.so.2 (0x00007f87e5254000)
               libc.so.6 => /lib64/libc.so.6 (0x00007f87e4e92000)
               libpcre.so.1 => /lib64/libpcre.so.1 (0x00007f87e4c22000)
               libdl.so.2 => /lib64/libdl.so.2 (0x00007f87e4a1e000)
               /lib64/ld-linux-x86-64.so.2 (0x00005574bf12e000)
               libattr.so.1 => /lib64/libattr.so.1 (0x00007f87e4817000)
               libpthread.so.0 => /lib64/libpthread.so.0 (0x00007f87e45fa000)

       In the usual case, ldd invokes the standard dynamic linker (see
       ld.so(8)) with the LD_TRACE_LOADED_OBJECTS environment variable
       set to 1.  This causes the dynamic linker to inspect the
       program's dynamic dependencies, and find (according to the rules
       described in ld.so(8)) and load the objects that satisfy those
       dependencies.  For each dependency, ldd displays the location of
       the matching object and the (hexadecimal) address at which it is
       loaded.  (The linux-vdso and ld-linux shared dependencies are
       special; see vdso(7) and ld.so(8).)

   Security
       Be aware that in some circumstances (e.g., where the program
       specifies an ELF interpreter other than ld-linux.so), some
       versions of ldd may attempt to obtain the dependency information
       by attempting to directly execute the program, which may lead to
       the execution of whatever code is defined in the program's ELF
       interpreter, and perhaps to execution of the program itself.
       (Before glibc 2.27, the upstream ldd implementation did this for
       example, although most distributions provided a modified version
       that did not.)

       Thus, you should never employ ldd on an untrusted executable,
       since this may result in the execution of arbitrary code.  A
       safer alternative when dealing with untrusted executables is:

           $ objdump -p /path/to/program | grep NEEDED

       Note, however, that this alternative shows only the direct
       dependencies of the executable, while ldd shows the entire
       dependency tree of the executable.
OPTIONS
       --version
              Print the version number of ldd.

       --verbose
       -v     Print all information, including, for example, symbol
              versioning information.

       --unused
       -u     Print unused direct dependencies.  (Since glibc 2.3.4.)

       --data-relocs
       -d     Perform relocations and report any missing objects (ELF
              only).

       --function-relocs
       -r     Perform relocations for both data objects and functions,
              and report any missing objects or functions (ELF only).

       --help Usage information.
BUGS
       ldd does not work on a.out shared libraries.

       ldd does not work with some extremely old a.out programs which
       were built before ldd support was added to the compiler releases.
       If you use ldd on one of these programs, the program will attempt
       to run with argc = 0 and the results will be unpredictable.
SEE ALSO
       pldd(1), sprof(1), ld.so(8), ldconfig(8)
COLOPHON
       This page is part of the man-pages (Linux kernel and C library
       user-space interface documentation) project.  Information about
       the project can be found at 
       https://www.kernel.org/doc/man-pages/.  If you have a bug report
       for this manual page, see
       https://git.kernel.org/pub/scm/docs/man-pages/man-pages.git/tree/CONTRIBUTING.
       This page was obtained from the tarball man-pages-6.9.1.tar.gz
       fetched from
       https://mirrors.edge.kernel.org/pub/linux/docs/man-pages/ on
       2024-06-26.  If you discover any rendering problems in this HTML
       version of the page, or you believe there is a better or more up-
       to-date source for the page, or you have corrections or
       improvements to the information in this COLOPHON (which is not
       part of the original manual page), send a mail to
       man-pages@man7.org

Linux man-pages 6.9.1          2024-05-02                         ldd(1)
\end{lstlisting}
}}
\endinput  %  ==  ==  ==  ==  ==  ==  ==  ==  ==

\subsection{\refLdd: Print Shared Object Dependencies}

{\tiny{
\begin{lstlisting}[language=bash]
NAME
       ldd - print shared object dependencies
SYNOPSIS
       ldd [option]... file...
DESCRIPTION
       ldd prints the shared objects (shared libraries) required by each
       program or shared object specified on the command line.  An
       example of its use and output is the following:

           $ ldd /bin/ls
               linux-vdso.so.1 (0x00007ffcc3563000)
               libselinux.so.1 => /lib64/libselinux.so.1 (0x00007f87e5459000)
               libcap.so.2 => /lib64/libcap.so.2 (0x00007f87e5254000)
               libc.so.6 => /lib64/libc.so.6 (0x00007f87e4e92000)
               libpcre.so.1 => /lib64/libpcre.so.1 (0x00007f87e4c22000)
               libdl.so.2 => /lib64/libdl.so.2 (0x00007f87e4a1e000)
               /lib64/ld-linux-x86-64.so.2 (0x00005574bf12e000)
               libattr.so.1 => /lib64/libattr.so.1 (0x00007f87e4817000)
               libpthread.so.0 => /lib64/libpthread.so.0 (0x00007f87e45fa000)

       In the usual case, ldd invokes the standard dynamic linker (see
       ld.so(8)) with the LD_TRACE_LOADED_OBJECTS environment variable
       set to 1.  This causes the dynamic linker to inspect the
       program's dynamic dependencies, and find (according to the rules
       described in ld.so(8)) and load the objects that satisfy those
       dependencies.  For each dependency, ldd displays the location of
       the matching object and the (hexadecimal) address at which it is
       loaded.  (The linux-vdso and ld-linux shared dependencies are
       special; see vdso(7) and ld.so(8).)

   Security
       Be aware that in some circumstances (e.g., where the program
       specifies an ELF interpreter other than ld-linux.so), some
       versions of ldd may attempt to obtain the dependency information
       by attempting to directly execute the program, which may lead to
       the execution of whatever code is defined in the program's ELF
       interpreter, and perhaps to execution of the program itself.
       (Before glibc 2.27, the upstream ldd implementation did this for
       example, although most distributions provided a modified version
       that did not.)

       Thus, you should never employ ldd on an untrusted executable,
       since this may result in the execution of arbitrary code.  A
       safer alternative when dealing with untrusted executables is:

           $ objdump -p /path/to/program | grep NEEDED

       Note, however, that this alternative shows only the direct
       dependencies of the executable, while ldd shows the entire
       dependency tree of the executable.
OPTIONS
       --version
              Print the version number of ldd.

       --verbose
       -v     Print all information, including, for example, symbol
              versioning information.

       --unused
       -u     Print unused direct dependencies.  (Since glibc 2.3.4.)

       --data-relocs
       -d     Perform relocations and report any missing objects (ELF
              only).

       --function-relocs
       -r     Perform relocations for both data objects and functions,
              and report any missing objects or functions (ELF only).

       --help Usage information.
BUGS
       ldd does not work on a.out shared libraries.

       ldd does not work with some extremely old a.out programs which
       were built before ldd support was added to the compiler releases.
       If you use ldd on one of these programs, the program will attempt
       to run with argc = 0 and the results will be unpredictable.
SEE ALSO
       pldd(1), sprof(1), ld.so(8), ldconfig(8)
COLOPHON
       This page is part of the man-pages (Linux kernel and C library
       user-space interface documentation) project.  Information about
       the project can be found at 
       https://www.kernel.org/doc/man-pages/.  If you have a bug report
       for this manual page, see
       https://git.kernel.org/pub/scm/docs/man-pages/man-pages.git/tree/CONTRIBUTING.
       This page was obtained from the tarball man-pages-6.9.1.tar.gz
       fetched from
       https://mirrors.edge.kernel.org/pub/linux/docs/man-pages/ on
       2024-06-26.  If you discover any rendering problems in this HTML
       version of the page, or you believe there is a better or more up-
       to-date source for the page, or you have corrections or
       improvements to the information in this COLOPHON (which is not
       part of the original manual page), send a mail to
       man-pages@man7.org

Linux man-pages 6.9.1          2024-05-02                         ldd(1)
\end{lstlisting}
}}
\endinput  %  ==  ==  ==  ==  ==  ==  ==  ==  ==

\subsection{\refLddconfig: Configure Dynamic Linker Run-time Bindings}

{\tiny{
\begin{lstlisting}[language=bash]
NAME
       ldconfig - configure dynamic linker run-time bindings
SYNOPSIS
       /sbin/ldconfig [-nNvVX] [-C cache] [-f conf] [-r root]
                      directory ...

       /sbin/ldconfig -l [-v] library ...

       /sbin/ldconfig -p
DESCRIPTION
       ldconfig creates the necessary links and cache to the most recent
       shared libraries found in the directories specified on the
       command line, in the file /etc/ld.so.conf, and in the trusted
       directories, /lib and /usr/lib.  On some 64-bit architectures
       such as x86-64, /lib and /usr/lib are the trusted directories for
       32-bit libraries, while /lib64 and /usr/lib64 are used for 64-bit
       libraries.

       The cache is used by the run-time linker, ld.so or ld-linux.so.
       ldconfig checks the header and filenames of the libraries it
       encounters when determining which versions should have their
       links updated.  ldconfig should normally be run by the superuser
       as it may require write permission on some root owned directories
       and files.

       ldconfig will look only at files that are named lib*.so* (for
       regular shared objects) or ld-*.so* (for the dynamic loader
       itself).  Other files will be ignored.  Also, ldconfig expects a
       certain pattern to how the symbolic links are set up, like this
       example, where the middle file (libfoo.so.1 here) is the SONAME
       for the library:

           libfoo.so -> libfoo.so.1 -> libfoo.so.1.12

       Failure to follow this pattern may result in compatibility issues
       after an upgrade.
OPTIONS
       --format=fmt
       -c fmt (Since glibc 2.2) Use cache format fmt, which is one of
              old, new, or compat.  Since glibc 2.32, the default is
              new.  Before that, it was compat.

       -C cache
              Use cache instead of /etc/ld.so.cache.

       -f conf
              Use conf instead of /etc/ld.so.conf.

       --ignore-aux-cache
       -i     (Since glibc 2.7) Ignore auxiliary cache file.

       -l     (Since glibc 2.2) Interpret each operand as a library name
              and configure its links.  Intended for use only by
              experts.

       -n     Process only the directories specified on the command
              line; don't process the trusted directories, nor those
              specified in /etc/ld.so.conf.  Implies -N.

       -N     Don't rebuild the cache.  Unless -X is also specified,
              links are still updated.

       --print-cache
       -p     Print the lists of directories and candidate libraries
              stored in the current cache.

       -r root
              Change to and use root as the root directory.

       --verbose
       -v     Verbose mode.  Print current version number, the name of
              each directory as it is scanned, and any links that are
              created.  Overrides quiet mode.

       --version
       -V     Print program version.

       -X     Don't update links.  Unless -N is also specified, the
              cache is still rebuilt.
FILES
       /lib/ld.so
              is the run-time linker/loader.
       /etc/ld.so.conf
              contains a list of directories, one per line, in which to
              search for libraries.
       /etc/ld.so.cache
              contains an ordered list of libraries found in the
              directories specified in /etc/ld.so.conf, as well as those
              found in the trusted directories.
SEE ALSO
       ldd(1), ld.so(8)
COLOPHON
       This page is part of the man-pages (Linux kernel and C library
       user-space interface documentation) project.  Information about
       the project can be found at 
       https://www.kernel.org/doc/man-pages/.  If you have a bug report
       for this manual page, see
       https://git.kernel.org/pub/scm/docs/man-pages/man-pages.git/tree/CONTRIBUTING.
       This page was obtained from the tarball man-pages-6.9.1.tar.gz
       fetched from
       https://mirrors.edge.kernel.org/pub/linux/docs/man-pages/ on
       2024-06-26.  If you discover any rendering problems in this HTML
       version of the page, or you believe there is a better or more up-
       to-date source for the page, or you have corrections or
       improvements to the information in this COLOPHON (which is not
       part of the original manual page), send a mail to
       man-pages@man7.org

Linux man-pages 6.9.1          2024-05-02                    ldconfig(8)
\end{lstlisting}
}}
\endinput  %  ==  ==  ==  ==  ==  ==  ==  ==  ==
