% % % % \input{./components/man/man-readelf}
\subsection{\refReadelf: Display Information On \elf \ Files}

{\tiny{
\begin{lstlisting}[language=bash]
NAME
       readelf - display information about ELF files
SYNOPSIS
       readelf [-a|--all]
               [-h|--file-header]
               [-l|--program-headers|--segments]
               [-S|--section-headers|--sections]
               [-g|--section-groups]
               [-t|--section-details]
               [-e|--headers]
               [-s|--syms|--symbols]
               [--dyn-syms|--lto-syms]
               [--sym-base=[0|8|10|16]]
               [--demangle=style|--no-demangle]
               [--quiet]
               [--recurse-limit|--no-recurse-limit]
               [-U method|--unicode=method]
               [-X|--extra-sym-info|--no-extra-sym-info]
               [-n|--notes]
               [-r|--relocs]
               [-u|--unwind]
               [-d|--dynamic]
               [-V|--version-info]
               [-A|--arch-specific]
               [-D|--use-dynamic]
               [-L|--lint|--enable-checks]
               [-x <number or name>|--hex-dump=<number or name>]
               [-p <number or name>|--string-dump=<number or name>]
               [-R <number or name>|--relocated-dump=<number or name>]
               [-z|--decompress]
               [-c|--archive-index]
               [-w[lLiaprmfFsoORtUuTgAck]|
                --debug-dump[=rawline,=decodedline,=info,=abbrev,=pubnames,=aranges,=macro,=frames,=frames-interp,=str,=str-offsets,=loc,=Ranges,=pubtypes,=trace_info,=trace_abbrev,=trace_aranges,=gdb_index,=addr,=cu_index,=links]]
               [-wK|--debug-dump=follow-links]
               [-wN|--debug-dump=no-follow-links]
               [-wD|--debug-dump=use-debuginfod]
               [-wE|--debug-dump=do-not-use-debuginfod]
               [-P|--process-links]
               [--dwarf-depth=n]
               [--dwarf-start=n]
               [--ctf=section]
               [--ctf-parent=section]
               [--ctf-symbols=section]
               [--ctf-strings=section]
               [--sframe=section]
               [-I|--histogram]
               [-v|--version]
               [-W|--wide]
               [-T|--silent-truncation]
               [-H|--help]
               elffile...
DESCRIPTION
       readelf displays information about one or more ELF format object
       files.  The options control what particular information to
       display.

       elffile... are the object files to be examined.  32-bit and
       64-bit ELF files are supported, as are archives containing ELF
       files.

       This program performs a similar function to objdump but it goes
       into more detail and it exists independently of the BFD library,
       so if there is a bug in BFD then readelf will not be affected.
OPTIONS
       The long and short forms of options, shown here as alternatives,
       are equivalent.  At least one option besides -v or -H must be
       given.

       -a
       --all
           Equivalent to specifying --file-header, --program-headers,
           --sections, --symbols, --relocs, --dynamic, --notes,
           --version-info, --arch-specific, --unwind, --section-groups
           and --histogram.

           Note - this option does not enable --use-dynamic itself, so
           if that option is not present on the command line then
           dynamic symbols and dynamic relocs will not be displayed.

       -h
       --file-header
           Displays the information contained in the ELF header at the
           start of the file.

       -l
       --program-headers
       --segments
           Displays the information contained in the file's segment
           headers, if it has any.

       --quiet
           Suppress "no symbols" diagnostic.

       -S
       --sections
       --section-headers
           Displays the information contained in the file's section
           headers, if it has any.

       -g
       --section-groups
           Displays the information contained in the file's section
           groups, if it has any.

       -t
       --section-details
           Displays the detailed section information. Implies -S.

       -s
       --symbols
       --syms
           Displays the entries in symbol table section of the file, if
           it has one.  If a symbol has version information associated
           with it then this is displayed as well.  The version string
           is displayed as a suffix to the symbol name, preceded by an @
           character.  For example foo@VER_1.  If the version is the
           default version to be used when resolving unversioned
           references to the symbol then it is displayed as a suffix
           preceded by two @ characters.  For example foo@@VER_2.

       --dyn-syms
           Displays the entries in dynamic symbol table section of the
           file, if it has one.  The output format is the same as the
           format used by the --syms option.

       --lto-syms
           Displays the contents of any LTO symbol tables in the file.

       --sym-base=[0|8|10|16]
           Forces the size field of the symbol table to use the given
           base.  Any unrecognized options will be treated as 0.
           --sym-base=0 represents the default and legacy behaviour.
           This will output sizes as decimal for numbers less than
           100000.  For sizes 100000 and greater hexadecimal notation
           will be used with a 0x prefix.  --sym-base=8 will give the
           symbol sizes in octal.  --sym-base=10 will always give the
           symbol sizes in decimal.  --sym-base=16 will always give the
           symbol sizes in hexadecimal with a 0x prefix.

       -C
       --demangle[=style]
           Decode (demangle) low-level symbol names into user-level
           names.  This makes C++ function names readable.  Different
           compilers have different mangling styles.  The optional
           demangling style argument can be used to choose an
           appropriate demangling style for your compiler.

       --no-demangle
           Do not demangle low-level symbol names.  This is the default.

       --recurse-limit
       --no-recurse-limit
       --recursion-limit
       --no-recursion-limit
           Enables or disables a limit on the amount of recursion
           performed whilst demangling strings.  Since the name mangling
           formats allow for an infinite level of recursion it is
           possible to create strings whose decoding will exhaust the
           amount of stack space available on the host machine,
           triggering a memory fault.  The limit tries to prevent this
           from happening by restricting recursion to 2048 levels of
           nesting.

           The default is for this limit to be enabled, but disabling it
           may be necessary in order to demangle truly complicated
           names.  Note however that if the recursion limit is disabled
           then stack exhaustion is possible and any bug reports about
           such an event will be rejected.

       -U [d|i|l|e|x|h]
       --unicode=[default|invalid|locale|escape|hex|highlight]
           Controls the display of non-ASCII characters in identifier
           names.  The default (--unicode=locale or --unicode=default)
           is to treat them as multibyte characters and display them in
           the current locale.  All other versions of this option treat
           the bytes as UTF-8 encoded values and attempt to interpret
           them.  If they cannot be interpreted or if the
           --unicode=invalid option is used then they are displayed as a
           sequence of hex bytes, encloses in curly parethesis
           characters.

           Using the --unicode=escape option will display the characters
           as as unicode escape sequences (\uxxxx).  Using the
           --unicode=hex will display the characters as hex byte
           sequences enclosed between angle brackets.

           Using the --unicode=highlight will display the characters as
           unicode escape sequences but it will also highlighted them in
           red, assuming that colouring is supported by the output
           device.  The colouring is intended to draw attention to the
           presence of unicode sequences when they might not be
           expected.

       -X
       --extra-sym-info
           When displaying details of symbols, include extra information
           not normally presented.  Currently this just adds the name of
           the section referenced by the symbol's index field, if there
           is one.  In the future more information may be displayed when
           this option is enabled.

           Enabling this option effectively enables the --wide option as
           well, at least when displaying symbol information.

       --no-extra-sym-info
           Disables the effect of the --extra-sym-info option.  This is
           the default.

       -e
       --headers
           Display all the headers in the file.  Equivalent to -h -l -S.

       -n
       --notes
           Displays the contents of the NOTE segments and/or sections,
           if any.

       -r
       --relocs
           Displays the contents of the file's relocation section, if it
           has one.

       -u
       --unwind
           Displays the contents of the file's unwind section, if it has
           one.  Only the unwind sections for IA64 ELF files, as well as
           ARM unwind tables (".ARM.exidx" / ".ARM.extab") are currently
           supported.  If support is not yet implemented for your
           architecture you could try dumping the contents of the
           .eh_frames section using the --debug-dump=frames or
           --debug-dump=frames-interp options.

       -d
       --dynamic
           Displays the contents of the file's dynamic section, if it
           has one.

       -V
       --version-info
           Displays the contents of the version sections in the file, it
           they exist.

       -A
       --arch-specific
           Displays architecture-specific information in the file, if
           there is any.

       -D
       --use-dynamic
           When displaying symbols, this option makes readelf use the
           symbol hash tables in the file's dynamic section, rather than
           the symbol table sections.

           When displaying relocations, this option makes readelf
           display the dynamic relocations rather than the static
           relocations.

       -L
       --lint
       --enable-checks
           Displays warning messages about possible problems with the
           file(s) being examined.  If used on its own then all of the
           contents of the file(s) will be examined.  If used with one
           of the dumping options then the warning messages will only be
           produced for the things being displayed.

       -x <number or name>
       --hex-dump=<number or name>
           Displays the contents of the indicated section as a
           hexadecimal bytes.  A number identifies a particular section
           by index in the section table; any other string identifies
           all sections with that name in the object file.

       -R <number or name>
       --relocated-dump=<number or name>
           Displays the contents of the indicated section as a
           hexadecimal bytes.  A number identifies a particular section
           by index in the section table; any other string identifies
           all sections with that name in the object file.  The contents
           of the section will be relocated before they are displayed.

       -p <number or name>
       --string-dump=<number or name>
           Displays the contents of the indicated section as printable
           strings.  A number identifies a particular section by index
           in the section table; any other string identifies all
           sections with that name in the object file.

       -z
       --decompress
           Requests that the section(s) being dumped by x, R or p
           options are decompressed before being displayed.  If the
           section(s) are not compressed then they are displayed as is.

       -c
       --archive-index
           Displays the file symbol index information contained in the
           header part of binary archives.  Performs the same function
           as the t command to ar, but without using the BFD library.

       -w[lLiaprmfFsOoRtUuTgAckK]
       --debug-dump[=rawline,=decodedline,=info,=abbrev,=pubnames,=aranges,=macro,=frames,=frames-interp,=str,=str-offsets,=loc,=Ranges,=pubtypes,=trace_info,=trace_abbrev,=trace_aranges,=gdb_index,=addr,=cu_index,=links,=follow-links]
           Displays the contents of the DWARF debug sections in the
           file, if any are present.  Compressed debug sections are
           automatically decompressed (temporarily) before they are
           displayed.  If one or more of the optional letters or words
           follows the switch then only those type(s) of data will be
           dumped.  The letters and words refer to the following
           information:

           "a"
           "=abbrev"
               Displays the contents of the .debug_abbrev section.

           "A"
           "=addr"
               Displays the contents of the .debug_addr section.

           "c"
           "=cu_index"
               Displays the contents of the .debug_cu_index and/or
               .debug_tu_index sections.

           "f"
           "=frames"
               Display the raw contents of a .debug_frame section.

           "F"
           "=frames-interp"
               Display the interpreted contents of a .debug_frame
               section.

           "g"
           "=gdb_index"
               Displays the contents of the .gdb_index and/or
               .debug_names sections.

           "i"
           "=info"
               Displays the contents of the .debug_info section.  Note:
               the output from this option can also be restricted by the
               use of the --dwarf-depth and --dwarf-start options.

           "k"
           "=links"
               Displays the contents of the .gnu_debuglink,
               .gnu_debugaltlink and .debug_sup sections, if any of them
               are present.  Also displays any links to separate dwarf
               object files (dwo), if they are specified by the
               DW_AT_GNU_dwo_name or DW_AT_dwo_name attributes in the
               .debug_info section.

           "K"
           "=follow-links"
               Display the contents of any selected debug sections that
               are found in linked, separate debug info file(s).  This
               can result in multiple versions of the same debug section
               being displayed if it exists in more than one file.

               In addition, when displaying DWARF attributes, if a form
               is found that references the separate debug info file,
               then the referenced contents will also be displayed.

               Note - in some distributions this option is enabled by
               default.  It can be disabled via the N debug option.  The
               default can be chosen when configuring the binutils via
               the --enable-follow-debug-links=yes or
               --enable-follow-debug-links=no options.  If these are not
               used then the default is to enable the following of debug
               links.

               Note - if support for the debuginfod protocol was enabled
               when the binutils were built then this option will also
               include an attempt to contact any debuginfod servers
               mentioned in the DEBUGINFOD_URLS environment variable.
               This could take some time to resolve.  This behaviour can
               be disabled via the =do-not-use-debuginfod debug option.

           "N"
           "=no-follow-links"
               Disables the following of links to separate debug info
               files.

           "D"
           "=use-debuginfod"
               Enables contacting debuginfod servers if there is a need
               to follow debug links.  This is the default behaviour.

           "E"
           "=do-not-use-debuginfod"
               Disables contacting debuginfod servers when there is a
               need to follow debug links.

           "l"
           "=rawline"
               Displays the contents of the .debug_line section in a raw
               format.

           "L"
           "=decodedline"
               Displays the interpreted contents of the .debug_line
               section.

           "m"
           "=macro"
               Displays the contents of the .debug_macro and/or
               .debug_macinfo sections.

           "o"
           "=loc"
               Displays the contents of the .debug_loc and/or
               .debug_loclists sections.

           "O"
           "=str-offsets"
               Displays the contents of the .debug_str_offsets section.

           "p"
           "=pubnames"
               Displays the contents of the .debug_pubnames and/or
               .debug_gnu_pubnames sections.

           "r"
           "=aranges"
               Displays the contents of the .debug_aranges section.

           "R"
           "=Ranges"
               Displays the contents of the .debug_ranges and/or
               .debug_rnglists sections.

           "s"
           "=str"
               Displays the contents of the .debug_str, .debug_line_str
               and/or .debug_str_offsets sections.

           "t"
           "=pubtype"
               Displays the contents of the .debug_pubtypes and/or
               .debug_gnu_pubtypes sections.

           "T"
           "=trace_aranges"
               Displays the contents of the .trace_aranges section.

           "u"
           "=trace_abbrev"
               Displays the contents of the .trace_abbrev section.

           "U"
           "=trace_info"
               Displays the contents of the .trace_info section.

           Note: displaying the contents of .debug_static_funcs,
           .debug_static_vars and debug_weaknames sections is not
           currently supported.

       --dwarf-depth=n
           Limit the dump of the ".debug_info" section to n children.
           This is only useful with --debug-dump=info.  The default is
           to print all DIEs; the special value 0 for n will also have
           this effect.

           With a non-zero value for n, DIEs at or deeper than n levels
           will not be printed.  The range for n is zero-based.

       --dwarf-start=n
           Print only DIEs beginning with the DIE numbered n.  This is
           only useful with --debug-dump=info.

           If specified, this option will suppress printing of any
           header information and all DIEs before the DIE numbered n.
           Only siblings and children of the specified DIE will be
           printed.

           This can be used in conjunction with --dwarf-depth.

       -P
       --process-links
           Display the contents of non-debug sections found in separate
           debuginfo files that are linked to the main file.  This
           option automatically implies the -wK option, and only
           sections requested by other command line options will be
           displayed.

       --ctf[=section]
           Display the contents of the specified CTF section.  CTF
           sections themselves contain many subsections, all of which
           are displayed in order.

           By default, display the name of the section named .ctf, which
           is the name emitted by ld.

       --ctf-parent=member
           If the CTF section contains ambiguously-defined types, it
           will consist of an archive of many CTF dictionaries, all
           inheriting from one dictionary containing unambiguous types.
           This member is by default named .ctf, like the section
           containing it, but it is possible to change this name using
           the "ctf_link_set_memb_name_changer" function at link time.
           When looking at CTF archives that have been created by a
           linker that uses the name changer to rename the parent
           archive member, --ctf-parent can be used to specify the name
           used for the parent.

       --ctf-symbols=section
       --ctf-strings=section
           Specify the name of another section from which the CTF file
           can inherit strings and symbols.  By default, the ".symtab"
           and its linked string table are used.

           If either of --ctf-symbols or --ctf-strings is specified, the
           other must be specified as well.

       -I
       --histogram
           Display a histogram of bucket list lengths when displaying
           the contents of the symbol tables.

       -v
       --version
           Display the version number of readelf.

       -W
       --wide
           Don't break output lines to fit into 80 columns. By default
           readelf breaks section header and segment listing lines for
           64-bit ELF files, so that they fit into 80 columns. This
           option causes readelf to print each section header resp. each
           segment one a single line, which is far more readable on
           terminals wider than 80 columns.

       -T
       --silent-truncation
           Normally when readelf is displaying a symbol name, and it has
           to truncate the name to fit into an 80 column display, it
           will add a suffix of "[...]" to the name.  This command line
           option disables this behaviour, allowing 5 more characters of
           the name to be displayed and restoring the old behaviour of
           readelf (prior to release 2.35).

       -H
       --help
           Display the command-line options understood by readelf.

       @file
           Read command-line options from file.  The options read are
           inserted in place of the original @file option.  If file does
           not exist, or cannot be read, then the option will be treated
           literally, and not removed.

           Options in file are separated by whitespace.  A whitespace
           character may be included in an option by surrounding the
           entire option in either single or double quotes.  Any
           character (including a backslash) may be included by
           prefixing the character to be included with a backslash.  The
           file may itself contain additional @file options; any such
           options will be processed recursively.
SEE ALSO
       objdump(1), and the Info entries for binutils.
COPYRIGHT
       Copyright (c) 1991-2024 Free Software Foundation, Inc.

       Permission is granted to copy, distribute and/or modify this
       document under the terms of the GNU Free Documentation License,
       Version 1.3 or any later version published by the Free Software
       Foundation; with no Invariant Sections, with no Front-Cover
       Texts, and with no Back-Cover Texts.  A copy of the license is
       included in the section entitled "GNU Free Documentation
       License".
COLOPHON
       This page is part of the binutils (a collection of tools for
       working with executable binaries) project.  Information about the
       project can be found at http://www.gnu.org/software/binutils/.
       If you have a bug report for this manual page, see
       http://sourceware.org/bugzilla/enter_bug.cgi?product=binutils.
       This page was obtained from the tarball binutils-2.42.tar.gz
       fetched from https://ftp.gnu.org/gnu/binutils/ on 2024-06-14.
       If you discover any rendering problems in this HTML version of
       the page, or you believe there is a better or more up-to-date
       source for the page, or you have corrections or improvements to
       the information in this COLOPHON (which is not part of the
       original manual page), send a mail to man-pages@man7.org

binutils-2.42                  2024-06-14                     READELF(1)
\end{lstlisting}
}}
\endinput  %  ==  ==  ==  ==  ==  ==  ==  ==  ==

\subsection{\refReadelf: Display Information On \elf \ Files}

{\tiny{
\begin{lstlisting}[language=bash]
NAME
       readelf - display information about ELF files
SYNOPSIS
       readelf [-a|--all]
               [-h|--file-header]
               [-l|--program-headers|--segments]
               [-S|--section-headers|--sections]
               [-g|--section-groups]
               [-t|--section-details]
               [-e|--headers]
               [-s|--syms|--symbols]
               [--dyn-syms|--lto-syms]
               [--sym-base=[0|8|10|16]]
               [--demangle=style|--no-demangle]
               [--quiet]
               [--recurse-limit|--no-recurse-limit]
               [-U method|--unicode=method]
               [-X|--extra-sym-info|--no-extra-sym-info]
               [-n|--notes]
               [-r|--relocs]
               [-u|--unwind]
               [-d|--dynamic]
               [-V|--version-info]
               [-A|--arch-specific]
               [-D|--use-dynamic]
               [-L|--lint|--enable-checks]
               [-x <number or name>|--hex-dump=<number or name>]
               [-p <number or name>|--string-dump=<number or name>]
               [-R <number or name>|--relocated-dump=<number or name>]
               [-z|--decompress]
               [-c|--archive-index]
               [-w[lLiaprmfFsoORtUuTgAck]|
                --debug-dump[=rawline,=decodedline,=info,=abbrev,=pubnames,=aranges,=macro,=frames,=frames-interp,=str,=str-offsets,=loc,=Ranges,=pubtypes,=trace_info,=trace_abbrev,=trace_aranges,=gdb_index,=addr,=cu_index,=links]]
               [-wK|--debug-dump=follow-links]
               [-wN|--debug-dump=no-follow-links]
               [-wD|--debug-dump=use-debuginfod]
               [-wE|--debug-dump=do-not-use-debuginfod]
               [-P|--process-links]
               [--dwarf-depth=n]
               [--dwarf-start=n]
               [--ctf=section]
               [--ctf-parent=section]
               [--ctf-symbols=section]
               [--ctf-strings=section]
               [--sframe=section]
               [-I|--histogram]
               [-v|--version]
               [-W|--wide]
               [-T|--silent-truncation]
               [-H|--help]
               elffile...
DESCRIPTION
       readelf displays information about one or more ELF format object
       files.  The options control what particular information to
       display.

       elffile... are the object files to be examined.  32-bit and
       64-bit ELF files are supported, as are archives containing ELF
       files.

       This program performs a similar function to objdump but it goes
       into more detail and it exists independently of the BFD library,
       so if there is a bug in BFD then readelf will not be affected.
OPTIONS
       The long and short forms of options, shown here as alternatives,
       are equivalent.  At least one option besides -v or -H must be
       given.

       -a
       --all
           Equivalent to specifying --file-header, --program-headers,
           --sections, --symbols, --relocs, --dynamic, --notes,
           --version-info, --arch-specific, --unwind, --section-groups
           and --histogram.

           Note - this option does not enable --use-dynamic itself, so
           if that option is not present on the command line then
           dynamic symbols and dynamic relocs will not be displayed.

       -h
       --file-header
           Displays the information contained in the ELF header at the
           start of the file.

       -l
       --program-headers
       --segments
           Displays the information contained in the file's segment
           headers, if it has any.

       --quiet
           Suppress "no symbols" diagnostic.

       -S
       --sections
       --section-headers
           Displays the information contained in the file's section
           headers, if it has any.

       -g
       --section-groups
           Displays the information contained in the file's section
           groups, if it has any.

       -t
       --section-details
           Displays the detailed section information. Implies -S.

       -s
       --symbols
       --syms
           Displays the entries in symbol table section of the file, if
           it has one.  If a symbol has version information associated
           with it then this is displayed as well.  The version string
           is displayed as a suffix to the symbol name, preceded by an @
           character.  For example foo@VER_1.  If the version is the
           default version to be used when resolving unversioned
           references to the symbol then it is displayed as a suffix
           preceded by two @ characters.  For example foo@@VER_2.

       --dyn-syms
           Displays the entries in dynamic symbol table section of the
           file, if it has one.  The output format is the same as the
           format used by the --syms option.

       --lto-syms
           Displays the contents of any LTO symbol tables in the file.

       --sym-base=[0|8|10|16]
           Forces the size field of the symbol table to use the given
           base.  Any unrecognized options will be treated as 0.
           --sym-base=0 represents the default and legacy behaviour.
           This will output sizes as decimal for numbers less than
           100000.  For sizes 100000 and greater hexadecimal notation
           will be used with a 0x prefix.  --sym-base=8 will give the
           symbol sizes in octal.  --sym-base=10 will always give the
           symbol sizes in decimal.  --sym-base=16 will always give the
           symbol sizes in hexadecimal with a 0x prefix.

       -C
       --demangle[=style]
           Decode (demangle) low-level symbol names into user-level
           names.  This makes C++ function names readable.  Different
           compilers have different mangling styles.  The optional
           demangling style argument can be used to choose an
           appropriate demangling style for your compiler.

       --no-demangle
           Do not demangle low-level symbol names.  This is the default.

       --recurse-limit
       --no-recurse-limit
       --recursion-limit
       --no-recursion-limit
           Enables or disables a limit on the amount of recursion
           performed whilst demangling strings.  Since the name mangling
           formats allow for an infinite level of recursion it is
           possible to create strings whose decoding will exhaust the
           amount of stack space available on the host machine,
           triggering a memory fault.  The limit tries to prevent this
           from happening by restricting recursion to 2048 levels of
           nesting.

           The default is for this limit to be enabled, but disabling it
           may be necessary in order to demangle truly complicated
           names.  Note however that if the recursion limit is disabled
           then stack exhaustion is possible and any bug reports about
           such an event will be rejected.

       -U [d|i|l|e|x|h]
       --unicode=[default|invalid|locale|escape|hex|highlight]
           Controls the display of non-ASCII characters in identifier
           names.  The default (--unicode=locale or --unicode=default)
           is to treat them as multibyte characters and display them in
           the current locale.  All other versions of this option treat
           the bytes as UTF-8 encoded values and attempt to interpret
           them.  If they cannot be interpreted or if the
           --unicode=invalid option is used then they are displayed as a
           sequence of hex bytes, encloses in curly parethesis
           characters.

           Using the --unicode=escape option will display the characters
           as as unicode escape sequences (\uxxxx).  Using the
           --unicode=hex will display the characters as hex byte
           sequences enclosed between angle brackets.

           Using the --unicode=highlight will display the characters as
           unicode escape sequences but it will also highlighted them in
           red, assuming that colouring is supported by the output
           device.  The colouring is intended to draw attention to the
           presence of unicode sequences when they might not be
           expected.

       -X
       --extra-sym-info
           When displaying details of symbols, include extra information
           not normally presented.  Currently this just adds the name of
           the section referenced by the symbol's index field, if there
           is one.  In the future more information may be displayed when
           this option is enabled.

           Enabling this option effectively enables the --wide option as
           well, at least when displaying symbol information.

       --no-extra-sym-info
           Disables the effect of the --extra-sym-info option.  This is
           the default.

       -e
       --headers
           Display all the headers in the file.  Equivalent to -h -l -S.

       -n
       --notes
           Displays the contents of the NOTE segments and/or sections,
           if any.

       -r
       --relocs
           Displays the contents of the file's relocation section, if it
           has one.

       -u
       --unwind
           Displays the contents of the file's unwind section, if it has
           one.  Only the unwind sections for IA64 ELF files, as well as
           ARM unwind tables (".ARM.exidx" / ".ARM.extab") are currently
           supported.  If support is not yet implemented for your
           architecture you could try dumping the contents of the
           .eh_frames section using the --debug-dump=frames or
           --debug-dump=frames-interp options.

       -d
       --dynamic
           Displays the contents of the file's dynamic section, if it
           has one.

       -V
       --version-info
           Displays the contents of the version sections in the file, it
           they exist.

       -A
       --arch-specific
           Displays architecture-specific information in the file, if
           there is any.

       -D
       --use-dynamic
           When displaying symbols, this option makes readelf use the
           symbol hash tables in the file's dynamic section, rather than
           the symbol table sections.

           When displaying relocations, this option makes readelf
           display the dynamic relocations rather than the static
           relocations.

       -L
       --lint
       --enable-checks
           Displays warning messages about possible problems with the
           file(s) being examined.  If used on its own then all of the
           contents of the file(s) will be examined.  If used with one
           of the dumping options then the warning messages will only be
           produced for the things being displayed.

       -x <number or name>
       --hex-dump=<number or name>
           Displays the contents of the indicated section as a
           hexadecimal bytes.  A number identifies a particular section
           by index in the section table; any other string identifies
           all sections with that name in the object file.

       -R <number or name>
       --relocated-dump=<number or name>
           Displays the contents of the indicated section as a
           hexadecimal bytes.  A number identifies a particular section
           by index in the section table; any other string identifies
           all sections with that name in the object file.  The contents
           of the section will be relocated before they are displayed.

       -p <number or name>
       --string-dump=<number or name>
           Displays the contents of the indicated section as printable
           strings.  A number identifies a particular section by index
           in the section table; any other string identifies all
           sections with that name in the object file.

       -z
       --decompress
           Requests that the section(s) being dumped by x, R or p
           options are decompressed before being displayed.  If the
           section(s) are not compressed then they are displayed as is.

       -c
       --archive-index
           Displays the file symbol index information contained in the
           header part of binary archives.  Performs the same function
           as the t command to ar, but without using the BFD library.

       -w[lLiaprmfFsOoRtUuTgAckK]
       --debug-dump[=rawline,=decodedline,=info,=abbrev,=pubnames,=aranges,=macro,=frames,=frames-interp,=str,=str-offsets,=loc,=Ranges,=pubtypes,=trace_info,=trace_abbrev,=trace_aranges,=gdb_index,=addr,=cu_index,=links,=follow-links]
           Displays the contents of the DWARF debug sections in the
           file, if any are present.  Compressed debug sections are
           automatically decompressed (temporarily) before they are
           displayed.  If one or more of the optional letters or words
           follows the switch then only those type(s) of data will be
           dumped.  The letters and words refer to the following
           information:

           "a"
           "=abbrev"
               Displays the contents of the .debug_abbrev section.

           "A"
           "=addr"
               Displays the contents of the .debug_addr section.

           "c"
           "=cu_index"
               Displays the contents of the .debug_cu_index and/or
               .debug_tu_index sections.

           "f"
           "=frames"
               Display the raw contents of a .debug_frame section.

           "F"
           "=frames-interp"
               Display the interpreted contents of a .debug_frame
               section.

           "g"
           "=gdb_index"
               Displays the contents of the .gdb_index and/or
               .debug_names sections.

           "i"
           "=info"
               Displays the contents of the .debug_info section.  Note:
               the output from this option can also be restricted by the
               use of the --dwarf-depth and --dwarf-start options.

           "k"
           "=links"
               Displays the contents of the .gnu_debuglink,
               .gnu_debugaltlink and .debug_sup sections, if any of them
               are present.  Also displays any links to separate dwarf
               object files (dwo), if they are specified by the
               DW_AT_GNU_dwo_name or DW_AT_dwo_name attributes in the
               .debug_info section.

           "K"
           "=follow-links"
               Display the contents of any selected debug sections that
               are found in linked, separate debug info file(s).  This
               can result in multiple versions of the same debug section
               being displayed if it exists in more than one file.

               In addition, when displaying DWARF attributes, if a form
               is found that references the separate debug info file,
               then the referenced contents will also be displayed.

               Note - in some distributions this option is enabled by
               default.  It can be disabled via the N debug option.  The
               default can be chosen when configuring the binutils via
               the --enable-follow-debug-links=yes or
               --enable-follow-debug-links=no options.  If these are not
               used then the default is to enable the following of debug
               links.

               Note - if support for the debuginfod protocol was enabled
               when the binutils were built then this option will also
               include an attempt to contact any debuginfod servers
               mentioned in the DEBUGINFOD_URLS environment variable.
               This could take some time to resolve.  This behaviour can
               be disabled via the =do-not-use-debuginfod debug option.

           "N"
           "=no-follow-links"
               Disables the following of links to separate debug info
               files.

           "D"
           "=use-debuginfod"
               Enables contacting debuginfod servers if there is a need
               to follow debug links.  This is the default behaviour.

           "E"
           "=do-not-use-debuginfod"
               Disables contacting debuginfod servers when there is a
               need to follow debug links.

           "l"
           "=rawline"
               Displays the contents of the .debug_line section in a raw
               format.

           "L"
           "=decodedline"
               Displays the interpreted contents of the .debug_line
               section.

           "m"
           "=macro"
               Displays the contents of the .debug_macro and/or
               .debug_macinfo sections.

           "o"
           "=loc"
               Displays the contents of the .debug_loc and/or
               .debug_loclists sections.

           "O"
           "=str-offsets"
               Displays the contents of the .debug_str_offsets section.

           "p"
           "=pubnames"
               Displays the contents of the .debug_pubnames and/or
               .debug_gnu_pubnames sections.

           "r"
           "=aranges"
               Displays the contents of the .debug_aranges section.

           "R"
           "=Ranges"
               Displays the contents of the .debug_ranges and/or
               .debug_rnglists sections.

           "s"
           "=str"
               Displays the contents of the .debug_str, .debug_line_str
               and/or .debug_str_offsets sections.

           "t"
           "=pubtype"
               Displays the contents of the .debug_pubtypes and/or
               .debug_gnu_pubtypes sections.

           "T"
           "=trace_aranges"
               Displays the contents of the .trace_aranges section.

           "u"
           "=trace_abbrev"
               Displays the contents of the .trace_abbrev section.

           "U"
           "=trace_info"
               Displays the contents of the .trace_info section.

           Note: displaying the contents of .debug_static_funcs,
           .debug_static_vars and debug_weaknames sections is not
           currently supported.

       --dwarf-depth=n
           Limit the dump of the ".debug_info" section to n children.
           This is only useful with --debug-dump=info.  The default is
           to print all DIEs; the special value 0 for n will also have
           this effect.

           With a non-zero value for n, DIEs at or deeper than n levels
           will not be printed.  The range for n is zero-based.

       --dwarf-start=n
           Print only DIEs beginning with the DIE numbered n.  This is
           only useful with --debug-dump=info.

           If specified, this option will suppress printing of any
           header information and all DIEs before the DIE numbered n.
           Only siblings and children of the specified DIE will be
           printed.

           This can be used in conjunction with --dwarf-depth.

       -P
       --process-links
           Display the contents of non-debug sections found in separate
           debuginfo files that are linked to the main file.  This
           option automatically implies the -wK option, and only
           sections requested by other command line options will be
           displayed.

       --ctf[=section]
           Display the contents of the specified CTF section.  CTF
           sections themselves contain many subsections, all of which
           are displayed in order.

           By default, display the name of the section named .ctf, which
           is the name emitted by ld.

       --ctf-parent=member
           If the CTF section contains ambiguously-defined types, it
           will consist of an archive of many CTF dictionaries, all
           inheriting from one dictionary containing unambiguous types.
           This member is by default named .ctf, like the section
           containing it, but it is possible to change this name using
           the "ctf_link_set_memb_name_changer" function at link time.
           When looking at CTF archives that have been created by a
           linker that uses the name changer to rename the parent
           archive member, --ctf-parent can be used to specify the name
           used for the parent.

       --ctf-symbols=section
       --ctf-strings=section
           Specify the name of another section from which the CTF file
           can inherit strings and symbols.  By default, the ".symtab"
           and its linked string table are used.

           If either of --ctf-symbols or --ctf-strings is specified, the
           other must be specified as well.

       -I
       --histogram
           Display a histogram of bucket list lengths when displaying
           the contents of the symbol tables.

       -v
       --version
           Display the version number of readelf.

       -W
       --wide
           Don't break output lines to fit into 80 columns. By default
           readelf breaks section header and segment listing lines for
           64-bit ELF files, so that they fit into 80 columns. This
           option causes readelf to print each section header resp. each
           segment one a single line, which is far more readable on
           terminals wider than 80 columns.

       -T
       --silent-truncation
           Normally when readelf is displaying a symbol name, and it has
           to truncate the name to fit into an 80 column display, it
           will add a suffix of "[...]" to the name.  This command line
           option disables this behaviour, allowing 5 more characters of
           the name to be displayed and restoring the old behaviour of
           readelf (prior to release 2.35).

       -H
       --help
           Display the command-line options understood by readelf.

       @file
           Read command-line options from file.  The options read are
           inserted in place of the original @file option.  If file does
           not exist, or cannot be read, then the option will be treated
           literally, and not removed.

           Options in file are separated by whitespace.  A whitespace
           character may be included in an option by surrounding the
           entire option in either single or double quotes.  Any
           character (including a backslash) may be included by
           prefixing the character to be included with a backslash.  The
           file may itself contain additional @file options; any such
           options will be processed recursively.
SEE ALSO
       objdump(1), and the Info entries for binutils.
COPYRIGHT
       Copyright (c) 1991-2024 Free Software Foundation, Inc.

       Permission is granted to copy, distribute and/or modify this
       document under the terms of the GNU Free Documentation License,
       Version 1.3 or any later version published by the Free Software
       Foundation; with no Invariant Sections, with no Front-Cover
       Texts, and with no Back-Cover Texts.  A copy of the license is
       included in the section entitled "GNU Free Documentation
       License".
COLOPHON
       This page is part of the binutils (a collection of tools for
       working with executable binaries) project.  Information about the
       project can be found at http://www.gnu.org/software/binutils/.
       If you have a bug report for this manual page, see
       http://sourceware.org/bugzilla/enter_bug.cgi?product=binutils.
       This page was obtained from the tarball binutils-2.42.tar.gz
       fetched from https://ftp.gnu.org/gnu/binutils/ on 2024-06-14.
       If you discover any rendering problems in this HTML version of
       the page, or you believe there is a better or more up-to-date
       source for the page, or you have corrections or improvements to
       the information in this COLOPHON (which is not part of the
       original manual page), send a mail to man-pages@man7.org

binutils-2.42                  2024-06-14                     READELF(1)
\end{lstlisting}
}}
\endinput  %  ==  ==  ==  ==  ==  ==  ==  ==  ==

\subsection{\refReadelf: Display Information On \elf \ Files}

{\tiny{
\begin{lstlisting}[language=bash]
NAME
       readelf - display information about ELF files
SYNOPSIS
       readelf [-a|--all]
               [-h|--file-header]
               [-l|--program-headers|--segments]
               [-S|--section-headers|--sections]
               [-g|--section-groups]
               [-t|--section-details]
               [-e|--headers]
               [-s|--syms|--symbols]
               [--dyn-syms|--lto-syms]
               [--sym-base=[0|8|10|16]]
               [--demangle=style|--no-demangle]
               [--quiet]
               [--recurse-limit|--no-recurse-limit]
               [-U method|--unicode=method]
               [-X|--extra-sym-info|--no-extra-sym-info]
               [-n|--notes]
               [-r|--relocs]
               [-u|--unwind]
               [-d|--dynamic]
               [-V|--version-info]
               [-A|--arch-specific]
               [-D|--use-dynamic]
               [-L|--lint|--enable-checks]
               [-x <number or name>|--hex-dump=<number or name>]
               [-p <number or name>|--string-dump=<number or name>]
               [-R <number or name>|--relocated-dump=<number or name>]
               [-z|--decompress]
               [-c|--archive-index]
               [-w[lLiaprmfFsoORtUuTgAck]|
                --debug-dump[=rawline,=decodedline,=info,=abbrev,=pubnames,=aranges,=macro,=frames,=frames-interp,=str,=str-offsets,=loc,=Ranges,=pubtypes,=trace_info,=trace_abbrev,=trace_aranges,=gdb_index,=addr,=cu_index,=links]]
               [-wK|--debug-dump=follow-links]
               [-wN|--debug-dump=no-follow-links]
               [-wD|--debug-dump=use-debuginfod]
               [-wE|--debug-dump=do-not-use-debuginfod]
               [-P|--process-links]
               [--dwarf-depth=n]
               [--dwarf-start=n]
               [--ctf=section]
               [--ctf-parent=section]
               [--ctf-symbols=section]
               [--ctf-strings=section]
               [--sframe=section]
               [-I|--histogram]
               [-v|--version]
               [-W|--wide]
               [-T|--silent-truncation]
               [-H|--help]
               elffile...
DESCRIPTION
       readelf displays information about one or more ELF format object
       files.  The options control what particular information to
       display.

       elffile... are the object files to be examined.  32-bit and
       64-bit ELF files are supported, as are archives containing ELF
       files.

       This program performs a similar function to objdump but it goes
       into more detail and it exists independently of the BFD library,
       so if there is a bug in BFD then readelf will not be affected.
OPTIONS
       The long and short forms of options, shown here as alternatives,
       are equivalent.  At least one option besides -v or -H must be
       given.

       -a
       --all
           Equivalent to specifying --file-header, --program-headers,
           --sections, --symbols, --relocs, --dynamic, --notes,
           --version-info, --arch-specific, --unwind, --section-groups
           and --histogram.

           Note - this option does not enable --use-dynamic itself, so
           if that option is not present on the command line then
           dynamic symbols and dynamic relocs will not be displayed.

       -h
       --file-header
           Displays the information contained in the ELF header at the
           start of the file.

       -l
       --program-headers
       --segments
           Displays the information contained in the file's segment
           headers, if it has any.

       --quiet
           Suppress "no symbols" diagnostic.

       -S
       --sections
       --section-headers
           Displays the information contained in the file's section
           headers, if it has any.

       -g
       --section-groups
           Displays the information contained in the file's section
           groups, if it has any.

       -t
       --section-details
           Displays the detailed section information. Implies -S.

       -s
       --symbols
       --syms
           Displays the entries in symbol table section of the file, if
           it has one.  If a symbol has version information associated
           with it then this is displayed as well.  The version string
           is displayed as a suffix to the symbol name, preceded by an @
           character.  For example foo@VER_1.  If the version is the
           default version to be used when resolving unversioned
           references to the symbol then it is displayed as a suffix
           preceded by two @ characters.  For example foo@@VER_2.

       --dyn-syms
           Displays the entries in dynamic symbol table section of the
           file, if it has one.  The output format is the same as the
           format used by the --syms option.

       --lto-syms
           Displays the contents of any LTO symbol tables in the file.

       --sym-base=[0|8|10|16]
           Forces the size field of the symbol table to use the given
           base.  Any unrecognized options will be treated as 0.
           --sym-base=0 represents the default and legacy behaviour.
           This will output sizes as decimal for numbers less than
           100000.  For sizes 100000 and greater hexadecimal notation
           will be used with a 0x prefix.  --sym-base=8 will give the
           symbol sizes in octal.  --sym-base=10 will always give the
           symbol sizes in decimal.  --sym-base=16 will always give the
           symbol sizes in hexadecimal with a 0x prefix.

       -C
       --demangle[=style]
           Decode (demangle) low-level symbol names into user-level
           names.  This makes C++ function names readable.  Different
           compilers have different mangling styles.  The optional
           demangling style argument can be used to choose an
           appropriate demangling style for your compiler.

       --no-demangle
           Do not demangle low-level symbol names.  This is the default.

       --recurse-limit
       --no-recurse-limit
       --recursion-limit
       --no-recursion-limit
           Enables or disables a limit on the amount of recursion
           performed whilst demangling strings.  Since the name mangling
           formats allow for an infinite level of recursion it is
           possible to create strings whose decoding will exhaust the
           amount of stack space available on the host machine,
           triggering a memory fault.  The limit tries to prevent this
           from happening by restricting recursion to 2048 levels of
           nesting.

           The default is for this limit to be enabled, but disabling it
           may be necessary in order to demangle truly complicated
           names.  Note however that if the recursion limit is disabled
           then stack exhaustion is possible and any bug reports about
           such an event will be rejected.

       -U [d|i|l|e|x|h]
       --unicode=[default|invalid|locale|escape|hex|highlight]
           Controls the display of non-ASCII characters in identifier
           names.  The default (--unicode=locale or --unicode=default)
           is to treat them as multibyte characters and display them in
           the current locale.  All other versions of this option treat
           the bytes as UTF-8 encoded values and attempt to interpret
           them.  If they cannot be interpreted or if the
           --unicode=invalid option is used then they are displayed as a
           sequence of hex bytes, encloses in curly parethesis
           characters.

           Using the --unicode=escape option will display the characters
           as as unicode escape sequences (\uxxxx).  Using the
           --unicode=hex will display the characters as hex byte
           sequences enclosed between angle brackets.

           Using the --unicode=highlight will display the characters as
           unicode escape sequences but it will also highlighted them in
           red, assuming that colouring is supported by the output
           device.  The colouring is intended to draw attention to the
           presence of unicode sequences when they might not be
           expected.

       -X
       --extra-sym-info
           When displaying details of symbols, include extra information
           not normally presented.  Currently this just adds the name of
           the section referenced by the symbol's index field, if there
           is one.  In the future more information may be displayed when
           this option is enabled.

           Enabling this option effectively enables the --wide option as
           well, at least when displaying symbol information.

       --no-extra-sym-info
           Disables the effect of the --extra-sym-info option.  This is
           the default.

       -e
       --headers
           Display all the headers in the file.  Equivalent to -h -l -S.

       -n
       --notes
           Displays the contents of the NOTE segments and/or sections,
           if any.

       -r
       --relocs
           Displays the contents of the file's relocation section, if it
           has one.

       -u
       --unwind
           Displays the contents of the file's unwind section, if it has
           one.  Only the unwind sections for IA64 ELF files, as well as
           ARM unwind tables (".ARM.exidx" / ".ARM.extab") are currently
           supported.  If support is not yet implemented for your
           architecture you could try dumping the contents of the
           .eh_frames section using the --debug-dump=frames or
           --debug-dump=frames-interp options.

       -d
       --dynamic
           Displays the contents of the file's dynamic section, if it
           has one.

       -V
       --version-info
           Displays the contents of the version sections in the file, it
           they exist.

       -A
       --arch-specific
           Displays architecture-specific information in the file, if
           there is any.

       -D
       --use-dynamic
           When displaying symbols, this option makes readelf use the
           symbol hash tables in the file's dynamic section, rather than
           the symbol table sections.

           When displaying relocations, this option makes readelf
           display the dynamic relocations rather than the static
           relocations.

       -L
       --lint
       --enable-checks
           Displays warning messages about possible problems with the
           file(s) being examined.  If used on its own then all of the
           contents of the file(s) will be examined.  If used with one
           of the dumping options then the warning messages will only be
           produced for the things being displayed.

       -x <number or name>
       --hex-dump=<number or name>
           Displays the contents of the indicated section as a
           hexadecimal bytes.  A number identifies a particular section
           by index in the section table; any other string identifies
           all sections with that name in the object file.

       -R <number or name>
       --relocated-dump=<number or name>
           Displays the contents of the indicated section as a
           hexadecimal bytes.  A number identifies a particular section
           by index in the section table; any other string identifies
           all sections with that name in the object file.  The contents
           of the section will be relocated before they are displayed.

       -p <number or name>
       --string-dump=<number or name>
           Displays the contents of the indicated section as printable
           strings.  A number identifies a particular section by index
           in the section table; any other string identifies all
           sections with that name in the object file.

       -z
       --decompress
           Requests that the section(s) being dumped by x, R or p
           options are decompressed before being displayed.  If the
           section(s) are not compressed then they are displayed as is.

       -c
       --archive-index
           Displays the file symbol index information contained in the
           header part of binary archives.  Performs the same function
           as the t command to ar, but without using the BFD library.

       -w[lLiaprmfFsOoRtUuTgAckK]
       --debug-dump[=rawline,=decodedline,=info,=abbrev,=pubnames,=aranges,=macro,=frames,=frames-interp,=str,=str-offsets,=loc,=Ranges,=pubtypes,=trace_info,=trace_abbrev,=trace_aranges,=gdb_index,=addr,=cu_index,=links,=follow-links]
           Displays the contents of the DWARF debug sections in the
           file, if any are present.  Compressed debug sections are
           automatically decompressed (temporarily) before they are
           displayed.  If one or more of the optional letters or words
           follows the switch then only those type(s) of data will be
           dumped.  The letters and words refer to the following
           information:

           "a"
           "=abbrev"
               Displays the contents of the .debug_abbrev section.

           "A"
           "=addr"
               Displays the contents of the .debug_addr section.

           "c"
           "=cu_index"
               Displays the contents of the .debug_cu_index and/or
               .debug_tu_index sections.

           "f"
           "=frames"
               Display the raw contents of a .debug_frame section.

           "F"
           "=frames-interp"
               Display the interpreted contents of a .debug_frame
               section.

           "g"
           "=gdb_index"
               Displays the contents of the .gdb_index and/or
               .debug_names sections.

           "i"
           "=info"
               Displays the contents of the .debug_info section.  Note:
               the output from this option can also be restricted by the
               use of the --dwarf-depth and --dwarf-start options.

           "k"
           "=links"
               Displays the contents of the .gnu_debuglink,
               .gnu_debugaltlink and .debug_sup sections, if any of them
               are present.  Also displays any links to separate dwarf
               object files (dwo), if they are specified by the
               DW_AT_GNU_dwo_name or DW_AT_dwo_name attributes in the
               .debug_info section.

           "K"
           "=follow-links"
               Display the contents of any selected debug sections that
               are found in linked, separate debug info file(s).  This
               can result in multiple versions of the same debug section
               being displayed if it exists in more than one file.

               In addition, when displaying DWARF attributes, if a form
               is found that references the separate debug info file,
               then the referenced contents will also be displayed.

               Note - in some distributions this option is enabled by
               default.  It can be disabled via the N debug option.  The
               default can be chosen when configuring the binutils via
               the --enable-follow-debug-links=yes or
               --enable-follow-debug-links=no options.  If these are not
               used then the default is to enable the following of debug
               links.

               Note - if support for the debuginfod protocol was enabled
               when the binutils were built then this option will also
               include an attempt to contact any debuginfod servers
               mentioned in the DEBUGINFOD_URLS environment variable.
               This could take some time to resolve.  This behaviour can
               be disabled via the =do-not-use-debuginfod debug option.

           "N"
           "=no-follow-links"
               Disables the following of links to separate debug info
               files.

           "D"
           "=use-debuginfod"
               Enables contacting debuginfod servers if there is a need
               to follow debug links.  This is the default behaviour.

           "E"
           "=do-not-use-debuginfod"
               Disables contacting debuginfod servers when there is a
               need to follow debug links.

           "l"
           "=rawline"
               Displays the contents of the .debug_line section in a raw
               format.

           "L"
           "=decodedline"
               Displays the interpreted contents of the .debug_line
               section.

           "m"
           "=macro"
               Displays the contents of the .debug_macro and/or
               .debug_macinfo sections.

           "o"
           "=loc"
               Displays the contents of the .debug_loc and/or
               .debug_loclists sections.

           "O"
           "=str-offsets"
               Displays the contents of the .debug_str_offsets section.

           "p"
           "=pubnames"
               Displays the contents of the .debug_pubnames and/or
               .debug_gnu_pubnames sections.

           "r"
           "=aranges"
               Displays the contents of the .debug_aranges section.

           "R"
           "=Ranges"
               Displays the contents of the .debug_ranges and/or
               .debug_rnglists sections.

           "s"
           "=str"
               Displays the contents of the .debug_str, .debug_line_str
               and/or .debug_str_offsets sections.

           "t"
           "=pubtype"
               Displays the contents of the .debug_pubtypes and/or
               .debug_gnu_pubtypes sections.

           "T"
           "=trace_aranges"
               Displays the contents of the .trace_aranges section.

           "u"
           "=trace_abbrev"
               Displays the contents of the .trace_abbrev section.

           "U"
           "=trace_info"
               Displays the contents of the .trace_info section.

           Note: displaying the contents of .debug_static_funcs,
           .debug_static_vars and debug_weaknames sections is not
           currently supported.

       --dwarf-depth=n
           Limit the dump of the ".debug_info" section to n children.
           This is only useful with --debug-dump=info.  The default is
           to print all DIEs; the special value 0 for n will also have
           this effect.

           With a non-zero value for n, DIEs at or deeper than n levels
           will not be printed.  The range for n is zero-based.

       --dwarf-start=n
           Print only DIEs beginning with the DIE numbered n.  This is
           only useful with --debug-dump=info.

           If specified, this option will suppress printing of any
           header information and all DIEs before the DIE numbered n.
           Only siblings and children of the specified DIE will be
           printed.

           This can be used in conjunction with --dwarf-depth.

       -P
       --process-links
           Display the contents of non-debug sections found in separate
           debuginfo files that are linked to the main file.  This
           option automatically implies the -wK option, and only
           sections requested by other command line options will be
           displayed.

       --ctf[=section]
           Display the contents of the specified CTF section.  CTF
           sections themselves contain many subsections, all of which
           are displayed in order.

           By default, display the name of the section named .ctf, which
           is the name emitted by ld.

       --ctf-parent=member
           If the CTF section contains ambiguously-defined types, it
           will consist of an archive of many CTF dictionaries, all
           inheriting from one dictionary containing unambiguous types.
           This member is by default named .ctf, like the section
           containing it, but it is possible to change this name using
           the "ctf_link_set_memb_name_changer" function at link time.
           When looking at CTF archives that have been created by a
           linker that uses the name changer to rename the parent
           archive member, --ctf-parent can be used to specify the name
           used for the parent.

       --ctf-symbols=section
       --ctf-strings=section
           Specify the name of another section from which the CTF file
           can inherit strings and symbols.  By default, the ".symtab"
           and its linked string table are used.

           If either of --ctf-symbols or --ctf-strings is specified, the
           other must be specified as well.

       -I
       --histogram
           Display a histogram of bucket list lengths when displaying
           the contents of the symbol tables.

       -v
       --version
           Display the version number of readelf.

       -W
       --wide
           Don't break output lines to fit into 80 columns. By default
           readelf breaks section header and segment listing lines for
           64-bit ELF files, so that they fit into 80 columns. This
           option causes readelf to print each section header resp. each
           segment one a single line, which is far more readable on
           terminals wider than 80 columns.

       -T
       --silent-truncation
           Normally when readelf is displaying a symbol name, and it has
           to truncate the name to fit into an 80 column display, it
           will add a suffix of "[...]" to the name.  This command line
           option disables this behaviour, allowing 5 more characters of
           the name to be displayed and restoring the old behaviour of
           readelf (prior to release 2.35).

       -H
       --help
           Display the command-line options understood by readelf.

       @file
           Read command-line options from file.  The options read are
           inserted in place of the original @file option.  If file does
           not exist, or cannot be read, then the option will be treated
           literally, and not removed.

           Options in file are separated by whitespace.  A whitespace
           character may be included in an option by surrounding the
           entire option in either single or double quotes.  Any
           character (including a backslash) may be included by
           prefixing the character to be included with a backslash.  The
           file may itself contain additional @file options; any such
           options will be processed recursively.
SEE ALSO
       objdump(1), and the Info entries for binutils.
COPYRIGHT
       Copyright (c) 1991-2024 Free Software Foundation, Inc.

       Permission is granted to copy, distribute and/or modify this
       document under the terms of the GNU Free Documentation License,
       Version 1.3 or any later version published by the Free Software
       Foundation; with no Invariant Sections, with no Front-Cover
       Texts, and with no Back-Cover Texts.  A copy of the license is
       included in the section entitled "GNU Free Documentation
       License".
COLOPHON
       This page is part of the binutils (a collection of tools for
       working with executable binaries) project.  Information about the
       project can be found at http://www.gnu.org/software/binutils/.
       If you have a bug report for this manual page, see
       http://sourceware.org/bugzilla/enter_bug.cgi?product=binutils.
       This page was obtained from the tarball binutils-2.42.tar.gz
       fetched from https://ftp.gnu.org/gnu/binutils/ on 2024-06-14.
       If you discover any rendering problems in this HTML version of
       the page, or you believe there is a better or more up-to-date
       source for the page, or you have corrections or improvements to
       the information in this COLOPHON (which is not part of the
       original manual page), send a mail to man-pages@man7.org

binutils-2.42                  2024-06-14                     READELF(1)
\end{lstlisting}
}}
\endinput  %  ==  ==  ==  ==  ==  ==  ==  ==  ==

\subsection{\refReadelf: Display Information On \elf \ Files}

{\tiny{
\begin{lstlisting}[language=bash]
NAME
       readelf - display information about ELF files
SYNOPSIS
       readelf [-a|--all]
               [-h|--file-header]
               [-l|--program-headers|--segments]
               [-S|--section-headers|--sections]
               [-g|--section-groups]
               [-t|--section-details]
               [-e|--headers]
               [-s|--syms|--symbols]
               [--dyn-syms|--lto-syms]
               [--sym-base=[0|8|10|16]]
               [--demangle=style|--no-demangle]
               [--quiet]
               [--recurse-limit|--no-recurse-limit]
               [-U method|--unicode=method]
               [-X|--extra-sym-info|--no-extra-sym-info]
               [-n|--notes]
               [-r|--relocs]
               [-u|--unwind]
               [-d|--dynamic]
               [-V|--version-info]
               [-A|--arch-specific]
               [-D|--use-dynamic]
               [-L|--lint|--enable-checks]
               [-x <number or name>|--hex-dump=<number or name>]
               [-p <number or name>|--string-dump=<number or name>]
               [-R <number or name>|--relocated-dump=<number or name>]
               [-z|--decompress]
               [-c|--archive-index]
               [-w[lLiaprmfFsoORtUuTgAck]|
                --debug-dump[=rawline,=decodedline,=info,=abbrev,=pubnames,=aranges,=macro,=frames,=frames-interp,=str,=str-offsets,=loc,=Ranges,=pubtypes,=trace_info,=trace_abbrev,=trace_aranges,=gdb_index,=addr,=cu_index,=links]]
               [-wK|--debug-dump=follow-links]
               [-wN|--debug-dump=no-follow-links]
               [-wD|--debug-dump=use-debuginfod]
               [-wE|--debug-dump=do-not-use-debuginfod]
               [-P|--process-links]
               [--dwarf-depth=n]
               [--dwarf-start=n]
               [--ctf=section]
               [--ctf-parent=section]
               [--ctf-symbols=section]
               [--ctf-strings=section]
               [--sframe=section]
               [-I|--histogram]
               [-v|--version]
               [-W|--wide]
               [-T|--silent-truncation]
               [-H|--help]
               elffile...
DESCRIPTION
       readelf displays information about one or more ELF format object
       files.  The options control what particular information to
       display.

       elffile... are the object files to be examined.  32-bit and
       64-bit ELF files are supported, as are archives containing ELF
       files.

       This program performs a similar function to objdump but it goes
       into more detail and it exists independently of the BFD library,
       so if there is a bug in BFD then readelf will not be affected.
OPTIONS
       The long and short forms of options, shown here as alternatives,
       are equivalent.  At least one option besides -v or -H must be
       given.

       -a
       --all
           Equivalent to specifying --file-header, --program-headers,
           --sections, --symbols, --relocs, --dynamic, --notes,
           --version-info, --arch-specific, --unwind, --section-groups
           and --histogram.

           Note - this option does not enable --use-dynamic itself, so
           if that option is not present on the command line then
           dynamic symbols and dynamic relocs will not be displayed.

       -h
       --file-header
           Displays the information contained in the ELF header at the
           start of the file.

       -l
       --program-headers
       --segments
           Displays the information contained in the file's segment
           headers, if it has any.

       --quiet
           Suppress "no symbols" diagnostic.

       -S
       --sections
       --section-headers
           Displays the information contained in the file's section
           headers, if it has any.

       -g
       --section-groups
           Displays the information contained in the file's section
           groups, if it has any.

       -t
       --section-details
           Displays the detailed section information. Implies -S.

       -s
       --symbols
       --syms
           Displays the entries in symbol table section of the file, if
           it has one.  If a symbol has version information associated
           with it then this is displayed as well.  The version string
           is displayed as a suffix to the symbol name, preceded by an @
           character.  For example foo@VER_1.  If the version is the
           default version to be used when resolving unversioned
           references to the symbol then it is displayed as a suffix
           preceded by two @ characters.  For example foo@@VER_2.

       --dyn-syms
           Displays the entries in dynamic symbol table section of the
           file, if it has one.  The output format is the same as the
           format used by the --syms option.

       --lto-syms
           Displays the contents of any LTO symbol tables in the file.

       --sym-base=[0|8|10|16]
           Forces the size field of the symbol table to use the given
           base.  Any unrecognized options will be treated as 0.
           --sym-base=0 represents the default and legacy behaviour.
           This will output sizes as decimal for numbers less than
           100000.  For sizes 100000 and greater hexadecimal notation
           will be used with a 0x prefix.  --sym-base=8 will give the
           symbol sizes in octal.  --sym-base=10 will always give the
           symbol sizes in decimal.  --sym-base=16 will always give the
           symbol sizes in hexadecimal with a 0x prefix.

       -C
       --demangle[=style]
           Decode (demangle) low-level symbol names into user-level
           names.  This makes C++ function names readable.  Different
           compilers have different mangling styles.  The optional
           demangling style argument can be used to choose an
           appropriate demangling style for your compiler.

       --no-demangle
           Do not demangle low-level symbol names.  This is the default.

       --recurse-limit
       --no-recurse-limit
       --recursion-limit
       --no-recursion-limit
           Enables or disables a limit on the amount of recursion
           performed whilst demangling strings.  Since the name mangling
           formats allow for an infinite level of recursion it is
           possible to create strings whose decoding will exhaust the
           amount of stack space available on the host machine,
           triggering a memory fault.  The limit tries to prevent this
           from happening by restricting recursion to 2048 levels of
           nesting.

           The default is for this limit to be enabled, but disabling it
           may be necessary in order to demangle truly complicated
           names.  Note however that if the recursion limit is disabled
           then stack exhaustion is possible and any bug reports about
           such an event will be rejected.

       -U [d|i|l|e|x|h]
       --unicode=[default|invalid|locale|escape|hex|highlight]
           Controls the display of non-ASCII characters in identifier
           names.  The default (--unicode=locale or --unicode=default)
           is to treat them as multibyte characters and display them in
           the current locale.  All other versions of this option treat
           the bytes as UTF-8 encoded values and attempt to interpret
           them.  If they cannot be interpreted or if the
           --unicode=invalid option is used then they are displayed as a
           sequence of hex bytes, encloses in curly parethesis
           characters.

           Using the --unicode=escape option will display the characters
           as as unicode escape sequences (\uxxxx).  Using the
           --unicode=hex will display the characters as hex byte
           sequences enclosed between angle brackets.

           Using the --unicode=highlight will display the characters as
           unicode escape sequences but it will also highlighted them in
           red, assuming that colouring is supported by the output
           device.  The colouring is intended to draw attention to the
           presence of unicode sequences when they might not be
           expected.

       -X
       --extra-sym-info
           When displaying details of symbols, include extra information
           not normally presented.  Currently this just adds the name of
           the section referenced by the symbol's index field, if there
           is one.  In the future more information may be displayed when
           this option is enabled.

           Enabling this option effectively enables the --wide option as
           well, at least when displaying symbol information.

       --no-extra-sym-info
           Disables the effect of the --extra-sym-info option.  This is
           the default.

       -e
       --headers
           Display all the headers in the file.  Equivalent to -h -l -S.

       -n
       --notes
           Displays the contents of the NOTE segments and/or sections,
           if any.

       -r
       --relocs
           Displays the contents of the file's relocation section, if it
           has one.

       -u
       --unwind
           Displays the contents of the file's unwind section, if it has
           one.  Only the unwind sections for IA64 ELF files, as well as
           ARM unwind tables (".ARM.exidx" / ".ARM.extab") are currently
           supported.  If support is not yet implemented for your
           architecture you could try dumping the contents of the
           .eh_frames section using the --debug-dump=frames or
           --debug-dump=frames-interp options.

       -d
       --dynamic
           Displays the contents of the file's dynamic section, if it
           has one.

       -V
       --version-info
           Displays the contents of the version sections in the file, it
           they exist.

       -A
       --arch-specific
           Displays architecture-specific information in the file, if
           there is any.

       -D
       --use-dynamic
           When displaying symbols, this option makes readelf use the
           symbol hash tables in the file's dynamic section, rather than
           the symbol table sections.

           When displaying relocations, this option makes readelf
           display the dynamic relocations rather than the static
           relocations.

       -L
       --lint
       --enable-checks
           Displays warning messages about possible problems with the
           file(s) being examined.  If used on its own then all of the
           contents of the file(s) will be examined.  If used with one
           of the dumping options then the warning messages will only be
           produced for the things being displayed.

       -x <number or name>
       --hex-dump=<number or name>
           Displays the contents of the indicated section as a
           hexadecimal bytes.  A number identifies a particular section
           by index in the section table; any other string identifies
           all sections with that name in the object file.

       -R <number or name>
       --relocated-dump=<number or name>
           Displays the contents of the indicated section as a
           hexadecimal bytes.  A number identifies a particular section
           by index in the section table; any other string identifies
           all sections with that name in the object file.  The contents
           of the section will be relocated before they are displayed.

       -p <number or name>
       --string-dump=<number or name>
           Displays the contents of the indicated section as printable
           strings.  A number identifies a particular section by index
           in the section table; any other string identifies all
           sections with that name in the object file.

       -z
       --decompress
           Requests that the section(s) being dumped by x, R or p
           options are decompressed before being displayed.  If the
           section(s) are not compressed then they are displayed as is.

       -c
       --archive-index
           Displays the file symbol index information contained in the
           header part of binary archives.  Performs the same function
           as the t command to ar, but without using the BFD library.

       -w[lLiaprmfFsOoRtUuTgAckK]
       --debug-dump[=rawline,=decodedline,=info,=abbrev,=pubnames,=aranges,=macro,=frames,=frames-interp,=str,=str-offsets,=loc,=Ranges,=pubtypes,=trace_info,=trace_abbrev,=trace_aranges,=gdb_index,=addr,=cu_index,=links,=follow-links]
           Displays the contents of the DWARF debug sections in the
           file, if any are present.  Compressed debug sections are
           automatically decompressed (temporarily) before they are
           displayed.  If one or more of the optional letters or words
           follows the switch then only those type(s) of data will be
           dumped.  The letters and words refer to the following
           information:

           "a"
           "=abbrev"
               Displays the contents of the .debug_abbrev section.

           "A"
           "=addr"
               Displays the contents of the .debug_addr section.

           "c"
           "=cu_index"
               Displays the contents of the .debug_cu_index and/or
               .debug_tu_index sections.

           "f"
           "=frames"
               Display the raw contents of a .debug_frame section.

           "F"
           "=frames-interp"
               Display the interpreted contents of a .debug_frame
               section.

           "g"
           "=gdb_index"
               Displays the contents of the .gdb_index and/or
               .debug_names sections.

           "i"
           "=info"
               Displays the contents of the .debug_info section.  Note:
               the output from this option can also be restricted by the
               use of the --dwarf-depth and --dwarf-start options.

           "k"
           "=links"
               Displays the contents of the .gnu_debuglink,
               .gnu_debugaltlink and .debug_sup sections, if any of them
               are present.  Also displays any links to separate dwarf
               object files (dwo), if they are specified by the
               DW_AT_GNU_dwo_name or DW_AT_dwo_name attributes in the
               .debug_info section.

           "K"
           "=follow-links"
               Display the contents of any selected debug sections that
               are found in linked, separate debug info file(s).  This
               can result in multiple versions of the same debug section
               being displayed if it exists in more than one file.

               In addition, when displaying DWARF attributes, if a form
               is found that references the separate debug info file,
               then the referenced contents will also be displayed.

               Note - in some distributions this option is enabled by
               default.  It can be disabled via the N debug option.  The
               default can be chosen when configuring the binutils via
               the --enable-follow-debug-links=yes or
               --enable-follow-debug-links=no options.  If these are not
               used then the default is to enable the following of debug
               links.

               Note - if support for the debuginfod protocol was enabled
               when the binutils were built then this option will also
               include an attempt to contact any debuginfod servers
               mentioned in the DEBUGINFOD_URLS environment variable.
               This could take some time to resolve.  This behaviour can
               be disabled via the =do-not-use-debuginfod debug option.

           "N"
           "=no-follow-links"
               Disables the following of links to separate debug info
               files.

           "D"
           "=use-debuginfod"
               Enables contacting debuginfod servers if there is a need
               to follow debug links.  This is the default behaviour.

           "E"
           "=do-not-use-debuginfod"
               Disables contacting debuginfod servers when there is a
               need to follow debug links.

           "l"
           "=rawline"
               Displays the contents of the .debug_line section in a raw
               format.

           "L"
           "=decodedline"
               Displays the interpreted contents of the .debug_line
               section.

           "m"
           "=macro"
               Displays the contents of the .debug_macro and/or
               .debug_macinfo sections.

           "o"
           "=loc"
               Displays the contents of the .debug_loc and/or
               .debug_loclists sections.

           "O"
           "=str-offsets"
               Displays the contents of the .debug_str_offsets section.

           "p"
           "=pubnames"
               Displays the contents of the .debug_pubnames and/or
               .debug_gnu_pubnames sections.

           "r"
           "=aranges"
               Displays the contents of the .debug_aranges section.

           "R"
           "=Ranges"
               Displays the contents of the .debug_ranges and/or
               .debug_rnglists sections.

           "s"
           "=str"
               Displays the contents of the .debug_str, .debug_line_str
               and/or .debug_str_offsets sections.

           "t"
           "=pubtype"
               Displays the contents of the .debug_pubtypes and/or
               .debug_gnu_pubtypes sections.

           "T"
           "=trace_aranges"
               Displays the contents of the .trace_aranges section.

           "u"
           "=trace_abbrev"
               Displays the contents of the .trace_abbrev section.

           "U"
           "=trace_info"
               Displays the contents of the .trace_info section.

           Note: displaying the contents of .debug_static_funcs,
           .debug_static_vars and debug_weaknames sections is not
           currently supported.

       --dwarf-depth=n
           Limit the dump of the ".debug_info" section to n children.
           This is only useful with --debug-dump=info.  The default is
           to print all DIEs; the special value 0 for n will also have
           this effect.

           With a non-zero value for n, DIEs at or deeper than n levels
           will not be printed.  The range for n is zero-based.

       --dwarf-start=n
           Print only DIEs beginning with the DIE numbered n.  This is
           only useful with --debug-dump=info.

           If specified, this option will suppress printing of any
           header information and all DIEs before the DIE numbered n.
           Only siblings and children of the specified DIE will be
           printed.

           This can be used in conjunction with --dwarf-depth.

       -P
       --process-links
           Display the contents of non-debug sections found in separate
           debuginfo files that are linked to the main file.  This
           option automatically implies the -wK option, and only
           sections requested by other command line options will be
           displayed.

       --ctf[=section]
           Display the contents of the specified CTF section.  CTF
           sections themselves contain many subsections, all of which
           are displayed in order.

           By default, display the name of the section named .ctf, which
           is the name emitted by ld.

       --ctf-parent=member
           If the CTF section contains ambiguously-defined types, it
           will consist of an archive of many CTF dictionaries, all
           inheriting from one dictionary containing unambiguous types.
           This member is by default named .ctf, like the section
           containing it, but it is possible to change this name using
           the "ctf_link_set_memb_name_changer" function at link time.
           When looking at CTF archives that have been created by a
           linker that uses the name changer to rename the parent
           archive member, --ctf-parent can be used to specify the name
           used for the parent.

       --ctf-symbols=section
       --ctf-strings=section
           Specify the name of another section from which the CTF file
           can inherit strings and symbols.  By default, the ".symtab"
           and its linked string table are used.

           If either of --ctf-symbols or --ctf-strings is specified, the
           other must be specified as well.

       -I
       --histogram
           Display a histogram of bucket list lengths when displaying
           the contents of the symbol tables.

       -v
       --version
           Display the version number of readelf.

       -W
       --wide
           Don't break output lines to fit into 80 columns. By default
           readelf breaks section header and segment listing lines for
           64-bit ELF files, so that they fit into 80 columns. This
           option causes readelf to print each section header resp. each
           segment one a single line, which is far more readable on
           terminals wider than 80 columns.

       -T
       --silent-truncation
           Normally when readelf is displaying a symbol name, and it has
           to truncate the name to fit into an 80 column display, it
           will add a suffix of "[...]" to the name.  This command line
           option disables this behaviour, allowing 5 more characters of
           the name to be displayed and restoring the old behaviour of
           readelf (prior to release 2.35).

       -H
       --help
           Display the command-line options understood by readelf.

       @file
           Read command-line options from file.  The options read are
           inserted in place of the original @file option.  If file does
           not exist, or cannot be read, then the option will be treated
           literally, and not removed.

           Options in file are separated by whitespace.  A whitespace
           character may be included in an option by surrounding the
           entire option in either single or double quotes.  Any
           character (including a backslash) may be included by
           prefixing the character to be included with a backslash.  The
           file may itself contain additional @file options; any such
           options will be processed recursively.
SEE ALSO
       objdump(1), and the Info entries for binutils.
COPYRIGHT
       Copyright (c) 1991-2024 Free Software Foundation, Inc.

       Permission is granted to copy, distribute and/or modify this
       document under the terms of the GNU Free Documentation License,
       Version 1.3 or any later version published by the Free Software
       Foundation; with no Invariant Sections, with no Front-Cover
       Texts, and with no Back-Cover Texts.  A copy of the license is
       included in the section entitled "GNU Free Documentation
       License".
COLOPHON
       This page is part of the binutils (a collection of tools for
       working with executable binaries) project.  Information about the
       project can be found at http://www.gnu.org/software/binutils/.
       If you have a bug report for this manual page, see
       http://sourceware.org/bugzilla/enter_bug.cgi?product=binutils.
       This page was obtained from the tarball binutils-2.42.tar.gz
       fetched from https://ftp.gnu.org/gnu/binutils/ on 2024-06-14.
       If you discover any rendering problems in this HTML version of
       the page, or you believe there is a better or more up-to-date
       source for the page, or you have corrections or improvements to
       the information in this COLOPHON (which is not part of the
       original manual page), send a mail to man-pages@man7.org

binutils-2.42                  2024-06-14                     READELF(1)
\end{lstlisting}
}}
\endinput  %  ==  ==  ==  ==  ==  ==  ==  ==  ==
