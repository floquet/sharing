% % % % \input{./components/man/man-objdump}
\subsection{\refObjdump: Display Information From Object Files}

{\tiny{
\begin{lstlisting}[language=bash]
NAME
       objdump - display information from object files
SYNOPSIS
       objdump [-a|--archive-headers]
               [-b bfdname|--target=bfdname]
               [-C|--demangle[=style] ]
               [-d|--disassemble[=symbol]]
               [-D|--disassemble-all]
               [-z|--disassemble-zeroes]
               [-EB|-EL|--endian={big | little }]
               [-f|--file-headers]
               [-F|--file-offsets]
               [--file-start-context]
               [-g|--debugging]
               [-e|--debugging-tags]
               [-h|--section-headers|--headers]
               [-i|--info]
               [-j section|--section=section]
               [-l|--line-numbers]
               [-S|--source]
               [--source-comment[=text]]
               [-m machine|--architecture=machine]
               [-M options|--disassembler-options=options]
               [-p|--private-headers]
               [-P options|--private=options]
               [-r|--reloc]
               [-R|--dynamic-reloc]
               [-s|--full-contents]
               [-Z|--decompress]
               [-W[lLiaprmfFsoORtUuTgAck]|
                --dwarf[=rawline,=decodedline,=info,=abbrev,=pubnames,=aranges,=macro,=frames,=frames-interp,=str,=str-offsets,=loc,=Ranges,=pubtypes,=trace_info,=trace_abbrev,=trace_aranges,=gdb_index,=addr,=cu_index,=links]]
               [-WK|--dwarf=follow-links]
               [-WN|--dwarf=no-follow-links]
               [-wD|--dwarf=use-debuginfod]
               [-wE|--dwarf=do-not-use-debuginfod]
               [-L|--process-links]
               [--ctf=section]
               [--sframe=section]
               [-G|--stabs]
               [-t|--syms]
               [-T|--dynamic-syms]
               [-x|--all-headers]
               [-w|--wide]
               [--start-address=address]
               [--stop-address=address]
               [--no-addresses]
               [--prefix-addresses]
               [--[no-]show-raw-insn]
               [--adjust-vma=offset]
               [--show-all-symbols]
               [--dwarf-depth=n]
               [--dwarf-start=n]
               [--ctf-parent=section]
               [--no-recurse-limit|--recurse-limit]
               [--special-syms]
               [--prefix=prefix]
               [--prefix-strip=level]
               [--insn-width=width]
               [--visualize-jumps[=color|=extended-color|=off]
               [--disassembler-color=[off|terminal|on|extended]
               [-U method] [--unicode=method]
               [-V|--version]
               [-H|--help]
               objfile...
DESCRIPTION
       objdump displays information about one or more object files.  The
       options control what particular information to display.  This
       information is mostly useful to programmers who are working on
       the compilation tools, as opposed to programmers who just want
       their program to compile and work.

       objfile... are the object files to be examined.  When you specify
       archives, objdump shows information on each of the member object
       files.
OPTIONS
       The long and short forms of options, shown here as alternatives,
       are equivalent.  At least one option from the list
       -a,-d,-D,-e,-f,-g,-G,-h,-H,-p,-P,-r,-R,-s,-S,-t,-T,-V,-x must be
       given.

       -a
       --archive-header
           If any of the objfile files are archives, display the archive
           header information (in a format similar to ls -l).  Besides
           the information you could list with ar tv, objdump -a shows
           the object file format of each archive member.

       --adjust-vma=offset
           When dumping information, first add offset to all the section
           addresses.  This is useful if the section addresses do not
           correspond to the symbol table, which can happen when putting
           sections at particular addresses when using a format which
           can not represent section addresses, such as a.out.

       -b bfdname
       --target=bfdname
           Specify that the object-code format for the object files is
           bfdname.  This option may not be necessary; objdump can
           automatically recognize many formats.

           For example,

                   objdump -b oasys -m vax -h fu.o

           displays summary information from the section headers (-h) of
           fu.o, which is explicitly identified (-m) as a VAX object
           file in the format produced by Oasys compilers.  You can list
           the formats available with the -i option.

       -C
       --demangle[=style]
           Decode (demangle) low-level symbol names into user-level
           names.  Besides removing any initial underscore prepended by
           the system, this makes C++ function names readable.
           Different compilers have different mangling styles. The
           optional demangling style argument can be used to choose an
           appropriate demangling style for your compiler.

       --recurse-limit
       --no-recurse-limit
       --recursion-limit
       --no-recursion-limit
           Enables or disables a limit on the amount of recursion
           performed whilst demangling strings.  Since the name mangling
           formats allow for an infinite level of recursion it is
           possible to create strings whose decoding will exhaust the
           amount of stack space available on the host machine,
           triggering a memory fault.  The limit tries to prevent this
           from happening by restricting recursion to 2048 levels of
           nesting.

           The default is for this limit to be enabled, but disabling it
           may be necessary in order to demangle truly complicated
           names.  Note however that if the recursion limit is disabled
           then stack exhaustion is possible and any bug reports about
           such an event will be rejected.

       -g
       --debugging
           Display debugging information.  This attempts to parse STABS
           debugging format information stored in the file and print it
           out using a C like syntax.  If no STABS debugging was found
           this option falls back on the -W option to print any DWARF
           information in the file.

       -e
       --debugging-tags
           Like -g, but the information is generated in a format
           compatible with ctags tool.

       -d
       --disassemble
       --disassemble=symbol
           Display the assembler mnemonics for the machine instructions
           from the input file.  This option only disassembles those
           sections which are expected to contain instructions.  If the
           optional symbol argument is given, then display the assembler
           mnemonics starting at symbol.  If symbol is a function name
           then disassembly will stop at the end of the function,
           otherwise it will stop when the next symbol is encountered.
           If there are no matches for symbol then nothing will be
           displayed.

           Note if the --dwarf=follow-links option is enabled then any
           symbol tables in linked debug info files will be read in and
           used when disassembling.

       -D
       --disassemble-all
           Like -d, but disassemble the contents of all non-empty non-
           bss sections, not just those expected to contain
           instructions.  -j may be used to select specific sections.

           This option also has a subtle effect on the disassembly of
           instructions in code sections.  When option -d is in effect
           objdump will assume that any symbols present in a code
           section occur on the boundary between instructions and it
           will refuse to disassemble across such a boundary.  When
           option -D is in effect however this assumption is supressed.
           This means that it is possible for the output of -d and -D to
           differ if, for example, data is stored in code sections.

           If the target is an ARM architecture this switch also has the
           effect of forcing the disassembler to decode pieces of data
           found in code sections as if they were instructions.

           Note if the --dwarf=follow-links option is enabled then any
           symbol tables in linked debug info files will be read in and
           used when disassembling.

       --no-addresses
           When disassembling, don't print addresses on each line or for
           symbols and relocation offsets.  In combination with
           --no-show-raw-insn this may be useful for comparing compiler
           output.

       --prefix-addresses
           When disassembling, print the complete address on each line.
           This is the older disassembly format.

       -EB
       -EL
       --endian={big|little}
           Specify the endianness of the object files.  This only
           affects disassembly.  This can be useful when disassembling a
           file format which does not describe endianness information,
           such as S-records.

       -f
       --file-headers
           Display summary information from the overall header of each
           of the objfile files.

       -F
       --file-offsets
           When disassembling sections, whenever a symbol is displayed,
           also display the file offset of the region of data that is
           about to be dumped.  If zeroes are being skipped, then when
           disassembly resumes, tell the user how many zeroes were
           skipped and the file offset of the location from where the
           disassembly resumes.  When dumping sections, display the file
           offset of the location from where the dump starts.

       --file-start-context
           Specify that when displaying interlisted source
           code/disassembly (assumes -S) from a file that has not yet
           been displayed, extend the context to the start of the file.

       -h
       --section-headers
       --headers
           Display summary information from the section headers of the
           object file.

           File segments may be relocated to nonstandard addresses, for
           example by using the -Ttext, -Tdata, or -Tbss options to ld.
           However, some object file formats, such as a.out, do not
           store the starting address of the file segments.  In those
           situations, although ld relocates the sections correctly,
           using objdump -h to list the file section headers cannot show
           the correct addresses.  Instead, it shows the usual
           addresses, which are implicit for the target.

           Note, in some cases it is possible for a section to have both
           the READONLY and the NOREAD attributes set.  In such cases
           the NOREAD attribute takes precedence, but objdump will
           report both since the exact setting of the flag bits might be
           important.

       -H
       --help
           Print a summary of the options to objdump and exit.

       -i
       --info
           Display a list showing all architectures and object formats
           available for specification with -b or -m.

       -j name
       --section=name
           Display information for section name.  This option may be
           specified multiple times.

       -L
       --process-links
           Display the contents of non-debug sections found in separate
           debuginfo files that are linked to the main file.  This
           option automatically implies the -WK option, and only
           sections requested by other command line options will be
           displayed.

       -l
       --line-numbers
           Label the display (using debugging information) with the
           filename and source line numbers corresponding to the object
           code or relocs shown.  Only useful with -d, -D, or -r.

       -m machine
       --architecture=machine
           Specify the architecture to use when disassembling object
           files.  This can be useful when disassembling object files
           which do not describe architecture information, such as
           S-records.  You can list the available architectures with the
           -i option.

           For most architectures it is possible to supply an
           architecture name and a machine name, separated by a colon.
           For example foo:bar would refer to the bar machine type in
           the foo architecture.  This can be helpful if objdump has
           been configured to support multiple architectures.

           If the target is an ARM architecture then this switch has an
           additional effect.  It restricts the disassembly to only
           those instructions supported by the architecture specified by
           machine.  If it is necessary to use this switch because the
           input file does not contain any architecture information, but
           it is also desired to disassemble all the instructions use
           -marm.

       -M options
       --disassembler-options=options
           Pass target specific information to the disassembler.  Only
           supported on some targets.  If it is necessary to specify
           more than one disassembler option then multiple -M options
           can be used or can be placed together into a comma separated
           list.

           For ARC, dsp controls the printing of DSP instructions, spfp
           selects the printing of FPX single precision FP instructions,
           dpfp selects the printing of FPX double precision FP
           instructions, quarkse_em selects the printing of special
           QuarkSE-EM instructions, fpuda selects the printing of double
           precision assist instructions, fpus selects the printing of
           FPU single precision FP instructions, while fpud selects the
           printing of FPU double precision FP instructions.
           Additionally, one can choose to have all the immediates
           printed in hexadecimal using hex.  By default, the short
           immediates are printed using the decimal representation,
           while the long immediate values are printed as hexadecimal.

           cpu=... allows one to enforce a particular ISA when
           disassembling instructions, overriding the -m value or
           whatever is in the ELF file.  This might be useful to select
           ARC EM or HS ISA, because architecture is same for those and
           disassembler relies on private ELF header data to decide if
           code is for EM or HS.  This option might be specified
           multiple times - only the latest value will be used.  Valid
           values are same as for the assembler -mcpu=... option.

           If the target is an ARM architecture then this switch can be
           used to select which register name set is used during
           disassembler.  Specifying -M reg-names-std (the default) will
           select the register names as used in ARM's instruction set
           documentation, but with register 13 called 'sp', register 14
           called 'lr' and register 15 called 'pc'.  Specifying -M reg-
           names-apcs will select the name set used by the ARM Procedure
           Call Standard, whilst specifying -M reg-names-raw will just
           use r followed by the register number.

           There are also two variants on the APCS register naming
           scheme enabled by -M reg-names-atpcs and -M reg-names-
           special-atpcs which use the ARM/Thumb Procedure Call Standard
           naming conventions.  (Either with the normal register names
           or the special register names).

           This option can also be used for ARM architectures to force
           the disassembler to interpret all instructions as Thumb
           instructions by using the switch
           --disassembler-options=force-thumb.  This can be useful when
           attempting to disassemble thumb code produced by other
           compilers.

           For AArch64 targets this switch can be used to set whether
           instructions are disassembled as the most general instruction
           using the -M no-aliases option or whether instruction notes
           should be generated as comments in the disasssembly using -M
           notes.

           For the x86, some of the options duplicate functions of the
           -m switch, but allow finer grained control.

           "x86-64"
           "i386"
           "i8086"
               Select disassembly for the given architecture.

           "intel"
           "att"
               Select between intel syntax mode and AT&T syntax mode.

           "amd64"
           "intel64"
               Select between AMD64 ISA and Intel64 ISA.

           "intel-mnemonic"
           "att-mnemonic"
               Select between intel mnemonic mode and AT&T mnemonic
               mode.  Note: "intel-mnemonic" implies "intel" and
               "att-mnemonic" implies "att".

           "addr64"
           "addr32"
           "addr16"
           "data32"
           "data16"
               Specify the default address size and operand size.  These
               five options will be overridden if "x86-64", "i386" or
               "i8086" appear later in the option string.

           "suffix"
               When in AT&T mode and also for a limited set of
               instructions when in Intel mode, instructs the
               disassembler to print a mnemonic suffix even when the
               suffix could be inferred by the operands or, for certain
               instructions, the execution mode's defaults.

           For PowerPC, the -M argument raw selects disasssembly of
           hardware insns rather than aliases.  For example, you will
           see "rlwinm" rather than "clrlwi", and "addi" rather than
           "li".  All of the -m arguments for gas that select a CPU are
           supported.  These are: 403, 405, 440, 464, 476, 601, 603,
           604, 620, 7400, 7410, 7450, 7455, 750cl, 821, 850, 860, a2,
           booke, booke32, cell, com, e200z2, e200z4, e300, e500,
           e500mc, e500mc64, e500x2, e5500, e6500, efs, power4, power5,
           power6, power7, power8, power9, power10, ppc, ppc32, ppc64,
           ppc64bridge, ppcps, pwr, pwr2, pwr4, pwr5, pwr5x, pwr6, pwr7,
           pwr8, pwr9, pwr10, pwrx, titan, vle, and future.  32 and 64
           modify the default or a prior CPU selection, disabling and
           enabling 64-bit insns respectively.  In addition, altivec,
           any, lsp, htm, vsx, spe and  spe2 add capabilities to a
           previous or later CPU selection.  any will disassemble any
           opcode known to binutils, but in cases where an opcode has
           two different meanings or different arguments, you may not
           see the disassembly you expect.  If you disassemble without
           giving a CPU selection, a default will be chosen from
           information gleaned by BFD from the object files headers, but
           the result again may not be as you expect.

           For MIPS, this option controls the printing of instruction
           mnemonic names and register names in disassembled
           instructions.  Multiple selections from the following may be
           specified as a comma separated string, and invalid options
           are ignored:

           "no-aliases"
               Print the 'raw' instruction mnemonic instead of some
               pseudo instruction mnemonic.  I.e., print 'daddu' or 'or'
               instead of 'move', 'sll' instead of 'nop', etc.

           "msa"
               Disassemble MSA instructions.

           "virt"
               Disassemble the virtualization ASE instructions.

           "xpa"
               Disassemble the eXtended Physical Address (XPA) ASE
               instructions.

           "gpr-names=ABI"
               Print GPR (general-purpose register) names as appropriate
               for the specified ABI.  By default, GPR names are
               selected according to the ABI of the binary being
               disassembled.

           "fpr-names=ABI"
               Print FPR (floating-point register) names as appropriate
               for the specified ABI.  By default, FPR numbers are
               printed rather than names.

           "cp0-names=ARCH"
               Print CP0 (system control coprocessor; coprocessor 0)
               register names as appropriate for the CPU or architecture
               specified by ARCH.  By default, CP0 register names are
               selected according to the architecture and CPU of the
               binary being disassembled.

           "hwr-names=ARCH"
               Print HWR (hardware register, used by the "rdhwr"
               instruction) names as appropriate for the CPU or
               architecture specified by ARCH.  By default, HWR names
               are selected according to the architecture and CPU of the
               binary being disassembled.

           "reg-names=ABI"
               Print GPR and FPR names as appropriate for the selected
               ABI.

           "reg-names=ARCH"
               Print CPU-specific register names (CP0 register and HWR
               names) as appropriate for the selected CPU or
               architecture.

           For any of the options listed above, ABI or ARCH may be
           specified as numeric to have numbers printed rather than
           names, for the selected types of registers.  You can list the
           available values of ABI and ARCH using the --help option.

           For VAX, you can specify function entry addresses with -M
           entry:0xf00ba.  You can use this multiple times to properly
           disassemble VAX binary files that don't contain symbol tables
           (like ROM dumps).  In these cases, the function entry mask
           would otherwise be decoded as VAX instructions, which would
           probably lead the rest of the function being wrongly
           disassembled.

       -p
       --private-headers
           Print information that is specific to the object file format.
           The exact information printed depends upon the object file
           format.  For some object file formats, no additional
           information is printed.

       -P options
       --private=options
           Print information that is specific to the object file format.
           The argument options is a comma separated list that depends
           on the format (the lists of options is displayed with the
           help).

           For XCOFF, the available options are:

           "header"
           "aout"
           "sections"
           "syms"
           "relocs"
           "lineno,"
           "loader"
           "except"
           "typchk"
           "traceback"
           "toc"
           "ldinfo"

           For PE, the available options are:

           "header"
           "sections"

           Not all object formats support this option.  In particular
           the ELF format does not use it.

       -r
       --reloc
           Print the relocation entries of the file.  If used with -d or
           -D, the relocations are printed interspersed with the
           disassembly.

       -R
       --dynamic-reloc
           Print the dynamic relocation entries of the file.  This is
           only meaningful for dynamic objects, such as certain types of
           shared libraries.  As for -r, if used with -d or -D, the
           relocations are printed interspersed with the disassembly.

       -s
       --full-contents
           Display the full contents of sections, often used in
           combination with -j to request specific sections.  By default
           all non-empty non-bss sections are displayed.  By default any
           compressed section will be displayed in its compressed form.
           In order to see the contents in a decompressed form add the
           -Z option to the command line.

       -S
       --source
           Display source code intermixed with disassembly, if possible.
           Implies -d.

       --show-all-symbols
           When disassembling, show all the symbols that match a given
           address, not just the first one.

       --source-comment[=txt]
           Like the -S option, but all source code lines are displayed
           with a prefix of txt.  Typically txt will be a comment string
           which can be used to distinguish the assembler code from the
           source code.  If txt is not provided then a default string of
           "# " (hash followed by a space), will be used.

       --prefix=prefix
           Specify prefix to add to the absolute paths when used with
           -S.

       --prefix-strip=level
           Indicate how many initial directory names to strip off the
           hardwired absolute paths. It has no effect without
           --prefix=prefix.

       --show-raw-insn
           When disassembling instructions, print the instruction in hex
           as well as in symbolic form.  This is the default except when
           --prefix-addresses is used.

       --no-show-raw-insn
           When disassembling instructions, do not print the instruction
           bytes.  This is the default when --prefix-addresses is used.

       --insn-width=width
           Display width bytes on a single line when disassembling
           instructions.

       --visualize-jumps[=color|=extended-color|=off]
           Visualize jumps that stay inside a function by drawing ASCII
           art between the start and target addresses.  The optional
           =color argument adds color to the output using simple
           terminal colors.  Alternatively the =extended-color argument
           will add color using 8bit colors, but these might not work on
           all terminals.

           If it is necessary to disable the visualize-jumps option
           after it has previously been enabled then use
           visualize-jumps=off.

       --disassembler-color=off
       --disassembler-color=terminal
       --disassembler-color=on|color|colour
       --disassembler-color=extened|extended-color|extened-colour
           Enables or disables the use of colored syntax highlighting in
           disassembly output.  The default behaviour is determined via
           a configure time option.  Note, not all architectures support
           colored syntax highlighting, and depending upon the terminal
           used, colored output may not actually be legible.

           The on argument adds colors using simple terminal colors.

           The terminal argument does the same, but only if the output
           device is a terminal.

           The extended-color argument is similar to the on argument,
           but it uses 8-bit colors.  These may not work on all
           terminals.

           The off argument disables colored disassembly.

       -W[lLiaprmfFsoORtUuTgAckK]
       --dwarf[=rawline,=decodedline,=info,=abbrev,=pubnames,=aranges,=macro,=frames,=frames-interp,=str,=str-offsets,=loc,=Ranges,=pubtypes,=trace_info,=trace_abbrev,=trace_aranges,=gdb_index,=addr,=cu_index,=links,=follow-links]
           Displays the contents of the DWARF debug sections in the
           file, if any are present.  Compressed debug sections are
           automatically decompressed (temporarily) before they are
           displayed.  If one or more of the optional letters or words
           follows the switch then only those type(s) of data will be
           dumped.  The letters and words refer to the following
           information:

           "a"
           "=abbrev"
               Displays the contents of the .debug_abbrev section.

           "A"
           "=addr"
               Displays the contents of the .debug_addr section.

           "c"
           "=cu_index"
               Displays the contents of the .debug_cu_index and/or
               .debug_tu_index sections.

           "f"
           "=frames"
               Display the raw contents of a .debug_frame section.

           "F"
           "=frames-interp"
               Display the interpreted contents of a .debug_frame
               section.

           "g"
           "=gdb_index"
               Displays the contents of the .gdb_index and/or
               .debug_names sections.

           "i"
           "=info"
               Displays the contents of the .debug_info section.  Note:
               the output from this option can also be restricted by the
               use of the --dwarf-depth and --dwarf-start options.

           "k"
           "=links"
               Displays the contents of the .gnu_debuglink,
               .gnu_debugaltlink and .debug_sup sections, if any of them
               are present.  Also displays any links to separate dwarf
               object files (dwo), if they are specified by the
               DW_AT_GNU_dwo_name or DW_AT_dwo_name attributes in the
               .debug_info section.

           "K"
           "=follow-links"
               Display the contents of any selected debug sections that
               are found in linked, separate debug info file(s).  This
               can result in multiple versions of the same debug section
               being displayed if it exists in more than one file.

               In addition, when displaying DWARF attributes, if a form
               is found that references the separate debug info file,
               then the referenced contents will also be displayed.

               Note - in some distributions this option is enabled by
               default.  It can be disabled via the N debug option.  The
               default can be chosen when configuring the binutils via
               the --enable-follow-debug-links=yes or
               --enable-follow-debug-links=no options.  If these are not
               used then the default is to enable the following of debug
               links.

               Note - if support for the debuginfod protocol was enabled
               when the binutils were built then this option will also
               include an attempt to contact any debuginfod servers
               mentioned in the DEBUGINFOD_URLS environment variable.
               This could take some time to resolve.  This behaviour can
               be disabled via the =do-not-use-debuginfod debug option.

           "N"
           "=no-follow-links"
               Disables the following of links to separate debug info
               files.

           "D"
           "=use-debuginfod"
               Enables contacting debuginfod servers if there is a need
               to follow debug links.  This is the default behaviour.

           "E"
           "=do-not-use-debuginfod"
               Disables contacting debuginfod servers when there is a
               need to follow debug links.

           "l"
           "=rawline"
               Displays the contents of the .debug_line section in a raw
               format.

           "L"
           "=decodedline"
               Displays the interpreted contents of the .debug_line
               section.

           "m"
           "=macro"
               Displays the contents of the .debug_macro and/or
               .debug_macinfo sections.

           "o"
           "=loc"
               Displays the contents of the .debug_loc and/or
               .debug_loclists sections.

           "O"
           "=str-offsets"
               Displays the contents of the .debug_str_offsets section.

           "p"
           "=pubnames"
               Displays the contents of the .debug_pubnames and/or
               .debug_gnu_pubnames sections.

           "r"
           "=aranges"
               Displays the contents of the .debug_aranges section.

           "R"
           "=Ranges"
               Displays the contents of the .debug_ranges and/or
               .debug_rnglists sections.

           "s"
           "=str"
               Displays the contents of the .debug_str, .debug_line_str
               and/or .debug_str_offsets sections.

           "t"
           "=pubtype"
               Displays the contents of the .debug_pubtypes and/or
               .debug_gnu_pubtypes sections.

           "T"
           "=trace_aranges"
               Displays the contents of the .trace_aranges section.

           "u"
           "=trace_abbrev"
               Displays the contents of the .trace_abbrev section.

           "U"
           "=trace_info"
               Displays the contents of the .trace_info section.

           Note: displaying the contents of .debug_static_funcs,
           .debug_static_vars and debug_weaknames sections is not
           currently supported.

       --dwarf-depth=n
           Limit the dump of the ".debug_info" section to n children.
           This is only useful with --debug-dump=info.  The default is
           to print all DIEs; the special value 0 for n will also have
           this effect.

           With a non-zero value for n, DIEs at or deeper than n levels
           will not be printed.  The range for n is zero-based.

       --dwarf-start=n
           Print only DIEs beginning with the DIE numbered n.  This is
           only useful with --debug-dump=info.

           If specified, this option will suppress printing of any
           header information and all DIEs before the DIE numbered n.
           Only siblings and children of the specified DIE will be
           printed.

           This can be used in conjunction with --dwarf-depth.

       --dwarf-check
           Enable additional checks for consistency of Dwarf
           information.

       --ctf[=section]
           Display the contents of the specified CTF section.  CTF
           sections themselves contain many subsections, all of which
           are displayed in order.

           By default, display the name of the section named .ctf, which
           is the name emitted by ld.

       --ctf-parent=member
           If the CTF section contains ambiguously-defined types, it
           will consist of an archive of many CTF dictionaries, all
           inheriting from one dictionary containing unambiguous types.
           This member is by default named .ctf, like the section
           containing it, but it is possible to change this name using
           the "ctf_link_set_memb_name_changer" function at link time.
           When looking at CTF archives that have been created by a
           linker that uses the name changer to rename the parent
           archive member, --ctf-parent can be used to specify the name
           used for the parent.

       --sframe[=section]
           Display the contents of the specified SFrame section.

           By default, display the name of the section named .sframe,
           which is the name emitted by ld.

       -G
       --stabs
           Display the full contents of any sections requested.  Display
           the contents of the .stab and .stab.index and .stab.excl
           sections from an ELF file.  This is only useful on systems
           (such as Solaris 2.0) in which ".stab" debugging symbol-table
           entries are carried in an ELF section.  In most other file
           formats, debugging symbol-table entries are interleaved with
           linkage symbols, and are visible in the --syms output.

       --start-address=address
           Start displaying data at the specified address.  This affects
           the output of the -d, -r and -s options.

       --stop-address=address
           Stop displaying data at the specified address.  This affects
           the output of the -d, -r and -s options.

       -t
       --syms
           Print the symbol table entries of the file.  This is similar
           to the information provided by the nm program, although the
           display format is different.  The format of the output
           depends upon the format of the file being dumped, but there
           are two main types.  One looks like this:

                   [  4](sec  3)(fl 0x00)(ty   0)(scl   3) (nx 1) 0x00000000 .bss
                   [  6](sec  1)(fl 0x00)(ty   0)(scl   2) (nx 0) 0x00000000 fred

           where the number inside the square brackets is the number of
           the entry in the symbol table, the sec number is the section
           number, the fl value are the symbol's flag bits, the ty
           number is the symbol's type, the scl number is the symbol's
           storage class and the nx value is the number of auxiliary
           entries associated with the symbol.  The last two fields are
           the symbol's value and its name.

           The other common output format, usually seen with ELF based
           files, looks like this:

                   00000000 l    d  .bss   00000000 .bss
                   00000000 g       .text  00000000 fred

           Here the first number is the symbol's value (sometimes
           referred to as its address).  The next field is actually a
           set of characters and spaces indicating the flag bits that
           are set on the symbol.  These characters are described below.
           Next is the section with which the symbol is associated or
           *ABS* if the section is absolute (ie not connected with any
           section), or *UND* if the section is referenced in the file
           being dumped, but not defined there.

           After the section name comes another field, a number, which
           for common symbols is the alignment and for other symbol is
           the size.  Finally the symbol's name is displayed.

           The flag characters are divided into 7 groups as follows:

           "l"
           "g"
           "u"
           "!" The symbol is a local (l), global (g), unique global (u),
               neither global nor local (a space) or both global and
               local (!).  A symbol can be neither local or global for a
               variety of reasons, e.g., because it is used for
               debugging, but it is probably an indication of a bug if
               it is ever both local and global.  Unique global symbols
               are a GNU extension to the standard set of ELF symbol
               bindings.  For such a symbol the dynamic linker will make
               sure that in the entire process there is just one symbol
               with this name and type in use.

           "w" The symbol is weak (w) or strong (a space).

           "C" The symbol denotes a constructor (C) or an ordinary
               symbol (a space).

           "W" The symbol is a warning (W) or a normal symbol (a space).
               A warning symbol's name is a message to be displayed if
               the symbol following the warning symbol is ever
               referenced.

           "I"
           "i" The symbol is an indirect reference to another symbol
               (I), a function to be evaluated during reloc processing
               (i) or a normal symbol (a space).

           "d"
           "D" The symbol is a debugging symbol (d) or a dynamic symbol
               (D) or a normal symbol (a space).

           "F"
           "f"
           "O" The symbol is the name of a function (F) or a file (f) or
               an object (O) or just a normal symbol (a space).

       -T
       --dynamic-syms
           Print the dynamic symbol table entries of the file.  This is
           only meaningful for dynamic objects, such as certain types of
           shared libraries.  This is similar to the information
           provided by the nm program when given the -D (--dynamic)
           option.

           The output format is similar to that produced by the --syms
           option, except that an extra field is inserted before the
           symbol's name, giving the version information associated with
           the symbol.  If the version is the default version to be used
           when resolving unversioned references to the symbol then it's
           displayed as is, otherwise it's put into parentheses.

       --special-syms
           When displaying symbols include those which the target
           considers to be special in some way and which would not
           normally be of interest to the user.

       -U [d|i|l|e|x|h]
       --unicode=[default|invalid|locale|escape|hex|highlight]
           Controls the display of UTF-8 encoded multibyte characters in
           strings.  The default (--unicode=default) is to give them no
           special treatment.  The --unicode=locale option displays the
           sequence in the current locale, which may or may not support
           them.  The options --unicode=hex and --unicode=invalid
           display them as hex byte sequences enclosed by either angle
           brackets or curly braces.

           The --unicode=escape option displays them as escape sequences
           (\uxxxx) and the --unicode=highlight option displays them as
           escape sequences highlighted in red (if supported by the
           output device).  The colouring is intended to draw attention
           to the presence of unicode sequences where they might not be
           expected.

       -V
       --version
           Print the version number of objdump and exit.

       -x
       --all-headers
           Display all available header information, including the
           symbol table and relocation entries.  Using -x is equivalent
           to specifying all of -a -f -h -p -r -t.

       -w
       --wide
           Format some lines for output devices that have more than 80
           columns.  Also do not truncate symbol names when they are
           displayed.

       -z
       --disassemble-zeroes
           Normally the disassembly output will skip blocks of zeroes.
           This option directs the disassembler to disassemble those
           blocks, just like any other data.

       -Z
       --decompress
           The -Z option is meant to be used in conunction with the -s
           option.  It instructs objdump to decompress any compressed
           sections before displaying their contents.

       @file
           Read command-line options from file.  The options read are
           inserted in place of the original @file option.  If file does
           not exist, or cannot be read, then the option will be treated
           literally, and not removed.

           Options in file are separated by whitespace.  A whitespace
           character may be included in an option by surrounding the
           entire option in either single or double quotes.  Any
           character (including a backslash) may be included by
           prefixing the character to be included with a backslash.  The
           file may itself contain additional @file options; any such
           options will be processed recursively.
SEE ALSO
       nm(1), readelf(1), and the Info entries for binutils.
COPYRIGHT
       Copyright (c) 1991-2024 Free Software Foundation, Inc.

       Permission is granted to copy, distribute and/or modify this
       document under the terms of the GNU Free Documentation License,
       Version 1.3 or any later version published by the Free Software
       Foundation; with no Invariant Sections, with no Front-Cover
       Texts, and with no Back-Cover Texts.  A copy of the license is
       included in the section entitled "GNU Free Documentation License".
COLOPHON
       This page is part of the binutils (a collection of tools for
       working with executable binaries) project.  Information about the
       project can be found at http://www.gnu.org/software/binutils/.
       If you have a bug report for this manual page, see
       http://sourceware.org/bugzilla/enter_bug.cgi?product=binutils.
       This page was obtained from the tarball binutils-2.42.tar.gz
       fetched from https://ftp.gnu.org/gnu/binutils/ on 2024-06-14.
       If you discover any rendering problems in this HTML version of
       the page, or you believe there is a better or more up-to-date
       source for the page, or you have corrections or improvements to
       the information in this COLOPHON (which is not part of the
       original manual page), send a mail to man-pages@man7.org

binutils-2.42                  2024-06-14                     OBJDUMP(1)
\end{lstlisting}
}}
\endinput  %  ==  ==  ==  ==  ==  ==  ==  ==  ==

\subsection{\refObjdump: Display Information From Object Files}

{\tiny{
\begin{lstlisting}[language=bash]
NAME
       objdump - display information from object files
SYNOPSIS
       objdump [-a|--archive-headers]
               [-b bfdname|--target=bfdname]
               [-C|--demangle[=style] ]
               [-d|--disassemble[=symbol]]
               [-D|--disassemble-all]
               [-z|--disassemble-zeroes]
               [-EB|-EL|--endian={big | little }]
               [-f|--file-headers]
               [-F|--file-offsets]
               [--file-start-context]
               [-g|--debugging]
               [-e|--debugging-tags]
               [-h|--section-headers|--headers]
               [-i|--info]
               [-j section|--section=section]
               [-l|--line-numbers]
               [-S|--source]
               [--source-comment[=text]]
               [-m machine|--architecture=machine]
               [-M options|--disassembler-options=options]
               [-p|--private-headers]
               [-P options|--private=options]
               [-r|--reloc]
               [-R|--dynamic-reloc]
               [-s|--full-contents]
               [-Z|--decompress]
               [-W[lLiaprmfFsoORtUuTgAck]|
                --dwarf[=rawline,=decodedline,=info,=abbrev,=pubnames,=aranges,=macro,=frames,=frames-interp,=str,=str-offsets,=loc,=Ranges,=pubtypes,=trace_info,=trace_abbrev,=trace_aranges,=gdb_index,=addr,=cu_index,=links]]
               [-WK|--dwarf=follow-links]
               [-WN|--dwarf=no-follow-links]
               [-wD|--dwarf=use-debuginfod]
               [-wE|--dwarf=do-not-use-debuginfod]
               [-L|--process-links]
               [--ctf=section]
               [--sframe=section]
               [-G|--stabs]
               [-t|--syms]
               [-T|--dynamic-syms]
               [-x|--all-headers]
               [-w|--wide]
               [--start-address=address]
               [--stop-address=address]
               [--no-addresses]
               [--prefix-addresses]
               [--[no-]show-raw-insn]
               [--adjust-vma=offset]
               [--show-all-symbols]
               [--dwarf-depth=n]
               [--dwarf-start=n]
               [--ctf-parent=section]
               [--no-recurse-limit|--recurse-limit]
               [--special-syms]
               [--prefix=prefix]
               [--prefix-strip=level]
               [--insn-width=width]
               [--visualize-jumps[=color|=extended-color|=off]
               [--disassembler-color=[off|terminal|on|extended]
               [-U method] [--unicode=method]
               [-V|--version]
               [-H|--help]
               objfile...
DESCRIPTION
       objdump displays information about one or more object files.  The
       options control what particular information to display.  This
       information is mostly useful to programmers who are working on
       the compilation tools, as opposed to programmers who just want
       their program to compile and work.

       objfile... are the object files to be examined.  When you specify
       archives, objdump shows information on each of the member object
       files.
OPTIONS
       The long and short forms of options, shown here as alternatives,
       are equivalent.  At least one option from the list
       -a,-d,-D,-e,-f,-g,-G,-h,-H,-p,-P,-r,-R,-s,-S,-t,-T,-V,-x must be
       given.

       -a
       --archive-header
           If any of the objfile files are archives, display the archive
           header information (in a format similar to ls -l).  Besides
           the information you could list with ar tv, objdump -a shows
           the object file format of each archive member.

       --adjust-vma=offset
           When dumping information, first add offset to all the section
           addresses.  This is useful if the section addresses do not
           correspond to the symbol table, which can happen when putting
           sections at particular addresses when using a format which
           can not represent section addresses, such as a.out.

       -b bfdname
       --target=bfdname
           Specify that the object-code format for the object files is
           bfdname.  This option may not be necessary; objdump can
           automatically recognize many formats.

           For example,

                   objdump -b oasys -m vax -h fu.o

           displays summary information from the section headers (-h) of
           fu.o, which is explicitly identified (-m) as a VAX object
           file in the format produced by Oasys compilers.  You can list
           the formats available with the -i option.

       -C
       --demangle[=style]
           Decode (demangle) low-level symbol names into user-level
           names.  Besides removing any initial underscore prepended by
           the system, this makes C++ function names readable.
           Different compilers have different mangling styles. The
           optional demangling style argument can be used to choose an
           appropriate demangling style for your compiler.

       --recurse-limit
       --no-recurse-limit
       --recursion-limit
       --no-recursion-limit
           Enables or disables a limit on the amount of recursion
           performed whilst demangling strings.  Since the name mangling
           formats allow for an infinite level of recursion it is
           possible to create strings whose decoding will exhaust the
           amount of stack space available on the host machine,
           triggering a memory fault.  The limit tries to prevent this
           from happening by restricting recursion to 2048 levels of
           nesting.

           The default is for this limit to be enabled, but disabling it
           may be necessary in order to demangle truly complicated
           names.  Note however that if the recursion limit is disabled
           then stack exhaustion is possible and any bug reports about
           such an event will be rejected.

       -g
       --debugging
           Display debugging information.  This attempts to parse STABS
           debugging format information stored in the file and print it
           out using a C like syntax.  If no STABS debugging was found
           this option falls back on the -W option to print any DWARF
           information in the file.

       -e
       --debugging-tags
           Like -g, but the information is generated in a format
           compatible with ctags tool.

       -d
       --disassemble
       --disassemble=symbol
           Display the assembler mnemonics for the machine instructions
           from the input file.  This option only disassembles those
           sections which are expected to contain instructions.  If the
           optional symbol argument is given, then display the assembler
           mnemonics starting at symbol.  If symbol is a function name
           then disassembly will stop at the end of the function,
           otherwise it will stop when the next symbol is encountered.
           If there are no matches for symbol then nothing will be
           displayed.

           Note if the --dwarf=follow-links option is enabled then any
           symbol tables in linked debug info files will be read in and
           used when disassembling.

       -D
       --disassemble-all
           Like -d, but disassemble the contents of all non-empty non-
           bss sections, not just those expected to contain
           instructions.  -j may be used to select specific sections.

           This option also has a subtle effect on the disassembly of
           instructions in code sections.  When option -d is in effect
           objdump will assume that any symbols present in a code
           section occur on the boundary between instructions and it
           will refuse to disassemble across such a boundary.  When
           option -D is in effect however this assumption is supressed.
           This means that it is possible for the output of -d and -D to
           differ if, for example, data is stored in code sections.

           If the target is an ARM architecture this switch also has the
           effect of forcing the disassembler to decode pieces of data
           found in code sections as if they were instructions.

           Note if the --dwarf=follow-links option is enabled then any
           symbol tables in linked debug info files will be read in and
           used when disassembling.

       --no-addresses
           When disassembling, don't print addresses on each line or for
           symbols and relocation offsets.  In combination with
           --no-show-raw-insn this may be useful for comparing compiler
           output.

       --prefix-addresses
           When disassembling, print the complete address on each line.
           This is the older disassembly format.

       -EB
       -EL
       --endian={big|little}
           Specify the endianness of the object files.  This only
           affects disassembly.  This can be useful when disassembling a
           file format which does not describe endianness information,
           such as S-records.

       -f
       --file-headers
           Display summary information from the overall header of each
           of the objfile files.

       -F
       --file-offsets
           When disassembling sections, whenever a symbol is displayed,
           also display the file offset of the region of data that is
           about to be dumped.  If zeroes are being skipped, then when
           disassembly resumes, tell the user how many zeroes were
           skipped and the file offset of the location from where the
           disassembly resumes.  When dumping sections, display the file
           offset of the location from where the dump starts.

       --file-start-context
           Specify that when displaying interlisted source
           code/disassembly (assumes -S) from a file that has not yet
           been displayed, extend the context to the start of the file.

       -h
       --section-headers
       --headers
           Display summary information from the section headers of the
           object file.

           File segments may be relocated to nonstandard addresses, for
           example by using the -Ttext, -Tdata, or -Tbss options to ld.
           However, some object file formats, such as a.out, do not
           store the starting address of the file segments.  In those
           situations, although ld relocates the sections correctly,
           using objdump -h to list the file section headers cannot show
           the correct addresses.  Instead, it shows the usual
           addresses, which are implicit for the target.

           Note, in some cases it is possible for a section to have both
           the READONLY and the NOREAD attributes set.  In such cases
           the NOREAD attribute takes precedence, but objdump will
           report both since the exact setting of the flag bits might be
           important.

       -H
       --help
           Print a summary of the options to objdump and exit.

       -i
       --info
           Display a list showing all architectures and object formats
           available for specification with -b or -m.

       -j name
       --section=name
           Display information for section name.  This option may be
           specified multiple times.

       -L
       --process-links
           Display the contents of non-debug sections found in separate
           debuginfo files that are linked to the main file.  This
           option automatically implies the -WK option, and only
           sections requested by other command line options will be
           displayed.

       -l
       --line-numbers
           Label the display (using debugging information) with the
           filename and source line numbers corresponding to the object
           code or relocs shown.  Only useful with -d, -D, or -r.

       -m machine
       --architecture=machine
           Specify the architecture to use when disassembling object
           files.  This can be useful when disassembling object files
           which do not describe architecture information, such as
           S-records.  You can list the available architectures with the
           -i option.

           For most architectures it is possible to supply an
           architecture name and a machine name, separated by a colon.
           For example foo:bar would refer to the bar machine type in
           the foo architecture.  This can be helpful if objdump has
           been configured to support multiple architectures.

           If the target is an ARM architecture then this switch has an
           additional effect.  It restricts the disassembly to only
           those instructions supported by the architecture specified by
           machine.  If it is necessary to use this switch because the
           input file does not contain any architecture information, but
           it is also desired to disassemble all the instructions use
           -marm.

       -M options
       --disassembler-options=options
           Pass target specific information to the disassembler.  Only
           supported on some targets.  If it is necessary to specify
           more than one disassembler option then multiple -M options
           can be used or can be placed together into a comma separated
           list.

           For ARC, dsp controls the printing of DSP instructions, spfp
           selects the printing of FPX single precision FP instructions,
           dpfp selects the printing of FPX double precision FP
           instructions, quarkse_em selects the printing of special
           QuarkSE-EM instructions, fpuda selects the printing of double
           precision assist instructions, fpus selects the printing of
           FPU single precision FP instructions, while fpud selects the
           printing of FPU double precision FP instructions.
           Additionally, one can choose to have all the immediates
           printed in hexadecimal using hex.  By default, the short
           immediates are printed using the decimal representation,
           while the long immediate values are printed as hexadecimal.

           cpu=... allows one to enforce a particular ISA when
           disassembling instructions, overriding the -m value or
           whatever is in the ELF file.  This might be useful to select
           ARC EM or HS ISA, because architecture is same for those and
           disassembler relies on private ELF header data to decide if
           code is for EM or HS.  This option might be specified
           multiple times - only the latest value will be used.  Valid
           values are same as for the assembler -mcpu=... option.

           If the target is an ARM architecture then this switch can be
           used to select which register name set is used during
           disassembler.  Specifying -M reg-names-std (the default) will
           select the register names as used in ARM's instruction set
           documentation, but with register 13 called 'sp', register 14
           called 'lr' and register 15 called 'pc'.  Specifying -M reg-
           names-apcs will select the name set used by the ARM Procedure
           Call Standard, whilst specifying -M reg-names-raw will just
           use r followed by the register number.

           There are also two variants on the APCS register naming
           scheme enabled by -M reg-names-atpcs and -M reg-names-
           special-atpcs which use the ARM/Thumb Procedure Call Standard
           naming conventions.  (Either with the normal register names
           or the special register names).

           This option can also be used for ARM architectures to force
           the disassembler to interpret all instructions as Thumb
           instructions by using the switch
           --disassembler-options=force-thumb.  This can be useful when
           attempting to disassemble thumb code produced by other
           compilers.

           For AArch64 targets this switch can be used to set whether
           instructions are disassembled as the most general instruction
           using the -M no-aliases option or whether instruction notes
           should be generated as comments in the disasssembly using -M
           notes.

           For the x86, some of the options duplicate functions of the
           -m switch, but allow finer grained control.

           "x86-64"
           "i386"
           "i8086"
               Select disassembly for the given architecture.

           "intel"
           "att"
               Select between intel syntax mode and AT&T syntax mode.

           "amd64"
           "intel64"
               Select between AMD64 ISA and Intel64 ISA.

           "intel-mnemonic"
           "att-mnemonic"
               Select between intel mnemonic mode and AT&T mnemonic
               mode.  Note: "intel-mnemonic" implies "intel" and
               "att-mnemonic" implies "att".

           "addr64"
           "addr32"
           "addr16"
           "data32"
           "data16"
               Specify the default address size and operand size.  These
               five options will be overridden if "x86-64", "i386" or
               "i8086" appear later in the option string.

           "suffix"
               When in AT&T mode and also for a limited set of
               instructions when in Intel mode, instructs the
               disassembler to print a mnemonic suffix even when the
               suffix could be inferred by the operands or, for certain
               instructions, the execution mode's defaults.

           For PowerPC, the -M argument raw selects disasssembly of
           hardware insns rather than aliases.  For example, you will
           see "rlwinm" rather than "clrlwi", and "addi" rather than
           "li".  All of the -m arguments for gas that select a CPU are
           supported.  These are: 403, 405, 440, 464, 476, 601, 603,
           604, 620, 7400, 7410, 7450, 7455, 750cl, 821, 850, 860, a2,
           booke, booke32, cell, com, e200z2, e200z4, e300, e500,
           e500mc, e500mc64, e500x2, e5500, e6500, efs, power4, power5,
           power6, power7, power8, power9, power10, ppc, ppc32, ppc64,
           ppc64bridge, ppcps, pwr, pwr2, pwr4, pwr5, pwr5x, pwr6, pwr7,
           pwr8, pwr9, pwr10, pwrx, titan, vle, and future.  32 and 64
           modify the default or a prior CPU selection, disabling and
           enabling 64-bit insns respectively.  In addition, altivec,
           any, lsp, htm, vsx, spe and  spe2 add capabilities to a
           previous or later CPU selection.  any will disassemble any
           opcode known to binutils, but in cases where an opcode has
           two different meanings or different arguments, you may not
           see the disassembly you expect.  If you disassemble without
           giving a CPU selection, a default will be chosen from
           information gleaned by BFD from the object files headers, but
           the result again may not be as you expect.

           For MIPS, this option controls the printing of instruction
           mnemonic names and register names in disassembled
           instructions.  Multiple selections from the following may be
           specified as a comma separated string, and invalid options
           are ignored:

           "no-aliases"
               Print the 'raw' instruction mnemonic instead of some
               pseudo instruction mnemonic.  I.e., print 'daddu' or 'or'
               instead of 'move', 'sll' instead of 'nop', etc.

           "msa"
               Disassemble MSA instructions.

           "virt"
               Disassemble the virtualization ASE instructions.

           "xpa"
               Disassemble the eXtended Physical Address (XPA) ASE
               instructions.

           "gpr-names=ABI"
               Print GPR (general-purpose register) names as appropriate
               for the specified ABI.  By default, GPR names are
               selected according to the ABI of the binary being
               disassembled.

           "fpr-names=ABI"
               Print FPR (floating-point register) names as appropriate
               for the specified ABI.  By default, FPR numbers are
               printed rather than names.

           "cp0-names=ARCH"
               Print CP0 (system control coprocessor; coprocessor 0)
               register names as appropriate for the CPU or architecture
               specified by ARCH.  By default, CP0 register names are
               selected according to the architecture and CPU of the
               binary being disassembled.

           "hwr-names=ARCH"
               Print HWR (hardware register, used by the "rdhwr"
               instruction) names as appropriate for the CPU or
               architecture specified by ARCH.  By default, HWR names
               are selected according to the architecture and CPU of the
               binary being disassembled.

           "reg-names=ABI"
               Print GPR and FPR names as appropriate for the selected
               ABI.

           "reg-names=ARCH"
               Print CPU-specific register names (CP0 register and HWR
               names) as appropriate for the selected CPU or
               architecture.

           For any of the options listed above, ABI or ARCH may be
           specified as numeric to have numbers printed rather than
           names, for the selected types of registers.  You can list the
           available values of ABI and ARCH using the --help option.

           For VAX, you can specify function entry addresses with -M
           entry:0xf00ba.  You can use this multiple times to properly
           disassemble VAX binary files that don't contain symbol tables
           (like ROM dumps).  In these cases, the function entry mask
           would otherwise be decoded as VAX instructions, which would
           probably lead the rest of the function being wrongly
           disassembled.

       -p
       --private-headers
           Print information that is specific to the object file format.
           The exact information printed depends upon the object file
           format.  For some object file formats, no additional
           information is printed.

       -P options
       --private=options
           Print information that is specific to the object file format.
           The argument options is a comma separated list that depends
           on the format (the lists of options is displayed with the
           help).

           For XCOFF, the available options are:

           "header"
           "aout"
           "sections"
           "syms"
           "relocs"
           "lineno,"
           "loader"
           "except"
           "typchk"
           "traceback"
           "toc"
           "ldinfo"

           For PE, the available options are:

           "header"
           "sections"

           Not all object formats support this option.  In particular
           the ELF format does not use it.

       -r
       --reloc
           Print the relocation entries of the file.  If used with -d or
           -D, the relocations are printed interspersed with the
           disassembly.

       -R
       --dynamic-reloc
           Print the dynamic relocation entries of the file.  This is
           only meaningful for dynamic objects, such as certain types of
           shared libraries.  As for -r, if used with -d or -D, the
           relocations are printed interspersed with the disassembly.

       -s
       --full-contents
           Display the full contents of sections, often used in
           combination with -j to request specific sections.  By default
           all non-empty non-bss sections are displayed.  By default any
           compressed section will be displayed in its compressed form.
           In order to see the contents in a decompressed form add the
           -Z option to the command line.

       -S
       --source
           Display source code intermixed with disassembly, if possible.
           Implies -d.

       --show-all-symbols
           When disassembling, show all the symbols that match a given
           address, not just the first one.

       --source-comment[=txt]
           Like the -S option, but all source code lines are displayed
           with a prefix of txt.  Typically txt will be a comment string
           which can be used to distinguish the assembler code from the
           source code.  If txt is not provided then a default string of
           "# " (hash followed by a space), will be used.

       --prefix=prefix
           Specify prefix to add to the absolute paths when used with
           -S.

       --prefix-strip=level
           Indicate how many initial directory names to strip off the
           hardwired absolute paths. It has no effect without
           --prefix=prefix.

       --show-raw-insn
           When disassembling instructions, print the instruction in hex
           as well as in symbolic form.  This is the default except when
           --prefix-addresses is used.

       --no-show-raw-insn
           When disassembling instructions, do not print the instruction
           bytes.  This is the default when --prefix-addresses is used.

       --insn-width=width
           Display width bytes on a single line when disassembling
           instructions.

       --visualize-jumps[=color|=extended-color|=off]
           Visualize jumps that stay inside a function by drawing ASCII
           art between the start and target addresses.  The optional
           =color argument adds color to the output using simple
           terminal colors.  Alternatively the =extended-color argument
           will add color using 8bit colors, but these might not work on
           all terminals.

           If it is necessary to disable the visualize-jumps option
           after it has previously been enabled then use
           visualize-jumps=off.

       --disassembler-color=off
       --disassembler-color=terminal
       --disassembler-color=on|color|colour
       --disassembler-color=extened|extended-color|extened-colour
           Enables or disables the use of colored syntax highlighting in
           disassembly output.  The default behaviour is determined via
           a configure time option.  Note, not all architectures support
           colored syntax highlighting, and depending upon the terminal
           used, colored output may not actually be legible.

           The on argument adds colors using simple terminal colors.

           The terminal argument does the same, but only if the output
           device is a terminal.

           The extended-color argument is similar to the on argument,
           but it uses 8-bit colors.  These may not work on all
           terminals.

           The off argument disables colored disassembly.

       -W[lLiaprmfFsoORtUuTgAckK]
       --dwarf[=rawline,=decodedline,=info,=abbrev,=pubnames,=aranges,=macro,=frames,=frames-interp,=str,=str-offsets,=loc,=Ranges,=pubtypes,=trace_info,=trace_abbrev,=trace_aranges,=gdb_index,=addr,=cu_index,=links,=follow-links]
           Displays the contents of the DWARF debug sections in the
           file, if any are present.  Compressed debug sections are
           automatically decompressed (temporarily) before they are
           displayed.  If one or more of the optional letters or words
           follows the switch then only those type(s) of data will be
           dumped.  The letters and words refer to the following
           information:

           "a"
           "=abbrev"
               Displays the contents of the .debug_abbrev section.

           "A"
           "=addr"
               Displays the contents of the .debug_addr section.

           "c"
           "=cu_index"
               Displays the contents of the .debug_cu_index and/or
               .debug_tu_index sections.

           "f"
           "=frames"
               Display the raw contents of a .debug_frame section.

           "F"
           "=frames-interp"
               Display the interpreted contents of a .debug_frame
               section.

           "g"
           "=gdb_index"
               Displays the contents of the .gdb_index and/or
               .debug_names sections.

           "i"
           "=info"
               Displays the contents of the .debug_info section.  Note:
               the output from this option can also be restricted by the
               use of the --dwarf-depth and --dwarf-start options.

           "k"
           "=links"
               Displays the contents of the .gnu_debuglink,
               .gnu_debugaltlink and .debug_sup sections, if any of them
               are present.  Also displays any links to separate dwarf
               object files (dwo), if they are specified by the
               DW_AT_GNU_dwo_name or DW_AT_dwo_name attributes in the
               .debug_info section.

           "K"
           "=follow-links"
               Display the contents of any selected debug sections that
               are found in linked, separate debug info file(s).  This
               can result in multiple versions of the same debug section
               being displayed if it exists in more than one file.

               In addition, when displaying DWARF attributes, if a form
               is found that references the separate debug info file,
               then the referenced contents will also be displayed.

               Note - in some distributions this option is enabled by
               default.  It can be disabled via the N debug option.  The
               default can be chosen when configuring the binutils via
               the --enable-follow-debug-links=yes or
               --enable-follow-debug-links=no options.  If these are not
               used then the default is to enable the following of debug
               links.

               Note - if support for the debuginfod protocol was enabled
               when the binutils were built then this option will also
               include an attempt to contact any debuginfod servers
               mentioned in the DEBUGINFOD_URLS environment variable.
               This could take some time to resolve.  This behaviour can
               be disabled via the =do-not-use-debuginfod debug option.

           "N"
           "=no-follow-links"
               Disables the following of links to separate debug info
               files.

           "D"
           "=use-debuginfod"
               Enables contacting debuginfod servers if there is a need
               to follow debug links.  This is the default behaviour.

           "E"
           "=do-not-use-debuginfod"
               Disables contacting debuginfod servers when there is a
               need to follow debug links.

           "l"
           "=rawline"
               Displays the contents of the .debug_line section in a raw
               format.

           "L"
           "=decodedline"
               Displays the interpreted contents of the .debug_line
               section.

           "m"
           "=macro"
               Displays the contents of the .debug_macro and/or
               .debug_macinfo sections.

           "o"
           "=loc"
               Displays the contents of the .debug_loc and/or
               .debug_loclists sections.

           "O"
           "=str-offsets"
               Displays the contents of the .debug_str_offsets section.

           "p"
           "=pubnames"
               Displays the contents of the .debug_pubnames and/or
               .debug_gnu_pubnames sections.

           "r"
           "=aranges"
               Displays the contents of the .debug_aranges section.

           "R"
           "=Ranges"
               Displays the contents of the .debug_ranges and/or
               .debug_rnglists sections.

           "s"
           "=str"
               Displays the contents of the .debug_str, .debug_line_str
               and/or .debug_str_offsets sections.

           "t"
           "=pubtype"
               Displays the contents of the .debug_pubtypes and/or
               .debug_gnu_pubtypes sections.

           "T"
           "=trace_aranges"
               Displays the contents of the .trace_aranges section.

           "u"
           "=trace_abbrev"
               Displays the contents of the .trace_abbrev section.

           "U"
           "=trace_info"
               Displays the contents of the .trace_info section.

           Note: displaying the contents of .debug_static_funcs,
           .debug_static_vars and debug_weaknames sections is not
           currently supported.

       --dwarf-depth=n
           Limit the dump of the ".debug_info" section to n children.
           This is only useful with --debug-dump=info.  The default is
           to print all DIEs; the special value 0 for n will also have
           this effect.

           With a non-zero value for n, DIEs at or deeper than n levels
           will not be printed.  The range for n is zero-based.

       --dwarf-start=n
           Print only DIEs beginning with the DIE numbered n.  This is
           only useful with --debug-dump=info.

           If specified, this option will suppress printing of any
           header information and all DIEs before the DIE numbered n.
           Only siblings and children of the specified DIE will be
           printed.

           This can be used in conjunction with --dwarf-depth.

       --dwarf-check
           Enable additional checks for consistency of Dwarf
           information.

       --ctf[=section]
           Display the contents of the specified CTF section.  CTF
           sections themselves contain many subsections, all of which
           are displayed in order.

           By default, display the name of the section named .ctf, which
           is the name emitted by ld.

       --ctf-parent=member
           If the CTF section contains ambiguously-defined types, it
           will consist of an archive of many CTF dictionaries, all
           inheriting from one dictionary containing unambiguous types.
           This member is by default named .ctf, like the section
           containing it, but it is possible to change this name using
           the "ctf_link_set_memb_name_changer" function at link time.
           When looking at CTF archives that have been created by a
           linker that uses the name changer to rename the parent
           archive member, --ctf-parent can be used to specify the name
           used for the parent.

       --sframe[=section]
           Display the contents of the specified SFrame section.

           By default, display the name of the section named .sframe,
           which is the name emitted by ld.

       -G
       --stabs
           Display the full contents of any sections requested.  Display
           the contents of the .stab and .stab.index and .stab.excl
           sections from an ELF file.  This is only useful on systems
           (such as Solaris 2.0) in which ".stab" debugging symbol-table
           entries are carried in an ELF section.  In most other file
           formats, debugging symbol-table entries are interleaved with
           linkage symbols, and are visible in the --syms output.

       --start-address=address
           Start displaying data at the specified address.  This affects
           the output of the -d, -r and -s options.

       --stop-address=address
           Stop displaying data at the specified address.  This affects
           the output of the -d, -r and -s options.

       -t
       --syms
           Print the symbol table entries of the file.  This is similar
           to the information provided by the nm program, although the
           display format is different.  The format of the output
           depends upon the format of the file being dumped, but there
           are two main types.  One looks like this:

                   [  4](sec  3)(fl 0x00)(ty   0)(scl   3) (nx 1) 0x00000000 .bss
                   [  6](sec  1)(fl 0x00)(ty   0)(scl   2) (nx 0) 0x00000000 fred

           where the number inside the square brackets is the number of
           the entry in the symbol table, the sec number is the section
           number, the fl value are the symbol's flag bits, the ty
           number is the symbol's type, the scl number is the symbol's
           storage class and the nx value is the number of auxiliary
           entries associated with the symbol.  The last two fields are
           the symbol's value and its name.

           The other common output format, usually seen with ELF based
           files, looks like this:

                   00000000 l    d  .bss   00000000 .bss
                   00000000 g       .text  00000000 fred

           Here the first number is the symbol's value (sometimes
           referred to as its address).  The next field is actually a
           set of characters and spaces indicating the flag bits that
           are set on the symbol.  These characters are described below.
           Next is the section with which the symbol is associated or
           *ABS* if the section is absolute (ie not connected with any
           section), or *UND* if the section is referenced in the file
           being dumped, but not defined there.

           After the section name comes another field, a number, which
           for common symbols is the alignment and for other symbol is
           the size.  Finally the symbol's name is displayed.

           The flag characters are divided into 7 groups as follows:

           "l"
           "g"
           "u"
           "!" The symbol is a local (l), global (g), unique global (u),
               neither global nor local (a space) or both global and
               local (!).  A symbol can be neither local or global for a
               variety of reasons, e.g., because it is used for
               debugging, but it is probably an indication of a bug if
               it is ever both local and global.  Unique global symbols
               are a GNU extension to the standard set of ELF symbol
               bindings.  For such a symbol the dynamic linker will make
               sure that in the entire process there is just one symbol
               with this name and type in use.

           "w" The symbol is weak (w) or strong (a space).

           "C" The symbol denotes a constructor (C) or an ordinary
               symbol (a space).

           "W" The symbol is a warning (W) or a normal symbol (a space).
               A warning symbol's name is a message to be displayed if
               the symbol following the warning symbol is ever
               referenced.

           "I"
           "i" The symbol is an indirect reference to another symbol
               (I), a function to be evaluated during reloc processing
               (i) or a normal symbol (a space).

           "d"
           "D" The symbol is a debugging symbol (d) or a dynamic symbol
               (D) or a normal symbol (a space).

           "F"
           "f"
           "O" The symbol is the name of a function (F) or a file (f) or
               an object (O) or just a normal symbol (a space).

       -T
       --dynamic-syms
           Print the dynamic symbol table entries of the file.  This is
           only meaningful for dynamic objects, such as certain types of
           shared libraries.  This is similar to the information
           provided by the nm program when given the -D (--dynamic)
           option.

           The output format is similar to that produced by the --syms
           option, except that an extra field is inserted before the
           symbol's name, giving the version information associated with
           the symbol.  If the version is the default version to be used
           when resolving unversioned references to the symbol then it's
           displayed as is, otherwise it's put into parentheses.

       --special-syms
           When displaying symbols include those which the target
           considers to be special in some way and which would not
           normally be of interest to the user.

       -U [d|i|l|e|x|h]
       --unicode=[default|invalid|locale|escape|hex|highlight]
           Controls the display of UTF-8 encoded multibyte characters in
           strings.  The default (--unicode=default) is to give them no
           special treatment.  The --unicode=locale option displays the
           sequence in the current locale, which may or may not support
           them.  The options --unicode=hex and --unicode=invalid
           display them as hex byte sequences enclosed by either angle
           brackets or curly braces.

           The --unicode=escape option displays them as escape sequences
           (\uxxxx) and the --unicode=highlight option displays them as
           escape sequences highlighted in red (if supported by the
           output device).  The colouring is intended to draw attention
           to the presence of unicode sequences where they might not be
           expected.

       -V
       --version
           Print the version number of objdump and exit.

       -x
       --all-headers
           Display all available header information, including the
           symbol table and relocation entries.  Using -x is equivalent
           to specifying all of -a -f -h -p -r -t.

       -w
       --wide
           Format some lines for output devices that have more than 80
           columns.  Also do not truncate symbol names when they are
           displayed.

       -z
       --disassemble-zeroes
           Normally the disassembly output will skip blocks of zeroes.
           This option directs the disassembler to disassemble those
           blocks, just like any other data.

       -Z
       --decompress
           The -Z option is meant to be used in conunction with the -s
           option.  It instructs objdump to decompress any compressed
           sections before displaying their contents.

       @file
           Read command-line options from file.  The options read are
           inserted in place of the original @file option.  If file does
           not exist, or cannot be read, then the option will be treated
           literally, and not removed.

           Options in file are separated by whitespace.  A whitespace
           character may be included in an option by surrounding the
           entire option in either single or double quotes.  Any
           character (including a backslash) may be included by
           prefixing the character to be included with a backslash.  The
           file may itself contain additional @file options; any such
           options will be processed recursively.
SEE ALSO
       nm(1), readelf(1), and the Info entries for binutils.
COPYRIGHT
       Copyright (c) 1991-2024 Free Software Foundation, Inc.

       Permission is granted to copy, distribute and/or modify this
       document under the terms of the GNU Free Documentation License,
       Version 1.3 or any later version published by the Free Software
       Foundation; with no Invariant Sections, with no Front-Cover
       Texts, and with no Back-Cover Texts.  A copy of the license is
       included in the section entitled "GNU Free Documentation License".
COLOPHON
       This page is part of the binutils (a collection of tools for
       working with executable binaries) project.  Information about the
       project can be found at http://www.gnu.org/software/binutils/.
       If you have a bug report for this manual page, see
       http://sourceware.org/bugzilla/enter_bug.cgi?product=binutils.
       This page was obtained from the tarball binutils-2.42.tar.gz
       fetched from https://ftp.gnu.org/gnu/binutils/ on 2024-06-14.
       If you discover any rendering problems in this HTML version of
       the page, or you believe there is a better or more up-to-date
       source for the page, or you have corrections or improvements to
       the information in this COLOPHON (which is not part of the
       original manual page), send a mail to man-pages@man7.org

binutils-2.42                  2024-06-14                     OBJDUMP(1)
\end{lstlisting}
}}
\endinput  %  ==  ==  ==  ==  ==  ==  ==  ==  ==

\subsection{\refObjdump: Display Information From Object Files}

{\tiny{
\begin{lstlisting}[language=bash]
NAME
       objdump - display information from object files
SYNOPSIS
       objdump [-a|--archive-headers]
               [-b bfdname|--target=bfdname]
               [-C|--demangle[=style] ]
               [-d|--disassemble[=symbol]]
               [-D|--disassemble-all]
               [-z|--disassemble-zeroes]
               [-EB|-EL|--endian={big | little }]
               [-f|--file-headers]
               [-F|--file-offsets]
               [--file-start-context]
               [-g|--debugging]
               [-e|--debugging-tags]
               [-h|--section-headers|--headers]
               [-i|--info]
               [-j section|--section=section]
               [-l|--line-numbers]
               [-S|--source]
               [--source-comment[=text]]
               [-m machine|--architecture=machine]
               [-M options|--disassembler-options=options]
               [-p|--private-headers]
               [-P options|--private=options]
               [-r|--reloc]
               [-R|--dynamic-reloc]
               [-s|--full-contents]
               [-Z|--decompress]
               [-W[lLiaprmfFsoORtUuTgAck]|
                --dwarf[=rawline,=decodedline,=info,=abbrev,=pubnames,=aranges,=macro,=frames,=frames-interp,=str,=str-offsets,=loc,=Ranges,=pubtypes,=trace_info,=trace_abbrev,=trace_aranges,=gdb_index,=addr,=cu_index,=links]]
               [-WK|--dwarf=follow-links]
               [-WN|--dwarf=no-follow-links]
               [-wD|--dwarf=use-debuginfod]
               [-wE|--dwarf=do-not-use-debuginfod]
               [-L|--process-links]
               [--ctf=section]
               [--sframe=section]
               [-G|--stabs]
               [-t|--syms]
               [-T|--dynamic-syms]
               [-x|--all-headers]
               [-w|--wide]
               [--start-address=address]
               [--stop-address=address]
               [--no-addresses]
               [--prefix-addresses]
               [--[no-]show-raw-insn]
               [--adjust-vma=offset]
               [--show-all-symbols]
               [--dwarf-depth=n]
               [--dwarf-start=n]
               [--ctf-parent=section]
               [--no-recurse-limit|--recurse-limit]
               [--special-syms]
               [--prefix=prefix]
               [--prefix-strip=level]
               [--insn-width=width]
               [--visualize-jumps[=color|=extended-color|=off]
               [--disassembler-color=[off|terminal|on|extended]
               [-U method] [--unicode=method]
               [-V|--version]
               [-H|--help]
               objfile...
DESCRIPTION
       objdump displays information about one or more object files.  The
       options control what particular information to display.  This
       information is mostly useful to programmers who are working on
       the compilation tools, as opposed to programmers who just want
       their program to compile and work.

       objfile... are the object files to be examined.  When you specify
       archives, objdump shows information on each of the member object
       files.
OPTIONS
       The long and short forms of options, shown here as alternatives,
       are equivalent.  At least one option from the list
       -a,-d,-D,-e,-f,-g,-G,-h,-H,-p,-P,-r,-R,-s,-S,-t,-T,-V,-x must be
       given.

       -a
       --archive-header
           If any of the objfile files are archives, display the archive
           header information (in a format similar to ls -l).  Besides
           the information you could list with ar tv, objdump -a shows
           the object file format of each archive member.

       --adjust-vma=offset
           When dumping information, first add offset to all the section
           addresses.  This is useful if the section addresses do not
           correspond to the symbol table, which can happen when putting
           sections at particular addresses when using a format which
           can not represent section addresses, such as a.out.

       -b bfdname
       --target=bfdname
           Specify that the object-code format for the object files is
           bfdname.  This option may not be necessary; objdump can
           automatically recognize many formats.

           For example,

                   objdump -b oasys -m vax -h fu.o

           displays summary information from the section headers (-h) of
           fu.o, which is explicitly identified (-m) as a VAX object
           file in the format produced by Oasys compilers.  You can list
           the formats available with the -i option.

       -C
       --demangle[=style]
           Decode (demangle) low-level symbol names into user-level
           names.  Besides removing any initial underscore prepended by
           the system, this makes C++ function names readable.
           Different compilers have different mangling styles. The
           optional demangling style argument can be used to choose an
           appropriate demangling style for your compiler.

       --recurse-limit
       --no-recurse-limit
       --recursion-limit
       --no-recursion-limit
           Enables or disables a limit on the amount of recursion
           performed whilst demangling strings.  Since the name mangling
           formats allow for an infinite level of recursion it is
           possible to create strings whose decoding will exhaust the
           amount of stack space available on the host machine,
           triggering a memory fault.  The limit tries to prevent this
           from happening by restricting recursion to 2048 levels of
           nesting.

           The default is for this limit to be enabled, but disabling it
           may be necessary in order to demangle truly complicated
           names.  Note however that if the recursion limit is disabled
           then stack exhaustion is possible and any bug reports about
           such an event will be rejected.

       -g
       --debugging
           Display debugging information.  This attempts to parse STABS
           debugging format information stored in the file and print it
           out using a C like syntax.  If no STABS debugging was found
           this option falls back on the -W option to print any DWARF
           information in the file.

       -e
       --debugging-tags
           Like -g, but the information is generated in a format
           compatible with ctags tool.

       -d
       --disassemble
       --disassemble=symbol
           Display the assembler mnemonics for the machine instructions
           from the input file.  This option only disassembles those
           sections which are expected to contain instructions.  If the
           optional symbol argument is given, then display the assembler
           mnemonics starting at symbol.  If symbol is a function name
           then disassembly will stop at the end of the function,
           otherwise it will stop when the next symbol is encountered.
           If there are no matches for symbol then nothing will be
           displayed.

           Note if the --dwarf=follow-links option is enabled then any
           symbol tables in linked debug info files will be read in and
           used when disassembling.

       -D
       --disassemble-all
           Like -d, but disassemble the contents of all non-empty non-
           bss sections, not just those expected to contain
           instructions.  -j may be used to select specific sections.

           This option also has a subtle effect on the disassembly of
           instructions in code sections.  When option -d is in effect
           objdump will assume that any symbols present in a code
           section occur on the boundary between instructions and it
           will refuse to disassemble across such a boundary.  When
           option -D is in effect however this assumption is supressed.
           This means that it is possible for the output of -d and -D to
           differ if, for example, data is stored in code sections.

           If the target is an ARM architecture this switch also has the
           effect of forcing the disassembler to decode pieces of data
           found in code sections as if they were instructions.

           Note if the --dwarf=follow-links option is enabled then any
           symbol tables in linked debug info files will be read in and
           used when disassembling.

       --no-addresses
           When disassembling, don't print addresses on each line or for
           symbols and relocation offsets.  In combination with
           --no-show-raw-insn this may be useful for comparing compiler
           output.

       --prefix-addresses
           When disassembling, print the complete address on each line.
           This is the older disassembly format.

       -EB
       -EL
       --endian={big|little}
           Specify the endianness of the object files.  This only
           affects disassembly.  This can be useful when disassembling a
           file format which does not describe endianness information,
           such as S-records.

       -f
       --file-headers
           Display summary information from the overall header of each
           of the objfile files.

       -F
       --file-offsets
           When disassembling sections, whenever a symbol is displayed,
           also display the file offset of the region of data that is
           about to be dumped.  If zeroes are being skipped, then when
           disassembly resumes, tell the user how many zeroes were
           skipped and the file offset of the location from where the
           disassembly resumes.  When dumping sections, display the file
           offset of the location from where the dump starts.

       --file-start-context
           Specify that when displaying interlisted source
           code/disassembly (assumes -S) from a file that has not yet
           been displayed, extend the context to the start of the file.

       -h
       --section-headers
       --headers
           Display summary information from the section headers of the
           object file.

           File segments may be relocated to nonstandard addresses, for
           example by using the -Ttext, -Tdata, or -Tbss options to ld.
           However, some object file formats, such as a.out, do not
           store the starting address of the file segments.  In those
           situations, although ld relocates the sections correctly,
           using objdump -h to list the file section headers cannot show
           the correct addresses.  Instead, it shows the usual
           addresses, which are implicit for the target.

           Note, in some cases it is possible for a section to have both
           the READONLY and the NOREAD attributes set.  In such cases
           the NOREAD attribute takes precedence, but objdump will
           report both since the exact setting of the flag bits might be
           important.

       -H
       --help
           Print a summary of the options to objdump and exit.

       -i
       --info
           Display a list showing all architectures and object formats
           available for specification with -b or -m.

       -j name
       --section=name
           Display information for section name.  This option may be
           specified multiple times.

       -L
       --process-links
           Display the contents of non-debug sections found in separate
           debuginfo files that are linked to the main file.  This
           option automatically implies the -WK option, and only
           sections requested by other command line options will be
           displayed.

       -l
       --line-numbers
           Label the display (using debugging information) with the
           filename and source line numbers corresponding to the object
           code or relocs shown.  Only useful with -d, -D, or -r.

       -m machine
       --architecture=machine
           Specify the architecture to use when disassembling object
           files.  This can be useful when disassembling object files
           which do not describe architecture information, such as
           S-records.  You can list the available architectures with the
           -i option.

           For most architectures it is possible to supply an
           architecture name and a machine name, separated by a colon.
           For example foo:bar would refer to the bar machine type in
           the foo architecture.  This can be helpful if objdump has
           been configured to support multiple architectures.

           If the target is an ARM architecture then this switch has an
           additional effect.  It restricts the disassembly to only
           those instructions supported by the architecture specified by
           machine.  If it is necessary to use this switch because the
           input file does not contain any architecture information, but
           it is also desired to disassemble all the instructions use
           -marm.

       -M options
       --disassembler-options=options
           Pass target specific information to the disassembler.  Only
           supported on some targets.  If it is necessary to specify
           more than one disassembler option then multiple -M options
           can be used or can be placed together into a comma separated
           list.

           For ARC, dsp controls the printing of DSP instructions, spfp
           selects the printing of FPX single precision FP instructions,
           dpfp selects the printing of FPX double precision FP
           instructions, quarkse_em selects the printing of special
           QuarkSE-EM instructions, fpuda selects the printing of double
           precision assist instructions, fpus selects the printing of
           FPU single precision FP instructions, while fpud selects the
           printing of FPU double precision FP instructions.
           Additionally, one can choose to have all the immediates
           printed in hexadecimal using hex.  By default, the short
           immediates are printed using the decimal representation,
           while the long immediate values are printed as hexadecimal.

           cpu=... allows one to enforce a particular ISA when
           disassembling instructions, overriding the -m value or
           whatever is in the ELF file.  This might be useful to select
           ARC EM or HS ISA, because architecture is same for those and
           disassembler relies on private ELF header data to decide if
           code is for EM or HS.  This option might be specified
           multiple times - only the latest value will be used.  Valid
           values are same as for the assembler -mcpu=... option.

           If the target is an ARM architecture then this switch can be
           used to select which register name set is used during
           disassembler.  Specifying -M reg-names-std (the default) will
           select the register names as used in ARM's instruction set
           documentation, but with register 13 called 'sp', register 14
           called 'lr' and register 15 called 'pc'.  Specifying -M reg-
           names-apcs will select the name set used by the ARM Procedure
           Call Standard, whilst specifying -M reg-names-raw will just
           use r followed by the register number.

           There are also two variants on the APCS register naming
           scheme enabled by -M reg-names-atpcs and -M reg-names-
           special-atpcs which use the ARM/Thumb Procedure Call Standard
           naming conventions.  (Either with the normal register names
           or the special register names).

           This option can also be used for ARM architectures to force
           the disassembler to interpret all instructions as Thumb
           instructions by using the switch
           --disassembler-options=force-thumb.  This can be useful when
           attempting to disassemble thumb code produced by other
           compilers.

           For AArch64 targets this switch can be used to set whether
           instructions are disassembled as the most general instruction
           using the -M no-aliases option or whether instruction notes
           should be generated as comments in the disasssembly using -M
           notes.

           For the x86, some of the options duplicate functions of the
           -m switch, but allow finer grained control.

           "x86-64"
           "i386"
           "i8086"
               Select disassembly for the given architecture.

           "intel"
           "att"
               Select between intel syntax mode and AT&T syntax mode.

           "amd64"
           "intel64"
               Select between AMD64 ISA and Intel64 ISA.

           "intel-mnemonic"
           "att-mnemonic"
               Select between intel mnemonic mode and AT&T mnemonic
               mode.  Note: "intel-mnemonic" implies "intel" and
               "att-mnemonic" implies "att".

           "addr64"
           "addr32"
           "addr16"
           "data32"
           "data16"
               Specify the default address size and operand size.  These
               five options will be overridden if "x86-64", "i386" or
               "i8086" appear later in the option string.

           "suffix"
               When in AT&T mode and also for a limited set of
               instructions when in Intel mode, instructs the
               disassembler to print a mnemonic suffix even when the
               suffix could be inferred by the operands or, for certain
               instructions, the execution mode's defaults.

           For PowerPC, the -M argument raw selects disasssembly of
           hardware insns rather than aliases.  For example, you will
           see "rlwinm" rather than "clrlwi", and "addi" rather than
           "li".  All of the -m arguments for gas that select a CPU are
           supported.  These are: 403, 405, 440, 464, 476, 601, 603,
           604, 620, 7400, 7410, 7450, 7455, 750cl, 821, 850, 860, a2,
           booke, booke32, cell, com, e200z2, e200z4, e300, e500,
           e500mc, e500mc64, e500x2, e5500, e6500, efs, power4, power5,
           power6, power7, power8, power9, power10, ppc, ppc32, ppc64,
           ppc64bridge, ppcps, pwr, pwr2, pwr4, pwr5, pwr5x, pwr6, pwr7,
           pwr8, pwr9, pwr10, pwrx, titan, vle, and future.  32 and 64
           modify the default or a prior CPU selection, disabling and
           enabling 64-bit insns respectively.  In addition, altivec,
           any, lsp, htm, vsx, spe and  spe2 add capabilities to a
           previous or later CPU selection.  any will disassemble any
           opcode known to binutils, but in cases where an opcode has
           two different meanings or different arguments, you may not
           see the disassembly you expect.  If you disassemble without
           giving a CPU selection, a default will be chosen from
           information gleaned by BFD from the object files headers, but
           the result again may not be as you expect.

           For MIPS, this option controls the printing of instruction
           mnemonic names and register names in disassembled
           instructions.  Multiple selections from the following may be
           specified as a comma separated string, and invalid options
           are ignored:

           "no-aliases"
               Print the 'raw' instruction mnemonic instead of some
               pseudo instruction mnemonic.  I.e., print 'daddu' or 'or'
               instead of 'move', 'sll' instead of 'nop', etc.

           "msa"
               Disassemble MSA instructions.

           "virt"
               Disassemble the virtualization ASE instructions.

           "xpa"
               Disassemble the eXtended Physical Address (XPA) ASE
               instructions.

           "gpr-names=ABI"
               Print GPR (general-purpose register) names as appropriate
               for the specified ABI.  By default, GPR names are
               selected according to the ABI of the binary being
               disassembled.

           "fpr-names=ABI"
               Print FPR (floating-point register) names as appropriate
               for the specified ABI.  By default, FPR numbers are
               printed rather than names.

           "cp0-names=ARCH"
               Print CP0 (system control coprocessor; coprocessor 0)
               register names as appropriate for the CPU or architecture
               specified by ARCH.  By default, CP0 register names are
               selected according to the architecture and CPU of the
               binary being disassembled.

           "hwr-names=ARCH"
               Print HWR (hardware register, used by the "rdhwr"
               instruction) names as appropriate for the CPU or
               architecture specified by ARCH.  By default, HWR names
               are selected according to the architecture and CPU of the
               binary being disassembled.

           "reg-names=ABI"
               Print GPR and FPR names as appropriate for the selected
               ABI.

           "reg-names=ARCH"
               Print CPU-specific register names (CP0 register and HWR
               names) as appropriate for the selected CPU or
               architecture.

           For any of the options listed above, ABI or ARCH may be
           specified as numeric to have numbers printed rather than
           names, for the selected types of registers.  You can list the
           available values of ABI and ARCH using the --help option.

           For VAX, you can specify function entry addresses with -M
           entry:0xf00ba.  You can use this multiple times to properly
           disassemble VAX binary files that don't contain symbol tables
           (like ROM dumps).  In these cases, the function entry mask
           would otherwise be decoded as VAX instructions, which would
           probably lead the rest of the function being wrongly
           disassembled.

       -p
       --private-headers
           Print information that is specific to the object file format.
           The exact information printed depends upon the object file
           format.  For some object file formats, no additional
           information is printed.

       -P options
       --private=options
           Print information that is specific to the object file format.
           The argument options is a comma separated list that depends
           on the format (the lists of options is displayed with the
           help).

           For XCOFF, the available options are:

           "header"
           "aout"
           "sections"
           "syms"
           "relocs"
           "lineno,"
           "loader"
           "except"
           "typchk"
           "traceback"
           "toc"
           "ldinfo"

           For PE, the available options are:

           "header"
           "sections"

           Not all object formats support this option.  In particular
           the ELF format does not use it.

       -r
       --reloc
           Print the relocation entries of the file.  If used with -d or
           -D, the relocations are printed interspersed with the
           disassembly.

       -R
       --dynamic-reloc
           Print the dynamic relocation entries of the file.  This is
           only meaningful for dynamic objects, such as certain types of
           shared libraries.  As for -r, if used with -d or -D, the
           relocations are printed interspersed with the disassembly.

       -s
       --full-contents
           Display the full contents of sections, often used in
           combination with -j to request specific sections.  By default
           all non-empty non-bss sections are displayed.  By default any
           compressed section will be displayed in its compressed form.
           In order to see the contents in a decompressed form add the
           -Z option to the command line.

       -S
       --source
           Display source code intermixed with disassembly, if possible.
           Implies -d.

       --show-all-symbols
           When disassembling, show all the symbols that match a given
           address, not just the first one.

       --source-comment[=txt]
           Like the -S option, but all source code lines are displayed
           with a prefix of txt.  Typically txt will be a comment string
           which can be used to distinguish the assembler code from the
           source code.  If txt is not provided then a default string of
           "# " (hash followed by a space), will be used.

       --prefix=prefix
           Specify prefix to add to the absolute paths when used with
           -S.

       --prefix-strip=level
           Indicate how many initial directory names to strip off the
           hardwired absolute paths. It has no effect without
           --prefix=prefix.

       --show-raw-insn
           When disassembling instructions, print the instruction in hex
           as well as in symbolic form.  This is the default except when
           --prefix-addresses is used.

       --no-show-raw-insn
           When disassembling instructions, do not print the instruction
           bytes.  This is the default when --prefix-addresses is used.

       --insn-width=width
           Display width bytes on a single line when disassembling
           instructions.

       --visualize-jumps[=color|=extended-color|=off]
           Visualize jumps that stay inside a function by drawing ASCII
           art between the start and target addresses.  The optional
           =color argument adds color to the output using simple
           terminal colors.  Alternatively the =extended-color argument
           will add color using 8bit colors, but these might not work on
           all terminals.

           If it is necessary to disable the visualize-jumps option
           after it has previously been enabled then use
           visualize-jumps=off.

       --disassembler-color=off
       --disassembler-color=terminal
       --disassembler-color=on|color|colour
       --disassembler-color=extened|extended-color|extened-colour
           Enables or disables the use of colored syntax highlighting in
           disassembly output.  The default behaviour is determined via
           a configure time option.  Note, not all architectures support
           colored syntax highlighting, and depending upon the terminal
           used, colored output may not actually be legible.

           The on argument adds colors using simple terminal colors.

           The terminal argument does the same, but only if the output
           device is a terminal.

           The extended-color argument is similar to the on argument,
           but it uses 8-bit colors.  These may not work on all
           terminals.

           The off argument disables colored disassembly.

       -W[lLiaprmfFsoORtUuTgAckK]
       --dwarf[=rawline,=decodedline,=info,=abbrev,=pubnames,=aranges,=macro,=frames,=frames-interp,=str,=str-offsets,=loc,=Ranges,=pubtypes,=trace_info,=trace_abbrev,=trace_aranges,=gdb_index,=addr,=cu_index,=links,=follow-links]
           Displays the contents of the DWARF debug sections in the
           file, if any are present.  Compressed debug sections are
           automatically decompressed (temporarily) before they are
           displayed.  If one or more of the optional letters or words
           follows the switch then only those type(s) of data will be
           dumped.  The letters and words refer to the following
           information:

           "a"
           "=abbrev"
               Displays the contents of the .debug_abbrev section.

           "A"
           "=addr"
               Displays the contents of the .debug_addr section.

           "c"
           "=cu_index"
               Displays the contents of the .debug_cu_index and/or
               .debug_tu_index sections.

           "f"
           "=frames"
               Display the raw contents of a .debug_frame section.

           "F"
           "=frames-interp"
               Display the interpreted contents of a .debug_frame
               section.

           "g"
           "=gdb_index"
               Displays the contents of the .gdb_index and/or
               .debug_names sections.

           "i"
           "=info"
               Displays the contents of the .debug_info section.  Note:
               the output from this option can also be restricted by the
               use of the --dwarf-depth and --dwarf-start options.

           "k"
           "=links"
               Displays the contents of the .gnu_debuglink,
               .gnu_debugaltlink and .debug_sup sections, if any of them
               are present.  Also displays any links to separate dwarf
               object files (dwo), if they are specified by the
               DW_AT_GNU_dwo_name or DW_AT_dwo_name attributes in the
               .debug_info section.

           "K"
           "=follow-links"
               Display the contents of any selected debug sections that
               are found in linked, separate debug info file(s).  This
               can result in multiple versions of the same debug section
               being displayed if it exists in more than one file.

               In addition, when displaying DWARF attributes, if a form
               is found that references the separate debug info file,
               then the referenced contents will also be displayed.

               Note - in some distributions this option is enabled by
               default.  It can be disabled via the N debug option.  The
               default can be chosen when configuring the binutils via
               the --enable-follow-debug-links=yes or
               --enable-follow-debug-links=no options.  If these are not
               used then the default is to enable the following of debug
               links.

               Note - if support for the debuginfod protocol was enabled
               when the binutils were built then this option will also
               include an attempt to contact any debuginfod servers
               mentioned in the DEBUGINFOD_URLS environment variable.
               This could take some time to resolve.  This behaviour can
               be disabled via the =do-not-use-debuginfod debug option.

           "N"
           "=no-follow-links"
               Disables the following of links to separate debug info
               files.

           "D"
           "=use-debuginfod"
               Enables contacting debuginfod servers if there is a need
               to follow debug links.  This is the default behaviour.

           "E"
           "=do-not-use-debuginfod"
               Disables contacting debuginfod servers when there is a
               need to follow debug links.

           "l"
           "=rawline"
               Displays the contents of the .debug_line section in a raw
               format.

           "L"
           "=decodedline"
               Displays the interpreted contents of the .debug_line
               section.

           "m"
           "=macro"
               Displays the contents of the .debug_macro and/or
               .debug_macinfo sections.

           "o"
           "=loc"
               Displays the contents of the .debug_loc and/or
               .debug_loclists sections.

           "O"
           "=str-offsets"
               Displays the contents of the .debug_str_offsets section.

           "p"
           "=pubnames"
               Displays the contents of the .debug_pubnames and/or
               .debug_gnu_pubnames sections.

           "r"
           "=aranges"
               Displays the contents of the .debug_aranges section.

           "R"
           "=Ranges"
               Displays the contents of the .debug_ranges and/or
               .debug_rnglists sections.

           "s"
           "=str"
               Displays the contents of the .debug_str, .debug_line_str
               and/or .debug_str_offsets sections.

           "t"
           "=pubtype"
               Displays the contents of the .debug_pubtypes and/or
               .debug_gnu_pubtypes sections.

           "T"
           "=trace_aranges"
               Displays the contents of the .trace_aranges section.

           "u"
           "=trace_abbrev"
               Displays the contents of the .trace_abbrev section.

           "U"
           "=trace_info"
               Displays the contents of the .trace_info section.

           Note: displaying the contents of .debug_static_funcs,
           .debug_static_vars and debug_weaknames sections is not
           currently supported.

       --dwarf-depth=n
           Limit the dump of the ".debug_info" section to n children.
           This is only useful with --debug-dump=info.  The default is
           to print all DIEs; the special value 0 for n will also have
           this effect.

           With a non-zero value for n, DIEs at or deeper than n levels
           will not be printed.  The range for n is zero-based.

       --dwarf-start=n
           Print only DIEs beginning with the DIE numbered n.  This is
           only useful with --debug-dump=info.

           If specified, this option will suppress printing of any
           header information and all DIEs before the DIE numbered n.
           Only siblings and children of the specified DIE will be
           printed.

           This can be used in conjunction with --dwarf-depth.

       --dwarf-check
           Enable additional checks for consistency of Dwarf
           information.

       --ctf[=section]
           Display the contents of the specified CTF section.  CTF
           sections themselves contain many subsections, all of which
           are displayed in order.

           By default, display the name of the section named .ctf, which
           is the name emitted by ld.

       --ctf-parent=member
           If the CTF section contains ambiguously-defined types, it
           will consist of an archive of many CTF dictionaries, all
           inheriting from one dictionary containing unambiguous types.
           This member is by default named .ctf, like the section
           containing it, but it is possible to change this name using
           the "ctf_link_set_memb_name_changer" function at link time.
           When looking at CTF archives that have been created by a
           linker that uses the name changer to rename the parent
           archive member, --ctf-parent can be used to specify the name
           used for the parent.

       --sframe[=section]
           Display the contents of the specified SFrame section.

           By default, display the name of the section named .sframe,
           which is the name emitted by ld.

       -G
       --stabs
           Display the full contents of any sections requested.  Display
           the contents of the .stab and .stab.index and .stab.excl
           sections from an ELF file.  This is only useful on systems
           (such as Solaris 2.0) in which ".stab" debugging symbol-table
           entries are carried in an ELF section.  In most other file
           formats, debugging symbol-table entries are interleaved with
           linkage symbols, and are visible in the --syms output.

       --start-address=address
           Start displaying data at the specified address.  This affects
           the output of the -d, -r and -s options.

       --stop-address=address
           Stop displaying data at the specified address.  This affects
           the output of the -d, -r and -s options.

       -t
       --syms
           Print the symbol table entries of the file.  This is similar
           to the information provided by the nm program, although the
           display format is different.  The format of the output
           depends upon the format of the file being dumped, but there
           are two main types.  One looks like this:

                   [  4](sec  3)(fl 0x00)(ty   0)(scl   3) (nx 1) 0x00000000 .bss
                   [  6](sec  1)(fl 0x00)(ty   0)(scl   2) (nx 0) 0x00000000 fred

           where the number inside the square brackets is the number of
           the entry in the symbol table, the sec number is the section
           number, the fl value are the symbol's flag bits, the ty
           number is the symbol's type, the scl number is the symbol's
           storage class and the nx value is the number of auxiliary
           entries associated with the symbol.  The last two fields are
           the symbol's value and its name.

           The other common output format, usually seen with ELF based
           files, looks like this:

                   00000000 l    d  .bss   00000000 .bss
                   00000000 g       .text  00000000 fred

           Here the first number is the symbol's value (sometimes
           referred to as its address).  The next field is actually a
           set of characters and spaces indicating the flag bits that
           are set on the symbol.  These characters are described below.
           Next is the section with which the symbol is associated or
           *ABS* if the section is absolute (ie not connected with any
           section), or *UND* if the section is referenced in the file
           being dumped, but not defined there.

           After the section name comes another field, a number, which
           for common symbols is the alignment and for other symbol is
           the size.  Finally the symbol's name is displayed.

           The flag characters are divided into 7 groups as follows:

           "l"
           "g"
           "u"
           "!" The symbol is a local (l), global (g), unique global (u),
               neither global nor local (a space) or both global and
               local (!).  A symbol can be neither local or global for a
               variety of reasons, e.g., because it is used for
               debugging, but it is probably an indication of a bug if
               it is ever both local and global.  Unique global symbols
               are a GNU extension to the standard set of ELF symbol
               bindings.  For such a symbol the dynamic linker will make
               sure that in the entire process there is just one symbol
               with this name and type in use.

           "w" The symbol is weak (w) or strong (a space).

           "C" The symbol denotes a constructor (C) or an ordinary
               symbol (a space).

           "W" The symbol is a warning (W) or a normal symbol (a space).
               A warning symbol's name is a message to be displayed if
               the symbol following the warning symbol is ever
               referenced.

           "I"
           "i" The symbol is an indirect reference to another symbol
               (I), a function to be evaluated during reloc processing
               (i) or a normal symbol (a space).

           "d"
           "D" The symbol is a debugging symbol (d) or a dynamic symbol
               (D) or a normal symbol (a space).

           "F"
           "f"
           "O" The symbol is the name of a function (F) or a file (f) or
               an object (O) or just a normal symbol (a space).

       -T
       --dynamic-syms
           Print the dynamic symbol table entries of the file.  This is
           only meaningful for dynamic objects, such as certain types of
           shared libraries.  This is similar to the information
           provided by the nm program when given the -D (--dynamic)
           option.

           The output format is similar to that produced by the --syms
           option, except that an extra field is inserted before the
           symbol's name, giving the version information associated with
           the symbol.  If the version is the default version to be used
           when resolving unversioned references to the symbol then it's
           displayed as is, otherwise it's put into parentheses.

       --special-syms
           When displaying symbols include those which the target
           considers to be special in some way and which would not
           normally be of interest to the user.

       -U [d|i|l|e|x|h]
       --unicode=[default|invalid|locale|escape|hex|highlight]
           Controls the display of UTF-8 encoded multibyte characters in
           strings.  The default (--unicode=default) is to give them no
           special treatment.  The --unicode=locale option displays the
           sequence in the current locale, which may or may not support
           them.  The options --unicode=hex and --unicode=invalid
           display them as hex byte sequences enclosed by either angle
           brackets or curly braces.

           The --unicode=escape option displays them as escape sequences
           (\uxxxx) and the --unicode=highlight option displays them as
           escape sequences highlighted in red (if supported by the
           output device).  The colouring is intended to draw attention
           to the presence of unicode sequences where they might not be
           expected.

       -V
       --version
           Print the version number of objdump and exit.

       -x
       --all-headers
           Display all available header information, including the
           symbol table and relocation entries.  Using -x is equivalent
           to specifying all of -a -f -h -p -r -t.

       -w
       --wide
           Format some lines for output devices that have more than 80
           columns.  Also do not truncate symbol names when they are
           displayed.

       -z
       --disassemble-zeroes
           Normally the disassembly output will skip blocks of zeroes.
           This option directs the disassembler to disassemble those
           blocks, just like any other data.

       -Z
       --decompress
           The -Z option is meant to be used in conunction with the -s
           option.  It instructs objdump to decompress any compressed
           sections before displaying their contents.

       @file
           Read command-line options from file.  The options read are
           inserted in place of the original @file option.  If file does
           not exist, or cannot be read, then the option will be treated
           literally, and not removed.

           Options in file are separated by whitespace.  A whitespace
           character may be included in an option by surrounding the
           entire option in either single or double quotes.  Any
           character (including a backslash) may be included by
           prefixing the character to be included with a backslash.  The
           file may itself contain additional @file options; any such
           options will be processed recursively.
SEE ALSO
       nm(1), readelf(1), and the Info entries for binutils.
COPYRIGHT
       Copyright (c) 1991-2024 Free Software Foundation, Inc.

       Permission is granted to copy, distribute and/or modify this
       document under the terms of the GNU Free Documentation License,
       Version 1.3 or any later version published by the Free Software
       Foundation; with no Invariant Sections, with no Front-Cover
       Texts, and with no Back-Cover Texts.  A copy of the license is
       included in the section entitled "GNU Free Documentation License".
COLOPHON
       This page is part of the binutils (a collection of tools for
       working with executable binaries) project.  Information about the
       project can be found at http://www.gnu.org/software/binutils/.
       If you have a bug report for this manual page, see
       http://sourceware.org/bugzilla/enter_bug.cgi?product=binutils.
       This page was obtained from the tarball binutils-2.42.tar.gz
       fetched from https://ftp.gnu.org/gnu/binutils/ on 2024-06-14.
       If you discover any rendering problems in this HTML version of
       the page, or you believe there is a better or more up-to-date
       source for the page, or you have corrections or improvements to
       the information in this COLOPHON (which is not part of the
       original manual page), send a mail to man-pages@man7.org

binutils-2.42                  2024-06-14                     OBJDUMP(1)
\end{lstlisting}
}}
\endinput  %  ==  ==  ==  ==  ==  ==  ==  ==  ==

\subsection{\refObjdump: Display Information From Object Files}

{\tiny{
\begin{lstlisting}[language=bash]
NAME
       objdump - display information from object files
SYNOPSIS
       objdump [-a|--archive-headers]
               [-b bfdname|--target=bfdname]
               [-C|--demangle[=style] ]
               [-d|--disassemble[=symbol]]
               [-D|--disassemble-all]
               [-z|--disassemble-zeroes]
               [-EB|-EL|--endian={big | little }]
               [-f|--file-headers]
               [-F|--file-offsets]
               [--file-start-context]
               [-g|--debugging]
               [-e|--debugging-tags]
               [-h|--section-headers|--headers]
               [-i|--info]
               [-j section|--section=section]
               [-l|--line-numbers]
               [-S|--source]
               [--source-comment[=text]]
               [-m machine|--architecture=machine]
               [-M options|--disassembler-options=options]
               [-p|--private-headers]
               [-P options|--private=options]
               [-r|--reloc]
               [-R|--dynamic-reloc]
               [-s|--full-contents]
               [-Z|--decompress]
               [-W[lLiaprmfFsoORtUuTgAck]|
                --dwarf[=rawline,=decodedline,=info,=abbrev,=pubnames,=aranges,=macro,=frames,=frames-interp,=str,=str-offsets,=loc,=Ranges,=pubtypes,=trace_info,=trace_abbrev,=trace_aranges,=gdb_index,=addr,=cu_index,=links]]
               [-WK|--dwarf=follow-links]
               [-WN|--dwarf=no-follow-links]
               [-wD|--dwarf=use-debuginfod]
               [-wE|--dwarf=do-not-use-debuginfod]
               [-L|--process-links]
               [--ctf=section]
               [--sframe=section]
               [-G|--stabs]
               [-t|--syms]
               [-T|--dynamic-syms]
               [-x|--all-headers]
               [-w|--wide]
               [--start-address=address]
               [--stop-address=address]
               [--no-addresses]
               [--prefix-addresses]
               [--[no-]show-raw-insn]
               [--adjust-vma=offset]
               [--show-all-symbols]
               [--dwarf-depth=n]
               [--dwarf-start=n]
               [--ctf-parent=section]
               [--no-recurse-limit|--recurse-limit]
               [--special-syms]
               [--prefix=prefix]
               [--prefix-strip=level]
               [--insn-width=width]
               [--visualize-jumps[=color|=extended-color|=off]
               [--disassembler-color=[off|terminal|on|extended]
               [-U method] [--unicode=method]
               [-V|--version]
               [-H|--help]
               objfile...
DESCRIPTION
       objdump displays information about one or more object files.  The
       options control what particular information to display.  This
       information is mostly useful to programmers who are working on
       the compilation tools, as opposed to programmers who just want
       their program to compile and work.

       objfile... are the object files to be examined.  When you specify
       archives, objdump shows information on each of the member object
       files.
OPTIONS
       The long and short forms of options, shown here as alternatives,
       are equivalent.  At least one option from the list
       -a,-d,-D,-e,-f,-g,-G,-h,-H,-p,-P,-r,-R,-s,-S,-t,-T,-V,-x must be
       given.

       -a
       --archive-header
           If any of the objfile files are archives, display the archive
           header information (in a format similar to ls -l).  Besides
           the information you could list with ar tv, objdump -a shows
           the object file format of each archive member.

       --adjust-vma=offset
           When dumping information, first add offset to all the section
           addresses.  This is useful if the section addresses do not
           correspond to the symbol table, which can happen when putting
           sections at particular addresses when using a format which
           can not represent section addresses, such as a.out.

       -b bfdname
       --target=bfdname
           Specify that the object-code format for the object files is
           bfdname.  This option may not be necessary; objdump can
           automatically recognize many formats.

           For example,

                   objdump -b oasys -m vax -h fu.o

           displays summary information from the section headers (-h) of
           fu.o, which is explicitly identified (-m) as a VAX object
           file in the format produced by Oasys compilers.  You can list
           the formats available with the -i option.

       -C
       --demangle[=style]
           Decode (demangle) low-level symbol names into user-level
           names.  Besides removing any initial underscore prepended by
           the system, this makes C++ function names readable.
           Different compilers have different mangling styles. The
           optional demangling style argument can be used to choose an
           appropriate demangling style for your compiler.

       --recurse-limit
       --no-recurse-limit
       --recursion-limit
       --no-recursion-limit
           Enables or disables a limit on the amount of recursion
           performed whilst demangling strings.  Since the name mangling
           formats allow for an infinite level of recursion it is
           possible to create strings whose decoding will exhaust the
           amount of stack space available on the host machine,
           triggering a memory fault.  The limit tries to prevent this
           from happening by restricting recursion to 2048 levels of
           nesting.

           The default is for this limit to be enabled, but disabling it
           may be necessary in order to demangle truly complicated
           names.  Note however that if the recursion limit is disabled
           then stack exhaustion is possible and any bug reports about
           such an event will be rejected.

       -g
       --debugging
           Display debugging information.  This attempts to parse STABS
           debugging format information stored in the file and print it
           out using a C like syntax.  If no STABS debugging was found
           this option falls back on the -W option to print any DWARF
           information in the file.

       -e
       --debugging-tags
           Like -g, but the information is generated in a format
           compatible with ctags tool.

       -d
       --disassemble
       --disassemble=symbol
           Display the assembler mnemonics for the machine instructions
           from the input file.  This option only disassembles those
           sections which are expected to contain instructions.  If the
           optional symbol argument is given, then display the assembler
           mnemonics starting at symbol.  If symbol is a function name
           then disassembly will stop at the end of the function,
           otherwise it will stop when the next symbol is encountered.
           If there are no matches for symbol then nothing will be
           displayed.

           Note if the --dwarf=follow-links option is enabled then any
           symbol tables in linked debug info files will be read in and
           used when disassembling.

       -D
       --disassemble-all
           Like -d, but disassemble the contents of all non-empty non-
           bss sections, not just those expected to contain
           instructions.  -j may be used to select specific sections.

           This option also has a subtle effect on the disassembly of
           instructions in code sections.  When option -d is in effect
           objdump will assume that any symbols present in a code
           section occur on the boundary between instructions and it
           will refuse to disassemble across such a boundary.  When
           option -D is in effect however this assumption is supressed.
           This means that it is possible for the output of -d and -D to
           differ if, for example, data is stored in code sections.

           If the target is an ARM architecture this switch also has the
           effect of forcing the disassembler to decode pieces of data
           found in code sections as if they were instructions.

           Note if the --dwarf=follow-links option is enabled then any
           symbol tables in linked debug info files will be read in and
           used when disassembling.

       --no-addresses
           When disassembling, don't print addresses on each line or for
           symbols and relocation offsets.  In combination with
           --no-show-raw-insn this may be useful for comparing compiler
           output.

       --prefix-addresses
           When disassembling, print the complete address on each line.
           This is the older disassembly format.

       -EB
       -EL
       --endian={big|little}
           Specify the endianness of the object files.  This only
           affects disassembly.  This can be useful when disassembling a
           file format which does not describe endianness information,
           such as S-records.

       -f
       --file-headers
           Display summary information from the overall header of each
           of the objfile files.

       -F
       --file-offsets
           When disassembling sections, whenever a symbol is displayed,
           also display the file offset of the region of data that is
           about to be dumped.  If zeroes are being skipped, then when
           disassembly resumes, tell the user how many zeroes were
           skipped and the file offset of the location from where the
           disassembly resumes.  When dumping sections, display the file
           offset of the location from where the dump starts.

       --file-start-context
           Specify that when displaying interlisted source
           code/disassembly (assumes -S) from a file that has not yet
           been displayed, extend the context to the start of the file.

       -h
       --section-headers
       --headers
           Display summary information from the section headers of the
           object file.

           File segments may be relocated to nonstandard addresses, for
           example by using the -Ttext, -Tdata, or -Tbss options to ld.
           However, some object file formats, such as a.out, do not
           store the starting address of the file segments.  In those
           situations, although ld relocates the sections correctly,
           using objdump -h to list the file section headers cannot show
           the correct addresses.  Instead, it shows the usual
           addresses, which are implicit for the target.

           Note, in some cases it is possible for a section to have both
           the READONLY and the NOREAD attributes set.  In such cases
           the NOREAD attribute takes precedence, but objdump will
           report both since the exact setting of the flag bits might be
           important.

       -H
       --help
           Print a summary of the options to objdump and exit.

       -i
       --info
           Display a list showing all architectures and object formats
           available for specification with -b or -m.

       -j name
       --section=name
           Display information for section name.  This option may be
           specified multiple times.

       -L
       --process-links
           Display the contents of non-debug sections found in separate
           debuginfo files that are linked to the main file.  This
           option automatically implies the -WK option, and only
           sections requested by other command line options will be
           displayed.

       -l
       --line-numbers
           Label the display (using debugging information) with the
           filename and source line numbers corresponding to the object
           code or relocs shown.  Only useful with -d, -D, or -r.

       -m machine
       --architecture=machine
           Specify the architecture to use when disassembling object
           files.  This can be useful when disassembling object files
           which do not describe architecture information, such as
           S-records.  You can list the available architectures with the
           -i option.

           For most architectures it is possible to supply an
           architecture name and a machine name, separated by a colon.
           For example foo:bar would refer to the bar machine type in
           the foo architecture.  This can be helpful if objdump has
           been configured to support multiple architectures.

           If the target is an ARM architecture then this switch has an
           additional effect.  It restricts the disassembly to only
           those instructions supported by the architecture specified by
           machine.  If it is necessary to use this switch because the
           input file does not contain any architecture information, but
           it is also desired to disassemble all the instructions use
           -marm.

       -M options
       --disassembler-options=options
           Pass target specific information to the disassembler.  Only
           supported on some targets.  If it is necessary to specify
           more than one disassembler option then multiple -M options
           can be used or can be placed together into a comma separated
           list.

           For ARC, dsp controls the printing of DSP instructions, spfp
           selects the printing of FPX single precision FP instructions,
           dpfp selects the printing of FPX double precision FP
           instructions, quarkse_em selects the printing of special
           QuarkSE-EM instructions, fpuda selects the printing of double
           precision assist instructions, fpus selects the printing of
           FPU single precision FP instructions, while fpud selects the
           printing of FPU double precision FP instructions.
           Additionally, one can choose to have all the immediates
           printed in hexadecimal using hex.  By default, the short
           immediates are printed using the decimal representation,
           while the long immediate values are printed as hexadecimal.

           cpu=... allows one to enforce a particular ISA when
           disassembling instructions, overriding the -m value or
           whatever is in the ELF file.  This might be useful to select
           ARC EM or HS ISA, because architecture is same for those and
           disassembler relies on private ELF header data to decide if
           code is for EM or HS.  This option might be specified
           multiple times - only the latest value will be used.  Valid
           values are same as for the assembler -mcpu=... option.

           If the target is an ARM architecture then this switch can be
           used to select which register name set is used during
           disassembler.  Specifying -M reg-names-std (the default) will
           select the register names as used in ARM's instruction set
           documentation, but with register 13 called 'sp', register 14
           called 'lr' and register 15 called 'pc'.  Specifying -M reg-
           names-apcs will select the name set used by the ARM Procedure
           Call Standard, whilst specifying -M reg-names-raw will just
           use r followed by the register number.

           There are also two variants on the APCS register naming
           scheme enabled by -M reg-names-atpcs and -M reg-names-
           special-atpcs which use the ARM/Thumb Procedure Call Standard
           naming conventions.  (Either with the normal register names
           or the special register names).

           This option can also be used for ARM architectures to force
           the disassembler to interpret all instructions as Thumb
           instructions by using the switch
           --disassembler-options=force-thumb.  This can be useful when
           attempting to disassemble thumb code produced by other
           compilers.

           For AArch64 targets this switch can be used to set whether
           instructions are disassembled as the most general instruction
           using the -M no-aliases option or whether instruction notes
           should be generated as comments in the disasssembly using -M
           notes.

           For the x86, some of the options duplicate functions of the
           -m switch, but allow finer grained control.

           "x86-64"
           "i386"
           "i8086"
               Select disassembly for the given architecture.

           "intel"
           "att"
               Select between intel syntax mode and AT&T syntax mode.

           "amd64"
           "intel64"
               Select between AMD64 ISA and Intel64 ISA.

           "intel-mnemonic"
           "att-mnemonic"
               Select between intel mnemonic mode and AT&T mnemonic
               mode.  Note: "intel-mnemonic" implies "intel" and
               "att-mnemonic" implies "att".

           "addr64"
           "addr32"
           "addr16"
           "data32"
           "data16"
               Specify the default address size and operand size.  These
               five options will be overridden if "x86-64", "i386" or
               "i8086" appear later in the option string.

           "suffix"
               When in AT&T mode and also for a limited set of
               instructions when in Intel mode, instructs the
               disassembler to print a mnemonic suffix even when the
               suffix could be inferred by the operands or, for certain
               instructions, the execution mode's defaults.

           For PowerPC, the -M argument raw selects disasssembly of
           hardware insns rather than aliases.  For example, you will
           see "rlwinm" rather than "clrlwi", and "addi" rather than
           "li".  All of the -m arguments for gas that select a CPU are
           supported.  These are: 403, 405, 440, 464, 476, 601, 603,
           604, 620, 7400, 7410, 7450, 7455, 750cl, 821, 850, 860, a2,
           booke, booke32, cell, com, e200z2, e200z4, e300, e500,
           e500mc, e500mc64, e500x2, e5500, e6500, efs, power4, power5,
           power6, power7, power8, power9, power10, ppc, ppc32, ppc64,
           ppc64bridge, ppcps, pwr, pwr2, pwr4, pwr5, pwr5x, pwr6, pwr7,
           pwr8, pwr9, pwr10, pwrx, titan, vle, and future.  32 and 64
           modify the default or a prior CPU selection, disabling and
           enabling 64-bit insns respectively.  In addition, altivec,
           any, lsp, htm, vsx, spe and  spe2 add capabilities to a
           previous or later CPU selection.  any will disassemble any
           opcode known to binutils, but in cases where an opcode has
           two different meanings or different arguments, you may not
           see the disassembly you expect.  If you disassemble without
           giving a CPU selection, a default will be chosen from
           information gleaned by BFD from the object files headers, but
           the result again may not be as you expect.

           For MIPS, this option controls the printing of instruction
           mnemonic names and register names in disassembled
           instructions.  Multiple selections from the following may be
           specified as a comma separated string, and invalid options
           are ignored:

           "no-aliases"
               Print the 'raw' instruction mnemonic instead of some
               pseudo instruction mnemonic.  I.e., print 'daddu' or 'or'
               instead of 'move', 'sll' instead of 'nop', etc.

           "msa"
               Disassemble MSA instructions.

           "virt"
               Disassemble the virtualization ASE instructions.

           "xpa"
               Disassemble the eXtended Physical Address (XPA) ASE
               instructions.

           "gpr-names=ABI"
               Print GPR (general-purpose register) names as appropriate
               for the specified ABI.  By default, GPR names are
               selected according to the ABI of the binary being
               disassembled.

           "fpr-names=ABI"
               Print FPR (floating-point register) names as appropriate
               for the specified ABI.  By default, FPR numbers are
               printed rather than names.

           "cp0-names=ARCH"
               Print CP0 (system control coprocessor; coprocessor 0)
               register names as appropriate for the CPU or architecture
               specified by ARCH.  By default, CP0 register names are
               selected according to the architecture and CPU of the
               binary being disassembled.

           "hwr-names=ARCH"
               Print HWR (hardware register, used by the "rdhwr"
               instruction) names as appropriate for the CPU or
               architecture specified by ARCH.  By default, HWR names
               are selected according to the architecture and CPU of the
               binary being disassembled.

           "reg-names=ABI"
               Print GPR and FPR names as appropriate for the selected
               ABI.

           "reg-names=ARCH"
               Print CPU-specific register names (CP0 register and HWR
               names) as appropriate for the selected CPU or
               architecture.

           For any of the options listed above, ABI or ARCH may be
           specified as numeric to have numbers printed rather than
           names, for the selected types of registers.  You can list the
           available values of ABI and ARCH using the --help option.

           For VAX, you can specify function entry addresses with -M
           entry:0xf00ba.  You can use this multiple times to properly
           disassemble VAX binary files that don't contain symbol tables
           (like ROM dumps).  In these cases, the function entry mask
           would otherwise be decoded as VAX instructions, which would
           probably lead the rest of the function being wrongly
           disassembled.

       -p
       --private-headers
           Print information that is specific to the object file format.
           The exact information printed depends upon the object file
           format.  For some object file formats, no additional
           information is printed.

       -P options
       --private=options
           Print information that is specific to the object file format.
           The argument options is a comma separated list that depends
           on the format (the lists of options is displayed with the
           help).

           For XCOFF, the available options are:

           "header"
           "aout"
           "sections"
           "syms"
           "relocs"
           "lineno,"
           "loader"
           "except"
           "typchk"
           "traceback"
           "toc"
           "ldinfo"

           For PE, the available options are:

           "header"
           "sections"

           Not all object formats support this option.  In particular
           the ELF format does not use it.

       -r
       --reloc
           Print the relocation entries of the file.  If used with -d or
           -D, the relocations are printed interspersed with the
           disassembly.

       -R
       --dynamic-reloc
           Print the dynamic relocation entries of the file.  This is
           only meaningful for dynamic objects, such as certain types of
           shared libraries.  As for -r, if used with -d or -D, the
           relocations are printed interspersed with the disassembly.

       -s
       --full-contents
           Display the full contents of sections, often used in
           combination with -j to request specific sections.  By default
           all non-empty non-bss sections are displayed.  By default any
           compressed section will be displayed in its compressed form.
           In order to see the contents in a decompressed form add the
           -Z option to the command line.

       -S
       --source
           Display source code intermixed with disassembly, if possible.
           Implies -d.

       --show-all-symbols
           When disassembling, show all the symbols that match a given
           address, not just the first one.

       --source-comment[=txt]
           Like the -S option, but all source code lines are displayed
           with a prefix of txt.  Typically txt will be a comment string
           which can be used to distinguish the assembler code from the
           source code.  If txt is not provided then a default string of
           "# " (hash followed by a space), will be used.

       --prefix=prefix
           Specify prefix to add to the absolute paths when used with
           -S.

       --prefix-strip=level
           Indicate how many initial directory names to strip off the
           hardwired absolute paths. It has no effect without
           --prefix=prefix.

       --show-raw-insn
           When disassembling instructions, print the instruction in hex
           as well as in symbolic form.  This is the default except when
           --prefix-addresses is used.

       --no-show-raw-insn
           When disassembling instructions, do not print the instruction
           bytes.  This is the default when --prefix-addresses is used.

       --insn-width=width
           Display width bytes on a single line when disassembling
           instructions.

       --visualize-jumps[=color|=extended-color|=off]
           Visualize jumps that stay inside a function by drawing ASCII
           art between the start and target addresses.  The optional
           =color argument adds color to the output using simple
           terminal colors.  Alternatively the =extended-color argument
           will add color using 8bit colors, but these might not work on
           all terminals.

           If it is necessary to disable the visualize-jumps option
           after it has previously been enabled then use
           visualize-jumps=off.

       --disassembler-color=off
       --disassembler-color=terminal
       --disassembler-color=on|color|colour
       --disassembler-color=extened|extended-color|extened-colour
           Enables or disables the use of colored syntax highlighting in
           disassembly output.  The default behaviour is determined via
           a configure time option.  Note, not all architectures support
           colored syntax highlighting, and depending upon the terminal
           used, colored output may not actually be legible.

           The on argument adds colors using simple terminal colors.

           The terminal argument does the same, but only if the output
           device is a terminal.

           The extended-color argument is similar to the on argument,
           but it uses 8-bit colors.  These may not work on all
           terminals.

           The off argument disables colored disassembly.

       -W[lLiaprmfFsoORtUuTgAckK]
       --dwarf[=rawline,=decodedline,=info,=abbrev,=pubnames,=aranges,=macro,=frames,=frames-interp,=str,=str-offsets,=loc,=Ranges,=pubtypes,=trace_info,=trace_abbrev,=trace_aranges,=gdb_index,=addr,=cu_index,=links,=follow-links]
           Displays the contents of the DWARF debug sections in the
           file, if any are present.  Compressed debug sections are
           automatically decompressed (temporarily) before they are
           displayed.  If one or more of the optional letters or words
           follows the switch then only those type(s) of data will be
           dumped.  The letters and words refer to the following
           information:

           "a"
           "=abbrev"
               Displays the contents of the .debug_abbrev section.

           "A"
           "=addr"
               Displays the contents of the .debug_addr section.

           "c"
           "=cu_index"
               Displays the contents of the .debug_cu_index and/or
               .debug_tu_index sections.

           "f"
           "=frames"
               Display the raw contents of a .debug_frame section.

           "F"
           "=frames-interp"
               Display the interpreted contents of a .debug_frame
               section.

           "g"
           "=gdb_index"
               Displays the contents of the .gdb_index and/or
               .debug_names sections.

           "i"
           "=info"
               Displays the contents of the .debug_info section.  Note:
               the output from this option can also be restricted by the
               use of the --dwarf-depth and --dwarf-start options.

           "k"
           "=links"
               Displays the contents of the .gnu_debuglink,
               .gnu_debugaltlink and .debug_sup sections, if any of them
               are present.  Also displays any links to separate dwarf
               object files (dwo), if they are specified by the
               DW_AT_GNU_dwo_name or DW_AT_dwo_name attributes in the
               .debug_info section.

           "K"
           "=follow-links"
               Display the contents of any selected debug sections that
               are found in linked, separate debug info file(s).  This
               can result in multiple versions of the same debug section
               being displayed if it exists in more than one file.

               In addition, when displaying DWARF attributes, if a form
               is found that references the separate debug info file,
               then the referenced contents will also be displayed.

               Note - in some distributions this option is enabled by
               default.  It can be disabled via the N debug option.  The
               default can be chosen when configuring the binutils via
               the --enable-follow-debug-links=yes or
               --enable-follow-debug-links=no options.  If these are not
               used then the default is to enable the following of debug
               links.

               Note - if support for the debuginfod protocol was enabled
               when the binutils were built then this option will also
               include an attempt to contact any debuginfod servers
               mentioned in the DEBUGINFOD_URLS environment variable.
               This could take some time to resolve.  This behaviour can
               be disabled via the =do-not-use-debuginfod debug option.

           "N"
           "=no-follow-links"
               Disables the following of links to separate debug info
               files.

           "D"
           "=use-debuginfod"
               Enables contacting debuginfod servers if there is a need
               to follow debug links.  This is the default behaviour.

           "E"
           "=do-not-use-debuginfod"
               Disables contacting debuginfod servers when there is a
               need to follow debug links.

           "l"
           "=rawline"
               Displays the contents of the .debug_line section in a raw
               format.

           "L"
           "=decodedline"
               Displays the interpreted contents of the .debug_line
               section.

           "m"
           "=macro"
               Displays the contents of the .debug_macro and/or
               .debug_macinfo sections.

           "o"
           "=loc"
               Displays the contents of the .debug_loc and/or
               .debug_loclists sections.

           "O"
           "=str-offsets"
               Displays the contents of the .debug_str_offsets section.

           "p"
           "=pubnames"
               Displays the contents of the .debug_pubnames and/or
               .debug_gnu_pubnames sections.

           "r"
           "=aranges"
               Displays the contents of the .debug_aranges section.

           "R"
           "=Ranges"
               Displays the contents of the .debug_ranges and/or
               .debug_rnglists sections.

           "s"
           "=str"
               Displays the contents of the .debug_str, .debug_line_str
               and/or .debug_str_offsets sections.

           "t"
           "=pubtype"
               Displays the contents of the .debug_pubtypes and/or
               .debug_gnu_pubtypes sections.

           "T"
           "=trace_aranges"
               Displays the contents of the .trace_aranges section.

           "u"
           "=trace_abbrev"
               Displays the contents of the .trace_abbrev section.

           "U"
           "=trace_info"
               Displays the contents of the .trace_info section.

           Note: displaying the contents of .debug_static_funcs,
           .debug_static_vars and debug_weaknames sections is not
           currently supported.

       --dwarf-depth=n
           Limit the dump of the ".debug_info" section to n children.
           This is only useful with --debug-dump=info.  The default is
           to print all DIEs; the special value 0 for n will also have
           this effect.

           With a non-zero value for n, DIEs at or deeper than n levels
           will not be printed.  The range for n is zero-based.

       --dwarf-start=n
           Print only DIEs beginning with the DIE numbered n.  This is
           only useful with --debug-dump=info.

           If specified, this option will suppress printing of any
           header information and all DIEs before the DIE numbered n.
           Only siblings and children of the specified DIE will be
           printed.

           This can be used in conjunction with --dwarf-depth.

       --dwarf-check
           Enable additional checks for consistency of Dwarf
           information.

       --ctf[=section]
           Display the contents of the specified CTF section.  CTF
           sections themselves contain many subsections, all of which
           are displayed in order.

           By default, display the name of the section named .ctf, which
           is the name emitted by ld.

       --ctf-parent=member
           If the CTF section contains ambiguously-defined types, it
           will consist of an archive of many CTF dictionaries, all
           inheriting from one dictionary containing unambiguous types.
           This member is by default named .ctf, like the section
           containing it, but it is possible to change this name using
           the "ctf_link_set_memb_name_changer" function at link time.
           When looking at CTF archives that have been created by a
           linker that uses the name changer to rename the parent
           archive member, --ctf-parent can be used to specify the name
           used for the parent.

       --sframe[=section]
           Display the contents of the specified SFrame section.

           By default, display the name of the section named .sframe,
           which is the name emitted by ld.

       -G
       --stabs
           Display the full contents of any sections requested.  Display
           the contents of the .stab and .stab.index and .stab.excl
           sections from an ELF file.  This is only useful on systems
           (such as Solaris 2.0) in which ".stab" debugging symbol-table
           entries are carried in an ELF section.  In most other file
           formats, debugging symbol-table entries are interleaved with
           linkage symbols, and are visible in the --syms output.

       --start-address=address
           Start displaying data at the specified address.  This affects
           the output of the -d, -r and -s options.

       --stop-address=address
           Stop displaying data at the specified address.  This affects
           the output of the -d, -r and -s options.

       -t
       --syms
           Print the symbol table entries of the file.  This is similar
           to the information provided by the nm program, although the
           display format is different.  The format of the output
           depends upon the format of the file being dumped, but there
           are two main types.  One looks like this:

                   [  4](sec  3)(fl 0x00)(ty   0)(scl   3) (nx 1) 0x00000000 .bss
                   [  6](sec  1)(fl 0x00)(ty   0)(scl   2) (nx 0) 0x00000000 fred

           where the number inside the square brackets is the number of
           the entry in the symbol table, the sec number is the section
           number, the fl value are the symbol's flag bits, the ty
           number is the symbol's type, the scl number is the symbol's
           storage class and the nx value is the number of auxiliary
           entries associated with the symbol.  The last two fields are
           the symbol's value and its name.

           The other common output format, usually seen with ELF based
           files, looks like this:

                   00000000 l    d  .bss   00000000 .bss
                   00000000 g       .text  00000000 fred

           Here the first number is the symbol's value (sometimes
           referred to as its address).  The next field is actually a
           set of characters and spaces indicating the flag bits that
           are set on the symbol.  These characters are described below.
           Next is the section with which the symbol is associated or
           *ABS* if the section is absolute (ie not connected with any
           section), or *UND* if the section is referenced in the file
           being dumped, but not defined there.

           After the section name comes another field, a number, which
           for common symbols is the alignment and for other symbol is
           the size.  Finally the symbol's name is displayed.

           The flag characters are divided into 7 groups as follows:

           "l"
           "g"
           "u"
           "!" The symbol is a local (l), global (g), unique global (u),
               neither global nor local (a space) or both global and
               local (!).  A symbol can be neither local or global for a
               variety of reasons, e.g., because it is used for
               debugging, but it is probably an indication of a bug if
               it is ever both local and global.  Unique global symbols
               are a GNU extension to the standard set of ELF symbol
               bindings.  For such a symbol the dynamic linker will make
               sure that in the entire process there is just one symbol
               with this name and type in use.

           "w" The symbol is weak (w) or strong (a space).

           "C" The symbol denotes a constructor (C) or an ordinary
               symbol (a space).

           "W" The symbol is a warning (W) or a normal symbol (a space).
               A warning symbol's name is a message to be displayed if
               the symbol following the warning symbol is ever
               referenced.

           "I"
           "i" The symbol is an indirect reference to another symbol
               (I), a function to be evaluated during reloc processing
               (i) or a normal symbol (a space).

           "d"
           "D" The symbol is a debugging symbol (d) or a dynamic symbol
               (D) or a normal symbol (a space).

           "F"
           "f"
           "O" The symbol is the name of a function (F) or a file (f) or
               an object (O) or just a normal symbol (a space).

       -T
       --dynamic-syms
           Print the dynamic symbol table entries of the file.  This is
           only meaningful for dynamic objects, such as certain types of
           shared libraries.  This is similar to the information
           provided by the nm program when given the -D (--dynamic)
           option.

           The output format is similar to that produced by the --syms
           option, except that an extra field is inserted before the
           symbol's name, giving the version information associated with
           the symbol.  If the version is the default version to be used
           when resolving unversioned references to the symbol then it's
           displayed as is, otherwise it's put into parentheses.

       --special-syms
           When displaying symbols include those which the target
           considers to be special in some way and which would not
           normally be of interest to the user.

       -U [d|i|l|e|x|h]
       --unicode=[default|invalid|locale|escape|hex|highlight]
           Controls the display of UTF-8 encoded multibyte characters in
           strings.  The default (--unicode=default) is to give them no
           special treatment.  The --unicode=locale option displays the
           sequence in the current locale, which may or may not support
           them.  The options --unicode=hex and --unicode=invalid
           display them as hex byte sequences enclosed by either angle
           brackets or curly braces.

           The --unicode=escape option displays them as escape sequences
           (\uxxxx) and the --unicode=highlight option displays them as
           escape sequences highlighted in red (if supported by the
           output device).  The colouring is intended to draw attention
           to the presence of unicode sequences where they might not be
           expected.

       -V
       --version
           Print the version number of objdump and exit.

       -x
       --all-headers
           Display all available header information, including the
           symbol table and relocation entries.  Using -x is equivalent
           to specifying all of -a -f -h -p -r -t.

       -w
       --wide
           Format some lines for output devices that have more than 80
           columns.  Also do not truncate symbol names when they are
           displayed.

       -z
       --disassemble-zeroes
           Normally the disassembly output will skip blocks of zeroes.
           This option directs the disassembler to disassemble those
           blocks, just like any other data.

       -Z
       --decompress
           The -Z option is meant to be used in conunction with the -s
           option.  It instructs objdump to decompress any compressed
           sections before displaying their contents.

       @file
           Read command-line options from file.  The options read are
           inserted in place of the original @file option.  If file does
           not exist, or cannot be read, then the option will be treated
           literally, and not removed.

           Options in file are separated by whitespace.  A whitespace
           character may be included in an option by surrounding the
           entire option in either single or double quotes.  Any
           character (including a backslash) may be included by
           prefixing the character to be included with a backslash.  The
           file may itself contain additional @file options; any such
           options will be processed recursively.
SEE ALSO
       nm(1), readelf(1), and the Info entries for binutils.
COPYRIGHT
       Copyright (c) 1991-2024 Free Software Foundation, Inc.

       Permission is granted to copy, distribute and/or modify this
       document under the terms of the GNU Free Documentation License,
       Version 1.3 or any later version published by the Free Software
       Foundation; with no Invariant Sections, with no Front-Cover
       Texts, and with no Back-Cover Texts.  A copy of the license is
       included in the section entitled "GNU Free Documentation License".
COLOPHON
       This page is part of the binutils (a collection of tools for
       working with executable binaries) project.  Information about the
       project can be found at http://www.gnu.org/software/binutils/.
       If you have a bug report for this manual page, see
       http://sourceware.org/bugzilla/enter_bug.cgi?product=binutils.
       This page was obtained from the tarball binutils-2.42.tar.gz
       fetched from https://ftp.gnu.org/gnu/binutils/ on 2024-06-14.
       If you discover any rendering problems in this HTML version of
       the page, or you believe there is a better or more up-to-date
       source for the page, or you have corrections or improvements to
       the information in this COLOPHON (which is not part of the
       original manual page), send a mail to man-pages@man7.org

binutils-2.42                  2024-06-14                     OBJDUMP(1)
\end{lstlisting}
}}
\endinput  %  ==  ==  ==  ==  ==  ==  ==  ==  ==
