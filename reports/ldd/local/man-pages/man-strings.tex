% % % % \input{./components/man/man-strings}
\subsection{\refStrings: Print Sequences Of Printable Characters}

{\tiny{
\begin{lstlisting}[language=bash]
NAME
       strings - print the sequences of printable characters in files
SYNOPSIS
       strings [-afovV] [-min-len]
               [-n min-len] [--bytes=min-len]
               [-t radix] [--radix=radix]
               [-e encoding] [--encoding=encoding]
               [-U method] [--unicode=method]
               [-] [--all] [--print-file-name]
               [-T bfdname] [--target=bfdname]
               [-w] [--include-all-whitespace]
               [-s] [--output-separator sep_string]
               [--help] [--version] file...
DESCRIPTION
       For each file given, GNU strings prints the printable character
       sequences that are at least 4 characters long (or the number
       given with the options below) and are followed by an unprintable
       character.

       Depending upon how the strings program was configured it will
       default to either displaying all the printable sequences that it
       can find in each file, or only those sequences that are in
       loadable, initialized data sections.  If the file type is
       unrecognizable, or if strings is reading from stdin then it will
       always display all of the printable sequences that it can find.

       For backwards compatibility any file that occurs after a command-
       line option of just - will also be scanned in full, regardless of
       the presence of any -d option.

       strings is mainly useful for determining the contents of non-text
       files.
OPTIONS
       -a
       --all
       -   Scan the whole file, regardless of what sections it contains
           or whether those sections are loaded or initialized.
           Normally this is the default behaviour, but strings can be
           configured so that the -d is the default instead.

           The - option is position dependent and forces strings to
           perform full scans of any file that is mentioned after the -
           on the command line, even if the -d option has been
           specified.

       -d
       --data
           Only print strings from initialized, loaded data sections in
           the file.  This may reduce the amount of garbage in the
           output, but it also exposes the strings program to any
           security flaws that may be present in the BFD library used to
           scan and load sections.  Strings can be configured so that
           this option is the default behaviour.  In such cases the -a
           option can be used to avoid using the BFD library and instead
           just print all of the strings found in the file.

       -f
       --print-file-name
           Print the name of the file before each string.

       --help
           Print a summary of the program usage on the standard output
           and exit.

       -min-len
       -n min-len
       --bytes=min-len
           Print sequences of displayable characters that are at least
           min-len characters long.  If not specified a default minimum
           length of 4 is used.  The distinction between displayable and
           non-displayable characters depends upon the setting of the -e
           and -U options.  Sequences are always terminated at control
           characters such as new-line and carriage-return, but not the
           tab character.

       -o  Like -t o.  Some other versions of strings have -o act like
           -t d instead.  Since we can not be compatible with both ways,
           we simply chose one.

       -t radix
       --radix=radix
           Print the offset within the file before each string.  The
           single character argument specifies the radix of the
           offset---o for octal, x for hexadecimal, or d for decimal.

       -e encoding
       --encoding=encoding
           Select the character encoding of the strings that are to be
           found.  Possible values for encoding are: s =
           single-7-bit-byte characters (default), S = single-8-bit-byte
           characters, b = 16-bit bigendian, l = 16-bit littleendian, B
           = 32-bit bigendian, L = 32-bit littleendian.  Useful for
           finding wide character strings. (l and b apply to, for
           example, Unicode UTF-16/UCS-2 encodings).

       -U [d|i|l|e|x|h]
       --unicode=[default|invalid|locale|escape|hex|highlight]
           Controls the display of UTF-8 encoded multibyte characters in
           strings.  The default (--unicode=default) is to give them no
           special treatment, and instead rely upon the setting of the
           --encoding option.  The other values for this option
           automatically enable --encoding=S.

           The --unicode=invalid option treats them as non-graphic
           characters and hence not part of a valid string.  All the
           remaining options treat them as valid string characters.

           The --unicode=locale option displays them in the current
           locale, which may or may not support UTF-8 encoding.  The
           --unicode=hex option displays them as hex byte sequences
           enclosed between <> characters.  The --unicode=escape option
           displays them as escape sequences (\uxxxx) and the
           --unicode=highlight option displays them as escape sequences
           highlighted in red (if supported by the output device).  The
           colouring is intended to draw attention to the presence of
           unicode sequences where they might not be expected.

       -T bfdname
       --target=bfdname
           Specify an object code format other than your system's
           default format.

       -v
       -V
       --version
           Print the program version number on the standard output and
           exit.

       -w
       --include-all-whitespace
           By default tab and space characters are included in the
           strings that are displayed, but other whitespace characters,
           such a newlines and carriage returns, are not.  The -w option
           changes this so that all whitespace characters are considered
           to be part of a string.

       -s
       --output-separator
           By default, output strings are delimited by a new-line. This
           option allows you to supply any string to be used as the
           output record separator.  Useful with
           --include-all-whitespace where strings may contain new-lines
           internally.

       @file
           Read command-line options from file.  The options read are
           inserted in place of the original @file option.  If file does
           not exist, or cannot be read, then the option will be treated
           literally, and not removed.

           Options in file are separated by whitespace.  A whitespace
           character may be included in an option by surrounding the
           entire option in either single or double quotes.  Any
           character (including a backslash) may be included by
           prefixing the character to be included with a backslash.  The
           file may itself contain additional @file options; any such
           options will be processed recursively.
SEE ALSO
       ar(1), nm(1), objdump(1), ranlib(1), readelf(1) and the Info
       entries for binutils.
COPYRIGHT
       Copyright (c) 1991-2024 Free Software Foundation, Inc.

       Permission is granted to copy, distribute and/or modify this
       document under the terms of the GNU Free Documentation License,
       Version 1.3 or any later version published by the Free Software
       Foundation; with no Invariant Sections, with no Front-Cover
       Texts, and with no Back-Cover Texts.  A copy of the license is
       included in the section entitled "GNU Free Documentation
       License".
COLOPHON
       This page is part of the binutils (a collection of tools for
       working with executable binaries) project.  Information about the
       project can be found at http://www.gnu.org/software/binutils/.
       If you have a bug report for this manual page, see
       http://sourceware.org/bugzilla/enter_bug.cgi?product=binutils.
       This page was obtained from the tarball binutils-2.42.tar.gz
       fetched from https://ftp.gnu.org/gnu/binutils/ on 2024-06-14.
       If you discover any rendering problems in this HTML version of
       the page, or you believe there is a better or more up-to-date
       source for the page, or you have corrections or improvements to
       the information in this COLOPHON (which is not part of the
       original manual page), send a mail to man-pages@man7.org

binutils-2.42                  2024-06-14                     STRINGS(1)
\end{lstlisting}
}}
\endinput  %  ==  ==  ==  ==  ==  ==  ==  ==  ==

\subsection{\refStrings: Print Sequences Of Printable Characters}

{\tiny{
\begin{lstlisting}[language=bash]
NAME
       strings - print the sequences of printable characters in files
SYNOPSIS
       strings [-afovV] [-min-len]
               [-n min-len] [--bytes=min-len]
               [-t radix] [--radix=radix]
               [-e encoding] [--encoding=encoding]
               [-U method] [--unicode=method]
               [-] [--all] [--print-file-name]
               [-T bfdname] [--target=bfdname]
               [-w] [--include-all-whitespace]
               [-s] [--output-separator sep_string]
               [--help] [--version] file...
DESCRIPTION
       For each file given, GNU strings prints the printable character
       sequences that are at least 4 characters long (or the number
       given with the options below) and are followed by an unprintable
       character.

       Depending upon how the strings program was configured it will
       default to either displaying all the printable sequences that it
       can find in each file, or only those sequences that are in
       loadable, initialized data sections.  If the file type is
       unrecognizable, or if strings is reading from stdin then it will
       always display all of the printable sequences that it can find.

       For backwards compatibility any file that occurs after a command-
       line option of just - will also be scanned in full, regardless of
       the presence of any -d option.

       strings is mainly useful for determining the contents of non-text
       files.
OPTIONS
       -a
       --all
       -   Scan the whole file, regardless of what sections it contains
           or whether those sections are loaded or initialized.
           Normally this is the default behaviour, but strings can be
           configured so that the -d is the default instead.

           The - option is position dependent and forces strings to
           perform full scans of any file that is mentioned after the -
           on the command line, even if the -d option has been
           specified.

       -d
       --data
           Only print strings from initialized, loaded data sections in
           the file.  This may reduce the amount of garbage in the
           output, but it also exposes the strings program to any
           security flaws that may be present in the BFD library used to
           scan and load sections.  Strings can be configured so that
           this option is the default behaviour.  In such cases the -a
           option can be used to avoid using the BFD library and instead
           just print all of the strings found in the file.

       -f
       --print-file-name
           Print the name of the file before each string.

       --help
           Print a summary of the program usage on the standard output
           and exit.

       -min-len
       -n min-len
       --bytes=min-len
           Print sequences of displayable characters that are at least
           min-len characters long.  If not specified a default minimum
           length of 4 is used.  The distinction between displayable and
           non-displayable characters depends upon the setting of the -e
           and -U options.  Sequences are always terminated at control
           characters such as new-line and carriage-return, but not the
           tab character.

       -o  Like -t o.  Some other versions of strings have -o act like
           -t d instead.  Since we can not be compatible with both ways,
           we simply chose one.

       -t radix
       --radix=radix
           Print the offset within the file before each string.  The
           single character argument specifies the radix of the
           offset---o for octal, x for hexadecimal, or d for decimal.

       -e encoding
       --encoding=encoding
           Select the character encoding of the strings that are to be
           found.  Possible values for encoding are: s =
           single-7-bit-byte characters (default), S = single-8-bit-byte
           characters, b = 16-bit bigendian, l = 16-bit littleendian, B
           = 32-bit bigendian, L = 32-bit littleendian.  Useful for
           finding wide character strings. (l and b apply to, for
           example, Unicode UTF-16/UCS-2 encodings).

       -U [d|i|l|e|x|h]
       --unicode=[default|invalid|locale|escape|hex|highlight]
           Controls the display of UTF-8 encoded multibyte characters in
           strings.  The default (--unicode=default) is to give them no
           special treatment, and instead rely upon the setting of the
           --encoding option.  The other values for this option
           automatically enable --encoding=S.

           The --unicode=invalid option treats them as non-graphic
           characters and hence not part of a valid string.  All the
           remaining options treat them as valid string characters.

           The --unicode=locale option displays them in the current
           locale, which may or may not support UTF-8 encoding.  The
           --unicode=hex option displays them as hex byte sequences
           enclosed between <> characters.  The --unicode=escape option
           displays them as escape sequences (\uxxxx) and the
           --unicode=highlight option displays them as escape sequences
           highlighted in red (if supported by the output device).  The
           colouring is intended to draw attention to the presence of
           unicode sequences where they might not be expected.

       -T bfdname
       --target=bfdname
           Specify an object code format other than your system's
           default format.

       -v
       -V
       --version
           Print the program version number on the standard output and
           exit.

       -w
       --include-all-whitespace
           By default tab and space characters are included in the
           strings that are displayed, but other whitespace characters,
           such a newlines and carriage returns, are not.  The -w option
           changes this so that all whitespace characters are considered
           to be part of a string.

       -s
       --output-separator
           By default, output strings are delimited by a new-line. This
           option allows you to supply any string to be used as the
           output record separator.  Useful with
           --include-all-whitespace where strings may contain new-lines
           internally.

       @file
           Read command-line options from file.  The options read are
           inserted in place of the original @file option.  If file does
           not exist, or cannot be read, then the option will be treated
           literally, and not removed.

           Options in file are separated by whitespace.  A whitespace
           character may be included in an option by surrounding the
           entire option in either single or double quotes.  Any
           character (including a backslash) may be included by
           prefixing the character to be included with a backslash.  The
           file may itself contain additional @file options; any such
           options will be processed recursively.
SEE ALSO
       ar(1), nm(1), objdump(1), ranlib(1), readelf(1) and the Info
       entries for binutils.
COPYRIGHT
       Copyright (c) 1991-2024 Free Software Foundation, Inc.

       Permission is granted to copy, distribute and/or modify this
       document under the terms of the GNU Free Documentation License,
       Version 1.3 or any later version published by the Free Software
       Foundation; with no Invariant Sections, with no Front-Cover
       Texts, and with no Back-Cover Texts.  A copy of the license is
       included in the section entitled "GNU Free Documentation
       License".
COLOPHON
       This page is part of the binutils (a collection of tools for
       working with executable binaries) project.  Information about the
       project can be found at http://www.gnu.org/software/binutils/.
       If you have a bug report for this manual page, see
       http://sourceware.org/bugzilla/enter_bug.cgi?product=binutils.
       This page was obtained from the tarball binutils-2.42.tar.gz
       fetched from https://ftp.gnu.org/gnu/binutils/ on 2024-06-14.
       If you discover any rendering problems in this HTML version of
       the page, or you believe there is a better or more up-to-date
       source for the page, or you have corrections or improvements to
       the information in this COLOPHON (which is not part of the
       original manual page), send a mail to man-pages@man7.org

binutils-2.42                  2024-06-14                     STRINGS(1)
\end{lstlisting}
}}
\endinput  %  ==  ==  ==  ==  ==  ==  ==  ==  ==

\subsection{\refStrings: Print Sequences Of Printable Characters}

{\tiny{
\begin{lstlisting}[language=bash]
NAME
       strings - print the sequences of printable characters in files
SYNOPSIS
       strings [-afovV] [-min-len]
               [-n min-len] [--bytes=min-len]
               [-t radix] [--radix=radix]
               [-e encoding] [--encoding=encoding]
               [-U method] [--unicode=method]
               [-] [--all] [--print-file-name]
               [-T bfdname] [--target=bfdname]
               [-w] [--include-all-whitespace]
               [-s] [--output-separator sep_string]
               [--help] [--version] file...
DESCRIPTION
       For each file given, GNU strings prints the printable character
       sequences that are at least 4 characters long (or the number
       given with the options below) and are followed by an unprintable
       character.

       Depending upon how the strings program was configured it will
       default to either displaying all the printable sequences that it
       can find in each file, or only those sequences that are in
       loadable, initialized data sections.  If the file type is
       unrecognizable, or if strings is reading from stdin then it will
       always display all of the printable sequences that it can find.

       For backwards compatibility any file that occurs after a command-
       line option of just - will also be scanned in full, regardless of
       the presence of any -d option.

       strings is mainly useful for determining the contents of non-text
       files.
OPTIONS
       -a
       --all
       -   Scan the whole file, regardless of what sections it contains
           or whether those sections are loaded or initialized.
           Normally this is the default behaviour, but strings can be
           configured so that the -d is the default instead.

           The - option is position dependent and forces strings to
           perform full scans of any file that is mentioned after the -
           on the command line, even if the -d option has been
           specified.

       -d
       --data
           Only print strings from initialized, loaded data sections in
           the file.  This may reduce the amount of garbage in the
           output, but it also exposes the strings program to any
           security flaws that may be present in the BFD library used to
           scan and load sections.  Strings can be configured so that
           this option is the default behaviour.  In such cases the -a
           option can be used to avoid using the BFD library and instead
           just print all of the strings found in the file.

       -f
       --print-file-name
           Print the name of the file before each string.

       --help
           Print a summary of the program usage on the standard output
           and exit.

       -min-len
       -n min-len
       --bytes=min-len
           Print sequences of displayable characters that are at least
           min-len characters long.  If not specified a default minimum
           length of 4 is used.  The distinction between displayable and
           non-displayable characters depends upon the setting of the -e
           and -U options.  Sequences are always terminated at control
           characters such as new-line and carriage-return, but not the
           tab character.

       -o  Like -t o.  Some other versions of strings have -o act like
           -t d instead.  Since we can not be compatible with both ways,
           we simply chose one.

       -t radix
       --radix=radix
           Print the offset within the file before each string.  The
           single character argument specifies the radix of the
           offset---o for octal, x for hexadecimal, or d for decimal.

       -e encoding
       --encoding=encoding
           Select the character encoding of the strings that are to be
           found.  Possible values for encoding are: s =
           single-7-bit-byte characters (default), S = single-8-bit-byte
           characters, b = 16-bit bigendian, l = 16-bit littleendian, B
           = 32-bit bigendian, L = 32-bit littleendian.  Useful for
           finding wide character strings. (l and b apply to, for
           example, Unicode UTF-16/UCS-2 encodings).

       -U [d|i|l|e|x|h]
       --unicode=[default|invalid|locale|escape|hex|highlight]
           Controls the display of UTF-8 encoded multibyte characters in
           strings.  The default (--unicode=default) is to give them no
           special treatment, and instead rely upon the setting of the
           --encoding option.  The other values for this option
           automatically enable --encoding=S.

           The --unicode=invalid option treats them as non-graphic
           characters and hence not part of a valid string.  All the
           remaining options treat them as valid string characters.

           The --unicode=locale option displays them in the current
           locale, which may or may not support UTF-8 encoding.  The
           --unicode=hex option displays them as hex byte sequences
           enclosed between <> characters.  The --unicode=escape option
           displays them as escape sequences (\uxxxx) and the
           --unicode=highlight option displays them as escape sequences
           highlighted in red (if supported by the output device).  The
           colouring is intended to draw attention to the presence of
           unicode sequences where they might not be expected.

       -T bfdname
       --target=bfdname
           Specify an object code format other than your system's
           default format.

       -v
       -V
       --version
           Print the program version number on the standard output and
           exit.

       -w
       --include-all-whitespace
           By default tab and space characters are included in the
           strings that are displayed, but other whitespace characters,
           such a newlines and carriage returns, are not.  The -w option
           changes this so that all whitespace characters are considered
           to be part of a string.

       -s
       --output-separator
           By default, output strings are delimited by a new-line. This
           option allows you to supply any string to be used as the
           output record separator.  Useful with
           --include-all-whitespace where strings may contain new-lines
           internally.

       @file
           Read command-line options from file.  The options read are
           inserted in place of the original @file option.  If file does
           not exist, or cannot be read, then the option will be treated
           literally, and not removed.

           Options in file are separated by whitespace.  A whitespace
           character may be included in an option by surrounding the
           entire option in either single or double quotes.  Any
           character (including a backslash) may be included by
           prefixing the character to be included with a backslash.  The
           file may itself contain additional @file options; any such
           options will be processed recursively.
SEE ALSO
       ar(1), nm(1), objdump(1), ranlib(1), readelf(1) and the Info
       entries for binutils.
COPYRIGHT
       Copyright (c) 1991-2024 Free Software Foundation, Inc.

       Permission is granted to copy, distribute and/or modify this
       document under the terms of the GNU Free Documentation License,
       Version 1.3 or any later version published by the Free Software
       Foundation; with no Invariant Sections, with no Front-Cover
       Texts, and with no Back-Cover Texts.  A copy of the license is
       included in the section entitled "GNU Free Documentation
       License".
COLOPHON
       This page is part of the binutils (a collection of tools for
       working with executable binaries) project.  Information about the
       project can be found at http://www.gnu.org/software/binutils/.
       If you have a bug report for this manual page, see
       http://sourceware.org/bugzilla/enter_bug.cgi?product=binutils.
       This page was obtained from the tarball binutils-2.42.tar.gz
       fetched from https://ftp.gnu.org/gnu/binutils/ on 2024-06-14.
       If you discover any rendering problems in this HTML version of
       the page, or you believe there is a better or more up-to-date
       source for the page, or you have corrections or improvements to
       the information in this COLOPHON (which is not part of the
       original manual page), send a mail to man-pages@man7.org

binutils-2.42                  2024-06-14                     STRINGS(1)
\end{lstlisting}
}}
\endinput  %  ==  ==  ==  ==  ==  ==  ==  ==  ==

\subsection{\refStrings: Print Sequences Of Printable Characters}

{\tiny{
\begin{lstlisting}[language=bash]
NAME
       strings - print the sequences of printable characters in files
SYNOPSIS
       strings [-afovV] [-min-len]
               [-n min-len] [--bytes=min-len]
               [-t radix] [--radix=radix]
               [-e encoding] [--encoding=encoding]
               [-U method] [--unicode=method]
               [-] [--all] [--print-file-name]
               [-T bfdname] [--target=bfdname]
               [-w] [--include-all-whitespace]
               [-s] [--output-separator sep_string]
               [--help] [--version] file...
DESCRIPTION
       For each file given, GNU strings prints the printable character
       sequences that are at least 4 characters long (or the number
       given with the options below) and are followed by an unprintable
       character.

       Depending upon how the strings program was configured it will
       default to either displaying all the printable sequences that it
       can find in each file, or only those sequences that are in
       loadable, initialized data sections.  If the file type is
       unrecognizable, or if strings is reading from stdin then it will
       always display all of the printable sequences that it can find.

       For backwards compatibility any file that occurs after a command-
       line option of just - will also be scanned in full, regardless of
       the presence of any -d option.

       strings is mainly useful for determining the contents of non-text
       files.
OPTIONS
       -a
       --all
       -   Scan the whole file, regardless of what sections it contains
           or whether those sections are loaded or initialized.
           Normally this is the default behaviour, but strings can be
           configured so that the -d is the default instead.

           The - option is position dependent and forces strings to
           perform full scans of any file that is mentioned after the -
           on the command line, even if the -d option has been
           specified.

       -d
       --data
           Only print strings from initialized, loaded data sections in
           the file.  This may reduce the amount of garbage in the
           output, but it also exposes the strings program to any
           security flaws that may be present in the BFD library used to
           scan and load sections.  Strings can be configured so that
           this option is the default behaviour.  In such cases the -a
           option can be used to avoid using the BFD library and instead
           just print all of the strings found in the file.

       -f
       --print-file-name
           Print the name of the file before each string.

       --help
           Print a summary of the program usage on the standard output
           and exit.

       -min-len
       -n min-len
       --bytes=min-len
           Print sequences of displayable characters that are at least
           min-len characters long.  If not specified a default minimum
           length of 4 is used.  The distinction between displayable and
           non-displayable characters depends upon the setting of the -e
           and -U options.  Sequences are always terminated at control
           characters such as new-line and carriage-return, but not the
           tab character.

       -o  Like -t o.  Some other versions of strings have -o act like
           -t d instead.  Since we can not be compatible with both ways,
           we simply chose one.

       -t radix
       --radix=radix
           Print the offset within the file before each string.  The
           single character argument specifies the radix of the
           offset---o for octal, x for hexadecimal, or d for decimal.

       -e encoding
       --encoding=encoding
           Select the character encoding of the strings that are to be
           found.  Possible values for encoding are: s =
           single-7-bit-byte characters (default), S = single-8-bit-byte
           characters, b = 16-bit bigendian, l = 16-bit littleendian, B
           = 32-bit bigendian, L = 32-bit littleendian.  Useful for
           finding wide character strings. (l and b apply to, for
           example, Unicode UTF-16/UCS-2 encodings).

       -U [d|i|l|e|x|h]
       --unicode=[default|invalid|locale|escape|hex|highlight]
           Controls the display of UTF-8 encoded multibyte characters in
           strings.  The default (--unicode=default) is to give them no
           special treatment, and instead rely upon the setting of the
           --encoding option.  The other values for this option
           automatically enable --encoding=S.

           The --unicode=invalid option treats them as non-graphic
           characters and hence not part of a valid string.  All the
           remaining options treat them as valid string characters.

           The --unicode=locale option displays them in the current
           locale, which may or may not support UTF-8 encoding.  The
           --unicode=hex option displays them as hex byte sequences
           enclosed between <> characters.  The --unicode=escape option
           displays them as escape sequences (\uxxxx) and the
           --unicode=highlight option displays them as escape sequences
           highlighted in red (if supported by the output device).  The
           colouring is intended to draw attention to the presence of
           unicode sequences where they might not be expected.

       -T bfdname
       --target=bfdname
           Specify an object code format other than your system's
           default format.

       -v
       -V
       --version
           Print the program version number on the standard output and
           exit.

       -w
       --include-all-whitespace
           By default tab and space characters are included in the
           strings that are displayed, but other whitespace characters,
           such a newlines and carriage returns, are not.  The -w option
           changes this so that all whitespace characters are considered
           to be part of a string.

       -s
       --output-separator
           By default, output strings are delimited by a new-line. This
           option allows you to supply any string to be used as the
           output record separator.  Useful with
           --include-all-whitespace where strings may contain new-lines
           internally.

       @file
           Read command-line options from file.  The options read are
           inserted in place of the original @file option.  If file does
           not exist, or cannot be read, then the option will be treated
           literally, and not removed.

           Options in file are separated by whitespace.  A whitespace
           character may be included in an option by surrounding the
           entire option in either single or double quotes.  Any
           character (including a backslash) may be included by
           prefixing the character to be included with a backslash.  The
           file may itself contain additional @file options; any such
           options will be processed recursively.
SEE ALSO
       ar(1), nm(1), objdump(1), ranlib(1), readelf(1) and the Info
       entries for binutils.
COPYRIGHT
       Copyright (c) 1991-2024 Free Software Foundation, Inc.

       Permission is granted to copy, distribute and/or modify this
       document under the terms of the GNU Free Documentation License,
       Version 1.3 or any later version published by the Free Software
       Foundation; with no Invariant Sections, with no Front-Cover
       Texts, and with no Back-Cover Texts.  A copy of the license is
       included in the section entitled "GNU Free Documentation
       License".
COLOPHON
       This page is part of the binutils (a collection of tools for
       working with executable binaries) project.  Information about the
       project can be found at http://www.gnu.org/software/binutils/.
       If you have a bug report for this manual page, see
       http://sourceware.org/bugzilla/enter_bug.cgi?product=binutils.
       This page was obtained from the tarball binutils-2.42.tar.gz
       fetched from https://ftp.gnu.org/gnu/binutils/ on 2024-06-14.
       If you discover any rendering problems in this HTML version of
       the page, or you believe there is a better or more up-to-date
       source for the page, or you have corrections or improvements to
       the information in this COLOPHON (which is not part of the
       original manual page), send a mail to man-pages@man7.org

binutils-2.42                  2024-06-14                     STRINGS(1)
\end{lstlisting}
}}
\endinput  %  ==  ==  ==  ==  ==  ==  ==  ==  ==
