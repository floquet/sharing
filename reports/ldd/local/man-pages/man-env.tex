% % % % \input{./components/man/man-env}
\subsection{\refEnv: Show \textt{ENVIRONMENT} variables.}

{\tiny{
\begin{lstlisting}[language=bash]
NAME
       env - run a program in a modified environment
SYNOPSIS
       env [OPTION]... [-] [NAME=VALUE]... [COMMAND [ARG]...]
DESCRIPTION
       Set each NAME to VALUE in the environment and run COMMAND.

       Mandatory arguments to long options are mandatory for short
       options too.

       -a, --argv0=ARG
              pass ARG as the zeroth argument of COMMAND

       -i, --ignore-environment
              start with an empty environment

       -0, --null
              end each output line with NUL, not newline

       -u, --unset=NAME
              remove variable from the environment

       -C, --chdir=DIR
              change working directory to DIR

       -S, --split-string=S
              process and split S into separate arguments; used to pass
              multiple arguments on shebang lines

       --block-signal[=SIG]
              block delivery of SIG signal(s) to COMMAND

       --default-signal[=SIG]
              reset handling of SIG signal(s) to the default

       --ignore-signal[=SIG]
              set handling of SIG signal(s) to do nothing

       --list-signal-handling
              list non default signal handling to stderr

       -v, --debug
              print verbose information for each processing step

       --help display this help and exit

       --version
              output version information and exit

       A mere - implies -i.  If no COMMAND, print the resulting
       environment.

       SIG may be a signal name like 'PIPE', or a signal number like
       '13'.  Without SIG, all known signals are included.  Multiple
       signals can be comma-separated.  An empty SIG argument is a
       no-op.

   Exit status:
       125    if the env command itself fails

       126    if COMMAND is found but cannot be invoked

       127    if COMMAND cannot be found

       -      the exit status of COMMAND otherwise
OPTIONS
   -S/--split-string usage in scripts
       The -S option allows specifying multiple parameters in a script.
       Running a script named 1.pl containing the following first line:

              #!/usr/bin/env -S perl -w -T
              ...

       Will execute perl -w -T 1.pl .

       Without the '-S' parameter the script will likely fail with:

              /usr/bin/env: 'perl -w -T': No such file or directory

       See the full documentation for more details.

   --default-signal[=SIG] usage
       This option allows setting a signal handler to its default
       action, which is not possible using the traditional shell trap
       command.  The following example ensures that seq will be
       terminated by SIGPIPE no matter how this signal is being handled
       in the process invoking the command.

              sh -c 'env --default-signal=PIPE seq inf | head -n1'
NOTES
       POSIX's exec(3p) pages says:
              "many existing applications wrongly assume that they start
              with certain signals set to the default action and/or
              unblocked.... Therefore, it is best not to block or ignore
              signals across execs without explicit reason to do so, and
              especially not to block signals across execs of arbitrary
              (not closely cooperating) programs."
AUTHOR
       Written by Richard Mlynarik, David MacKenzie, and Assaf Gordon.
REPORTING BUGS
       GNU coreutils online help:
       <https://www.gnu.org/software/coreutils/>
       Report any translation bugs to
       <https://translationproject.org/team/>
COPYRIGHT
       Copyright (C) 2024 Free Software Foundation, Inc.  License GPLv3+:
       GNU GPL version 3 or later <https://gnu.org/licenses/gpl.html>.
       This is free software: you are free to change and redistribute
       it.  There is NO WARRANTY, to the extent permitted by law.
SEE ALSO
       sigaction(2), sigprocmask(2), signal(7)

       Full documentation <https://www.gnu.org/software/coreutils/env>
       or available locally via: info '(coreutils) env invocation'
COLOPHON
       This page is part of the coreutils (basic file, shell and text
       manipulation utilities) project.  Information about the project
       can be found at http://www.gnu.org/software/coreutils/.  If you
       have a bug report for this manual page, see
       http://www.gnu.org/software/coreutils/.  This page was obtained
       from the tarball coreutils-9.5.tar.xz fetched from
       http://ftp.gnu.org/gnu/coreutils/ on 2024-06-14.  If you
       discover any rendering problems in this HTML version of the page,
       or you believe there is a better or more up-to-date source for
       the page, or you have corrections or improvements to the
       information in this COLOPHON (which is not part of the original
       manual page), send a mail to man-pages@man7.org

GNU coreutils 9.5              March 2024                         ENV(1)
\end{lstlisting}
}}
\endinput  %  ==  ==  ==  ==  ==  ==  ==  ==  ==

\subsection{\refEnv: Show \textt{ENVIRONMENT} variables.}

{\tiny{
\begin{lstlisting}[language=bash]
NAME
       env - run a program in a modified environment
SYNOPSIS
       env [OPTION]... [-] [NAME=VALUE]... [COMMAND [ARG]...]
DESCRIPTION
       Set each NAME to VALUE in the environment and run COMMAND.

       Mandatory arguments to long options are mandatory for short
       options too.

       -a, --argv0=ARG
              pass ARG as the zeroth argument of COMMAND

       -i, --ignore-environment
              start with an empty environment

       -0, --null
              end each output line with NUL, not newline

       -u, --unset=NAME
              remove variable from the environment

       -C, --chdir=DIR
              change working directory to DIR

       -S, --split-string=S
              process and split S into separate arguments; used to pass
              multiple arguments on shebang lines

       --block-signal[=SIG]
              block delivery of SIG signal(s) to COMMAND

       --default-signal[=SIG]
              reset handling of SIG signal(s) to the default

       --ignore-signal[=SIG]
              set handling of SIG signal(s) to do nothing

       --list-signal-handling
              list non default signal handling to stderr

       -v, --debug
              print verbose information for each processing step

       --help display this help and exit

       --version
              output version information and exit

       A mere - implies -i.  If no COMMAND, print the resulting
       environment.

       SIG may be a signal name like 'PIPE', or a signal number like
       '13'.  Without SIG, all known signals are included.  Multiple
       signals can be comma-separated.  An empty SIG argument is a
       no-op.

   Exit status:
       125    if the env command itself fails

       126    if COMMAND is found but cannot be invoked

       127    if COMMAND cannot be found

       -      the exit status of COMMAND otherwise
OPTIONS
   -S/--split-string usage in scripts
       The -S option allows specifying multiple parameters in a script.
       Running a script named 1.pl containing the following first line:

              #!/usr/bin/env -S perl -w -T
              ...

       Will execute perl -w -T 1.pl .

       Without the '-S' parameter the script will likely fail with:

              /usr/bin/env: 'perl -w -T': No such file or directory

       See the full documentation for more details.

   --default-signal[=SIG] usage
       This option allows setting a signal handler to its default
       action, which is not possible using the traditional shell trap
       command.  The following example ensures that seq will be
       terminated by SIGPIPE no matter how this signal is being handled
       in the process invoking the command.

              sh -c 'env --default-signal=PIPE seq inf | head -n1'
NOTES
       POSIX's exec(3p) pages says:
              "many existing applications wrongly assume that they start
              with certain signals set to the default action and/or
              unblocked.... Therefore, it is best not to block or ignore
              signals across execs without explicit reason to do so, and
              especially not to block signals across execs of arbitrary
              (not closely cooperating) programs."
AUTHOR
       Written by Richard Mlynarik, David MacKenzie, and Assaf Gordon.
REPORTING BUGS
       GNU coreutils online help:
       <https://www.gnu.org/software/coreutils/>
       Report any translation bugs to
       <https://translationproject.org/team/>
COPYRIGHT
       Copyright (C) 2024 Free Software Foundation, Inc.  License GPLv3+:
       GNU GPL version 3 or later <https://gnu.org/licenses/gpl.html>.
       This is free software: you are free to change and redistribute
       it.  There is NO WARRANTY, to the extent permitted by law.
SEE ALSO
       sigaction(2), sigprocmask(2), signal(7)

       Full documentation <https://www.gnu.org/software/coreutils/env>
       or available locally via: info '(coreutils) env invocation'
COLOPHON
       This page is part of the coreutils (basic file, shell and text
       manipulation utilities) project.  Information about the project
       can be found at http://www.gnu.org/software/coreutils/.  If you
       have a bug report for this manual page, see
       http://www.gnu.org/software/coreutils/.  This page was obtained
       from the tarball coreutils-9.5.tar.xz fetched from
       http://ftp.gnu.org/gnu/coreutils/ on 2024-06-14.  If you
       discover any rendering problems in this HTML version of the page,
       or you believe there is a better or more up-to-date source for
       the page, or you have corrections or improvements to the
       information in this COLOPHON (which is not part of the original
       manual page), send a mail to man-pages@man7.org

GNU coreutils 9.5              March 2024                         ENV(1)
\end{lstlisting}
}}
\endinput  %  ==  ==  ==  ==  ==  ==  ==  ==  ==

\subsection{\refEnv: Show \textt{ENVIRONMENT} variables.}

{\tiny{
\begin{lstlisting}[language=bash]
NAME
       env - run a program in a modified environment
SYNOPSIS
       env [OPTION]... [-] [NAME=VALUE]... [COMMAND [ARG]...]
DESCRIPTION
       Set each NAME to VALUE in the environment and run COMMAND.

       Mandatory arguments to long options are mandatory for short
       options too.

       -a, --argv0=ARG
              pass ARG as the zeroth argument of COMMAND

       -i, --ignore-environment
              start with an empty environment

       -0, --null
              end each output line with NUL, not newline

       -u, --unset=NAME
              remove variable from the environment

       -C, --chdir=DIR
              change working directory to DIR

       -S, --split-string=S
              process and split S into separate arguments; used to pass
              multiple arguments on shebang lines

       --block-signal[=SIG]
              block delivery of SIG signal(s) to COMMAND

       --default-signal[=SIG]
              reset handling of SIG signal(s) to the default

       --ignore-signal[=SIG]
              set handling of SIG signal(s) to do nothing

       --list-signal-handling
              list non default signal handling to stderr

       -v, --debug
              print verbose information for each processing step

       --help display this help and exit

       --version
              output version information and exit

       A mere - implies -i.  If no COMMAND, print the resulting
       environment.

       SIG may be a signal name like 'PIPE', or a signal number like
       '13'.  Without SIG, all known signals are included.  Multiple
       signals can be comma-separated.  An empty SIG argument is a
       no-op.

   Exit status:
       125    if the env command itself fails

       126    if COMMAND is found but cannot be invoked

       127    if COMMAND cannot be found

       -      the exit status of COMMAND otherwise
OPTIONS
   -S/--split-string usage in scripts
       The -S option allows specifying multiple parameters in a script.
       Running a script named 1.pl containing the following first line:

              #!/usr/bin/env -S perl -w -T
              ...

       Will execute perl -w -T 1.pl .

       Without the '-S' parameter the script will likely fail with:

              /usr/bin/env: 'perl -w -T': No such file or directory

       See the full documentation for more details.

   --default-signal[=SIG] usage
       This option allows setting a signal handler to its default
       action, which is not possible using the traditional shell trap
       command.  The following example ensures that seq will be
       terminated by SIGPIPE no matter how this signal is being handled
       in the process invoking the command.

              sh -c 'env --default-signal=PIPE seq inf | head -n1'
NOTES
       POSIX's exec(3p) pages says:
              "many existing applications wrongly assume that they start
              with certain signals set to the default action and/or
              unblocked.... Therefore, it is best not to block or ignore
              signals across execs without explicit reason to do so, and
              especially not to block signals across execs of arbitrary
              (not closely cooperating) programs."
AUTHOR
       Written by Richard Mlynarik, David MacKenzie, and Assaf Gordon.
REPORTING BUGS
       GNU coreutils online help:
       <https://www.gnu.org/software/coreutils/>
       Report any translation bugs to
       <https://translationproject.org/team/>
COPYRIGHT
       Copyright (C) 2024 Free Software Foundation, Inc.  License GPLv3+:
       GNU GPL version 3 or later <https://gnu.org/licenses/gpl.html>.
       This is free software: you are free to change and redistribute
       it.  There is NO WARRANTY, to the extent permitted by law.
SEE ALSO
       sigaction(2), sigprocmask(2), signal(7)

       Full documentation <https://www.gnu.org/software/coreutils/env>
       or available locally via: info '(coreutils) env invocation'
COLOPHON
       This page is part of the coreutils (basic file, shell and text
       manipulation utilities) project.  Information about the project
       can be found at http://www.gnu.org/software/coreutils/.  If you
       have a bug report for this manual page, see
       http://www.gnu.org/software/coreutils/.  This page was obtained
       from the tarball coreutils-9.5.tar.xz fetched from
       http://ftp.gnu.org/gnu/coreutils/ on 2024-06-14.  If you
       discover any rendering problems in this HTML version of the page,
       or you believe there is a better or more up-to-date source for
       the page, or you have corrections or improvements to the
       information in this COLOPHON (which is not part of the original
       manual page), send a mail to man-pages@man7.org

GNU coreutils 9.5              March 2024                         ENV(1)
\end{lstlisting}
}}
\endinput  %  ==  ==  ==  ==  ==  ==  ==  ==  ==

\subsection{\refEnv: Show \textt{ENVIRONMENT} variables.}

{\tiny{
\begin{lstlisting}[language=bash]
NAME
       env - run a program in a modified environment
SYNOPSIS
       env [OPTION]... [-] [NAME=VALUE]... [COMMAND [ARG]...]
DESCRIPTION
       Set each NAME to VALUE in the environment and run COMMAND.

       Mandatory arguments to long options are mandatory for short
       options too.

       -a, --argv0=ARG
              pass ARG as the zeroth argument of COMMAND

       -i, --ignore-environment
              start with an empty environment

       -0, --null
              end each output line with NUL, not newline

       -u, --unset=NAME
              remove variable from the environment

       -C, --chdir=DIR
              change working directory to DIR

       -S, --split-string=S
              process and split S into separate arguments; used to pass
              multiple arguments on shebang lines

       --block-signal[=SIG]
              block delivery of SIG signal(s) to COMMAND

       --default-signal[=SIG]
              reset handling of SIG signal(s) to the default

       --ignore-signal[=SIG]
              set handling of SIG signal(s) to do nothing

       --list-signal-handling
              list non default signal handling to stderr

       -v, --debug
              print verbose information for each processing step

       --help display this help and exit

       --version
              output version information and exit

       A mere - implies -i.  If no COMMAND, print the resulting
       environment.

       SIG may be a signal name like 'PIPE', or a signal number like
       '13'.  Without SIG, all known signals are included.  Multiple
       signals can be comma-separated.  An empty SIG argument is a
       no-op.

   Exit status:
       125    if the env command itself fails

       126    if COMMAND is found but cannot be invoked

       127    if COMMAND cannot be found

       -      the exit status of COMMAND otherwise
OPTIONS
   -S/--split-string usage in scripts
       The -S option allows specifying multiple parameters in a script.
       Running a script named 1.pl containing the following first line:

              #!/usr/bin/env -S perl -w -T
              ...

       Will execute perl -w -T 1.pl .

       Without the '-S' parameter the script will likely fail with:

              /usr/bin/env: 'perl -w -T': No such file or directory

       See the full documentation for more details.

   --default-signal[=SIG] usage
       This option allows setting a signal handler to its default
       action, which is not possible using the traditional shell trap
       command.  The following example ensures that seq will be
       terminated by SIGPIPE no matter how this signal is being handled
       in the process invoking the command.

              sh -c 'env --default-signal=PIPE seq inf | head -n1'
NOTES
       POSIX's exec(3p) pages says:
              "many existing applications wrongly assume that they start
              with certain signals set to the default action and/or
              unblocked.... Therefore, it is best not to block or ignore
              signals across execs without explicit reason to do so, and
              especially not to block signals across execs of arbitrary
              (not closely cooperating) programs."
AUTHOR
       Written by Richard Mlynarik, David MacKenzie, and Assaf Gordon.
REPORTING BUGS
       GNU coreutils online help:
       <https://www.gnu.org/software/coreutils/>
       Report any translation bugs to
       <https://translationproject.org/team/>
COPYRIGHT
       Copyright (C) 2024 Free Software Foundation, Inc.  License GPLv3+:
       GNU GPL version 3 or later <https://gnu.org/licenses/gpl.html>.
       This is free software: you are free to change and redistribute
       it.  There is NO WARRANTY, to the extent permitted by law.
SEE ALSO
       sigaction(2), sigprocmask(2), signal(7)

       Full documentation <https://www.gnu.org/software/coreutils/env>
       or available locally via: info '(coreutils) env invocation'
COLOPHON
       This page is part of the coreutils (basic file, shell and text
       manipulation utilities) project.  Information about the project
       can be found at http://www.gnu.org/software/coreutils/.  If you
       have a bug report for this manual page, see
       http://www.gnu.org/software/coreutils/.  This page was obtained
       from the tarball coreutils-9.5.tar.xz fetched from
       http://ftp.gnu.org/gnu/coreutils/ on 2024-06-14.  If you
       discover any rendering problems in this HTML version of the page,
       or you believe there is a better or more up-to-date source for
       the page, or you have corrections or improvements to the
       information in this COLOPHON (which is not part of the original
       manual page), send a mail to man-pages@man7.org

GNU coreutils 9.5              March 2024                         ENV(1)
\end{lstlisting}
}}
\endinput  %  ==  ==  ==  ==  ==  ==  ==  ==  ==
