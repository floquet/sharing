% % % % \input{./components/man/man-nm}
\subsection{\refNm: List Symbols From Object Files}

{\tiny{
\begin{lstlisting}[language=bash]
NAME
       nm - list symbols from object files
SYNOPSIS
       nm [-A|-o|--print-file-name]
          [-a|--debug-syms]
          [-B|--format=bsd]
          [-C|--demangle[=style]]
          [-D|--dynamic]
          [-fformat|--format=format]
          [-g|--extern-only]
          [-h|--help]
          [--ifunc-chars=CHARS]
          [-j|--format=just-symbols]
          [-l|--line-numbers] [--inlines]
          [-n|-v|--numeric-sort]
          [-P|--portability]
          [-p|--no-sort]
          [-r|--reverse-sort]
          [-S|--print-size]
          [-s|--print-armap]
          [-t radix|--radix=radix]
          [-u|--undefined-only]
          [-U|--defined-only]
          [-V|--version]
          [-W|--no-weak]
          [-X 32_64]
          [--no-demangle]
          [--no-recurse-limit|--recurse-limit]]
          [--plugin name]
          [--size-sort]
          [--special-syms]
          [--synthetic]
          [--target=bfdname]
          [--unicode=method]
          [--with-symbol-versions]
          [--without-symbol-versions]
          [objfile...]
DESCRIPTION
       GNU nm lists the symbols from object files objfile....  If no
       object files are listed as arguments, nm assumes the file a.out.

       For each symbol, nm shows:

       *   The symbol value, in the radix selected by options (see
           below), or hexadecimal by default.

       *   The symbol type.  At least the following types are used;
           others are, as well, depending on the object file format.  If
           lowercase, the symbol is usually local; if uppercase, the
           symbol is global (external).  There are however a few
           lowercase symbols that are shown for special global symbols
           ("u", "v" and "w").

           "A" The symbol's value is absolute, and will not be changed
               by further linking.

           "B"
           "b" The symbol is in the BSS data section.  This section
               typically contains zero-initialized or uninitialized
               data, although the exact behavior is system dependent.

           "C"
           "c" The symbol is common.  Common symbols are uninitialized
               data.  When linking, multiple common symbols may appear
               with the same name.  If the symbol is defined anywhere,
               the common symbols are treated as undefined references.
               The lower case c character is used when the symbol is in
               a special section for small commons.

           "D"
           "d" The symbol is in the initialized data section.

           "G"
           "g" The symbol is in an initialized data section for small
               objects.  Some object file formats permit more efficient
               access to small data objects, such as a global int
               variable as opposed to a large global array.

           "i" For PE format files this indicates that the symbol is in
               a section specific to the implementation of DLLs.

               For ELF format files this indicates that the symbol is an
               indirect function.  This is a GNU extension to the
               standard set of ELF symbol types.  It indicates a symbol
               which if referenced by a relocation does not evaluate to
               its address, but instead must be invoked at runtime.  The
               runtime execution will then return the value to be used
               in the relocation.

               Note - the actual symbols display for GNU indirect
               symbols is controlled by the --ifunc-chars command line
               option.  If this option has been provided then the first
               character in the string will be used for global indirect
               function symbols.  If the string contains a second
               character then that will be used for local indirect
               function symbols.

           "I" The symbol is an indirect reference to another symbol.

           "N" The symbol is a debugging symbol.

           "n" The symbol is in a non-data, non-code, non-debug read-
               only section.

           "p" The symbol is in a stack unwind section.

           "R"
           "r" The symbol is in a read only data section.

           "S"
           "s" The symbol is in an uninitialized or zero-initialized
               data section for small objects.

           "T"
           "t" The symbol is in the text (code) section.

           "U" The symbol is undefined.

           "u" The symbol is a unique global symbol.  This is a GNU
               extension to the standard set of ELF symbol bindings.
               For such a symbol the dynamic linker will make sure that
               in the entire process there is just one symbol with this
               name and type in use.

           "V"
           "v" The symbol is a weak object.  When a weak defined symbol
               is linked with a normal defined symbol, the normal
               defined symbol is used with no error.  When a weak
               undefined symbol is linked and the symbol is not defined,
               the value of the weak symbol becomes zero with no error.
               On some systems, uppercase indicates that a default value
               has been specified.

           "W"
           "w" The symbol is a weak symbol that has not been
               specifically tagged as a weak object symbol.  When a weak
               defined symbol is linked with a normal defined symbol,
               the normal defined symbol is used with no error.  When a
               weak undefined symbol is linked and the symbol is not
               defined, the value of the symbol is determined in a
               system-specific manner without error.  On some systems,
               uppercase indicates that a default value has been
               specified.

           "-" The symbol is a stabs symbol in an a.out object file.  In
               this case, the next values printed are the stabs other
               field, the stabs desc field, and the stab type.  Stabs
               symbols are used to hold debugging information.

           "?" The symbol type is unknown, or object file format
               specific.

       *   The symbol name.  If a symbol has version information
           associated with it, then the version information is displayed
           as well.  If the versioned symbol is undefined or hidden from
           linker, the version string is displayed as a suffix to the
           symbol name, preceded by an @ character.  For example
           foo@VER_1.  If the version is the default version to be used
           when resolving unversioned references to the symbol, then it
           is displayed as a suffix preceded by two @ characters.  For
           example foo@@VER_2.
OPTIONS
       The long and short forms of options, shown here as alternatives,
       are equivalent.

       -A
       -o
       --print-file-name
           Precede each symbol by the name of the input file (or archive
           member) in which it was found, rather than identifying the
           input file once only, before all of its symbols.

       -a
       --debug-syms
           Display all symbols, even debugger-only symbols; normally
           these are not listed.

       -B  The same as --format=bsd (for compatibility with the MIPS
           nm).

       -C
       --demangle[=style]
           Decode (demangle) low-level symbol names into user-level
           names.  Besides removing any initial underscore prepended by
           the system, this makes C++ function names readable. Different
           compilers have different mangling styles. The optional
           demangling style argument can be used to choose an
           appropriate demangling style for your compiler.

       --no-demangle
           Do not demangle low-level symbol names.  This is the default.

       --recurse-limit
       --no-recurse-limit
       --recursion-limit
       --no-recursion-limit
           Enables or disables a limit on the amount of recursion
           performed whilst demangling strings.  Since the name mangling
           formats allow for an infinite level of recursion it is
           possible to create strings whose decoding will exhaust the
           amount of stack space available on the host machine,
           triggering a memory fault.  The limit tries to prevent this
           from happening by restricting recursion to 2048 levels of
           nesting.

           The default is for this limit to be enabled, but disabling it
           may be necessary in order to demangle truly complicated
           names.  Note however that if the recursion limit is disabled
           then stack exhaustion is possible and any bug reports about
           such an event will be rejected.

       -D
       --dynamic
           Display the dynamic symbols rather than the normal symbols.
           This is only meaningful for dynamic objects, such as certain
           types of shared libraries.

       -f format
       --format=format
           Use the output format format, which can be "bsd", "sysv",
           "posix" or "just-symbols".  The default is "bsd".  Only the
           first character of format is significant; it can be either
           upper or lower case.

       -g
       --extern-only
           Display only external symbols.

       -h
       --help
           Show a summary of the options to nm and exit.

       --ifunc-chars=CHARS
           When display GNU indirect function symbols nm will default to
           using the "i" character for both local indirect functions and
           global indirect functions.  The --ifunc-chars option allows
           the user to specify a string containing one or two
           characters. The first character will be used for global
           indirect function symbols and the second character, if
           present, will be used for local indirect function symbols.

       j   The same as --format=just-symbols.

       -l
       --line-numbers
           For each symbol, use debugging information to try to find a
           filename and line number.  For a defined symbol, look for the
           line number of the address of the symbol.  For an undefined
           symbol, look for the line number of a relocation entry which
           refers to the symbol.  If line number information can be
           found, print it after the other symbol information.

       --inlines
           When option -l is active, if the address belongs to a
           function that was inlined, then this option causes the source
           information for all enclosing scopes back to the first non-
           inlined function to be printed as well.  For example, if
           "main" inlines "callee1" which inlines "callee2", and address
           is from "callee2", the source information for "callee1" and
           "main" will also be printed.

       -n
       -v
       --numeric-sort
           Sort symbols numerically by their addresses, rather than
           alphabetically by their names.

       -p
       --no-sort
           Do not bother to sort the symbols in any order; print them in
           the order encountered.

       -P
       --portability
           Use the POSIX.2 standard output format instead of the default
           format.  Equivalent to -f posix.

       -r
       --reverse-sort
           Reverse the order of the sort (whether numeric or
           alphabetic); let the last come first.

       -S
       --print-size
           Print both value and size of defined symbols for the "bsd"
           output style.  This option has no effect for object formats
           that do not record symbol sizes, unless --size-sort is also
           used in which case a calculated size is displayed.

       -s
       --print-armap
           When listing symbols from archive members, include the index:
           a mapping (stored in the archive by ar or ranlib) of which
           modules contain definitions for which names.

       -t radix
       --radix=radix
           Use radix as the radix for printing the symbol values.  It
           must be d for decimal, o for octal, or x for hexadecimal.

       -u
       --undefined-only
           Display only undefined symbols (those external to each object
           file).  By default both defined and undefined symbols are
           displayed.

       -U
       --defined-only
           Display only defined symbols for each object file.  By
           default both defined and undefined symbols are displayed.

       -V
       --version
           Show the version number of nm and exit.

       -X  This option is ignored for compatibility with the AIX version
           of nm.  It takes one parameter which must be the string
           32_64.  The default mode of AIX nm corresponds to -X 32,
           which is not supported by GNU nm.

       --plugin name
           Load the plugin called name to add support for extra target
           types.  This option is only available if the toolchain has
           been built with plugin support enabled.

           If --plugin is not provided, but plugin support has been
           enabled then nm iterates over the files in
           ${libdir}/bfd-plugins in alphabetic order and the first
           plugin that claims the object in question is used.

           Please note that this plugin search directory is not the one
           used by ld's -plugin option.  In order to make nm use the
           linker plugin it must be copied into the
           ${libdir}/bfd-plugins directory.  For GCC based compilations
           the linker plugin is called liblto_plugin.so.0.0.0.  For
           Clang based compilations it is called LLVMgold.so.  The GCC
           plugin is always backwards compatible with earlier versions,
           so it is sufficient to just copy the newest one.

       --size-sort
           Sort symbols by size.  For ELF objects symbol sizes are read
           from the ELF, for other object types the symbol sizes are
           computed as the difference between the value of the symbol
           and the value of the symbol with the next higher value.  If
           the "bsd" output format is used the size of the symbol is
           printed, rather than the value, and -S must be used in order
           both size and value to be printed.

           Note - this option does not work if --undefined-only has been
           enabled as undefined symbols have no size.

       --special-syms
           Display symbols which have a target-specific special meaning.
           These symbols are usually used by the target for some special
           processing and are not normally helpful when included in the
           normal symbol lists.  For example for ARM targets this option
           would skip the mapping symbols used to mark transitions
           between ARM code, THUMB code and data.

       --synthetic
           Include synthetic symbols in the output.  These are special
           symbols created by the linker for various purposes.  They are
           not shown by default since they are not part of the binary's
           original source code.

       --unicode=[default|invalid|locale|escape|hex|highlight]
           Controls the display of UTF-8 encoded multibyte characters in
           strings.  The default (--unicode=default) is to give them no
           special treatment.  The --unicode=locale option displays the
           sequence in the current locale, which may or may not support
           them.  The options --unicode=hex and --unicode=invalid
           display them as hex byte sequences enclosed by either angle
           brackets or curly braces.

           The --unicode=escape option displays them as escape sequences
           (\uxxxx) and the --unicode=highlight option displays them as
           escape sequences highlighted in red (if supported by the
           output device).  The colouring is intended to draw attention
           to the presence of unicode sequences where they might not be
           expected.

       -W
       --no-weak
           Do not display weak symbols.

       --with-symbol-versions
       --without-symbol-versions
           Enables or disables the display of symbol version
           information.  The version string is displayed as a suffix to
           the symbol name, preceded by an @ character.  For example
           foo@VER_1.  If the version is the default version to be used
           when resolving unversioned references to the symbol then it
           is displayed as a suffix preceded by two @ characters.  For
           example foo@@VER_2.  By default, symbol version information
           is displayed.

       --target=bfdname
           Specify an object code format other than your system's
           default format.

       @file
           Read command-line options from file.  The options read are
           inserted in place of the original @file option.  If file does
           not exist, or cannot be read, then the option will be treated
           literally, and not removed.

           Options in file are separated by whitespace.  A whitespace
           character may be included in an option by surrounding the
           entire option in either single or double quotes.  Any
           character (including a backslash) may be included by
           prefixing the character to be included with a backslash.  The
           file may itself contain additional @file options; any such
           options will be processed recursively.
SEE ALSO
       ar(1), objdump(1), ranlib(1), and the Info entries for binutils.
COPYRIGHT
       Copyright (c) 1991-2024 Free Software Foundation, Inc.

       Permission is granted to copy, distribute and/or modify this
       document under the terms of the GNU Free Documentation License,
       Version 1.3 or any later version published by the Free Software
       Foundation; with no Invariant Sections, with no Front-Cover
       Texts, and with no Back-Cover Texts.  A copy of the license is
       included in the section entitled "GNU Free Documentation
       License".
COLOPHON
       This page is part of the binutils (a collection of tools for
       working with executable binaries) project.  Information about the
       project can be found at http://www.gnu.org/software/binutils/.
       If you have a bug report for this manual page, see
       http://sourceware.org/bugzilla/enter_bug.cgi?product=binutils.
       This page was obtained from the tarball binutils-2.42.tar.gz
       fetched from https://ftp.gnu.org/gnu/binutils/ on 2024-06-14.
       If you discover any rendering problems in this HTML version of
       the page, or you believe there is a better or more up-to-date
       source for the page, or you have corrections or improvements to
       the information in this COLOPHON (which is not part of the
       original manual page), send a mail to man-pages@man7.org

binutils-2.42                  2024-06-14                          NM(1)\end{lstlisting}
}}
\endinput  %  ==  ==  ==  ==  ==  ==  ==  ==  ==

\subsection{\refNm: List Symbols From Object Files}

{\tiny{
\begin{lstlisting}[language=bash]
NAME
       nm - list symbols from object files
SYNOPSIS
       nm [-A|-o|--print-file-name]
          [-a|--debug-syms]
          [-B|--format=bsd]
          [-C|--demangle[=style]]
          [-D|--dynamic]
          [-fformat|--format=format]
          [-g|--extern-only]
          [-h|--help]
          [--ifunc-chars=CHARS]
          [-j|--format=just-symbols]
          [-l|--line-numbers] [--inlines]
          [-n|-v|--numeric-sort]
          [-P|--portability]
          [-p|--no-sort]
          [-r|--reverse-sort]
          [-S|--print-size]
          [-s|--print-armap]
          [-t radix|--radix=radix]
          [-u|--undefined-only]
          [-U|--defined-only]
          [-V|--version]
          [-W|--no-weak]
          [-X 32_64]
          [--no-demangle]
          [--no-recurse-limit|--recurse-limit]]
          [--plugin name]
          [--size-sort]
          [--special-syms]
          [--synthetic]
          [--target=bfdname]
          [--unicode=method]
          [--with-symbol-versions]
          [--without-symbol-versions]
          [objfile...]
DESCRIPTION
       GNU nm lists the symbols from object files objfile....  If no
       object files are listed as arguments, nm assumes the file a.out.

       For each symbol, nm shows:

       *   The symbol value, in the radix selected by options (see
           below), or hexadecimal by default.

       *   The symbol type.  At least the following types are used;
           others are, as well, depending on the object file format.  If
           lowercase, the symbol is usually local; if uppercase, the
           symbol is global (external).  There are however a few
           lowercase symbols that are shown for special global symbols
           ("u", "v" and "w").

           "A" The symbol's value is absolute, and will not be changed
               by further linking.

           "B"
           "b" The symbol is in the BSS data section.  This section
               typically contains zero-initialized or uninitialized
               data, although the exact behavior is system dependent.

           "C"
           "c" The symbol is common.  Common symbols are uninitialized
               data.  When linking, multiple common symbols may appear
               with the same name.  If the symbol is defined anywhere,
               the common symbols are treated as undefined references.
               The lower case c character is used when the symbol is in
               a special section for small commons.

           "D"
           "d" The symbol is in the initialized data section.

           "G"
           "g" The symbol is in an initialized data section for small
               objects.  Some object file formats permit more efficient
               access to small data objects, such as a global int
               variable as opposed to a large global array.

           "i" For PE format files this indicates that the symbol is in
               a section specific to the implementation of DLLs.

               For ELF format files this indicates that the symbol is an
               indirect function.  This is a GNU extension to the
               standard set of ELF symbol types.  It indicates a symbol
               which if referenced by a relocation does not evaluate to
               its address, but instead must be invoked at runtime.  The
               runtime execution will then return the value to be used
               in the relocation.

               Note - the actual symbols display for GNU indirect
               symbols is controlled by the --ifunc-chars command line
               option.  If this option has been provided then the first
               character in the string will be used for global indirect
               function symbols.  If the string contains a second
               character then that will be used for local indirect
               function symbols.

           "I" The symbol is an indirect reference to another symbol.

           "N" The symbol is a debugging symbol.

           "n" The symbol is in a non-data, non-code, non-debug read-
               only section.

           "p" The symbol is in a stack unwind section.

           "R"
           "r" The symbol is in a read only data section.

           "S"
           "s" The symbol is in an uninitialized or zero-initialized
               data section for small objects.

           "T"
           "t" The symbol is in the text (code) section.

           "U" The symbol is undefined.

           "u" The symbol is a unique global symbol.  This is a GNU
               extension to the standard set of ELF symbol bindings.
               For such a symbol the dynamic linker will make sure that
               in the entire process there is just one symbol with this
               name and type in use.

           "V"
           "v" The symbol is a weak object.  When a weak defined symbol
               is linked with a normal defined symbol, the normal
               defined symbol is used with no error.  When a weak
               undefined symbol is linked and the symbol is not defined,
               the value of the weak symbol becomes zero with no error.
               On some systems, uppercase indicates that a default value
               has been specified.

           "W"
           "w" The symbol is a weak symbol that has not been
               specifically tagged as a weak object symbol.  When a weak
               defined symbol is linked with a normal defined symbol,
               the normal defined symbol is used with no error.  When a
               weak undefined symbol is linked and the symbol is not
               defined, the value of the symbol is determined in a
               system-specific manner without error.  On some systems,
               uppercase indicates that a default value has been
               specified.

           "-" The symbol is a stabs symbol in an a.out object file.  In
               this case, the next values printed are the stabs other
               field, the stabs desc field, and the stab type.  Stabs
               symbols are used to hold debugging information.

           "?" The symbol type is unknown, or object file format
               specific.

       *   The symbol name.  If a symbol has version information
           associated with it, then the version information is displayed
           as well.  If the versioned symbol is undefined or hidden from
           linker, the version string is displayed as a suffix to the
           symbol name, preceded by an @ character.  For example
           foo@VER_1.  If the version is the default version to be used
           when resolving unversioned references to the symbol, then it
           is displayed as a suffix preceded by two @ characters.  For
           example foo@@VER_2.
OPTIONS
       The long and short forms of options, shown here as alternatives,
       are equivalent.

       -A
       -o
       --print-file-name
           Precede each symbol by the name of the input file (or archive
           member) in which it was found, rather than identifying the
           input file once only, before all of its symbols.

       -a
       --debug-syms
           Display all symbols, even debugger-only symbols; normally
           these are not listed.

       -B  The same as --format=bsd (for compatibility with the MIPS
           nm).

       -C
       --demangle[=style]
           Decode (demangle) low-level symbol names into user-level
           names.  Besides removing any initial underscore prepended by
           the system, this makes C++ function names readable. Different
           compilers have different mangling styles. The optional
           demangling style argument can be used to choose an
           appropriate demangling style for your compiler.

       --no-demangle
           Do not demangle low-level symbol names.  This is the default.

       --recurse-limit
       --no-recurse-limit
       --recursion-limit
       --no-recursion-limit
           Enables or disables a limit on the amount of recursion
           performed whilst demangling strings.  Since the name mangling
           formats allow for an infinite level of recursion it is
           possible to create strings whose decoding will exhaust the
           amount of stack space available on the host machine,
           triggering a memory fault.  The limit tries to prevent this
           from happening by restricting recursion to 2048 levels of
           nesting.

           The default is for this limit to be enabled, but disabling it
           may be necessary in order to demangle truly complicated
           names.  Note however that if the recursion limit is disabled
           then stack exhaustion is possible and any bug reports about
           such an event will be rejected.

       -D
       --dynamic
           Display the dynamic symbols rather than the normal symbols.
           This is only meaningful for dynamic objects, such as certain
           types of shared libraries.

       -f format
       --format=format
           Use the output format format, which can be "bsd", "sysv",
           "posix" or "just-symbols".  The default is "bsd".  Only the
           first character of format is significant; it can be either
           upper or lower case.

       -g
       --extern-only
           Display only external symbols.

       -h
       --help
           Show a summary of the options to nm and exit.

       --ifunc-chars=CHARS
           When display GNU indirect function symbols nm will default to
           using the "i" character for both local indirect functions and
           global indirect functions.  The --ifunc-chars option allows
           the user to specify a string containing one or two
           characters. The first character will be used for global
           indirect function symbols and the second character, if
           present, will be used for local indirect function symbols.

       j   The same as --format=just-symbols.

       -l
       --line-numbers
           For each symbol, use debugging information to try to find a
           filename and line number.  For a defined symbol, look for the
           line number of the address of the symbol.  For an undefined
           symbol, look for the line number of a relocation entry which
           refers to the symbol.  If line number information can be
           found, print it after the other symbol information.

       --inlines
           When option -l is active, if the address belongs to a
           function that was inlined, then this option causes the source
           information for all enclosing scopes back to the first non-
           inlined function to be printed as well.  For example, if
           "main" inlines "callee1" which inlines "callee2", and address
           is from "callee2", the source information for "callee1" and
           "main" will also be printed.

       -n
       -v
       --numeric-sort
           Sort symbols numerically by their addresses, rather than
           alphabetically by their names.

       -p
       --no-sort
           Do not bother to sort the symbols in any order; print them in
           the order encountered.

       -P
       --portability
           Use the POSIX.2 standard output format instead of the default
           format.  Equivalent to -f posix.

       -r
       --reverse-sort
           Reverse the order of the sort (whether numeric or
           alphabetic); let the last come first.

       -S
       --print-size
           Print both value and size of defined symbols for the "bsd"
           output style.  This option has no effect for object formats
           that do not record symbol sizes, unless --size-sort is also
           used in which case a calculated size is displayed.

       -s
       --print-armap
           When listing symbols from archive members, include the index:
           a mapping (stored in the archive by ar or ranlib) of which
           modules contain definitions for which names.

       -t radix
       --radix=radix
           Use radix as the radix for printing the symbol values.  It
           must be d for decimal, o for octal, or x for hexadecimal.

       -u
       --undefined-only
           Display only undefined symbols (those external to each object
           file).  By default both defined and undefined symbols are
           displayed.

       -U
       --defined-only
           Display only defined symbols for each object file.  By
           default both defined and undefined symbols are displayed.

       -V
       --version
           Show the version number of nm and exit.

       -X  This option is ignored for compatibility with the AIX version
           of nm.  It takes one parameter which must be the string
           32_64.  The default mode of AIX nm corresponds to -X 32,
           which is not supported by GNU nm.

       --plugin name
           Load the plugin called name to add support for extra target
           types.  This option is only available if the toolchain has
           been built with plugin support enabled.

           If --plugin is not provided, but plugin support has been
           enabled then nm iterates over the files in
           ${libdir}/bfd-plugins in alphabetic order and the first
           plugin that claims the object in question is used.

           Please note that this plugin search directory is not the one
           used by ld's -plugin option.  In order to make nm use the
           linker plugin it must be copied into the
           ${libdir}/bfd-plugins directory.  For GCC based compilations
           the linker plugin is called liblto_plugin.so.0.0.0.  For
           Clang based compilations it is called LLVMgold.so.  The GCC
           plugin is always backwards compatible with earlier versions,
           so it is sufficient to just copy the newest one.

       --size-sort
           Sort symbols by size.  For ELF objects symbol sizes are read
           from the ELF, for other object types the symbol sizes are
           computed as the difference between the value of the symbol
           and the value of the symbol with the next higher value.  If
           the "bsd" output format is used the size of the symbol is
           printed, rather than the value, and -S must be used in order
           both size and value to be printed.

           Note - this option does not work if --undefined-only has been
           enabled as undefined symbols have no size.

       --special-syms
           Display symbols which have a target-specific special meaning.
           These symbols are usually used by the target for some special
           processing and are not normally helpful when included in the
           normal symbol lists.  For example for ARM targets this option
           would skip the mapping symbols used to mark transitions
           between ARM code, THUMB code and data.

       --synthetic
           Include synthetic symbols in the output.  These are special
           symbols created by the linker for various purposes.  They are
           not shown by default since they are not part of the binary's
           original source code.

       --unicode=[default|invalid|locale|escape|hex|highlight]
           Controls the display of UTF-8 encoded multibyte characters in
           strings.  The default (--unicode=default) is to give them no
           special treatment.  The --unicode=locale option displays the
           sequence in the current locale, which may or may not support
           them.  The options --unicode=hex and --unicode=invalid
           display them as hex byte sequences enclosed by either angle
           brackets or curly braces.

           The --unicode=escape option displays them as escape sequences
           (\uxxxx) and the --unicode=highlight option displays them as
           escape sequences highlighted in red (if supported by the
           output device).  The colouring is intended to draw attention
           to the presence of unicode sequences where they might not be
           expected.

       -W
       --no-weak
           Do not display weak symbols.

       --with-symbol-versions
       --without-symbol-versions
           Enables or disables the display of symbol version
           information.  The version string is displayed as a suffix to
           the symbol name, preceded by an @ character.  For example
           foo@VER_1.  If the version is the default version to be used
           when resolving unversioned references to the symbol then it
           is displayed as a suffix preceded by two @ characters.  For
           example foo@@VER_2.  By default, symbol version information
           is displayed.

       --target=bfdname
           Specify an object code format other than your system's
           default format.

       @file
           Read command-line options from file.  The options read are
           inserted in place of the original @file option.  If file does
           not exist, or cannot be read, then the option will be treated
           literally, and not removed.

           Options in file are separated by whitespace.  A whitespace
           character may be included in an option by surrounding the
           entire option in either single or double quotes.  Any
           character (including a backslash) may be included by
           prefixing the character to be included with a backslash.  The
           file may itself contain additional @file options; any such
           options will be processed recursively.
SEE ALSO
       ar(1), objdump(1), ranlib(1), and the Info entries for binutils.
COPYRIGHT
       Copyright (c) 1991-2024 Free Software Foundation, Inc.

       Permission is granted to copy, distribute and/or modify this
       document under the terms of the GNU Free Documentation License,
       Version 1.3 or any later version published by the Free Software
       Foundation; with no Invariant Sections, with no Front-Cover
       Texts, and with no Back-Cover Texts.  A copy of the license is
       included in the section entitled "GNU Free Documentation
       License".
COLOPHON
       This page is part of the binutils (a collection of tools for
       working with executable binaries) project.  Information about the
       project can be found at http://www.gnu.org/software/binutils/.
       If you have a bug report for this manual page, see
       http://sourceware.org/bugzilla/enter_bug.cgi?product=binutils.
       This page was obtained from the tarball binutils-2.42.tar.gz
       fetched from https://ftp.gnu.org/gnu/binutils/ on 2024-06-14.
       If you discover any rendering problems in this HTML version of
       the page, or you believe there is a better or more up-to-date
       source for the page, or you have corrections or improvements to
       the information in this COLOPHON (which is not part of the
       original manual page), send a mail to man-pages@man7.org

binutils-2.42                  2024-06-14                          NM(1)\end{lstlisting}
}}
\endinput  %  ==  ==  ==  ==  ==  ==  ==  ==  ==

\subsection{\refNm: List Symbols From Object Files}

{\tiny{
\begin{lstlisting}[language=bash]
NAME
       nm - list symbols from object files
SYNOPSIS
       nm [-A|-o|--print-file-name]
          [-a|--debug-syms]
          [-B|--format=bsd]
          [-C|--demangle[=style]]
          [-D|--dynamic]
          [-fformat|--format=format]
          [-g|--extern-only]
          [-h|--help]
          [--ifunc-chars=CHARS]
          [-j|--format=just-symbols]
          [-l|--line-numbers] [--inlines]
          [-n|-v|--numeric-sort]
          [-P|--portability]
          [-p|--no-sort]
          [-r|--reverse-sort]
          [-S|--print-size]
          [-s|--print-armap]
          [-t radix|--radix=radix]
          [-u|--undefined-only]
          [-U|--defined-only]
          [-V|--version]
          [-W|--no-weak]
          [-X 32_64]
          [--no-demangle]
          [--no-recurse-limit|--recurse-limit]]
          [--plugin name]
          [--size-sort]
          [--special-syms]
          [--synthetic]
          [--target=bfdname]
          [--unicode=method]
          [--with-symbol-versions]
          [--without-symbol-versions]
          [objfile...]
DESCRIPTION
       GNU nm lists the symbols from object files objfile....  If no
       object files are listed as arguments, nm assumes the file a.out.

       For each symbol, nm shows:

       *   The symbol value, in the radix selected by options (see
           below), or hexadecimal by default.

       *   The symbol type.  At least the following types are used;
           others are, as well, depending on the object file format.  If
           lowercase, the symbol is usually local; if uppercase, the
           symbol is global (external).  There are however a few
           lowercase symbols that are shown for special global symbols
           ("u", "v" and "w").

           "A" The symbol's value is absolute, and will not be changed
               by further linking.

           "B"
           "b" The symbol is in the BSS data section.  This section
               typically contains zero-initialized or uninitialized
               data, although the exact behavior is system dependent.

           "C"
           "c" The symbol is common.  Common symbols are uninitialized
               data.  When linking, multiple common symbols may appear
               with the same name.  If the symbol is defined anywhere,
               the common symbols are treated as undefined references.
               The lower case c character is used when the symbol is in
               a special section for small commons.

           "D"
           "d" The symbol is in the initialized data section.

           "G"
           "g" The symbol is in an initialized data section for small
               objects.  Some object file formats permit more efficient
               access to small data objects, such as a global int
               variable as opposed to a large global array.

           "i" For PE format files this indicates that the symbol is in
               a section specific to the implementation of DLLs.

               For ELF format files this indicates that the symbol is an
               indirect function.  This is a GNU extension to the
               standard set of ELF symbol types.  It indicates a symbol
               which if referenced by a relocation does not evaluate to
               its address, but instead must be invoked at runtime.  The
               runtime execution will then return the value to be used
               in the relocation.

               Note - the actual symbols display for GNU indirect
               symbols is controlled by the --ifunc-chars command line
               option.  If this option has been provided then the first
               character in the string will be used for global indirect
               function symbols.  If the string contains a second
               character then that will be used for local indirect
               function symbols.

           "I" The symbol is an indirect reference to another symbol.

           "N" The symbol is a debugging symbol.

           "n" The symbol is in a non-data, non-code, non-debug read-
               only section.

           "p" The symbol is in a stack unwind section.

           "R"
           "r" The symbol is in a read only data section.

           "S"
           "s" The symbol is in an uninitialized or zero-initialized
               data section for small objects.

           "T"
           "t" The symbol is in the text (code) section.

           "U" The symbol is undefined.

           "u" The symbol is a unique global symbol.  This is a GNU
               extension to the standard set of ELF symbol bindings.
               For such a symbol the dynamic linker will make sure that
               in the entire process there is just one symbol with this
               name and type in use.

           "V"
           "v" The symbol is a weak object.  When a weak defined symbol
               is linked with a normal defined symbol, the normal
               defined symbol is used with no error.  When a weak
               undefined symbol is linked and the symbol is not defined,
               the value of the weak symbol becomes zero with no error.
               On some systems, uppercase indicates that a default value
               has been specified.

           "W"
           "w" The symbol is a weak symbol that has not been
               specifically tagged as a weak object symbol.  When a weak
               defined symbol is linked with a normal defined symbol,
               the normal defined symbol is used with no error.  When a
               weak undefined symbol is linked and the symbol is not
               defined, the value of the symbol is determined in a
               system-specific manner without error.  On some systems,
               uppercase indicates that a default value has been
               specified.

           "-" The symbol is a stabs symbol in an a.out object file.  In
               this case, the next values printed are the stabs other
               field, the stabs desc field, and the stab type.  Stabs
               symbols are used to hold debugging information.

           "?" The symbol type is unknown, or object file format
               specific.

       *   The symbol name.  If a symbol has version information
           associated with it, then the version information is displayed
           as well.  If the versioned symbol is undefined or hidden from
           linker, the version string is displayed as a suffix to the
           symbol name, preceded by an @ character.  For example
           foo@VER_1.  If the version is the default version to be used
           when resolving unversioned references to the symbol, then it
           is displayed as a suffix preceded by two @ characters.  For
           example foo@@VER_2.
OPTIONS
       The long and short forms of options, shown here as alternatives,
       are equivalent.

       -A
       -o
       --print-file-name
           Precede each symbol by the name of the input file (or archive
           member) in which it was found, rather than identifying the
           input file once only, before all of its symbols.

       -a
       --debug-syms
           Display all symbols, even debugger-only symbols; normally
           these are not listed.

       -B  The same as --format=bsd (for compatibility with the MIPS
           nm).

       -C
       --demangle[=style]
           Decode (demangle) low-level symbol names into user-level
           names.  Besides removing any initial underscore prepended by
           the system, this makes C++ function names readable. Different
           compilers have different mangling styles. The optional
           demangling style argument can be used to choose an
           appropriate demangling style for your compiler.

       --no-demangle
           Do not demangle low-level symbol names.  This is the default.

       --recurse-limit
       --no-recurse-limit
       --recursion-limit
       --no-recursion-limit
           Enables or disables a limit on the amount of recursion
           performed whilst demangling strings.  Since the name mangling
           formats allow for an infinite level of recursion it is
           possible to create strings whose decoding will exhaust the
           amount of stack space available on the host machine,
           triggering a memory fault.  The limit tries to prevent this
           from happening by restricting recursion to 2048 levels of
           nesting.

           The default is for this limit to be enabled, but disabling it
           may be necessary in order to demangle truly complicated
           names.  Note however that if the recursion limit is disabled
           then stack exhaustion is possible and any bug reports about
           such an event will be rejected.

       -D
       --dynamic
           Display the dynamic symbols rather than the normal symbols.
           This is only meaningful for dynamic objects, such as certain
           types of shared libraries.

       -f format
       --format=format
           Use the output format format, which can be "bsd", "sysv",
           "posix" or "just-symbols".  The default is "bsd".  Only the
           first character of format is significant; it can be either
           upper or lower case.

       -g
       --extern-only
           Display only external symbols.

       -h
       --help
           Show a summary of the options to nm and exit.

       --ifunc-chars=CHARS
           When display GNU indirect function symbols nm will default to
           using the "i" character for both local indirect functions and
           global indirect functions.  The --ifunc-chars option allows
           the user to specify a string containing one or two
           characters. The first character will be used for global
           indirect function symbols and the second character, if
           present, will be used for local indirect function symbols.

       j   The same as --format=just-symbols.

       -l
       --line-numbers
           For each symbol, use debugging information to try to find a
           filename and line number.  For a defined symbol, look for the
           line number of the address of the symbol.  For an undefined
           symbol, look for the line number of a relocation entry which
           refers to the symbol.  If line number information can be
           found, print it after the other symbol information.

       --inlines
           When option -l is active, if the address belongs to a
           function that was inlined, then this option causes the source
           information for all enclosing scopes back to the first non-
           inlined function to be printed as well.  For example, if
           "main" inlines "callee1" which inlines "callee2", and address
           is from "callee2", the source information for "callee1" and
           "main" will also be printed.

       -n
       -v
       --numeric-sort
           Sort symbols numerically by their addresses, rather than
           alphabetically by their names.

       -p
       --no-sort
           Do not bother to sort the symbols in any order; print them in
           the order encountered.

       -P
       --portability
           Use the POSIX.2 standard output format instead of the default
           format.  Equivalent to -f posix.

       -r
       --reverse-sort
           Reverse the order of the sort (whether numeric or
           alphabetic); let the last come first.

       -S
       --print-size
           Print both value and size of defined symbols for the "bsd"
           output style.  This option has no effect for object formats
           that do not record symbol sizes, unless --size-sort is also
           used in which case a calculated size is displayed.

       -s
       --print-armap
           When listing symbols from archive members, include the index:
           a mapping (stored in the archive by ar or ranlib) of which
           modules contain definitions for which names.

       -t radix
       --radix=radix
           Use radix as the radix for printing the symbol values.  It
           must be d for decimal, o for octal, or x for hexadecimal.

       -u
       --undefined-only
           Display only undefined symbols (those external to each object
           file).  By default both defined and undefined symbols are
           displayed.

       -U
       --defined-only
           Display only defined symbols for each object file.  By
           default both defined and undefined symbols are displayed.

       -V
       --version
           Show the version number of nm and exit.

       -X  This option is ignored for compatibility with the AIX version
           of nm.  It takes one parameter which must be the string
           32_64.  The default mode of AIX nm corresponds to -X 32,
           which is not supported by GNU nm.

       --plugin name
           Load the plugin called name to add support for extra target
           types.  This option is only available if the toolchain has
           been built with plugin support enabled.

           If --plugin is not provided, but plugin support has been
           enabled then nm iterates over the files in
           ${libdir}/bfd-plugins in alphabetic order and the first
           plugin that claims the object in question is used.

           Please note that this plugin search directory is not the one
           used by ld's -plugin option.  In order to make nm use the
           linker plugin it must be copied into the
           ${libdir}/bfd-plugins directory.  For GCC based compilations
           the linker plugin is called liblto_plugin.so.0.0.0.  For
           Clang based compilations it is called LLVMgold.so.  The GCC
           plugin is always backwards compatible with earlier versions,
           so it is sufficient to just copy the newest one.

       --size-sort
           Sort symbols by size.  For ELF objects symbol sizes are read
           from the ELF, for other object types the symbol sizes are
           computed as the difference between the value of the symbol
           and the value of the symbol with the next higher value.  If
           the "bsd" output format is used the size of the symbol is
           printed, rather than the value, and -S must be used in order
           both size and value to be printed.

           Note - this option does not work if --undefined-only has been
           enabled as undefined symbols have no size.

       --special-syms
           Display symbols which have a target-specific special meaning.
           These symbols are usually used by the target for some special
           processing and are not normally helpful when included in the
           normal symbol lists.  For example for ARM targets this option
           would skip the mapping symbols used to mark transitions
           between ARM code, THUMB code and data.

       --synthetic
           Include synthetic symbols in the output.  These are special
           symbols created by the linker for various purposes.  They are
           not shown by default since they are not part of the binary's
           original source code.

       --unicode=[default|invalid|locale|escape|hex|highlight]
           Controls the display of UTF-8 encoded multibyte characters in
           strings.  The default (--unicode=default) is to give them no
           special treatment.  The --unicode=locale option displays the
           sequence in the current locale, which may or may not support
           them.  The options --unicode=hex and --unicode=invalid
           display them as hex byte sequences enclosed by either angle
           brackets or curly braces.

           The --unicode=escape option displays them as escape sequences
           (\uxxxx) and the --unicode=highlight option displays them as
           escape sequences highlighted in red (if supported by the
           output device).  The colouring is intended to draw attention
           to the presence of unicode sequences where they might not be
           expected.

       -W
       --no-weak
           Do not display weak symbols.

       --with-symbol-versions
       --without-symbol-versions
           Enables or disables the display of symbol version
           information.  The version string is displayed as a suffix to
           the symbol name, preceded by an @ character.  For example
           foo@VER_1.  If the version is the default version to be used
           when resolving unversioned references to the symbol then it
           is displayed as a suffix preceded by two @ characters.  For
           example foo@@VER_2.  By default, symbol version information
           is displayed.

       --target=bfdname
           Specify an object code format other than your system's
           default format.

       @file
           Read command-line options from file.  The options read are
           inserted in place of the original @file option.  If file does
           not exist, or cannot be read, then the option will be treated
           literally, and not removed.

           Options in file are separated by whitespace.  A whitespace
           character may be included in an option by surrounding the
           entire option in either single or double quotes.  Any
           character (including a backslash) may be included by
           prefixing the character to be included with a backslash.  The
           file may itself contain additional @file options; any such
           options will be processed recursively.
SEE ALSO
       ar(1), objdump(1), ranlib(1), and the Info entries for binutils.
COPYRIGHT
       Copyright (c) 1991-2024 Free Software Foundation, Inc.

       Permission is granted to copy, distribute and/or modify this
       document under the terms of the GNU Free Documentation License,
       Version 1.3 or any later version published by the Free Software
       Foundation; with no Invariant Sections, with no Front-Cover
       Texts, and with no Back-Cover Texts.  A copy of the license is
       included in the section entitled "GNU Free Documentation
       License".
COLOPHON
       This page is part of the binutils (a collection of tools for
       working with executable binaries) project.  Information about the
       project can be found at http://www.gnu.org/software/binutils/.
       If you have a bug report for this manual page, see
       http://sourceware.org/bugzilla/enter_bug.cgi?product=binutils.
       This page was obtained from the tarball binutils-2.42.tar.gz
       fetched from https://ftp.gnu.org/gnu/binutils/ on 2024-06-14.
       If you discover any rendering problems in this HTML version of
       the page, or you believe there is a better or more up-to-date
       source for the page, or you have corrections or improvements to
       the information in this COLOPHON (which is not part of the
       original manual page), send a mail to man-pages@man7.org

binutils-2.42                  2024-06-14                          NM(1)\end{lstlisting}
}}
\endinput  %  ==  ==  ==  ==  ==  ==  ==  ==  ==

\subsection{\refNm: List Symbols From Object Files}

{\tiny{
\begin{lstlisting}[language=bash]
NAME
       nm - list symbols from object files
SYNOPSIS
       nm [-A|-o|--print-file-name]
          [-a|--debug-syms]
          [-B|--format=bsd]
          [-C|--demangle[=style]]
          [-D|--dynamic]
          [-fformat|--format=format]
          [-g|--extern-only]
          [-h|--help]
          [--ifunc-chars=CHARS]
          [-j|--format=just-symbols]
          [-l|--line-numbers] [--inlines]
          [-n|-v|--numeric-sort]
          [-P|--portability]
          [-p|--no-sort]
          [-r|--reverse-sort]
          [-S|--print-size]
          [-s|--print-armap]
          [-t radix|--radix=radix]
          [-u|--undefined-only]
          [-U|--defined-only]
          [-V|--version]
          [-W|--no-weak]
          [-X 32_64]
          [--no-demangle]
          [--no-recurse-limit|--recurse-limit]]
          [--plugin name]
          [--size-sort]
          [--special-syms]
          [--synthetic]
          [--target=bfdname]
          [--unicode=method]
          [--with-symbol-versions]
          [--without-symbol-versions]
          [objfile...]
DESCRIPTION
       GNU nm lists the symbols from object files objfile....  If no
       object files are listed as arguments, nm assumes the file a.out.

       For each symbol, nm shows:

       *   The symbol value, in the radix selected by options (see
           below), or hexadecimal by default.

       *   The symbol type.  At least the following types are used;
           others are, as well, depending on the object file format.  If
           lowercase, the symbol is usually local; if uppercase, the
           symbol is global (external).  There are however a few
           lowercase symbols that are shown for special global symbols
           ("u", "v" and "w").

           "A" The symbol's value is absolute, and will not be changed
               by further linking.

           "B"
           "b" The symbol is in the BSS data section.  This section
               typically contains zero-initialized or uninitialized
               data, although the exact behavior is system dependent.

           "C"
           "c" The symbol is common.  Common symbols are uninitialized
               data.  When linking, multiple common symbols may appear
               with the same name.  If the symbol is defined anywhere,
               the common symbols are treated as undefined references.
               The lower case c character is used when the symbol is in
               a special section for small commons.

           "D"
           "d" The symbol is in the initialized data section.

           "G"
           "g" The symbol is in an initialized data section for small
               objects.  Some object file formats permit more efficient
               access to small data objects, such as a global int
               variable as opposed to a large global array.

           "i" For PE format files this indicates that the symbol is in
               a section specific to the implementation of DLLs.

               For ELF format files this indicates that the symbol is an
               indirect function.  This is a GNU extension to the
               standard set of ELF symbol types.  It indicates a symbol
               which if referenced by a relocation does not evaluate to
               its address, but instead must be invoked at runtime.  The
               runtime execution will then return the value to be used
               in the relocation.

               Note - the actual symbols display for GNU indirect
               symbols is controlled by the --ifunc-chars command line
               option.  If this option has been provided then the first
               character in the string will be used for global indirect
               function symbols.  If the string contains a second
               character then that will be used for local indirect
               function symbols.

           "I" The symbol is an indirect reference to another symbol.

           "N" The symbol is a debugging symbol.

           "n" The symbol is in a non-data, non-code, non-debug read-
               only section.

           "p" The symbol is in a stack unwind section.

           "R"
           "r" The symbol is in a read only data section.

           "S"
           "s" The symbol is in an uninitialized or zero-initialized
               data section for small objects.

           "T"
           "t" The symbol is in the text (code) section.

           "U" The symbol is undefined.

           "u" The symbol is a unique global symbol.  This is a GNU
               extension to the standard set of ELF symbol bindings.
               For such a symbol the dynamic linker will make sure that
               in the entire process there is just one symbol with this
               name and type in use.

           "V"
           "v" The symbol is a weak object.  When a weak defined symbol
               is linked with a normal defined symbol, the normal
               defined symbol is used with no error.  When a weak
               undefined symbol is linked and the symbol is not defined,
               the value of the weak symbol becomes zero with no error.
               On some systems, uppercase indicates that a default value
               has been specified.

           "W"
           "w" The symbol is a weak symbol that has not been
               specifically tagged as a weak object symbol.  When a weak
               defined symbol is linked with a normal defined symbol,
               the normal defined symbol is used with no error.  When a
               weak undefined symbol is linked and the symbol is not
               defined, the value of the symbol is determined in a
               system-specific manner without error.  On some systems,
               uppercase indicates that a default value has been
               specified.

           "-" The symbol is a stabs symbol in an a.out object file.  In
               this case, the next values printed are the stabs other
               field, the stabs desc field, and the stab type.  Stabs
               symbols are used to hold debugging information.

           "?" The symbol type is unknown, or object file format
               specific.

       *   The symbol name.  If a symbol has version information
           associated with it, then the version information is displayed
           as well.  If the versioned symbol is undefined or hidden from
           linker, the version string is displayed as a suffix to the
           symbol name, preceded by an @ character.  For example
           foo@VER_1.  If the version is the default version to be used
           when resolving unversioned references to the symbol, then it
           is displayed as a suffix preceded by two @ characters.  For
           example foo@@VER_2.
OPTIONS
       The long and short forms of options, shown here as alternatives,
       are equivalent.

       -A
       -o
       --print-file-name
           Precede each symbol by the name of the input file (or archive
           member) in which it was found, rather than identifying the
           input file once only, before all of its symbols.

       -a
       --debug-syms
           Display all symbols, even debugger-only symbols; normally
           these are not listed.

       -B  The same as --format=bsd (for compatibility with the MIPS
           nm).

       -C
       --demangle[=style]
           Decode (demangle) low-level symbol names into user-level
           names.  Besides removing any initial underscore prepended by
           the system, this makes C++ function names readable. Different
           compilers have different mangling styles. The optional
           demangling style argument can be used to choose an
           appropriate demangling style for your compiler.

       --no-demangle
           Do not demangle low-level symbol names.  This is the default.

       --recurse-limit
       --no-recurse-limit
       --recursion-limit
       --no-recursion-limit
           Enables or disables a limit on the amount of recursion
           performed whilst demangling strings.  Since the name mangling
           formats allow for an infinite level of recursion it is
           possible to create strings whose decoding will exhaust the
           amount of stack space available on the host machine,
           triggering a memory fault.  The limit tries to prevent this
           from happening by restricting recursion to 2048 levels of
           nesting.

           The default is for this limit to be enabled, but disabling it
           may be necessary in order to demangle truly complicated
           names.  Note however that if the recursion limit is disabled
           then stack exhaustion is possible and any bug reports about
           such an event will be rejected.

       -D
       --dynamic
           Display the dynamic symbols rather than the normal symbols.
           This is only meaningful for dynamic objects, such as certain
           types of shared libraries.

       -f format
       --format=format
           Use the output format format, which can be "bsd", "sysv",
           "posix" or "just-symbols".  The default is "bsd".  Only the
           first character of format is significant; it can be either
           upper or lower case.

       -g
       --extern-only
           Display only external symbols.

       -h
       --help
           Show a summary of the options to nm and exit.

       --ifunc-chars=CHARS
           When display GNU indirect function symbols nm will default to
           using the "i" character for both local indirect functions and
           global indirect functions.  The --ifunc-chars option allows
           the user to specify a string containing one or two
           characters. The first character will be used for global
           indirect function symbols and the second character, if
           present, will be used for local indirect function symbols.

       j   The same as --format=just-symbols.

       -l
       --line-numbers
           For each symbol, use debugging information to try to find a
           filename and line number.  For a defined symbol, look for the
           line number of the address of the symbol.  For an undefined
           symbol, look for the line number of a relocation entry which
           refers to the symbol.  If line number information can be
           found, print it after the other symbol information.

       --inlines
           When option -l is active, if the address belongs to a
           function that was inlined, then this option causes the source
           information for all enclosing scopes back to the first non-
           inlined function to be printed as well.  For example, if
           "main" inlines "callee1" which inlines "callee2", and address
           is from "callee2", the source information for "callee1" and
           "main" will also be printed.

       -n
       -v
       --numeric-sort
           Sort symbols numerically by their addresses, rather than
           alphabetically by their names.

       -p
       --no-sort
           Do not bother to sort the symbols in any order; print them in
           the order encountered.

       -P
       --portability
           Use the POSIX.2 standard output format instead of the default
           format.  Equivalent to -f posix.

       -r
       --reverse-sort
           Reverse the order of the sort (whether numeric or
           alphabetic); let the last come first.

       -S
       --print-size
           Print both value and size of defined symbols for the "bsd"
           output style.  This option has no effect for object formats
           that do not record symbol sizes, unless --size-sort is also
           used in which case a calculated size is displayed.

       -s
       --print-armap
           When listing symbols from archive members, include the index:
           a mapping (stored in the archive by ar or ranlib) of which
           modules contain definitions for which names.

       -t radix
       --radix=radix
           Use radix as the radix for printing the symbol values.  It
           must be d for decimal, o for octal, or x for hexadecimal.

       -u
       --undefined-only
           Display only undefined symbols (those external to each object
           file).  By default both defined and undefined symbols are
           displayed.

       -U
       --defined-only
           Display only defined symbols for each object file.  By
           default both defined and undefined symbols are displayed.

       -V
       --version
           Show the version number of nm and exit.

       -X  This option is ignored for compatibility with the AIX version
           of nm.  It takes one parameter which must be the string
           32_64.  The default mode of AIX nm corresponds to -X 32,
           which is not supported by GNU nm.

       --plugin name
           Load the plugin called name to add support for extra target
           types.  This option is only available if the toolchain has
           been built with plugin support enabled.

           If --plugin is not provided, but plugin support has been
           enabled then nm iterates over the files in
           ${libdir}/bfd-plugins in alphabetic order and the first
           plugin that claims the object in question is used.

           Please note that this plugin search directory is not the one
           used by ld's -plugin option.  In order to make nm use the
           linker plugin it must be copied into the
           ${libdir}/bfd-plugins directory.  For GCC based compilations
           the linker plugin is called liblto_plugin.so.0.0.0.  For
           Clang based compilations it is called LLVMgold.so.  The GCC
           plugin is always backwards compatible with earlier versions,
           so it is sufficient to just copy the newest one.

       --size-sort
           Sort symbols by size.  For ELF objects symbol sizes are read
           from the ELF, for other object types the symbol sizes are
           computed as the difference between the value of the symbol
           and the value of the symbol with the next higher value.  If
           the "bsd" output format is used the size of the symbol is
           printed, rather than the value, and -S must be used in order
           both size and value to be printed.

           Note - this option does not work if --undefined-only has been
           enabled as undefined symbols have no size.

       --special-syms
           Display symbols which have a target-specific special meaning.
           These symbols are usually used by the target for some special
           processing and are not normally helpful when included in the
           normal symbol lists.  For example for ARM targets this option
           would skip the mapping symbols used to mark transitions
           between ARM code, THUMB code and data.

       --synthetic
           Include synthetic symbols in the output.  These are special
           symbols created by the linker for various purposes.  They are
           not shown by default since they are not part of the binary's
           original source code.

       --unicode=[default|invalid|locale|escape|hex|highlight]
           Controls the display of UTF-8 encoded multibyte characters in
           strings.  The default (--unicode=default) is to give them no
           special treatment.  The --unicode=locale option displays the
           sequence in the current locale, which may or may not support
           them.  The options --unicode=hex and --unicode=invalid
           display them as hex byte sequences enclosed by either angle
           brackets or curly braces.

           The --unicode=escape option displays them as escape sequences
           (\uxxxx) and the --unicode=highlight option displays them as
           escape sequences highlighted in red (if supported by the
           output device).  The colouring is intended to draw attention
           to the presence of unicode sequences where they might not be
           expected.

       -W
       --no-weak
           Do not display weak symbols.

       --with-symbol-versions
       --without-symbol-versions
           Enables or disables the display of symbol version
           information.  The version string is displayed as a suffix to
           the symbol name, preceded by an @ character.  For example
           foo@VER_1.  If the version is the default version to be used
           when resolving unversioned references to the symbol then it
           is displayed as a suffix preceded by two @ characters.  For
           example foo@@VER_2.  By default, symbol version information
           is displayed.

       --target=bfdname
           Specify an object code format other than your system's
           default format.

       @file
           Read command-line options from file.  The options read are
           inserted in place of the original @file option.  If file does
           not exist, or cannot be read, then the option will be treated
           literally, and not removed.

           Options in file are separated by whitespace.  A whitespace
           character may be included in an option by surrounding the
           entire option in either single or double quotes.  Any
           character (including a backslash) may be included by
           prefixing the character to be included with a backslash.  The
           file may itself contain additional @file options; any such
           options will be processed recursively.
SEE ALSO
       ar(1), objdump(1), ranlib(1), and the Info entries for binutils.
COPYRIGHT
       Copyright (c) 1991-2024 Free Software Foundation, Inc.

       Permission is granted to copy, distribute and/or modify this
       document under the terms of the GNU Free Documentation License,
       Version 1.3 or any later version published by the Free Software
       Foundation; with no Invariant Sections, with no Front-Cover
       Texts, and with no Back-Cover Texts.  A copy of the license is
       included in the section entitled "GNU Free Documentation
       License".
COLOPHON
       This page is part of the binutils (a collection of tools for
       working with executable binaries) project.  Information about the
       project can be found at http://www.gnu.org/software/binutils/.
       If you have a bug report for this manual page, see
       http://sourceware.org/bugzilla/enter_bug.cgi?product=binutils.
       This page was obtained from the tarball binutils-2.42.tar.gz
       fetched from https://ftp.gnu.org/gnu/binutils/ on 2024-06-14.
       If you discover any rendering problems in this HTML version of
       the page, or you believe there is a better or more up-to-date
       source for the page, or you have corrections or improvements to
       the information in this COLOPHON (which is not part of the
       original manual page), send a mail to man-pages@man7.org

binutils-2.42                  2024-06-14                          NM(1)\end{lstlisting}
}}
\endinput  %  ==  ==  ==  ==  ==  ==  ==  ==  ==
