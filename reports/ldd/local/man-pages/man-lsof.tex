% % % % \input{./components/man/man-ldd}
\subsection{\refLdd: Print Shared Object Dependencies}

{\tiny{
\begin{lstlisting}[language=bash]
NAME
       ldd - print shared object dependencies
SYNOPSIS
       ldd [option]... file...
DESCRIPTION
       ldd prints the shared objects (shared libraries) required by each
       program or shared object specified on the command line.  An
       example of its use and output is the following:

           $ ldd /bin/ls
               linux-vdso.so.1 (0x00007ffcc3563000)
               libselinux.so.1 => /lib64/libselinux.so.1 (0x00007f87e5459000)
               libcap.so.2 => /lib64/libcap.so.2 (0x00007f87e5254000)
               libc.so.6 => /lib64/libc.so.6 (0x00007f87e4e92000)
               libpcre.so.1 => /lib64/libpcre.so.1 (0x00007f87e4c22000)
               libdl.so.2 => /lib64/libdl.so.2 (0x00007f87e4a1e000)
               /lib64/ld-linux-x86-64.so.2 (0x00005574bf12e000)
               libattr.so.1 => /lib64/libattr.so.1 (0x00007f87e4817000)
               libpthread.so.0 => /lib64/libpthread.so.0 (0x00007f87e45fa000)

       In the usual case, ldd invokes the standard dynamic linker (see
       ld.so(8)) with the LD_TRACE_LOADED_OBJECTS environment variable
       set to 1.  This causes the dynamic linker to inspect the
       program's dynamic dependencies, and find (according to the rules
       described in ld.so(8)) and load the objects that satisfy those
       dependencies.  For each dependency, ldd displays the location of
       the matching object and the (hexadecimal) address at which it is
       loaded.  (The linux-vdso and ld-linux shared dependencies are
       special; see vdso(7) and ld.so(8).)

   Security
       Be aware that in some circumstances (e.g., where the program
       specifies an ELF interpreter other than ld-linux.so), some
       versions of ldd may attempt to obtain the dependency information
       by attempting to directly execute the program, which may lead to
       the execution of whatever code is defined in the program's ELF
       interpreter, and perhaps to execution of the program itself.
       (Before glibc 2.27, the upstream ldd implementation did this for
       example, although most distributions provided a modified version
       that did not.)

       Thus, you should never employ ldd on an untrusted executable,
       since this may result in the execution of arbitrary code.  A
       safer alternative when dealing with untrusted executables is:

           $ objdump -p /path/to/program | grep NEEDED

       Note, however, that this alternative shows only the direct
       dependencies of the executable, while ldd shows the entire
       dependency tree of the executable.
OPTIONS
       --version
              Print the version number of ldd.

       --verbose
       -v     Print all information, including, for example, symbol
              versioning information.

       --unused
       -u     Print unused direct dependencies.  (Since glibc 2.3.4.)

       --data-relocs
       -d     Perform relocations and report any missing objects (ELF
              only).

       --function-relocs
       -r     Perform relocations for both data objects and functions,
              and report any missing objects or functions (ELF only).

       --help Usage information.
BUGS
       ldd does not work on a.out shared libraries.

       ldd does not work with some extremely old a.out programs which
       were built before ldd support was added to the compiler releases.
       If you use ldd on one of these programs, the program will attempt
       to run with argc = 0 and the results will be unpredictable.
SEE ALSO
       pldd(1), sprof(1), ld.so(8), ldconfig(8)
COLOPHON
       This page is part of the man-pages (Linux kernel and C library
       user-space interface documentation) project.  Information about
       the project can be found at 
       https://www.kernel.org/doc/man-pages/.  If you have a bug report
       for this manual page, see
       https://git.kernel.org/pub/scm/docs/man-pages/man-pages.git/tree/CONTRIBUTING.
       This page was obtained from the tarball man-pages-6.9.1.tar.gz
       fetched from
       https://mirrors.edge.kernel.org/pub/linux/docs/man-pages/ on
       2024-06-26.  If you discover any rendering problems in this HTML
       version of the page, or you believe there is a better or more up-
       to-date source for the page, or you have corrections or
       improvements to the information in this COLOPHON (which is not
       part of the original manual page), send a mail to
       man-pages@man7.org

Linux man-pages 6.9.1          2024-05-02                         ldd(1)
\end{lstlisting}
}}
\endinput  %  ==  ==  ==  ==  ==  ==  ==  ==  ==

\subsection{\refLdd: Print Shared Object Dependencies}

{\tiny{
\begin{lstlisting}[language=bash]
NAME
       ldd - print shared object dependencies
SYNOPSIS
       ldd [option]... file...
DESCRIPTION
       ldd prints the shared objects (shared libraries) required by each
       program or shared object specified on the command line.  An
       example of its use and output is the following:

           $ ldd /bin/ls
               linux-vdso.so.1 (0x00007ffcc3563000)
               libselinux.so.1 => /lib64/libselinux.so.1 (0x00007f87e5459000)
               libcap.so.2 => /lib64/libcap.so.2 (0x00007f87e5254000)
               libc.so.6 => /lib64/libc.so.6 (0x00007f87e4e92000)
               libpcre.so.1 => /lib64/libpcre.so.1 (0x00007f87e4c22000)
               libdl.so.2 => /lib64/libdl.so.2 (0x00007f87e4a1e000)
               /lib64/ld-linux-x86-64.so.2 (0x00005574bf12e000)
               libattr.so.1 => /lib64/libattr.so.1 (0x00007f87e4817000)
               libpthread.so.0 => /lib64/libpthread.so.0 (0x00007f87e45fa000)

       In the usual case, ldd invokes the standard dynamic linker (see
       ld.so(8)) with the LD_TRACE_LOADED_OBJECTS environment variable
       set to 1.  This causes the dynamic linker to inspect the
       program's dynamic dependencies, and find (according to the rules
       described in ld.so(8)) and load the objects that satisfy those
       dependencies.  For each dependency, ldd displays the location of
       the matching object and the (hexadecimal) address at which it is
       loaded.  (The linux-vdso and ld-linux shared dependencies are
       special; see vdso(7) and ld.so(8).)

   Security
       Be aware that in some circumstances (e.g., where the program
       specifies an ELF interpreter other than ld-linux.so), some
       versions of ldd may attempt to obtain the dependency information
       by attempting to directly execute the program, which may lead to
       the execution of whatever code is defined in the program's ELF
       interpreter, and perhaps to execution of the program itself.
       (Before glibc 2.27, the upstream ldd implementation did this for
       example, although most distributions provided a modified version
       that did not.)

       Thus, you should never employ ldd on an untrusted executable,
       since this may result in the execution of arbitrary code.  A
       safer alternative when dealing with untrusted executables is:

           $ objdump -p /path/to/program | grep NEEDED

       Note, however, that this alternative shows only the direct
       dependencies of the executable, while ldd shows the entire
       dependency tree of the executable.
OPTIONS
       --version
              Print the version number of ldd.

       --verbose
       -v     Print all information, including, for example, symbol
              versioning information.

       --unused
       -u     Print unused direct dependencies.  (Since glibc 2.3.4.)

       --data-relocs
       -d     Perform relocations and report any missing objects (ELF
              only).

       --function-relocs
       -r     Perform relocations for both data objects and functions,
              and report any missing objects or functions (ELF only).

       --help Usage information.
BUGS
       ldd does not work on a.out shared libraries.

       ldd does not work with some extremely old a.out programs which
       were built before ldd support was added to the compiler releases.
       If you use ldd on one of these programs, the program will attempt
       to run with argc = 0 and the results will be unpredictable.
SEE ALSO
       pldd(1), sprof(1), ld.so(8), ldconfig(8)
COLOPHON
       This page is part of the man-pages (Linux kernel and C library
       user-space interface documentation) project.  Information about
       the project can be found at 
       https://www.kernel.org/doc/man-pages/.  If you have a bug report
       for this manual page, see
       https://git.kernel.org/pub/scm/docs/man-pages/man-pages.git/tree/CONTRIBUTING.
       This page was obtained from the tarball man-pages-6.9.1.tar.gz
       fetched from
       https://mirrors.edge.kernel.org/pub/linux/docs/man-pages/ on
       2024-06-26.  If you discover any rendering problems in this HTML
       version of the page, or you believe there is a better or more up-
       to-date source for the page, or you have corrections or
       improvements to the information in this COLOPHON (which is not
       part of the original manual page), send a mail to
       man-pages@man7.org

Linux man-pages 6.9.1          2024-05-02                         ldd(1)
\end{lstlisting}
}}
\endinput  %  ==  ==  ==  ==  ==  ==  ==  ==  ==

\subsection{\refLdd: Print Shared Object Dependencies}

{\tiny{
\begin{lstlisting}[language=bash]
NAME
       ldd - print shared object dependencies
SYNOPSIS
       ldd [option]... file...
DESCRIPTION
       ldd prints the shared objects (shared libraries) required by each
       program or shared object specified on the command line.  An
       example of its use and output is the following:

           $ ldd /bin/ls
               linux-vdso.so.1 (0x00007ffcc3563000)
               libselinux.so.1 => /lib64/libselinux.so.1 (0x00007f87e5459000)
               libcap.so.2 => /lib64/libcap.so.2 (0x00007f87e5254000)
               libc.so.6 => /lib64/libc.so.6 (0x00007f87e4e92000)
               libpcre.so.1 => /lib64/libpcre.so.1 (0x00007f87e4c22000)
               libdl.so.2 => /lib64/libdl.so.2 (0x00007f87e4a1e000)
               /lib64/ld-linux-x86-64.so.2 (0x00005574bf12e000)
               libattr.so.1 => /lib64/libattr.so.1 (0x00007f87e4817000)
               libpthread.so.0 => /lib64/libpthread.so.0 (0x00007f87e45fa000)

       In the usual case, ldd invokes the standard dynamic linker (see
       ld.so(8)) with the LD_TRACE_LOADED_OBJECTS environment variable
       set to 1.  This causes the dynamic linker to inspect the
       program's dynamic dependencies, and find (according to the rules
       described in ld.so(8)) and load the objects that satisfy those
       dependencies.  For each dependency, ldd displays the location of
       the matching object and the (hexadecimal) address at which it is
       loaded.  (The linux-vdso and ld-linux shared dependencies are
       special; see vdso(7) and ld.so(8).)

   Security
       Be aware that in some circumstances (e.g., where the program
       specifies an ELF interpreter other than ld-linux.so), some
       versions of ldd may attempt to obtain the dependency information
       by attempting to directly execute the program, which may lead to
       the execution of whatever code is defined in the program's ELF
       interpreter, and perhaps to execution of the program itself.
       (Before glibc 2.27, the upstream ldd implementation did this for
       example, although most distributions provided a modified version
       that did not.)

       Thus, you should never employ ldd on an untrusted executable,
       since this may result in the execution of arbitrary code.  A
       safer alternative when dealing with untrusted executables is:

           $ objdump -p /path/to/program | grep NEEDED

       Note, however, that this alternative shows only the direct
       dependencies of the executable, while ldd shows the entire
       dependency tree of the executable.
OPTIONS
       --version
              Print the version number of ldd.

       --verbose
       -v     Print all information, including, for example, symbol
              versioning information.

       --unused
       -u     Print unused direct dependencies.  (Since glibc 2.3.4.)

       --data-relocs
       -d     Perform relocations and report any missing objects (ELF
              only).

       --function-relocs
       -r     Perform relocations for both data objects and functions,
              and report any missing objects or functions (ELF only).

       --help Usage information.
BUGS
       ldd does not work on a.out shared libraries.

       ldd does not work with some extremely old a.out programs which
       were built before ldd support was added to the compiler releases.
       If you use ldd on one of these programs, the program will attempt
       to run with argc = 0 and the results will be unpredictable.
SEE ALSO
       pldd(1), sprof(1), ld.so(8), ldconfig(8)
COLOPHON
       This page is part of the man-pages (Linux kernel and C library
       user-space interface documentation) project.  Information about
       the project can be found at 
       https://www.kernel.org/doc/man-pages/.  If you have a bug report
       for this manual page, see
       https://git.kernel.org/pub/scm/docs/man-pages/man-pages.git/tree/CONTRIBUTING.
       This page was obtained from the tarball man-pages-6.9.1.tar.gz
       fetched from
       https://mirrors.edge.kernel.org/pub/linux/docs/man-pages/ on
       2024-06-26.  If you discover any rendering problems in this HTML
       version of the page, or you believe there is a better or more up-
       to-date source for the page, or you have corrections or
       improvements to the information in this COLOPHON (which is not
       part of the original manual page), send a mail to
       man-pages@man7.org

Linux man-pages 6.9.1          2024-05-02                         ldd(1)
\end{lstlisting}
}}
\endinput  %  ==  ==  ==  ==  ==  ==  ==  ==  ==

\subsection{\refLsof: Show Open Files}

{\tiny{
\begin{lstlisting}[language=bash]
NAME
       lsof - list open files
SYNOPSIS
       lsof [ -?abChlnNOPRtUvVX ] [ -A A ] [ -c c ] [ +c c ] [ +|-d d ]
       [ +|-D D ] [ +|-e s ] [ +|-E ] [ +|-f [cfgGn] ] [ -F [f] ] [ -g
       [s] ] [ -i [i] ] [ -k k ] [ -K k ] [ +|-L [l] ] [ +|-m m ] [ +|-M
       ] [ -o [o] ] [ -p s ] [ +|-r [t[m<fmt>]] ] [ -s [p:s] ] [ -S [t]
       ] [ -T [t] ] [ -u s ] [ +|-w ] [ -x [fl] ] [ -z [z] ] [ -Z [Z] ]
       [ -- ] [names]
DESCRIPTION
       Lsof revision 4.91 lists on its standard output file information
       about files opened by processes for the following UNIX dialects:

            Apple Darwin 9 and Mac OS X 10.[567]
            FreeBSD 8.[234], 9.0 and 1[012].0 for AMD64-based systems
            Linux 2.1.72 and above for x86-based systems
            Solaris 9, 10 and 11

       (See the DISTRIBUTION section of this manual page for information
       on how to obtain the latest lsof revision.)

       An open file may be a regular file, a directory, a block special
       file, a character special file, an executing text reference, a
       library, a stream or a network file (Internet socket, NFS file or
       UNIX domain socket.)  A specific file or all the files in a file
       system may be selected by path.

       Instead of a formatted display, lsof will produce output that can
       be parsed by other programs.  See the -F, option description, and
       the OUTPUT FOR OTHER PROGRAMS section for more information.

       In addition to producing a single output list, lsof will run in
       repeat mode.  In repeat mode it will produce output, delay, then
       repeat the output operation until stopped with an interrupt or
       quit signal.  See the +|-r [t[m<fmt>]] option description for
       more information.
OPTIONS
       In the absence of any options, lsof lists all open files
       belonging to all active processes.

       If any list request option is specified, other list requests must
       be specifically requested - e.g., if -U is specified for the
       listing of UNIX socket files, NFS files won't be listed unless -N
       is also specified; or if a user list is specified with the -u
       option, UNIX domain socket files, belonging to users not in the
       list, won't be listed unless the -U option is also specified.

       Normally list options that are specifically stated are ORed -
       i.e., specifying the -i option without an address and the -ufoo
       option produces a listing of all network files OR files belonging
       to processes owned by user ``foo''.  The exceptions are:

       1) the `^' (negated) login name or user ID (UID), specified with
          the -u option;

       2) the `^' (negated) process ID (PID), specified with the -p
          option;

       3) the `^' (negated) process group ID (PGID), specified with the
          -g option;

       4) the `^' (negated) command, specified with the -c option;

       5) the (`^') negated TCP or UDP protocol state names, specified
          with the -s [p:s] option.

       Since they represent exclusions, they are applied without ORing
       or ANDing and take effect before any other selection criteria are
       applied.

       The -a option may be used to AND the selections.  For example,
       specifying -a, -U, and -ufoo produces a listing of only UNIX
       socket files that belong to processes owned by user ``foo''.

       Caution: the -a option causes all list selection options to be
       ANDed; it can't be used to cause ANDing of selected pairs of
       selection options by placing it between them, even though its
       placement there is acceptable.  Wherever -a is placed, it causes
       the ANDing of all selection options.

       Items of the same selection set - command names, file
       descriptors, network addresses, process identifiers, user
       identifiers, zone names, security contexts - are joined in a
       single ORed set and applied before the result participates in
       ANDing.  Thus, for example, specifying -i@aaa.bbb, -i@ccc.ddd,
       -a, and -ufff,ggg will select the listing of files that belong to
       either login ``fff'' OR ``ggg'' AND have network connections to
       either host aaa.bbb OR ccc.ddd.

       Options may be grouped together following a single prefix --
       e.g., the option set ``-a -b -C'' may be stated as -abC.
       However, since values are optional following +|-f, -F, -g, -i,
       +|-L, -o, +|-r, -s, -S, -T, -x and -z.  when you have no values
       for them be careful that the following character isn't ambiguous.
       For example, -Fn might represent the -F and -n options, or it
       might represent the n field identifier character following the -F
       option.  When ambiguity is possible, start a new option with a
       `-' character - e.g., ``-F -n''.  If the next option is a file
       name, follow the possibly ambiguous option with ``--'' - e.g.,
       ``-F -- name''.

       Either the `+' or the `-' prefix may be applied to a group of
       options.  Options that don't take on separate meanings for each
       prefix - e.g., -i - may be grouped under either prefix.  Thus,
       for example, ``+M -i'' may be stated as ``+Mi'' and the group
       means the same as the separate options.  Be careful of prefix
       grouping when one or more options in the group does take on
       separate meanings under different prefixes - e.g., +|-M; ``-iM''
       is not the same request as ``-i +M''.  When in doubt, use
       separate options with appropriate prefixes.

       -? -h  These two equivalent options select a usage (help) output
              list.  Lsof displays a shortened form of this output when
              it detects an error in the options supplied to it, after
              it has displayed messages explaining each error.  (Escape
              the `?' character as your shell requires.)

       -a     causes list selection options to be ANDed, as described
              above.

       -A A   is available on systems configured for AFS whose AFS
              kernel code is implemented via dynamic modules.  It allows
              the lsof user to specify A as an alternate name list file
              where the kernel addresses of the dynamic modules might be
              found.  See the lsof FAQ (The FAQ section gives its
              location.)  for more information about dynamic modules,
              their symbols, and how they affect lsof.

       -b     causes lsof to avoid kernel functions that might block -
              lstat(2), readlink(2), and stat(2).

              See the BLOCKS AND TIMEOUTS and AVOIDING KERNEL BLOCKS
              sections for information on using this option.

       -c c   selects the listing of files for processes executing the
              command that begins with the characters of c.  Multiple
              commands may be specified, using multiple -c options.
              They are joined in a single ORed set before participating
              in AND option selection.

              If c begins with a `^', then the following characters
              specify a command name whose processes are to be ignored
              (excluded.)

              If c begins and ends with a slash ('/'), the characters
              between the slashes are interpreted as a regular
              expression.  Shell meta-characters in the regular
              expression must be quoted to prevent their interpretation
              by the shell.  The closing slash may be followed by these
              modifiers:

                   b    the regular expression is a basic one.
                   i    ignore the case of letters.
                   x    the regular expression is an extended one
                        (default).

              See the lsof FAQ (The FAQ section gives its location.)
              for more information on basic and extended regular
              expressions.

              The simple command specification is tested first.  If that
              test fails, the command regular expression is applied.  If
              the simple command test succeeds, the command regular
              expression test isn't made.  This may result in ``no
              command found for regex:'' messages when lsof's -V option
              is specified.

       +c w   defines the maximum number of initial characters of the
              name, supplied by the UNIX dialect, of the UNIX command
              associated with a process to be printed in the COMMAND
              column.  (The lsof default is nine.)

              Note that many UNIX dialects do not supply all command
              name characters to lsof in the files and structures from
              which lsof obtains command name.  Often dialects limit the
              number of characters supplied in those sources.  For
              example, Linux 2.4.27 and Solaris 9 both limit command
              name length to 16 characters.

              If w is zero ('0'), all command characters supplied to
              lsof by the UNIX dialect will be printed.

              If w is less than the length of the column title,
              ``COMMAND'', it will be raised to that length.

       -C     disables the reporting of any path name components from
              the kernel's name cache.  See the KERNEL NAME CACHE
              section for more information.

       +d s   causes lsof to search for all open instances of directory
              s and the files and directories it contains at its top
              level.  +d does NOT descend the directory tree, rooted at
              s.  The +D D option may be used to request a full-descent
              directory tree search, rooted at directory D.

              Processing of the +d option does not follow symbolic links
              within s unless the -x or -x  l option is also specified.
              Nor does it search for open files on file system mount
              points on subdirectories of s unless the -x or -x  f
              option is also specified.

              Note: the authority of the user of this option limits it
              to searching for files that the user has permission to
              examine with the system stat(2) function.

       -d s   specifies a list of file descriptors (FDs) to exclude from
              or include in the output listing.  The file descriptors
              are specified in the comma-separated set s - e.g.,
              ``cwd,1,3'', ``^6,^2''.  (There should be no spaces in the
              set.)

              The list is an exclusion list if all entries of the set
              begin with `^'.  It is an inclusion list if no entry
              begins with `^'.  Mixed lists are not permitted.

              A file descriptor number range may be in the set as long
              as neither member is empty, both members are numbers, and
              the ending member is larger than the starting one - e.g.,
              ``0-7'' or ``3-10''.  Ranges may be specified for
              exclusion if they have the `^' prefix - e.g., ``^0-7''
              excludes all file descriptors 0 through 7.

              Multiple file descriptor numbers are joined in a single
              ORed set before participating in AND option selection.

              When there are exclusion and inclusion members in the set,
              lsof reports them as errors and exits with a non-zero
              return code.

              See the description of File Descriptor (FD) output values
              in the OUTPUT section for more information on file
              descriptor names.

       +D D   causes lsof to search for all open instances of directory
              D and all the files and directories it contains to its
              complete depth.

              Processing of the +D option does not follow symbolic links
              within D unless the -x or -x  l option is also specified.
              Nor does it search for open files on file system mount
              points on subdirectories of D unless the -x or -x  f
              option is also specified.

              Note: the authority of the user of this option limits it
              to searching for files that the user has permission to
              examine with the system stat(2) function.

              Further note: lsof may process this option slowly and
              require a large amount of dynamic memory to do it.  This
              is because it must descend the entire directory tree,
              rooted at D, calling stat(2) for each file and directory,
              building a list of all the files it finds, and searching
              that list for a match with every open file.  When
              directory D is large, these steps can take a long time, so
              use this option prudently.

       -D D   directs lsof's use of the device cache file.  The use of
              this option is sometimes restricted.  See the DEVICE CACHE
              FILE section and the sections that follow it for more
              information on this option.

              -D must be followed by a function letter; the function
              letter may optionally be followed by a path name.  Lsof
              recognizes these function letters:

                   ? - report device cache file paths
                   b - build the device cache file
                   i - ignore the device cache file
                   r - read the device cache file
                   u - read and update the device cache file

              The b, r, and u functions, accompanied by a path name, are
              sometimes restricted.  When these functions are
              restricted, they will not appear in the description of the
              -D option that accompanies -h or -?  option output.  See
              the DEVICE CACHE FILE section and the sections that follow
              it for more information on these functions and when
              they're restricted.

              The ?  function reports the read-only and write paths that
              lsof can use for the device cache file, the names of any
              environment variables whose values lsof will examine when
              forming the device cache file path, and the format for the
              personal device cache file path.  (Escape the `?'
              character as your shell requires.)

              When available, the b, r, and u functions may be followed
              by the device cache file's path.  The standard default is
              .lsof_hostname in the home directory of the real user ID
              that executes lsof, but this could have been changed when
              lsof was configured and compiled.  (The output of the -h
              and -?  options show the current default prefix - e.g.,
              ``.lsof''.)  The suffix, hostname, is the first component
              of the host's name returned by gethostname(2).

              When available, the b function directs lsof to build a new
              device cache file at the default or specified path.

              The i function directs lsof to ignore the default device
              cache file and obtain its information about devices via
              direct calls to the kernel.

              The r function directs lsof to read the device cache at
              the default or specified path, but prevents it from
              creating a new device cache file when none exists or the
              existing one is improperly structured.  The r function,
              when specified without a path name, prevents lsof from
              updating an incorrect or outdated device cache file, or
              creating a new one in its place.  The r function is always
              available when it is specified without a path name
              argument; it may be restricted by the permissions of the
              lsof process.

              When available, the u function directs lsof to read the
              device cache file at the default or specified path, if
              possible, and to rebuild it, if necessary.  This is the
              default device cache file function when no -D option has
              been specified.

       +|-e s exempts the file system whose path name is s from being
              subjected to kernel function calls that might block.  The
              +e option exempts stat(2), lstat(2) and most readlink(2)
              kernel function calls.  The -e option exempts only stat(2)
              and lstat(2) kernel function calls.  Multiple file systems
              may be specified with separate +|-e specifications and
              each may have readlink(2) calls exempted or not.

              This option is currently implemented only for Linux.

              CAUTION: this option can easily be mis-applied to other
              than the file system of interest, because it uses path
              name rather than the more reliable device and inode
              numbers.  (Device and inode numbers are acquired via the
              potentially blocking stat(2) kernel call and are thus not
              available, but see the +|-m m option as a possible
              alternative way to supply device numbers.)  Use this
              option with great care and fully specify the path name of
              the file system to be exempted.

              When open files on exempted file systems are reported, it
              may not be possible to obtain all their information.
              Therefore, some information columns will be blank, the
              characters ``UNKN'' preface the values in the TYPE column,
              and the applicable exemption option is added in
              parentheses to the end of the NAME column.  (Some device
              number information might be made available via the +|-m m
              option.)

       +|-E   +E specifies that Linux pipe, Linux UNIX socket and Linux
              pseudoterminal files should be displayed with endpoint
              information and the files of the endpoints should also be
              displayed.  Note: UNIX socket file endpoint information is
              only available when the compile flags line of -v output
              contains HASUXSOCKEPT, and psudoterminal endpoint
              information is only available when the compile flags line
              contains HASPTYEPT.

              Pipe endpoint information is displayed in the NAME column
              in the form ``PID,cmd,FDmode'', where PID is the endpoint
              process ID; cmd is the endpoint process command; FD is the
              endpoint file's descriptor; and mode is the endpoint
              file's access mode.

              Pseudoterminal endpoint information is displayed in the
              NAME column as ``->/dev/ptsmin PID,cmd,FDmode'' or
              ``PID,cmd,FDmode''.  The first form is for a master
              device; the second, for a slave device.  min is a slave
              device's minor device number; and PID, cmd, FD and mode
              are the same as with pipe endpoint information.  Note:
              psudoterminal endpoint information is only available when
              the compile flags line of -V output contains HASPTYEPT.

              UNIX socket file endpoint information is displayed in the
              NAME column in the form
              ``type=TYPE ->INO=INODE PID,cmd,FDmode'', where TYPE is
              the socket type; INODE is the i-node number of the
              connected socket; and PID, cmd, FD and mode are the same
              as with pipe endpoint information.  Note: UNIX socket file
              endpoint information is available only when the compile
              flags line of -v output contains HASUXSOCKEPT.

              Multiple occurrences of this information can appear in a
              file's NAME column.

              -E specfies that Linux pipe and Linux UNIX socket files
              should be displayed with endpoint information, but not the
              files of the endpoints.

       +|-f [cfgGn]
              f by itself clarifies how path name arguments are to be
              interpreted.  When followed by c, f, g, G, or n in any
              combination it specifies that the listing of kernel file
              structure information is to be enabled (`+') or inhibited
              (`-').

              Normally a path name argument is taken to be a file system
              name if it matches a mounted-on directory name reported by
              mount(8), or if it represents a block device, named in the
              mount output and associated with a mounted directory name.
              When +f is specified, all path name arguments will be
              taken to be file system names, and lsof will complain if
              any are not.  This can be useful, for example, when the
              file system name (mounted-on device) isn't a block device.
              This happens for some CD-ROM file systems.

              When -f is specified by itself, all path name arguments
              will be taken to be simple files.  Thus, for example, the
              ``-f -- /'' arguments direct lsof to search for open files
              with a `/' path name, not all open files in the `/' (root)
              file system.

              Be careful to make sure +f and -f are properly terminated
              and aren't followed by a character (e.g., of the file or
              file system name) that might be taken as a parameter.  For
              example, use ``--'' after +f and -f as in these examples.

                   $ lsof +f -- /file/system/name
                   $ lsof -f -- /file/name

              The listing of information from kernel file structures,
              requested with the +f [cfgGn] option form, is normally
              inhibited, and is not available in whole or part for some
              dialects - e.g., /proc-based Linux kernels below 2.6.22.
              When the prefix to f is a plus sign (`+'), these
              characters request file structure information:

                   c    file structure use count (not Linux)
                   f    file structure address (not Linux)
                   g    file flag abbreviations (Linux 2.6.22 and up)
                   G    file flags in hexadecimal (Linux 2.6.22 and up)
                   n    file structure node address (not Linux)

              When the prefix is minus (`-') the same characters disable
              the listing of the indicated values.

              File structure addresses, use counts, flags, and node
              addresses may be used to detect more readily identical
              files inherited by child processes and identical files in
              use by different processes.  Lsof column output can be
              sorted by output columns holding the values and listed to
              identify identical file use, or lsof field output can be
              parsed by an AWK or Perl post-filter script, or by a C
              program.

       -F f   specifies a character list, f, that selects the fields to
              be output for processing by another program, and the
              character that terminates each output field.  Each field
              to be output is specified with a single character in f.
              The field terminator defaults to NL, but may be changed to
              NUL (000).  See the OUTPUT FOR OTHER PROGRAMS section for
              a description of the field identification characters and
              the field output process.

              When the field selection character list is empty, all
              standard fields are selected (except the raw device field,
              security context and zone field for compatibility reasons)
              and the NL field terminator is used.

              When the field selection character list contains only a
              zero (`0'), all fields are selected (except the raw device
              field for compatibility reasons) and the NUL terminator
              character is used.

              Other combinations of fields and their associated field
              terminator character must be set with explicit entries in
              f, as described in the OUTPUT FOR OTHER PROGRAMS section.

              When a field selection character identifies an item lsof
              does not normally list - e.g., PPID, selected with -R -
              specification of the field character - e.g., ``-FR'' -
              also selects the listing of the item.

              When the field selection character list contains the
              single character `?', lsof will display a help list of the
              field identification characters.  (Escape the `?'
              character as your shell requires.)

       -g [s] excludes or selects the listing of files for the processes
              whose optional process group IDentification (PGID) numbers
              are in the comma-separated set s - e.g., ``123'' or
              ``123,^456''.  (There should be no spaces in the set.)

              PGID numbers that begin with `^' (negation) represent
              exclusions.

              Multiple PGID numbers are joined in a single ORed set
              before participating in AND option selection.  However,
              PGID exclusions are applied without ORing or ANDing and
              take effect before other selection criteria are applied.

              The -g option also enables the output display of PGID
              numbers.  When specified without a PGID set that's all it
              does.

       -i [i] selects the listing of files any of whose Internet address
              matches the address specified in i.  If no address is
              specified, this option selects the listing of all Internet
              and x.25 (HP-UX) network files.

              If -i4 or -i6 is specified with no following address, only
              files of the indicated IP version, IPv4 or IPv6, are
              displayed.  (An IPv6 specification may be used only if the
              dialects supports IPv6, as indicated by ``[46]'' and
              ``IPv[46]'' in lsof's -h or -?  output.)  Sequentially
              specifying -i4, followed by -i6 is the same as specifying
              -i, and vice-versa.  Specifying -i4, or -i6 after -i is
              the same as specifying -i4 or -i6 by itself.

              Multiple addresses (up to a limit of 100) may be specified
              with multiple -i options.  (A port number or service name
              range is counted as one address.)  They are joined in a
              single ORed set before participating in AND option
              selection.

              An Internet address is specified in the form (Items in
              square brackets are optional.):

              [46][protocol][@hostname|hostaddr][:service|port]

              where:
                   46 specifies the IP version, IPv4 or IPv6
                        that applies to the following address.
                        '6' may be be specified only if the UNIX
                        dialect supports IPv6.  If neither '4' nor
                        '6' is specified, the following address
                        applies to all IP versions.
                   protocol is a protocol name - TCP, UDP
                   hostname is an Internet host name.  Unless a
                        specific IP version is specified, open
                        network files associated with host names
                        of all versions will be selected.
                   hostaddr is a numeric Internet IPv4 address in
                        dot form; or an IPv6 numeric address in
                        colon form, enclosed in brackets, if the
                        UNIX dialect supports IPv6.  When an IP
                        version is selected, only its numeric
                        addresses may be specified.
                   service is an /etc/services name - e.g., smtp -
                        or a list of them.
                   port is a port number, or a list of them.

              IPv6 options may be used only if the UNIX dialect supports
              IPv6.  To see if the dialect supports IPv6, run lsof and
              specify the -h or -?  (help) option.  If the displayed
              description of the -i option contains ``[46]'' and
              ``IPv[46]'', IPv6 is supported.

              IPv4 host names and addresses may not be specified if
              network file selection is limited to IPv6 with -i 6.  IPv6
              host names and addresses may not be specified if network
              file selection is limited to IPv4 with -i 4.  When an open
              IPv4 network file's address is mapped in an IPv6 address,
              the open file's type will be IPv6, not IPv4, and its
              display will be selected by '6', not '4'.

              At least one address component - 4, 6, protocol, hostname,
              hostaddr, or service - must be supplied.  The `@'
              character, leading the host specification, is always
              required; as is the `:', leading the port specification.
              Specify either hostname or hostaddr.  Specify either
              service name list or port number list.  If a service name
              list is specified, the protocol may also need to be
              specified if the TCP, UDP and UDPLITE port numbers for the
              service name are different.  Use any case - lower or upper
              - for protocol.

              Service names and port numbers may be combined in a list
              whose entries are separated by commas and whose numeric
              range entries are separated by minus signs.  There may be
              no embedded spaces, and all service names must belong to
              the specified protocol.  Since service names may contain
              embedded minus signs, the starting entry of a range can't
              be a service name; it can be a port number, however.

              Here are some sample addresses:

                   -i6 - IPv6 only
                   TCP:25 - TCP and port 25
                   @1.2.3.4 - Internet IPv4 host address 1.2.3.4
                   @[3ffe:1ebc::1]:1234 - Internet IPv6 host address
                        3ffe:1ebc::1, port 1234
                   UDP:who - UDP who service port
                   TCP@lsof.itap:513 - TCP, port 513 and host name lsof.itap
                   tcp@foo:1-10,smtp,99 - TCP, ports 1 through 10,
                        service name smtp, port 99, host name foo
                   tcp@bar:1-smtp - TCP, ports 1 through smtp, host bar
                   :time - either TCP, UDP or UDPLITE time service port

       -K k   selects the listing of tasks (threads) of processes, on
              dialects where task (thread) reporting is supported.  (If
              help output - i.e., the output of the -h or -?  options -
              shows this option, then task (thread) reporting is
              supported by the dialect.)

              If -K is followed by a value, k, it must be ``i''.  That
              causes lsof to ignore tasks, particularly in the default,
              list-everything case when no other options are specified.

              When -K and -a are both specified on Linux, and the tasks
              of a main process are selected by other options, the main
              process will also be listed as though it were a task, but
              without a task ID.  (See the description of the TID column
              in the OUTPUT section.)

              Where the FreeBSD version supports threads, all threads
              will be listed with their IDs.

              In general threads and tasks inherit the files of the
              caller, but may close some and open others, so lsof always
              reports all the open files of threads and tasks.

       -k k   specifies a kernel name list file, k, in place of /vmunix,
              /mach, etc.  -k is not available under AIX on the IBM
              RISC/System 6000.

       -l     inhibits the conversion of user ID numbers to login names.
              It is also useful when login name lookup is working
              improperly or slowly.

       +|-L [l]
              enables (`+') or disables (`-') the listing of file link
              counts, where they are available - e.g., they aren't
              available for sockets, or most FIFOs and pipes.

              When +L is specified without a following number, all link
              counts will be listed.  When -L is specified (the
              default), no link counts will be listed.

              When +L is followed by a number, only files having a link
              count less than that number will be listed.  (No number
              may follow -L.)  A specification of the form ``+L1'' will
              select open files that have been unlinked.  A
              specification of the form ``+aL1 <file_system>'' will
              select unlinked open files on the specified file system.

              For other link count comparisons, use field output (-F)
              and a post-processing script or program.

       +|-m m specifies an alternate kernel memory file or activates
              mount table supplement processing.

              The option form -m m specifies a kernel memory file, m, in
              place of /dev/kmem or /dev/mem - e.g., a crash dump file.

              The option form +m requests that a mount supplement file
              be written to the standard output file.  All other options
              are silently ignored.

              There will be a line in the mount supplement file for each
              mounted file system, containing the mounted file system
              directory, followed by a single space, followed by the
              device number in hexadecimal "0x" format - e.g.,

                   / 0x801

              Lsof can use the mount supplement file to get device
              numbers for file systems when it can't get them via
              stat(2) or lstat(2).

              The option form +m m identifies m as a mount supplement
              file.

              Note: the +m and +m m options are not available for all
              supported dialects.  Check the output of lsof's -h or -?
              options to see if the +m and +m m options are available.

       +|-M   Enables (+) or disables (-) the reporting of portmapper
              registrations for local TCP, UDP and UDPLITE ports, where
              port mapping is supported.  (See the last paragraph of
              this option description for information about where
              portmapper registration reporting is supported.)

              The default reporting mode is set by the lsof builder with
              the HASPMAPENABLED #define in the dialect's machine.h
              header file; lsof is distributed with the HASPMAPENABLED
              #define deactivated, so portmapper reporting is disabled
              by default and must be requested with +M.  Specifying
              lsof's -h or -?  option will report the default mode.
              Disabling portmapper registration when it is already
              disabled or enabling it when already enabled is
              acceptable.  When portmapper registration reporting is
              enabled, lsof displays the portmapper registration (if
              any) for local TCP, UDP or UDPLITE ports in square
              brackets immediately following the port numbers or service
              names - e.g., ``:1234[name]'' or ``:name[100083]''.  The
              registration information may be a name or number,
              depending on what the registering program supplied to the
              portmapper when it registered the port.

              When portmapper registration reporting is enabled, lsof
              may run a little more slowly or even become blocked when
              access to the portmapper becomes congested or stopped.
              Reverse the reporting mode to determine if portmapper
              registration reporting is slowing or blocking lsof.

              For purposes of portmapper registration reporting lsof
              considers a TCP, UDP or UDPLITE port local if: it is found
              in the local part of its containing kernel structure; or
              if it is located in the foreign part of its containing
              kernel structure and the local and foreign Internet
              addresses are the same; or if it is located in the foreign
              part of its containing kernel structure and the foreign
              Internet address is INADDR_LOOPBACK (127.0.0.1).  This
              rule may make lsof ignore some foreign ports on machines
              with multiple interfaces when the foreign Internet address
              is on a different interface from the local one.

              See the lsof FAQ (The FAQ section gives its location.)
              for further discussion of portmapper registration
              reporting issues.

              Portmapper registration reporting is supported only on
              dialects that have RPC header files.  (Some Linux
              distributions with GlibC 2.14 do not have them.)  When
              portmapper registration reporting is supported, the -h or
              -?  help output will show the +|-M option.

       -n     inhibits the conversion of network numbers to host names
              for network files.  Inhibiting conversion may make lsof
              run faster.  It is also useful when host name lookup is
              not working properly.

       -N     selects the listing of NFS files.

       -o     directs lsof to display file offset at all times.  It
              causes the SIZE/OFF output column title to be changed to
              OFFSET.  Note: on some UNIX dialects lsof can't obtain
              accurate or consistent file offset information from its
              kernel data sources, sometimes just for particular kinds
              of files (e.g., socket files.)  Consult the lsof FAQ (The
              FAQ section gives its location.)  for more information.

              The -o and -s options are mutually exclusive; they can't
              both be specified.  When neither is specified, lsof
              displays whatever value - size or offset - is appropriate
              and available for the type of the file.

       -o o   defines the number of decimal digits (o) to be printed
              after the ``0t'' for a file offset before the form is
              switched to ``0x...''.  An o value of zero (unlimited)
              directs lsof to use the ``0t'' form for all offset output.

              This option does NOT direct lsof to display offset at all
              times; specify -o (without a trailing number) to do that.
              -o o only specifies the number of digits after ``0t'' in
              either mixed size and offset or offset-only output.  Thus,
              for example, to direct lsof to display offset at all times
              with a decimal digit count of 10, use:

                   -o -o 10
              or
                   -oo10

              The default number of digits allowed after ``0t'' is
              normally 8, but may have been changed by the lsof builder.
              Consult the description of the -o o option in the output
              of the -h or -?  option to determine the default that is
              in effect.

       -O     directs lsof to bypass the strategy it uses to avoid being
              blocked by some kernel operations - i.e., doing them in
              forked child processes.  See the BLOCKS AND TIMEOUTS and
              AVOIDING KERNEL BLOCKS sections for more information on
              kernel operations that may block lsof.

              While use of this option will reduce lsof startup
              overhead, it may also cause lsof to hang when the kernel
              doesn't respond to a function.  Use this option
              cautiously.

       -p s   excludes or selects the listing of files for the processes
              whose optional process IDentification (PID) numbers are in
              the comma-separated set s - e.g., ``123'' or ``123,^456''.
              (There should be no spaces in the set.)

              PID numbers that begin with `^' (negation) represent
              exclusions.

              Multiple process ID numbers are joined in a single ORed
              set before participating in AND option selection.
              However, PID exclusions are applied without ORing or
              ANDing and take effect before other selection criteria are
              applied.

       -P     inhibits the conversion of port numbers to port names for
              network files.  Inhibiting the conversion may make lsof
              run a little faster.  It is also useful when port name
              lookup is not working properly.

       +|-r [t[m<fmt>]]
              puts lsof in repeat mode.  There lsof lists open files as
              selected by other options, delays t seconds (default
              fifteen), then repeats the listing, delaying and listing
              repetitively until stopped by a condition defined by the
              prefix to the option.

              If the prefix is a `-', repeat mode is endless.  Lsof must
              be terminated with an interrupt or quit signal.

              If the prefix is `+', repeat mode will end the first cycle
              no open files are listed - and of course when lsof is
              stopped with an interrupt or quit signal.  When repeat
              mode ends because no files are listed, the process exit
              code will be zero if any open files were ever listed; one,
              if none were ever listed.

              Lsof marks the end of each listing: if field output is in
              progress (the -F, option has been specified), the default
              marker is `m'; otherwise the default marker is
              ``========''.  The marker is followed by a NL character.

              The optional "m<fmt>" argument specifies a format for the
              marker line.  The <fmt> characters following `m' are
              interpreted as a format specification to the strftime(3)
              function, when both it and the localtime(3) function are
              available in the dialect's C library.  Consult the
              strftime(3) documentation for what may appear in its
              format specification.  Note that when field output is
              requested with the -F option, <fmt> cannot contain the NL
              format, ``%n''.  Note also that when <fmt> contains spaces
              or other characters that affect the shell's interpretation
              of arguments, <fmt> must be quoted appropriately.

              Repeat mode reduces lsof startup overhead, so it is more
              efficient to use this mode than to call lsof repetitively
              from a shell script, for example.

              To use repeat mode most efficiently, accompany +|-r with
              specification of other lsof selection options, so the
              amount of kernel memory access lsof does will be kept to a
              minimum.  Options that filter at the process level - e.g.,
              -c, -g, -p, -u - are the most efficient selectors.

              Repeat mode is useful when coupled with field output (see
              the -F, option description) and a supervising awk or Perl
              script, or a C program.

       -R     directs lsof to list the Parent Process IDentification
              number in the PPID column.

       -s [p:s]
              s alone directs lsof to display file size at all times.
              It causes the SIZE/OFF output column title to be changed
              to SIZE.  If the file does not have a size, nothing is
              displayed.

              The optional -s p:s form is available only for selected
              dialects, and only when the -h or -?  help output lists
              it.

              When the optional form is available, the s may be followed
              by a protocol name (p), either TCP or UDP, a colon (`:')
              and a comma-separated protocol state name list, the option
              causes open TCP and UDP files to be excluded if their
              state name(s) are in the list (s) preceded by a `^'; or
              included if their name(s) are not preceded by a `^'.

              Dialects that support this option may support only one
              protocol.  When an unsupported protocol is specified, a
              message will be displayed indicating state names for the
              protocol are unavailable.

              When an inclusion list is defined, only network files with
              state names in the list will be present in the lsof
              output.  Thus, specifying one state name means that only
              network files with that lone state name will be listed.

              Case is unimportant in the protocol or state names, but
              there may be no spaces and the colon (`:') separating the
              protocol name (p) and the state name list (s) is required.

              If only TCP and UDP files are to be listed, as controlled
              by the specified exclusions and inclusions, the -i option
              must be specified, too.  If only a single protocol's files
              are to be listed, add its name as an argument to the -i
              option.

              For example, to list only network files with TCP state
              LISTEN, use:

                   -iTCP -sTCP:LISTEN

              Or, for example, to list network files with all UDP states
              except Idle, use:

                   -iUDP -sUDP:Idle

              State names vary with UNIX dialects, so it's not possible
              to provide a complete list.  Some common TCP state names
              are: CLOSED, IDLE, BOUND, LISTEN, ESTABLISHED, SYN_SENT,
              SYN_RCDV, ESTABLISHED, CLOSE_WAIT, FIN_WAIT1, CLOSING,
              LAST_ACK, FIN_WAIT_2, and TIME_WAIT.  Two common UDP state
              names are Unbound and Idle.

              See the lsof FAQ (The FAQ section gives its location.)
              for more information on how to use protocol state
              exclusion and inclusion, including examples.

              The -o (without a following decimal digit count) and -s
              option (without a following protocol and state name list)
              are mutually exclusive; they can't both be specified.
              When neither is specified, lsof displays whatever value -
              size or offset - is appropriate and available for the type
              of file.

              Since some types of files don't have true sizes - sockets,
              FIFOs, pipes, etc. - lsof displays for their sizes the
              content amounts in their associated kernel buffers, if
              possible.

       -S [t] specifies an optional time-out seconds value for kernel
              functions - lstat(2), readlink(2), and stat(2) - that
              might otherwise deadlock.  The minimum for t is two; the
              default, fifteen; when no value is specified, the default
              is used.

              See the BLOCKS AND TIMEOUTS section for more information.

       -T [t] controls the reporting of some TCP/TPI information, also
              reported by netstat(1), following the network addresses.
              In normal output the information appears in parentheses,
              each item except TCP or TPI state name identified by a
              keyword, followed by `=', separated from others by a
              single space:

                   <TCP or TPI state name>
                   QR=<read queue length>
                   QS=<send queue length>
                   SO=<socket options and values>
                   SS=<socket states>
                   TF=<TCP flags and values>
                   WR=<window read length>
                   WW=<window write length>

              Not all values are reported for all UNIX dialects.  Items
              values (when available) are reported after the item name
              and '='.

              When the field output mode is in effect (See OUTPUT FOR
              OTHER PROGRAMS.)  each item appears as a field with a `T'
              leading character.

              -T with no following key characters disables TCP/TPI
              information reporting.

              -T with following characters selects the reporting of
              specific TCP/TPI information:

                   f    selects reporting of socket options,
                        states and values, and TCP flags and
                        values.
                   q    selects queue length reporting.
                   s    selects connection state reporting.
                   w    selects window size reporting.

              Not all selections are enabled for some UNIX dialects.
              State may be selected for all dialects and is reported by
              default.  The -h or -?  help output for the -T option will
              show what selections may be used with the UNIX dialect.

              When -T is used to select information - i.e., it is
              followed by one or more selection characters - the
              displaying of state is disabled by default, and it must be
              explicitly selected again in the characters following -T.
              (In effect, then, the default is equivalent to -Ts.)  For
              example, if queue lengths and state are desired, use -Tqs.

              Socket options, socket states, some socket values, TCP
              flags and one TCP value may be reported (when available in
              the UNIX dialect) in the form of the names that commonly
              appear after SO_, so_, SS_, TCP_  and TF_ in the dialect's
              header files - most often <sys/socket.h>,
              <sys/socketvar.h> and <netinet/tcp_var.h>.  Consult those
              header files for the meaning of the flags, options, states
              and values.

              ``SO='' precedes socket options and values; ``SS='',
              socket states; and ``TF='', TCP flags and values.

              If a flag or option has a value, the value will follow an
              '=' and the name -- e.g., ``SO=LINGER=5'', ``SO=QLIM=5'',
              ``TF=MSS=512''.  The following seven values may be
              reported:

                   Name
                   Reported  Description (Common Symbol)

                   KEEPALIVE keep alive time (SO_KEEPALIVE)
                   LINGER    linger time (SO_LINGER)
                   MSS       maximum segment size (TCP_MAXSEG)
                   PQLEN          partial listen queue connections
                   QLEN      established listen queue connections
                   QLIM      established listen queue limit
                   RCVBUF    receive buffer length (SO_RCVBUF)
                   SNDBUF    send buffer length (SO_SNDBUF)

              Details on what socket options and values, socket states,
              and TCP flags and values may be displayed for particular
              UNIX dialects may be found in the answer to the ``Why
              doesn't lsof report socket options, socket states, and TCP
              flags and values for my dialect?'' and ``Why doesn't lsof
              report the partial listen queue connection count for my
              dialect?''  questions in the lsof FAQ (The FAQ section
              gives its location.)

       -t     specifies that lsof should produce terse output with
              process identifiers only and no header - e.g., so that the
              output may be piped to kill(1).  -t selects the -w option.

       -u s   selects the listing of files for the user whose login
              names or user ID numbers are in the comma-separated set s
              - e.g., ``abe'', or ``548,root''.  (There should be no
              spaces in the set.)

              Multiple login names or user ID numbers are joined in a
              single ORed set before participating in AND option
              selection.

              If a login name or user ID is preceded by a `^', it
              becomes a negation - i.e., files of processes owned by the
              login name or user ID will never be listed.  A negated
              login name or user ID selection is neither ANDed nor ORed
              with other selections; it is applied before all other
              selections and absolutely excludes the listing of the
              files of the process.  For example, to direct lsof to
              exclude the listing of files belonging to root processes,
              specify ``-u^root'' or ``-u^0''.

       -U     selects the listing of UNIX domain socket files.

       -v     selects the listing of lsof version information,
              including: revision number; when the lsof binary was
              constructed; who constructed the binary and where; the
              name of the compiler used to construct the lsof binary;
              the version number of the compiler when readily available;
              the compiler and loader flags used to construct the lsof
              binary; and system information, typically the output of
              uname's -a option.

       -V     directs lsof to indicate the items it was asked to list
              and failed to find - command names, file names, Internet
              addresses or files, login names, NFS files, PIDs, PGIDs,
              and UIDs.

              When other options are ANDed to search options, or
              compile-time options restrict the listing of some files,
              lsof may not report that it failed to find a search item
              when an ANDed option or compile-time option prevents the
              listing of the open file containing the located search
              item.

              For example, ``lsof -V -iTCP@foobar -a -d 999'' may not
              report a failure to locate open files at ``TCP@foobar''
              and may not list any, if none have a file descriptor
              number of 999.  A similar situation arises when
              HASSECURITY and HASNOSOCKSECURITY are defined at compile
              time and they prevent the listing of open files.

       +|-w   Enables (+) or disables (-) the suppression of warning
              messages.

              The lsof builder may choose to have warning messages
              disabled or enabled by default.  The default warning
              message state is indicated in the output of the -h or -?
              option.  Disabling warning messages when they are already
              disabled or enabling them when already enabled is
              acceptable.

              The -t option selects the -w option.

       -x [fl]
              may accompany the +d and +D options to direct their
              processing to cross over symbolic links and|or file system
              mount points encountered when scanning the directory (+d)
              or directory tree (+D).

              If -x is specified by itself without a following
              parameter, cross-over processing of both symbolic links
              and file system mount points is enabled.  Note that when
              -x is specified without a parameter, the next argument
              must begin with '-' or '+'.

              The optional 'f' parameter enables file system mount point
              cross-over processing; 'l', symbolic link cross-over
              processing.

              The -x option may not be supplied without also supplying a
              +d or +D option.

       -X     This is a dialect-specific option.

           AIX:
                This IBM AIX RISC/System 6000 option requests the
                reporting of executed text file and shared library
                references.

                WARNING: because this option uses the kernel readx()
                function, its use on a busy AIX system might cause an
                application process to hang so completely that it can
                neither be killed nor stopped.  I have never seen this
                happen or had a report of its happening, but I think
                there is a remote possibility it could happen.

                By default use of readx() is disabled.  On AIX 5L and
                above lsof may need setuid-root permission to perform
                the actions this option requests.

                The lsof builder may specify that the -X option be
                restricted to processes whose real UID is root.  If that
                has been done, the -X option will not appear in the -h
                or -?  help output unless the real UID of the lsof
                process is root.  The default lsof distribution allows
                any UID to specify -X, so by default it will appear in
                the help output.

                When AIX readx() use is disabled, lsof may not be able
                to report information for all text and loader file
                references, but it may also avoid exacerbating an AIX
                kernel directory search kernel error, known as the Stale
                Segment ID bug.

                The readx() function, used by lsof or any other program
                to access some sections of kernel virtual memory, can
                trigger the Stale Segment ID bug.  It can cause the
                kernel's dir_search() function to believe erroneously
                that part of an in-memory copy of a file system
                directory has been zeroed.  Another application process,
                distinct from lsof, asking the kernel to search the
                directory - e.g., by using open(2) - can cause
                dir_search() to loop forever, thus hanging the
                application process.

                Consult the lsof FAQ (The FAQ section gives its
                location.)  and the 00README file of the lsof
                distribution for a more complete description of the
                Stale Segment ID bug, its APAR, and methods for defining
                readx() use when compiling lsof.

           Linux:
                This Linux option requests that lsof skip the reporting
                of information on all open TCP, UDP and UDPLITE IPv4 and
                IPv6 files.

                This Linux option is most useful when the system has an
                extremely large number of open TCP, UDP and UDPLITE
                files, the processing of whose information in the
                /proc/net/tcp* and /proc/net/udp* files would take lsof
                a long time, and whose reporting is not of interest.

                Use this option with care and only when you are sure
                that the information you want lsof to display isn't
                associated with open TCP, UDP or UDPLITE socket files.

           Solaris 10 and above:
                This Solaris 10 and above option requests the reporting
                of cached paths for files that have been deleted - i.e.,
                removed with rm(1) or unlink(2).

                The cached path is followed by the string `` (deleted)''
                to indicate that the path by which the file was opened
                has been deleted.

                Because intervening changes made to the path - i.e.,
                renames with mv(1) or rename(2) - are not recorded in
                the cached path, what lsof reports is only the path by
                which the file was opened, not its possibly different
                final path.

       -z [z]   specifies how Solaris 10 and higher zone information is
                to be handled.

                Without a following argument - e.g., NO z - the option
                specifies that zone names are to be listed in the ZONE
                output column.

                The -z option may be followed by a zone name, z.  That
                causes lsof to list only open files for processes in
                that zone.  Multiple -z z option and argument pairs may
                be specified to form a list of named zones.  Any open
                file of any process in any of the zones will be listed,
                subject to other conditions specified by other options
                and arguments.

       -Z [Z]   specifies how SELinux security contexts are to be
                handled.  It and 'Z' field output character support are
                inhibited when SELinux is disabled in the running Linux
                kernel.  See OUTPUT FOR OTHER PROGRAMS for more
                information on the 'Z' field output character.

                Without a following argument - e.g., NO Z - the option
                specifies that security contexts are to be listed in the
                SECURITY-CONTEXT output column.

                The -Z option may be followed by a wildcard security
                context name, Z.  That causes lsof to list only open
                files for processes in that security context.  Multiple
                -Z Z option and argument pairs may be specified to form
                a list of security contexts.  Any open file of any
                process in any of the security contexts will be listed,
                subject to other conditions specified by other options
                and arguments.  Note that Z can be A:B:C or *:B:C or
                A:B:* or *:*:C to match against the A:B:C context.

       --       The double minus sign option is a marker that signals
                the end of the keyed options.  It may be used, for
                example, when the first file name begins with a minus
                sign.  It may also be used when the absence of a value
                for the last keyed option must be signified by the
                presence of a minus sign in the following option and
                before the start of the file names.

       names    These are path names of specific files to list.
                Symbolic links are resolved before use.  The first name
                may be separated from the preceding options with the
                ``--'' option.

                If a name is the mounted-on directory of a file system
                or the device of the file system, lsof will list all the
                files open on the file system.  To be considered a file
                system, the name must match a mounted-on directory name
                in mount(8) output, or match the name of a block device
                associated with a mounted-on directory name.  The +|-f
                option may be used to force lsof to consider a name a
                file system identifier (+f) or a simple file (-f).

                If name is a path to a directory that is not the
                mounted-on directory name of a file system, it is
                treated just as a regular file is treated - i.e., its
                listing is restricted to processes that have it open as
                a file or as a process-specific directory, such as the
                root or current working directory.  To request that lsof
                look for open files inside a directory name, use the +d
                s and +D D options.

                If a name is the base name of a family of multiplexed
                files - e.g, AIX's /dev/pt[cs] - lsof will list all the
                associated multiplexed files on the device that are open
                - e.g., /dev/pt[cs]/1, /dev/pt[cs]/2, etc.

                If a name is a UNIX domain socket name, lsof will
                usually search for it by the characters of the name
                alone - exactly as it is specified and is recorded in
                the kernel socket structure.  (See the next paragraph
                for an exception to that rule for Linux.)  Specifying a
                relative path - e.g., ./file - in place of the file's
                absolute path - e.g., /tmp/file - won't work because
                lsof must match the characters you specify with what it
                finds in the kernel UNIX domain socket structures.

                If a name is a Linux UNIX domain socket name, in one
                case lsof is able to search for it by its device and
                inode number, allowing name to be a relative path.  The
                case requires that the absolute path -- i.e., one
                beginning with a slash ('/') be used by the process that
                created the socket, and hence be stored in the
                /proc/net/unix file; and it requires that lsof be able
                to obtain the device and node numbers of both the
                absolute path in /proc/net/unix and name via successful
                stat(2) system calls.  When those conditions are met,
                lsof will be able to search for the UNIX domain socket
                when some path to it is is specified in name.  Thus, for
                example, if the path is /dev/log, and an lsof search is
                initiated when the working directory is /dev, then name
                could be ./log.

                If a name is none of the above, lsof will list any open
                files whose device and inode match that of the specified
                path name.

                If you have also specified the -b option, the only names
                you may safely specify are file systems for which your
                mount table supplies alternate device numbers.  See the
                AVOIDING KERNEL BLOCKS and ALTERNATE DEVICE NUMBERS
                sections for more information.

                Multiple file names are joined in a single ORed set
                before participating in AND option selection.
AFS
       Lsof supports the recognition of AFS files for these dialects
       (and AFS versions):

            AIX 4.1.4 (AFS 3.4a)
            HP-UX 9.0.5 (AFS 3.4a)
            Linux 1.2.13 (AFS 3.3)
            Solaris 2.[56] (AFS 3.4a)

       It may recognize AFS files on other versions of these dialects,
       but has not been tested there.  Depending on how AFS is
       implemented, lsof may recognize AFS files in other dialects, or
       may have difficulties recognizing AFS files in the supported
       dialects.

       Lsof may have trouble identifying all aspects of AFS files in
       supported dialects when AFS kernel support is implemented via
       dynamic modules whose addresses do not appear in the kernel's
       variable name list.  In that case, lsof may have to guess at the
       identity of AFS files, and might not be able to obtain volume
       information from the kernel that is needed for calculating AFS
       volume node numbers.  When lsof can't compute volume node
       numbers, it reports blank in the NODE column.

       The -A A option is available in some dialect implementations of
       lsof for specifying the name list file where dynamic module
       kernel addresses may be found.  When this option is available, it
       will be listed in the lsof help output, presented in response to
       the -h or -?

       See the lsof FAQ (The FAQ section gives its location.)  for more
       information about dynamic modules, their symbols, and how they
       affect lsof options.

       Because AFS path lookups don't seem to participate in the
       kernel's name cache operations, lsof can't identify path name
       components for AFS files.
SECURITY
       Lsof has three features that may cause security concerns.  First,
       its default compilation mode allows anyone to list all open files
       with it.  Second, by default it creates a user-readable and
       user-writable device cache file in the home directory of the real
       user ID that executes lsof.  (The list-all-open-files and device
       cache features may be disabled when lsof is compiled.)  Third,
       its -k and -m options name alternate kernel name list or memory
       files.

       Restricting the listing of all open files is controlled by the
       compile-time HASSECURITY and HASNOSOCKSECURITY options.  When
       HASSECURITY is defined, lsof will allow only the root user to
       list all open files.  The non-root user may list only open files
       of processes with the same user IDentification number as the real
       user ID number of the lsof process (the one that its user logged
       on with).

       However, if HASSECURITY and HASNOSOCKSECURITY are both defined,
       anyone may list open socket files, provided they are selected
       with the -i option.

       When HASSECURITY is not defined, anyone may list all open files.

       Help output, presented in response to the -h or -?  option, gives
       the status of the HASSECURITY and HASNOSOCKSECURITY definitions.

       See the Security section of the 00README file of the lsof
       distribution for information on building lsof with the
       HASSECURITY and HASNOSOCKSECURITY options enabled.

       Creation and use of a user-readable and user-writable device
       cache file is controlled by the compile-time HASDCACHE option.
       See the DEVICE CACHE FILE section and the sections that follow it
       for details on how its path is formed.  For security
       considerations it is important to note that in the default lsof
       distribution, if the real user ID under which lsof is executed is
       root, the device cache file will be written in root's home
       directory - e.g., / or /root.  When HASDCACHE is not defined,
       lsof does not write or attempt to read a device cache file.

       When HASDCACHE is defined, the lsof help output, presented in
       response to the -h, -D?, or -?  options, will provide device
       cache file handling information.  When HASDCACHE is not defined,
       the -h or -?  output will have no -D option description.

       Before you decide to disable the device cache file feature -
       enabling it improves the performance of lsof by reducing the
       startup overhead of examining all the nodes in /dev (or /devices)
       - read the discussion of it in the 00DCACHE file of the lsof
       distribution and the lsof FAQ (The FAQ section gives its
       location.)

       WHEN IN DOUBT, YOU CAN TEMPORARILY DISABLE THE USE OF THE DEVICE
       CACHE FILE WITH THE -Di OPTION.

       When lsof user declares alternate kernel name list or memory
       files with the -k and -m options, lsof checks the user's
       authority to read them with access(2).  This is intended to
       prevent whatever special power lsof's modes might confer on it
       from letting it read files not normally accessible via the
       authority of the real user ID.
OUTPUT
       This section describes the information lsof lists for each open
       file.  See the OUTPUT FOR OTHER PROGRAMS section for additional
       information on output that can be processed by another program.

       Lsof only outputs printable (declared so by isprint(3)) 8 bit
       characters.  Non-printable characters are printed in one of three
       forms: the C ``\[bfrnt]'' form; the control character `^' form
       (e.g., ``^@''); or hexadecimal leading ``\x'' form (e.g.,
       ``\xab'').  Space is non-printable in the COMMAND column
       (``\x20'') and printable elsewhere.

       For some dialects - if HASSETLOCALE is defined in the dialect's
       machine.h header file - lsof will print the extended 8 bit
       characters of a language locale.  The lsof process must be
       supplied a language locale environment variable (e.g., LANG)
       whose value represents a known language locale in which the
       extended characters are considered printable by isprint(3).
       Otherwise lsof considers the extended characters non-printable
       and prints them according to its rules for non-printable
       characters, stated above.  Consult your dialect's setlocale(3)
       man page for the names of other environment variables that may be
       used in place of LANG - e.g., LC_ALL, LC_CTYPE, etc.

       Lsof's language locale support for a dialect also covers wide
       characters - e.g., UTF-8 - when HASSETLOCALE and HASWIDECHAR are
       defined in the dialect's machine.h header file, and when a
       suitable language locale has been defined in the appropriate
       environment variable for the lsof process.  Wide characters are
       printable under those conditions if iswprint(3) reports them to
       be.  If HASSETLOCALE, HASWIDECHAR and a suitable language locale
       aren't defined, or if iswprint(3) reports wide characters that
       aren't printable, lsof considers the wide characters
       non-printable and prints each of their 8 bits according to its
       rules for non-printable characters, stated above.

       Consult the answers to the "Language locale support" questions in
       the lsof FAQ (The FAQ section gives its location.) for more
       information.

       Lsof dynamically sizes the output columns each time it runs,
       guaranteeing that each column is a minimum size.  It also
       guarantees that each column is separated from its predecessor by
       at least one space.

       COMMAND
              contains the first nine characters of the name of the UNIX
              command associated with the process.  If a non-zero w
              value is specified to the +c w option, the column contains
              the first w characters of the name of the UNIX command
              associated with the process up to the limit of characters
              supplied to lsof by the UNIX dialect.  (See the
              description of the +c w command or the lsof FAQ for more
              information.  The FAQ section gives its location.)

              If w is less than the length of the column title,
              ``COMMAND'', it will be raised to that length.

              If a zero w value is specified to the +c w option, the
              column contains all the characters of the name of the UNIX
              command associated with the process.

              All command name characters maintained by the kernel in
              its structures are displayed in field output when the
              command name descriptor (`c') is specified.  See the
              OUTPUT FOR OTHER COMMANDS section for information on
              selecting field output and the associated command name
              descriptor.

       PID    is the Process IDentification number of the process.

       TID    is the task (thread) IDentification number, if task
              (thread) reporting is supported by the dialect and a task
              (thread) is being listed.  (If help output - i.e., the
              output of the -h or -?  options - shows this option, then
              task (thread) reporting is supported by the dialect.)

              A blank TID column in Linux indicates a process - i.e., a
              non-task.

       TASKCMD
              is the task command name.  Generally this will be the same
              as the process named in the COMMAND column, but some task
              implementations (e.g., Linux) permit a task to change its
              command name.

              The TASKCMD column width is subject to the same size
              limitation as the COMMAND column.

       ZONE   is the Solaris 10 and higher zone name.  This column must
              be selected with the -z option.

       SECURITY-CONTEXT
              is the SELinux security context.  This column must be
              selected with the -Z option.  Note that the -Z option is
              inhibited when SELinux is disabled in the running Linux
              kernel.

       PPID   is the Parent Process IDentification number of the
              process.  It is only displayed when the -R option has been
              specified.

       PGID   is the process group IDentification number associated with
              the process.  It is only displayed when the -g option has
              been specified.

       USER   is the user ID number or login name of the user to whom
              the process belongs, usually the same as reported by
              ps(1).  However, on Linux USER is the user ID number or
              login that owns the directory in /proc where lsof finds
              information about the process.  Usually that is the same
              value reported by ps(1), but may differ when the process
              has changed its effective user ID.  (See the -l option
              description for information on when a user ID number or
              login name is displayed.)

       FD     is the File Descriptor number of the file or:

                   cwd  current working directory;
                   Lnn  library references (AIX);
                   err  FD information error (see NAME column);
                   jld  jail directory (FreeBSD);
                   ltx  shared library text (code and data);
                   Mxx  hex memory-mapped type number xx.
                   m86  DOS Merge mapped file;
                   mem  memory-mapped file;
                   mmap memory-mapped device;
                   pd   parent directory;
                   rtd  root directory;
                   tr   kernel trace file (OpenBSD);
                   txt  program text (code and data);
                   v86  VP/ix mapped file;

              FD is followed by one of these characters, describing the
              mode under which the file is open:

                   r for read access;
                   w for write access;
                   u for read and write access;
                   space if mode unknown and no lock
                        character follows;
                   `-' if mode unknown and lock
                        character follows.

              The mode character is followed by one of these lock
              characters, describing the type of lock applied to the
              file:

                   N for a Solaris NFS lock of unknown type;
                   r for read lock on part of the file;
                   R for a read lock on the entire file;
                   w for a write lock on part of the file;
                   W for a write lock on the entire file;
                   u for a read and write lock of any length;
                   U for a lock of unknown type;
                   x for an SCO OpenServer Xenix lock on part      of
              the file;
                   X for an SCO OpenServer Xenix lock on the entire
              file;
                   space if there is no lock.

              See the LOCKS section for more information on the lock
              information character.

              The FD column contents constitutes a single field for
              parsing in post-processing scripts.

       TYPE   is the type of the node associated with the file - e.g.,
              GDIR, GREG, VDIR, VREG, etc.

              or ``IPv4'' for an IPv4 socket;

              or ``IPv6'' for an open IPv6 network file - even if its
              address is IPv4, mapped in an IPv6 address;

              or ``ax25'' for a Linux AX.25 socket;

              or ``inet'' for an Internet domain socket;

              or ``lla'' for a HP-UX link level access file;

              or ``rte'' for an AF_ROUTE socket;

              or ``sock'' for a socket of unknown domain;

              or ``unix'' for a UNIX domain socket;

              or ``x.25'' for an HP-UX x.25 socket;

              or ``BLK'' for a block special file;

              or ``CHR'' for a character special file;

              or ``DEL'' for a Linux map file that has been deleted;

              or ``DIR'' for a directory;

              or ``DOOR'' for a VDOOR file;

              or ``FIFO'' for a FIFO special file;

              or ``KQUEUE'' for a BSD style kernel event queue file;

              or ``LINK'' for a symbolic link file;

              or ``MPB'' for a multiplexed block file;

              or ``MPC'' for a multiplexed character file;

              or ``NOFD'' for a Linux /proc/<PID>/fd directory that
              can't be opened -- the directory path appears in the NAME
              column, followed by an error message;

              or ``PAS'' for a /proc/as file;

              or ``PAXV'' for a /proc/auxv file;

              or ``PCRE'' for a /proc/cred file;

              or ``PCTL'' for a /proc control file;

              or ``PCUR'' for the current /proc process;

              or ``PCWD'' for a /proc current working directory;

              or ``PDIR'' for a /proc directory;

              or ``PETY'' for a /proc executable type (etype);

              or ``PFD'' for a /proc file descriptor;

              or ``PFDR'' for a /proc file descriptor directory;

              or ``PFIL'' for an executable /proc file;

              or ``PFPR'' for a /proc FP register set;

              or ``PGD'' for a /proc/pagedata file;

              or ``PGID'' for a /proc group notifier file;

              or ``PIPE'' for pipes;

              or ``PLC'' for a /proc/lwpctl file;

              or ``PLDR'' for a /proc/lpw directory;

              or ``PLDT'' for a /proc/ldt file;

              or ``PLPI'' for a /proc/lpsinfo file;

              or ``PLST'' for a /proc/lstatus file;

              or ``PLU'' for a /proc/lusage file;

              or ``PLWG'' for a /proc/gwindows file;

              or ``PLWI'' for a /proc/lwpsinfo file;

              or ``PLWS'' for a /proc/lwpstatus file;

              or ``PLWU'' for a /proc/lwpusage file;

              or ``PLWX'' for a /proc/xregs file;

              or ``PMAP'' for a /proc map file (map);

              or ``PMEM'' for a /proc memory image file;

              or ``PNTF'' for a /proc process notifier file;

              or ``POBJ'' for a /proc/object file;

              or ``PODR'' for a /proc/object directory;

              or ``POLP'' for an old format /proc light weight process
              file;

              or ``POPF'' for an old format /proc PID file;

              or ``POPG'' for an old format /proc page data file;

              or ``PORT'' for a SYSV named pipe;

              or ``PREG'' for a /proc register file;

              or ``PRMP'' for a /proc/rmap file;

              or ``PRTD'' for a /proc root directory;

              or ``PSGA'' for a /proc/sigact file;

              or ``PSIN'' for a /proc/psinfo file;

              or ``PSTA'' for a /proc status file;

              or ``PSXSEM'' for a POSIX semaphore file;

              or ``PSXSHM'' for a POSIX shared memory file;

              or ``PTS'' for a /dev/pts file;

              or ``PUSG'' for a /proc/usage file;

              or ``PW'' for a /proc/watch file;

              or ``PXMP'' for a /proc/xmap file;

              or ``REG'' for a regular file;

              or ``SMT'' for a shared memory transport file;

              or ``STSO'' for a stream socket;

              or ``UNNM'' for an unnamed type file;

              or ``XNAM'' for an OpenServer Xenix special file of
              unknown type;

              or ``XSEM'' for an OpenServer Xenix semaphore file;

              or ``XSD'' for an OpenServer Xenix shared data file;

              or the four type number octets if the corresponding name
              isn't known.

       FILE-ADDR
              contains the kernel file structure address when f has been
              specified to +f;

       FCT    contains the file reference count from the kernel file
              structure when c has been specified to +f;

       FILE-FLAG
              when g or G has been specified to +f, this field contains
              the contents of the f_flag[s] member of the kernel file
              structure and the kernel's per-process open file flags (if
              available); `G' causes them to be displayed in
              hexadecimal; `g', as short-hand names; two lists may be
              displayed with entries separated by commas, the lists
              separated by a semicolon (`;'); the first list may contain
              short-hand names for f_flag[s] values from the following
              table:

                   AIO       asynchronous I/O (e.g., FAIO)
                   AP        append
                   ASYN      asynchronous I/O (e.g., FASYNC)
                   BAS       block, test, and set in use
                   BKIU      block if in use
                   BL        use block offsets
                   BSK       block seek
                   CA        copy avoid
                   CIO       concurrent I/O
                   CLON      clone
                   CLRD      CL read
                   CR        create
                   DF        defer
                   DFI       defer IND
                   DFLU      data flush
                   DIR       direct
                   DLY       delay
                   DOCL      do clone
                   DSYN      data-only integrity
                   DTY       must be a directory
                   EVO       event only
                   EX        open for exec
                   EXCL      exclusive open
                   FSYN      synchronous writes
                   GCDF      defer during unp_gc() (AIX)
                   GCMK      mark during unp_gc() (AIX)
                   GTTY      accessed via /dev/tty
                   HUP       HUP in progress
                   KERN      kernel
                   KIOC      kernel-issued ioctl
                   LCK       has lock
                   LG        large file
                   MBLK      stream message block
                   MK        mark
                   MNT       mount
                   MSYN      multiplex synchronization
                   NATM      don't update atime
                   NB        non-blocking I/O
                   NBDR      no BDRM check
                   NBIO      SYSV non-blocking I/O
                   NBF       n-buffering in effect
                   NC        no cache
                   ND        no delay
                   NDSY      no data synchronization
                   NET       network
                   NFLK      don't follow links
                   NMFS      NM file system
                   NOTO      disable background stop
                   NSH       no share
                   NTTY      no controlling TTY
                   OLRM      OLR mirror
                   PAIO      POSIX asynchronous I/O
                   PP        POSIX pipe
                   R         read
                   RC        file and record locking cache
                   REV       revoked
                   RSH       shared read
                   RSYN      read synchronization
                   RW        read and write access
                   SL        shared lock
                   SNAP      cooked snapshot
                   SOCK      socket
                   SQSH      Sequent shared set on open
                   SQSV      Sequent SVM set on open
                   SQR       Sequent set repair on open
                   SQS1      Sequent full shared open
                   SQS2      Sequent partial shared open
                   STPI      stop I/O
                   SWR       synchronous read
                   SYN       file integrity while writing
                   TCPM      avoid TCP collision
                   TR        truncate
                   W         write
                   WKUP      parallel I/O synchronization
                   WTG       parallel I/O synchronization
                   VH        vhangup pending
                   VTXT      virtual text
                   XL        exclusive lock

              this list of names was derived from F* #define's in
              dialect header files <fcntl.h>, <linux</fs.h>,
              <sys/fcntl.c>, <sys/fcntlcom.h>, and <sys/file.h>; see the
              lsof.h header file for a list showing the correspondence
              between the above short-hand names and the header file
              definitions;

              the second list (after the semicolon) may contain
              short-hand names for kernel per-process open file flags
              from this table:

                   ALLC      allocated
                   BR        the file has been read
                   BHUP      activity stopped by SIGHUP
                   BW        the file has been written
                   CLSG      closing
                   CX        close-on-exec (see fcntl(F_SETFD))
                   LCK       lock was applied
                   MP        memory-mapped
                   OPIP      open pending - in progress
                   RSVW      reserved wait
                   SHMT      UF_FSHMAT set (AIX)
                   USE       in use (multi-threaded)

       NODE-ID
              (or INODE-ADDR for some dialects) contains a unique
              identifier for the file node (usually the kernel vnode or
              inode address, but also occasionally a concatenation of
              device and node number) when n has been specified to +f;

       DEVICE contains the device numbers, separated by commas, for a
              character special, block special, regular, directory or
              NFS file;

              or ``memory'' for a memory file system node under Tru64
              UNIX;

              or the address of the private data area of a Solaris
              socket stream;

              or a kernel reference address that identifies the file
              (The kernel reference address may be used for FIFO's, for
              example.);

              or the base address or device name of a Linux AX.25 socket
              device.

              Usually only the lower thirty two bits of Tru64 UNIX
              kernel addresses are displayed.

       SIZE, SIZE/OFF, or OFFSET
              is the size of the file or the file offset in bytes.  A
              value is displayed in this column only if it is available.
              Lsof displays whatever value - size or offset - is
              appropriate for the type of the file and the version of
              lsof.

              On some UNIX dialects lsof can't obtain accurate or
              consistent file offset information from its kernel data
              sources, sometimes just for particular kinds of files
              (e.g., socket files.)  In other cases, files don't have
              true sizes - e.g., sockets, FIFOs, pipes - so lsof
              displays for their sizes the content amounts it finds in
              their kernel buffer descriptors (e.g., socket buffer size
              counts or TCP/IP window sizes.)  Consult the lsof FAQ (The
              FAQ section gives its location.)  for more information.

              The file size is displayed in decimal; the offset is
              normally displayed in decimal with a leading ``0t'' if it
              contains 8 digits or less; in hexadecimal with a leading
              ``0x'' if it is longer than 8 digits.  (Consult the -o o
              option description for information on when 8 might default
              to some other value.)

              Thus the leading ``0t'' and ``0x'' identify an offset when
              the column may contain both a size and an offset (i.e.,
              its title is SIZE/OFF).

              If the -o option is specified, lsof always displays the
              file offset (or nothing if no offset is available) and
              labels the column OFFSET.  The offset always begins with
              ``0t'' or ``0x'' as described above.

              The lsof user can control the switch from ``0t'' to ``0x''
              with the -o o option.  Consult its description for more
              information.

              If the -s option is specified, lsof always displays the
              file size (or nothing if no size is available) and labels
              the column SIZE.  The -o and -s options are mutually
              exclusive; they can't both be specified.

              For files that don't have a fixed size - e.g., don't
              reside on a disk device - lsof will display appropriate
              information about the current size or position of the file
              if it is available in the kernel structures that define
              the file.

       NLINK  contains the file link count when +L has been specified;

       NODE   is the node number of a local file;

              or the inode number of an NFS file in the server host;

              or the Internet protocol type - e.g, ``TCP'';

              or ``STR'' for a stream;

              or ``CCITT'' for an HP-UX x.25 socket;

              or the IRQ or inode number of a Linux AX.25 socket device.

       NAME   is the name of the mount point and file system on which
              the file resides;

              or the name of a file specified in the names option (after
              any symbolic links have been resolved);

              or the name of a character special or block special
              device;

              or the local and remote Internet addresses of a network
              file; the local host name or IP number is followed by a
              colon (':'), the port, ``->'', and the two-part remote
              address; IP addresses may be reported as numbers or names,
              depending on the +|-M, -n, and -P options; colon-separated
              IPv6 numbers are enclosed in square brackets; IPv4
              INADDR_ANY and IPv6 IN6_IS_ADDR_UNSPECIFIED addresses, and
              zero port numbers are represented by an asterisk ('*'); a
              UDP destination address may be followed by the amount of
              time elapsed since the last packet was sent to the
              destination; TCP, UDP and UDPLITE remote addresses may be
              followed by TCP/TPI information in parentheses - state
              (e.g., ``(ESTABLISHED)'', ``(Unbound)''), queue sizes, and
              window sizes (not all dialects) - in a fashion similar to
              what netstat(1) reports; see the -T option description or
              the description of the TCP/TPI field in OUTPUT FOR OTHER
              PROGRAMS for more information on state, queue size, and
              window size;

              or the address or name of a UNIX domain socket, possibly
              including a stream clone device name, a file system
              object's path name, local and foreign kernel addresses,
              socket pair information, and a bound vnode address;

              or the local and remote mount point names of an NFS file;

              or ``STR'', followed by the stream name;

              or a stream character device name, followed by ``->'' and
              the stream name or a list of stream module names,
              separated by ``->'';

              or ``STR:'' followed by the SCO OpenServer stream device
              and module names, separated by ``->'';

              or system directory name, `` -- '', and as many components
              of the path name as lsof can find in the kernel's name
              cache for selected dialects (See the KERNEL NAME CACHE
              section for more information.);

              or ``PIPE->'', followed by a Solaris kernel pipe
              destination address;

              or ``COMMON:'', followed by the vnode device information
              structure's device name, for a Solaris common vnode;

              or the address family, followed by a slash (`/'), followed
              by fourteen comma-separated bytes of a non-Internet raw
              socket address;

              or the HP-UX x.25 local address, followed by the virtual
              connection number (if any), followed by the remote address
              (if any);

              or ``(dead)'' for disassociated Tru64 UNIX files -
              typically terminal files that have been flagged with the
              TIOCNOTTY ioctl and closed by daemons;

              or ``rd=<offset>'' and ``wr=<offset>'' for the values of
              the read and write offsets of a FIFO;

              or ``clone n:/dev/event'' for SCO OpenServer file clones
              of the /dev/event device, where n is the minor device
              number of the file;

              or ``(socketpair: n)'' for a Solaris 2.6, 8, 9  or 10 UNIX
              domain socket, created by the socketpair(3N) network
              function;

              or ``no PCB'' for socket files that do not have a protocol
              block associated with them, optionally followed by ``,
              CANTSENDMORE'' if sending on the socket has been disabled,
              or ``, CANTRCVMORE'' if receiving on the socket has been
              disabled (e.g., by the shutdown(2) function);

              or the local and remote addresses of a Linux IPX socket
              file in the form <net>:[<node>:]<port>, followed in
              parentheses by the transmit and receive queue sizes, and
              the connection state;

              or ``dgram'' or ``stream'' for the type UnixWare 7.1.1 and
              above in-kernel UNIX domain sockets, followed by a colon
              (':') and the local path name when available, followed by
              ``->'' and the remote path name or kernel socket address
              in hexadecimal when available;

              or the association value, association index, endpoint
              value, local address, local port, remote address and
              remote port for Linux SCTP sockets;

              or ``protocol: '' followed by the Linux socket's protocol
              attribute.

       For dialects that support a ``namefs'' file system, allowing one
       file to be attached to another with fattach(3C), lsof will add
       ``(FA:<address1><direction><address2>)'' to the NAME column.
       <address1> and <address2> are hexadecimal vnode addresses.
       <direction> will be ``<-'' if <address2> has been fattach'ed to
       this vnode whose address is <address1>; and ``->'' if <address1>,
       the vnode address of this vnode, has been fattach'ed to
       <address2>.  <address1> may be omitted if it already appears in
       the DEVICE column.

       Lsof may add two parenthetical notes to the NAME column for open
       Solaris 10 files: ``(?)'' if lsof considers the path name of
       questionable accuracy; and ``(deleted)'' if the -X option has
       been specified and lsof detects the open file's path name has
       been deleted.  Consult the lsof FAQ (The FAQ section gives its
       location.)  for more information on these NAME column additions.
LOCKS
       Lsof can't adequately report the wide variety of UNIX dialect
       file locks in a single character.  What it reports in a single
       character is a compromise between the information it finds in the
       kernel and the limitations of the reporting format.

       Moreover, when a process holds several byte level locks on a
       file, lsof only reports the status of the first lock it
       encounters.  If it is a byte level lock, then the lock character
       will be reported in lower case - i.e., `r', `w', or `x' - rather
       than the upper case equivalent reported for a full file lock.

       Generally lsof can only report on locks held by local processes
       on local files.  When a local process sets a lock on a remotely
       mounted (e.g., NFS) file, the remote server host usually records
       the lock state.  One exception is Solaris - at some patch levels
       of 2.3, and in all versions above 2.4, the Solaris kernel records
       information on remote locks in local structures.

       Lsof has trouble reporting locks for some UNIX dialects.  Consult
       the BUGS section of this manual page or the lsof FAQ (The FAQ
       section gives its location.)  for more information.
OUTPUT FOR OTHER PROGRAMS
       When the -F option is specified, lsof produces output that is
       suitable for processing by another program - e.g, an awk or Perl
       script, or a C program.

       Each unit of information is output in a field that is identified
       with a leading character and terminated by a NL (012) (or a NUL
       (000) if the 0 (zero) field identifier character is specified.)
       The data of the field follows immediately after the field
       identification character and extends to the field terminator.

       It is possible to think of field output as process and file sets.
       A process set begins with a field whose identifier is `p' (for
       process IDentifier (PID)).  It extends to the beginning of the
       next PID field or the beginning of the first file set of the
       process, whichever comes first.  Included in the process set are
       fields that identify the command, the process group
       IDentification (PGID) number, the task (thread) ID (TID), and the
       user ID (UID) number or login name.

       A file set begins with a field whose identifier is `f' (for file
       descriptor).  It is followed by lines that describe the file's
       access mode, lock state, type, device, size, offset, inode,
       protocol, name and stream module names.  It extends to the
       beginning of the next file or process set, whichever comes first.

       When the NUL (000) field terminator has been selected with the 0
       (zero) field identifier character, lsof ends each process and
       file set with a NL (012) character.

       Lsof always produces one field, the PID (`p') field.  All other
       fields may be declared optionally in the field identifier
       character list that follows the -F option.  When a field
       selection character identifies an item lsof does not normally
       list - e.g., PPID, selected with -R - specification of the field
       character - e.g., ``-FR'' - also selects the listing of the item.

       It is entirely possible to select a set of fields that cannot
       easily be parsed - e.g., if the field descriptor field is not
       selected, it may be difficult to identify file sets.  To help you
       avoid this difficulty, lsof supports the -F option; it selects
       the output of all fields with NL terminators (the -F0 option pair
       selects the output of all fields with NUL terminators).  For
       compatibility reasons neither -F nor -F0 select the raw device
       field.

       These are the fields that lsof will produce.  The single
       character listed first is the field identifier.

            a    file access mode
            c    process command name (all characters from proc or
                 user structure)
            C    file structure share count
            d    file's device character code
            D    file's major/minor device number (0x<hexadecimal>)
            f    file descriptor (always selected)
            F    file structure address (0x<hexadecimal>)
            G    file flaGs (0x<hexadecimal>; names if +fg follows)
            g    process group ID
            i    file's inode number
            K    tasK ID
            k    link count
            l    file's lock status
            L    process login name
            m    marker between repeated output
            M    the task comMand name
            n    file name, comment, Internet address
            N    node identifier (ox<hexadecimal>
            o    file's offset (decimal)
            p    process ID (always selected)
            P    protocol name
            r    raw device number (0x<hexadecimal>)
            R    parent process ID
            s    file's size (decimal)
            S    file's stream identification
            t    file's type
            T    TCP/TPI information, identified by prefixes (the
                 `=' is part of the prefix):
                     QR=<read queue size>
                     QS=<send queue size>
                     SO=<socket options and values> (not all dialects)
                     SS=<socket states> (not all dialects)
                     ST=<connection state>
                     TF=<TCP flags and values> (not all dialects)
                     WR=<window read size>  (not all dialects)
                     WW=<window write size>  (not all dialects)
                 (TCP/TPI information isn't reported for all supported
                   UNIX dialects. The -h or -? help output for the
                   -T option will show what TCP/TPI reporting can be
                   requested.)
            u    process user ID
            z    Solaris 10 and higher zone name
            Z    SELinux security context (inhibited when SELinux is disabled)
            0    use NUL field terminator character in place of NL
            1-9  dialect-specific field identifiers (The output
                 of -F? identifies the information to be found
                 in dialect-specific fields.)

       You can get on-line help information on these characters and
       their descriptions by specifying the -F?  option pair.  (Escape
       the `?' character as your shell requires.)  Additional
       information on field content can be found in the OUTPUT section.

       As an example, ``-F pcfn'' will select the process ID (`p'),
       command name (`c'), file descriptor (`f') and file name (`n')
       fields with an NL field terminator character; ``-F pcfn0''
       selects the same output with a NUL (000) field terminator
       character.

       Lsof doesn't produce all fields for every process or file set,
       only those that are available.  Some fields are mutually
       exclusive: file device characters and file major/minor device
       numbers; file inode number and protocol name; file name and
       stream identification; file size and offset.  One or the other
       member of these mutually exclusive sets will appear in field
       output, but not both.

       Normally lsof ends each field with a NL (012) character.  The 0
       (zero) field identifier character may be specified to change the
       field terminator character to a NUL (000).  A NUL terminator may
       be easier to process with xargs(1), for example, or with programs
       whose quoting mechanisms may not easily cope with the range of
       characters in the field output.  When the NUL field terminator is
       in use, lsof ends each process and file set with a NL (012).

       Three aids to producing programs that can process lsof field
       output are included in the lsof distribution.  The first is a C
       header file, lsof_fields.h, that contains symbols for the field
       identification characters, indexes for storing them in a table,
       and explanation strings that may be compiled into programs.  Lsof
       uses this header file.

       The second aid is a set of sample scripts that process field
       output, written in awk, Perl 4, and Perl 5.  They're located in
       the scripts subdirectory of the lsof distribution.

       The third aid is the C library used for the lsof test suite.  The
       test suite is written in C and uses field output to validate the
       correct operation of lsof.  The library can be found in the
       tests/LTlib.c file of the lsof distribution.  The library uses
       the first aid, the lsof_fields.h header file.
BLOCKS AND TIMEOUTS
       Lsof can be blocked by some kernel functions that it uses -
       lstat(2), readlink(2), and stat(2).  These functions are stalled
       in the kernel, for example, when the hosts where mounted NFS file
       systems reside become inaccessible.

       Lsof attempts to break these blocks with timers and child
       processes, but the techniques are not wholly reliable.  When lsof
       does manage to break a block, it will report the break with an
       error message.  The messages may be suppressed with the -t and -w
       options.

       The default timeout value may be displayed with the -h or -?
       option, and it may be changed with the -S [t] option.  The
       minimum for t is two seconds, but you should avoid small values,
       since slow system responsiveness can cause short timeouts to
       expire unexpectedly and perhaps stop lsof before it can produce
       any output.

       When lsof has to break a block during its access of mounted file
       system information, it normally continues, although with less
       information available to display about open files.

       Lsof can also be directed to avoid the protection of timers and
       child processes when using the kernel functions that might block
       by specifying the -O option.  While this will allow lsof to start
       up with less overhead, it exposes lsof completely to the kernel
       situations that might block it.  Use this option cautiously.
AVOIDING KERNEL BLOCKS
       You can use the -b option to tell lsof to avoid using kernel
       functions that would block.  Some cautions apply.

       First, using this option usually requires that your system supply
       alternate device numbers in place of the device numbers that lsof
       would normally obtain with the lstat(2) and stat(2) kernel
       functions.  See the ALTERNATE DEVICE NUMBERS section for more
       information on alternate device numbers.

       Second, you can't specify names for lsof to locate unless they're
       file system names.  This is because lsof needs to know the device
       and inode numbers of files listed with names in the lsof options,
       and the -b option prevents lsof from obtaining them.  Moreover,
       since lsof only has device numbers for the file systems that have
       alternates, its ability to locate files on file systems depends
       completely on the availability and accuracy of the alternates.
       If no alternates are available, or if they're incorrect, lsof
       won't be able to locate files on the named file systems.

       Third, if the names of your file system directories that lsof
       obtains from your system's mount table are symbolic links, lsof
       won't be able to resolve the links.  This is because the -b
       option causes lsof to avoid the kernel readlink(2) function it
       uses to resolve symbolic links.

       Finally, using the -b option causes lsof to issue warning
       messages when it needs to use the kernel functions that the -b
       option directs it to avoid.  You can suppress these messages by
       specifying the -w option, but if you do, you won't see the
       alternate device numbers reported in the warning messages.
ALTERNATE DEVICE NUMBERS
       On some dialects, when lsof has to break a block because it can't
       get information about a mounted file system via the lstat(2) and
       stat(2) kernel functions, or because you specified the -b option,
       lsof can obtain some of the information it needs - the device
       number and possibly the file system type - from the system mount
       table.  When that is possible, lsof will report the device number
       it obtained.  (You can suppress the report by specifying the -w
       option.)

       You can assist this process if your mount table is supported with
       an /etc/mtab or /etc/mnttab file that contains an options field
       by adding a ``dev=xxxx'' field for mount points that do not have
       one in their options strings.  Note: you must be able to edit the
       file - i.e., some mount tables like recent Solaris /etc/mnttab or
       Linux /proc/mounts are read-only and can't be modified.

       You may also be able to supply device numbers using the +m and +m
       m options, provided they are supported by your dialect.  Check
       the output of lsof's -h or -?  options to see if the +m and +m m
       options are available.

       The ``xxxx'' portion of the field is the hexadecimal value of the
       file system's device number.  (Consult the st_dev field of the
       output of the lstat(2) and stat(2) functions for the appropriate
       values for your file systems.)  Here's an example from a Sun
       Solaris 2.6 /etc/mnttab for a file system remotely mounted via
       NFS:

            nfs  ignore,noquota,dev=2a40001

       There's an advantage to having ``dev=xxxx'' entries in your mount
       table file, especially for file systems that are mounted from
       remote NFS servers.  When a remote server crashes and you want to
       identify its users by running lsof on one of its clients, lsof
       probably won't be able to get output from the lstat(2) and
       stat(2) functions for the file system.  If it can obtain the file
       system's device number from the mount table, it will be able to
       display the files open on the crashed NFS server.

       Some dialects that do not use an ASCII /etc/mtab or /etc/mnttab
       file for the mount table may still provide an alternative device
       number in their internal mount tables.  This includes AIX, Apple
       Darwin, FreeBSD, NetBSD, OpenBSD, and Tru64 UNIX.  Lsof knows how
       to obtain the alternative device number for these dialects and
       uses it when its attempt to lstat(2) or stat(2) the file system
       is blocked.

       If you're not sure your dialect supplies alternate device numbers
       for file systems from its mount table, use this lsof incantation
       to see if it reports any alternate device numbers:

              lsof -b

       Look for standard error file warning messages that begin
       ``assuming "dev=xxxx" from ...''.
KERNEL NAME CACHE
       Lsof is able to examine the kernel's name cache or use other
       kernel facilities (e.g., the ADVFS 4.x tag_to_path() function
       under Tru64 UNIX) on some dialects for most file system types,
       excluding AFS, and extract recently used path name components
       from it.  (AFS file system path lookups don't use the kernel's
       name cache; some Solaris VxFS file system operations apparently
       don't use it, either.)

       Lsof reports the complete paths it finds in the NAME column.  If
       lsof can't report all components in a path, it reports in the
       NAME column the file system name, followed by a space, two `-'
       characters, another space, and the name components it has
       located, separated by the `/' character.

       When lsof is run in repeat mode - i.e., with the -r option
       specified - the extent to which it can report path name
       components for the same file may vary from cycle to cycle.
       That's because other running processes can cause the kernel to
       remove entries from its name cache and replace them with others.

       Lsof's use of the kernel name cache to identify the paths of
       files can lead it to report incorrect components under some
       circumstances.  This can happen when the kernel name cache uses
       device and node number as a key (e.g., SCO OpenServer) and a key
       on a rapidly changing file system is reused.  If the UNIX
       dialect's kernel doesn't purge the name cache entry for a file
       when it is unlinked, lsof may find a reference to the wrong entry
       in the cache.  The lsof FAQ (The FAQ section gives its location.)
       has more information on this situation.

       Lsof can report path name components for these dialects:

            FreeBSD
            HP-UX
            Linux
            NetBSD
            NEXTSTEP
            OpenBSD
            OPENSTEP
            SCO OpenServer
            SCO|Caldera UnixWare
            Solaris
            Tru64 UNIX

       Lsof can't report path name components for these dialects:

            AIX

       If you want to know why lsof can't report path name components
       for some dialects, see the lsof FAQ (The FAQ section gives its
       location.)
DEVICE CACHE FILE
       Examining all members of the /dev (or /devices) node tree with
       stat(2) functions can be time consuming.  What's more, the
       information that lsof needs - device number, inode number, and
       path - rarely changes.

       Consequently, lsof normally maintains an ASCII text file of
       cached /dev (or /devices) information (exception: the /proc-based
       Linux lsof where it's not needed.)  The local system
       administrator who builds lsof can control the way the device
       cache file path is formed, selecting from these options:

            Path from the -D option;
            Path from an environment variable;
            System-wide path;
            Personal path (the default);
            Personal path, modified by an environment variable.

       Consult the output of the -h, -D? , or -?  help options for the
       current state of device cache support.  The help output lists the
       default read-mode device cache file path that is in effect for
       the current invocation of lsof.  The -D?  option output lists the
       read-only and write device cache file paths, the names of any
       applicable environment variables, and the personal device cache
       path format.

       Lsof can detect that the current device cache file has been
       accidentally or maliciously modified by integrity checks,
       including the computation and verification of a sixteen bit
       Cyclic Redundancy Check (CRC) sum on the file's contents.  When
       lsof senses something wrong with the file, it issues a warning
       and attempts to remove the current cache file and create a new
       copy, but only to a path that the process can legitimately write.

       The path from which a lsof process may attempt to read a device
       cache file may not be the same as the path to which it can
       legitimately write.  Thus when lsof senses that it needs to
       update the device cache file, it may choose a different path for
       writing it from the path from which it read an incorrect or
       outdated version.

       If available, the -Dr option will inhibit the writing of a new
       device cache file.  (It's always available when specified without
       a path name argument.)

       When a new device is added to the system, the device cache file
       may need to be recreated.  Since lsof compares the mtime of the
       device cache file with the mtime and ctime of the /dev (or
       /devices) directory, it usually detects that a new device has
       been added; in that case lsof issues a warning message and
       attempts to rebuild the device cache file.

       Whenever lsof writes a device cache file, it sets its ownership
       to the real UID of the executing process, and its permission
       modes to 0600, this restricting its reading and writing to the
       file's owner.
LSOF PERMISSIONS THAT AFFECT DEVICE CACHE FILE ACCESS
       Two permissions of the lsof executable affect its ability to
       access device cache files.  The permissions are set by the local
       system administrator when lsof is installed.

       The first and rarer permission is setuid-root.  It comes into
       effect when lsof is executed; its effective UID is then root,
       while its real (i.e., that of the logged-on user) UID is not.
       The lsof distribution recommends that versions for these dialects
       run setuid-root.

            HP-UX 11.11 and 11.23
            Linux

       The second and more common permission is setgid.  It comes into
       effect when the effective group IDentification number (GID) of
       the lsof process is set to one that can access kernel memory
       devices - e.g., ``kmem'', ``sys'', or ``system''.

       An lsof process that has setgid permission usually surrenders the
       permission after it has accessed the kernel memory devices.  When
       it does that, lsof can allow more liberal device cache path
       formations.  The lsof distribution recommends that versions for
       these dialects run setgid and be allowed to surrender setgid
       permission.

            AIX 5.[12] and 5.3-ML1
            Apple Darwin 7.x Power Macintosh systems
            FreeBSD 4.x, 4.1x, 5.x and [6789].x for x86-based systems
            FreeBSD 5.x, [6789].x and 1[012].8for Alpha, AMD64 and Sparc64
                based systems
            HP-UX 11.00
            NetBSD 1.[456], 2.x and 3.x for Alpha, x86, and SPARC-based
                systems
            NEXTSTEP 3.[13] for NEXTSTEP architectures
            OpenBSD 2.[89] and 3.[0-9] for x86-based systems
            OPENSTEP 4.x
            SCO OpenServer Release 5.0.6 for x86-based systems
            SCO|Caldera UnixWare 7.1.4 for x86-based systems
            Solaris 2.6, 8, 9 and 10
            Tru64 UNIX 5.1

       (Note: lsof for AIX 5L and above needs setuid-root permission if
       its -X option is used.)

       Lsof for these dialects does not support a device cache, so the
       permissions given to the executable don't apply to the device
       cache file.

            Linux
DEVICE CACHE FILE PATH FROM THE -D OPTION
       The -D option provides limited means for specifying the device
       cache file path.  Its ?  function will report the read-only and
       write device cache file paths that lsof will use.

       When the -D b, r, and u functions are available, you can use them
       to request that the cache file be built in a specific location
       (b[path]); read but not rebuilt (r[path]); or read and rebuilt
       (u[path]).  The b, r, and u functions are restricted under some
       conditions.  They are restricted when the lsof process is
       setuid-root.  The path specified with the r function is always
       read-only, even when it is available.

       The b, r, and u functions are also restricted when the lsof
       process runs setgid and lsof doesn't surrender the setgid
       permission.  (See the LSOF PERMISSIONS THAT AFFECT DEVICE CACHE
       FILE ACCESS section for a list of implementations that normally
       don't surrender their setgid permission.)

       A further -D function, i (for ignore), is always available.

       When available, the b function tells lsof to read device
       information from the kernel with the stat(2) function and build a
       device cache file at the indicated path.

       When available, the r function tells lsof to read the device
       cache file, but not update it.  When a path argument accompanies
       -Dr, it names the device cache file path.  The r function is
       always available when it is specified without a path name
       argument.  If lsof is not running setuid-root and surrenders its
       setgid permission, a path name argument may accompany the r
       function.

       When available, the u function tells lsof to attempt to read and
       use the device cache file.  If it can't read the file, or if it
       finds the contents of the file incorrect or outdated, it will
       read information from the kernel, and attempt to write an updated
       version of the device cache file, but only to a path it considers
       legitimate for the lsof process effective and real UIDs.
DEVICE CACHE PATH FROM AN ENVIRONMENT VARIABLE
       Lsof's second choice for the device cache file is the contents of
       the LSOFDEVCACHE environment variable.  It avoids this choice if
       the lsof process is setuid-root, or the real UID of the process
       is root.

       A further restriction applies to a device cache file path taken
       from the LSOFDEVCACHE environment variable: lsof will not write a
       device cache file to the path if the lsof process doesn't
       surrender its setgid permission.  (See the LSOF PERMISSIONS THAT
       AFFECT DEVICE CACHE FILE ACCESS section for information on
       implementations that don't surrender their setgid permission.)

       The local system administrator can disable the use of the
       LSOFDEVCACHE environment variable or change its name when
       building lsof.  Consult the output of -D?  for the environment
       variable's name.
SYSTEM-WIDE DEVICE CACHE PATH
       The local system administrator may choose to have a system-wide
       device cache file when building lsof.  That file will generally
       be constructed by a special system administration procedure when
       the system is booted or when the contents of /dev or /devices)
       changes.  If defined, it is lsof's third device cache file path
       choice.

       You can tell that a system-wide device cache file is in effect
       for your local installation by examining the lsof help option
       output - i.e., the output from the -h or -?  option.

       Lsof will never write to the system-wide device cache file path
       by default.  It must be explicitly named with a -D function in a
       root-owned procedure.  Once the file has been written, the
       procedure must change its permission modes to 0644 (owner-read
       and owner-write, group-read, and other-read).
PERSONAL DEVICE CACHE PATH (DEFAULT)
       The default device cache file path of the lsof distribution is
       one recorded in the home directory of the real UID that executes
       lsof.  Added to the home directory is a second path component of
       the form .lsof_hostname.

       This is lsof's fourth device cache file path choice, and is
       usually the default.  If a system-wide device cache file path was
       defined when lsof was built, this fourth choice will be applied
       when lsof can't find the system-wide device cache file.  This is
       the only time lsof uses two paths when reading the device cache
       file.

       The hostname part of the second component is the base name of the
       executing host, as returned by gethostname(2).  The base name is
       defined to be the characters preceding the first `.'  in the
       gethostname(2) output, or all the gethostname(2) output if it
       contains no `.'.

       The device cache file belongs to the user ID and is readable and
       writable by the user ID alone - i.e., its modes are 0600.  Each
       distinct real user ID on a given host that executes lsof has a
       distinct device cache file.  The hostname part of the path
       distinguishes device cache files in an NFS-mounted home directory
       into which device cache files are written from several different
       hosts.

       The personal device cache file path formed by this method
       represents a device cache file that lsof will attempt to read,
       and will attempt to write should it not exist or should its
       contents be incorrect or outdated.

       The -Dr option without a path name argument will inhibit the
       writing of a new device cache file.

       The -D?  option will list the format specification for
       constructing the personal device cache file.  The conversions
       used in the format specification are described in the 00DCACHE
       file of the lsof distribution.
MODIFIED PERSONAL DEVICE CACHE PATH
       If this option is defined by the local system administrator when
       lsof is built, the LSOFPERSDCPATH environment variable contents
       may be used to add a component of the personal device cache file
       path.

       The LSOFPERSDCPATH variable contents are inserted in the path at
       the place marked by the local system administrator with the
       ``%p'' conversion in the HASPERSDC format specification of the
       dialect's machine.h header file.  (It's placed right after the
       home directory in the default lsof distribution.)

       Thus, for example, if LSOFPERSDCPATH contains ``LSOF'', the home
       directory is ``/Homes/abe'', the host name is
       ``lsof.itap.purdue.edu'', and the HASPERSDC format is the default
       (``%h/%p.lsof_%L''), the modified personal device cache file path
       is:

            /Homes/abe/LSOF/.lsof_vic

       The LSOFPERSDCPATH environment variable is ignored when the lsof
       process is setuid-root or when the real UID of the process is
       root.

       Lsof will not write to a modified personal device cache file path
       if the lsof process doesn't surrender setgid permission.  (See
       the LSOF PERMISSIONS THAT AFFECT DEVICE CACHE FILE ACCESS section
       for a list of implementations that normally don't surrender their
       setgid permission.)

       If, for example, you want to create a sub-directory of personal
       device cache file paths by using the LSOFPERSDCPATH environment
       variable to name it, and lsof doesn't surrender its setgid
       permission, you will have to allow lsof to create device cache
       files at the standard personal path and move them to your
       subdirectory with shell commands.

       The local system administrator may: disable this option when lsof
       is built; change the name of the environment variable from
       LSOFPERSDCPATH to something else; change the HASPERSDC format to
       include the personal path component in another place; or exclude
       the personal path component entirely.  Consult the output of the
       -D?  option for the environment variable's name and the HASPERSDC
       format specification.
DIAGNOSTICS
       Errors are identified with messages on the standard error file.

       Lsof returns a one (1) if any error was detected, including the
       failure to locate command names, file names, Internet addresses
       or files, login names, NFS files, PIDs, PGIDs, or UIDs it was
       asked to list.  If the -V option is specified, lsof will indicate
       the search items it failed to list.

       It returns a zero (0) if no errors were detected and if it was
       able to list some information about all the specified search
       arguments.

       When lsof cannot open access to /dev (or /devices) or one of its
       subdirectories, or get information on a file in them with
       stat(2), it issues a warning message and continues.  That lsof
       will issue warning messages about inaccessible files in /dev (or
       /devices) is indicated in its help output - requested with the -h
       or >B -?  options -  with the message:

            Inaccessible /dev warnings are enabled.

       The warning message may be suppressed with the -w option.  It may
       also have been suppressed by the system administrator when lsof
       was compiled by the setting of the WARNDEVACCESS definition.  In
       this case, the output from the help options will include the
       message:

            Inaccessible /dev warnings are disabled.

       Inaccessible device warning messages usually disappear after lsof
       has created a working device cache file.
EXAMPLES
       For a more extensive set of examples, documented more fully, see
       the 00QUICKSTART file of the lsof distribution.

       To list all open files, use:

              lsof

       To list all open Internet, x.25 (HP-UX), and UNIX domain files,
       use:

              lsof -i -U

       To list all open IPv4 network files in use by the process whose
       PID is 1234, use:

              lsof -i 4 -a -p 1234

       Presuming the UNIX dialect supports IPv6, to list only open IPv6
       network files, use:

              lsof -i 6

       To list all files using any protocol on ports 513, 514, or 515 of
       host wonderland.cc.purdue.edu, use:

              lsof -i @wonderland.cc.purdue.edu:513-515

       To list all files using any protocol on any port of
       mace.cc.purdue.edu (cc.purdue.edu is the default domain), use:

              lsof -i @mace

       To list all open files for login name ``abe'', or user ID 1234,
       or process 456, or process 123, or process 789, use:

              lsof -p 456,123,789 -u 1234,abe

       To list all open files on device /dev/hd4, use:

              lsof /dev/hd4

       To find the process that has /u/abe/foo open, use:

              lsof /u/abe/foo

       To send a SIGHUP to the processes that have /u/abe/bar open, use:

              kill -HUP `lsof -t /u/abe/bar`

       To find any open file, including an open UNIX domain socket file,
       with the name /dev/log, use:

              lsof /dev/log

       To find processes with open files on the NFS file system named
       /nfs/mount/point whose server is inaccessible, and presuming your
       mount table supplies the device number for /nfs/mount/point, use:

              lsof -b /nfs/mount/point

       To do the preceding search with warning messages suppressed, use:

              lsof -bw /nfs/mount/point

       To ignore the device cache file, use:

              lsof -Di

       To obtain PID and command name field output for each process,
       file descriptor, file device number, and file inode number for
       each file of each process, use:

              lsof -FpcfDi

       To list the files at descriptors 1 and 3 of every process running
       the lsof command for login ID ``abe'' every 10 seconds, use:

              lsof -c lsof -a -d 1 -d 3 -u abe -r10

       To list the current working directory of processes running a
       command that is exactly four characters long and has an 'o' or
       'O' in character three, use this regular expression form of the
       -c c option:

              lsof -c /^..o.$/i -a -d cwd

       To find an IP version 4 socket file by its associated numeric
       dot-form address, use:

              lsof -i@128.210.15.17

       To find an IP version 6 socket file (when the UNIX dialect
       supports IPv6) by its associated numeric colon-form address, use:

              lsof -i@[0:1:2:3:4:5:6:7]

       To find an IP version 6 socket file (when the UNIX dialect
       supports IPv6) by an associated numeric colon-form address that
       has a run of zeroes in it - e.g., the loop-back address - use:

              lsof -i@[::1]

       To obtain a repeat mode marker line that contains the current
       time, use:

              lsof -rm====%T====

       To add spaces to the previous marker line, use:

              lsof -r "m==== %T ===="
BUGS
       Since lsof reads kernel memory in its search for open files,
       rapid changes in kernel memory may produce unpredictable results.

       When a file has multiple record locks, the lock status character
       (following the file descriptor) is derived from a test of the
       first lock structure, not from any combination of the individual
       record locks that might be described by multiple lock structures.

       Lsof can't search for files with restrictive access permissions
       by name unless it is installed with root set-UID permission.
       Otherwise it is limited to searching for files to which its user
       or its set-GID group (if any) has access permission.

       The display of the destination address of a raw socket (e.g., for
       ping) depends on the UNIX operating system.  Some dialects store
       the destination address in the raw socket's protocol control
       block, some do not.

       Lsof can't always represent Solaris device numbers in the same
       way that ls(1) does.  For example, the major and minor device
       numbers that the lstat(2) and stat(2) functions report for the
       directory on which CD-ROM files are mounted (typically /cdrom)
       are not the same as the ones that it reports for the device on
       which CD-ROM files are mounted (typically /dev/sr0).  (Lsof
       reports the directory numbers.)

       The support for /proc file systems is available only for BSD and
       Tru64 UNIX dialects, Linux, and dialects derived from SYSV R4 -
       e.g., FreeBSD, NetBSD, OpenBSD, Solaris, UnixWare.

       Some /proc file items - device number, inode number, and file
       size - are unavailable in some dialects.  Searching for files in
       a /proc file system may require that the full path name be
       specified.

       No text (txt) file descriptors are displayed for Linux processes.
       All entries for files other than the current working directory,
       the root directory, and numerical file descriptors are labeled
       mem descriptors.

       Lsof can't search for Tru64 UNIX named pipes by name, because
       their kernel implementation of lstat(2) returns an improper
       device number for a named pipe.

       Lsof can't report fully or correctly on HP-UX 9.01, 10.20, and
       11.00 locks because of insufficient access to kernel data or
       errors in the kernel data.  See the lsof FAQ (The FAQ section
       gives its location.)  for details.

       The AIX SMT file type is a fabrication.  It's made up for file
       structures whose type (15) isn't defined in the AIX
       /usr/include/sys/file.h header file.  One way to create such file
       structures is to run X clients with the DISPLAY variable set to
       ``:0.0''.

       The +|-f[cfgGn] option is not supported under /proc-based Linux
       lsof, because it doesn't read kernel structures from kernel
       memory.
ENVIRONMENT
       Lsof may access these environment variables.

       LANG   defines a language locale.  See setlocale(3) for the names
              of other variables that can be used in place of LANG -
              e.g., LC_ALL, LC_TYPE, etc.

       LSOFDEVCACHE
              defines the path to a device cache file.  See the DEVICE
              CACHE PATH FROM AN ENVIRONMENT VARIABLE section for more
              information.

       LSOFPERSDCPATH
              defines the middle component of a modified personal device
              cache file path.  See the MODIFIED PERSONAL DEVICE CACHE
              PATH section for more information.
FAQ
       Frequently-asked questions and their answers (an FAQ) are
       available in the 00FAQ file of the lsof distribution.

       That file is also available via anonymous ftp from
       lsof.itap.purdue.edu at pub/tools/unix/lsofFAQ.  The URL is:

              ftp://lsof.itap.purdue.edu/pub/tools/unix/lsof/FAQ
FILES
       /dev/kmem
              kernel virtual memory device

       /dev/mem
              physical memory device

       /dev/swap
              system paging device

       .lsof_hostname
              lsof's device cache file (The suffix, hostname, is the
              first component of the host's name returned by
              gethostname(2).)
AUTHORS
       Lsof was written by Victor A.Abell <abe@purdue.edu> of Purdue
       University.  Many others have contributed to lsof.  They're
       listed in the 00CREDITS file of the lsof distribution.
DISTRIBUTION
       The latest distribution of lsof is available via anonymous ftp
       from the host lsof.itap.purdue.edu.  You'll find the lsof
       distribution in the pub/tools/unix/lsof directory.

       You can also use this URL:

              ftp://lsof.itap.purdue.edu/pub/tools/unix/lsof

       Lsof is also mirrored elsewhere.  When you access
       lsof.itap.purdue.edu and change to its pub/tools/unix/lsof
       directory, you'll be given a list of some mirror sites.  The
       pub/tools/unix/lsof directory also contains a more complete list
       in its mirrors file.  Use mirrors with caution - not all mirrors
       always have the latest lsof revision.

       Some pre-compiled Lsof executables are available on
       lsof.itap.purdue.edu, but their use is discouraged - it's better
       that you build your own from the sources.  If you feel you must
       use a pre-compiled executable, please read the cautions that
       appear in the README files of the pub/tools/unix/lsof/binaries
       subdirectories and in the 00* files of the distribution.

       More information on the lsof distribution can be found in its
       README.lsof_<version> file.  If you intend to get the lsof
       distribution and build it, please read README.lsof_<version> and
       the other 00* files of the distribution before sending questions
       to the author.
SEE ALSO
       Not all the following manual pages may exist in every UNIX
       dialect to which lsof has been ported.

       access(2), awk(1), crash(1), fattach(3C), ff(1), fstat(8),
       fuser(1), gethostname(2), isprint(3), kill(1), localtime(3),
       lstat(2), modload(8), mount(8), netstat(1), ofiles(8L), perl(1),
       ps(1), readlink(2), setlocale(3), stat(2), strftime(3), time(2),
       uname(1).
COLOPHON
       This page is part of the lsof (LiSt Open Files) project.
       Information about the project can be found at 
       http://people.freebsd.org/~abe/.  If you have a bug report for
       this manual page, send it to abe@purdue.edu.  This page was
       obtained from the tarball lsof_4.91_src.tar fetched from
       ftp://ftp.fu-berlin.de/pub/unix/tools/lsof/lsof.tar.gz on
       2024-06-14.  If you discover any rendering problems in this HTML
       version of the page, or you believe there is a better or more up-
       to-date source for the page, or you have corrections or
       improvements to the information in this COLOPHON (which is not
       part of the original manual page), send a mail to
       man-pages@man7.org

                              Revision-4.91                      LSOF(8)\end{lstlisting}
}}
\endinput  %  ==  ==  ==  ==  ==  ==  ==  ==  ==
