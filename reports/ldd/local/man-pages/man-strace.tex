% % % % \input{./components/man/man-strace}
\subsection{\refStrace: Trace System Calls and Signals}

{\tiny{
\begin{lstlisting}[language=bash]
NAME
       strace - trace system calls and signals
SYNOPSIS
       strace [-ACdffhikkqqrtttTvVwxxyyYzZ] [-a column] [-b execve]
              [-e expr]... [-I n] [-o file] [-O overhead] [-p pid]...
              [-P path]... [-s strsize] [-S sortby] [-U columns]
              [-X format] [--seccomp-bpf]
              [--stack-trace-frame-limit=limit] [--syscall-limit=limit]
              [--secontext[=format]] [--tips[=format]] { -p pid | [-DDD]
              [-E var[=val]]... [-u username] command [args] }

       strace -c [-dfwzZ] [-b execve] [-e expr]... [-I n] [-O overhead]
              [-p pid]... [-P path]... [-S sortby] [-U columns]
              [--seccomp-bpf] [--syscall-limit=limit] [--tips[=format]]
              { -p pid | [-DDD] [-E var[=val]]... [-u username] command
              [args] }

       strace --tips[=format]
DESCRIPTION
       In the simplest case strace runs the specified command until it
       exits.  It intercepts and records the system calls which are
       called by a process and the signals which are received by a
       process.  The name of each system call, its arguments and its
       return value are printed on standard error or to the file
       specified with the -o option.

       strace is a useful diagnostic, instructional, and debugging tool.
       System administrators, diagnosticians and trouble-shooters will
       find it invaluable for solving problems with programs for which
       the source is not readily available since they do not need to be
       recompiled in order to trace them.  Students, hackers and the
       overly-curious will find that a great deal can be learned about a
       system and its system calls by tracing even ordinary programs.
       And programmers will find that since system calls and signals are
       events that happen at the user/kernel interface, a close
       examination of this boundary is very useful for bug isolation,
       sanity checking and attempting to capture race conditions.

       Each line in the trace contains the system call name, followed by
       its arguments in parentheses and its return value.  An example
       from stracing the command "cat /dev/null" is:

           open("/dev/null", O_RDONLY) = 3

       Errors (typically a return value of -1) have the errno symbol and
       error string appended.

           open("/foo/bar", O_RDONLY) = -1 ENOENT (No such file or directory)

       Signals are printed as signal symbol and decoded siginfo
       structure.  An excerpt from stracing and interrupting the command
       "sleep 666" is:

           sigsuspend([] <unfinished ...>
           --- SIGINT {si_signo=SIGINT, si_code=SI_USER, si_pid=...} ---
           +++ killed by SIGINT +++

       If a system call is being executed and meanwhile another one is
       being called from a different thread/process then strace will try
       to preserve the order of those events and mark the ongoing call
       as being unfinished.  When the call returns it will be marked as
       resumed.

           [pid 28772] select(4, [3], NULL, NULL, NULL <unfinished ...>
           [pid 28779] clock_gettime(CLOCK_REALTIME, {tv_sec=1130322148, tv_nsec=3977000}) = 0
           [pid 28772] <... select resumed> )      = 1 (in [3])

       Interruption of a (restartable) system call by a signal delivery
       is processed differently as kernel terminates the system call and
       also arranges its immediate reexecution after the signal handler
       completes.

           read(0, 0x7ffff72cf5cf, 1)              = ? ERESTARTSYS (To be restarted)
           --- SIGALRM {si_signo=SIGALRM, si_code=SI_KERNEL} ---
           rt_sigreturn({mask=[]})                 = 0
           read(0, "", 1)                          = 0

       Arguments are printed in symbolic form with passion.  This
       example shows the shell performing ">>xyzzy" output redirection:

           open("xyzzy", O_WRONLY|O_APPEND|O_CREAT, 0666) = 3

       Here, the second and the third argument of open(2) are decoded by
       breaking down the flag argument into its three bitwise-OR
       constituents and printing the mode value in octal by tradition.
       Where the traditional or native usage differs from ANSI or POSIX,
       the latter forms are preferred.  In some cases, strace output is
       proven to be more readable than the source.

       Structure pointers are dereferenced and the members are displayed
       as appropriate.  In most cases, arguments are formatted in the
       most C-like fashion possible.  For example, the essence of the
       command "ls -l /dev/null" is captured as:

           lstat("/dev/null", {st_mode=S_IFCHR|0666, st_rdev=makedev(0x1, 0x3), ...}) = 0

       Notice how the 'struct stat' argument is dereferenced and how
       each member is displayed symbolically.  In particular, observe
       how the st_mode member is carefully decoded into a bitwise-OR of
       symbolic and numeric values.  Also notice in this example that
       the first argument to lstat(2) is an input to the system call and
       the second argument is an output.  Since output arguments are not
       modified if the system call fails, arguments may not always be
       dereferenced.  For example, retrying the "ls -l" example with a
       non-existent file produces the following line:

           lstat("/foo/bar", 0xb004) = -1 ENOENT (No such file or directory)

       In this case the porch light is on but nobody is home.

       Syscalls unknown to strace are printed raw, with the unknown
       system call number printed in hexadecimal form and prefixed with
       "syscall_":

           syscall_0xbad(0x1, 0x2, 0x3, 0x4, 0x5, 0x6) = -1 ENOSYS (Function not implemented)

       Character pointers are dereferenced and printed as C strings.
       Non-printing characters in strings are normally represented by
       ordinary C escape codes.  Only the first strsize (32 by default)
       bytes of strings are printed; longer strings have an ellipsis
       appended following the closing quote.  Here is a line from "ls
       -l" where the getpwuid(3) library routine is reading the password
       file:

           read(3, "root::0:0:System Administrator:/"..., 1024) = 422

       While structures are annotated using curly braces, pointers to
       basic types and arrays are printed using square brackets with
       commas separating the elements.  Here is an example from the
       command id(1) on a system with supplementary group ids:

           getgroups(32, [100, 0]) = 2

       On the other hand, bit-sets are also shown using square brackets,
       but set elements are separated only by a space.  Here is the
       shell, preparing to execute an external command:

           sigprocmask(SIG_BLOCK, [CHLD TTOU], []) = 0

       Here, the second argument is a bit-set of two signals, SIGCHLD
       and SIGTTOU.  In some cases, the bit-set is so full that printing
       out the unset elements is more valuable.  In that case, the bit-
       set is prefixed by a tilde like this:

           sigprocmask(SIG_UNBLOCK, ~[], NULL) = 0

       Here, the second argument represents the full set of all signals.
OPTIONS
   General
       -e expr
              A qualifying expression which modifies which events to
              trace or how to trace them.  The format of the expression
              is:

                             [qualifier=][!]value[,value]...

              where qualifier is one of trace (or t), trace-fds (or
              trace-fd or fd or fds), abbrev (or a), verbose (or v), raw
              (or x), signal (or signals or s), read (or reads or r),
              write (or writes or w), fault, inject, status, quiet (or
              silent or silence or q), secontext, decode-fds (or
              decode-fd), decode-pids (or decode-pid), or kvm, and value
              is a qualifier-dependent symbol or number.  The default
              qualifier is trace.  Using an exclamation mark negates the
              set of values.  For example, -e open means literally
              -e trace=open which in turn means trace only the open
              system call.  By contrast, -e trace=!open means to trace
              every system call except open.  In addition, the special
              values all and none have the obvious meanings.

              Note that some shells use the exclamation point for
              history expansion even inside quoted arguments.  If so,
              you must escape the exclamation point with a backslash.

   Startup
       -E var=val
       --env=var=val
              Run command with var=val in its list of environment
              variables.

       -E var
       --env=var
              Remove var from the inherited list of environment
              variables before passing it on to the command.

       -p pid
       --attach=pid
              Attach to the process with the process ID pid and begin
              tracing.  The trace may be terminated at any time by a
              keyboard interrupt signal (CTRL-C).  strace will respond
              by detaching itself from the traced process(es) leaving it
              (them) to continue running.  Multiple -p options can be
              used to attach to many processes in addition to command
              (which is optional if at least one -p option is given).
              Multiple process IDs, separated by either comma (",''),
              space (" "), tab, or newline character, can be provided as
              an argument to a single -p option, so, for example, -p
              "$(pidof PROG)" and -p "$(pgrep PROG)" syntaxes are
              supported.

       -u username
       --user=username
              Run command with the user ID, group ID, and supplementary
              groups of username.  This option is only useful when
              running as root and enables the correct execution of
              setuid and/or setgid binaries.  Unless this option is used
              setuid and setgid programs are executed without effective
              privileges.
       -u UID:GID
       --user=UID:GID
              Alternative syntax where the program is started with
              exactly the given user and group IDs, and an empty list of
              supplementary groups.  In this case, user and group name
              lookups are not performed.

       --argv0=name
              Set argv[0] of the command being executed to name.  Useful
              for tracing multi-call executables which interpret
              argv[0], such as busybox or kmod.

   Tracing
       -b syscall
       --detach-on=syscall
              If specified syscall is reached, detach from traced
              process.  Currently, only execve(2) syscall is supported.
              This option is useful if you want to trace multi-threaded
              process and therefore require -f, but don't want to trace
              its (potentially very complex) children.

       -D
       --daemonize
       --daemonize=grandchild
              Run tracer process as a grandchild, not as the parent of
              the tracee.  This reduces the visible effect of strace by
              keeping the tracee a direct child of the calling process.

       -DD
       --daemonize=pgroup
       --daemonize=pgrp
              Run tracer process as tracee's grandchild in a separate
              process group.  In addition to reduction of the visible
              effect of strace, it also avoids killing of strace with
              kill(2) issued to the whole process group.

       -DDD
       --daemonize=session
              Run tracer process as tracee's grandchild in a separate
              session ("true daemonisation").  In addition to reduction
              of the visible effect of strace, it also avoids killing of
              strace upon session termination.

       -f
       --follow-forks
              Trace child processes as they are created by currently
              traced processes as a result of the fork(2), vfork(2) and
              clone(2) system calls.  Note that -p PID -f will attach
              all threads of process PID if it is multi-threaded, not
              only thread with thread_id = PID.

       --output-separately
              If the --output=filename option is in effect, each
              processes trace is written to filename.pid where pid is
              the numeric process id of each process.

       -ff
       --follow-forks --output-separately
              Combine the effects of --follow-forks and
              --output-separately options.  This is incompatible with
              -c, since no per-process counts are kept.

              One might want to consider using strace-log-merge(1) to
              obtain a combined strace log view.

       -I interruptible
       --interruptible=interruptible
              When strace can be interrupted by signals (such as
              pressing CTRL-C).

              1, anywhere
                     no signals are blocked;
              2, waiting
                     fatal signals are blocked while decoding syscall
                     (default);
              3, never
                     fatal signals are always blocked (default if -o
                     FILE PROG);
              4, never_tstp
                     fatal signals and SIGTSTP (CTRL-Z) are always
                     blocked (useful to make strace -o FILE PROG not
                     stop on CTRL-Z, default if -D).

       --syscall-limit=limit
              Detach all tracees when limit number of syscalls have been
              captured. Syscalls filtered out via --trace, --trace-path
              or --status options are not considered when keeping track
              of the number of syscalls that are captured.

       --kill-on-exit
              Apply PTRACE_O_EXITKILL ptrace option to all tracee
              processes (which sends a SIGKILL signal to the tracee if
              the tracer exits) and do not detach them on cleanup so
              they will not be left running after the tracer exit.
              --kill-on-exit is not compatible with -p/--attach options.

   Filtering
       -e trace=syscall_set
       -e t=syscall_set
       --trace=syscall_set
              Trace only the specified set of system calls.  syscall_set
              is defined as [!]value[,value], and value can be one of
              the following:

              syscall
                     Trace specific syscall, specified by its name (see
                     syscalls(2) for a reference, but also see NOTES).

              ?value Question mark before the syscall qualification
                     allows suppression of error in case no syscalls
                     matched the qualification provided.

              value@64
                     Limit the syscall specification described by value
                     to 64-bit personality.

              value@32
                     Limit the syscall specification described by value
                     to 32-bit personality.

              value@x32
                     Limit the syscall specification described by value
                     to x32 personality.

              all    Trace all system calls.

              /regex Trace only those system calls that match the regex.
                     You can use POSIX Extended Regular Expression
                     syntax (see regex(7)).

              %file
              file   Trace all system calls which take a file name as an
                     argument.  You can think of this as an abbreviation
                     for -e trace=open,stat,chmod,unlink,...  which is
                     useful to seeing what files the process is
                     referencing.  Furthermore, using the abbreviation
                     will ensure that you don't accidentally forget to
                     include a call like lstat(2) in the list.  Betchya
                     woulda forgot that one.  The syntax without a
                     preceding percent sign ("-e trace=file") is
                     deprecated.

              %process
              process
                     Trace system calls associated with process
                     lifecycle (creation, exec, termination).  The
                     syntax without a preceding percent sign ("-e
                     trace=process") is deprecated.

              %net
              %network
              network
                     Trace all the network related system calls.  The
                     syntax without a preceding percent sign ("-e
                     trace=network") is deprecated.

              %signal
              signal Trace all signal related system calls.  The syntax
                     without a preceding percent sign ("-e
                     trace=signal") is deprecated.

              %ipc
              ipc    Trace all IPC related system calls.  The syntax
                     without a preceding percent sign ("-e trace=ipc")
                     is deprecated.

              %desc
              desc   Trace all file descriptor related system calls.
                     The syntax without a preceding percent sign ("-e
                     trace=desc") is deprecated.

              %memory
              memory Trace all memory mapping related system calls.  The
                     syntax without a preceding percent sign ("-e
                     trace=memory") is deprecated.

              %creds Trace system calls that read or modify user and
                     group identifiers or capability sets.

              %stat  Trace stat syscall variants.

              %lstat Trace lstat syscall variants.

              %fstat Trace fstat, fstatat, and statx syscall variants.

              %%stat Trace syscalls used for requesting file status
                     (stat, lstat, fstat, fstatat, statx, and their
                     variants).

              %statfs
                     Trace statfs, statfs64, statvfs, osf_statfs, and
                     osf_statfs64 system calls.  The same effect can be
                     achieved with -e trace=/^(.*_)?statv?fs regular
                     expression.

              %fstatfs
                     Trace fstatfs, fstatfs64, fstatvfs, osf_fstatfs,
                     and osf_fstatfs64 system calls.  The same effect
                     can be achieved with -e trace=/fstatv?fs regular
                     expression.

              %%statfs
                     Trace syscalls related to file system statistics
                     (statfs-like, fstatfs-like, and ustat).  The same
                     effect can be achieved with
                     -e trace=/statv?fs|fsstat|ustat regular expression.

              %clock Trace system calls that read or modify system
                     clocks.

              %pure  Trace syscalls that always succeed and have no
                     arguments.  Currently, this list includes
                     arc_gettls(2), getdtablesize(2), getegid(2),
                     getegid32(2), geteuid(2), geteuid32(2), getgid(2),
                     getgid32(2), getpagesize(2), getpgrp(2), getpid(2),
                     getppid(2), get_thread_area(2) (on architectures
                     other than x86), gettid(2), get_tls(2), getuid(2),
                     getuid32(2), getxgid(2), getxpid(2), getxuid(2),
                     kern_features(2), and metag_get_tls(2) syscalls.

              The -c option is useful for determining which system calls
              might be useful to trace.  For example,
              trace=open,close,read,write means to only trace those four
              system calls.  Be careful when making inferences about the
              user/kernel boundary if only a subset of system calls are
              being monitored.  The default is trace=all.

       -e trace-fd=set
       -e trace-fds=set
       -e fd=set
       -e fds=set
       --trace-fds=set
              Trace only the syscalls that operate on the specified
              subset of (non-negative) file descriptors.  Note that
              usage of this option also filters out all the syscalls
              that do not operate on file descriptors at all.  Applies
              in (inclusive) disjunction with the --trace-path option.

       -e signal=set
       -e signals=set
       -e s=set
       --signal=set
              Trace only the specified subset of signals.  The default
              is signal=all.  For example, signal=!SIGIO (or signal=!io)
              causes SIGIO signals not to be traced.

       -e status=set
       --status=set
              Print only system calls with the specified return status.
              The default is status=all.  When using the status
              qualifier, because strace waits for system calls to return
              before deciding whether they should be printed or not, the
              traditional order of events may not be preserved anymore.
              If two system calls are executed by concurrent threads,
              strace will first print both the entry and exit of the
              first system call to exit, regardless of their respective
              entry time.  The entry and exit of the second system call
              to exit will be printed afterwards.  Here is an example
              when select(2) is called, but a different thread calls
              clock_gettime(2) before select(2) finishes:

                  [pid 28779] 1130322148.939977 clock_gettime(CLOCK_REALTIME, {1130322148, 939977000}) = 0
                  [pid 28772] 1130322148.438139 select(4, [3], NULL, NULL, NULL) = 1 (in [3])

              set can include the following elements:

              successful
                     Trace system calls that returned without an error
                     code.  The -z option has the effect of
                     status=successful.
              failed Trace system calls that returned with an error
                     code.  The -Z option has the effect of
                     status=failed.
              unfinished
                     Trace system calls that did not return.  This might
                     happen, for example, due to an execve call in a
                     neighbour thread.
              unavailable
                     Trace system calls that returned but strace failed
                     to fetch the error status.
              detached
                     Trace system calls for which strace detached before
                     the return.

       -P path
       --trace-path=path
              Trace only system calls accessing path.  Multiple -P
              options can be used to specify several paths.  Applies in
              (inclusive) disjunction with the --trace-fds option.

       -z
       --successful-only
              Print only syscalls that returned without an error code.

       -Z
       --failed-only
              Print only syscalls that returned with an error code.

   Output format
       -a column
       --columns=column
              Align return values in a specific column (default column
              40).

       -e abbrev=syscall_set
       -e a=syscall_set
       --abbrev=syscall_set
              Abbreviate the output from printing each member of large
              structures.  The syntax of the syscall_set specification
              is the same as in the -e trace option.  The default is
              abbrev=all.  The -v option has the effect of abbrev=none.

       -e verbose=syscall_set
       -e v=syscall_set
       --verbose=syscall_set
              Dereference structures for the specified set of system
              calls.  The syntax of the syscall_set specification is the
              same as in the -e trace option.  The default is
              verbose=all.

       -e raw=syscall_set
       -e x=syscall_set
       --raw=syscall_set
              Print raw, undecoded arguments for the specified set of
              system calls.  The syntax of the syscall_set specification
              is the same as in the -e trace option.  This option has
              the effect of causing all arguments to be printed in
              hexadecimal.  This is mostly useful if you don't trust the
              decoding or you need to know the actual numeric value of
              an argument.  See also -X raw option.

       -e read=set
       -e reads=set
       -e r=set
       --read=set
              Perform a full hexadecimal and ASCII dump of all the data
              read from file descriptors listed in the specified set.
              For example, to see all input activity on file descriptors
              3 and 5 use -e read=3,5.  Note that this is independent
              from the normal tracing of the read(2) system call which
              is controlled by the option -e trace=read.

       -e write=set
       -e writes=set
       -e w=set
       --write=set
              Perform a full hexadecimal and ASCII dump of all the data
              written to file descriptors listed in the specified set.
              For example, to see all output activity on file
              descriptors 3 and 5 use -e write=3,5.  Note that this is
              independent from the normal tracing of the write(2) system
              call which is controlled by the option -e trace=write.

       -e quiet=set
       -e silent=set
       -e silence=set
       -e q=set
       --quiet=set
       --silent=set
       --silence=set
              Suppress various information messages.  The default is
              quiet=none.  set can include the following elements:

              attach Suppress messages about attaching and detaching ("[
                     Process NNNN attached ]", "[ Process NNNN detached
                     ]").
              exit   Suppress messages about process exits ("+++ exited
                     with SSS +++").
              path-resolution
                     Suppress messages about resolution of paths
                     provided via the -P option ("Requested path "..."
                     resolved into "..."").
              personality
                     Suppress messages about process personality changes
                     ("[ Process PID=NNNN runs in PPP mode. ]").
              thread-execve
              superseded
                     Suppress messages about process being superseded by
                     execve(2) in another thread ("+++ superseded by
                     execve in pid NNNN +++").

       -e decode-fds=set
       --decode-fds=set
              Decode various information associated with file
              descriptors.  The default is decode-fds=none.  set can
              include the following elements:

              path     Print file paths.  Also enables printing of
                       tracee's current working directory when AT_FDCWD
                       constant is used.
              socket   Print socket protocol-specific information,
              dev      Print character/block device numbers.
              pidfd    Print PIDs associated with pidfd file
                       descriptors.
              signalfd Print signal masks associated with signalfd file
                       descriptors.

       -e decode-pids=set
       --decode-pids=set
              Decode various information associated with process IDs
              (and also thread IDs, process group IDs, and session IDs).
              The default is decode-pids=none.  set can include the
              following elements:

              comm    Print command names associated with thread or
                      process IDs.
              pidns   Print thread, process, process group, and session
                      IDs in strace's PID namespace if the tracee is in
                      a different PID namespace.

       -e kvm=vcpu
       --kvm=vcpu
              Print the exit reason of kvm vcpu.  Requires Linux kernel
              version 4.16.0 or higher.

       -i
       --instruction-pointer
              Print the instruction pointer at the time of the system
              call.

       -n
       --syscall-number
              Print the syscall number.

       -k
       --stack-trace[=symbol]
              Print the execution stack trace of the traced processes
              after each system call.

       -kk
       --stack-trace=source
              Print the execution stack trace and source code
              information of the traced processes after each system
              call. This option expects the target program is compiled
              with appropriate debug options: "-g" (gcc), or "-g
              -gdwarf-aranges" (clang).

       --stack-trace-frame-limit=limit
              Print no more than this amount of stack trace frames when
              backtracing a system call (the default is 256).  Use this
              option with the --stack-trace (or -k) option.

       -o filename
       --output=filename
              Write the trace output to the file filename rather than to
              stderr.  filename.pid form is used if -ff option is
              supplied.  If the argument begins with '|' or '!', the
              rest of the argument is treated as a command and all
              output is piped to it.  This is convenient for piping the
              debugging output to a program without affecting the
              redirections of executed programs.  The latter is not
              compatible with -ff option currently.

       -A
       --output-append-mode
              Open the file provided in the -o option in append mode.

       -q
       --quiet
       --quiet=attach,personality
              Suppress messages about attaching, detaching, and
              personality changes.  This happens automatically when
              output is redirected to a file and the command is run
              directly instead of attaching.

       -qq
       --quiet=attach,personality,exit
              Suppress messages attaching, detaching, personality
              changes, and about process exit status.

       -qqq
       --quiet=all
              Suppress all suppressible messages (please refer to the -e
              quiet option description for the full list of suppressible
              messages).

       -r
       --relative-timestamps[=precision]
              Print a relative timestamp upon entry to each system call.
              This records the time difference between the beginning of
              successive system calls.  precision can be one of s (for
              seconds), ms (milliseconds), us (microseconds), or ns
              (nanoseconds), and allows setting the precision of time
              value being printed.  Default is us (microseconds).  Note
              that since -r option uses the monotonic clock time for
              measuring time difference and not the wall clock time, its
              measurements can differ from the difference in time
              reported by the -t option.

       -s strsize
       --string-limit=strsize
              Specify the maximum string size to print (the default is
              32).  Note that filenames are not considered strings and
              are always printed in full.

       --absolute-timestamps[=[[format:]format],[[precision:]precision]]
       --timestamps[=[[format:]format],[[precision:]precision]]
              Prefix each line of the trace with the wall clock time in
              the specified format with the specified precision.  format
              can be one of the following:

              none   No time stamp is printed.  Can be used to override
                     the previous setting.
              time   Wall clock time (strftime(3) format string is %T).
              unix   Number of seconds since the epoch (strftime(3)
                     format string is %s).

              precision can be one of s (for seconds), ms
              (milliseconds), us (microseconds), or ns (nanoseconds).
              Default arguments for the option are
              format:time,precision:s.

       -t
       --absolute-timestamps
              Prefix each line of the trace with the wall clock time.

       -tt
       --absolute-timestamps=precision:us
              If given twice, the time printed will include the
              microseconds.

       -ttt
       --absolute-timestamps=format:unix,precision:us
              If given thrice, the time printed will include the
              microseconds and the leading portion will be printed as
              the number of seconds since the epoch.

       -T
       --syscall-times[=precision]
              Show the time spent in system calls.  This records the
              time difference between the beginning and the end of each
              system call.  precision can be one of s (for seconds), ms
              (milliseconds), us (microseconds), or ns (nanoseconds),
              and allows setting the precision of time value being
              printed.  Default is us (microseconds).

       -v
       --no-abbrev
              Print unabbreviated versions of environment, stat,
              termios, etc.  calls.  These structures are very common in
              calls and so the default behavior displays a reasonable
              subset of structure members.  Use this option to get all
              of the gory details.

       --strings-in-hex[=option]
              Control usage of escape sequences with hexadecimal numbers
              in the printed strings.  Normally (when no
              --strings-in-hex or -x option is supplied), escape
              sequences are used to print non-printable and non-ASCII
              characters (that is, characters with a character code less
              than 32 or greater than 127), or to disambiguate the
              output (so, for quotes and other characters that encase
              the printed string, for example, angle brackets, in case
              of file descriptor path output); for the former use case,
              unless it is a white space character that has a symbolic
              escape sequence defined in the C standard (that is, "\textbackslash t"
              for a horizontal tab, "\textbackslash n" for a newline, "\textbackslash v" for a
              vertical tab, "\textbackslash f" for a form feed page break, and "\textbackslash r"
              for a carriage return) are printed using escape sequences
              with numbers that correspond to their byte values, with
              octal number format being the default.  option can be one
              of the following:

              none   Hexadecimal numbers are not used in the output at
                     all.  When there is a need to emit an escape
                     sequence, octal numbers are used.
              non-ascii-chars
                     Hexadecimal numbers are used instead of octal in
                     the escape sequences.
              non-ascii
                     Strings that contain non-ASCII characters are
                     printed using escape sequences with hexadecimal
                     numbers.
              all    All strings are printed using escape sequences with
                     hexadecimal numbers.

              When the option is supplied without an argument, all is
              assumed.

       -x
       --strings-in-hex=non-ascii
              Print all non-ASCII strings in hexadecimal string format.

       -xx
       --strings-in-hex[=all]
              Print all strings in hexadecimal string format.

       -X format
       --const-print-style=format
              Set the format for printing of named constants and flags.
              Supported format values are:

              raw    Raw number output, without decoding.
              abbrev Output a named constant or a set of flags instead
                     of the raw number if they are found.  This is the
                     default strace behaviour.
              verbose
                     Output both the raw value and the decoded string
                     (as a comment).

       -y
       --decode-fds
       --decode-fds=path
              Print paths associated with file descriptor arguments and
              with the AT_FDCWD constant.

       -yy
       --decode-fds=all
              Print all available information associated with file
              descriptors: protocol-specific information associated with
              socket file descriptors, block/character device number
              associated with device file descriptors, and PIDs
              associated with pidfd file descriptors.

       --pidns-translation
       --decode-pids=pidns
              If strace and tracee are in different PID namespaces,
              print PIDs in strace's namespace, too.

       -Y
       --decode-pids=comm
              Print command names for PIDs.

       --secontext[=format]
       -e secontext=format
              When SELinux is available and is not disabled, print in
              square brackets SELinux contexts of processes, files, and
              descriptors.  The format argument is a comma-separated
              list of items being one of the following:

              full              Print the full context (user, role, type
                                level and category).
              mismatch          Also print the context recorded by the
                                SELinux database in case the current
                                context differs.  The latter is printed
                                after two exclamation marks (!!).

              The default value for --secontext is !full,mismatch which
              prints only the type instead of full context and doesn't
              check for context mismatches.

       --always-show-pid
              Show PID prefix also for the process started by strace.
              Implied when -f and -o are both specified.

   Statistics
       -c
       --summary-only
              Count time, calls, and errors for each system call and
              report a summary on program exit, suppressing the regular
              output.  This attempts to show system time (CPU time spent
              running in the kernel) independent of wall clock time.  If
              -c is used with -f, only aggregate totals for all traced
              processes are kept.

       -C
       --summary
              Like -c but also print regular output while processes are
              running.

       -O overhead
       --summary-syscall-overhead=overhead
              Set the overhead for tracing system calls to overhead.
              This is useful for overriding the default heuristic for
              guessing how much time is spent in mere measuring when
              timing system calls using the -c option.  The accuracy of
              the heuristic can be gauged by timing a given program run
              without tracing (using time(1)) and comparing the
              accumulated system call time to the total produced using
              -c.

              The format of overhead specification is described in
              section Time specification format description.

       -S sortby
       --summary-sort-by=sortby
              Sort the output of the histogram printed by the -c option
              by the specified criterion.  Legal values are time (or
              time-percent or time-total or total-time), min-time (or
              shortest or time-min), max-time (or longest or time-max),
              avg-time (or time-avg), calls (or count), errors (or
              error), name (or syscall or syscall-name), and nothing (or
              none); default is time.

       -U columns
       --summary-columns=columns
              Configure a set (and order) of columns being shown in the
              call summary.  The columns argument is a comma-separated
              list with items being one of the following:

              time-percent (or time)
                     Percentage of cumulative time consumed by a
                     specific system call.
              total-time (or time-total)
                     Total system (or wall clock, if -w option is
                     provided) time consumed by a specific system call.
              min-time (or shortest or time-min)
                     Minimum observed call duration.
              max-time (or longest or time-max)
                     Maximum observed call duration.
              avg-time (or time-avg)
                     Average call duration.
              calls (or count)
                     Call count.
              errors (or error)
                     Error count.
              name (or syscall or syscall-name)
                     Syscall name.

              The default value is
              time-percent,total-time,avg-time,calls,errors,name.  If
              the name field is not supplied explicitly, it is added as
              the last column.

       -w
       --summary-wall-clock
              Summarise the time difference between the beginning and
              end of each system call.  The default is to summarise the
              system time.

   Tampering
       -e inject=syscall_set[:error=errno|:retval=value][:signal=sig]
       [:syscall=syscall][:delay_enter=delay][:delay_exit=delay]
       [:poke_enter=@argN=DATAN,@argM=DATAM...]
       [:poke_exit=@argN=DATAN,@argM=DATAM...][:when=expr]
       --inject=syscall_set[:error=errno|:retval=value][:signal=sig]
       [:syscall=syscall][:delay_enter=delay][:delay_exit=delay]
       [:poke_enter=@argN=DATAN,@argM=DATAM...]
       [:poke_exit=@argN=DATAN,@argM=DATAM...][:when=expr]
              Perform   syscall  tampering  for  the  specified  set  of
              syscalls.  The syntax of the syscall_set specification  is
              the same as in the -e trace option.

              At  least  one  of  error,  retval,  signal,  delay_enter,
              delay_exit, poke_enter, or poke_exit  options  has  to  be
              specified.  error and retval are mutually exclusive.

              If  :error=errno  option is specified, a fault is injected
              into a syscall invocation: the syscall number is  replaced
              by  -1  which  corresponds to an invalid syscall (unless a
              syscall is specified with :syscall= option), and the error
              code is specified using a symbolic errno value like ENOSYS
              or a numeric value within 1..4095 range.

              If :retval=value option is specified, success injection is
              performed: the syscall number is replaced  by  -1,  but  a
              bogus success value is returned to the callee.

              If  :signal=sig option is specified with either a symbolic
              value like SIGSEGV or a numeric value  within  1..SIGRTMAX
              range,  that signal is delivered on entering every syscall
              specified by the set.

              If :delay_enter=delay  or  :delay_exit=delay  options  are
              specified,  delay  injection  is  performed: the tracee is
              delayed by time period specified by delay on  entering  or
              exiting  the  syscall,  respectively.  The format of delay
              specification is described in section  Time  specification
              format description.

              If        :poke_enter=@argN=DATAN,@argM=DATAM...        or
              :poke_exit=@argN=DATAN,@argM=DATAM...     options      are
              specified,  tracee's  memory  at  locations, pointed to by
              system call arguments argN and argM (going  from  arg1  to
              arg7) is overwritten by data DATAN and DATAM (specified in
              hexadecimal          format;          for          example
              :poke_enter=@arg1=0000DEAD0000BEEF).  :poke_enter modifies
              memory on syscall enter, and :poke_exit - on exit.

              If :signal=sig option is specified  without  :error=errno,
              :retval=value  or  :delay_{enter,exit}=usecs options, then
              only a signal sig is delivered without a syscall fault  or
              delay     injection.     Conversely,    :error=errno    or
              :retval=value    option    without     :delay_enter=delay,
              :delay_exit=delay  or  :signal=sig options injects a fault
              without delivering a signal or injecting a delay, etc.

              If  :signal=sig  option   is   specified   together   with
              :error=errno  or  :retval=value,  then both injection of a
              fault or success and signal delivery are performed.

              if :syscall=syscall option is specified, the corresponding
              syscall with no side effects is injected  instead  of  -1.
              Currently,  only  "pure"  (see -e trace=%pure description)
              syscalls can be specified there.

              Unless  a  :when=expr  subexpression  is   specified,   an
              injection  is  being  made  into  every invocation of each
              syscall from the set.

              The format of the subexpression is:

                             first[..last][+[step]]

              Number first stands for the first invocation number in the
              range, number last stands for the last  invocation  number
              in  the  range,  and  step stands for the step between two
              consecutive invocations.  The following  combinations  are
              useful:

              first  For every syscall from the set, perform an
                     injection for the syscall invocation number first
                     only.
              first..last
                     For every syscall from the set, perform an
                     injection for the syscall invocation number first
                     and all subsequent invocations until the invocation
                     number last (inclusive).
              first+ For every syscall from the set, perform injections
                     for the syscall invocation number first and all
                     subsequent invocations.
              first..last+
                     For every syscall from the set, perform injections
                     for the syscall invocation number first and all
                     subsequent invocations until the invocation number
                     last (inclusive).
              first+step
                     For every syscall from the set, perform injections
                     for syscall invocations number first, first+step,
                     first+step+step, and so on.
              first..last+step
                     Same as the previous, but consider only syscall
                     invocations with numbers up to last (inclusive).

              For example, to fail each third and subsequent chdir
              syscalls with ENOENT, use
              -e inject=chdir:error=ENOENT:when=3+.

              The valid range for numbers first and step is 1..65535,
              and for number last is 1..65534.

              An injection expression can contain only one error= or
              retval= specification, and only one signal= specification.
              If an injection expression contains multiple when=
              specifications, the last one takes precedence.

              Accounting of syscalls that are subject to injection is
              done per syscall and per tracee.

              Specification of syscall injection can be combined with
              other syscall filtering options, for example, -P
              /dev/urandom -e inject=file:error=ENOENT.

       -e fault=syscall_set[:error=errno][:when=expr]
       --fault=syscall_set[:error=errno][:when=expr]
              Perform syscall fault injection for the specified set of
              syscalls.

              This is equivalent to more generic -e inject= expression
              with default value of errno option set to ENOSYS.

   Miscellaneous
       -d
       --debug
              Show some debugging output of strace itself on the
              standard error.

       -F     This option is deprecated.  It is retained for backward
              compatibility only and may be removed in future releases.
              Usage of multiple instances of -F option is still
              equivalent to a single -f, and it is ignored at all if
              used along with one or more instances of -f option.

       -h
       --help Print the help summary.

       --seccomp-bpf
              Try to enable use of seccomp-bpf (see seccomp(2)) to have
              ptrace(2)-stops only when system calls that are being
              traced occur in the traced processes.

              This option has no effect unless -f/--follow-forks is also
              specified.  --seccomp-bpf is not compatible with
              --syscall-limit and -b/--detach-on options.  It is also
              not applicable to processes attached using -p/--attach
              option.

              An attempt to enable system calls filtering using seccomp-
              bpf may fail for various reasons, e.g. there are too many
              system calls to filter, the seccomp API is not available,
              or strace itself is being traced.  In cases when seccomp-
              bpf filter setup failed, strace proceeds as usual and
              stops traced processes on every system call.

              When --seccomp-bpf is activated and -p/--attach option is
              not used, --kill-on-exit option is activated as well.

              Note that in cases when the tracee has another seccomp
              filter that returns an action value with a precedence
              greater than SECCOMP_RET_TRACE, strace --seccomp-bpf will
              not be notified.  That is, if another seccomp filter, for
              example, disables the syscall or kills the tracee, then
              strace --seccomp-bpf will not be aware of that syscall
              invocation at all.

       --tips[=[[id:]id],[[format:]format]]
              Show strace tips, tricks, and tweaks before exit.  id can
              be a non-negative integer number, which enables printing
              of specific tip, trick, or tweak (these ID are not
              guaranteed to be stable), or random (the default), in
              which case a random tip is printed.  format can be one of
              the following:

              none     No tip is printed.  Can be used to override the
                       previous setting.
              compact  Print the tip just big enough to contain all the
                       text.
              full     Print the tip in its full glory.

              Default is id:random,format:compact.

       -V
       --version
              Print the version number of strace.  Multiple instances of
              the option beyond specific threshold tend to increase
              Strauss awareness.

   Time specification format description
       Time values can be specified as a decimal floating point number
       (in a format accepted by strtod(3)), optionally followed by one
       of the following suffices that specify the unit of time: s
       (seconds), ms (milliseconds), us (microseconds), or ns
       (nanoseconds).  If no suffix is specified, the value is
       interpreted as microseconds.

       The described format is used for -O, -e inject=delay_enter, and
       -e inject=delay_exit options.
DIAGNOSTICS
       When command exits, strace exits with the same exit status.  If
       command is terminated by a signal, strace terminates itself with
       the same signal, so that strace can be used as a wrapper process
       transparent to the invoking parent process.  Note that parent-
       child relationship (signal stop notifications, getppid(2) value,
       etc) between traced process and its parent are not preserved
       unless -D is used.

       When using -p without a command, the exit status of strace is
       zero unless no processes has been attached or there was an
       unexpected error in doing the tracing.
SETUID INSTALLATION
       If strace is installed setuid to root then the invoking user will
       be able to attach to and trace processes owned by any user.  In
       addition setuid and setgid programs will be executed and traced
       with the correct effective privileges.  Since only users trusted
       with full root privileges should be allowed to do these things,
       it only makes sense to install strace as setuid to root when the
       users who can execute it are restricted to those users who have
       this trust.  For example, it makes sense to install a special
       version of strace with mode 'rwsr-xr--', user root and group
       trace, where members of the trace group are trusted users.  If
       you do use this feature, please remember to install a regular
       non-setuid version of strace for ordinary users to use.
MULTIPLE PERSONALITIES SUPPORT
       On some architectures, strace supports decoding of syscalls for
       processes that use different ABI rather than the one strace uses.
       Specifically, in addition to decoding native ABI, strace can
       decode the following ABIs on the following architectures:

       [1]  When strace is built as an x86_64 application
       [2]  When strace is built as an x32 application
       [3]  Big endian only

       This support is optional and relies on ability to generate and
       parse structure definitions during the build time.  Please refer
       to the output of the strace -V command in order to figure out
       what support is available in your strace build ("non-native"
       refers to an ABI that differs from the ABI strace has):

       m32-mpers
              strace can trace and properly decode non-native 32-bit
              binaries.
       no-m32-mpers
              strace can trace, but cannot properly decode non-native
              32-bit binaries.
       mx32-mpers
              strace can trace and properly decode non-native
              32-on-64-bit binaries.
       no-mx32-mpers
              strace can trace, but cannot properly decode non-native
              32-on-64-bit binaries.

       If the output contains neither m32-mpers nor no-m32-mpers, then
       decoding of non-native 32-bit binaries is not implemented at all
       or not applicable.

       Likewise, if the output contains neither mx32-mpers nor no-
       mx32-mpers, then decoding of non-native 32-on-64-bit binaries is
       not implemented at all or not applicable.
NOTES
       It is a pity that so much tracing clutter is produced by systems
       employing shared libraries.

       It is instructive to think about system call inputs and outputs
       as data-flow across the user/kernel boundary.  Because user-space
       and kernel-space are separate and address-protected, it is
       sometimes possible to make deductive inferences about process
       behavior using inputs and outputs as propositions.

       In some cases, a system call will differ from the documented
       behavior or have a different name.  For example, the faccessat(2)
       system call does not have flags argument, and the setrlimit(2)
       library function uses prlimit64(2) system call on modern
       (2.6.38+) kernels.  These discrepancies are normal but
       idiosyncratic characteristics of the system call interface and
       are accounted for by C library wrapper functions.

       Some system calls have different names in different architectures
       and personalities.  In these cases, system call filtering and
       printing uses the names that match corresponding __NR_* kernel
       macros of the tracee's architecture and personality.  There are
       two exceptions from this general rule: arm_fadvise64_64(2) ARM
       syscall and xtensa_fadvise64_64(2) Xtensa syscall are filtered
       and printed as fadvise64_64(2).

       On x32, syscalls that are intended to be used by 64-bit processes
       and not x32 ones (for example, readv(2), that has syscall number
       19 on x86_64, with its x32 counterpart has syscall number 515),
       but called with __X32_SYSCALL_BIT flag being set, are designated
       with #64 suffix.

       On some platforms a process that is attached to with the -p
       option may observe a spurious EINTR return from the current
       system call that is not restartable.  (Ideally, all system calls
       should be restarted on strace attach, making the attach invisible
       to the traced process, but a few system calls aren't.  Arguably,
       every instance of such behavior is a kernel bug.)  This may have
       an unpredictable effect on the process if the process takes no
       action to restart the system call.

       As strace executes the specified command directly and does not
       employ a shell for that, scripts without shebang that usually run
       just fine when invoked by shell fail to execute with ENOEXEC
       error.  It is advisable to manually supply a shell as a command
       with the script as its argument.
BUGS
       Programs that use the setuid bit do not have effective user ID
       privileges while being traced.

       A traced process runs slowly (but check out the --seccomp-bpf
       option).

       Unless --kill-on-exit option is used (or --seccomp-bpf option is
       used in a way that implies --kill-on-exit), traced processes
       which are descended from command may be left running after an
       interrupt signal (CTRL-C).

       By using CLONE_UNTRACED flag of clone system call a tracee can
       break the guarantee that --seccomp-bpf will not leave any
       processes with a seccomp program installed for syscall filtering
       purposes.
HISTORY
       The original strace was written by Paul Kranenburg for SunOS and
       was inspired by its trace utility.  The SunOS version of strace
       was ported to Linux and enhanced by Branko Lankester, who also
       wrote the Linux kernel support.  Even though Paul released strace
       2.5 in 1992, Branko's work was based on Paul's strace 1.5 release
       from 1991.  In 1993, Rick Sladkey merged strace 2.5 for SunOS and
       the second release of strace for Linux, added many of the
       features of truss(1) from SVR4, and produced an strace that
       worked on both platforms.  In 1994 Rick ported strace to SVR4 and
       Solaris and wrote the automatic configuration support.  In 1995
       he ported strace to Irix and became tired of writing about
       himself in the third person.

       Beginning with 1996, strace was maintained by Wichert Akkerman.
       During his tenure, strace development migrated to CVS; ports to
       FreeBSD and many architectures on Linux (including ARM, IA-64,
       MIPS, PA-RISC, PowerPC, s390, SPARC) were introduced.  In 2002,
       the burden of strace maintainership was transferred to Roland
       McGrath.  Since then, strace gained support for several new Linux
       architectures (AMD64, s390x, SuperH), bi-architecture support for
       some of them, and received numerous additions and improvements in
       syscalls decoders on Linux; strace development migrated to Git
       during that period.  Since 2009, strace is actively maintained by
       Dmitry Levin.  strace gained support for AArch64, ARC, AVR32,
       Blackfin, Meta, Nios II, OpenRISC 1000, RISC-V, Tile/TileGx,
       Xtensa architectures since that time.  In 2012, unmaintained and
       apparently broken support for non-Linux operating systems was
       removed.  Also, in 2012 strace gained support for path tracing
       and file descriptor path decoding.  In 2014, support for stack
       trace printing was added.  In 2016, syscall fault injection was
       implemented.

       For the additional information, please refer to the NEWS file and
       strace repository commit log.
REPORTING BUGS
       Problems with strace should be reported to the strace mailing
       list mailto:strace-devel@lists.strace.io.
SEE ALSO
       strace-log-merge(1), ltrace(1), perf-trace(1), trace-cmd(1),
       time(1), ptrace(2), seccomp(2), syscall(2), proc(5), signal(7)

       strace Home Page https://strace.io/
AUTHORS
       The complete list of strace contributors can be found in the
       CREDITS file.
COLOPHON
       This page is part of the strace (system call tracer) project.
       Information about the project can be found at 
       http://strace.io/.  If you have a bug report for this manual
       page, send it to strace-devel@lists.sourceforge.net.  This page
       was obtained from the project's upstream Git repository
       https://github.com/strace/strace.git on 2024-06-14.  (At that
       time, the date of the most recent commit that was found in the
       repository was 2024-06-04.)  If you discover any rendering
       problems in this HTML version of the page, or you believe there
       is a better or more up-to-date source for the page, or you have
       corrections or improvements to the information in this COLOPHON
       (which is not part of the original manual page), send a mail to
       man-pages@man7.org

strace 6.9.0.16.2a4c4          2024-06-04                      STRACE(1)
\end{lstlisting}
}}

\endinput  %  ==  ==  ==  ==  ==  ==  ==  ==  ==
\subsection{\refStrace: Trace System Calls and Signals}

{\tiny{
\begin{lstlisting}[language=bash]
NAME
       strace - trace system calls and signals
SYNOPSIS
       strace [-ACdffhikkqqrtttTvVwxxyyYzZ] [-a column] [-b execve]
              [-e expr]... [-I n] [-o file] [-O overhead] [-p pid]...
              [-P path]... [-s strsize] [-S sortby] [-U columns]
              [-X format] [--seccomp-bpf]
              [--stack-trace-frame-limit=limit] [--syscall-limit=limit]
              [--secontext[=format]] [--tips[=format]] { -p pid | [-DDD]
              [-E var[=val]]... [-u username] command [args] }

       strace -c [-dfwzZ] [-b execve] [-e expr]... [-I n] [-O overhead]
              [-p pid]... [-P path]... [-S sortby] [-U columns]
              [--seccomp-bpf] [--syscall-limit=limit] [--tips[=format]]
              { -p pid | [-DDD] [-E var[=val]]... [-u username] command
              [args] }

       strace --tips[=format]
DESCRIPTION
       In the simplest case strace runs the specified command until it
       exits.  It intercepts and records the system calls which are
       called by a process and the signals which are received by a
       process.  The name of each system call, its arguments and its
       return value are printed on standard error or to the file
       specified with the -o option.

       strace is a useful diagnostic, instructional, and debugging tool.
       System administrators, diagnosticians and trouble-shooters will
       find it invaluable for solving problems with programs for which
       the source is not readily available since they do not need to be
       recompiled in order to trace them.  Students, hackers and the
       overly-curious will find that a great deal can be learned about a
       system and its system calls by tracing even ordinary programs.
       And programmers will find that since system calls and signals are
       events that happen at the user/kernel interface, a close
       examination of this boundary is very useful for bug isolation,
       sanity checking and attempting to capture race conditions.

       Each line in the trace contains the system call name, followed by
       its arguments in parentheses and its return value.  An example
       from stracing the command "cat /dev/null" is:

           open("/dev/null", O_RDONLY) = 3

       Errors (typically a return value of -1) have the errno symbol and
       error string appended.

           open("/foo/bar", O_RDONLY) = -1 ENOENT (No such file or directory)

       Signals are printed as signal symbol and decoded siginfo
       structure.  An excerpt from stracing and interrupting the command
       "sleep 666" is:

           sigsuspend([] <unfinished ...>
           --- SIGINT {si_signo=SIGINT, si_code=SI_USER, si_pid=...} ---
           +++ killed by SIGINT +++

       If a system call is being executed and meanwhile another one is
       being called from a different thread/process then strace will try
       to preserve the order of those events and mark the ongoing call
       as being unfinished.  When the call returns it will be marked as
       resumed.

           [pid 28772] select(4, [3], NULL, NULL, NULL <unfinished ...>
           [pid 28779] clock_gettime(CLOCK_REALTIME, {tv_sec=1130322148, tv_nsec=3977000}) = 0
           [pid 28772] <... select resumed> )      = 1 (in [3])

       Interruption of a (restartable) system call by a signal delivery
       is processed differently as kernel terminates the system call and
       also arranges its immediate reexecution after the signal handler
       completes.

           read(0, 0x7ffff72cf5cf, 1)              = ? ERESTARTSYS (To be restarted)
           --- SIGALRM {si_signo=SIGALRM, si_code=SI_KERNEL} ---
           rt_sigreturn({mask=[]})                 = 0
           read(0, "", 1)                          = 0

       Arguments are printed in symbolic form with passion.  This
       example shows the shell performing ">>xyzzy" output redirection:

           open("xyzzy", O_WRONLY|O_APPEND|O_CREAT, 0666) = 3

       Here, the second and the third argument of open(2) are decoded by
       breaking down the flag argument into its three bitwise-OR
       constituents and printing the mode value in octal by tradition.
       Where the traditional or native usage differs from ANSI or POSIX,
       the latter forms are preferred.  In some cases, strace output is
       proven to be more readable than the source.

       Structure pointers are dereferenced and the members are displayed
       as appropriate.  In most cases, arguments are formatted in the
       most C-like fashion possible.  For example, the essence of the
       command "ls -l /dev/null" is captured as:

           lstat("/dev/null", {st_mode=S_IFCHR|0666, st_rdev=makedev(0x1, 0x3), ...}) = 0

       Notice how the 'struct stat' argument is dereferenced and how
       each member is displayed symbolically.  In particular, observe
       how the st_mode member is carefully decoded into a bitwise-OR of
       symbolic and numeric values.  Also notice in this example that
       the first argument to lstat(2) is an input to the system call and
       the second argument is an output.  Since output arguments are not
       modified if the system call fails, arguments may not always be
       dereferenced.  For example, retrying the "ls -l" example with a
       non-existent file produces the following line:

           lstat("/foo/bar", 0xb004) = -1 ENOENT (No such file or directory)

       In this case the porch light is on but nobody is home.

       Syscalls unknown to strace are printed raw, with the unknown
       system call number printed in hexadecimal form and prefixed with
       "syscall_":

           syscall_0xbad(0x1, 0x2, 0x3, 0x4, 0x5, 0x6) = -1 ENOSYS (Function not implemented)

       Character pointers are dereferenced and printed as C strings.
       Non-printing characters in strings are normally represented by
       ordinary C escape codes.  Only the first strsize (32 by default)
       bytes of strings are printed; longer strings have an ellipsis
       appended following the closing quote.  Here is a line from "ls
       -l" where the getpwuid(3) library routine is reading the password
       file:

           read(3, "root::0:0:System Administrator:/"..., 1024) = 422

       While structures are annotated using curly braces, pointers to
       basic types and arrays are printed using square brackets with
       commas separating the elements.  Here is an example from the
       command id(1) on a system with supplementary group ids:

           getgroups(32, [100, 0]) = 2

       On the other hand, bit-sets are also shown using square brackets,
       but set elements are separated only by a space.  Here is the
       shell, preparing to execute an external command:

           sigprocmask(SIG_BLOCK, [CHLD TTOU], []) = 0

       Here, the second argument is a bit-set of two signals, SIGCHLD
       and SIGTTOU.  In some cases, the bit-set is so full that printing
       out the unset elements is more valuable.  In that case, the bit-
       set is prefixed by a tilde like this:

           sigprocmask(SIG_UNBLOCK, ~[], NULL) = 0

       Here, the second argument represents the full set of all signals.
OPTIONS
   General
       -e expr
              A qualifying expression which modifies which events to
              trace or how to trace them.  The format of the expression
              is:

                             [qualifier=][!]value[,value]...

              where qualifier is one of trace (or t), trace-fds (or
              trace-fd or fd or fds), abbrev (or a), verbose (or v), raw
              (or x), signal (or signals or s), read (or reads or r),
              write (or writes or w), fault, inject, status, quiet (or
              silent or silence or q), secontext, decode-fds (or
              decode-fd), decode-pids (or decode-pid), or kvm, and value
              is a qualifier-dependent symbol or number.  The default
              qualifier is trace.  Using an exclamation mark negates the
              set of values.  For example, -e open means literally
              -e trace=open which in turn means trace only the open
              system call.  By contrast, -e trace=!open means to trace
              every system call except open.  In addition, the special
              values all and none have the obvious meanings.

              Note that some shells use the exclamation point for
              history expansion even inside quoted arguments.  If so,
              you must escape the exclamation point with a backslash.

   Startup
       -E var=val
       --env=var=val
              Run command with var=val in its list of environment
              variables.

       -E var
       --env=var
              Remove var from the inherited list of environment
              variables before passing it on to the command.

       -p pid
       --attach=pid
              Attach to the process with the process ID pid and begin
              tracing.  The trace may be terminated at any time by a
              keyboard interrupt signal (CTRL-C).  strace will respond
              by detaching itself from the traced process(es) leaving it
              (them) to continue running.  Multiple -p options can be
              used to attach to many processes in addition to command
              (which is optional if at least one -p option is given).
              Multiple process IDs, separated by either comma (",''),
              space (" "), tab, or newline character, can be provided as
              an argument to a single -p option, so, for example, -p
              "$(pidof PROG)" and -p "$(pgrep PROG)" syntaxes are
              supported.

       -u username
       --user=username
              Run command with the user ID, group ID, and supplementary
              groups of username.  This option is only useful when
              running as root and enables the correct execution of
              setuid and/or setgid binaries.  Unless this option is used
              setuid and setgid programs are executed without effective
              privileges.
       -u UID:GID
       --user=UID:GID
              Alternative syntax where the program is started with
              exactly the given user and group IDs, and an empty list of
              supplementary groups.  In this case, user and group name
              lookups are not performed.

       --argv0=name
              Set argv[0] of the command being executed to name.  Useful
              for tracing multi-call executables which interpret
              argv[0], such as busybox or kmod.

   Tracing
       -b syscall
       --detach-on=syscall
              If specified syscall is reached, detach from traced
              process.  Currently, only execve(2) syscall is supported.
              This option is useful if you want to trace multi-threaded
              process and therefore require -f, but don't want to trace
              its (potentially very complex) children.

       -D
       --daemonize
       --daemonize=grandchild
              Run tracer process as a grandchild, not as the parent of
              the tracee.  This reduces the visible effect of strace by
              keeping the tracee a direct child of the calling process.

       -DD
       --daemonize=pgroup
       --daemonize=pgrp
              Run tracer process as tracee's grandchild in a separate
              process group.  In addition to reduction of the visible
              effect of strace, it also avoids killing of strace with
              kill(2) issued to the whole process group.

       -DDD
       --daemonize=session
              Run tracer process as tracee's grandchild in a separate
              session ("true daemonisation").  In addition to reduction
              of the visible effect of strace, it also avoids killing of
              strace upon session termination.

       -f
       --follow-forks
              Trace child processes as they are created by currently
              traced processes as a result of the fork(2), vfork(2) and
              clone(2) system calls.  Note that -p PID -f will attach
              all threads of process PID if it is multi-threaded, not
              only thread with thread_id = PID.

       --output-separately
              If the --output=filename option is in effect, each
              processes trace is written to filename.pid where pid is
              the numeric process id of each process.

       -ff
       --follow-forks --output-separately
              Combine the effects of --follow-forks and
              --output-separately options.  This is incompatible with
              -c, since no per-process counts are kept.

              One might want to consider using strace-log-merge(1) to
              obtain a combined strace log view.

       -I interruptible
       --interruptible=interruptible
              When strace can be interrupted by signals (such as
              pressing CTRL-C).

              1, anywhere
                     no signals are blocked;
              2, waiting
                     fatal signals are blocked while decoding syscall
                     (default);
              3, never
                     fatal signals are always blocked (default if -o
                     FILE PROG);
              4, never_tstp
                     fatal signals and SIGTSTP (CTRL-Z) are always
                     blocked (useful to make strace -o FILE PROG not
                     stop on CTRL-Z, default if -D).

       --syscall-limit=limit
              Detach all tracees when limit number of syscalls have been
              captured. Syscalls filtered out via --trace, --trace-path
              or --status options are not considered when keeping track
              of the number of syscalls that are captured.

       --kill-on-exit
              Apply PTRACE_O_EXITKILL ptrace option to all tracee
              processes (which sends a SIGKILL signal to the tracee if
              the tracer exits) and do not detach them on cleanup so
              they will not be left running after the tracer exit.
              --kill-on-exit is not compatible with -p/--attach options.

   Filtering
       -e trace=syscall_set
       -e t=syscall_set
       --trace=syscall_set
              Trace only the specified set of system calls.  syscall_set
              is defined as [!]value[,value], and value can be one of
              the following:

              syscall
                     Trace specific syscall, specified by its name (see
                     syscalls(2) for a reference, but also see NOTES).

              ?value Question mark before the syscall qualification
                     allows suppression of error in case no syscalls
                     matched the qualification provided.

              value@64
                     Limit the syscall specification described by value
                     to 64-bit personality.

              value@32
                     Limit the syscall specification described by value
                     to 32-bit personality.

              value@x32
                     Limit the syscall specification described by value
                     to x32 personality.

              all    Trace all system calls.

              /regex Trace only those system calls that match the regex.
                     You can use POSIX Extended Regular Expression
                     syntax (see regex(7)).

              %file
              file   Trace all system calls which take a file name as an
                     argument.  You can think of this as an abbreviation
                     for -e trace=open,stat,chmod,unlink,...  which is
                     useful to seeing what files the process is
                     referencing.  Furthermore, using the abbreviation
                     will ensure that you don't accidentally forget to
                     include a call like lstat(2) in the list.  Betchya
                     woulda forgot that one.  The syntax without a
                     preceding percent sign ("-e trace=file") is
                     deprecated.

              %process
              process
                     Trace system calls associated with process
                     lifecycle (creation, exec, termination).  The
                     syntax without a preceding percent sign ("-e
                     trace=process") is deprecated.

              %net
              %network
              network
                     Trace all the network related system calls.  The
                     syntax without a preceding percent sign ("-e
                     trace=network") is deprecated.

              %signal
              signal Trace all signal related system calls.  The syntax
                     without a preceding percent sign ("-e
                     trace=signal") is deprecated.

              %ipc
              ipc    Trace all IPC related system calls.  The syntax
                     without a preceding percent sign ("-e trace=ipc")
                     is deprecated.

              %desc
              desc   Trace all file descriptor related system calls.
                     The syntax without a preceding percent sign ("-e
                     trace=desc") is deprecated.

              %memory
              memory Trace all memory mapping related system calls.  The
                     syntax without a preceding percent sign ("-e
                     trace=memory") is deprecated.

              %creds Trace system calls that read or modify user and
                     group identifiers or capability sets.

              %stat  Trace stat syscall variants.

              %lstat Trace lstat syscall variants.

              %fstat Trace fstat, fstatat, and statx syscall variants.

              %%stat Trace syscalls used for requesting file status
                     (stat, lstat, fstat, fstatat, statx, and their
                     variants).

              %statfs
                     Trace statfs, statfs64, statvfs, osf_statfs, and
                     osf_statfs64 system calls.  The same effect can be
                     achieved with -e trace=/^(.*_)?statv?fs regular
                     expression.

              %fstatfs
                     Trace fstatfs, fstatfs64, fstatvfs, osf_fstatfs,
                     and osf_fstatfs64 system calls.  The same effect
                     can be achieved with -e trace=/fstatv?fs regular
                     expression.

              %%statfs
                     Trace syscalls related to file system statistics
                     (statfs-like, fstatfs-like, and ustat).  The same
                     effect can be achieved with
                     -e trace=/statv?fs|fsstat|ustat regular expression.

              %clock Trace system calls that read or modify system
                     clocks.

              %pure  Trace syscalls that always succeed and have no
                     arguments.  Currently, this list includes
                     arc_gettls(2), getdtablesize(2), getegid(2),
                     getegid32(2), geteuid(2), geteuid32(2), getgid(2),
                     getgid32(2), getpagesize(2), getpgrp(2), getpid(2),
                     getppid(2), get_thread_area(2) (on architectures
                     other than x86), gettid(2), get_tls(2), getuid(2),
                     getuid32(2), getxgid(2), getxpid(2), getxuid(2),
                     kern_features(2), and metag_get_tls(2) syscalls.

              The -c option is useful for determining which system calls
              might be useful to trace.  For example,
              trace=open,close,read,write means to only trace those four
              system calls.  Be careful when making inferences about the
              user/kernel boundary if only a subset of system calls are
              being monitored.  The default is trace=all.

       -e trace-fd=set
       -e trace-fds=set
       -e fd=set
       -e fds=set
       --trace-fds=set
              Trace only the syscalls that operate on the specified
              subset of (non-negative) file descriptors.  Note that
              usage of this option also filters out all the syscalls
              that do not operate on file descriptors at all.  Applies
              in (inclusive) disjunction with the --trace-path option.

       -e signal=set
       -e signals=set
       -e s=set
       --signal=set
              Trace only the specified subset of signals.  The default
              is signal=all.  For example, signal=!SIGIO (or signal=!io)
              causes SIGIO signals not to be traced.

       -e status=set
       --status=set
              Print only system calls with the specified return status.
              The default is status=all.  When using the status
              qualifier, because strace waits for system calls to return
              before deciding whether they should be printed or not, the
              traditional order of events may not be preserved anymore.
              If two system calls are executed by concurrent threads,
              strace will first print both the entry and exit of the
              first system call to exit, regardless of their respective
              entry time.  The entry and exit of the second system call
              to exit will be printed afterwards.  Here is an example
              when select(2) is called, but a different thread calls
              clock_gettime(2) before select(2) finishes:

                  [pid 28779] 1130322148.939977 clock_gettime(CLOCK_REALTIME, {1130322148, 939977000}) = 0
                  [pid 28772] 1130322148.438139 select(4, [3], NULL, NULL, NULL) = 1 (in [3])

              set can include the following elements:

              successful
                     Trace system calls that returned without an error
                     code.  The -z option has the effect of
                     status=successful.
              failed Trace system calls that returned with an error
                     code.  The -Z option has the effect of
                     status=failed.
              unfinished
                     Trace system calls that did not return.  This might
                     happen, for example, due to an execve call in a
                     neighbour thread.
              unavailable
                     Trace system calls that returned but strace failed
                     to fetch the error status.
              detached
                     Trace system calls for which strace detached before
                     the return.

       -P path
       --trace-path=path
              Trace only system calls accessing path.  Multiple -P
              options can be used to specify several paths.  Applies in
              (inclusive) disjunction with the --trace-fds option.

       -z
       --successful-only
              Print only syscalls that returned without an error code.

       -Z
       --failed-only
              Print only syscalls that returned with an error code.

   Output format
       -a column
       --columns=column
              Align return values in a specific column (default column
              40).

       -e abbrev=syscall_set
       -e a=syscall_set
       --abbrev=syscall_set
              Abbreviate the output from printing each member of large
              structures.  The syntax of the syscall_set specification
              is the same as in the -e trace option.  The default is
              abbrev=all.  The -v option has the effect of abbrev=none.

       -e verbose=syscall_set
       -e v=syscall_set
       --verbose=syscall_set
              Dereference structures for the specified set of system
              calls.  The syntax of the syscall_set specification is the
              same as in the -e trace option.  The default is
              verbose=all.

       -e raw=syscall_set
       -e x=syscall_set
       --raw=syscall_set
              Print raw, undecoded arguments for the specified set of
              system calls.  The syntax of the syscall_set specification
              is the same as in the -e trace option.  This option has
              the effect of causing all arguments to be printed in
              hexadecimal.  This is mostly useful if you don't trust the
              decoding or you need to know the actual numeric value of
              an argument.  See also -X raw option.

       -e read=set
       -e reads=set
       -e r=set
       --read=set
              Perform a full hexadecimal and ASCII dump of all the data
              read from file descriptors listed in the specified set.
              For example, to see all input activity on file descriptors
              3 and 5 use -e read=3,5.  Note that this is independent
              from the normal tracing of the read(2) system call which
              is controlled by the option -e trace=read.

       -e write=set
       -e writes=set
       -e w=set
       --write=set
              Perform a full hexadecimal and ASCII dump of all the data
              written to file descriptors listed in the specified set.
              For example, to see all output activity on file
              descriptors 3 and 5 use -e write=3,5.  Note that this is
              independent from the normal tracing of the write(2) system
              call which is controlled by the option -e trace=write.

       -e quiet=set
       -e silent=set
       -e silence=set
       -e q=set
       --quiet=set
       --silent=set
       --silence=set
              Suppress various information messages.  The default is
              quiet=none.  set can include the following elements:

              attach Suppress messages about attaching and detaching ("[
                     Process NNNN attached ]", "[ Process NNNN detached
                     ]").
              exit   Suppress messages about process exits ("+++ exited
                     with SSS +++").
              path-resolution
                     Suppress messages about resolution of paths
                     provided via the -P option ("Requested path "..."
                     resolved into "..."").
              personality
                     Suppress messages about process personality changes
                     ("[ Process PID=NNNN runs in PPP mode. ]").
              thread-execve
              superseded
                     Suppress messages about process being superseded by
                     execve(2) in another thread ("+++ superseded by
                     execve in pid NNNN +++").

       -e decode-fds=set
       --decode-fds=set
              Decode various information associated with file
              descriptors.  The default is decode-fds=none.  set can
              include the following elements:

              path     Print file paths.  Also enables printing of
                       tracee's current working directory when AT_FDCWD
                       constant is used.
              socket   Print socket protocol-specific information,
              dev      Print character/block device numbers.
              pidfd    Print PIDs associated with pidfd file
                       descriptors.
              signalfd Print signal masks associated with signalfd file
                       descriptors.

       -e decode-pids=set
       --decode-pids=set
              Decode various information associated with process IDs
              (and also thread IDs, process group IDs, and session IDs).
              The default is decode-pids=none.  set can include the
              following elements:

              comm    Print command names associated with thread or
                      process IDs.
              pidns   Print thread, process, process group, and session
                      IDs in strace's PID namespace if the tracee is in
                      a different PID namespace.

       -e kvm=vcpu
       --kvm=vcpu
              Print the exit reason of kvm vcpu.  Requires Linux kernel
              version 4.16.0 or higher.

       -i
       --instruction-pointer
              Print the instruction pointer at the time of the system
              call.

       -n
       --syscall-number
              Print the syscall number.

       -k
       --stack-trace[=symbol]
              Print the execution stack trace of the traced processes
              after each system call.

       -kk
       --stack-trace=source
              Print the execution stack trace and source code
              information of the traced processes after each system
              call. This option expects the target program is compiled
              with appropriate debug options: "-g" (gcc), or "-g
              -gdwarf-aranges" (clang).

       --stack-trace-frame-limit=limit
              Print no more than this amount of stack trace frames when
              backtracing a system call (the default is 256).  Use this
              option with the --stack-trace (or -k) option.

       -o filename
       --output=filename
              Write the trace output to the file filename rather than to
              stderr.  filename.pid form is used if -ff option is
              supplied.  If the argument begins with '|' or '!', the
              rest of the argument is treated as a command and all
              output is piped to it.  This is convenient for piping the
              debugging output to a program without affecting the
              redirections of executed programs.  The latter is not
              compatible with -ff option currently.

       -A
       --output-append-mode
              Open the file provided in the -o option in append mode.

       -q
       --quiet
       --quiet=attach,personality
              Suppress messages about attaching, detaching, and
              personality changes.  This happens automatically when
              output is redirected to a file and the command is run
              directly instead of attaching.

       -qq
       --quiet=attach,personality,exit
              Suppress messages attaching, detaching, personality
              changes, and about process exit status.

       -qqq
       --quiet=all
              Suppress all suppressible messages (please refer to the -e
              quiet option description for the full list of suppressible
              messages).

       -r
       --relative-timestamps[=precision]
              Print a relative timestamp upon entry to each system call.
              This records the time difference between the beginning of
              successive system calls.  precision can be one of s (for
              seconds), ms (milliseconds), us (microseconds), or ns
              (nanoseconds), and allows setting the precision of time
              value being printed.  Default is us (microseconds).  Note
              that since -r option uses the monotonic clock time for
              measuring time difference and not the wall clock time, its
              measurements can differ from the difference in time
              reported by the -t option.

       -s strsize
       --string-limit=strsize
              Specify the maximum string size to print (the default is
              32).  Note that filenames are not considered strings and
              are always printed in full.

       --absolute-timestamps[=[[format:]format],[[precision:]precision]]
       --timestamps[=[[format:]format],[[precision:]precision]]
              Prefix each line of the trace with the wall clock time in
              the specified format with the specified precision.  format
              can be one of the following:

              none   No time stamp is printed.  Can be used to override
                     the previous setting.
              time   Wall clock time (strftime(3) format string is %T).
              unix   Number of seconds since the epoch (strftime(3)
                     format string is %s).

              precision can be one of s (for seconds), ms
              (milliseconds), us (microseconds), or ns (nanoseconds).
              Default arguments for the option are
              format:time,precision:s.

       -t
       --absolute-timestamps
              Prefix each line of the trace with the wall clock time.

       -tt
       --absolute-timestamps=precision:us
              If given twice, the time printed will include the
              microseconds.

       -ttt
       --absolute-timestamps=format:unix,precision:us
              If given thrice, the time printed will include the
              microseconds and the leading portion will be printed as
              the number of seconds since the epoch.

       -T
       --syscall-times[=precision]
              Show the time spent in system calls.  This records the
              time difference between the beginning and the end of each
              system call.  precision can be one of s (for seconds), ms
              (milliseconds), us (microseconds), or ns (nanoseconds),
              and allows setting the precision of time value being
              printed.  Default is us (microseconds).

       -v
       --no-abbrev
              Print unabbreviated versions of environment, stat,
              termios, etc.  calls.  These structures are very common in
              calls and so the default behavior displays a reasonable
              subset of structure members.  Use this option to get all
              of the gory details.

       --strings-in-hex[=option]
              Control usage of escape sequences with hexadecimal numbers
              in the printed strings.  Normally (when no
              --strings-in-hex or -x option is supplied), escape
              sequences are used to print non-printable and non-ASCII
              characters (that is, characters with a character code less
              than 32 or greater than 127), or to disambiguate the
              output (so, for quotes and other characters that encase
              the printed string, for example, angle brackets, in case
              of file descriptor path output); for the former use case,
              unless it is a white space character that has a symbolic
              escape sequence defined in the C standard (that is, "\textbackslash t"
              for a horizontal tab, "\textbackslash n" for a newline, "\textbackslash v" for a
              vertical tab, "\textbackslash f" for a form feed page break, and "\textbackslash r"
              for a carriage return) are printed using escape sequences
              with numbers that correspond to their byte values, with
              octal number format being the default.  option can be one
              of the following:

              none   Hexadecimal numbers are not used in the output at
                     all.  When there is a need to emit an escape
                     sequence, octal numbers are used.
              non-ascii-chars
                     Hexadecimal numbers are used instead of octal in
                     the escape sequences.
              non-ascii
                     Strings that contain non-ASCII characters are
                     printed using escape sequences with hexadecimal
                     numbers.
              all    All strings are printed using escape sequences with
                     hexadecimal numbers.

              When the option is supplied without an argument, all is
              assumed.

       -x
       --strings-in-hex=non-ascii
              Print all non-ASCII strings in hexadecimal string format.

       -xx
       --strings-in-hex[=all]
              Print all strings in hexadecimal string format.

       -X format
       --const-print-style=format
              Set the format for printing of named constants and flags.
              Supported format values are:

              raw    Raw number output, without decoding.
              abbrev Output a named constant or a set of flags instead
                     of the raw number if they are found.  This is the
                     default strace behaviour.
              verbose
                     Output both the raw value and the decoded string
                     (as a comment).

       -y
       --decode-fds
       --decode-fds=path
              Print paths associated with file descriptor arguments and
              with the AT_FDCWD constant.

       -yy
       --decode-fds=all
              Print all available information associated with file
              descriptors: protocol-specific information associated with
              socket file descriptors, block/character device number
              associated with device file descriptors, and PIDs
              associated with pidfd file descriptors.

       --pidns-translation
       --decode-pids=pidns
              If strace and tracee are in different PID namespaces,
              print PIDs in strace's namespace, too.

       -Y
       --decode-pids=comm
              Print command names for PIDs.

       --secontext[=format]
       -e secontext=format
              When SELinux is available and is not disabled, print in
              square brackets SELinux contexts of processes, files, and
              descriptors.  The format argument is a comma-separated
              list of items being one of the following:

              full              Print the full context (user, role, type
                                level and category).
              mismatch          Also print the context recorded by the
                                SELinux database in case the current
                                context differs.  The latter is printed
                                after two exclamation marks (!!).

              The default value for --secontext is !full,mismatch which
              prints only the type instead of full context and doesn't
              check for context mismatches.

       --always-show-pid
              Show PID prefix also for the process started by strace.
              Implied when -f and -o are both specified.

   Statistics
       -c
       --summary-only
              Count time, calls, and errors for each system call and
              report a summary on program exit, suppressing the regular
              output.  This attempts to show system time (CPU time spent
              running in the kernel) independent of wall clock time.  If
              -c is used with -f, only aggregate totals for all traced
              processes are kept.

       -C
       --summary
              Like -c but also print regular output while processes are
              running.

       -O overhead
       --summary-syscall-overhead=overhead
              Set the overhead for tracing system calls to overhead.
              This is useful for overriding the default heuristic for
              guessing how much time is spent in mere measuring when
              timing system calls using the -c option.  The accuracy of
              the heuristic can be gauged by timing a given program run
              without tracing (using time(1)) and comparing the
              accumulated system call time to the total produced using
              -c.

              The format of overhead specification is described in
              section Time specification format description.

       -S sortby
       --summary-sort-by=sortby
              Sort the output of the histogram printed by the -c option
              by the specified criterion.  Legal values are time (or
              time-percent or time-total or total-time), min-time (or
              shortest or time-min), max-time (or longest or time-max),
              avg-time (or time-avg), calls (or count), errors (or
              error), name (or syscall or syscall-name), and nothing (or
              none); default is time.

       -U columns
       --summary-columns=columns
              Configure a set (and order) of columns being shown in the
              call summary.  The columns argument is a comma-separated
              list with items being one of the following:

              time-percent (or time)
                     Percentage of cumulative time consumed by a
                     specific system call.
              total-time (or time-total)
                     Total system (or wall clock, if -w option is
                     provided) time consumed by a specific system call.
              min-time (or shortest or time-min)
                     Minimum observed call duration.
              max-time (or longest or time-max)
                     Maximum observed call duration.
              avg-time (or time-avg)
                     Average call duration.
              calls (or count)
                     Call count.
              errors (or error)
                     Error count.
              name (or syscall or syscall-name)
                     Syscall name.

              The default value is
              time-percent,total-time,avg-time,calls,errors,name.  If
              the name field is not supplied explicitly, it is added as
              the last column.

       -w
       --summary-wall-clock
              Summarise the time difference between the beginning and
              end of each system call.  The default is to summarise the
              system time.

   Tampering
       -e inject=syscall_set[:error=errno|:retval=value][:signal=sig]
       [:syscall=syscall][:delay_enter=delay][:delay_exit=delay]
       [:poke_enter=@argN=DATAN,@argM=DATAM...]
       [:poke_exit=@argN=DATAN,@argM=DATAM...][:when=expr]
       --inject=syscall_set[:error=errno|:retval=value][:signal=sig]
       [:syscall=syscall][:delay_enter=delay][:delay_exit=delay]
       [:poke_enter=@argN=DATAN,@argM=DATAM...]
       [:poke_exit=@argN=DATAN,@argM=DATAM...][:when=expr]
              Perform   syscall  tampering  for  the  specified  set  of
              syscalls.  The syntax of the syscall_set specification  is
              the same as in the -e trace option.

              At  least  one  of  error,  retval,  signal,  delay_enter,
              delay_exit, poke_enter, or poke_exit  options  has  to  be
              specified.  error and retval are mutually exclusive.

              If  :error=errno  option is specified, a fault is injected
              into a syscall invocation: the syscall number is  replaced
              by  -1  which  corresponds to an invalid syscall (unless a
              syscall is specified with :syscall= option), and the error
              code is specified using a symbolic errno value like ENOSYS
              or a numeric value within 1..4095 range.

              If :retval=value option is specified, success injection is
              performed: the syscall number is replaced  by  -1,  but  a
              bogus success value is returned to the callee.

              If  :signal=sig option is specified with either a symbolic
              value like SIGSEGV or a numeric value  within  1..SIGRTMAX
              range,  that signal is delivered on entering every syscall
              specified by the set.

              If :delay_enter=delay  or  :delay_exit=delay  options  are
              specified,  delay  injection  is  performed: the tracee is
              delayed by time period specified by delay on  entering  or
              exiting  the  syscall,  respectively.  The format of delay
              specification is described in section  Time  specification
              format description.

              If        :poke_enter=@argN=DATAN,@argM=DATAM...        or
              :poke_exit=@argN=DATAN,@argM=DATAM...     options      are
              specified,  tracee's  memory  at  locations, pointed to by
              system call arguments argN and argM (going  from  arg1  to
              arg7) is overwritten by data DATAN and DATAM (specified in
              hexadecimal          format;          for          example
              :poke_enter=@arg1=0000DEAD0000BEEF).  :poke_enter modifies
              memory on syscall enter, and :poke_exit - on exit.

              If :signal=sig option is specified  without  :error=errno,
              :retval=value  or  :delay_{enter,exit}=usecs options, then
              only a signal sig is delivered without a syscall fault  or
              delay     injection.     Conversely,    :error=errno    or
              :retval=value    option    without     :delay_enter=delay,
              :delay_exit=delay  or  :signal=sig options injects a fault
              without delivering a signal or injecting a delay, etc.

              If  :signal=sig  option   is   specified   together   with
              :error=errno  or  :retval=value,  then both injection of a
              fault or success and signal delivery are performed.

              if :syscall=syscall option is specified, the corresponding
              syscall with no side effects is injected  instead  of  -1.
              Currently,  only  "pure"  (see -e trace=%pure description)
              syscalls can be specified there.

              Unless  a  :when=expr  subexpression  is   specified,   an
              injection  is  being  made  into  every invocation of each
              syscall from the set.

              The format of the subexpression is:

                             first[..last][+[step]]

              Number first stands for the first invocation number in the
              range, number last stands for the last  invocation  number
              in  the  range,  and  step stands for the step between two
              consecutive invocations.  The following  combinations  are
              useful:

              first  For every syscall from the set, perform an
                     injection for the syscall invocation number first
                     only.
              first..last
                     For every syscall from the set, perform an
                     injection for the syscall invocation number first
                     and all subsequent invocations until the invocation
                     number last (inclusive).
              first+ For every syscall from the set, perform injections
                     for the syscall invocation number first and all
                     subsequent invocations.
              first..last+
                     For every syscall from the set, perform injections
                     for the syscall invocation number first and all
                     subsequent invocations until the invocation number
                     last (inclusive).
              first+step
                     For every syscall from the set, perform injections
                     for syscall invocations number first, first+step,
                     first+step+step, and so on.
              first..last+step
                     Same as the previous, but consider only syscall
                     invocations with numbers up to last (inclusive).

              For example, to fail each third and subsequent chdir
              syscalls with ENOENT, use
              -e inject=chdir:error=ENOENT:when=3+.

              The valid range for numbers first and step is 1..65535,
              and for number last is 1..65534.

              An injection expression can contain only one error= or
              retval= specification, and only one signal= specification.
              If an injection expression contains multiple when=
              specifications, the last one takes precedence.

              Accounting of syscalls that are subject to injection is
              done per syscall and per tracee.

              Specification of syscall injection can be combined with
              other syscall filtering options, for example, -P
              /dev/urandom -e inject=file:error=ENOENT.

       -e fault=syscall_set[:error=errno][:when=expr]
       --fault=syscall_set[:error=errno][:when=expr]
              Perform syscall fault injection for the specified set of
              syscalls.

              This is equivalent to more generic -e inject= expression
              with default value of errno option set to ENOSYS.

   Miscellaneous
       -d
       --debug
              Show some debugging output of strace itself on the
              standard error.

       -F     This option is deprecated.  It is retained for backward
              compatibility only and may be removed in future releases.
              Usage of multiple instances of -F option is still
              equivalent to a single -f, and it is ignored at all if
              used along with one or more instances of -f option.

       -h
       --help Print the help summary.

       --seccomp-bpf
              Try to enable use of seccomp-bpf (see seccomp(2)) to have
              ptrace(2)-stops only when system calls that are being
              traced occur in the traced processes.

              This option has no effect unless -f/--follow-forks is also
              specified.  --seccomp-bpf is not compatible with
              --syscall-limit and -b/--detach-on options.  It is also
              not applicable to processes attached using -p/--attach
              option.

              An attempt to enable system calls filtering using seccomp-
              bpf may fail for various reasons, e.g. there are too many
              system calls to filter, the seccomp API is not available,
              or strace itself is being traced.  In cases when seccomp-
              bpf filter setup failed, strace proceeds as usual and
              stops traced processes on every system call.

              When --seccomp-bpf is activated and -p/--attach option is
              not used, --kill-on-exit option is activated as well.

              Note that in cases when the tracee has another seccomp
              filter that returns an action value with a precedence
              greater than SECCOMP_RET_TRACE, strace --seccomp-bpf will
              not be notified.  That is, if another seccomp filter, for
              example, disables the syscall or kills the tracee, then
              strace --seccomp-bpf will not be aware of that syscall
              invocation at all.

       --tips[=[[id:]id],[[format:]format]]
              Show strace tips, tricks, and tweaks before exit.  id can
              be a non-negative integer number, which enables printing
              of specific tip, trick, or tweak (these ID are not
              guaranteed to be stable), or random (the default), in
              which case a random tip is printed.  format can be one of
              the following:

              none     No tip is printed.  Can be used to override the
                       previous setting.
              compact  Print the tip just big enough to contain all the
                       text.
              full     Print the tip in its full glory.

              Default is id:random,format:compact.

       -V
       --version
              Print the version number of strace.  Multiple instances of
              the option beyond specific threshold tend to increase
              Strauss awareness.

   Time specification format description
       Time values can be specified as a decimal floating point number
       (in a format accepted by strtod(3)), optionally followed by one
       of the following suffices that specify the unit of time: s
       (seconds), ms (milliseconds), us (microseconds), or ns
       (nanoseconds).  If no suffix is specified, the value is
       interpreted as microseconds.

       The described format is used for -O, -e inject=delay_enter, and
       -e inject=delay_exit options.
DIAGNOSTICS
       When command exits, strace exits with the same exit status.  If
       command is terminated by a signal, strace terminates itself with
       the same signal, so that strace can be used as a wrapper process
       transparent to the invoking parent process.  Note that parent-
       child relationship (signal stop notifications, getppid(2) value,
       etc) between traced process and its parent are not preserved
       unless -D is used.

       When using -p without a command, the exit status of strace is
       zero unless no processes has been attached or there was an
       unexpected error in doing the tracing.
SETUID INSTALLATION
       If strace is installed setuid to root then the invoking user will
       be able to attach to and trace processes owned by any user.  In
       addition setuid and setgid programs will be executed and traced
       with the correct effective privileges.  Since only users trusted
       with full root privileges should be allowed to do these things,
       it only makes sense to install strace as setuid to root when the
       users who can execute it are restricted to those users who have
       this trust.  For example, it makes sense to install a special
       version of strace with mode 'rwsr-xr--', user root and group
       trace, where members of the trace group are trusted users.  If
       you do use this feature, please remember to install a regular
       non-setuid version of strace for ordinary users to use.
MULTIPLE PERSONALITIES SUPPORT
       On some architectures, strace supports decoding of syscalls for
       processes that use different ABI rather than the one strace uses.
       Specifically, in addition to decoding native ABI, strace can
       decode the following ABIs on the following architectures:

       [1]  When strace is built as an x86_64 application
       [2]  When strace is built as an x32 application
       [3]  Big endian only

       This support is optional and relies on ability to generate and
       parse structure definitions during the build time.  Please refer
       to the output of the strace -V command in order to figure out
       what support is available in your strace build ("non-native"
       refers to an ABI that differs from the ABI strace has):

       m32-mpers
              strace can trace and properly decode non-native 32-bit
              binaries.
       no-m32-mpers
              strace can trace, but cannot properly decode non-native
              32-bit binaries.
       mx32-mpers
              strace can trace and properly decode non-native
              32-on-64-bit binaries.
       no-mx32-mpers
              strace can trace, but cannot properly decode non-native
              32-on-64-bit binaries.

       If the output contains neither m32-mpers nor no-m32-mpers, then
       decoding of non-native 32-bit binaries is not implemented at all
       or not applicable.

       Likewise, if the output contains neither mx32-mpers nor no-
       mx32-mpers, then decoding of non-native 32-on-64-bit binaries is
       not implemented at all or not applicable.
NOTES
       It is a pity that so much tracing clutter is produced by systems
       employing shared libraries.

       It is instructive to think about system call inputs and outputs
       as data-flow across the user/kernel boundary.  Because user-space
       and kernel-space are separate and address-protected, it is
       sometimes possible to make deductive inferences about process
       behavior using inputs and outputs as propositions.

       In some cases, a system call will differ from the documented
       behavior or have a different name.  For example, the faccessat(2)
       system call does not have flags argument, and the setrlimit(2)
       library function uses prlimit64(2) system call on modern
       (2.6.38+) kernels.  These discrepancies are normal but
       idiosyncratic characteristics of the system call interface and
       are accounted for by C library wrapper functions.

       Some system calls have different names in different architectures
       and personalities.  In these cases, system call filtering and
       printing uses the names that match corresponding __NR_* kernel
       macros of the tracee's architecture and personality.  There are
       two exceptions from this general rule: arm_fadvise64_64(2) ARM
       syscall and xtensa_fadvise64_64(2) Xtensa syscall are filtered
       and printed as fadvise64_64(2).

       On x32, syscalls that are intended to be used by 64-bit processes
       and not x32 ones (for example, readv(2), that has syscall number
       19 on x86_64, with its x32 counterpart has syscall number 515),
       but called with __X32_SYSCALL_BIT flag being set, are designated
       with #64 suffix.

       On some platforms a process that is attached to with the -p
       option may observe a spurious EINTR return from the current
       system call that is not restartable.  (Ideally, all system calls
       should be restarted on strace attach, making the attach invisible
       to the traced process, but a few system calls aren't.  Arguably,
       every instance of such behavior is a kernel bug.)  This may have
       an unpredictable effect on the process if the process takes no
       action to restart the system call.

       As strace executes the specified command directly and does not
       employ a shell for that, scripts without shebang that usually run
       just fine when invoked by shell fail to execute with ENOEXEC
       error.  It is advisable to manually supply a shell as a command
       with the script as its argument.
BUGS
       Programs that use the setuid bit do not have effective user ID
       privileges while being traced.

       A traced process runs slowly (but check out the --seccomp-bpf
       option).

       Unless --kill-on-exit option is used (or --seccomp-bpf option is
       used in a way that implies --kill-on-exit), traced processes
       which are descended from command may be left running after an
       interrupt signal (CTRL-C).

       By using CLONE_UNTRACED flag of clone system call a tracee can
       break the guarantee that --seccomp-bpf will not leave any
       processes with a seccomp program installed for syscall filtering
       purposes.
HISTORY
       The original strace was written by Paul Kranenburg for SunOS and
       was inspired by its trace utility.  The SunOS version of strace
       was ported to Linux and enhanced by Branko Lankester, who also
       wrote the Linux kernel support.  Even though Paul released strace
       2.5 in 1992, Branko's work was based on Paul's strace 1.5 release
       from 1991.  In 1993, Rick Sladkey merged strace 2.5 for SunOS and
       the second release of strace for Linux, added many of the
       features of truss(1) from SVR4, and produced an strace that
       worked on both platforms.  In 1994 Rick ported strace to SVR4 and
       Solaris and wrote the automatic configuration support.  In 1995
       he ported strace to Irix and became tired of writing about
       himself in the third person.

       Beginning with 1996, strace was maintained by Wichert Akkerman.
       During his tenure, strace development migrated to CVS; ports to
       FreeBSD and many architectures on Linux (including ARM, IA-64,
       MIPS, PA-RISC, PowerPC, s390, SPARC) were introduced.  In 2002,
       the burden of strace maintainership was transferred to Roland
       McGrath.  Since then, strace gained support for several new Linux
       architectures (AMD64, s390x, SuperH), bi-architecture support for
       some of them, and received numerous additions and improvements in
       syscalls decoders on Linux; strace development migrated to Git
       during that period.  Since 2009, strace is actively maintained by
       Dmitry Levin.  strace gained support for AArch64, ARC, AVR32,
       Blackfin, Meta, Nios II, OpenRISC 1000, RISC-V, Tile/TileGx,
       Xtensa architectures since that time.  In 2012, unmaintained and
       apparently broken support for non-Linux operating systems was
       removed.  Also, in 2012 strace gained support for path tracing
       and file descriptor path decoding.  In 2014, support for stack
       trace printing was added.  In 2016, syscall fault injection was
       implemented.

       For the additional information, please refer to the NEWS file and
       strace repository commit log.
REPORTING BUGS
       Problems with strace should be reported to the strace mailing
       list mailto:strace-devel@lists.strace.io.
SEE ALSO
       strace-log-merge(1), ltrace(1), perf-trace(1), trace-cmd(1),
       time(1), ptrace(2), seccomp(2), syscall(2), proc(5), signal(7)

       strace Home Page https://strace.io/
AUTHORS
       The complete list of strace contributors can be found in the
       CREDITS file.
COLOPHON
       This page is part of the strace (system call tracer) project.
       Information about the project can be found at 
       http://strace.io/.  If you have a bug report for this manual
       page, send it to strace-devel@lists.sourceforge.net.  This page
       was obtained from the project's upstream Git repository
       https://github.com/strace/strace.git on 2024-06-14.  (At that
       time, the date of the most recent commit that was found in the
       repository was 2024-06-04.)  If you discover any rendering
       problems in this HTML version of the page, or you believe there
       is a better or more up-to-date source for the page, or you have
       corrections or improvements to the information in this COLOPHON
       (which is not part of the original manual page), send a mail to
       man-pages@man7.org

strace 6.9.0.16.2a4c4          2024-06-04                      STRACE(1)
\end{lstlisting}
}}

\endinput  %  ==  ==  ==  ==  ==  ==  ==  ==  ==
\subsection{\refStrace: Trace System Calls and Signals}

{\tiny{
\begin{lstlisting}[language=bash]
NAME
       strace - trace system calls and signals
SYNOPSIS
       strace [-ACdffhikkqqrtttTvVwxxyyYzZ] [-a column] [-b execve]
              [-e expr]... [-I n] [-o file] [-O overhead] [-p pid]...
              [-P path]... [-s strsize] [-S sortby] [-U columns]
              [-X format] [--seccomp-bpf]
              [--stack-trace-frame-limit=limit] [--syscall-limit=limit]
              [--secontext[=format]] [--tips[=format]] { -p pid | [-DDD]
              [-E var[=val]]... [-u username] command [args] }

       strace -c [-dfwzZ] [-b execve] [-e expr]... [-I n] [-O overhead]
              [-p pid]... [-P path]... [-S sortby] [-U columns]
              [--seccomp-bpf] [--syscall-limit=limit] [--tips[=format]]
              { -p pid | [-DDD] [-E var[=val]]... [-u username] command
              [args] }

       strace --tips[=format]
DESCRIPTION
       In the simplest case strace runs the specified command until it
       exits.  It intercepts and records the system calls which are
       called by a process and the signals which are received by a
       process.  The name of each system call, its arguments and its
       return value are printed on standard error or to the file
       specified with the -o option.

       strace is a useful diagnostic, instructional, and debugging tool.
       System administrators, diagnosticians and trouble-shooters will
       find it invaluable for solving problems with programs for which
       the source is not readily available since they do not need to be
       recompiled in order to trace them.  Students, hackers and the
       overly-curious will find that a great deal can be learned about a
       system and its system calls by tracing even ordinary programs.
       And programmers will find that since system calls and signals are
       events that happen at the user/kernel interface, a close
       examination of this boundary is very useful for bug isolation,
       sanity checking and attempting to capture race conditions.

       Each line in the trace contains the system call name, followed by
       its arguments in parentheses and its return value.  An example
       from stracing the command "cat /dev/null" is:

           open("/dev/null", O_RDONLY) = 3

       Errors (typically a return value of -1) have the errno symbol and
       error string appended.

           open("/foo/bar", O_RDONLY) = -1 ENOENT (No such file or directory)

       Signals are printed as signal symbol and decoded siginfo
       structure.  An excerpt from stracing and interrupting the command
       "sleep 666" is:

           sigsuspend([] <unfinished ...>
           --- SIGINT {si_signo=SIGINT, si_code=SI_USER, si_pid=...} ---
           +++ killed by SIGINT +++

       If a system call is being executed and meanwhile another one is
       being called from a different thread/process then strace will try
       to preserve the order of those events and mark the ongoing call
       as being unfinished.  When the call returns it will be marked as
       resumed.

           [pid 28772] select(4, [3], NULL, NULL, NULL <unfinished ...>
           [pid 28779] clock_gettime(CLOCK_REALTIME, {tv_sec=1130322148, tv_nsec=3977000}) = 0
           [pid 28772] <... select resumed> )      = 1 (in [3])

       Interruption of a (restartable) system call by a signal delivery
       is processed differently as kernel terminates the system call and
       also arranges its immediate reexecution after the signal handler
       completes.

           read(0, 0x7ffff72cf5cf, 1)              = ? ERESTARTSYS (To be restarted)
           --- SIGALRM {si_signo=SIGALRM, si_code=SI_KERNEL} ---
           rt_sigreturn({mask=[]})                 = 0
           read(0, "", 1)                          = 0

       Arguments are printed in symbolic form with passion.  This
       example shows the shell performing ">>xyzzy" output redirection:

           open("xyzzy", O_WRONLY|O_APPEND|O_CREAT, 0666) = 3

       Here, the second and the third argument of open(2) are decoded by
       breaking down the flag argument into its three bitwise-OR
       constituents and printing the mode value in octal by tradition.
       Where the traditional or native usage differs from ANSI or POSIX,
       the latter forms are preferred.  In some cases, strace output is
       proven to be more readable than the source.

       Structure pointers are dereferenced and the members are displayed
       as appropriate.  In most cases, arguments are formatted in the
       most C-like fashion possible.  For example, the essence of the
       command "ls -l /dev/null" is captured as:

           lstat("/dev/null", {st_mode=S_IFCHR|0666, st_rdev=makedev(0x1, 0x3), ...}) = 0

       Notice how the 'struct stat' argument is dereferenced and how
       each member is displayed symbolically.  In particular, observe
       how the st_mode member is carefully decoded into a bitwise-OR of
       symbolic and numeric values.  Also notice in this example that
       the first argument to lstat(2) is an input to the system call and
       the second argument is an output.  Since output arguments are not
       modified if the system call fails, arguments may not always be
       dereferenced.  For example, retrying the "ls -l" example with a
       non-existent file produces the following line:

           lstat("/foo/bar", 0xb004) = -1 ENOENT (No such file or directory)

       In this case the porch light is on but nobody is home.

       Syscalls unknown to strace are printed raw, with the unknown
       system call number printed in hexadecimal form and prefixed with
       "syscall_":

           syscall_0xbad(0x1, 0x2, 0x3, 0x4, 0x5, 0x6) = -1 ENOSYS (Function not implemented)

       Character pointers are dereferenced and printed as C strings.
       Non-printing characters in strings are normally represented by
       ordinary C escape codes.  Only the first strsize (32 by default)
       bytes of strings are printed; longer strings have an ellipsis
       appended following the closing quote.  Here is a line from "ls
       -l" where the getpwuid(3) library routine is reading the password
       file:

           read(3, "root::0:0:System Administrator:/"..., 1024) = 422

       While structures are annotated using curly braces, pointers to
       basic types and arrays are printed using square brackets with
       commas separating the elements.  Here is an example from the
       command id(1) on a system with supplementary group ids:

           getgroups(32, [100, 0]) = 2

       On the other hand, bit-sets are also shown using square brackets,
       but set elements are separated only by a space.  Here is the
       shell, preparing to execute an external command:

           sigprocmask(SIG_BLOCK, [CHLD TTOU], []) = 0

       Here, the second argument is a bit-set of two signals, SIGCHLD
       and SIGTTOU.  In some cases, the bit-set is so full that printing
       out the unset elements is more valuable.  In that case, the bit-
       set is prefixed by a tilde like this:

           sigprocmask(SIG_UNBLOCK, ~[], NULL) = 0

       Here, the second argument represents the full set of all signals.
OPTIONS
   General
       -e expr
              A qualifying expression which modifies which events to
              trace or how to trace them.  The format of the expression
              is:

                             [qualifier=][!]value[,value]...

              where qualifier is one of trace (or t), trace-fds (or
              trace-fd or fd or fds), abbrev (or a), verbose (or v), raw
              (or x), signal (or signals or s), read (or reads or r),
              write (or writes or w), fault, inject, status, quiet (or
              silent or silence or q), secontext, decode-fds (or
              decode-fd), decode-pids (or decode-pid), or kvm, and value
              is a qualifier-dependent symbol or number.  The default
              qualifier is trace.  Using an exclamation mark negates the
              set of values.  For example, -e open means literally
              -e trace=open which in turn means trace only the open
              system call.  By contrast, -e trace=!open means to trace
              every system call except open.  In addition, the special
              values all and none have the obvious meanings.

              Note that some shells use the exclamation point for
              history expansion even inside quoted arguments.  If so,
              you must escape the exclamation point with a backslash.

   Startup
       -E var=val
       --env=var=val
              Run command with var=val in its list of environment
              variables.

       -E var
       --env=var
              Remove var from the inherited list of environment
              variables before passing it on to the command.

       -p pid
       --attach=pid
              Attach to the process with the process ID pid and begin
              tracing.  The trace may be terminated at any time by a
              keyboard interrupt signal (CTRL-C).  strace will respond
              by detaching itself from the traced process(es) leaving it
              (them) to continue running.  Multiple -p options can be
              used to attach to many processes in addition to command
              (which is optional if at least one -p option is given).
              Multiple process IDs, separated by either comma (",''),
              space (" "), tab, or newline character, can be provided as
              an argument to a single -p option, so, for example, -p
              "$(pidof PROG)" and -p "$(pgrep PROG)" syntaxes are
              supported.

       -u username
       --user=username
              Run command with the user ID, group ID, and supplementary
              groups of username.  This option is only useful when
              running as root and enables the correct execution of
              setuid and/or setgid binaries.  Unless this option is used
              setuid and setgid programs are executed without effective
              privileges.
       -u UID:GID
       --user=UID:GID
              Alternative syntax where the program is started with
              exactly the given user and group IDs, and an empty list of
              supplementary groups.  In this case, user and group name
              lookups are not performed.

       --argv0=name
              Set argv[0] of the command being executed to name.  Useful
              for tracing multi-call executables which interpret
              argv[0], such as busybox or kmod.

   Tracing
       -b syscall
       --detach-on=syscall
              If specified syscall is reached, detach from traced
              process.  Currently, only execve(2) syscall is supported.
              This option is useful if you want to trace multi-threaded
              process and therefore require -f, but don't want to trace
              its (potentially very complex) children.

       -D
       --daemonize
       --daemonize=grandchild
              Run tracer process as a grandchild, not as the parent of
              the tracee.  This reduces the visible effect of strace by
              keeping the tracee a direct child of the calling process.

       -DD
       --daemonize=pgroup
       --daemonize=pgrp
              Run tracer process as tracee's grandchild in a separate
              process group.  In addition to reduction of the visible
              effect of strace, it also avoids killing of strace with
              kill(2) issued to the whole process group.

       -DDD
       --daemonize=session
              Run tracer process as tracee's grandchild in a separate
              session ("true daemonisation").  In addition to reduction
              of the visible effect of strace, it also avoids killing of
              strace upon session termination.

       -f
       --follow-forks
              Trace child processes as they are created by currently
              traced processes as a result of the fork(2), vfork(2) and
              clone(2) system calls.  Note that -p PID -f will attach
              all threads of process PID if it is multi-threaded, not
              only thread with thread_id = PID.

       --output-separately
              If the --output=filename option is in effect, each
              processes trace is written to filename.pid where pid is
              the numeric process id of each process.

       -ff
       --follow-forks --output-separately
              Combine the effects of --follow-forks and
              --output-separately options.  This is incompatible with
              -c, since no per-process counts are kept.

              One might want to consider using strace-log-merge(1) to
              obtain a combined strace log view.

       -I interruptible
       --interruptible=interruptible
              When strace can be interrupted by signals (such as
              pressing CTRL-C).

              1, anywhere
                     no signals are blocked;
              2, waiting
                     fatal signals are blocked while decoding syscall
                     (default);
              3, never
                     fatal signals are always blocked (default if -o
                     FILE PROG);
              4, never_tstp
                     fatal signals and SIGTSTP (CTRL-Z) are always
                     blocked (useful to make strace -o FILE PROG not
                     stop on CTRL-Z, default if -D).

       --syscall-limit=limit
              Detach all tracees when limit number of syscalls have been
              captured. Syscalls filtered out via --trace, --trace-path
              or --status options are not considered when keeping track
              of the number of syscalls that are captured.

       --kill-on-exit
              Apply PTRACE_O_EXITKILL ptrace option to all tracee
              processes (which sends a SIGKILL signal to the tracee if
              the tracer exits) and do not detach them on cleanup so
              they will not be left running after the tracer exit.
              --kill-on-exit is not compatible with -p/--attach options.

   Filtering
       -e trace=syscall_set
       -e t=syscall_set
       --trace=syscall_set
              Trace only the specified set of system calls.  syscall_set
              is defined as [!]value[,value], and value can be one of
              the following:

              syscall
                     Trace specific syscall, specified by its name (see
                     syscalls(2) for a reference, but also see NOTES).

              ?value Question mark before the syscall qualification
                     allows suppression of error in case no syscalls
                     matched the qualification provided.

              value@64
                     Limit the syscall specification described by value
                     to 64-bit personality.

              value@32
                     Limit the syscall specification described by value
                     to 32-bit personality.

              value@x32
                     Limit the syscall specification described by value
                     to x32 personality.

              all    Trace all system calls.

              /regex Trace only those system calls that match the regex.
                     You can use POSIX Extended Regular Expression
                     syntax (see regex(7)).

              %file
              file   Trace all system calls which take a file name as an
                     argument.  You can think of this as an abbreviation
                     for -e trace=open,stat,chmod,unlink,...  which is
                     useful to seeing what files the process is
                     referencing.  Furthermore, using the abbreviation
                     will ensure that you don't accidentally forget to
                     include a call like lstat(2) in the list.  Betchya
                     woulda forgot that one.  The syntax without a
                     preceding percent sign ("-e trace=file") is
                     deprecated.

              %process
              process
                     Trace system calls associated with process
                     lifecycle (creation, exec, termination).  The
                     syntax without a preceding percent sign ("-e
                     trace=process") is deprecated.

              %net
              %network
              network
                     Trace all the network related system calls.  The
                     syntax without a preceding percent sign ("-e
                     trace=network") is deprecated.

              %signal
              signal Trace all signal related system calls.  The syntax
                     without a preceding percent sign ("-e
                     trace=signal") is deprecated.

              %ipc
              ipc    Trace all IPC related system calls.  The syntax
                     without a preceding percent sign ("-e trace=ipc")
                     is deprecated.

              %desc
              desc   Trace all file descriptor related system calls.
                     The syntax without a preceding percent sign ("-e
                     trace=desc") is deprecated.

              %memory
              memory Trace all memory mapping related system calls.  The
                     syntax without a preceding percent sign ("-e
                     trace=memory") is deprecated.

              %creds Trace system calls that read or modify user and
                     group identifiers or capability sets.

              %stat  Trace stat syscall variants.

              %lstat Trace lstat syscall variants.

              %fstat Trace fstat, fstatat, and statx syscall variants.

              %%stat Trace syscalls used for requesting file status
                     (stat, lstat, fstat, fstatat, statx, and their
                     variants).

              %statfs
                     Trace statfs, statfs64, statvfs, osf_statfs, and
                     osf_statfs64 system calls.  The same effect can be
                     achieved with -e trace=/^(.*_)?statv?fs regular
                     expression.

              %fstatfs
                     Trace fstatfs, fstatfs64, fstatvfs, osf_fstatfs,
                     and osf_fstatfs64 system calls.  The same effect
                     can be achieved with -e trace=/fstatv?fs regular
                     expression.

              %%statfs
                     Trace syscalls related to file system statistics
                     (statfs-like, fstatfs-like, and ustat).  The same
                     effect can be achieved with
                     -e trace=/statv?fs|fsstat|ustat regular expression.

              %clock Trace system calls that read or modify system
                     clocks.

              %pure  Trace syscalls that always succeed and have no
                     arguments.  Currently, this list includes
                     arc_gettls(2), getdtablesize(2), getegid(2),
                     getegid32(2), geteuid(2), geteuid32(2), getgid(2),
                     getgid32(2), getpagesize(2), getpgrp(2), getpid(2),
                     getppid(2), get_thread_area(2) (on architectures
                     other than x86), gettid(2), get_tls(2), getuid(2),
                     getuid32(2), getxgid(2), getxpid(2), getxuid(2),
                     kern_features(2), and metag_get_tls(2) syscalls.

              The -c option is useful for determining which system calls
              might be useful to trace.  For example,
              trace=open,close,read,write means to only trace those four
              system calls.  Be careful when making inferences about the
              user/kernel boundary if only a subset of system calls are
              being monitored.  The default is trace=all.

       -e trace-fd=set
       -e trace-fds=set
       -e fd=set
       -e fds=set
       --trace-fds=set
              Trace only the syscalls that operate on the specified
              subset of (non-negative) file descriptors.  Note that
              usage of this option also filters out all the syscalls
              that do not operate on file descriptors at all.  Applies
              in (inclusive) disjunction with the --trace-path option.

       -e signal=set
       -e signals=set
       -e s=set
       --signal=set
              Trace only the specified subset of signals.  The default
              is signal=all.  For example, signal=!SIGIO (or signal=!io)
              causes SIGIO signals not to be traced.

       -e status=set
       --status=set
              Print only system calls with the specified return status.
              The default is status=all.  When using the status
              qualifier, because strace waits for system calls to return
              before deciding whether they should be printed or not, the
              traditional order of events may not be preserved anymore.
              If two system calls are executed by concurrent threads,
              strace will first print both the entry and exit of the
              first system call to exit, regardless of their respective
              entry time.  The entry and exit of the second system call
              to exit will be printed afterwards.  Here is an example
              when select(2) is called, but a different thread calls
              clock_gettime(2) before select(2) finishes:

                  [pid 28779] 1130322148.939977 clock_gettime(CLOCK_REALTIME, {1130322148, 939977000}) = 0
                  [pid 28772] 1130322148.438139 select(4, [3], NULL, NULL, NULL) = 1 (in [3])

              set can include the following elements:

              successful
                     Trace system calls that returned without an error
                     code.  The -z option has the effect of
                     status=successful.
              failed Trace system calls that returned with an error
                     code.  The -Z option has the effect of
                     status=failed.
              unfinished
                     Trace system calls that did not return.  This might
                     happen, for example, due to an execve call in a
                     neighbour thread.
              unavailable
                     Trace system calls that returned but strace failed
                     to fetch the error status.
              detached
                     Trace system calls for which strace detached before
                     the return.

       -P path
       --trace-path=path
              Trace only system calls accessing path.  Multiple -P
              options can be used to specify several paths.  Applies in
              (inclusive) disjunction with the --trace-fds option.

       -z
       --successful-only
              Print only syscalls that returned without an error code.

       -Z
       --failed-only
              Print only syscalls that returned with an error code.

   Output format
       -a column
       --columns=column
              Align return values in a specific column (default column
              40).

       -e abbrev=syscall_set
       -e a=syscall_set
       --abbrev=syscall_set
              Abbreviate the output from printing each member of large
              structures.  The syntax of the syscall_set specification
              is the same as in the -e trace option.  The default is
              abbrev=all.  The -v option has the effect of abbrev=none.

       -e verbose=syscall_set
       -e v=syscall_set
       --verbose=syscall_set
              Dereference structures for the specified set of system
              calls.  The syntax of the syscall_set specification is the
              same as in the -e trace option.  The default is
              verbose=all.

       -e raw=syscall_set
       -e x=syscall_set
       --raw=syscall_set
              Print raw, undecoded arguments for the specified set of
              system calls.  The syntax of the syscall_set specification
              is the same as in the -e trace option.  This option has
              the effect of causing all arguments to be printed in
              hexadecimal.  This is mostly useful if you don't trust the
              decoding or you need to know the actual numeric value of
              an argument.  See also -X raw option.

       -e read=set
       -e reads=set
       -e r=set
       --read=set
              Perform a full hexadecimal and ASCII dump of all the data
              read from file descriptors listed in the specified set.
              For example, to see all input activity on file descriptors
              3 and 5 use -e read=3,5.  Note that this is independent
              from the normal tracing of the read(2) system call which
              is controlled by the option -e trace=read.

       -e write=set
       -e writes=set
       -e w=set
       --write=set
              Perform a full hexadecimal and ASCII dump of all the data
              written to file descriptors listed in the specified set.
              For example, to see all output activity on file
              descriptors 3 and 5 use -e write=3,5.  Note that this is
              independent from the normal tracing of the write(2) system
              call which is controlled by the option -e trace=write.

       -e quiet=set
       -e silent=set
       -e silence=set
       -e q=set
       --quiet=set
       --silent=set
       --silence=set
              Suppress various information messages.  The default is
              quiet=none.  set can include the following elements:

              attach Suppress messages about attaching and detaching ("[
                     Process NNNN attached ]", "[ Process NNNN detached
                     ]").
              exit   Suppress messages about process exits ("+++ exited
                     with SSS +++").
              path-resolution
                     Suppress messages about resolution of paths
                     provided via the -P option ("Requested path "..."
                     resolved into "..."").
              personality
                     Suppress messages about process personality changes
                     ("[ Process PID=NNNN runs in PPP mode. ]").
              thread-execve
              superseded
                     Suppress messages about process being superseded by
                     execve(2) in another thread ("+++ superseded by
                     execve in pid NNNN +++").

       -e decode-fds=set
       --decode-fds=set
              Decode various information associated with file
              descriptors.  The default is decode-fds=none.  set can
              include the following elements:

              path     Print file paths.  Also enables printing of
                       tracee's current working directory when AT_FDCWD
                       constant is used.
              socket   Print socket protocol-specific information,
              dev      Print character/block device numbers.
              pidfd    Print PIDs associated with pidfd file
                       descriptors.
              signalfd Print signal masks associated with signalfd file
                       descriptors.

       -e decode-pids=set
       --decode-pids=set
              Decode various information associated with process IDs
              (and also thread IDs, process group IDs, and session IDs).
              The default is decode-pids=none.  set can include the
              following elements:

              comm    Print command names associated with thread or
                      process IDs.
              pidns   Print thread, process, process group, and session
                      IDs in strace's PID namespace if the tracee is in
                      a different PID namespace.

       -e kvm=vcpu
       --kvm=vcpu
              Print the exit reason of kvm vcpu.  Requires Linux kernel
              version 4.16.0 or higher.

       -i
       --instruction-pointer
              Print the instruction pointer at the time of the system
              call.

       -n
       --syscall-number
              Print the syscall number.

       -k
       --stack-trace[=symbol]
              Print the execution stack trace of the traced processes
              after each system call.

       -kk
       --stack-trace=source
              Print the execution stack trace and source code
              information of the traced processes after each system
              call. This option expects the target program is compiled
              with appropriate debug options: "-g" (gcc), or "-g
              -gdwarf-aranges" (clang).

       --stack-trace-frame-limit=limit
              Print no more than this amount of stack trace frames when
              backtracing a system call (the default is 256).  Use this
              option with the --stack-trace (or -k) option.

       -o filename
       --output=filename
              Write the trace output to the file filename rather than to
              stderr.  filename.pid form is used if -ff option is
              supplied.  If the argument begins with '|' or '!', the
              rest of the argument is treated as a command and all
              output is piped to it.  This is convenient for piping the
              debugging output to a program without affecting the
              redirections of executed programs.  The latter is not
              compatible with -ff option currently.

       -A
       --output-append-mode
              Open the file provided in the -o option in append mode.

       -q
       --quiet
       --quiet=attach,personality
              Suppress messages about attaching, detaching, and
              personality changes.  This happens automatically when
              output is redirected to a file and the command is run
              directly instead of attaching.

       -qq
       --quiet=attach,personality,exit
              Suppress messages attaching, detaching, personality
              changes, and about process exit status.

       -qqq
       --quiet=all
              Suppress all suppressible messages (please refer to the -e
              quiet option description for the full list of suppressible
              messages).

       -r
       --relative-timestamps[=precision]
              Print a relative timestamp upon entry to each system call.
              This records the time difference between the beginning of
              successive system calls.  precision can be one of s (for
              seconds), ms (milliseconds), us (microseconds), or ns
              (nanoseconds), and allows setting the precision of time
              value being printed.  Default is us (microseconds).  Note
              that since -r option uses the monotonic clock time for
              measuring time difference and not the wall clock time, its
              measurements can differ from the difference in time
              reported by the -t option.

       -s strsize
       --string-limit=strsize
              Specify the maximum string size to print (the default is
              32).  Note that filenames are not considered strings and
              are always printed in full.

       --absolute-timestamps[=[[format:]format],[[precision:]precision]]
       --timestamps[=[[format:]format],[[precision:]precision]]
              Prefix each line of the trace with the wall clock time in
              the specified format with the specified precision.  format
              can be one of the following:

              none   No time stamp is printed.  Can be used to override
                     the previous setting.
              time   Wall clock time (strftime(3) format string is %T).
              unix   Number of seconds since the epoch (strftime(3)
                     format string is %s).

              precision can be one of s (for seconds), ms
              (milliseconds), us (microseconds), or ns (nanoseconds).
              Default arguments for the option are
              format:time,precision:s.

       -t
       --absolute-timestamps
              Prefix each line of the trace with the wall clock time.

       -tt
       --absolute-timestamps=precision:us
              If given twice, the time printed will include the
              microseconds.

       -ttt
       --absolute-timestamps=format:unix,precision:us
              If given thrice, the time printed will include the
              microseconds and the leading portion will be printed as
              the number of seconds since the epoch.

       -T
       --syscall-times[=precision]
              Show the time spent in system calls.  This records the
              time difference between the beginning and the end of each
              system call.  precision can be one of s (for seconds), ms
              (milliseconds), us (microseconds), or ns (nanoseconds),
              and allows setting the precision of time value being
              printed.  Default is us (microseconds).

       -v
       --no-abbrev
              Print unabbreviated versions of environment, stat,
              termios, etc.  calls.  These structures are very common in
              calls and so the default behavior displays a reasonable
              subset of structure members.  Use this option to get all
              of the gory details.

       --strings-in-hex[=option]
              Control usage of escape sequences with hexadecimal numbers
              in the printed strings.  Normally (when no
              --strings-in-hex or -x option is supplied), escape
              sequences are used to print non-printable and non-ASCII
              characters (that is, characters with a character code less
              than 32 or greater than 127), or to disambiguate the
              output (so, for quotes and other characters that encase
              the printed string, for example, angle brackets, in case
              of file descriptor path output); for the former use case,
              unless it is a white space character that has a symbolic
              escape sequence defined in the C standard (that is, "\textbackslash t"
              for a horizontal tab, "\textbackslash n" for a newline, "\textbackslash v" for a
              vertical tab, "\textbackslash f" for a form feed page break, and "\textbackslash r"
              for a carriage return) are printed using escape sequences
              with numbers that correspond to their byte values, with
              octal number format being the default.  option can be one
              of the following:

              none   Hexadecimal numbers are not used in the output at
                     all.  When there is a need to emit an escape
                     sequence, octal numbers are used.
              non-ascii-chars
                     Hexadecimal numbers are used instead of octal in
                     the escape sequences.
              non-ascii
                     Strings that contain non-ASCII characters are
                     printed using escape sequences with hexadecimal
                     numbers.
              all    All strings are printed using escape sequences with
                     hexadecimal numbers.

              When the option is supplied without an argument, all is
              assumed.

       -x
       --strings-in-hex=non-ascii
              Print all non-ASCII strings in hexadecimal string format.

       -xx
       --strings-in-hex[=all]
              Print all strings in hexadecimal string format.

       -X format
       --const-print-style=format
              Set the format for printing of named constants and flags.
              Supported format values are:

              raw    Raw number output, without decoding.
              abbrev Output a named constant or a set of flags instead
                     of the raw number if they are found.  This is the
                     default strace behaviour.
              verbose
                     Output both the raw value and the decoded string
                     (as a comment).

       -y
       --decode-fds
       --decode-fds=path
              Print paths associated with file descriptor arguments and
              with the AT_FDCWD constant.

       -yy
       --decode-fds=all
              Print all available information associated with file
              descriptors: protocol-specific information associated with
              socket file descriptors, block/character device number
              associated with device file descriptors, and PIDs
              associated with pidfd file descriptors.

       --pidns-translation
       --decode-pids=pidns
              If strace and tracee are in different PID namespaces,
              print PIDs in strace's namespace, too.

       -Y
       --decode-pids=comm
              Print command names for PIDs.

       --secontext[=format]
       -e secontext=format
              When SELinux is available and is not disabled, print in
              square brackets SELinux contexts of processes, files, and
              descriptors.  The format argument is a comma-separated
              list of items being one of the following:

              full              Print the full context (user, role, type
                                level and category).
              mismatch          Also print the context recorded by the
                                SELinux database in case the current
                                context differs.  The latter is printed
                                after two exclamation marks (!!).

              The default value for --secontext is !full,mismatch which
              prints only the type instead of full context and doesn't
              check for context mismatches.

       --always-show-pid
              Show PID prefix also for the process started by strace.
              Implied when -f and -o are both specified.

   Statistics
       -c
       --summary-only
              Count time, calls, and errors for each system call and
              report a summary on program exit, suppressing the regular
              output.  This attempts to show system time (CPU time spent
              running in the kernel) independent of wall clock time.  If
              -c is used with -f, only aggregate totals for all traced
              processes are kept.

       -C
       --summary
              Like -c but also print regular output while processes are
              running.

       -O overhead
       --summary-syscall-overhead=overhead
              Set the overhead for tracing system calls to overhead.
              This is useful for overriding the default heuristic for
              guessing how much time is spent in mere measuring when
              timing system calls using the -c option.  The accuracy of
              the heuristic can be gauged by timing a given program run
              without tracing (using time(1)) and comparing the
              accumulated system call time to the total produced using
              -c.

              The format of overhead specification is described in
              section Time specification format description.

       -S sortby
       --summary-sort-by=sortby
              Sort the output of the histogram printed by the -c option
              by the specified criterion.  Legal values are time (or
              time-percent or time-total or total-time), min-time (or
              shortest or time-min), max-time (or longest or time-max),
              avg-time (or time-avg), calls (or count), errors (or
              error), name (or syscall or syscall-name), and nothing (or
              none); default is time.

       -U columns
       --summary-columns=columns
              Configure a set (and order) of columns being shown in the
              call summary.  The columns argument is a comma-separated
              list with items being one of the following:

              time-percent (or time)
                     Percentage of cumulative time consumed by a
                     specific system call.
              total-time (or time-total)
                     Total system (or wall clock, if -w option is
                     provided) time consumed by a specific system call.
              min-time (or shortest or time-min)
                     Minimum observed call duration.
              max-time (or longest or time-max)
                     Maximum observed call duration.
              avg-time (or time-avg)
                     Average call duration.
              calls (or count)
                     Call count.
              errors (or error)
                     Error count.
              name (or syscall or syscall-name)
                     Syscall name.

              The default value is
              time-percent,total-time,avg-time,calls,errors,name.  If
              the name field is not supplied explicitly, it is added as
              the last column.

       -w
       --summary-wall-clock
              Summarise the time difference between the beginning and
              end of each system call.  The default is to summarise the
              system time.

   Tampering
       -e inject=syscall_set[:error=errno|:retval=value][:signal=sig]
       [:syscall=syscall][:delay_enter=delay][:delay_exit=delay]
       [:poke_enter=@argN=DATAN,@argM=DATAM...]
       [:poke_exit=@argN=DATAN,@argM=DATAM...][:when=expr]
       --inject=syscall_set[:error=errno|:retval=value][:signal=sig]
       [:syscall=syscall][:delay_enter=delay][:delay_exit=delay]
       [:poke_enter=@argN=DATAN,@argM=DATAM...]
       [:poke_exit=@argN=DATAN,@argM=DATAM...][:when=expr]
              Perform   syscall  tampering  for  the  specified  set  of
              syscalls.  The syntax of the syscall_set specification  is
              the same as in the -e trace option.

              At  least  one  of  error,  retval,  signal,  delay_enter,
              delay_exit, poke_enter, or poke_exit  options  has  to  be
              specified.  error and retval are mutually exclusive.

              If  :error=errno  option is specified, a fault is injected
              into a syscall invocation: the syscall number is  replaced
              by  -1  which  corresponds to an invalid syscall (unless a
              syscall is specified with :syscall= option), and the error
              code is specified using a symbolic errno value like ENOSYS
              or a numeric value within 1..4095 range.

              If :retval=value option is specified, success injection is
              performed: the syscall number is replaced  by  -1,  but  a
              bogus success value is returned to the callee.

              If  :signal=sig option is specified with either a symbolic
              value like SIGSEGV or a numeric value  within  1..SIGRTMAX
              range,  that signal is delivered on entering every syscall
              specified by the set.

              If :delay_enter=delay  or  :delay_exit=delay  options  are
              specified,  delay  injection  is  performed: the tracee is
              delayed by time period specified by delay on  entering  or
              exiting  the  syscall,  respectively.  The format of delay
              specification is described in section  Time  specification
              format description.

              If        :poke_enter=@argN=DATAN,@argM=DATAM...        or
              :poke_exit=@argN=DATAN,@argM=DATAM...     options      are
              specified,  tracee's  memory  at  locations, pointed to by
              system call arguments argN and argM (going  from  arg1  to
              arg7) is overwritten by data DATAN and DATAM (specified in
              hexadecimal          format;          for          example
              :poke_enter=@arg1=0000DEAD0000BEEF).  :poke_enter modifies
              memory on syscall enter, and :poke_exit - on exit.

              If :signal=sig option is specified  without  :error=errno,
              :retval=value  or  :delay_{enter,exit}=usecs options, then
              only a signal sig is delivered without a syscall fault  or
              delay     injection.     Conversely,    :error=errno    or
              :retval=value    option    without     :delay_enter=delay,
              :delay_exit=delay  or  :signal=sig options injects a fault
              without delivering a signal or injecting a delay, etc.

              If  :signal=sig  option   is   specified   together   with
              :error=errno  or  :retval=value,  then both injection of a
              fault or success and signal delivery are performed.

              if :syscall=syscall option is specified, the corresponding
              syscall with no side effects is injected  instead  of  -1.
              Currently,  only  "pure"  (see -e trace=%pure description)
              syscalls can be specified there.

              Unless  a  :when=expr  subexpression  is   specified,   an
              injection  is  being  made  into  every invocation of each
              syscall from the set.

              The format of the subexpression is:

                             first[..last][+[step]]

              Number first stands for the first invocation number in the
              range, number last stands for the last  invocation  number
              in  the  range,  and  step stands for the step between two
              consecutive invocations.  The following  combinations  are
              useful:

              first  For every syscall from the set, perform an
                     injection for the syscall invocation number first
                     only.
              first..last
                     For every syscall from the set, perform an
                     injection for the syscall invocation number first
                     and all subsequent invocations until the invocation
                     number last (inclusive).
              first+ For every syscall from the set, perform injections
                     for the syscall invocation number first and all
                     subsequent invocations.
              first..last+
                     For every syscall from the set, perform injections
                     for the syscall invocation number first and all
                     subsequent invocations until the invocation number
                     last (inclusive).
              first+step
                     For every syscall from the set, perform injections
                     for syscall invocations number first, first+step,
                     first+step+step, and so on.
              first..last+step
                     Same as the previous, but consider only syscall
                     invocations with numbers up to last (inclusive).

              For example, to fail each third and subsequent chdir
              syscalls with ENOENT, use
              -e inject=chdir:error=ENOENT:when=3+.

              The valid range for numbers first and step is 1..65535,
              and for number last is 1..65534.

              An injection expression can contain only one error= or
              retval= specification, and only one signal= specification.
              If an injection expression contains multiple when=
              specifications, the last one takes precedence.

              Accounting of syscalls that are subject to injection is
              done per syscall and per tracee.

              Specification of syscall injection can be combined with
              other syscall filtering options, for example, -P
              /dev/urandom -e inject=file:error=ENOENT.

       -e fault=syscall_set[:error=errno][:when=expr]
       --fault=syscall_set[:error=errno][:when=expr]
              Perform syscall fault injection for the specified set of
              syscalls.

              This is equivalent to more generic -e inject= expression
              with default value of errno option set to ENOSYS.

   Miscellaneous
       -d
       --debug
              Show some debugging output of strace itself on the
              standard error.

       -F     This option is deprecated.  It is retained for backward
              compatibility only and may be removed in future releases.
              Usage of multiple instances of -F option is still
              equivalent to a single -f, and it is ignored at all if
              used along with one or more instances of -f option.

       -h
       --help Print the help summary.

       --seccomp-bpf
              Try to enable use of seccomp-bpf (see seccomp(2)) to have
              ptrace(2)-stops only when system calls that are being
              traced occur in the traced processes.

              This option has no effect unless -f/--follow-forks is also
              specified.  --seccomp-bpf is not compatible with
              --syscall-limit and -b/--detach-on options.  It is also
              not applicable to processes attached using -p/--attach
              option.

              An attempt to enable system calls filtering using seccomp-
              bpf may fail for various reasons, e.g. there are too many
              system calls to filter, the seccomp API is not available,
              or strace itself is being traced.  In cases when seccomp-
              bpf filter setup failed, strace proceeds as usual and
              stops traced processes on every system call.

              When --seccomp-bpf is activated and -p/--attach option is
              not used, --kill-on-exit option is activated as well.

              Note that in cases when the tracee has another seccomp
              filter that returns an action value with a precedence
              greater than SECCOMP_RET_TRACE, strace --seccomp-bpf will
              not be notified.  That is, if another seccomp filter, for
              example, disables the syscall or kills the tracee, then
              strace --seccomp-bpf will not be aware of that syscall
              invocation at all.

       --tips[=[[id:]id],[[format:]format]]
              Show strace tips, tricks, and tweaks before exit.  id can
              be a non-negative integer number, which enables printing
              of specific tip, trick, or tweak (these ID are not
              guaranteed to be stable), or random (the default), in
              which case a random tip is printed.  format can be one of
              the following:

              none     No tip is printed.  Can be used to override the
                       previous setting.
              compact  Print the tip just big enough to contain all the
                       text.
              full     Print the tip in its full glory.

              Default is id:random,format:compact.

       -V
       --version
              Print the version number of strace.  Multiple instances of
              the option beyond specific threshold tend to increase
              Strauss awareness.

   Time specification format description
       Time values can be specified as a decimal floating point number
       (in a format accepted by strtod(3)), optionally followed by one
       of the following suffices that specify the unit of time: s
       (seconds), ms (milliseconds), us (microseconds), or ns
       (nanoseconds).  If no suffix is specified, the value is
       interpreted as microseconds.

       The described format is used for -O, -e inject=delay_enter, and
       -e inject=delay_exit options.
DIAGNOSTICS
       When command exits, strace exits with the same exit status.  If
       command is terminated by a signal, strace terminates itself with
       the same signal, so that strace can be used as a wrapper process
       transparent to the invoking parent process.  Note that parent-
       child relationship (signal stop notifications, getppid(2) value,
       etc) between traced process and its parent are not preserved
       unless -D is used.

       When using -p without a command, the exit status of strace is
       zero unless no processes has been attached or there was an
       unexpected error in doing the tracing.
SETUID INSTALLATION
       If strace is installed setuid to root then the invoking user will
       be able to attach to and trace processes owned by any user.  In
       addition setuid and setgid programs will be executed and traced
       with the correct effective privileges.  Since only users trusted
       with full root privileges should be allowed to do these things,
       it only makes sense to install strace as setuid to root when the
       users who can execute it are restricted to those users who have
       this trust.  For example, it makes sense to install a special
       version of strace with mode 'rwsr-xr--', user root and group
       trace, where members of the trace group are trusted users.  If
       you do use this feature, please remember to install a regular
       non-setuid version of strace for ordinary users to use.
MULTIPLE PERSONALITIES SUPPORT
       On some architectures, strace supports decoding of syscalls for
       processes that use different ABI rather than the one strace uses.
       Specifically, in addition to decoding native ABI, strace can
       decode the following ABIs on the following architectures:

       [1]  When strace is built as an x86_64 application
       [2]  When strace is built as an x32 application
       [3]  Big endian only

       This support is optional and relies on ability to generate and
       parse structure definitions during the build time.  Please refer
       to the output of the strace -V command in order to figure out
       what support is available in your strace build ("non-native"
       refers to an ABI that differs from the ABI strace has):

       m32-mpers
              strace can trace and properly decode non-native 32-bit
              binaries.
       no-m32-mpers
              strace can trace, but cannot properly decode non-native
              32-bit binaries.
       mx32-mpers
              strace can trace and properly decode non-native
              32-on-64-bit binaries.
       no-mx32-mpers
              strace can trace, but cannot properly decode non-native
              32-on-64-bit binaries.

       If the output contains neither m32-mpers nor no-m32-mpers, then
       decoding of non-native 32-bit binaries is not implemented at all
       or not applicable.

       Likewise, if the output contains neither mx32-mpers nor no-
       mx32-mpers, then decoding of non-native 32-on-64-bit binaries is
       not implemented at all or not applicable.
NOTES
       It is a pity that so much tracing clutter is produced by systems
       employing shared libraries.

       It is instructive to think about system call inputs and outputs
       as data-flow across the user/kernel boundary.  Because user-space
       and kernel-space are separate and address-protected, it is
       sometimes possible to make deductive inferences about process
       behavior using inputs and outputs as propositions.

       In some cases, a system call will differ from the documented
       behavior or have a different name.  For example, the faccessat(2)
       system call does not have flags argument, and the setrlimit(2)
       library function uses prlimit64(2) system call on modern
       (2.6.38+) kernels.  These discrepancies are normal but
       idiosyncratic characteristics of the system call interface and
       are accounted for by C library wrapper functions.

       Some system calls have different names in different architectures
       and personalities.  In these cases, system call filtering and
       printing uses the names that match corresponding __NR_* kernel
       macros of the tracee's architecture and personality.  There are
       two exceptions from this general rule: arm_fadvise64_64(2) ARM
       syscall and xtensa_fadvise64_64(2) Xtensa syscall are filtered
       and printed as fadvise64_64(2).

       On x32, syscalls that are intended to be used by 64-bit processes
       and not x32 ones (for example, readv(2), that has syscall number
       19 on x86_64, with its x32 counterpart has syscall number 515),
       but called with __X32_SYSCALL_BIT flag being set, are designated
       with #64 suffix.

       On some platforms a process that is attached to with the -p
       option may observe a spurious EINTR return from the current
       system call that is not restartable.  (Ideally, all system calls
       should be restarted on strace attach, making the attach invisible
       to the traced process, but a few system calls aren't.  Arguably,
       every instance of such behavior is a kernel bug.)  This may have
       an unpredictable effect on the process if the process takes no
       action to restart the system call.

       As strace executes the specified command directly and does not
       employ a shell for that, scripts without shebang that usually run
       just fine when invoked by shell fail to execute with ENOEXEC
       error.  It is advisable to manually supply a shell as a command
       with the script as its argument.
BUGS
       Programs that use the setuid bit do not have effective user ID
       privileges while being traced.

       A traced process runs slowly (but check out the --seccomp-bpf
       option).

       Unless --kill-on-exit option is used (or --seccomp-bpf option is
       used in a way that implies --kill-on-exit), traced processes
       which are descended from command may be left running after an
       interrupt signal (CTRL-C).

       By using CLONE_UNTRACED flag of clone system call a tracee can
       break the guarantee that --seccomp-bpf will not leave any
       processes with a seccomp program installed for syscall filtering
       purposes.
HISTORY
       The original strace was written by Paul Kranenburg for SunOS and
       was inspired by its trace utility.  The SunOS version of strace
       was ported to Linux and enhanced by Branko Lankester, who also
       wrote the Linux kernel support.  Even though Paul released strace
       2.5 in 1992, Branko's work was based on Paul's strace 1.5 release
       from 1991.  In 1993, Rick Sladkey merged strace 2.5 for SunOS and
       the second release of strace for Linux, added many of the
       features of truss(1) from SVR4, and produced an strace that
       worked on both platforms.  In 1994 Rick ported strace to SVR4 and
       Solaris and wrote the automatic configuration support.  In 1995
       he ported strace to Irix and became tired of writing about
       himself in the third person.

       Beginning with 1996, strace was maintained by Wichert Akkerman.
       During his tenure, strace development migrated to CVS; ports to
       FreeBSD and many architectures on Linux (including ARM, IA-64,
       MIPS, PA-RISC, PowerPC, s390, SPARC) were introduced.  In 2002,
       the burden of strace maintainership was transferred to Roland
       McGrath.  Since then, strace gained support for several new Linux
       architectures (AMD64, s390x, SuperH), bi-architecture support for
       some of them, and received numerous additions and improvements in
       syscalls decoders on Linux; strace development migrated to Git
       during that period.  Since 2009, strace is actively maintained by
       Dmitry Levin.  strace gained support for AArch64, ARC, AVR32,
       Blackfin, Meta, Nios II, OpenRISC 1000, RISC-V, Tile/TileGx,
       Xtensa architectures since that time.  In 2012, unmaintained and
       apparently broken support for non-Linux operating systems was
       removed.  Also, in 2012 strace gained support for path tracing
       and file descriptor path decoding.  In 2014, support for stack
       trace printing was added.  In 2016, syscall fault injection was
       implemented.

       For the additional information, please refer to the NEWS file and
       strace repository commit log.
REPORTING BUGS
       Problems with strace should be reported to the strace mailing
       list mailto:strace-devel@lists.strace.io.
SEE ALSO
       strace-log-merge(1), ltrace(1), perf-trace(1), trace-cmd(1),
       time(1), ptrace(2), seccomp(2), syscall(2), proc(5), signal(7)

       strace Home Page https://strace.io/
AUTHORS
       The complete list of strace contributors can be found in the
       CREDITS file.
COLOPHON
       This page is part of the strace (system call tracer) project.
       Information about the project can be found at 
       http://strace.io/.  If you have a bug report for this manual
       page, send it to strace-devel@lists.sourceforge.net.  This page
       was obtained from the project's upstream Git repository
       https://github.com/strace/strace.git on 2024-06-14.  (At that
       time, the date of the most recent commit that was found in the
       repository was 2024-06-04.)  If you discover any rendering
       problems in this HTML version of the page, or you believe there
       is a better or more up-to-date source for the page, or you have
       corrections or improvements to the information in this COLOPHON
       (which is not part of the original manual page), send a mail to
       man-pages@man7.org

strace 6.9.0.16.2a4c4          2024-06-04                      STRACE(1)
\end{lstlisting}
}}

\endinput  %  ==  ==  ==  ==  ==  ==  ==  ==  ==
\subsection{\refStrace: Trace System Calls and Signals}

{\tiny{
\begin{lstlisting}[language=bash]
NAME
       strace - trace system calls and signals
SYNOPSIS
       strace [-ACdffhikkqqrtttTvVwxxyyYzZ] [-a column] [-b execve]
              [-e expr]... [-I n] [-o file] [-O overhead] [-p pid]...
              [-P path]... [-s strsize] [-S sortby] [-U columns]
              [-X format] [--seccomp-bpf]
              [--stack-trace-frame-limit=limit] [--syscall-limit=limit]
              [--secontext[=format]] [--tips[=format]] { -p pid | [-DDD]
              [-E var[=val]]... [-u username] command [args] }

       strace -c [-dfwzZ] [-b execve] [-e expr]... [-I n] [-O overhead]
              [-p pid]... [-P path]... [-S sortby] [-U columns]
              [--seccomp-bpf] [--syscall-limit=limit] [--tips[=format]]
              { -p pid | [-DDD] [-E var[=val]]... [-u username] command
              [args] }

       strace --tips[=format]
DESCRIPTION
       In the simplest case strace runs the specified command until it
       exits.  It intercepts and records the system calls which are
       called by a process and the signals which are received by a
       process.  The name of each system call, its arguments and its
       return value are printed on standard error or to the file
       specified with the -o option.

       strace is a useful diagnostic, instructional, and debugging tool.
       System administrators, diagnosticians and trouble-shooters will
       find it invaluable for solving problems with programs for which
       the source is not readily available since they do not need to be
       recompiled in order to trace them.  Students, hackers and the
       overly-curious will find that a great deal can be learned about a
       system and its system calls by tracing even ordinary programs.
       And programmers will find that since system calls and signals are
       events that happen at the user/kernel interface, a close
       examination of this boundary is very useful for bug isolation,
       sanity checking and attempting to capture race conditions.

       Each line in the trace contains the system call name, followed by
       its arguments in parentheses and its return value.  An example
       from stracing the command "cat /dev/null" is:

           open("/dev/null", O_RDONLY) = 3

       Errors (typically a return value of -1) have the errno symbol and
       error string appended.

           open("/foo/bar", O_RDONLY) = -1 ENOENT (No such file or directory)

       Signals are printed as signal symbol and decoded siginfo
       structure.  An excerpt from stracing and interrupting the command
       "sleep 666" is:

           sigsuspend([] <unfinished ...>
           --- SIGINT {si_signo=SIGINT, si_code=SI_USER, si_pid=...} ---
           +++ killed by SIGINT +++

       If a system call is being executed and meanwhile another one is
       being called from a different thread/process then strace will try
       to preserve the order of those events and mark the ongoing call
       as being unfinished.  When the call returns it will be marked as
       resumed.

           [pid 28772] select(4, [3], NULL, NULL, NULL <unfinished ...>
           [pid 28779] clock_gettime(CLOCK_REALTIME, {tv_sec=1130322148, tv_nsec=3977000}) = 0
           [pid 28772] <... select resumed> )      = 1 (in [3])

       Interruption of a (restartable) system call by a signal delivery
       is processed differently as kernel terminates the system call and
       also arranges its immediate reexecution after the signal handler
       completes.

           read(0, 0x7ffff72cf5cf, 1)              = ? ERESTARTSYS (To be restarted)
           --- SIGALRM {si_signo=SIGALRM, si_code=SI_KERNEL} ---
           rt_sigreturn({mask=[]})                 = 0
           read(0, "", 1)                          = 0

       Arguments are printed in symbolic form with passion.  This
       example shows the shell performing ">>xyzzy" output redirection:

           open("xyzzy", O_WRONLY|O_APPEND|O_CREAT, 0666) = 3

       Here, the second and the third argument of open(2) are decoded by
       breaking down the flag argument into its three bitwise-OR
       constituents and printing the mode value in octal by tradition.
       Where the traditional or native usage differs from ANSI or POSIX,
       the latter forms are preferred.  In some cases, strace output is
       proven to be more readable than the source.

       Structure pointers are dereferenced and the members are displayed
       as appropriate.  In most cases, arguments are formatted in the
       most C-like fashion possible.  For example, the essence of the
       command "ls -l /dev/null" is captured as:

           lstat("/dev/null", {st_mode=S_IFCHR|0666, st_rdev=makedev(0x1, 0x3), ...}) = 0

       Notice how the 'struct stat' argument is dereferenced and how
       each member is displayed symbolically.  In particular, observe
       how the st_mode member is carefully decoded into a bitwise-OR of
       symbolic and numeric values.  Also notice in this example that
       the first argument to lstat(2) is an input to the system call and
       the second argument is an output.  Since output arguments are not
       modified if the system call fails, arguments may not always be
       dereferenced.  For example, retrying the "ls -l" example with a
       non-existent file produces the following line:

           lstat("/foo/bar", 0xb004) = -1 ENOENT (No such file or directory)

       In this case the porch light is on but nobody is home.

       Syscalls unknown to strace are printed raw, with the unknown
       system call number printed in hexadecimal form and prefixed with
       "syscall_":

           syscall_0xbad(0x1, 0x2, 0x3, 0x4, 0x5, 0x6) = -1 ENOSYS (Function not implemented)

       Character pointers are dereferenced and printed as C strings.
       Non-printing characters in strings are normally represented by
       ordinary C escape codes.  Only the first strsize (32 by default)
       bytes of strings are printed; longer strings have an ellipsis
       appended following the closing quote.  Here is a line from "ls
       -l" where the getpwuid(3) library routine is reading the password
       file:

           read(3, "root::0:0:System Administrator:/"..., 1024) = 422

       While structures are annotated using curly braces, pointers to
       basic types and arrays are printed using square brackets with
       commas separating the elements.  Here is an example from the
       command id(1) on a system with supplementary group ids:

           getgroups(32, [100, 0]) = 2

       On the other hand, bit-sets are also shown using square brackets,
       but set elements are separated only by a space.  Here is the
       shell, preparing to execute an external command:

           sigprocmask(SIG_BLOCK, [CHLD TTOU], []) = 0

       Here, the second argument is a bit-set of two signals, SIGCHLD
       and SIGTTOU.  In some cases, the bit-set is so full that printing
       out the unset elements is more valuable.  In that case, the bit-
       set is prefixed by a tilde like this:

           sigprocmask(SIG_UNBLOCK, ~[], NULL) = 0

       Here, the second argument represents the full set of all signals.
OPTIONS
   General
       -e expr
              A qualifying expression which modifies which events to
              trace or how to trace them.  The format of the expression
              is:

                             [qualifier=][!]value[,value]...

              where qualifier is one of trace (or t), trace-fds (or
              trace-fd or fd or fds), abbrev (or a), verbose (or v), raw
              (or x), signal (or signals or s), read (or reads or r),
              write (or writes or w), fault, inject, status, quiet (or
              silent or silence or q), secontext, decode-fds (or
              decode-fd), decode-pids (or decode-pid), or kvm, and value
              is a qualifier-dependent symbol or number.  The default
              qualifier is trace.  Using an exclamation mark negates the
              set of values.  For example, -e open means literally
              -e trace=open which in turn means trace only the open
              system call.  By contrast, -e trace=!open means to trace
              every system call except open.  In addition, the special
              values all and none have the obvious meanings.

              Note that some shells use the exclamation point for
              history expansion even inside quoted arguments.  If so,
              you must escape the exclamation point with a backslash.

   Startup
       -E var=val
       --env=var=val
              Run command with var=val in its list of environment
              variables.

       -E var
       --env=var
              Remove var from the inherited list of environment
              variables before passing it on to the command.

       -p pid
       --attach=pid
              Attach to the process with the process ID pid and begin
              tracing.  The trace may be terminated at any time by a
              keyboard interrupt signal (CTRL-C).  strace will respond
              by detaching itself from the traced process(es) leaving it
              (them) to continue running.  Multiple -p options can be
              used to attach to many processes in addition to command
              (which is optional if at least one -p option is given).
              Multiple process IDs, separated by either comma (",''),
              space (" "), tab, or newline character, can be provided as
              an argument to a single -p option, so, for example, -p
              "$(pidof PROG)" and -p "$(pgrep PROG)" syntaxes are
              supported.

       -u username
       --user=username
              Run command with the user ID, group ID, and supplementary
              groups of username.  This option is only useful when
              running as root and enables the correct execution of
              setuid and/or setgid binaries.  Unless this option is used
              setuid and setgid programs are executed without effective
              privileges.
       -u UID:GID
       --user=UID:GID
              Alternative syntax where the program is started with
              exactly the given user and group IDs, and an empty list of
              supplementary groups.  In this case, user and group name
              lookups are not performed.

       --argv0=name
              Set argv[0] of the command being executed to name.  Useful
              for tracing multi-call executables which interpret
              argv[0], such as busybox or kmod.

   Tracing
       -b syscall
       --detach-on=syscall
              If specified syscall is reached, detach from traced
              process.  Currently, only execve(2) syscall is supported.
              This option is useful if you want to trace multi-threaded
              process and therefore require -f, but don't want to trace
              its (potentially very complex) children.

       -D
       --daemonize
       --daemonize=grandchild
              Run tracer process as a grandchild, not as the parent of
              the tracee.  This reduces the visible effect of strace by
              keeping the tracee a direct child of the calling process.

       -DD
       --daemonize=pgroup
       --daemonize=pgrp
              Run tracer process as tracee's grandchild in a separate
              process group.  In addition to reduction of the visible
              effect of strace, it also avoids killing of strace with
              kill(2) issued to the whole process group.

       -DDD
       --daemonize=session
              Run tracer process as tracee's grandchild in a separate
              session ("true daemonisation").  In addition to reduction
              of the visible effect of strace, it also avoids killing of
              strace upon session termination.

       -f
       --follow-forks
              Trace child processes as they are created by currently
              traced processes as a result of the fork(2), vfork(2) and
              clone(2) system calls.  Note that -p PID -f will attach
              all threads of process PID if it is multi-threaded, not
              only thread with thread_id = PID.

       --output-separately
              If the --output=filename option is in effect, each
              processes trace is written to filename.pid where pid is
              the numeric process id of each process.

       -ff
       --follow-forks --output-separately
              Combine the effects of --follow-forks and
              --output-separately options.  This is incompatible with
              -c, since no per-process counts are kept.

              One might want to consider using strace-log-merge(1) to
              obtain a combined strace log view.

       -I interruptible
       --interruptible=interruptible
              When strace can be interrupted by signals (such as
              pressing CTRL-C).

              1, anywhere
                     no signals are blocked;
              2, waiting
                     fatal signals are blocked while decoding syscall
                     (default);
              3, never
                     fatal signals are always blocked (default if -o
                     FILE PROG);
              4, never_tstp
                     fatal signals and SIGTSTP (CTRL-Z) are always
                     blocked (useful to make strace -o FILE PROG not
                     stop on CTRL-Z, default if -D).

       --syscall-limit=limit
              Detach all tracees when limit number of syscalls have been
              captured. Syscalls filtered out via --trace, --trace-path
              or --status options are not considered when keeping track
              of the number of syscalls that are captured.

       --kill-on-exit
              Apply PTRACE_O_EXITKILL ptrace option to all tracee
              processes (which sends a SIGKILL signal to the tracee if
              the tracer exits) and do not detach them on cleanup so
              they will not be left running after the tracer exit.
              --kill-on-exit is not compatible with -p/--attach options.

   Filtering
       -e trace=syscall_set
       -e t=syscall_set
       --trace=syscall_set
              Trace only the specified set of system calls.  syscall_set
              is defined as [!]value[,value], and value can be one of
              the following:

              syscall
                     Trace specific syscall, specified by its name (see
                     syscalls(2) for a reference, but also see NOTES).

              ?value Question mark before the syscall qualification
                     allows suppression of error in case no syscalls
                     matched the qualification provided.

              value@64
                     Limit the syscall specification described by value
                     to 64-bit personality.

              value@32
                     Limit the syscall specification described by value
                     to 32-bit personality.

              value@x32
                     Limit the syscall specification described by value
                     to x32 personality.

              all    Trace all system calls.

              /regex Trace only those system calls that match the regex.
                     You can use POSIX Extended Regular Expression
                     syntax (see regex(7)).

              %file
              file   Trace all system calls which take a file name as an
                     argument.  You can think of this as an abbreviation
                     for -e trace=open,stat,chmod,unlink,...  which is
                     useful to seeing what files the process is
                     referencing.  Furthermore, using the abbreviation
                     will ensure that you don't accidentally forget to
                     include a call like lstat(2) in the list.  Betchya
                     woulda forgot that one.  The syntax without a
                     preceding percent sign ("-e trace=file") is
                     deprecated.

              %process
              process
                     Trace system calls associated with process
                     lifecycle (creation, exec, termination).  The
                     syntax without a preceding percent sign ("-e
                     trace=process") is deprecated.

              %net
              %network
              network
                     Trace all the network related system calls.  The
                     syntax without a preceding percent sign ("-e
                     trace=network") is deprecated.

              %signal
              signal Trace all signal related system calls.  The syntax
                     without a preceding percent sign ("-e
                     trace=signal") is deprecated.

              %ipc
              ipc    Trace all IPC related system calls.  The syntax
                     without a preceding percent sign ("-e trace=ipc")
                     is deprecated.

              %desc
              desc   Trace all file descriptor related system calls.
                     The syntax without a preceding percent sign ("-e
                     trace=desc") is deprecated.

              %memory
              memory Trace all memory mapping related system calls.  The
                     syntax without a preceding percent sign ("-e
                     trace=memory") is deprecated.

              %creds Trace system calls that read or modify user and
                     group identifiers or capability sets.

              %stat  Trace stat syscall variants.

              %lstat Trace lstat syscall variants.

              %fstat Trace fstat, fstatat, and statx syscall variants.

              %%stat Trace syscalls used for requesting file status
                     (stat, lstat, fstat, fstatat, statx, and their
                     variants).

              %statfs
                     Trace statfs, statfs64, statvfs, osf_statfs, and
                     osf_statfs64 system calls.  The same effect can be
                     achieved with -e trace=/^(.*_)?statv?fs regular
                     expression.

              %fstatfs
                     Trace fstatfs, fstatfs64, fstatvfs, osf_fstatfs,
                     and osf_fstatfs64 system calls.  The same effect
                     can be achieved with -e trace=/fstatv?fs regular
                     expression.

              %%statfs
                     Trace syscalls related to file system statistics
                     (statfs-like, fstatfs-like, and ustat).  The same
                     effect can be achieved with
                     -e trace=/statv?fs|fsstat|ustat regular expression.

              %clock Trace system calls that read or modify system
                     clocks.

              %pure  Trace syscalls that always succeed and have no
                     arguments.  Currently, this list includes
                     arc_gettls(2), getdtablesize(2), getegid(2),
                     getegid32(2), geteuid(2), geteuid32(2), getgid(2),
                     getgid32(2), getpagesize(2), getpgrp(2), getpid(2),
                     getppid(2), get_thread_area(2) (on architectures
                     other than x86), gettid(2), get_tls(2), getuid(2),
                     getuid32(2), getxgid(2), getxpid(2), getxuid(2),
                     kern_features(2), and metag_get_tls(2) syscalls.

              The -c option is useful for determining which system calls
              might be useful to trace.  For example,
              trace=open,close,read,write means to only trace those four
              system calls.  Be careful when making inferences about the
              user/kernel boundary if only a subset of system calls are
              being monitored.  The default is trace=all.

       -e trace-fd=set
       -e trace-fds=set
       -e fd=set
       -e fds=set
       --trace-fds=set
              Trace only the syscalls that operate on the specified
              subset of (non-negative) file descriptors.  Note that
              usage of this option also filters out all the syscalls
              that do not operate on file descriptors at all.  Applies
              in (inclusive) disjunction with the --trace-path option.

       -e signal=set
       -e signals=set
       -e s=set
       --signal=set
              Trace only the specified subset of signals.  The default
              is signal=all.  For example, signal=!SIGIO (or signal=!io)
              causes SIGIO signals not to be traced.

       -e status=set
       --status=set
              Print only system calls with the specified return status.
              The default is status=all.  When using the status
              qualifier, because strace waits for system calls to return
              before deciding whether they should be printed or not, the
              traditional order of events may not be preserved anymore.
              If two system calls are executed by concurrent threads,
              strace will first print both the entry and exit of the
              first system call to exit, regardless of their respective
              entry time.  The entry and exit of the second system call
              to exit will be printed afterwards.  Here is an example
              when select(2) is called, but a different thread calls
              clock_gettime(2) before select(2) finishes:

                  [pid 28779] 1130322148.939977 clock_gettime(CLOCK_REALTIME, {1130322148, 939977000}) = 0
                  [pid 28772] 1130322148.438139 select(4, [3], NULL, NULL, NULL) = 1 (in [3])

              set can include the following elements:

              successful
                     Trace system calls that returned without an error
                     code.  The -z option has the effect of
                     status=successful.
              failed Trace system calls that returned with an error
                     code.  The -Z option has the effect of
                     status=failed.
              unfinished
                     Trace system calls that did not return.  This might
                     happen, for example, due to an execve call in a
                     neighbour thread.
              unavailable
                     Trace system calls that returned but strace failed
                     to fetch the error status.
              detached
                     Trace system calls for which strace detached before
                     the return.

       -P path
       --trace-path=path
              Trace only system calls accessing path.  Multiple -P
              options can be used to specify several paths.  Applies in
              (inclusive) disjunction with the --trace-fds option.

       -z
       --successful-only
              Print only syscalls that returned without an error code.

       -Z
       --failed-only
              Print only syscalls that returned with an error code.

   Output format
       -a column
       --columns=column
              Align return values in a specific column (default column
              40).

       -e abbrev=syscall_set
       -e a=syscall_set
       --abbrev=syscall_set
              Abbreviate the output from printing each member of large
              structures.  The syntax of the syscall_set specification
              is the same as in the -e trace option.  The default is
              abbrev=all.  The -v option has the effect of abbrev=none.

       -e verbose=syscall_set
       -e v=syscall_set
       --verbose=syscall_set
              Dereference structures for the specified set of system
              calls.  The syntax of the syscall_set specification is the
              same as in the -e trace option.  The default is
              verbose=all.

       -e raw=syscall_set
       -e x=syscall_set
       --raw=syscall_set
              Print raw, undecoded arguments for the specified set of
              system calls.  The syntax of the syscall_set specification
              is the same as in the -e trace option.  This option has
              the effect of causing all arguments to be printed in
              hexadecimal.  This is mostly useful if you don't trust the
              decoding or you need to know the actual numeric value of
              an argument.  See also -X raw option.

       -e read=set
       -e reads=set
       -e r=set
       --read=set
              Perform a full hexadecimal and ASCII dump of all the data
              read from file descriptors listed in the specified set.
              For example, to see all input activity on file descriptors
              3 and 5 use -e read=3,5.  Note that this is independent
              from the normal tracing of the read(2) system call which
              is controlled by the option -e trace=read.

       -e write=set
       -e writes=set
       -e w=set
       --write=set
              Perform a full hexadecimal and ASCII dump of all the data
              written to file descriptors listed in the specified set.
              For example, to see all output activity on file
              descriptors 3 and 5 use -e write=3,5.  Note that this is
              independent from the normal tracing of the write(2) system
              call which is controlled by the option -e trace=write.

       -e quiet=set
       -e silent=set
       -e silence=set
       -e q=set
       --quiet=set
       --silent=set
       --silence=set
              Suppress various information messages.  The default is
              quiet=none.  set can include the following elements:

              attach Suppress messages about attaching and detaching ("[
                     Process NNNN attached ]", "[ Process NNNN detached
                     ]").
              exit   Suppress messages about process exits ("+++ exited
                     with SSS +++").
              path-resolution
                     Suppress messages about resolution of paths
                     provided via the -P option ("Requested path "..."
                     resolved into "..."").
              personality
                     Suppress messages about process personality changes
                     ("[ Process PID=NNNN runs in PPP mode. ]").
              thread-execve
              superseded
                     Suppress messages about process being superseded by
                     execve(2) in another thread ("+++ superseded by
                     execve in pid NNNN +++").

       -e decode-fds=set
       --decode-fds=set
              Decode various information associated with file
              descriptors.  The default is decode-fds=none.  set can
              include the following elements:

              path     Print file paths.  Also enables printing of
                       tracee's current working directory when AT_FDCWD
                       constant is used.
              socket   Print socket protocol-specific information,
              dev      Print character/block device numbers.
              pidfd    Print PIDs associated with pidfd file
                       descriptors.
              signalfd Print signal masks associated with signalfd file
                       descriptors.

       -e decode-pids=set
       --decode-pids=set
              Decode various information associated with process IDs
              (and also thread IDs, process group IDs, and session IDs).
              The default is decode-pids=none.  set can include the
              following elements:

              comm    Print command names associated with thread or
                      process IDs.
              pidns   Print thread, process, process group, and session
                      IDs in strace's PID namespace if the tracee is in
                      a different PID namespace.

       -e kvm=vcpu
       --kvm=vcpu
              Print the exit reason of kvm vcpu.  Requires Linux kernel
              version 4.16.0 or higher.

       -i
       --instruction-pointer
              Print the instruction pointer at the time of the system
              call.

       -n
       --syscall-number
              Print the syscall number.

       -k
       --stack-trace[=symbol]
              Print the execution stack trace of the traced processes
              after each system call.

       -kk
       --stack-trace=source
              Print the execution stack trace and source code
              information of the traced processes after each system
              call. This option expects the target program is compiled
              with appropriate debug options: "-g" (gcc), or "-g
              -gdwarf-aranges" (clang).

       --stack-trace-frame-limit=limit
              Print no more than this amount of stack trace frames when
              backtracing a system call (the default is 256).  Use this
              option with the --stack-trace (or -k) option.

       -o filename
       --output=filename
              Write the trace output to the file filename rather than to
              stderr.  filename.pid form is used if -ff option is
              supplied.  If the argument begins with '|' or '!', the
              rest of the argument is treated as a command and all
              output is piped to it.  This is convenient for piping the
              debugging output to a program without affecting the
              redirections of executed programs.  The latter is not
              compatible with -ff option currently.

       -A
       --output-append-mode
              Open the file provided in the -o option in append mode.

       -q
       --quiet
       --quiet=attach,personality
              Suppress messages about attaching, detaching, and
              personality changes.  This happens automatically when
              output is redirected to a file and the command is run
              directly instead of attaching.

       -qq
       --quiet=attach,personality,exit
              Suppress messages attaching, detaching, personality
              changes, and about process exit status.

       -qqq
       --quiet=all
              Suppress all suppressible messages (please refer to the -e
              quiet option description for the full list of suppressible
              messages).

       -r
       --relative-timestamps[=precision]
              Print a relative timestamp upon entry to each system call.
              This records the time difference between the beginning of
              successive system calls.  precision can be one of s (for
              seconds), ms (milliseconds), us (microseconds), or ns
              (nanoseconds), and allows setting the precision of time
              value being printed.  Default is us (microseconds).  Note
              that since -r option uses the monotonic clock time for
              measuring time difference and not the wall clock time, its
              measurements can differ from the difference in time
              reported by the -t option.

       -s strsize
       --string-limit=strsize
              Specify the maximum string size to print (the default is
              32).  Note that filenames are not considered strings and
              are always printed in full.

       --absolute-timestamps[=[[format:]format],[[precision:]precision]]
       --timestamps[=[[format:]format],[[precision:]precision]]
              Prefix each line of the trace with the wall clock time in
              the specified format with the specified precision.  format
              can be one of the following:

              none   No time stamp is printed.  Can be used to override
                     the previous setting.
              time   Wall clock time (strftime(3) format string is %T).
              unix   Number of seconds since the epoch (strftime(3)
                     format string is %s).

              precision can be one of s (for seconds), ms
              (milliseconds), us (microseconds), or ns (nanoseconds).
              Default arguments for the option are
              format:time,precision:s.

       -t
       --absolute-timestamps
              Prefix each line of the trace with the wall clock time.

       -tt
       --absolute-timestamps=precision:us
              If given twice, the time printed will include the
              microseconds.

       -ttt
       --absolute-timestamps=format:unix,precision:us
              If given thrice, the time printed will include the
              microseconds and the leading portion will be printed as
              the number of seconds since the epoch.

       -T
       --syscall-times[=precision]
              Show the time spent in system calls.  This records the
              time difference between the beginning and the end of each
              system call.  precision can be one of s (for seconds), ms
              (milliseconds), us (microseconds), or ns (nanoseconds),
              and allows setting the precision of time value being
              printed.  Default is us (microseconds).

       -v
       --no-abbrev
              Print unabbreviated versions of environment, stat,
              termios, etc.  calls.  These structures are very common in
              calls and so the default behavior displays a reasonable
              subset of structure members.  Use this option to get all
              of the gory details.

       --strings-in-hex[=option]
              Control usage of escape sequences with hexadecimal numbers
              in the printed strings.  Normally (when no
              --strings-in-hex or -x option is supplied), escape
              sequences are used to print non-printable and non-ASCII
              characters (that is, characters with a character code less
              than 32 or greater than 127), or to disambiguate the
              output (so, for quotes and other characters that encase
              the printed string, for example, angle brackets, in case
              of file descriptor path output); for the former use case,
              unless it is a white space character that has a symbolic
              escape sequence defined in the C standard (that is, "\textbackslash t"
              for a horizontal tab, "\textbackslash n" for a newline, "\textbackslash v" for a
              vertical tab, "\textbackslash f" for a form feed page break, and "\textbackslash r"
              for a carriage return) are printed using escape sequences
              with numbers that correspond to their byte values, with
              octal number format being the default.  option can be one
              of the following:

              none   Hexadecimal numbers are not used in the output at
                     all.  When there is a need to emit an escape
                     sequence, octal numbers are used.
              non-ascii-chars
                     Hexadecimal numbers are used instead of octal in
                     the escape sequences.
              non-ascii
                     Strings that contain non-ASCII characters are
                     printed using escape sequences with hexadecimal
                     numbers.
              all    All strings are printed using escape sequences with
                     hexadecimal numbers.

              When the option is supplied without an argument, all is
              assumed.

       -x
       --strings-in-hex=non-ascii
              Print all non-ASCII strings in hexadecimal string format.

       -xx
       --strings-in-hex[=all]
              Print all strings in hexadecimal string format.

       -X format
       --const-print-style=format
              Set the format for printing of named constants and flags.
              Supported format values are:

              raw    Raw number output, without decoding.
              abbrev Output a named constant or a set of flags instead
                     of the raw number if they are found.  This is the
                     default strace behaviour.
              verbose
                     Output both the raw value and the decoded string
                     (as a comment).

       -y
       --decode-fds
       --decode-fds=path
              Print paths associated with file descriptor arguments and
              with the AT_FDCWD constant.

       -yy
       --decode-fds=all
              Print all available information associated with file
              descriptors: protocol-specific information associated with
              socket file descriptors, block/character device number
              associated with device file descriptors, and PIDs
              associated with pidfd file descriptors.

       --pidns-translation
       --decode-pids=pidns
              If strace and tracee are in different PID namespaces,
              print PIDs in strace's namespace, too.

       -Y
       --decode-pids=comm
              Print command names for PIDs.

       --secontext[=format]
       -e secontext=format
              When SELinux is available and is not disabled, print in
              square brackets SELinux contexts of processes, files, and
              descriptors.  The format argument is a comma-separated
              list of items being one of the following:

              full              Print the full context (user, role, type
                                level and category).
              mismatch          Also print the context recorded by the
                                SELinux database in case the current
                                context differs.  The latter is printed
                                after two exclamation marks (!!).

              The default value for --secontext is !full,mismatch which
              prints only the type instead of full context and doesn't
              check for context mismatches.

       --always-show-pid
              Show PID prefix also for the process started by strace.
              Implied when -f and -o are both specified.

   Statistics
       -c
       --summary-only
              Count time, calls, and errors for each system call and
              report a summary on program exit, suppressing the regular
              output.  This attempts to show system time (CPU time spent
              running in the kernel) independent of wall clock time.  If
              -c is used with -f, only aggregate totals for all traced
              processes are kept.

       -C
       --summary
              Like -c but also print regular output while processes are
              running.

       -O overhead
       --summary-syscall-overhead=overhead
              Set the overhead for tracing system calls to overhead.
              This is useful for overriding the default heuristic for
              guessing how much time is spent in mere measuring when
              timing system calls using the -c option.  The accuracy of
              the heuristic can be gauged by timing a given program run
              without tracing (using time(1)) and comparing the
              accumulated system call time to the total produced using
              -c.

              The format of overhead specification is described in
              section Time specification format description.

       -S sortby
       --summary-sort-by=sortby
              Sort the output of the histogram printed by the -c option
              by the specified criterion.  Legal values are time (or
              time-percent or time-total or total-time), min-time (or
              shortest or time-min), max-time (or longest or time-max),
              avg-time (or time-avg), calls (or count), errors (or
              error), name (or syscall or syscall-name), and nothing (or
              none); default is time.

       -U columns
       --summary-columns=columns
              Configure a set (and order) of columns being shown in the
              call summary.  The columns argument is a comma-separated
              list with items being one of the following:

              time-percent (or time)
                     Percentage of cumulative time consumed by a
                     specific system call.
              total-time (or time-total)
                     Total system (or wall clock, if -w option is
                     provided) time consumed by a specific system call.
              min-time (or shortest or time-min)
                     Minimum observed call duration.
              max-time (or longest or time-max)
                     Maximum observed call duration.
              avg-time (or time-avg)
                     Average call duration.
              calls (or count)
                     Call count.
              errors (or error)
                     Error count.
              name (or syscall or syscall-name)
                     Syscall name.

              The default value is
              time-percent,total-time,avg-time,calls,errors,name.  If
              the name field is not supplied explicitly, it is added as
              the last column.

       -w
       --summary-wall-clock
              Summarise the time difference between the beginning and
              end of each system call.  The default is to summarise the
              system time.

   Tampering
       -e inject=syscall_set[:error=errno|:retval=value][:signal=sig]
       [:syscall=syscall][:delay_enter=delay][:delay_exit=delay]
       [:poke_enter=@argN=DATAN,@argM=DATAM...]
       [:poke_exit=@argN=DATAN,@argM=DATAM...][:when=expr]
       --inject=syscall_set[:error=errno|:retval=value][:signal=sig]
       [:syscall=syscall][:delay_enter=delay][:delay_exit=delay]
       [:poke_enter=@argN=DATAN,@argM=DATAM...]
       [:poke_exit=@argN=DATAN,@argM=DATAM...][:when=expr]
              Perform   syscall  tampering  for  the  specified  set  of
              syscalls.  The syntax of the syscall_set specification  is
              the same as in the -e trace option.

              At  least  one  of  error,  retval,  signal,  delay_enter,
              delay_exit, poke_enter, or poke_exit  options  has  to  be
              specified.  error and retval are mutually exclusive.

              If  :error=errno  option is specified, a fault is injected
              into a syscall invocation: the syscall number is  replaced
              by  -1  which  corresponds to an invalid syscall (unless a
              syscall is specified with :syscall= option), and the error
              code is specified using a symbolic errno value like ENOSYS
              or a numeric value within 1..4095 range.

              If :retval=value option is specified, success injection is
              performed: the syscall number is replaced  by  -1,  but  a
              bogus success value is returned to the callee.

              If  :signal=sig option is specified with either a symbolic
              value like SIGSEGV or a numeric value  within  1..SIGRTMAX
              range,  that signal is delivered on entering every syscall
              specified by the set.

              If :delay_enter=delay  or  :delay_exit=delay  options  are
              specified,  delay  injection  is  performed: the tracee is
              delayed by time period specified by delay on  entering  or
              exiting  the  syscall,  respectively.  The format of delay
              specification is described in section  Time  specification
              format description.

              If        :poke_enter=@argN=DATAN,@argM=DATAM...        or
              :poke_exit=@argN=DATAN,@argM=DATAM...     options      are
              specified,  tracee's  memory  at  locations, pointed to by
              system call arguments argN and argM (going  from  arg1  to
              arg7) is overwritten by data DATAN and DATAM (specified in
              hexadecimal          format;          for          example
              :poke_enter=@arg1=0000DEAD0000BEEF).  :poke_enter modifies
              memory on syscall enter, and :poke_exit - on exit.

              If :signal=sig option is specified  without  :error=errno,
              :retval=value  or  :delay_{enter,exit}=usecs options, then
              only a signal sig is delivered without a syscall fault  or
              delay     injection.     Conversely,    :error=errno    or
              :retval=value    option    without     :delay_enter=delay,
              :delay_exit=delay  or  :signal=sig options injects a fault
              without delivering a signal or injecting a delay, etc.

              If  :signal=sig  option   is   specified   together   with
              :error=errno  or  :retval=value,  then both injection of a
              fault or success and signal delivery are performed.

              if :syscall=syscall option is specified, the corresponding
              syscall with no side effects is injected  instead  of  -1.
              Currently,  only  "pure"  (see -e trace=%pure description)
              syscalls can be specified there.

              Unless  a  :when=expr  subexpression  is   specified,   an
              injection  is  being  made  into  every invocation of each
              syscall from the set.

              The format of the subexpression is:

                             first[..last][+[step]]

              Number first stands for the first invocation number in the
              range, number last stands for the last  invocation  number
              in  the  range,  and  step stands for the step between two
              consecutive invocations.  The following  combinations  are
              useful:

              first  For every syscall from the set, perform an
                     injection for the syscall invocation number first
                     only.
              first..last
                     For every syscall from the set, perform an
                     injection for the syscall invocation number first
                     and all subsequent invocations until the invocation
                     number last (inclusive).
              first+ For every syscall from the set, perform injections
                     for the syscall invocation number first and all
                     subsequent invocations.
              first..last+
                     For every syscall from the set, perform injections
                     for the syscall invocation number first and all
                     subsequent invocations until the invocation number
                     last (inclusive).
              first+step
                     For every syscall from the set, perform injections
                     for syscall invocations number first, first+step,
                     first+step+step, and so on.
              first..last+step
                     Same as the previous, but consider only syscall
                     invocations with numbers up to last (inclusive).

              For example, to fail each third and subsequent chdir
              syscalls with ENOENT, use
              -e inject=chdir:error=ENOENT:when=3+.

              The valid range for numbers first and step is 1..65535,
              and for number last is 1..65534.

              An injection expression can contain only one error= or
              retval= specification, and only one signal= specification.
              If an injection expression contains multiple when=
              specifications, the last one takes precedence.

              Accounting of syscalls that are subject to injection is
              done per syscall and per tracee.

              Specification of syscall injection can be combined with
              other syscall filtering options, for example, -P
              /dev/urandom -e inject=file:error=ENOENT.

       -e fault=syscall_set[:error=errno][:when=expr]
       --fault=syscall_set[:error=errno][:when=expr]
              Perform syscall fault injection for the specified set of
              syscalls.

              This is equivalent to more generic -e inject= expression
              with default value of errno option set to ENOSYS.

   Miscellaneous
       -d
       --debug
              Show some debugging output of strace itself on the
              standard error.

       -F     This option is deprecated.  It is retained for backward
              compatibility only and may be removed in future releases.
              Usage of multiple instances of -F option is still
              equivalent to a single -f, and it is ignored at all if
              used along with one or more instances of -f option.

       -h
       --help Print the help summary.

       --seccomp-bpf
              Try to enable use of seccomp-bpf (see seccomp(2)) to have
              ptrace(2)-stops only when system calls that are being
              traced occur in the traced processes.

              This option has no effect unless -f/--follow-forks is also
              specified.  --seccomp-bpf is not compatible with
              --syscall-limit and -b/--detach-on options.  It is also
              not applicable to processes attached using -p/--attach
              option.

              An attempt to enable system calls filtering using seccomp-
              bpf may fail for various reasons, e.g. there are too many
              system calls to filter, the seccomp API is not available,
              or strace itself is being traced.  In cases when seccomp-
              bpf filter setup failed, strace proceeds as usual and
              stops traced processes on every system call.

              When --seccomp-bpf is activated and -p/--attach option is
              not used, --kill-on-exit option is activated as well.

              Note that in cases when the tracee has another seccomp
              filter that returns an action value with a precedence
              greater than SECCOMP_RET_TRACE, strace --seccomp-bpf will
              not be notified.  That is, if another seccomp filter, for
              example, disables the syscall or kills the tracee, then
              strace --seccomp-bpf will not be aware of that syscall
              invocation at all.

       --tips[=[[id:]id],[[format:]format]]
              Show strace tips, tricks, and tweaks before exit.  id can
              be a non-negative integer number, which enables printing
              of specific tip, trick, or tweak (these ID are not
              guaranteed to be stable), or random (the default), in
              which case a random tip is printed.  format can be one of
              the following:

              none     No tip is printed.  Can be used to override the
                       previous setting.
              compact  Print the tip just big enough to contain all the
                       text.
              full     Print the tip in its full glory.

              Default is id:random,format:compact.

       -V
       --version
              Print the version number of strace.  Multiple instances of
              the option beyond specific threshold tend to increase
              Strauss awareness.

   Time specification format description
       Time values can be specified as a decimal floating point number
       (in a format accepted by strtod(3)), optionally followed by one
       of the following suffices that specify the unit of time: s
       (seconds), ms (milliseconds), us (microseconds), or ns
       (nanoseconds).  If no suffix is specified, the value is
       interpreted as microseconds.

       The described format is used for -O, -e inject=delay_enter, and
       -e inject=delay_exit options.
DIAGNOSTICS
       When command exits, strace exits with the same exit status.  If
       command is terminated by a signal, strace terminates itself with
       the same signal, so that strace can be used as a wrapper process
       transparent to the invoking parent process.  Note that parent-
       child relationship (signal stop notifications, getppid(2) value,
       etc) between traced process and its parent are not preserved
       unless -D is used.

       When using -p without a command, the exit status of strace is
       zero unless no processes has been attached or there was an
       unexpected error in doing the tracing.
SETUID INSTALLATION
       If strace is installed setuid to root then the invoking user will
       be able to attach to and trace processes owned by any user.  In
       addition setuid and setgid programs will be executed and traced
       with the correct effective privileges.  Since only users trusted
       with full root privileges should be allowed to do these things,
       it only makes sense to install strace as setuid to root when the
       users who can execute it are restricted to those users who have
       this trust.  For example, it makes sense to install a special
       version of strace with mode 'rwsr-xr--', user root and group
       trace, where members of the trace group are trusted users.  If
       you do use this feature, please remember to install a regular
       non-setuid version of strace for ordinary users to use.
MULTIPLE PERSONALITIES SUPPORT
       On some architectures, strace supports decoding of syscalls for
       processes that use different ABI rather than the one strace uses.
       Specifically, in addition to decoding native ABI, strace can
       decode the following ABIs on the following architectures:

       [1]  When strace is built as an x86_64 application
       [2]  When strace is built as an x32 application
       [3]  Big endian only

       This support is optional and relies on ability to generate and
       parse structure definitions during the build time.  Please refer
       to the output of the strace -V command in order to figure out
       what support is available in your strace build ("non-native"
       refers to an ABI that differs from the ABI strace has):

       m32-mpers
              strace can trace and properly decode non-native 32-bit
              binaries.
       no-m32-mpers
              strace can trace, but cannot properly decode non-native
              32-bit binaries.
       mx32-mpers
              strace can trace and properly decode non-native
              32-on-64-bit binaries.
       no-mx32-mpers
              strace can trace, but cannot properly decode non-native
              32-on-64-bit binaries.

       If the output contains neither m32-mpers nor no-m32-mpers, then
       decoding of non-native 32-bit binaries is not implemented at all
       or not applicable.

       Likewise, if the output contains neither mx32-mpers nor no-
       mx32-mpers, then decoding of non-native 32-on-64-bit binaries is
       not implemented at all or not applicable.
NOTES
       It is a pity that so much tracing clutter is produced by systems
       employing shared libraries.

       It is instructive to think about system call inputs and outputs
       as data-flow across the user/kernel boundary.  Because user-space
       and kernel-space are separate and address-protected, it is
       sometimes possible to make deductive inferences about process
       behavior using inputs and outputs as propositions.

       In some cases, a system call will differ from the documented
       behavior or have a different name.  For example, the faccessat(2)
       system call does not have flags argument, and the setrlimit(2)
       library function uses prlimit64(2) system call on modern
       (2.6.38+) kernels.  These discrepancies are normal but
       idiosyncratic characteristics of the system call interface and
       are accounted for by C library wrapper functions.

       Some system calls have different names in different architectures
       and personalities.  In these cases, system call filtering and
       printing uses the names that match corresponding __NR_* kernel
       macros of the tracee's architecture and personality.  There are
       two exceptions from this general rule: arm_fadvise64_64(2) ARM
       syscall and xtensa_fadvise64_64(2) Xtensa syscall are filtered
       and printed as fadvise64_64(2).

       On x32, syscalls that are intended to be used by 64-bit processes
       and not x32 ones (for example, readv(2), that has syscall number
       19 on x86_64, with its x32 counterpart has syscall number 515),
       but called with __X32_SYSCALL_BIT flag being set, are designated
       with #64 suffix.

       On some platforms a process that is attached to with the -p
       option may observe a spurious EINTR return from the current
       system call that is not restartable.  (Ideally, all system calls
       should be restarted on strace attach, making the attach invisible
       to the traced process, but a few system calls aren't.  Arguably,
       every instance of such behavior is a kernel bug.)  This may have
       an unpredictable effect on the process if the process takes no
       action to restart the system call.

       As strace executes the specified command directly and does not
       employ a shell for that, scripts without shebang that usually run
       just fine when invoked by shell fail to execute with ENOEXEC
       error.  It is advisable to manually supply a shell as a command
       with the script as its argument.
BUGS
       Programs that use the setuid bit do not have effective user ID
       privileges while being traced.

       A traced process runs slowly (but check out the --seccomp-bpf
       option).

       Unless --kill-on-exit option is used (or --seccomp-bpf option is
       used in a way that implies --kill-on-exit), traced processes
       which are descended from command may be left running after an
       interrupt signal (CTRL-C).

       By using CLONE_UNTRACED flag of clone system call a tracee can
       break the guarantee that --seccomp-bpf will not leave any
       processes with a seccomp program installed for syscall filtering
       purposes.
HISTORY
       The original strace was written by Paul Kranenburg for SunOS and
       was inspired by its trace utility.  The SunOS version of strace
       was ported to Linux and enhanced by Branko Lankester, who also
       wrote the Linux kernel support.  Even though Paul released strace
       2.5 in 1992, Branko's work was based on Paul's strace 1.5 release
       from 1991.  In 1993, Rick Sladkey merged strace 2.5 for SunOS and
       the second release of strace for Linux, added many of the
       features of truss(1) from SVR4, and produced an strace that
       worked on both platforms.  In 1994 Rick ported strace to SVR4 and
       Solaris and wrote the automatic configuration support.  In 1995
       he ported strace to Irix and became tired of writing about
       himself in the third person.

       Beginning with 1996, strace was maintained by Wichert Akkerman.
       During his tenure, strace development migrated to CVS; ports to
       FreeBSD and many architectures on Linux (including ARM, IA-64,
       MIPS, PA-RISC, PowerPC, s390, SPARC) were introduced.  In 2002,
       the burden of strace maintainership was transferred to Roland
       McGrath.  Since then, strace gained support for several new Linux
       architectures (AMD64, s390x, SuperH), bi-architecture support for
       some of them, and received numerous additions and improvements in
       syscalls decoders on Linux; strace development migrated to Git
       during that period.  Since 2009, strace is actively maintained by
       Dmitry Levin.  strace gained support for AArch64, ARC, AVR32,
       Blackfin, Meta, Nios II, OpenRISC 1000, RISC-V, Tile/TileGx,
       Xtensa architectures since that time.  In 2012, unmaintained and
       apparently broken support for non-Linux operating systems was
       removed.  Also, in 2012 strace gained support for path tracing
       and file descriptor path decoding.  In 2014, support for stack
       trace printing was added.  In 2016, syscall fault injection was
       implemented.

       For the additional information, please refer to the NEWS file and
       strace repository commit log.
REPORTING BUGS
       Problems with strace should be reported to the strace mailing
       list mailto:strace-devel@lists.strace.io.
SEE ALSO
       strace-log-merge(1), ltrace(1), perf-trace(1), trace-cmd(1),
       time(1), ptrace(2), seccomp(2), syscall(2), proc(5), signal(7)

       strace Home Page https://strace.io/
AUTHORS
       The complete list of strace contributors can be found in the
       CREDITS file.
COLOPHON
       This page is part of the strace (system call tracer) project.
       Information about the project can be found at 
       http://strace.io/.  If you have a bug report for this manual
       page, send it to strace-devel@lists.sourceforge.net.  This page
       was obtained from the project's upstream Git repository
       https://github.com/strace/strace.git on 2024-06-14.  (At that
       time, the date of the most recent commit that was found in the
       repository was 2024-06-04.)  If you discover any rendering
       problems in this HTML version of the page, or you believe there
       is a better or more up-to-date source for the page, or you have
       corrections or improvements to the information in this COLOPHON
       (which is not part of the original manual page), send a mail to
       man-pages@man7.org

strace 6.9.0.16.2a4c4          2024-06-04                      STRACE(1)
\end{lstlisting}
}}

\endinput  %  ==  ==  ==  ==  ==  ==  ==  ==  ==