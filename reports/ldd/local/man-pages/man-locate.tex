% % % % \input{./components/man/man-locate}
\subsection{\refLocate: List File in Databases}

{\tiny{
\begin{lstlisting}[language=bash]
NAME
       locate - list files in databases that match a pattern
SYNOPSIS
       locate [-d path | --database=path] [-e | -E | --[non-]existing]
       [-i | --ignore-case] [-0 | --null] [-c | --count] [-w |
       --wholename] [-b | --basename] [-l N | --limit=N] [-S |
       --statistics] [-r | --regex ] [--regextype R] [--max-database-age
       D] [-P | -H | --nofollow] [-L | --follow] [--version] [-A |
       --all] [-p | --print] [--help] pattern...
DESCRIPTION
       This manual page documents the GNU version of locate.  For each
       given pattern, locate searches one or more databases of file
       names and displays the file names that contain the pattern.
       Patterns can contain shell-style metacharacters: `*', `?', and
       `[]'.  The metacharacters do not treat `/' or `.'  specially.
       Therefore, a pattern `foo*bar' can match a file name that
       contains `foo3/bar', and a pattern `*duck*' can match a file name
       that contains `lake/.ducky'.  Patterns that contain
       metacharacters should be quoted to protect them from expansion by
       the shell.

       If a pattern is a plain string - it contains no metacharacters -
       locate displays all file names in the database that contain that
       string anywhere.  If a pattern does contain metacharacters,
       locate only displays file names that match the pattern exactly.
       As a result, patterns that contain metacharacters should usually
       begin with a `*', and will most often end with one as well.  The
       exceptions are patterns that are intended to explicitly match the
       beginning or end of a file name.

       The file name databases contain lists of files that were on the
       system when the databases were last updated.  The system
       administrator can choose the file name of the default database,
       the frequency with which the databases are updated, and the
       directories for which they contain entries; see updatedb(1).

       If locate's output is going to a terminal, unusual characters in
       the output are escaped in the same way as for the -print action
       of the find command.  If the output is not going to a terminal,
       file names are printed exactly as-is.
OPTIONS
       -0, --null
              Use ASCII NUL as a separator, instead of newline.

       -A, --all
              Print only names which match all non-option arguments, not
              those matching one or more non-option arguments.

       -b, --basename
              Results are considered to match if the pattern specified
              matches the final component of the name of a file as
              listed in the database.  This final component is usually
              referred to as the `base name'.

       -c, --count
              Instead of printing the matched filenames, just print the
              total number of matches we found, unless --print (-p) is
              also present.

       -d path, --database=path
              Instead of searching the default file name database,
              search the file name databases in path, which is a colon-
              separated list of database file names.  You can also use
              the environment variable LOCATE_PATH to set the list of
              database files to search.  The option overrides the
              environment variable if both are used.  Empty elements in
              the path are taken to be synonyms for the file name of the
              default database.  A database can be supplied on stdin,
              using `-' as an element of path. If more than one element
              of path is `-', later instances are ignored (and a warning
              message is printed).

              The file name database format changed starting with GNU
              find and locate version 4.0 to allow machines with
              different byte orderings to share the databases.  This
              version of locate can automatically recognize and read
              databases produced for older versions of GNU locate or
              Unix versions of locate or find.  Support for the old
              locate database format will be discontinued in a future
              release.

       -e, --existing
              Only print out such names that currently exist (instead of
              such names that existed when the database was created).
              Note that this may slow down the program a lot, if there
              are many matches in the database.  If you are using this
              option within a program, please note that it is possible
              for the file to be deleted after locate has checked that
              it exists, but before you use it.

       -E, --non-existing
              Only print out such names that currently do not exist
              (instead of such names that existed when the database was
              created).  Note that this may slow down the program a lot,
              if there are many matches in the database.

       --help Print a summary of the options to locate and exit.

       -i, --ignore-case
              Ignore case distinctions in both the pattern and the file
              names.

       -l N, --limit=N
              Limit the number of matches to N.  If a limit is set via
              this option, the number of results printed for the -c
              option will never be larger than this number.

       -L, --follow
              If testing for the existence of files (with the -e or -E
              options), consider broken symbolic links to be non-
              existing.   This is the default.

       --max-database-age D
              Normally, locate will issue a warning message when it
              searches a database which is more than 8 days old.  This
              option changes that value to something other than 8.  The
              effect of specifying a negative value is undefined.

       -m, --mmap
              Accepted but does nothing, for compatibility with BSD
              locate.

       -P, -H, --nofollow
              If testing for the existence of files (with the -e or -E
              options), treat broken symbolic links as if they were
              existing files.  The -H form of this option is provided
              purely for similarity with find; the use of -P is
              recommended over -H.

       -p, --print
              Print search results when they normally would not, because
              of the presence of --statistics (-S) or --count (-c).

       -r, --regex
              The pattern specified on the command line is understood to
              be a regular expression, as opposed to a glob pattern.
              The Regular expressions work in the same was as in emacs
              except for the fact that "." will match a newline.  GNU
              find uses the same regular expressions.  Filenames whose
              full paths match the specified regular expression are
              printed (or, in the case of the -c option, counted).  If
              you wish to anchor your regular expression at the ends of
              the full path name, then as is usual with regular
              expressions, you should use the characters ^ and $ to
              signify this.

       --regextype R
              Use regular expression dialect R.  Supported dialects
              include `findutils-default', `posix-awk', `posix-basic',
              `posix-egrep', `posix-extended', `posix-minimal-basic',
              `awk', `ed', `egrep', `emacs', `gnu-awk', `grep' and
              `sed'.  See the Texinfo documentation for a detailed
              explanation of these dialects.

       -s, --stdio
              Accepted but does nothing, for compatibility with BSD
              locate.

       -S, --statistics
              Print various statistics about each locate database and
              then exit without performing a search, unless non-option
              arguments are given.  For compatibility with BSD, -S is
              accepted as a synonym for --statistics.  However, the
              output of locate -S is different for the GNU and BSD
              implementations of locate.

       --version
              Print the version number of locate and exit.

       -w, --wholename
              Match against the whole name of the file as listed in the
              database.  This is the default.
ENVIRONMENT
       LOCATE_PATH
              Colon-separated list of databases to search.  If the value
              has a leading or trailing colon, or has two colons in a
              row, you may get results that vary between different
              versions of locate.
HISTORY
       The locate program started life as the BSD fast find program,
       contributed to BSD by James A. Woods.  This was described by his
       paper Finding Files Fast which was published in Usenix ;login:,
       Vol 8, No 1, February/March, 1983, pp. 8-10.   When the find
       program began to assume a default -print action if no action was
       specified, this changed the interpretation of find pattern.  The
       BSD developers therefore moved the fast find functionality into
       locate.  The GNU implementation of locate appears to be derived
       from the same code.

       Significant changes to locate in reverse order:
       4.3.7     Byte-order independent support for old database format
       4.3.3     locate -i supports multi-byte characters correctly
                 Introduced --max_db_age
       4.3.2     Support for the slocate database format
       4.2.22    Introduced the --all option
       4.2.15    Introduced the --regex option
       4.2.14    Introduced options -L, -P, -H
       4.2.12    Empty items in LOCATE_PATH now indicate the default database
       4.2.11    Introduced the --statistics option
       4.2.4     Introduced --count and --limit
       4.2.0     Glob characters cause matching against the whole file name
       4.0       Introduced the LOCATE02 database format
       3.7       Locate can search multiple databases
BUGS
       The locate database correctly handles filenames containing
       newlines, but only if the system's sort command has a working -z
       option.  If you suspect that locate may need to return filenames
       containing newlines, consider using its --null option.
REPORTING BUGS
       GNU findutils online help:
       <https://www.gnu.org/software/findutils/#get-help>
       Report any translation bugs to
       <https://translationproject.org/team/>

       Report any other issue via the form at the GNU Savannah bug
       tracker:
              <https://savannah.gnu.org/bugs/?group=findutils>
       General topics about the GNU findutils package are discussed at
       the bug-findutils mailing list:
              <https://lists.gnu.org/mailman/listinfo/bug-findutils>
COPYRIGHT
       Copyright (C) 1994-2024 Free Software Foundation, Inc.  License
       GPLv3+: GNU GPL version 3 or later
       <https://gnu.org/licenses/gpl.html>.
       This is free software: you are free to change and redistribute
       it.  There is NO WARRANTY, to the extent permitted by law.
SEE ALSO
       find(1), updatedb(1), xargs(1), glob(3), locatedb(5)

       Full documentation
       <https://www.gnu.org/software/findutils/locate>
       or available locally via: info locate
COLOPHON
       This page is part of the findutils (find utilities) project.
       Information about the project can be found at 
       http://www.gnu.org/software/findutils/.  If you have a bug
       report for this manual page, see
       https://savannah.gnu.org/bugs/?group=findutils.  This page was
       obtained from the project's upstream Git repository
       git://git.savannah.gnu.org/findutils.git on 2024-06-14.  (At
       that time, the date of the most recent commit that was found in
       the repository was 2024-06-03.)  If you discover any rendering
       problems in this HTML version of the page, or you believe there
       is a better or more up-to-date source for the page, or you have
       corrections or improvements to the information in this COLOPHON
       (which is not part of the original manual page), send a mail to
       man-pages@man7.org

                                                               LOCATE(1)
\end{lstlisting}
}}
\endinput  %  ==  ==  ==  ==  ==  ==  ==  ==  ==

\subsection{\refLocate: List File in Databases}

{\tiny{
\begin{lstlisting}[language=bash]
NAME
       locate - list files in databases that match a pattern
SYNOPSIS
       locate [-d path | --database=path] [-e | -E | --[non-]existing]
       [-i | --ignore-case] [-0 | --null] [-c | --count] [-w |
       --wholename] [-b | --basename] [-l N | --limit=N] [-S |
       --statistics] [-r | --regex ] [--regextype R] [--max-database-age
       D] [-P | -H | --nofollow] [-L | --follow] [--version] [-A |
       --all] [-p | --print] [--help] pattern...
DESCRIPTION
       This manual page documents the GNU version of locate.  For each
       given pattern, locate searches one or more databases of file
       names and displays the file names that contain the pattern.
       Patterns can contain shell-style metacharacters: `*', `?', and
       `[]'.  The metacharacters do not treat `/' or `.'  specially.
       Therefore, a pattern `foo*bar' can match a file name that
       contains `foo3/bar', and a pattern `*duck*' can match a file name
       that contains `lake/.ducky'.  Patterns that contain
       metacharacters should be quoted to protect them from expansion by
       the shell.

       If a pattern is a plain string - it contains no metacharacters -
       locate displays all file names in the database that contain that
       string anywhere.  If a pattern does contain metacharacters,
       locate only displays file names that match the pattern exactly.
       As a result, patterns that contain metacharacters should usually
       begin with a `*', and will most often end with one as well.  The
       exceptions are patterns that are intended to explicitly match the
       beginning or end of a file name.

       The file name databases contain lists of files that were on the
       system when the databases were last updated.  The system
       administrator can choose the file name of the default database,
       the frequency with which the databases are updated, and the
       directories for which they contain entries; see updatedb(1).

       If locate's output is going to a terminal, unusual characters in
       the output are escaped in the same way as for the -print action
       of the find command.  If the output is not going to a terminal,
       file names are printed exactly as-is.
OPTIONS
       -0, --null
              Use ASCII NUL as a separator, instead of newline.

       -A, --all
              Print only names which match all non-option arguments, not
              those matching one or more non-option arguments.

       -b, --basename
              Results are considered to match if the pattern specified
              matches the final component of the name of a file as
              listed in the database.  This final component is usually
              referred to as the `base name'.

       -c, --count
              Instead of printing the matched filenames, just print the
              total number of matches we found, unless --print (-p) is
              also present.

       -d path, --database=path
              Instead of searching the default file name database,
              search the file name databases in path, which is a colon-
              separated list of database file names.  You can also use
              the environment variable LOCATE_PATH to set the list of
              database files to search.  The option overrides the
              environment variable if both are used.  Empty elements in
              the path are taken to be synonyms for the file name of the
              default database.  A database can be supplied on stdin,
              using `-' as an element of path. If more than one element
              of path is `-', later instances are ignored (and a warning
              message is printed).

              The file name database format changed starting with GNU
              find and locate version 4.0 to allow machines with
              different byte orderings to share the databases.  This
              version of locate can automatically recognize and read
              databases produced for older versions of GNU locate or
              Unix versions of locate or find.  Support for the old
              locate database format will be discontinued in a future
              release.

       -e, --existing
              Only print out such names that currently exist (instead of
              such names that existed when the database was created).
              Note that this may slow down the program a lot, if there
              are many matches in the database.  If you are using this
              option within a program, please note that it is possible
              for the file to be deleted after locate has checked that
              it exists, but before you use it.

       -E, --non-existing
              Only print out such names that currently do not exist
              (instead of such names that existed when the database was
              created).  Note that this may slow down the program a lot,
              if there are many matches in the database.

       --help Print a summary of the options to locate and exit.

       -i, --ignore-case
              Ignore case distinctions in both the pattern and the file
              names.

       -l N, --limit=N
              Limit the number of matches to N.  If a limit is set via
              this option, the number of results printed for the -c
              option will never be larger than this number.

       -L, --follow
              If testing for the existence of files (with the -e or -E
              options), consider broken symbolic links to be non-
              existing.   This is the default.

       --max-database-age D
              Normally, locate will issue a warning message when it
              searches a database which is more than 8 days old.  This
              option changes that value to something other than 8.  The
              effect of specifying a negative value is undefined.

       -m, --mmap
              Accepted but does nothing, for compatibility with BSD
              locate.

       -P, -H, --nofollow
              If testing for the existence of files (with the -e or -E
              options), treat broken symbolic links as if they were
              existing files.  The -H form of this option is provided
              purely for similarity with find; the use of -P is
              recommended over -H.

       -p, --print
              Print search results when they normally would not, because
              of the presence of --statistics (-S) or --count (-c).

       -r, --regex
              The pattern specified on the command line is understood to
              be a regular expression, as opposed to a glob pattern.
              The Regular expressions work in the same was as in emacs
              except for the fact that "." will match a newline.  GNU
              find uses the same regular expressions.  Filenames whose
              full paths match the specified regular expression are
              printed (or, in the case of the -c option, counted).  If
              you wish to anchor your regular expression at the ends of
              the full path name, then as is usual with regular
              expressions, you should use the characters ^ and $ to
              signify this.

       --regextype R
              Use regular expression dialect R.  Supported dialects
              include `findutils-default', `posix-awk', `posix-basic',
              `posix-egrep', `posix-extended', `posix-minimal-basic',
              `awk', `ed', `egrep', `emacs', `gnu-awk', `grep' and
              `sed'.  See the Texinfo documentation for a detailed
              explanation of these dialects.

       -s, --stdio
              Accepted but does nothing, for compatibility with BSD
              locate.

       -S, --statistics
              Print various statistics about each locate database and
              then exit without performing a search, unless non-option
              arguments are given.  For compatibility with BSD, -S is
              accepted as a synonym for --statistics.  However, the
              output of locate -S is different for the GNU and BSD
              implementations of locate.

       --version
              Print the version number of locate and exit.

       -w, --wholename
              Match against the whole name of the file as listed in the
              database.  This is the default.
ENVIRONMENT
       LOCATE_PATH
              Colon-separated list of databases to search.  If the value
              has a leading or trailing colon, or has two colons in a
              row, you may get results that vary between different
              versions of locate.
HISTORY
       The locate program started life as the BSD fast find program,
       contributed to BSD by James A. Woods.  This was described by his
       paper Finding Files Fast which was published in Usenix ;login:,
       Vol 8, No 1, February/March, 1983, pp. 8-10.   When the find
       program began to assume a default -print action if no action was
       specified, this changed the interpretation of find pattern.  The
       BSD developers therefore moved the fast find functionality into
       locate.  The GNU implementation of locate appears to be derived
       from the same code.

       Significant changes to locate in reverse order:
       4.3.7     Byte-order independent support for old database format
       4.3.3     locate -i supports multi-byte characters correctly
                 Introduced --max_db_age
       4.3.2     Support for the slocate database format
       4.2.22    Introduced the --all option
       4.2.15    Introduced the --regex option
       4.2.14    Introduced options -L, -P, -H
       4.2.12    Empty items in LOCATE_PATH now indicate the default database
       4.2.11    Introduced the --statistics option
       4.2.4     Introduced --count and --limit
       4.2.0     Glob characters cause matching against the whole file name
       4.0       Introduced the LOCATE02 database format
       3.7       Locate can search multiple databases
BUGS
       The locate database correctly handles filenames containing
       newlines, but only if the system's sort command has a working -z
       option.  If you suspect that locate may need to return filenames
       containing newlines, consider using its --null option.
REPORTING BUGS
       GNU findutils online help:
       <https://www.gnu.org/software/findutils/#get-help>
       Report any translation bugs to
       <https://translationproject.org/team/>

       Report any other issue via the form at the GNU Savannah bug
       tracker:
              <https://savannah.gnu.org/bugs/?group=findutils>
       General topics about the GNU findutils package are discussed at
       the bug-findutils mailing list:
              <https://lists.gnu.org/mailman/listinfo/bug-findutils>
COPYRIGHT
       Copyright (C) 1994-2024 Free Software Foundation, Inc.  License
       GPLv3+: GNU GPL version 3 or later
       <https://gnu.org/licenses/gpl.html>.
       This is free software: you are free to change and redistribute
       it.  There is NO WARRANTY, to the extent permitted by law.
SEE ALSO
       find(1), updatedb(1), xargs(1), glob(3), locatedb(5)

       Full documentation
       <https://www.gnu.org/software/findutils/locate>
       or available locally via: info locate
COLOPHON
       This page is part of the findutils (find utilities) project.
       Information about the project can be found at 
       http://www.gnu.org/software/findutils/.  If you have a bug
       report for this manual page, see
       https://savannah.gnu.org/bugs/?group=findutils.  This page was
       obtained from the project's upstream Git repository
       git://git.savannah.gnu.org/findutils.git on 2024-06-14.  (At
       that time, the date of the most recent commit that was found in
       the repository was 2024-06-03.)  If you discover any rendering
       problems in this HTML version of the page, or you believe there
       is a better or more up-to-date source for the page, or you have
       corrections or improvements to the information in this COLOPHON
       (which is not part of the original manual page), send a mail to
       man-pages@man7.org

                                                               LOCATE(1)
\end{lstlisting}
}}
\endinput  %  ==  ==  ==  ==  ==  ==  ==  ==  ==

\subsection{\refLocate: List File in Databases}

{\tiny{
\begin{lstlisting}[language=bash]
NAME
       locate - list files in databases that match a pattern
SYNOPSIS
       locate [-d path | --database=path] [-e | -E | --[non-]existing]
       [-i | --ignore-case] [-0 | --null] [-c | --count] [-w |
       --wholename] [-b | --basename] [-l N | --limit=N] [-S |
       --statistics] [-r | --regex ] [--regextype R] [--max-database-age
       D] [-P | -H | --nofollow] [-L | --follow] [--version] [-A |
       --all] [-p | --print] [--help] pattern...
DESCRIPTION
       This manual page documents the GNU version of locate.  For each
       given pattern, locate searches one or more databases of file
       names and displays the file names that contain the pattern.
       Patterns can contain shell-style metacharacters: `*', `?', and
       `[]'.  The metacharacters do not treat `/' or `.'  specially.
       Therefore, a pattern `foo*bar' can match a file name that
       contains `foo3/bar', and a pattern `*duck*' can match a file name
       that contains `lake/.ducky'.  Patterns that contain
       metacharacters should be quoted to protect them from expansion by
       the shell.

       If a pattern is a plain string - it contains no metacharacters -
       locate displays all file names in the database that contain that
       string anywhere.  If a pattern does contain metacharacters,
       locate only displays file names that match the pattern exactly.
       As a result, patterns that contain metacharacters should usually
       begin with a `*', and will most often end with one as well.  The
       exceptions are patterns that are intended to explicitly match the
       beginning or end of a file name.

       The file name databases contain lists of files that were on the
       system when the databases were last updated.  The system
       administrator can choose the file name of the default database,
       the frequency with which the databases are updated, and the
       directories for which they contain entries; see updatedb(1).

       If locate's output is going to a terminal, unusual characters in
       the output are escaped in the same way as for the -print action
       of the find command.  If the output is not going to a terminal,
       file names are printed exactly as-is.
OPTIONS
       -0, --null
              Use ASCII NUL as a separator, instead of newline.

       -A, --all
              Print only names which match all non-option arguments, not
              those matching one or more non-option arguments.

       -b, --basename
              Results are considered to match if the pattern specified
              matches the final component of the name of a file as
              listed in the database.  This final component is usually
              referred to as the `base name'.

       -c, --count
              Instead of printing the matched filenames, just print the
              total number of matches we found, unless --print (-p) is
              also present.

       -d path, --database=path
              Instead of searching the default file name database,
              search the file name databases in path, which is a colon-
              separated list of database file names.  You can also use
              the environment variable LOCATE_PATH to set the list of
              database files to search.  The option overrides the
              environment variable if both are used.  Empty elements in
              the path are taken to be synonyms for the file name of the
              default database.  A database can be supplied on stdin,
              using `-' as an element of path. If more than one element
              of path is `-', later instances are ignored (and a warning
              message is printed).

              The file name database format changed starting with GNU
              find and locate version 4.0 to allow machines with
              different byte orderings to share the databases.  This
              version of locate can automatically recognize and read
              databases produced for older versions of GNU locate or
              Unix versions of locate or find.  Support for the old
              locate database format will be discontinued in a future
              release.

       -e, --existing
              Only print out such names that currently exist (instead of
              such names that existed when the database was created).
              Note that this may slow down the program a lot, if there
              are many matches in the database.  If you are using this
              option within a program, please note that it is possible
              for the file to be deleted after locate has checked that
              it exists, but before you use it.

       -E, --non-existing
              Only print out such names that currently do not exist
              (instead of such names that existed when the database was
              created).  Note that this may slow down the program a lot,
              if there are many matches in the database.

       --help Print a summary of the options to locate and exit.

       -i, --ignore-case
              Ignore case distinctions in both the pattern and the file
              names.

       -l N, --limit=N
              Limit the number of matches to N.  If a limit is set via
              this option, the number of results printed for the -c
              option will never be larger than this number.

       -L, --follow
              If testing for the existence of files (with the -e or -E
              options), consider broken symbolic links to be non-
              existing.   This is the default.

       --max-database-age D
              Normally, locate will issue a warning message when it
              searches a database which is more than 8 days old.  This
              option changes that value to something other than 8.  The
              effect of specifying a negative value is undefined.

       -m, --mmap
              Accepted but does nothing, for compatibility with BSD
              locate.

       -P, -H, --nofollow
              If testing for the existence of files (with the -e or -E
              options), treat broken symbolic links as if they were
              existing files.  The -H form of this option is provided
              purely for similarity with find; the use of -P is
              recommended over -H.

       -p, --print
              Print search results when they normally would not, because
              of the presence of --statistics (-S) or --count (-c).

       -r, --regex
              The pattern specified on the command line is understood to
              be a regular expression, as opposed to a glob pattern.
              The Regular expressions work in the same was as in emacs
              except for the fact that "." will match a newline.  GNU
              find uses the same regular expressions.  Filenames whose
              full paths match the specified regular expression are
              printed (or, in the case of the -c option, counted).  If
              you wish to anchor your regular expression at the ends of
              the full path name, then as is usual with regular
              expressions, you should use the characters ^ and $ to
              signify this.

       --regextype R
              Use regular expression dialect R.  Supported dialects
              include `findutils-default', `posix-awk', `posix-basic',
              `posix-egrep', `posix-extended', `posix-minimal-basic',
              `awk', `ed', `egrep', `emacs', `gnu-awk', `grep' and
              `sed'.  See the Texinfo documentation for a detailed
              explanation of these dialects.

       -s, --stdio
              Accepted but does nothing, for compatibility with BSD
              locate.

       -S, --statistics
              Print various statistics about each locate database and
              then exit without performing a search, unless non-option
              arguments are given.  For compatibility with BSD, -S is
              accepted as a synonym for --statistics.  However, the
              output of locate -S is different for the GNU and BSD
              implementations of locate.

       --version
              Print the version number of locate and exit.

       -w, --wholename
              Match against the whole name of the file as listed in the
              database.  This is the default.
ENVIRONMENT
       LOCATE_PATH
              Colon-separated list of databases to search.  If the value
              has a leading or trailing colon, or has two colons in a
              row, you may get results that vary between different
              versions of locate.
HISTORY
       The locate program started life as the BSD fast find program,
       contributed to BSD by James A. Woods.  This was described by his
       paper Finding Files Fast which was published in Usenix ;login:,
       Vol 8, No 1, February/March, 1983, pp. 8-10.   When the find
       program began to assume a default -print action if no action was
       specified, this changed the interpretation of find pattern.  The
       BSD developers therefore moved the fast find functionality into
       locate.  The GNU implementation of locate appears to be derived
       from the same code.

       Significant changes to locate in reverse order:
       4.3.7     Byte-order independent support for old database format
       4.3.3     locate -i supports multi-byte characters correctly
                 Introduced --max_db_age
       4.3.2     Support for the slocate database format
       4.2.22    Introduced the --all option
       4.2.15    Introduced the --regex option
       4.2.14    Introduced options -L, -P, -H
       4.2.12    Empty items in LOCATE_PATH now indicate the default database
       4.2.11    Introduced the --statistics option
       4.2.4     Introduced --count and --limit
       4.2.0     Glob characters cause matching against the whole file name
       4.0       Introduced the LOCATE02 database format
       3.7       Locate can search multiple databases
BUGS
       The locate database correctly handles filenames containing
       newlines, but only if the system's sort command has a working -z
       option.  If you suspect that locate may need to return filenames
       containing newlines, consider using its --null option.
REPORTING BUGS
       GNU findutils online help:
       <https://www.gnu.org/software/findutils/#get-help>
       Report any translation bugs to
       <https://translationproject.org/team/>

       Report any other issue via the form at the GNU Savannah bug
       tracker:
              <https://savannah.gnu.org/bugs/?group=findutils>
       General topics about the GNU findutils package are discussed at
       the bug-findutils mailing list:
              <https://lists.gnu.org/mailman/listinfo/bug-findutils>
COPYRIGHT
       Copyright (C) 1994-2024 Free Software Foundation, Inc.  License
       GPLv3+: GNU GPL version 3 or later
       <https://gnu.org/licenses/gpl.html>.
       This is free software: you are free to change and redistribute
       it.  There is NO WARRANTY, to the extent permitted by law.
SEE ALSO
       find(1), updatedb(1), xargs(1), glob(3), locatedb(5)

       Full documentation
       <https://www.gnu.org/software/findutils/locate>
       or available locally via: info locate
COLOPHON
       This page is part of the findutils (find utilities) project.
       Information about the project can be found at 
       http://www.gnu.org/software/findutils/.  If you have a bug
       report for this manual page, see
       https://savannah.gnu.org/bugs/?group=findutils.  This page was
       obtained from the project's upstream Git repository
       git://git.savannah.gnu.org/findutils.git on 2024-06-14.  (At
       that time, the date of the most recent commit that was found in
       the repository was 2024-06-03.)  If you discover any rendering
       problems in this HTML version of the page, or you believe there
       is a better or more up-to-date source for the page, or you have
       corrections or improvements to the information in this COLOPHON
       (which is not part of the original manual page), send a mail to
       man-pages@man7.org

                                                               LOCATE(1)
\end{lstlisting}
}}
\endinput  %  ==  ==  ==  ==  ==  ==  ==  ==  ==

\subsection{\refLocate: List File in Databases}

{\tiny{
\begin{lstlisting}[language=bash]
NAME
       locate - list files in databases that match a pattern
SYNOPSIS
       locate [-d path | --database=path] [-e | -E | --[non-]existing]
       [-i | --ignore-case] [-0 | --null] [-c | --count] [-w |
       --wholename] [-b | --basename] [-l N | --limit=N] [-S |
       --statistics] [-r | --regex ] [--regextype R] [--max-database-age
       D] [-P | -H | --nofollow] [-L | --follow] [--version] [-A |
       --all] [-p | --print] [--help] pattern...
DESCRIPTION
       This manual page documents the GNU version of locate.  For each
       given pattern, locate searches one or more databases of file
       names and displays the file names that contain the pattern.
       Patterns can contain shell-style metacharacters: `*', `?', and
       `[]'.  The metacharacters do not treat `/' or `.'  specially.
       Therefore, a pattern `foo*bar' can match a file name that
       contains `foo3/bar', and a pattern `*duck*' can match a file name
       that contains `lake/.ducky'.  Patterns that contain
       metacharacters should be quoted to protect them from expansion by
       the shell.

       If a pattern is a plain string - it contains no metacharacters -
       locate displays all file names in the database that contain that
       string anywhere.  If a pattern does contain metacharacters,
       locate only displays file names that match the pattern exactly.
       As a result, patterns that contain metacharacters should usually
       begin with a `*', and will most often end with one as well.  The
       exceptions are patterns that are intended to explicitly match the
       beginning or end of a file name.

       The file name databases contain lists of files that were on the
       system when the databases were last updated.  The system
       administrator can choose the file name of the default database,
       the frequency with which the databases are updated, and the
       directories for which they contain entries; see updatedb(1).

       If locate's output is going to a terminal, unusual characters in
       the output are escaped in the same way as for the -print action
       of the find command.  If the output is not going to a terminal,
       file names are printed exactly as-is.
OPTIONS
       -0, --null
              Use ASCII NUL as a separator, instead of newline.

       -A, --all
              Print only names which match all non-option arguments, not
              those matching one or more non-option arguments.

       -b, --basename
              Results are considered to match if the pattern specified
              matches the final component of the name of a file as
              listed in the database.  This final component is usually
              referred to as the `base name'.

       -c, --count
              Instead of printing the matched filenames, just print the
              total number of matches we found, unless --print (-p) is
              also present.

       -d path, --database=path
              Instead of searching the default file name database,
              search the file name databases in path, which is a colon-
              separated list of database file names.  You can also use
              the environment variable LOCATE_PATH to set the list of
              database files to search.  The option overrides the
              environment variable if both are used.  Empty elements in
              the path are taken to be synonyms for the file name of the
              default database.  A database can be supplied on stdin,
              using `-' as an element of path. If more than one element
              of path is `-', later instances are ignored (and a warning
              message is printed).

              The file name database format changed starting with GNU
              find and locate version 4.0 to allow machines with
              different byte orderings to share the databases.  This
              version of locate can automatically recognize and read
              databases produced for older versions of GNU locate or
              Unix versions of locate or find.  Support for the old
              locate database format will be discontinued in a future
              release.

       -e, --existing
              Only print out such names that currently exist (instead of
              such names that existed when the database was created).
              Note that this may slow down the program a lot, if there
              are many matches in the database.  If you are using this
              option within a program, please note that it is possible
              for the file to be deleted after locate has checked that
              it exists, but before you use it.

       -E, --non-existing
              Only print out such names that currently do not exist
              (instead of such names that existed when the database was
              created).  Note that this may slow down the program a lot,
              if there are many matches in the database.

       --help Print a summary of the options to locate and exit.

       -i, --ignore-case
              Ignore case distinctions in both the pattern and the file
              names.

       -l N, --limit=N
              Limit the number of matches to N.  If a limit is set via
              this option, the number of results printed for the -c
              option will never be larger than this number.

       -L, --follow
              If testing for the existence of files (with the -e or -E
              options), consider broken symbolic links to be non-
              existing.   This is the default.

       --max-database-age D
              Normally, locate will issue a warning message when it
              searches a database which is more than 8 days old.  This
              option changes that value to something other than 8.  The
              effect of specifying a negative value is undefined.

       -m, --mmap
              Accepted but does nothing, for compatibility with BSD
              locate.

       -P, -H, --nofollow
              If testing for the existence of files (with the -e or -E
              options), treat broken symbolic links as if they were
              existing files.  The -H form of this option is provided
              purely for similarity with find; the use of -P is
              recommended over -H.

       -p, --print
              Print search results when they normally would not, because
              of the presence of --statistics (-S) or --count (-c).

       -r, --regex
              The pattern specified on the command line is understood to
              be a regular expression, as opposed to a glob pattern.
              The Regular expressions work in the same was as in emacs
              except for the fact that "." will match a newline.  GNU
              find uses the same regular expressions.  Filenames whose
              full paths match the specified regular expression are
              printed (or, in the case of the -c option, counted).  If
              you wish to anchor your regular expression at the ends of
              the full path name, then as is usual with regular
              expressions, you should use the characters ^ and $ to
              signify this.

       --regextype R
              Use regular expression dialect R.  Supported dialects
              include `findutils-default', `posix-awk', `posix-basic',
              `posix-egrep', `posix-extended', `posix-minimal-basic',
              `awk', `ed', `egrep', `emacs', `gnu-awk', `grep' and
              `sed'.  See the Texinfo documentation for a detailed
              explanation of these dialects.

       -s, --stdio
              Accepted but does nothing, for compatibility with BSD
              locate.

       -S, --statistics
              Print various statistics about each locate database and
              then exit without performing a search, unless non-option
              arguments are given.  For compatibility with BSD, -S is
              accepted as a synonym for --statistics.  However, the
              output of locate -S is different for the GNU and BSD
              implementations of locate.

       --version
              Print the version number of locate and exit.

       -w, --wholename
              Match against the whole name of the file as listed in the
              database.  This is the default.
ENVIRONMENT
       LOCATE_PATH
              Colon-separated list of databases to search.  If the value
              has a leading or trailing colon, or has two colons in a
              row, you may get results that vary between different
              versions of locate.
HISTORY
       The locate program started life as the BSD fast find program,
       contributed to BSD by James A. Woods.  This was described by his
       paper Finding Files Fast which was published in Usenix ;login:,
       Vol 8, No 1, February/March, 1983, pp. 8-10.   When the find
       program began to assume a default -print action if no action was
       specified, this changed the interpretation of find pattern.  The
       BSD developers therefore moved the fast find functionality into
       locate.  The GNU implementation of locate appears to be derived
       from the same code.

       Significant changes to locate in reverse order:
       4.3.7     Byte-order independent support for old database format
       4.3.3     locate -i supports multi-byte characters correctly
                 Introduced --max_db_age
       4.3.2     Support for the slocate database format
       4.2.22    Introduced the --all option
       4.2.15    Introduced the --regex option
       4.2.14    Introduced options -L, -P, -H
       4.2.12    Empty items in LOCATE_PATH now indicate the default database
       4.2.11    Introduced the --statistics option
       4.2.4     Introduced --count and --limit
       4.2.0     Glob characters cause matching against the whole file name
       4.0       Introduced the LOCATE02 database format
       3.7       Locate can search multiple databases
BUGS
       The locate database correctly handles filenames containing
       newlines, but only if the system's sort command has a working -z
       option.  If you suspect that locate may need to return filenames
       containing newlines, consider using its --null option.
REPORTING BUGS
       GNU findutils online help:
       <https://www.gnu.org/software/findutils/#get-help>
       Report any translation bugs to
       <https://translationproject.org/team/>

       Report any other issue via the form at the GNU Savannah bug
       tracker:
              <https://savannah.gnu.org/bugs/?group=findutils>
       General topics about the GNU findutils package are discussed at
       the bug-findutils mailing list:
              <https://lists.gnu.org/mailman/listinfo/bug-findutils>
COPYRIGHT
       Copyright (C) 1994-2024 Free Software Foundation, Inc.  License
       GPLv3+: GNU GPL version 3 or later
       <https://gnu.org/licenses/gpl.html>.
       This is free software: you are free to change and redistribute
       it.  There is NO WARRANTY, to the extent permitted by law.
SEE ALSO
       find(1), updatedb(1), xargs(1), glob(3), locatedb(5)

       Full documentation
       <https://www.gnu.org/software/findutils/locate>
       or available locally via: info locate
COLOPHON
       This page is part of the findutils (find utilities) project.
       Information about the project can be found at 
       http://www.gnu.org/software/findutils/.  If you have a bug
       report for this manual page, see
       https://savannah.gnu.org/bugs/?group=findutils.  This page was
       obtained from the project's upstream Git repository
       git://git.savannah.gnu.org/findutils.git on 2024-06-14.  (At
       that time, the date of the most recent commit that was found in
       the repository was 2024-06-03.)  If you discover any rendering
       problems in this HTML version of the page, or you believe there
       is a better or more up-to-date source for the page, or you have
       corrections or improvements to the information in this COLOPHON
       (which is not part of the original manual page), send a mail to
       man-pages@man7.org

                                                               LOCATE(1)
\end{lstlisting}
}}
\endinput  %  ==  ==  ==  ==  ==  ==  ==  ==  ==
