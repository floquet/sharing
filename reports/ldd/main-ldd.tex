\documentclass[10pt, oneside]{article}   	% use "amsart" instead of "article" for AMSLaTeX format
\usepackage{geometry}                		% See geometry.pdf to learn the layout options. There are lots.
\geometry{letterpaper}                   		% ... or a4paper or a5paper or ... 
%\geometry{landscape}                		% Activate for rotated page geometry
%\usepackage[parfill]{parskip}    		% Activate to begin paragraphs with an empty line rather than an indent
\usepackage{graphicx}				% Use pdf, png, jpg, or eps§ with pdflatex; use eps in DVI mode
								% TeX will automatically convert eps --> pdf in pdflatex		
\usepackage{amssymb}
\usepackage{fancyvrb}
\usepackage{hyperref}
\usepackage{overpic}
\usepackage{xcolor}

\newcommand{\escapepercent}{\%}

\usepackage{listings}
	\definecolor{textblue}{rgb}{.2,.2,.7}
	\definecolor{textred}{rgb}{0.54,0,0}
	\definecolor{textgreen}{rgb}{0,0.43,0}

% input{./setup/macros}

% macros to simplfy typing and make the product more reliable
\newcommand{\emailTopa}[0]			{\href{mailto:daniel.topa@hii-tsd.com}{daniel.topa@hii-tsd.com}}

\newcommand{\textt}[1]				{{\small{\texttt{#1}}}}

\newcommand{\urlMan}[0]				{https://man7.org/linux/man-pages/man1/}

\newcommand{\gdb}[0]				{\textt{gdb}}
\newcommand{\urlGdb}[0]				{https://www.sourceware.org/gdb/}
\newcommand{\refGdb}[0]				{\href{\urlGdb}{\gdb}}

\newcommand{\ldd}[0]				{\textt{ldd}}
\newcommand{\urlLdd}[0]				{\urlMan ldd.1.html}
\newcommand{\refLdd}[0]				{\href{\urlLdd}{\ldd}}

\newcommand{\lddconfig}[0]			{\textt{lddconfig}}
\newcommand{\urlLddconfig}[0]			{\urlMan lddconfig.1.html}
\newcommand{\refLddconfig}[0]		{\href{\urlLddconfig}{\lddconfig}}

\newcommand{\lsof}[0]				{\textt{lsof}}
\newcommand{\urlLsof}[0]				{\urlMan lsof.1.html}
\newcommand{\refLsof}[0]				{\href{\urlLsof}{\lsof}}

\newcommand{\locate}[0]				{\textt{locate}}
\newcommand{\urlLocate}[0]			{\urlMan locate.1.html}
\newcommand{\refLocate}[0]			{\href{\urlLocate}{\locate}}

\newcommand{\nm}[0]				{\textt{nm}}
\newcommand{\urlNm}[0]				{\urlMan nm.1.html}
\newcommand{\refNm}[0]				{\href{\urlNm}{\nm}}

\newcommand{\objdump}[0]			{\textt{objdump}}
\newcommand{\urlObjdump}[0]			{\urlMan objdump.1.html}
\newcommand{\refObjdump}[0]			{\href{\urlObjdump}{\objdump}}

\newcommand{\readelf}[0]				{\textt{readelf}}
\newcommand{\urlReadelf}[0]			{\urlMan readelf.1.html}
\newcommand{\refReadelf}[0]			{\href{\urlReadelf}{\readelf}}

\newcommand{\strace}[0]				{\textt{strace}}
\newcommand{\urlStrace}[0]			{\urlMan strace.1.html}
\newcommand{\refStrace}[0]			{\href{\urlStrace}{\strace}}

\newcommand{\strings}[0]				{\textt{strings}}
\newcommand{\urlStrings}[0]			{\urlMan strings.1.html}
\newcommand{\refStrings}[0]			{\href{\urlStrings}{\strings}}

\newcommand{\elf}[0]				{\href{https://en.wikipedia.org/wiki/Executable_and_Linkable_Format}{ELF}}


\endinput  %  ==  ==  ==  ==  ==  ==  ==  ==  ==


% input{./setup/listing-bash}
% https://tex.stackexchange.com/questions/310335/using-bash-listings-to-bold-variables-and-functions

\usepackage[T1]{fontenc}
\usepackage{
  color,
  beramono,
  listings,
  textcomp
}

\definecolor{lightgray}{RGB}{245,245,245}
\definecolor{darkgray}{RGB}{128,128,128}

\lstset{
  abovecaptionskip={0cm},
  backgroundcolor={\color{lightgray}},
  basicstyle={\small\ttfamily},
  breakatwhitespace=true,
  breaklines=true,
  captionpos=b,
  frame=tb,
  resetmargins=true,
  sensitive=true,
  stepnumber=1,
  tabsize=4,
  upquote=true
}

\AtBeginDocument{\lstdefinelanguage{bash}[]{sh}%
  {morekeywords={alias,bg,bind,builtin,caller,command,compgen,compopt,%
      complete,coproc,curl,declare,disown,dirs,enable,fc,fg,help,%
      history,jobs,let,local,logout,mapfile,printf,pushd,popd,%
      readarray,select,set,suspend,shopt,source,times,type,typeset,%
      ulimit,unalias,wait},%
   otherkeywords={ [, ], [[, ]], \{, \} }%
  }%

\lstdefinelanguage{sh}%
  {morekeywords={awk,break,case,cat,cd,continue,do,done,echo,elif,else,%
      env,esac,eval,exec,exit,export,expr,false,fi,for,function,getopts,%
      hash,history,if,in,kill,login,newgrp,nice,nohup,ps,pwd,read,%
      readonly,return,set,sed,shift,test,then,times,trap,true,type,%
      ulimit,umask,unset,until,wait,while},%
   morecomment=[l]\#,%
   morestring=[d]",%
   alsoletter={*"'0123456789.},%
   alsoother={\{\=\}},%
   literate={{=}{{{=}}}1},%
   literate={\$\{}{{{{\bfseries{}\$\{}}}}2,%
   otherkeywords={ [, ], \{, \} }%
  }[keywords,comments,strings]%
}



\endinput  %  ==  ==  ==  ==  ==  ==  ==  ==  ==



\title{Unix Tools for Probing Executable Files}
\author{Daniel Topa\\HII-TSD\\\href{mailto:daniel.topa@hii-tsd.com}{daniel.topa@hii-tsd.com}}

\begin{document}
\maketitle
\abstract{This article surveys Unix tools for the exploration of executable files, some of which depend upon the application being compiled with debug information. The \textt{man}ual pages are included, making this document usefull in siloed computing networks.}
\tableofcontents

\section{Overview}
Here are several Unix commands for probing executable files. The following section shows sample useage for each command and the final section contains the information from the \textt{man}ual page.
\begin{enumerate}
	\item gdb
	\item \hyperref[sec:ldd]{\ldd}	
	\item \hyperref[sec:ldd]{\lddconfig}
	\item \hyperref[sec:locate]{\locate}
	\item \hyperref[sec:objdump]{\objdump}
	\item \hyperref[sec:lsof]{\lsof}
	\item \hyperref[sec:readelf]{\readelf}
	\item \hyperref[sec:nm]{\nm}
	\item \hyperref[sec:strace]{\strace}
	\item \hyperref[sec:strings]{\strings}
\end{enumerate}

The goal is to be able to resolve the workings of an executable file exploiting the \elf \ structure show in figures \ref{fig:elf}. The next figure, \ref{fig:elf-II}, shows the relationship between source files, header files, shared objects, and the executable program.

\begin{figure}[htbp] %  figure placement: here, top, bottom, or page
\centering
	\href{https://camo.githubusercontent.com/00cd4e64df02caf11e9c7c8f67a4d7e9470ea03c244e6d5bce8444a674b9143c/68747470733a2f2f692e696d6775722e636f6d2f4169394f714f422e706e67}{
	\begin{overpic}[ scale = 0.4 ]
		{./local/graphics-local/elf-01}
		%\put(50, 50)	{\colorbox{white}{$a+b$}}
	\end{overpic}
	}
\caption{The structure of a Unix \elf \ file.}
\label{fig:elf}
\end{figure}

\begin{figure}[htbp] %  figure placement: here, top, bottom, or page
\centering
	\href{https://camo.githubusercontent.com/94b1128b885c29e21c64fb3b247d0184c54f4248e4195462bd15671003afc319/68747470733a2f2f692e696d6775722e636f6d2f4c4e6464546d6b2e706e67}{
	\begin{overpic}[ scale = 0.5 ]
		{./local/graphics-local/elf-02}
		%\put(50, 50)	{\colorbox{white}{$a+b$}}
	\end{overpic}
	}
\caption{Connecting source files, object files, libraries, and bindary executables.}
\label{fig:elf-II}
\end{figure}

\section{Command Examples}
		% % % \input{./sections/ssec-ldd}

\subsection{\ldd}
\label{sec:ldd}
The command \refLdd \ prints shared object dependencies, in this example, for the executable \texttt{bash}:
{\footnotesize{
\begin{Verbatim}[commandchars=\\\{\}]
{\color{darkgray}{root@69cb14a32689:/}}# ldd /bin/bash
{\color{darkgray}{	linux-vdso.so.1 (0x00007ffe64317000)}}
{\color{darkgray}{	libtinfo.so.6 }{\color{blue}{=>}}\color{darkgray}{ /lib/x86_64-linux-gnu/libtinfo.so.6 (0x00007f842112d000)}}
{\color{darkgray}{	libc.so.6 }{\color{blue}{=>}}\color{darkgray}{ /lib/x86_64-linux-gnu/libc.so.6 (0x00007f8420f04000)}}
{\color{darkgray}{	/lib64/ld-linux-x86-64.so.2 (0x00007f84212e3000)}}
\end{Verbatim}
}}
\href{https://en.wikipedia.org/wiki/Symbolic_link}{Symbolic link}s (symlinks) are highlighted with blue color.

\endinput  %  ==  ==  ==  ==  ==  ==  ==  ==  ==


\subsection{\ldd}
\label{sec:ldd}
The command \refLdd \ prints shared object dependencies, in this example, for the executable \texttt{bash}:
{\footnotesize{
\begin{Verbatim}[commandchars=\\\{\}]
{\color{darkgray}{root@69cb14a32689:/}}# ldd /bin/bash
{\color{darkgray}{	linux-vdso.so.1 (0x00007ffe64317000)}}
{\color{darkgray}{	libtinfo.so.6 }{\color{blue}{=>}}\color{darkgray}{ /lib/x86_64-linux-gnu/libtinfo.so.6 (0x00007f842112d000)}}
{\color{darkgray}{	libc.so.6 }{\color{blue}{=>}}\color{darkgray}{ /lib/x86_64-linux-gnu/libc.so.6 (0x00007f8420f04000)}}
{\color{darkgray}{	/lib64/ld-linux-x86-64.so.2 (0x00007f84212e3000)}}
\end{Verbatim}
}}
\href{https://en.wikipedia.org/wiki/Symbolic_link}{Symbolic link}s (symlinks) are highlighted with blue color.

\endinput  %  ==  ==  ==  ==  ==  ==  ==  ==  ==


\subsection{\ldd}
\label{sec:ldd}
The command \refLdd \ prints shared object dependencies, in this example, for the executable \texttt{bash}:
{\footnotesize{
\begin{Verbatim}[commandchars=\\\{\}]
{\color{darkgray}{root@69cb14a32689:/}}# ldd /bin/bash
{\color{darkgray}{	linux-vdso.so.1 (0x00007ffe64317000)}}
{\color{darkgray}{	libtinfo.so.6 }{\color{blue}{=>}}\color{darkgray}{ /lib/x86_64-linux-gnu/libtinfo.so.6 (0x00007f842112d000)}}
{\color{darkgray}{	libc.so.6 }{\color{blue}{=>}}\color{darkgray}{ /lib/x86_64-linux-gnu/libc.so.6 (0x00007f8420f04000)}}
{\color{darkgray}{	/lib64/ld-linux-x86-64.so.2 (0x00007f84212e3000)}}
\end{Verbatim}
}}
\href{https://en.wikipedia.org/wiki/Symbolic_link}{Symbolic link}s (symlinks) are highlighted with blue color.

\endinput  %  ==  ==  ==  ==  ==  ==  ==  ==  ==

		% % % \input{./sections/ssec-nm}

\subsection{\nm}
\label{sec:nm}

The \refNm \ command shows dependent shared objects and executables; 

\endinput  %  ==  ==  ==  ==  ==  ==  ==  ==  ==


\subsection{\nm}
\label{sec:nm}

The \refNm \ command shows dependent shared objects and executables; 

\endinput  %  ==  ==  ==  ==  ==  ==  ==  ==  ==


\subsection{\lddconfig}
\label{sec:lddconfig}

\refLddconfig: A tool to cnfigure run-time bindings for the dynamic linker. May reside in \textt{/sbin/lddconfig}.

		\subsubsection{Display First Five Current Libraries In The Cache}
{\footnotesize{
\begin{Verbatim}[commandchars=\\\{\}]
\href{https://linux.101hacks.com/unix/ldconfig/}{# ldconfig -p | head -5}
{\color{darkgray}{916 libs found in cache `/etc/ld.so.cache'}}
{\color{darkgray}{      libzephyr.so.4 (libc6) => /usr/lib/libzephyr.so.4}}
{\color{darkgray}{      libzbar.so.0 (libc6) => /usr/lib/libzbar.so.0}}
{\color{darkgray}{      libz.so.1 (libc6) => /lib/libz.so.1}}
{\color{darkgray}{      libz.so (libc6) => /usr/lib/libz.so}}
\end{Verbatim}
}}

		\subsubsection{Recursively Display Libraries For Each Directory}
{\footnotesize{
\begin{Verbatim}[commandchars=\\\{\}]
\href{https://linux.101hacks.com/unix/ldconfig/}{# ldconfig -v | head}
{\color{darkgray}{/usr/lib/mesa:}}
{\color{darkgray}{      libGL.so.1 -> libGL.so.1.2}}
{\color{darkgray}{/usr/lib/i686-linux-gnu:}}
{\color{darkgray}{      liblouis.so.2 -> liblouis.so.2.2.0}}
{\color{darkgray}{/usr/lib/alsa-lib:}}
{\color{darkgray}{      libasound_module_ctl_oss.so -> libasound_module_ctl_oss.so}}
{\color{darkgray}{      libasound_module_ctl_bluetooth.so -> libasound_module_ctl_bluetooth.so}}
{\color{darkgray}{      libasound_module_pcm_bluetooth.so -> libasound_module_pcm_bluetooth.so}}
{\color{darkgray}{      libasound_module_pcm_vdownmix.so -> libasound_module_pcm_vdownmix.so}}
 {\color{darkgray}{     libasound_module_rate_speexrate.so -> libasound_module_rate_speexrate_medium.so}}
\end{Verbatim}
}}

\endinput  %  ==  ==  ==  ==  ==  ==  ==  ==  ==

		% % % \input{./sections/ssec-locate}

\subsection{\locate}
\label{sec:locate}
The \refLocate \ \href{https://en.wikipedia.org/wiki/Locate_(Unix)}{command} lists files in a prebuilt database of files generated by the \textt{updatedb} command or by a daemon and compressed using incremental encoding.
{\footnotesize{
\begin{Verbatim}[commandchars=\\\{\}]
{\color{darkgray}{dantopa@92bc4c447e32:/}}$ locate libc.so.6
{\color{darkgray}{/usr/lib/x86_64-linux-gnu/libc.so.6}}
{\color{darkgray}{/usr/lib32/libc.so.6}}
\end{Verbatim}
}}

\endinput  %  ==  ==  ==  ==  ==  ==  ==  ==  ==


\subsection{\locate}
\label{sec:locate}
The \refLocate \ \href{https://en.wikipedia.org/wiki/Locate_(Unix)}{command} lists files in a prebuilt database of files generated by the \textt{updatedb} command or by a daemon and compressed using incremental encoding.
{\footnotesize{
\begin{Verbatim}[commandchars=\\\{\}]
{\color{darkgray}{dantopa@92bc4c447e32:/}}$ locate libc.so.6
{\color{darkgray}{/usr/lib/x86_64-linux-gnu/libc.so.6}}
{\color{darkgray}{/usr/lib32/libc.so.6}}
\end{Verbatim}
}}

\endinput  %  ==  ==  ==  ==  ==  ==  ==  ==  ==


\subsection{\locate}
\label{sec:locate}
The \refLocate \ \href{https://en.wikipedia.org/wiki/Locate_(Unix)}{command} lists files in a prebuilt database of files generated by the \textt{updatedb} command or by a daemon and compressed using incremental encoding.
{\footnotesize{
\begin{Verbatim}[commandchars=\\\{\}]
{\color{darkgray}{dantopa@92bc4c447e32:/}}$ locate libc.so.6
{\color{darkgray}{/usr/lib/x86_64-linux-gnu/libc.so.6}}
{\color{darkgray}{/usr/lib32/libc.so.6}}
\end{Verbatim}
}}

\endinput  %  ==  ==  ==  ==  ==  ==  ==  ==  ==

		% % % \input{./sections/ssec-losf}

\subsection{\lsof}
\label{sec:lsof}
This command does an \textt{ls} on open files. The example show how to query both a user and a process id (\textt{pid}).
		\subsubsection{\lsof \ on Process ID}
The \refLsof \ command shows open files, here for the bash process with PID = 10932:
{\footnotesize{
\begin{Verbatim}[commandchars=\\\{\}]
{\color{darkgray}{dantopa@92bc4c447e32:~}}$ ps
{\color{darkgray}{  PID TTY          TIME CMD}}
{\color{darkgray}{10932 pts/1    00:00:00 bash}}
{\color{darkgray}{11152 pts/1    00:00:00 ps}}
{\color{darkgray}{dantopa@92bc4c447e32:~}}$ lsof -p 10932
{\color{darkgray}{COMMAND   PID    USER   FD   TYPE DEVICE SIZE/OFF     NODE NAME}}
{\color{darkgray}{bash    10932 dantopa  cwd    DIR   0,71     4096  6820049 /home/dantopa}}
{\color{darkgray}{bash    10932 dantopa  rtd    DIR   0,71     4096 61653409 /}}
{\color{darkgray}{bash    10932 dantopa  txt    REG   0,71  1396520 62702252 /usr/bin/bash}}
{\color{darkgray}{bash    10932 dantopa  mem    REG  254,1          62702252 /usr/bin/bash (path dev=0,71)}}
{\color{darkgray}{bash    10932 dantopa  mem    REG  254,1          63095938 /usr/lib/x86_64-linux-gnu/libc.so.6 (path dev=0,71)}}
{\color{darkgray}{bash    10932 dantopa  mem    REG  254,1           1190606 /usr/lib/x86_64-linux-gnu/libtinfo.so.6.3 (path dev=0,71)}}
{\color{darkgray}{bash    10932 dantopa  mem    REG  254,1          63095935 /usr/lib/x86_64-linux-gnu/ld-linux-x86-64.so.2 (path dev=0,71)}}
{\color{darkgray}{bash    10932 dantopa    0u   CHR  136,1      0t0        4 /dev/pts/1}}
{\color{darkgray}{bash    10932 dantopa    1u   CHR  136,1      0t0        4 /dev/pts/1}}
{\color{darkgray}{bash    10932 dantopa    2u   CHR  136,1      0t0        4 /dev/pts/1}}
{\color{darkgray}{bash    10932 dantopa  255u   CHR  136,1      0t0        4 /dev/pts/1}}
\end{Verbatim}
}
		\subsubsection{\lsof \ on User}
These are open files for user \textt{dantopa}:
{\footnotesize{
\begin{Verbatim}[commandchars=\\\{\}]
{\color{darkgray}{dantopa@92bc4c447e32:~}}$ lsof -u dantopa
{\color{darkgray}{COMMAND   PID    USER   FD   TYPE DEVICE SIZE/OFF     NODE NAME}}
{\color{darkgray}{bash    10921 dantopa  cwd    DIR   0,71     4096 61653409 /}}
{\color{darkgray}{bash    10921 dantopa  rtd    DIR   0,71     4096 61653409 /}}
{\color{darkgray}{bash    10921 dantopa  txt    REG   0,71  1396520 62702252 /usr/bin/bash}}
{\color{darkgray}{bash    10921 dantopa  mem    REG  254,1          62702252 /usr/bin/bash (path dev=0,71)}}
{\color{darkgray}{bash    10921 dantopa  mem    REG  254,1          63095938 /usr/lib/x86_64-linux-gnu/libc.so.6 (path dev=0,71)}}
{\color{darkgray}{bash    10921 dantopa  mem    REG  254,1           1190606 /usr/lib/x86_64-linux-gnu/libtinfo.so.6.3 (path dev=0,71)}}
{\color{darkgray}{bash    10921 dantopa  mem    REG  254,1          63095935 /usr/lib/x86_64-linux-gnu/ld-linux-x86-64.so.2 (path dev=0,71)}}
{\color{darkgray}{bash    10921 dantopa    0u   CHR  136,0      0t0        3 /dev/pts/0}}
{\color{darkgray}{bash    10921 dantopa    1u   CHR  136,0      0t0        3 /dev/pts/0}}
{\color{darkgray}{bash    10921 dantopa    2u   CHR  136,0      0t0        3 /dev/pts/0}}
{\color{darkgray}{bash    10921 dantopa  255u   CHR  136,0      0t0        3 /dev/pts/0}}
{\color{darkgray}{bash    10932 dantopa  cwd    DIR   0,33      704     1572 /repos/github/vault-fortran/Xmodern-fortran/tau/apex}}
{\color{darkgray}{bash    10932 dantopa  rtd    DIR   0,71     4096 61653409 /}}
{\color{darkgray}{bash    10932 dantopa  txt    REG   0,71  1396520 62702252 /usr/bin/bash}}
{\color{darkgray}{bash    10932 dantopa  mem    REG  254,1          62702252 /usr/bin/bash (path dev=0,71)}}
{\color{darkgray}{bash    10932 dantopa  mem    REG  254,1          63095938 /usr/lib/x86_64-linux-gnu/libc.so.6 (path dev=0,71)}}
{\color{darkgray}{bash    10932 dantopa  mem    REG  254,1           1190606 /usr/lib/x86_64-linux-gnu/libtinfo.so.6.3 (path dev=0,71)}}
{\color{darkgray}{bash    10932 dantopa  mem    REG  254,1          63095935 /usr/lib/x86_64-linux-gnu/ld-linux-x86-64.so.2 (path dev=0,71)}}
{\color{darkgray}{bash    10932 dantopa    0u   CHR  136,1      0t0        4 /dev/pts/1}}
{\color{darkgray}{bash    10932 dantopa    1u   CHR  136,1      0t0        4 /dev/pts/1}}
{\color{darkgray}{bash    10932 dantopa    2u   CHR  136,1      0t0        4 /dev/pts/1}}
{\color{darkgray}{bash    10932 dantopa  255u   CHR  136,1      0t0        4 /dev/pts/1}}
{\color{darkgray}{lsof    11139 dantopa  cwd    DIR   0,33      704     1572 /repos/github/vault-fortran/Xmodern-fortran/tau/apex}}
{\color{darkgray}{lsof    11139 dantopa  rtd    DIR   0,71     4096 61653409 /}}
{\color{darkgray}{lsof    11139 dantopa  txt    REG   0,71   167544   709329 /usr/bin/lsof}}
{\color{darkgray}{lsof    11139 dantopa  mem    REG  254,1            709329 /usr/bin/lsof (path dev=0,71)}}
{\color{darkgray}{lsof    11139 dantopa  mem    REG  254,1          63095951 /usr/lib/x86_64-linux-gnu/libresolv.so.2 (path dev=0,71)}}
{\color{darkgray}{lsof    11139 dantopa  mem    REG  254,1           1190531 /usr/lib/x86_64-linux-gnu/libkeyutils.so.1.9 (path dev=0,71)}}
{\color{darkgray}{lsof    11139 dantopa  mem    REG  254,1          63096020 /usr/lib/x86_64-linux-gnu/libkrb5support.so.0.1 (path dev=0,71)}}
{\color{darkgray}{lsof    11139 dantopa  mem    REG  254,1          63096026 /usr/lib/x86_64-linux-gnu/libcom_err.so.2.1 (path dev=0,71)}}
{\color{darkgray}{lsof    11139 dantopa  mem    REG  254,1          63096018 /usr/lib/x86_64-linux-gnu/libk5crypto.so.3.1 (path dev=0,71)}}
{\color{darkgray}{lsof    11139 dantopa  mem    REG  254,1          63096022 /usr/lib/x86_64-linux-gnu/libkrb5.so.3.3 (path dev=0,71)}}
{\color{darkgray}{lsof    11139 dantopa  mem    REG  254,1           1190578 /usr/lib/x86_64-linux-gnu/libpcre2-8.so.0.10.4 (path dev=0,71)}}
{\color{darkgray}{lsof    11139 dantopa  mem    REG  254,1          63096024 /usr/lib/x86_64-linux-gnu/libgssapi_krb5.so.2.2 (path dev=0,71)}}
{\color{darkgray}{lsof    11139 dantopa  mem    REG  254,1          63095938 /usr/lib/x86_64-linux-gnu/libc.so.6 (path dev=0,71)}}
{\color{darkgray}{lsof    11139 dantopa  mem    REG  254,1           1190588 /usr/lib/x86_64-linux-gnu/libselinux.so.1 (path dev=0,71)}}
{\color{darkgray}{lsof    11139 dantopa  mem    REG  254,1           1190608 /usr/lib/x86_64-linux-gnu/libtirpc.so.3.0.0 (path dev=0,71)}}
{\color{darkgray}{lsof    11139 dantopa  mem    REG  254,1          63095935 /usr/lib/x86_64-linux-gnu/ld-linux-x86-64.so.2 (path dev=0,71)}}
{\color{darkgray}{lsof    11139 dantopa    0u   CHR  136,1      0t0        4 /dev/pts/1}}
{\color{darkgray}{lsof    11139 dantopa    1u   CHR  136,1      0t0        4 /dev/pts/1}}
{\color{darkgray}{lsof    11139 dantopa    2u   CHR  136,1      0t0        4 /dev/pts/1}}
{\color{darkgray}{lsof    11139 dantopa    3r   DIR   0,74        0        1 /proc}}
{\color{darkgray}{lsof    11139 dantopa    4r   DIR   0,74        7   123326 /proc/11139/fd}}
{\color{darkgray}{lsof    11139 dantopa    5w  FIFO   0,11      0t0   123331 pipe}}
{\color{darkgray}{lsof    11139 dantopa    6r  FIFO   0,11      0t0   123332 pipe}}
{\color{darkgray}{lsof    11140 dantopa  cwd    DIR   0,33      704     1572 /repos/github/vault-fortran/Xmodern-fortran/tau/apex}}
{\color{darkgray}{lsof    11140 dantopa  rtd    DIR   0,71     4096 61653409 /}}
{\color{darkgray}{lsof    11140 dantopa  txt    REG   0,71   167544   709329 /usr/bin/lsof}}
{\color{darkgray}{lsof    11140 dantopa  mem    REG  254,1            709329 /usr/bin/lsof (path dev=0,71)}}
{\color{darkgray}{lsof    11140 dantopa  mem    REG  254,1          63095951 /usr/lib/x86_64-linux-gnu/libresolv.so.2 (path dev=0,71)}}
{\color{darkgray}{lsof    11140 dantopa  mem    REG  254,1           1190531 /usr/lib/x86_64-linux-gnu/libkeyutils.so.1.9 (path dev=0,71)}}
{\color{darkgray}{lsof    11140 dantopa  mem    REG  254,1          63096020 /usr/lib/x86_64-linux-gnu/libkrb5support.so.0.1 (path dev=0,71)}}
{\color{darkgray}{lsof    11140 dantopa  mem    REG  254,1          63096026 /usr/lib/x86_64-linux-gnu/libcom_err.so.2.1 (path dev=0,71)}}
{\color{darkgray}{lsof    11140 dantopa  mem    REG  254,1          63096018 /usr/lib/x86_64-linux-gnu/libk5crypto.so.3.1 (path dev=0,71)}}
{\color{darkgray}{lsof    11140 dantopa  mem    REG  254,1          63096022 /usr/lib/x86_64-linux-gnu/libkrb5.so.3.3 (path dev=0,71)}}
{\color{darkgray}{lsof    11140 dantopa  mem    REG  254,1           1190578 /usr/lib/x86_64-linux-gnu/libpcre2-8.so.0.10.4 (path dev=0,71)}}
{\color{darkgray}{lsof    11140 dantopa  mem    REG  254,1          63096024 /usr/lib/x86_64-linux-gnu/libgssapi_krb5.so.2.2 (path dev=0,71)}}
{\color{darkgray}{lsof    11140 dantopa  mem    REG  254,1          63095938 /usr/lib/x86_64-linux-gnu/libc.so.6 (path dev=0,71)}}
{\color{darkgray}{lsof    11140 dantopa  mem    REG  254,1           1190588 /usr/lib/x86_64-linux-gnu/libselinux.so.1 (path dev=0,71)}}
{\color{darkgray}{lsof    11140 dantopa  mem    REG  254,1           1190608 /usr/lib/x86_64-linux-gnu/libtirpc.so.3.0.0 (path dev=0,71)}}
{\color{darkgray}{lsof    11140 dantopa  mem    REG  254,1          63095935 /usr/lib/x86_64-linux-gnu/ld-linux-x86-64.so.2 (path dev=0,71)}}
{\color{darkgray}{lsof    11140 dantopa    4r  FIFO   0,11      0t0   123331 pipe}}
{\color{darkgray}{lsof    11140 dantopa    7w  FIFO   0,11      0t0   123332 pipe}}
\end{Verbatim}
}}
\endinput  %  ==  ==  ==  ==  ==  ==  ==  ==  ==


\subsection{\lsof}
\label{sec:lsof}
This command does an \textt{ls} on open files. The example show how to query both a user and a process id (\textt{pid}).
		\subsubsection{\lsof \ on Process ID}
The \refLsof \ command shows open files, here for the bash process with PID = 10932:
{\footnotesize{
\begin{Verbatim}[commandchars=\\\{\}]
{\color{darkgray}{dantopa@92bc4c447e32:~}}$ ps
{\color{darkgray}{  PID TTY          TIME CMD}}
{\color{darkgray}{10932 pts/1    00:00:00 bash}}
{\color{darkgray}{11152 pts/1    00:00:00 ps}}
{\color{darkgray}{dantopa@92bc4c447e32:~}}$ lsof -p 10932
{\color{darkgray}{COMMAND   PID    USER   FD   TYPE DEVICE SIZE/OFF     NODE NAME}}
{\color{darkgray}{bash    10932 dantopa  cwd    DIR   0,71     4096  6820049 /home/dantopa}}
{\color{darkgray}{bash    10932 dantopa  rtd    DIR   0,71     4096 61653409 /}}
{\color{darkgray}{bash    10932 dantopa  txt    REG   0,71  1396520 62702252 /usr/bin/bash}}
{\color{darkgray}{bash    10932 dantopa  mem    REG  254,1          62702252 /usr/bin/bash (path dev=0,71)}}
{\color{darkgray}{bash    10932 dantopa  mem    REG  254,1          63095938 /usr/lib/x86_64-linux-gnu/libc.so.6 (path dev=0,71)}}
{\color{darkgray}{bash    10932 dantopa  mem    REG  254,1           1190606 /usr/lib/x86_64-linux-gnu/libtinfo.so.6.3 (path dev=0,71)}}
{\color{darkgray}{bash    10932 dantopa  mem    REG  254,1          63095935 /usr/lib/x86_64-linux-gnu/ld-linux-x86-64.so.2 (path dev=0,71)}}
{\color{darkgray}{bash    10932 dantopa    0u   CHR  136,1      0t0        4 /dev/pts/1}}
{\color{darkgray}{bash    10932 dantopa    1u   CHR  136,1      0t0        4 /dev/pts/1}}
{\color{darkgray}{bash    10932 dantopa    2u   CHR  136,1      0t0        4 /dev/pts/1}}
{\color{darkgray}{bash    10932 dantopa  255u   CHR  136,1      0t0        4 /dev/pts/1}}
\end{Verbatim}
}
		\subsubsection{\lsof \ on User}
These are open files for user \textt{dantopa}:
{\footnotesize{
\begin{Verbatim}[commandchars=\\\{\}]
{\color{darkgray}{dantopa@92bc4c447e32:~}}$ lsof -u dantopa
{\color{darkgray}{COMMAND   PID    USER   FD   TYPE DEVICE SIZE/OFF     NODE NAME}}
{\color{darkgray}{bash    10921 dantopa  cwd    DIR   0,71     4096 61653409 /}}
{\color{darkgray}{bash    10921 dantopa  rtd    DIR   0,71     4096 61653409 /}}
{\color{darkgray}{bash    10921 dantopa  txt    REG   0,71  1396520 62702252 /usr/bin/bash}}
{\color{darkgray}{bash    10921 dantopa  mem    REG  254,1          62702252 /usr/bin/bash (path dev=0,71)}}
{\color{darkgray}{bash    10921 dantopa  mem    REG  254,1          63095938 /usr/lib/x86_64-linux-gnu/libc.so.6 (path dev=0,71)}}
{\color{darkgray}{bash    10921 dantopa  mem    REG  254,1           1190606 /usr/lib/x86_64-linux-gnu/libtinfo.so.6.3 (path dev=0,71)}}
{\color{darkgray}{bash    10921 dantopa  mem    REG  254,1          63095935 /usr/lib/x86_64-linux-gnu/ld-linux-x86-64.so.2 (path dev=0,71)}}
{\color{darkgray}{bash    10921 dantopa    0u   CHR  136,0      0t0        3 /dev/pts/0}}
{\color{darkgray}{bash    10921 dantopa    1u   CHR  136,0      0t0        3 /dev/pts/0}}
{\color{darkgray}{bash    10921 dantopa    2u   CHR  136,0      0t0        3 /dev/pts/0}}
{\color{darkgray}{bash    10921 dantopa  255u   CHR  136,0      0t0        3 /dev/pts/0}}
{\color{darkgray}{bash    10932 dantopa  cwd    DIR   0,33      704     1572 /repos/github/vault-fortran/Xmodern-fortran/tau/apex}}
{\color{darkgray}{bash    10932 dantopa  rtd    DIR   0,71     4096 61653409 /}}
{\color{darkgray}{bash    10932 dantopa  txt    REG   0,71  1396520 62702252 /usr/bin/bash}}
{\color{darkgray}{bash    10932 dantopa  mem    REG  254,1          62702252 /usr/bin/bash (path dev=0,71)}}
{\color{darkgray}{bash    10932 dantopa  mem    REG  254,1          63095938 /usr/lib/x86_64-linux-gnu/libc.so.6 (path dev=0,71)}}
{\color{darkgray}{bash    10932 dantopa  mem    REG  254,1           1190606 /usr/lib/x86_64-linux-gnu/libtinfo.so.6.3 (path dev=0,71)}}
{\color{darkgray}{bash    10932 dantopa  mem    REG  254,1          63095935 /usr/lib/x86_64-linux-gnu/ld-linux-x86-64.so.2 (path dev=0,71)}}
{\color{darkgray}{bash    10932 dantopa    0u   CHR  136,1      0t0        4 /dev/pts/1}}
{\color{darkgray}{bash    10932 dantopa    1u   CHR  136,1      0t0        4 /dev/pts/1}}
{\color{darkgray}{bash    10932 dantopa    2u   CHR  136,1      0t0        4 /dev/pts/1}}
{\color{darkgray}{bash    10932 dantopa  255u   CHR  136,1      0t0        4 /dev/pts/1}}
{\color{darkgray}{lsof    11139 dantopa  cwd    DIR   0,33      704     1572 /repos/github/vault-fortran/Xmodern-fortran/tau/apex}}
{\color{darkgray}{lsof    11139 dantopa  rtd    DIR   0,71     4096 61653409 /}}
{\color{darkgray}{lsof    11139 dantopa  txt    REG   0,71   167544   709329 /usr/bin/lsof}}
{\color{darkgray}{lsof    11139 dantopa  mem    REG  254,1            709329 /usr/bin/lsof (path dev=0,71)}}
{\color{darkgray}{lsof    11139 dantopa  mem    REG  254,1          63095951 /usr/lib/x86_64-linux-gnu/libresolv.so.2 (path dev=0,71)}}
{\color{darkgray}{lsof    11139 dantopa  mem    REG  254,1           1190531 /usr/lib/x86_64-linux-gnu/libkeyutils.so.1.9 (path dev=0,71)}}
{\color{darkgray}{lsof    11139 dantopa  mem    REG  254,1          63096020 /usr/lib/x86_64-linux-gnu/libkrb5support.so.0.1 (path dev=0,71)}}
{\color{darkgray}{lsof    11139 dantopa  mem    REG  254,1          63096026 /usr/lib/x86_64-linux-gnu/libcom_err.so.2.1 (path dev=0,71)}}
{\color{darkgray}{lsof    11139 dantopa  mem    REG  254,1          63096018 /usr/lib/x86_64-linux-gnu/libk5crypto.so.3.1 (path dev=0,71)}}
{\color{darkgray}{lsof    11139 dantopa  mem    REG  254,1          63096022 /usr/lib/x86_64-linux-gnu/libkrb5.so.3.3 (path dev=0,71)}}
{\color{darkgray}{lsof    11139 dantopa  mem    REG  254,1           1190578 /usr/lib/x86_64-linux-gnu/libpcre2-8.so.0.10.4 (path dev=0,71)}}
{\color{darkgray}{lsof    11139 dantopa  mem    REG  254,1          63096024 /usr/lib/x86_64-linux-gnu/libgssapi_krb5.so.2.2 (path dev=0,71)}}
{\color{darkgray}{lsof    11139 dantopa  mem    REG  254,1          63095938 /usr/lib/x86_64-linux-gnu/libc.so.6 (path dev=0,71)}}
{\color{darkgray}{lsof    11139 dantopa  mem    REG  254,1           1190588 /usr/lib/x86_64-linux-gnu/libselinux.so.1 (path dev=0,71)}}
{\color{darkgray}{lsof    11139 dantopa  mem    REG  254,1           1190608 /usr/lib/x86_64-linux-gnu/libtirpc.so.3.0.0 (path dev=0,71)}}
{\color{darkgray}{lsof    11139 dantopa  mem    REG  254,1          63095935 /usr/lib/x86_64-linux-gnu/ld-linux-x86-64.so.2 (path dev=0,71)}}
{\color{darkgray}{lsof    11139 dantopa    0u   CHR  136,1      0t0        4 /dev/pts/1}}
{\color{darkgray}{lsof    11139 dantopa    1u   CHR  136,1      0t0        4 /dev/pts/1}}
{\color{darkgray}{lsof    11139 dantopa    2u   CHR  136,1      0t0        4 /dev/pts/1}}
{\color{darkgray}{lsof    11139 dantopa    3r   DIR   0,74        0        1 /proc}}
{\color{darkgray}{lsof    11139 dantopa    4r   DIR   0,74        7   123326 /proc/11139/fd}}
{\color{darkgray}{lsof    11139 dantopa    5w  FIFO   0,11      0t0   123331 pipe}}
{\color{darkgray}{lsof    11139 dantopa    6r  FIFO   0,11      0t0   123332 pipe}}
{\color{darkgray}{lsof    11140 dantopa  cwd    DIR   0,33      704     1572 /repos/github/vault-fortran/Xmodern-fortran/tau/apex}}
{\color{darkgray}{lsof    11140 dantopa  rtd    DIR   0,71     4096 61653409 /}}
{\color{darkgray}{lsof    11140 dantopa  txt    REG   0,71   167544   709329 /usr/bin/lsof}}
{\color{darkgray}{lsof    11140 dantopa  mem    REG  254,1            709329 /usr/bin/lsof (path dev=0,71)}}
{\color{darkgray}{lsof    11140 dantopa  mem    REG  254,1          63095951 /usr/lib/x86_64-linux-gnu/libresolv.so.2 (path dev=0,71)}}
{\color{darkgray}{lsof    11140 dantopa  mem    REG  254,1           1190531 /usr/lib/x86_64-linux-gnu/libkeyutils.so.1.9 (path dev=0,71)}}
{\color{darkgray}{lsof    11140 dantopa  mem    REG  254,1          63096020 /usr/lib/x86_64-linux-gnu/libkrb5support.so.0.1 (path dev=0,71)}}
{\color{darkgray}{lsof    11140 dantopa  mem    REG  254,1          63096026 /usr/lib/x86_64-linux-gnu/libcom_err.so.2.1 (path dev=0,71)}}
{\color{darkgray}{lsof    11140 dantopa  mem    REG  254,1          63096018 /usr/lib/x86_64-linux-gnu/libk5crypto.so.3.1 (path dev=0,71)}}
{\color{darkgray}{lsof    11140 dantopa  mem    REG  254,1          63096022 /usr/lib/x86_64-linux-gnu/libkrb5.so.3.3 (path dev=0,71)}}
{\color{darkgray}{lsof    11140 dantopa  mem    REG  254,1           1190578 /usr/lib/x86_64-linux-gnu/libpcre2-8.so.0.10.4 (path dev=0,71)}}
{\color{darkgray}{lsof    11140 dantopa  mem    REG  254,1          63096024 /usr/lib/x86_64-linux-gnu/libgssapi_krb5.so.2.2 (path dev=0,71)}}
{\color{darkgray}{lsof    11140 dantopa  mem    REG  254,1          63095938 /usr/lib/x86_64-linux-gnu/libc.so.6 (path dev=0,71)}}
{\color{darkgray}{lsof    11140 dantopa  mem    REG  254,1           1190588 /usr/lib/x86_64-linux-gnu/libselinux.so.1 (path dev=0,71)}}
{\color{darkgray}{lsof    11140 dantopa  mem    REG  254,1           1190608 /usr/lib/x86_64-linux-gnu/libtirpc.so.3.0.0 (path dev=0,71)}}
{\color{darkgray}{lsof    11140 dantopa  mem    REG  254,1          63095935 /usr/lib/x86_64-linux-gnu/ld-linux-x86-64.so.2 (path dev=0,71)}}
{\color{darkgray}{lsof    11140 dantopa    4r  FIFO   0,11      0t0   123331 pipe}}
{\color{darkgray}{lsof    11140 dantopa    7w  FIFO   0,11      0t0   123332 pipe}}
\end{Verbatim}
}}
\endinput  %  ==  ==  ==  ==  ==  ==  ==  ==  ==


\subsection{\lsof}
\label{sec:lsof}
This command does an \textt{ls} on open files. The example show how to query both a user and a process id (\textt{pid}).
		\subsubsection{\lsof \ on Process ID}
The \refLsof \ command shows open files, here for the bash process with PID = 10932:
{\footnotesize{
\begin{Verbatim}[commandchars=\\\{\}]
{\color{darkgray}{dantopa@92bc4c447e32:~}}$ ps
{\color{darkgray}{  PID TTY          TIME CMD}}
{\color{darkgray}{10932 pts/1    00:00:00 bash}}
{\color{darkgray}{11152 pts/1    00:00:00 ps}}
{\color{darkgray}{dantopa@92bc4c447e32:~}}$ lsof -p 10932
{\color{darkgray}{COMMAND   PID    USER   FD   TYPE DEVICE SIZE/OFF     NODE NAME}}
{\color{darkgray}{bash    10932 dantopa  cwd    DIR   0,71     4096  6820049 /home/dantopa}}
{\color{darkgray}{bash    10932 dantopa  rtd    DIR   0,71     4096 61653409 /}}
{\color{darkgray}{bash    10932 dantopa  txt    REG   0,71  1396520 62702252 /usr/bin/bash}}
{\color{darkgray}{bash    10932 dantopa  mem    REG  254,1          62702252 /usr/bin/bash (path dev=0,71)}}
{\color{darkgray}{bash    10932 dantopa  mem    REG  254,1          63095938 /usr/lib/x86_64-linux-gnu/libc.so.6 (path dev=0,71)}}
{\color{darkgray}{bash    10932 dantopa  mem    REG  254,1           1190606 /usr/lib/x86_64-linux-gnu/libtinfo.so.6.3 (path dev=0,71)}}
{\color{darkgray}{bash    10932 dantopa  mem    REG  254,1          63095935 /usr/lib/x86_64-linux-gnu/ld-linux-x86-64.so.2 (path dev=0,71)}}
{\color{darkgray}{bash    10932 dantopa    0u   CHR  136,1      0t0        4 /dev/pts/1}}
{\color{darkgray}{bash    10932 dantopa    1u   CHR  136,1      0t0        4 /dev/pts/1}}
{\color{darkgray}{bash    10932 dantopa    2u   CHR  136,1      0t0        4 /dev/pts/1}}
{\color{darkgray}{bash    10932 dantopa  255u   CHR  136,1      0t0        4 /dev/pts/1}}
\end{Verbatim}
}
		\subsubsection{\lsof \ on User}
These are open files for user \textt{dantopa}:
{\footnotesize{
\begin{Verbatim}[commandchars=\\\{\}]
{\color{darkgray}{dantopa@92bc4c447e32:~}}$ lsof -u dantopa
{\color{darkgray}{COMMAND   PID    USER   FD   TYPE DEVICE SIZE/OFF     NODE NAME}}
{\color{darkgray}{bash    10921 dantopa  cwd    DIR   0,71     4096 61653409 /}}
{\color{darkgray}{bash    10921 dantopa  rtd    DIR   0,71     4096 61653409 /}}
{\color{darkgray}{bash    10921 dantopa  txt    REG   0,71  1396520 62702252 /usr/bin/bash}}
{\color{darkgray}{bash    10921 dantopa  mem    REG  254,1          62702252 /usr/bin/bash (path dev=0,71)}}
{\color{darkgray}{bash    10921 dantopa  mem    REG  254,1          63095938 /usr/lib/x86_64-linux-gnu/libc.so.6 (path dev=0,71)}}
{\color{darkgray}{bash    10921 dantopa  mem    REG  254,1           1190606 /usr/lib/x86_64-linux-gnu/libtinfo.so.6.3 (path dev=0,71)}}
{\color{darkgray}{bash    10921 dantopa  mem    REG  254,1          63095935 /usr/lib/x86_64-linux-gnu/ld-linux-x86-64.so.2 (path dev=0,71)}}
{\color{darkgray}{bash    10921 dantopa    0u   CHR  136,0      0t0        3 /dev/pts/0}}
{\color{darkgray}{bash    10921 dantopa    1u   CHR  136,0      0t0        3 /dev/pts/0}}
{\color{darkgray}{bash    10921 dantopa    2u   CHR  136,0      0t0        3 /dev/pts/0}}
{\color{darkgray}{bash    10921 dantopa  255u   CHR  136,0      0t0        3 /dev/pts/0}}
{\color{darkgray}{bash    10932 dantopa  cwd    DIR   0,33      704     1572 /repos/github/vault-fortran/Xmodern-fortran/tau/apex}}
{\color{darkgray}{bash    10932 dantopa  rtd    DIR   0,71     4096 61653409 /}}
{\color{darkgray}{bash    10932 dantopa  txt    REG   0,71  1396520 62702252 /usr/bin/bash}}
{\color{darkgray}{bash    10932 dantopa  mem    REG  254,1          62702252 /usr/bin/bash (path dev=0,71)}}
{\color{darkgray}{bash    10932 dantopa  mem    REG  254,1          63095938 /usr/lib/x86_64-linux-gnu/libc.so.6 (path dev=0,71)}}
{\color{darkgray}{bash    10932 dantopa  mem    REG  254,1           1190606 /usr/lib/x86_64-linux-gnu/libtinfo.so.6.3 (path dev=0,71)}}
{\color{darkgray}{bash    10932 dantopa  mem    REG  254,1          63095935 /usr/lib/x86_64-linux-gnu/ld-linux-x86-64.so.2 (path dev=0,71)}}
{\color{darkgray}{bash    10932 dantopa    0u   CHR  136,1      0t0        4 /dev/pts/1}}
{\color{darkgray}{bash    10932 dantopa    1u   CHR  136,1      0t0        4 /dev/pts/1}}
{\color{darkgray}{bash    10932 dantopa    2u   CHR  136,1      0t0        4 /dev/pts/1}}
{\color{darkgray}{bash    10932 dantopa  255u   CHR  136,1      0t0        4 /dev/pts/1}}
{\color{darkgray}{lsof    11139 dantopa  cwd    DIR   0,33      704     1572 /repos/github/vault-fortran/Xmodern-fortran/tau/apex}}
{\color{darkgray}{lsof    11139 dantopa  rtd    DIR   0,71     4096 61653409 /}}
{\color{darkgray}{lsof    11139 dantopa  txt    REG   0,71   167544   709329 /usr/bin/lsof}}
{\color{darkgray}{lsof    11139 dantopa  mem    REG  254,1            709329 /usr/bin/lsof (path dev=0,71)}}
{\color{darkgray}{lsof    11139 dantopa  mem    REG  254,1          63095951 /usr/lib/x86_64-linux-gnu/libresolv.so.2 (path dev=0,71)}}
{\color{darkgray}{lsof    11139 dantopa  mem    REG  254,1           1190531 /usr/lib/x86_64-linux-gnu/libkeyutils.so.1.9 (path dev=0,71)}}
{\color{darkgray}{lsof    11139 dantopa  mem    REG  254,1          63096020 /usr/lib/x86_64-linux-gnu/libkrb5support.so.0.1 (path dev=0,71)}}
{\color{darkgray}{lsof    11139 dantopa  mem    REG  254,1          63096026 /usr/lib/x86_64-linux-gnu/libcom_err.so.2.1 (path dev=0,71)}}
{\color{darkgray}{lsof    11139 dantopa  mem    REG  254,1          63096018 /usr/lib/x86_64-linux-gnu/libk5crypto.so.3.1 (path dev=0,71)}}
{\color{darkgray}{lsof    11139 dantopa  mem    REG  254,1          63096022 /usr/lib/x86_64-linux-gnu/libkrb5.so.3.3 (path dev=0,71)}}
{\color{darkgray}{lsof    11139 dantopa  mem    REG  254,1           1190578 /usr/lib/x86_64-linux-gnu/libpcre2-8.so.0.10.4 (path dev=0,71)}}
{\color{darkgray}{lsof    11139 dantopa  mem    REG  254,1          63096024 /usr/lib/x86_64-linux-gnu/libgssapi_krb5.so.2.2 (path dev=0,71)}}
{\color{darkgray}{lsof    11139 dantopa  mem    REG  254,1          63095938 /usr/lib/x86_64-linux-gnu/libc.so.6 (path dev=0,71)}}
{\color{darkgray}{lsof    11139 dantopa  mem    REG  254,1           1190588 /usr/lib/x86_64-linux-gnu/libselinux.so.1 (path dev=0,71)}}
{\color{darkgray}{lsof    11139 dantopa  mem    REG  254,1           1190608 /usr/lib/x86_64-linux-gnu/libtirpc.so.3.0.0 (path dev=0,71)}}
{\color{darkgray}{lsof    11139 dantopa  mem    REG  254,1          63095935 /usr/lib/x86_64-linux-gnu/ld-linux-x86-64.so.2 (path dev=0,71)}}
{\color{darkgray}{lsof    11139 dantopa    0u   CHR  136,1      0t0        4 /dev/pts/1}}
{\color{darkgray}{lsof    11139 dantopa    1u   CHR  136,1      0t0        4 /dev/pts/1}}
{\color{darkgray}{lsof    11139 dantopa    2u   CHR  136,1      0t0        4 /dev/pts/1}}
{\color{darkgray}{lsof    11139 dantopa    3r   DIR   0,74        0        1 /proc}}
{\color{darkgray}{lsof    11139 dantopa    4r   DIR   0,74        7   123326 /proc/11139/fd}}
{\color{darkgray}{lsof    11139 dantopa    5w  FIFO   0,11      0t0   123331 pipe}}
{\color{darkgray}{lsof    11139 dantopa    6r  FIFO   0,11      0t0   123332 pipe}}
{\color{darkgray}{lsof    11140 dantopa  cwd    DIR   0,33      704     1572 /repos/github/vault-fortran/Xmodern-fortran/tau/apex}}
{\color{darkgray}{lsof    11140 dantopa  rtd    DIR   0,71     4096 61653409 /}}
{\color{darkgray}{lsof    11140 dantopa  txt    REG   0,71   167544   709329 /usr/bin/lsof}}
{\color{darkgray}{lsof    11140 dantopa  mem    REG  254,1            709329 /usr/bin/lsof (path dev=0,71)}}
{\color{darkgray}{lsof    11140 dantopa  mem    REG  254,1          63095951 /usr/lib/x86_64-linux-gnu/libresolv.so.2 (path dev=0,71)}}
{\color{darkgray}{lsof    11140 dantopa  mem    REG  254,1           1190531 /usr/lib/x86_64-linux-gnu/libkeyutils.so.1.9 (path dev=0,71)}}
{\color{darkgray}{lsof    11140 dantopa  mem    REG  254,1          63096020 /usr/lib/x86_64-linux-gnu/libkrb5support.so.0.1 (path dev=0,71)}}
{\color{darkgray}{lsof    11140 dantopa  mem    REG  254,1          63096026 /usr/lib/x86_64-linux-gnu/libcom_err.so.2.1 (path dev=0,71)}}
{\color{darkgray}{lsof    11140 dantopa  mem    REG  254,1          63096018 /usr/lib/x86_64-linux-gnu/libk5crypto.so.3.1 (path dev=0,71)}}
{\color{darkgray}{lsof    11140 dantopa  mem    REG  254,1          63096022 /usr/lib/x86_64-linux-gnu/libkrb5.so.3.3 (path dev=0,71)}}
{\color{darkgray}{lsof    11140 dantopa  mem    REG  254,1           1190578 /usr/lib/x86_64-linux-gnu/libpcre2-8.so.0.10.4 (path dev=0,71)}}
{\color{darkgray}{lsof    11140 dantopa  mem    REG  254,1          63096024 /usr/lib/x86_64-linux-gnu/libgssapi_krb5.so.2.2 (path dev=0,71)}}
{\color{darkgray}{lsof    11140 dantopa  mem    REG  254,1          63095938 /usr/lib/x86_64-linux-gnu/libc.so.6 (path dev=0,71)}}
{\color{darkgray}{lsof    11140 dantopa  mem    REG  254,1           1190588 /usr/lib/x86_64-linux-gnu/libselinux.so.1 (path dev=0,71)}}
{\color{darkgray}{lsof    11140 dantopa  mem    REG  254,1           1190608 /usr/lib/x86_64-linux-gnu/libtirpc.so.3.0.0 (path dev=0,71)}}
{\color{darkgray}{lsof    11140 dantopa  mem    REG  254,1          63095935 /usr/lib/x86_64-linux-gnu/ld-linux-x86-64.so.2 (path dev=0,71)}}
{\color{darkgray}{lsof    11140 dantopa    4r  FIFO   0,11      0t0   123331 pipe}}
{\color{darkgray}{lsof    11140 dantopa    7w  FIFO   0,11      0t0   123332 pipe}}
\end{Verbatim}
}}
\endinput  %  ==  ==  ==  ==  ==  ==  ==  ==  ==

		% % % \input{./sections/ssec-objdump}

\subsection{\objdump}
\label{sec:objdump}
The \refObjdump \ command shows dependent shared objects, typically libraries. Two versions of the shared library for the GNU standard C library -- one 32 bit, the other 64 bit -- are located.
{\footnotesize{
\begin{Verbatim}[commandchars=\\\{\}]
{\color{darkgray}{dantopa@92bc4c447e32:/}}$ locate libc.so.6
{\color{darkgray}{/usr/lib/x86_64-linux-gnu/libc.so.6}}
{\color{darkgray}{/usr/lib32/libc.so.6}}
\end{Verbatim}
}}

\endinput  %  ==  ==  ==  ==  ==  ==  ==  ==  ==


\subsection{\objdump}
\label{sec:objdump}
The \refObjdump \ command shows dependent shared objects, typically libraries. Two versions of the shared library for the GNU standard C library -- one 32 bit, the other 64 bit -- are located.
{\footnotesize{
\begin{Verbatim}[commandchars=\\\{\}]
{\color{darkgray}{dantopa@92bc4c447e32:/}}$ locate libc.so.6
{\color{darkgray}{/usr/lib/x86_64-linux-gnu/libc.so.6}}
{\color{darkgray}{/usr/lib32/libc.so.6}}
\end{Verbatim}
}}

\endinput  %  ==  ==  ==  ==  ==  ==  ==  ==  ==


\subsection{\objdump}
\label{sec:objdump}
The \refObjdump \ command shows dependent shared objects, typically libraries. Two versions of the shared library for the GNU standard C library -- one 32 bit, the other 64 bit -- are located.
{\footnotesize{
\begin{Verbatim}[commandchars=\\\{\}]
{\color{darkgray}{dantopa@92bc4c447e32:/}}$ locate libc.so.6
{\color{darkgray}{/usr/lib/x86_64-linux-gnu/libc.so.6}}
{\color{darkgray}{/usr/lib32/libc.so.6}}
\end{Verbatim}
}}

\endinput  %  ==  ==  ==  ==  ==  ==  ==  ==  ==

		\input{./sections/ssec-readelf}
		% % % \input{./sections/ssec-nm}

\subsection{\nm}
\label{sec:nm}

The \refNm \ command shows dependent shared objects and executables; 

\endinput  %  ==  ==  ==  ==  ==  ==  ==  ==  ==


\subsection{\nm}
\label{sec:nm}

The \refNm \ command shows dependent shared objects and executables; 

\endinput  %  ==  ==  ==  ==  ==  ==  ==  ==  ==


\subsection{\nm}
\label{sec:nm}

The \refNm \ command shows dependent shared objects and executables; 

\endinput  %  ==  ==  ==  ==  ==  ==  ==  ==  ==

		% % % \input{./sections/ssec-strace}

\subsection{\strace}
\label{sec:strace}
The \strace \ command is very powerful and the following examples.

		\subsubsection{Trace System Calls To A Given Path}
{\footnotesize{
\begin{Verbatim}[commandchars=\\\{\}]
{\color{darkgray}{root@169e8b2c1ae3:/#}} strace -P /etc/ld.so.cache ls /dev/null 
{\color{darkgray}{openat(AT_FDCWD, "/etc/ld.so.cache", O_RDONLY|O_CLOEXEC) = 3}}
{\color{darkgray}{newfstatat(3, "", {st_mode=S_IFREG|0644, st_size=135191, ...}, AT_EMPTY_PATH) = 0}}
{\color{darkgray}{mmap(NULL, 135191, PROT_READ, MAP_PRIVATE, 3, 0) = 0x7f03bba95000}}
{\color{darkgray}{close(3)                                = 0}}
{\color{darkgray}{/dev/null}}
{\color{darkgray}{+++ exited with 0 +++}}
\end{Verbatim}
}}


		\subsubsection{Inventory time, calls, and errors for every system call}
{\footnotesize{
\begin{Verbatim}[commandchars=\\\{\}]
{\color{darkgray}{root@169e8b2c1ae3:/}}# strace -c ls > /dev/null
{\color{darkgray}{\escapepercent time     seconds  usecs/call     calls    errors syscall}}
{\color{darkgray}{------ ----------- ----------- --------- --------- ----------------}}
{\color{darkgray}{ 71.76    0.013546        6773         2           getdents64}}
{\color{darkgray}{  7.85    0.001482         247         6           openat}}
{\color{darkgray}{  4.88    0.000922         922         1           execve}}
{\color{darkgray}{  4.44    0.000839          49        17           mmap}}
{\color{darkgray}{  1.84    0.000347          43         8           close}}
{\color{darkgray}{  1.48    0.000279          39         7           mprotect}}
{\color{darkgray}{  1.40    0.000265          37         7           newfstatat}}
{\color{darkgray}{  1.26    0.000237          47         5           read}}
{\color{darkgray}{  0.94    0.000178          44         4           pread64}}
{\color{darkgray}{  0.77    0.000145          48         3           brk}}
{\color{darkgray}{  0.57    0.000108          36         3         3 ioctl}}
{\color{darkgray}{  0.49    0.000092          46         2         2 statfs}}
{\color{darkgray}{  0.47    0.000088          44         2         2 access}}
{\color{darkgray}{  0.34    0.000065          32         2         1 arch_prctl}}
{\color{darkgray}{  0.34    0.000065          65         1           getrandom}}
{\color{darkgray}{  0.32    0.000061          61         1           munmap}}
{\color{darkgray}{  0.18    0.000034          34         1           rseq}}
{\color{darkgray}{  0.17    0.000032          32         1           set_robust_list}}
{\color{darkgray}{  0.16    0.000031          31         1           write}}
{\color{darkgray}{  0.16    0.000031          31         1           set_tid_address}}
{\color{darkgray}{  0.16    0.000031          31         1           prlimit64}}
{\color{darkgray}{------ ----------- ----------- --------- --------- ----------------}}
{\color{darkgray}{100.00    0.018878         248        76         8 total}}
\end{Verbatim}
}}

		\subsubsection{Identify Information Associated With File Descriptors}
{\footnotesize{
\begin{Verbatim}[commandchars=\\\{\}]
{\color{darkgray}{root@169e8b2c1ae3:/}}# strace -yy cat /dev/null
{\color{darkgray}{execve("/usr/bin/cat", ["cat", "/dev/null"], 0x7fffb8b235d0 /* 10 vars */) = 0}}
{\color{darkgray}{brk(NULL)                               = 0x5611c6a38000}}
{\color{darkgray}{arch_prctl(0x3001 /* ARCH_??? */, 0x7ffeede990c0) = -1 EINVAL (Invalid argument)}}
{\color{darkgray}{mmap(NULL, 8192, PROT_READ|PROT_WRITE, MAP_PRIVATE|MAP_ANONYMOUS, -1, 0) = 0x7f5c648b8000}}
{\color{darkgray}{access("/etc/ld.so.preload", R_OK)      = -1 ENOENT (No such file or directory)}}
{\color{darkgray}{openat(AT_FDCWD</>, "/etc/ld.so.cache", O_RDONLY|O_CLOEXEC) = 3</etc/ld.so.cache>}}
{\color{darkgray}{newfstatat(3</etc/ld.so.cache>, "", {st_mode=S_IFREG|0644, st_size=135191, ...}, AT_EMPTY_PATH) = 0}}
{\color{darkgray}{mmap(NULL, 135191, PROT_READ, MAP_PRIVATE, 3</etc/ld.so.cache>, 0) = 0x7f5c64896000}}
{\color{darkgray}{close(3</etc/ld.so.cache>)              = 0}}
{\color{darkgray}{openat(AT_FDCWD</>, "/lib/x86_64-linux-gnu/libc.so.6", O_RDONLY|O_CLOEXEC) = 3</usr/lib/x86_64-linux-gnu/libc.so.6>}}
{\color{darkgray}{read(3</usr/lib/x86_64-linux-gnu/libc.so.6>, "\textbackslash 177ELF\textbackslash2\textbackslash1\textbackslash1\textbackslash3\textbackslash0\textbackslash0\textbackslash0\textbackslash0\textbackslash0\textbackslash0\textbackslash0\textbackslash0\textbackslash3\textbackslash0>\textbackslash0\textbackslash1\textbackslash0\textbackslash0\textbackslash0P\textbackslash237\textbackslash2\textbackslash0\textbackslash0\textbackslash0\textbackslash0\textbackslash0"..., 832) = 832}}
{\color{darkgray}{pread64(3</usr/lib/x86_64-linux-gnu/libc.so.6>, "\textbackslash6\textbackslash0\textbackslash0\textbackslash0\textbackslash4\textbackslash0\textbackslash0\textbackslash0@\textbackslash0\textbackslash0\textbackslash0\textbackslash0\textbackslash0\textbackslash0\textbackslash0@\textbackslash0\textbackslash0\textbackslash0\textbackslash0\textbackslash0\textbackslash0\textbackslash0@\textbackslash0\textbackslash0\textbackslash0\textbackslash0\textbackslash0\textbackslash0\textbackslash0"..., 784, 64) = 784}}
{\color{darkgray}{pread64(3</usr/lib/x86_64-linux-gnu/libc.so.6>, "\textbackslash4\textbackslash0\textbackslash0\textbackslash0 \textbackslash0\textbackslash0\textbackslash0\textbackslash5\textbackslash0\textbackslash0\textbackslash0GNU\textbackslash0\textbackslash2\textbackslash0\textbackslash0\textbackslash300\textbackslash4\textbackslash0\textbackslash0\textbackslash0\textbackslash3\textbackslash0\textbackslash0\textbackslash0\textbackslash0\textbackslash0\textbackslash0\textbackslash0"..., 48, 848) = 48}}
{\color{darkgray}{pread64(3</usr/lib/x86_64-linux-gnu/libc.so.6>, "\textbackslash4\textbackslash0\textbackslash0\textbackslash0\textbackslash24\textbackslash0\textbackslash0\textbackslash0\textbackslash3\textbackslash0\textbackslash0\textbackslash0GNU\textbackslash0I\textbackslash17\textbackslash357\textbackslash204\textbackslash3$\textbackslash f\textbackslash221\textbackslash2039x\textbackslash324\textbackslash224\textbackslash323\textbackslash236S"..., 68, 896) = 68}}
{\color{darkgray}{newfstatat(3</usr/lib/x86_64-linux-gnu/libc.so.6>, "", {st_mode=S_IFREG|0755, st_size=2220400, ...}, AT_EMPTY_PATH) = 0}}
{\color{darkgray}{pread64(3</usr/lib/x86_64-linux-gnu/libc.so.6>, "\textbackslash6\textbackslash0\textbackslash0\textbackslash0\textbackslash4\textbackslash0\textbackslash0\textbackslash0@\textbackslash0\textbackslash0\textbackslash0\textbackslash0\textbackslash0\textbackslash0\textbackslash0@\textbackslash0\textbackslash0\textbackslash0\textbackslash0\textbackslash0\textbackslash0\textbackslash0@\textbackslash0\textbackslash0\textbackslash0\textbackslash0\textbackslash0\textbackslash0\textbackslash0"..., 784, 64) = 784}}
{\color{darkgray}{mmap(NULL, 2264656, PROT_READ, MAP_PRIVATE|MAP_DENYWRITE, 3</usr/lib/x86_64-linux-gnu/libc.so.6>, 0) = 0x7f5c6466d000}}
{\color{darkgray}{mprotect(0x7f5c64695000, 2023424, PROT_NONE) = 0}}
{\color{darkgray}{mmap(0x7f5c64695000, 1658880, PROT_READ|PROT_EXEC, MAP_PRIVATE|MAP_FIXED|MAP_DENYWRITE, 3</usr/lib/x86_64-linux-gnu/libc.so.6>, 0x28000) = 0x7f5c64695000}}
{\color{darkgray}{mmap(0x7f5c6482a000, 360448, PROT_READ, MAP_PRIVATE|MAP_FIXED|MAP_DENYWRITE, 3</usr/lib/x86_64-linux-gnu/libc.so.6>, 0x1bd000) = 0x7f5c6482a000}}
{\color{darkgray}{mmap(0x7f5c64883000, 24576, PROT_READ|PROT_WRITE, MAP_PRIVATE|MAP_FIXED|MAP_DENYWRITE, 3</usr/lib/x86_64-linux-gnu/libc.so.6>, 0x215000) = 0x7f5c64883000}}
{\color{darkgray}{mmap(0x7f5c64889000, 52816, PROT_READ|PROT_WRITE, MAP_PRIVATE|MAP_FIXED|MAP_ANONYMOUS, -1, 0) = 0x7f5c64889000}}
{\color{darkgray}{close(3</usr/lib/x86_64-linux-gnu/libc.so.6>) = 0}}
{\color{darkgray}{mmap(NULL, 12288, PROT_READ|PROT_WRITE, MAP_PRIVATE|MAP_ANONYMOUS, -1, 0) = 0x7f5c6466a000}}
{\color{darkgray}{arch_prctl(ARCH_SET_FS, 0x7f5c6466a740) = 0}}
{\color{darkgray}{set_tid_address(0x7f5c6466aa10)         = 23663}}
{\color{darkgray}{set_robust_list(0x7f5c6466aa20, 24)     = 0}}
{\color{darkgray}{rseq(0x7f5c6466b0e0, 0x20, 0, 0x53053053) = 0}}
{\color{darkgray}{mprotect(0x7f5c64883000, 16384, PROT_READ) = 0}}
{\color{darkgray}{mprotect(0x5611c4bde000, 4096, PROT_READ) = 0}}
{\color{darkgray}{mprotect(0x7f5c648f2000, 8192, PROT_READ) = 0}}
{\color{darkgray}{prlimit64(0, RLIMIT_STACK, NULL, {rlim_cur=8192*1024, rlim_max=RLIM64_INFINITY}) = 0}}
{\color{darkgray}{munmap(0x7f5c64896000, 135191)          = 0}}
{\color{darkgray}{getrandom("\textbackslash\textbackslash x7e\textbackslash x74\textbackslash x62\textbackslash xbc\textbackslash x66\textbackslash x05\textbackslash x81\textbackslash xf8", 8, GRND_NONBLOCK) = 8}}
{\color{darkgray}{brk(NULL)                               = 0x5611c6a38000}}
{\color{darkgray}{brk(0x5611c6a59000)                     = 0x5611c6a59000}}
{\color{darkgray}{newfstatat(1</dev/pts/0<char 136:0>>, "", {st_mode=S_IFCHR|0620, st_rdev=makedev(0x88, 0), ...}, AT_EMPTY_PATH) = 0}}
{\color{darkgray}{openat(AT_FDCWD</>, "/dev/null", O_RDONLY) = 3</dev/null<char 1:3>>}}
{\color{darkgray}{newfstatat(3</dev/null<char 1:3>>, "", {st_mode=S_IFCHR|0666, st_rdev=makedev(0x1, 0x3), ...}, AT_EMPTY_PATH) = 0}}
{\color{darkgray}{fadvise64(3</dev/null<char 1:3>>, 0, 0, POSIX_FADV_SEQUENTIAL) = 0}}
{\color{darkgray}{mmap(NULL, 139264, PROT_READ|PROT_WRITE, MAP_PRIVATE|MAP_ANONYMOUS, -1, 0) = 0x7f5c64896000}}
{\color{darkgray}{read(3</dev/null<char 1:3>>, "", 131072) = 0}}
{\color{darkgray}{munmap(0x7f5c64896000, 139264)          = 0}}
{\color{darkgray}{close(3</dev/null<char 1:3>>)           = 0}}
{\color{darkgray}{close(1</dev/pts/0<char 136:0>>)        = 0}}
{\color{darkgray}{close(2</dev/pts/0<char 136:0>>)        = 0}}
{\color{darkgray}{exit_group(0)                           = ?}}
{\color{darkgray}{+++ exited with 0 +++}}
\end{Verbatim}
}}
\endinput  %  ==  ==  ==  ==  ==  ==  ==  ==  ==

\subsection{\strace}
\label{sec:strace}
The \strace \ command is very powerful and the following examples.

		\subsubsection{Trace System Calls To A Given Path}
{\footnotesize{
\begin{Verbatim}[commandchars=\\\{\}]
{\color{darkgray}{root@169e8b2c1ae3:/#}} strace -P /etc/ld.so.cache ls /dev/null 
{\color{darkgray}{openat(AT_FDCWD, "/etc/ld.so.cache", O_RDONLY|O_CLOEXEC) = 3}}
{\color{darkgray}{newfstatat(3, "", {st_mode=S_IFREG|0644, st_size=135191, ...}, AT_EMPTY_PATH) = 0}}
{\color{darkgray}{mmap(NULL, 135191, PROT_READ, MAP_PRIVATE, 3, 0) = 0x7f03bba95000}}
{\color{darkgray}{close(3)                                = 0}}
{\color{darkgray}{/dev/null}}
{\color{darkgray}{+++ exited with 0 +++}}
\end{Verbatim}
}}


		\subsubsection{Inventory time, calls, and errors for every system call}
{\footnotesize{
\begin{Verbatim}[commandchars=\\\{\}]
{\color{darkgray}{root@169e8b2c1ae3:/}}# strace -c ls > /dev/null
{\color{darkgray}{\escapepercent time     seconds  usecs/call     calls    errors syscall}}
{\color{darkgray}{------ ----------- ----------- --------- --------- ----------------}}
{\color{darkgray}{ 71.76    0.013546        6773         2           getdents64}}
{\color{darkgray}{  7.85    0.001482         247         6           openat}}
{\color{darkgray}{  4.88    0.000922         922         1           execve}}
{\color{darkgray}{  4.44    0.000839          49        17           mmap}}
{\color{darkgray}{  1.84    0.000347          43         8           close}}
{\color{darkgray}{  1.48    0.000279          39         7           mprotect}}
{\color{darkgray}{  1.40    0.000265          37         7           newfstatat}}
{\color{darkgray}{  1.26    0.000237          47         5           read}}
{\color{darkgray}{  0.94    0.000178          44         4           pread64}}
{\color{darkgray}{  0.77    0.000145          48         3           brk}}
{\color{darkgray}{  0.57    0.000108          36         3         3 ioctl}}
{\color{darkgray}{  0.49    0.000092          46         2         2 statfs}}
{\color{darkgray}{  0.47    0.000088          44         2         2 access}}
{\color{darkgray}{  0.34    0.000065          32         2         1 arch_prctl}}
{\color{darkgray}{  0.34    0.000065          65         1           getrandom}}
{\color{darkgray}{  0.32    0.000061          61         1           munmap}}
{\color{darkgray}{  0.18    0.000034          34         1           rseq}}
{\color{darkgray}{  0.17    0.000032          32         1           set_robust_list}}
{\color{darkgray}{  0.16    0.000031          31         1           write}}
{\color{darkgray}{  0.16    0.000031          31         1           set_tid_address}}
{\color{darkgray}{  0.16    0.000031          31         1           prlimit64}}
{\color{darkgray}{------ ----------- ----------- --------- --------- ----------------}}
{\color{darkgray}{100.00    0.018878         248        76         8 total}}
\end{Verbatim}
}}

		\subsubsection{Identify Information Associated With File Descriptors}
{\footnotesize{
\begin{Verbatim}[commandchars=\\\{\}]
{\color{darkgray}{root@169e8b2c1ae3:/}}# strace -yy cat /dev/null
{\color{darkgray}{execve("/usr/bin/cat", ["cat", "/dev/null"], 0x7fffb8b235d0 /* 10 vars */) = 0}}
{\color{darkgray}{brk(NULL)                               = 0x5611c6a38000}}
{\color{darkgray}{arch_prctl(0x3001 /* ARCH_??? */, 0x7ffeede990c0) = -1 EINVAL (Invalid argument)}}
{\color{darkgray}{mmap(NULL, 8192, PROT_READ|PROT_WRITE, MAP_PRIVATE|MAP_ANONYMOUS, -1, 0) = 0x7f5c648b8000}}
{\color{darkgray}{access("/etc/ld.so.preload", R_OK)      = -1 ENOENT (No such file or directory)}}
{\color{darkgray}{openat(AT_FDCWD</>, "/etc/ld.so.cache", O_RDONLY|O_CLOEXEC) = 3</etc/ld.so.cache>}}
{\color{darkgray}{newfstatat(3</etc/ld.so.cache>, "", {st_mode=S_IFREG|0644, st_size=135191, ...}, AT_EMPTY_PATH) = 0}}
{\color{darkgray}{mmap(NULL, 135191, PROT_READ, MAP_PRIVATE, 3</etc/ld.so.cache>, 0) = 0x7f5c64896000}}
{\color{darkgray}{close(3</etc/ld.so.cache>)              = 0}}
{\color{darkgray}{openat(AT_FDCWD</>, "/lib/x86_64-linux-gnu/libc.so.6", O_RDONLY|O_CLOEXEC) = 3</usr/lib/x86_64-linux-gnu/libc.so.6>}}
{\color{darkgray}{read(3</usr/lib/x86_64-linux-gnu/libc.so.6>, "\textbackslash 177ELF\textbackslash2\textbackslash1\textbackslash1\textbackslash3\textbackslash0\textbackslash0\textbackslash0\textbackslash0\textbackslash0\textbackslash0\textbackslash0\textbackslash0\textbackslash3\textbackslash0>\textbackslash0\textbackslash1\textbackslash0\textbackslash0\textbackslash0P\textbackslash237\textbackslash2\textbackslash0\textbackslash0\textbackslash0\textbackslash0\textbackslash0"..., 832) = 832}}
{\color{darkgray}{pread64(3</usr/lib/x86_64-linux-gnu/libc.so.6>, "\textbackslash6\textbackslash0\textbackslash0\textbackslash0\textbackslash4\textbackslash0\textbackslash0\textbackslash0@\textbackslash0\textbackslash0\textbackslash0\textbackslash0\textbackslash0\textbackslash0\textbackslash0@\textbackslash0\textbackslash0\textbackslash0\textbackslash0\textbackslash0\textbackslash0\textbackslash0@\textbackslash0\textbackslash0\textbackslash0\textbackslash0\textbackslash0\textbackslash0\textbackslash0"..., 784, 64) = 784}}
{\color{darkgray}{pread64(3</usr/lib/x86_64-linux-gnu/libc.so.6>, "\textbackslash4\textbackslash0\textbackslash0\textbackslash0 \textbackslash0\textbackslash0\textbackslash0\textbackslash5\textbackslash0\textbackslash0\textbackslash0GNU\textbackslash0\textbackslash2\textbackslash0\textbackslash0\textbackslash300\textbackslash4\textbackslash0\textbackslash0\textbackslash0\textbackslash3\textbackslash0\textbackslash0\textbackslash0\textbackslash0\textbackslash0\textbackslash0\textbackslash0"..., 48, 848) = 48}}
{\color{darkgray}{pread64(3</usr/lib/x86_64-linux-gnu/libc.so.6>, "\textbackslash4\textbackslash0\textbackslash0\textbackslash0\textbackslash24\textbackslash0\textbackslash0\textbackslash0\textbackslash3\textbackslash0\textbackslash0\textbackslash0GNU\textbackslash0I\textbackslash17\textbackslash357\textbackslash204\textbackslash3$\textbackslash f\textbackslash221\textbackslash2039x\textbackslash324\textbackslash224\textbackslash323\textbackslash236S"..., 68, 896) = 68}}
{\color{darkgray}{newfstatat(3</usr/lib/x86_64-linux-gnu/libc.so.6>, "", {st_mode=S_IFREG|0755, st_size=2220400, ...}, AT_EMPTY_PATH) = 0}}
{\color{darkgray}{pread64(3</usr/lib/x86_64-linux-gnu/libc.so.6>, "\textbackslash6\textbackslash0\textbackslash0\textbackslash0\textbackslash4\textbackslash0\textbackslash0\textbackslash0@\textbackslash0\textbackslash0\textbackslash0\textbackslash0\textbackslash0\textbackslash0\textbackslash0@\textbackslash0\textbackslash0\textbackslash0\textbackslash0\textbackslash0\textbackslash0\textbackslash0@\textbackslash0\textbackslash0\textbackslash0\textbackslash0\textbackslash0\textbackslash0\textbackslash0"..., 784, 64) = 784}}
{\color{darkgray}{mmap(NULL, 2264656, PROT_READ, MAP_PRIVATE|MAP_DENYWRITE, 3</usr/lib/x86_64-linux-gnu/libc.so.6>, 0) = 0x7f5c6466d000}}
{\color{darkgray}{mprotect(0x7f5c64695000, 2023424, PROT_NONE) = 0}}
{\color{darkgray}{mmap(0x7f5c64695000, 1658880, PROT_READ|PROT_EXEC, MAP_PRIVATE|MAP_FIXED|MAP_DENYWRITE, 3</usr/lib/x86_64-linux-gnu/libc.so.6>, 0x28000) = 0x7f5c64695000}}
{\color{darkgray}{mmap(0x7f5c6482a000, 360448, PROT_READ, MAP_PRIVATE|MAP_FIXED|MAP_DENYWRITE, 3</usr/lib/x86_64-linux-gnu/libc.so.6>, 0x1bd000) = 0x7f5c6482a000}}
{\color{darkgray}{mmap(0x7f5c64883000, 24576, PROT_READ|PROT_WRITE, MAP_PRIVATE|MAP_FIXED|MAP_DENYWRITE, 3</usr/lib/x86_64-linux-gnu/libc.so.6>, 0x215000) = 0x7f5c64883000}}
{\color{darkgray}{mmap(0x7f5c64889000, 52816, PROT_READ|PROT_WRITE, MAP_PRIVATE|MAP_FIXED|MAP_ANONYMOUS, -1, 0) = 0x7f5c64889000}}
{\color{darkgray}{close(3</usr/lib/x86_64-linux-gnu/libc.so.6>) = 0}}
{\color{darkgray}{mmap(NULL, 12288, PROT_READ|PROT_WRITE, MAP_PRIVATE|MAP_ANONYMOUS, -1, 0) = 0x7f5c6466a000}}
{\color{darkgray}{arch_prctl(ARCH_SET_FS, 0x7f5c6466a740) = 0}}
{\color{darkgray}{set_tid_address(0x7f5c6466aa10)         = 23663}}
{\color{darkgray}{set_robust_list(0x7f5c6466aa20, 24)     = 0}}
{\color{darkgray}{rseq(0x7f5c6466b0e0, 0x20, 0, 0x53053053) = 0}}
{\color{darkgray}{mprotect(0x7f5c64883000, 16384, PROT_READ) = 0}}
{\color{darkgray}{mprotect(0x5611c4bde000, 4096, PROT_READ) = 0}}
{\color{darkgray}{mprotect(0x7f5c648f2000, 8192, PROT_READ) = 0}}
{\color{darkgray}{prlimit64(0, RLIMIT_STACK, NULL, {rlim_cur=8192*1024, rlim_max=RLIM64_INFINITY}) = 0}}
{\color{darkgray}{munmap(0x7f5c64896000, 135191)          = 0}}
{\color{darkgray}{getrandom("\textbackslash\textbackslash x7e\textbackslash x74\textbackslash x62\textbackslash xbc\textbackslash x66\textbackslash x05\textbackslash x81\textbackslash xf8", 8, GRND_NONBLOCK) = 8}}
{\color{darkgray}{brk(NULL)                               = 0x5611c6a38000}}
{\color{darkgray}{brk(0x5611c6a59000)                     = 0x5611c6a59000}}
{\color{darkgray}{newfstatat(1</dev/pts/0<char 136:0>>, "", {st_mode=S_IFCHR|0620, st_rdev=makedev(0x88, 0), ...}, AT_EMPTY_PATH) = 0}}
{\color{darkgray}{openat(AT_FDCWD</>, "/dev/null", O_RDONLY) = 3</dev/null<char 1:3>>}}
{\color{darkgray}{newfstatat(3</dev/null<char 1:3>>, "", {st_mode=S_IFCHR|0666, st_rdev=makedev(0x1, 0x3), ...}, AT_EMPTY_PATH) = 0}}
{\color{darkgray}{fadvise64(3</dev/null<char 1:3>>, 0, 0, POSIX_FADV_SEQUENTIAL) = 0}}
{\color{darkgray}{mmap(NULL, 139264, PROT_READ|PROT_WRITE, MAP_PRIVATE|MAP_ANONYMOUS, -1, 0) = 0x7f5c64896000}}
{\color{darkgray}{read(3</dev/null<char 1:3>>, "", 131072) = 0}}
{\color{darkgray}{munmap(0x7f5c64896000, 139264)          = 0}}
{\color{darkgray}{close(3</dev/null<char 1:3>>)           = 0}}
{\color{darkgray}{close(1</dev/pts/0<char 136:0>>)        = 0}}
{\color{darkgray}{close(2</dev/pts/0<char 136:0>>)        = 0}}
{\color{darkgray}{exit_group(0)                           = ?}}
{\color{darkgray}{+++ exited with 0 +++}}
\end{Verbatim}
}}
\endinput  %  ==  ==  ==  ==  ==  ==  ==  ==  ==

\subsection{\strace}
\label{sec:strace}
The \strace \ command is very powerful and the following examples.

		\subsubsection{Trace System Calls To A Given Path}
{\footnotesize{
\begin{Verbatim}[commandchars=\\\{\}]
{\color{darkgray}{root@169e8b2c1ae3:/#}} strace -P /etc/ld.so.cache ls /dev/null 
{\color{darkgray}{openat(AT_FDCWD, "/etc/ld.so.cache", O_RDONLY|O_CLOEXEC) = 3}}
{\color{darkgray}{newfstatat(3, "", {st_mode=S_IFREG|0644, st_size=135191, ...}, AT_EMPTY_PATH) = 0}}
{\color{darkgray}{mmap(NULL, 135191, PROT_READ, MAP_PRIVATE, 3, 0) = 0x7f03bba95000}}
{\color{darkgray}{close(3)                                = 0}}
{\color{darkgray}{/dev/null}}
{\color{darkgray}{+++ exited with 0 +++}}
\end{Verbatim}
}}


		\subsubsection{Inventory time, calls, and errors for every system call}
{\footnotesize{
\begin{Verbatim}[commandchars=\\\{\}]
{\color{darkgray}{root@169e8b2c1ae3:/}}# strace -c ls > /dev/null
{\color{darkgray}{\escapepercent time     seconds  usecs/call     calls    errors syscall}}
{\color{darkgray}{------ ----------- ----------- --------- --------- ----------------}}
{\color{darkgray}{ 71.76    0.013546        6773         2           getdents64}}
{\color{darkgray}{  7.85    0.001482         247         6           openat}}
{\color{darkgray}{  4.88    0.000922         922         1           execve}}
{\color{darkgray}{  4.44    0.000839          49        17           mmap}}
{\color{darkgray}{  1.84    0.000347          43         8           close}}
{\color{darkgray}{  1.48    0.000279          39         7           mprotect}}
{\color{darkgray}{  1.40    0.000265          37         7           newfstatat}}
{\color{darkgray}{  1.26    0.000237          47         5           read}}
{\color{darkgray}{  0.94    0.000178          44         4           pread64}}
{\color{darkgray}{  0.77    0.000145          48         3           brk}}
{\color{darkgray}{  0.57    0.000108          36         3         3 ioctl}}
{\color{darkgray}{  0.49    0.000092          46         2         2 statfs}}
{\color{darkgray}{  0.47    0.000088          44         2         2 access}}
{\color{darkgray}{  0.34    0.000065          32         2         1 arch_prctl}}
{\color{darkgray}{  0.34    0.000065          65         1           getrandom}}
{\color{darkgray}{  0.32    0.000061          61         1           munmap}}
{\color{darkgray}{  0.18    0.000034          34         1           rseq}}
{\color{darkgray}{  0.17    0.000032          32         1           set_robust_list}}
{\color{darkgray}{  0.16    0.000031          31         1           write}}
{\color{darkgray}{  0.16    0.000031          31         1           set_tid_address}}
{\color{darkgray}{  0.16    0.000031          31         1           prlimit64}}
{\color{darkgray}{------ ----------- ----------- --------- --------- ----------------}}
{\color{darkgray}{100.00    0.018878         248        76         8 total}}
\end{Verbatim}
}}

		\subsubsection{Identify Information Associated With File Descriptors}
{\footnotesize{
\begin{Verbatim}[commandchars=\\\{\}]
{\color{darkgray}{root@169e8b2c1ae3:/}}# strace -yy cat /dev/null
{\color{darkgray}{execve("/usr/bin/cat", ["cat", "/dev/null"], 0x7fffb8b235d0 /* 10 vars */) = 0}}
{\color{darkgray}{brk(NULL)                               = 0x5611c6a38000}}
{\color{darkgray}{arch_prctl(0x3001 /* ARCH_??? */, 0x7ffeede990c0) = -1 EINVAL (Invalid argument)}}
{\color{darkgray}{mmap(NULL, 8192, PROT_READ|PROT_WRITE, MAP_PRIVATE|MAP_ANONYMOUS, -1, 0) = 0x7f5c648b8000}}
{\color{darkgray}{access("/etc/ld.so.preload", R_OK)      = -1 ENOENT (No such file or directory)}}
{\color{darkgray}{openat(AT_FDCWD</>, "/etc/ld.so.cache", O_RDONLY|O_CLOEXEC) = 3</etc/ld.so.cache>}}
{\color{darkgray}{newfstatat(3</etc/ld.so.cache>, "", {st_mode=S_IFREG|0644, st_size=135191, ...}, AT_EMPTY_PATH) = 0}}
{\color{darkgray}{mmap(NULL, 135191, PROT_READ, MAP_PRIVATE, 3</etc/ld.so.cache>, 0) = 0x7f5c64896000}}
{\color{darkgray}{close(3</etc/ld.so.cache>)              = 0}}
{\color{darkgray}{openat(AT_FDCWD</>, "/lib/x86_64-linux-gnu/libc.so.6", O_RDONLY|O_CLOEXEC) = 3</usr/lib/x86_64-linux-gnu/libc.so.6>}}
{\color{darkgray}{read(3</usr/lib/x86_64-linux-gnu/libc.so.6>, "\textbackslash 177ELF\textbackslash2\textbackslash1\textbackslash1\textbackslash3\textbackslash0\textbackslash0\textbackslash0\textbackslash0\textbackslash0\textbackslash0\textbackslash0\textbackslash0\textbackslash3\textbackslash0>\textbackslash0\textbackslash1\textbackslash0\textbackslash0\textbackslash0P\textbackslash237\textbackslash2\textbackslash0\textbackslash0\textbackslash0\textbackslash0\textbackslash0"..., 832) = 832}}
{\color{darkgray}{pread64(3</usr/lib/x86_64-linux-gnu/libc.so.6>, "\textbackslash6\textbackslash0\textbackslash0\textbackslash0\textbackslash4\textbackslash0\textbackslash0\textbackslash0@\textbackslash0\textbackslash0\textbackslash0\textbackslash0\textbackslash0\textbackslash0\textbackslash0@\textbackslash0\textbackslash0\textbackslash0\textbackslash0\textbackslash0\textbackslash0\textbackslash0@\textbackslash0\textbackslash0\textbackslash0\textbackslash0\textbackslash0\textbackslash0\textbackslash0"..., 784, 64) = 784}}
{\color{darkgray}{pread64(3</usr/lib/x86_64-linux-gnu/libc.so.6>, "\textbackslash4\textbackslash0\textbackslash0\textbackslash0 \textbackslash0\textbackslash0\textbackslash0\textbackslash5\textbackslash0\textbackslash0\textbackslash0GNU\textbackslash0\textbackslash2\textbackslash0\textbackslash0\textbackslash300\textbackslash4\textbackslash0\textbackslash0\textbackslash0\textbackslash3\textbackslash0\textbackslash0\textbackslash0\textbackslash0\textbackslash0\textbackslash0\textbackslash0"..., 48, 848) = 48}}
{\color{darkgray}{pread64(3</usr/lib/x86_64-linux-gnu/libc.so.6>, "\textbackslash4\textbackslash0\textbackslash0\textbackslash0\textbackslash24\textbackslash0\textbackslash0\textbackslash0\textbackslash3\textbackslash0\textbackslash0\textbackslash0GNU\textbackslash0I\textbackslash17\textbackslash357\textbackslash204\textbackslash3$\textbackslash f\textbackslash221\textbackslash2039x\textbackslash324\textbackslash224\textbackslash323\textbackslash236S"..., 68, 896) = 68}}
{\color{darkgray}{newfstatat(3</usr/lib/x86_64-linux-gnu/libc.so.6>, "", {st_mode=S_IFREG|0755, st_size=2220400, ...}, AT_EMPTY_PATH) = 0}}
{\color{darkgray}{pread64(3</usr/lib/x86_64-linux-gnu/libc.so.6>, "\textbackslash6\textbackslash0\textbackslash0\textbackslash0\textbackslash4\textbackslash0\textbackslash0\textbackslash0@\textbackslash0\textbackslash0\textbackslash0\textbackslash0\textbackslash0\textbackslash0\textbackslash0@\textbackslash0\textbackslash0\textbackslash0\textbackslash0\textbackslash0\textbackslash0\textbackslash0@\textbackslash0\textbackslash0\textbackslash0\textbackslash0\textbackslash0\textbackslash0\textbackslash0"..., 784, 64) = 784}}
{\color{darkgray}{mmap(NULL, 2264656, PROT_READ, MAP_PRIVATE|MAP_DENYWRITE, 3</usr/lib/x86_64-linux-gnu/libc.so.6>, 0) = 0x7f5c6466d000}}
{\color{darkgray}{mprotect(0x7f5c64695000, 2023424, PROT_NONE) = 0}}
{\color{darkgray}{mmap(0x7f5c64695000, 1658880, PROT_READ|PROT_EXEC, MAP_PRIVATE|MAP_FIXED|MAP_DENYWRITE, 3</usr/lib/x86_64-linux-gnu/libc.so.6>, 0x28000) = 0x7f5c64695000}}
{\color{darkgray}{mmap(0x7f5c6482a000, 360448, PROT_READ, MAP_PRIVATE|MAP_FIXED|MAP_DENYWRITE, 3</usr/lib/x86_64-linux-gnu/libc.so.6>, 0x1bd000) = 0x7f5c6482a000}}
{\color{darkgray}{mmap(0x7f5c64883000, 24576, PROT_READ|PROT_WRITE, MAP_PRIVATE|MAP_FIXED|MAP_DENYWRITE, 3</usr/lib/x86_64-linux-gnu/libc.so.6>, 0x215000) = 0x7f5c64883000}}
{\color{darkgray}{mmap(0x7f5c64889000, 52816, PROT_READ|PROT_WRITE, MAP_PRIVATE|MAP_FIXED|MAP_ANONYMOUS, -1, 0) = 0x7f5c64889000}}
{\color{darkgray}{close(3</usr/lib/x86_64-linux-gnu/libc.so.6>) = 0}}
{\color{darkgray}{mmap(NULL, 12288, PROT_READ|PROT_WRITE, MAP_PRIVATE|MAP_ANONYMOUS, -1, 0) = 0x7f5c6466a000}}
{\color{darkgray}{arch_prctl(ARCH_SET_FS, 0x7f5c6466a740) = 0}}
{\color{darkgray}{set_tid_address(0x7f5c6466aa10)         = 23663}}
{\color{darkgray}{set_robust_list(0x7f5c6466aa20, 24)     = 0}}
{\color{darkgray}{rseq(0x7f5c6466b0e0, 0x20, 0, 0x53053053) = 0}}
{\color{darkgray}{mprotect(0x7f5c64883000, 16384, PROT_READ) = 0}}
{\color{darkgray}{mprotect(0x5611c4bde000, 4096, PROT_READ) = 0}}
{\color{darkgray}{mprotect(0x7f5c648f2000, 8192, PROT_READ) = 0}}
{\color{darkgray}{prlimit64(0, RLIMIT_STACK, NULL, {rlim_cur=8192*1024, rlim_max=RLIM64_INFINITY}) = 0}}
{\color{darkgray}{munmap(0x7f5c64896000, 135191)          = 0}}
{\color{darkgray}{getrandom("\textbackslash\textbackslash x7e\textbackslash x74\textbackslash x62\textbackslash xbc\textbackslash x66\textbackslash x05\textbackslash x81\textbackslash xf8", 8, GRND_NONBLOCK) = 8}}
{\color{darkgray}{brk(NULL)                               = 0x5611c6a38000}}
{\color{darkgray}{brk(0x5611c6a59000)                     = 0x5611c6a59000}}
{\color{darkgray}{newfstatat(1</dev/pts/0<char 136:0>>, "", {st_mode=S_IFCHR|0620, st_rdev=makedev(0x88, 0), ...}, AT_EMPTY_PATH) = 0}}
{\color{darkgray}{openat(AT_FDCWD</>, "/dev/null", O_RDONLY) = 3</dev/null<char 1:3>>}}
{\color{darkgray}{newfstatat(3</dev/null<char 1:3>>, "", {st_mode=S_IFCHR|0666, st_rdev=makedev(0x1, 0x3), ...}, AT_EMPTY_PATH) = 0}}
{\color{darkgray}{fadvise64(3</dev/null<char 1:3>>, 0, 0, POSIX_FADV_SEQUENTIAL) = 0}}
{\color{darkgray}{mmap(NULL, 139264, PROT_READ|PROT_WRITE, MAP_PRIVATE|MAP_ANONYMOUS, -1, 0) = 0x7f5c64896000}}
{\color{darkgray}{read(3</dev/null<char 1:3>>, "", 131072) = 0}}
{\color{darkgray}{munmap(0x7f5c64896000, 139264)          = 0}}
{\color{darkgray}{close(3</dev/null<char 1:3>>)           = 0}}
{\color{darkgray}{close(1</dev/pts/0<char 136:0>>)        = 0}}
{\color{darkgray}{close(2</dev/pts/0<char 136:0>>)        = 0}}
{\color{darkgray}{exit_group(0)                           = ?}}
{\color{darkgray}{+++ exited with 0 +++}}
\end{Verbatim}
}}
\endinput  %  ==  ==  ==  ==  ==  ==  ==  ==  ==
		% % % \input{./sections/ssec-strings}

\subsection{\strings}
\label{sec:strings}

Stub for \strings.

\endinput  %  ==  ==  ==  ==  ==  ==  ==  ==  ==


\subsection{\strings}
\label{sec:strings}

Stub for \strings.

\endinput  %  ==  ==  ==  ==  ==  ==  ==  ==  ==


\subsection{\strings}
\label{sec:strings}

Stub for \strings.

\endinput  %  ==  ==  ==  ==  ==  ==  ==  ==  ==


\section{\href{\urlMan}{Manual Pages}}
	% % % \input{./components/man/man-ldd}
\subsection{\refLdd: Print Shared Object Dependencies}

{\tiny{
\begin{lstlisting}[language=bash]
NAME
       ldd - print shared object dependencies
SYNOPSIS
       ldd [option]... file...
DESCRIPTION
       ldd prints the shared objects (shared libraries) required by each
       program or shared object specified on the command line.  An
       example of its use and output is the following:

           $ ldd /bin/ls
               linux-vdso.so.1 (0x00007ffcc3563000)
               libselinux.so.1 => /lib64/libselinux.so.1 (0x00007f87e5459000)
               libcap.so.2 => /lib64/libcap.so.2 (0x00007f87e5254000)
               libc.so.6 => /lib64/libc.so.6 (0x00007f87e4e92000)
               libpcre.so.1 => /lib64/libpcre.so.1 (0x00007f87e4c22000)
               libdl.so.2 => /lib64/libdl.so.2 (0x00007f87e4a1e000)
               /lib64/ld-linux-x86-64.so.2 (0x00005574bf12e000)
               libattr.so.1 => /lib64/libattr.so.1 (0x00007f87e4817000)
               libpthread.so.0 => /lib64/libpthread.so.0 (0x00007f87e45fa000)

       In the usual case, ldd invokes the standard dynamic linker (see
       ld.so(8)) with the LD_TRACE_LOADED_OBJECTS environment variable
       set to 1.  This causes the dynamic linker to inspect the
       program's dynamic dependencies, and find (according to the rules
       described in ld.so(8)) and load the objects that satisfy those
       dependencies.  For each dependency, ldd displays the location of
       the matching object and the (hexadecimal) address at which it is
       loaded.  (The linux-vdso and ld-linux shared dependencies are
       special; see vdso(7) and ld.so(8).)

   Security
       Be aware that in some circumstances (e.g., where the program
       specifies an ELF interpreter other than ld-linux.so), some
       versions of ldd may attempt to obtain the dependency information
       by attempting to directly execute the program, which may lead to
       the execution of whatever code is defined in the program's ELF
       interpreter, and perhaps to execution of the program itself.
       (Before glibc 2.27, the upstream ldd implementation did this for
       example, although most distributions provided a modified version
       that did not.)

       Thus, you should never employ ldd on an untrusted executable,
       since this may result in the execution of arbitrary code.  A
       safer alternative when dealing with untrusted executables is:

           $ objdump -p /path/to/program | grep NEEDED

       Note, however, that this alternative shows only the direct
       dependencies of the executable, while ldd shows the entire
       dependency tree of the executable.
OPTIONS
       --version
              Print the version number of ldd.

       --verbose
       -v     Print all information, including, for example, symbol
              versioning information.

       --unused
       -u     Print unused direct dependencies.  (Since glibc 2.3.4.)

       --data-relocs
       -d     Perform relocations and report any missing objects (ELF
              only).

       --function-relocs
       -r     Perform relocations for both data objects and functions,
              and report any missing objects or functions (ELF only).

       --help Usage information.
BUGS
       ldd does not work on a.out shared libraries.

       ldd does not work with some extremely old a.out programs which
       were built before ldd support was added to the compiler releases.
       If you use ldd on one of these programs, the program will attempt
       to run with argc = 0 and the results will be unpredictable.
SEE ALSO
       pldd(1), sprof(1), ld.so(8), ldconfig(8)
COLOPHON
       This page is part of the man-pages (Linux kernel and C library
       user-space interface documentation) project.  Information about
       the project can be found at 
       https://www.kernel.org/doc/man-pages/.  If you have a bug report
       for this manual page, see
       https://git.kernel.org/pub/scm/docs/man-pages/man-pages.git/tree/CONTRIBUTING.
       This page was obtained from the tarball man-pages-6.9.1.tar.gz
       fetched from
       https://mirrors.edge.kernel.org/pub/linux/docs/man-pages/ on
       2024-06-26.  If you discover any rendering problems in this HTML
       version of the page, or you believe there is a better or more up-
       to-date source for the page, or you have corrections or
       improvements to the information in this COLOPHON (which is not
       part of the original manual page), send a mail to
       man-pages@man7.org

Linux man-pages 6.9.1          2024-05-02                         ldd(1)
\end{lstlisting}
}}
\endinput  %  ==  ==  ==  ==  ==  ==  ==  ==  ==

\subsection{\refLdd: Print Shared Object Dependencies}

{\tiny{
\begin{lstlisting}[language=bash]
NAME
       ldd - print shared object dependencies
SYNOPSIS
       ldd [option]... file...
DESCRIPTION
       ldd prints the shared objects (shared libraries) required by each
       program or shared object specified on the command line.  An
       example of its use and output is the following:

           $ ldd /bin/ls
               linux-vdso.so.1 (0x00007ffcc3563000)
               libselinux.so.1 => /lib64/libselinux.so.1 (0x00007f87e5459000)
               libcap.so.2 => /lib64/libcap.so.2 (0x00007f87e5254000)
               libc.so.6 => /lib64/libc.so.6 (0x00007f87e4e92000)
               libpcre.so.1 => /lib64/libpcre.so.1 (0x00007f87e4c22000)
               libdl.so.2 => /lib64/libdl.so.2 (0x00007f87e4a1e000)
               /lib64/ld-linux-x86-64.so.2 (0x00005574bf12e000)
               libattr.so.1 => /lib64/libattr.so.1 (0x00007f87e4817000)
               libpthread.so.0 => /lib64/libpthread.so.0 (0x00007f87e45fa000)

       In the usual case, ldd invokes the standard dynamic linker (see
       ld.so(8)) with the LD_TRACE_LOADED_OBJECTS environment variable
       set to 1.  This causes the dynamic linker to inspect the
       program's dynamic dependencies, and find (according to the rules
       described in ld.so(8)) and load the objects that satisfy those
       dependencies.  For each dependency, ldd displays the location of
       the matching object and the (hexadecimal) address at which it is
       loaded.  (The linux-vdso and ld-linux shared dependencies are
       special; see vdso(7) and ld.so(8).)

   Security
       Be aware that in some circumstances (e.g., where the program
       specifies an ELF interpreter other than ld-linux.so), some
       versions of ldd may attempt to obtain the dependency information
       by attempting to directly execute the program, which may lead to
       the execution of whatever code is defined in the program's ELF
       interpreter, and perhaps to execution of the program itself.
       (Before glibc 2.27, the upstream ldd implementation did this for
       example, although most distributions provided a modified version
       that did not.)

       Thus, you should never employ ldd on an untrusted executable,
       since this may result in the execution of arbitrary code.  A
       safer alternative when dealing with untrusted executables is:

           $ objdump -p /path/to/program | grep NEEDED

       Note, however, that this alternative shows only the direct
       dependencies of the executable, while ldd shows the entire
       dependency tree of the executable.
OPTIONS
       --version
              Print the version number of ldd.

       --verbose
       -v     Print all information, including, for example, symbol
              versioning information.

       --unused
       -u     Print unused direct dependencies.  (Since glibc 2.3.4.)

       --data-relocs
       -d     Perform relocations and report any missing objects (ELF
              only).

       --function-relocs
       -r     Perform relocations for both data objects and functions,
              and report any missing objects or functions (ELF only).

       --help Usage information.
BUGS
       ldd does not work on a.out shared libraries.

       ldd does not work with some extremely old a.out programs which
       were built before ldd support was added to the compiler releases.
       If you use ldd on one of these programs, the program will attempt
       to run with argc = 0 and the results will be unpredictable.
SEE ALSO
       pldd(1), sprof(1), ld.so(8), ldconfig(8)
COLOPHON
       This page is part of the man-pages (Linux kernel and C library
       user-space interface documentation) project.  Information about
       the project can be found at 
       https://www.kernel.org/doc/man-pages/.  If you have a bug report
       for this manual page, see
       https://git.kernel.org/pub/scm/docs/man-pages/man-pages.git/tree/CONTRIBUTING.
       This page was obtained from the tarball man-pages-6.9.1.tar.gz
       fetched from
       https://mirrors.edge.kernel.org/pub/linux/docs/man-pages/ on
       2024-06-26.  If you discover any rendering problems in this HTML
       version of the page, or you believe there is a better or more up-
       to-date source for the page, or you have corrections or
       improvements to the information in this COLOPHON (which is not
       part of the original manual page), send a mail to
       man-pages@man7.org

Linux man-pages 6.9.1          2024-05-02                         ldd(1)
\end{lstlisting}
}}
\endinput  %  ==  ==  ==  ==  ==  ==  ==  ==  ==

\subsection{\refLdd: Print Shared Object Dependencies}

{\tiny{
\begin{lstlisting}[language=bash]
NAME
       ldd - print shared object dependencies
SYNOPSIS
       ldd [option]... file...
DESCRIPTION
       ldd prints the shared objects (shared libraries) required by each
       program or shared object specified on the command line.  An
       example of its use and output is the following:

           $ ldd /bin/ls
               linux-vdso.so.1 (0x00007ffcc3563000)
               libselinux.so.1 => /lib64/libselinux.so.1 (0x00007f87e5459000)
               libcap.so.2 => /lib64/libcap.so.2 (0x00007f87e5254000)
               libc.so.6 => /lib64/libc.so.6 (0x00007f87e4e92000)
               libpcre.so.1 => /lib64/libpcre.so.1 (0x00007f87e4c22000)
               libdl.so.2 => /lib64/libdl.so.2 (0x00007f87e4a1e000)
               /lib64/ld-linux-x86-64.so.2 (0x00005574bf12e000)
               libattr.so.1 => /lib64/libattr.so.1 (0x00007f87e4817000)
               libpthread.so.0 => /lib64/libpthread.so.0 (0x00007f87e45fa000)

       In the usual case, ldd invokes the standard dynamic linker (see
       ld.so(8)) with the LD_TRACE_LOADED_OBJECTS environment variable
       set to 1.  This causes the dynamic linker to inspect the
       program's dynamic dependencies, and find (according to the rules
       described in ld.so(8)) and load the objects that satisfy those
       dependencies.  For each dependency, ldd displays the location of
       the matching object and the (hexadecimal) address at which it is
       loaded.  (The linux-vdso and ld-linux shared dependencies are
       special; see vdso(7) and ld.so(8).)

   Security
       Be aware that in some circumstances (e.g., where the program
       specifies an ELF interpreter other than ld-linux.so), some
       versions of ldd may attempt to obtain the dependency information
       by attempting to directly execute the program, which may lead to
       the execution of whatever code is defined in the program's ELF
       interpreter, and perhaps to execution of the program itself.
       (Before glibc 2.27, the upstream ldd implementation did this for
       example, although most distributions provided a modified version
       that did not.)

       Thus, you should never employ ldd on an untrusted executable,
       since this may result in the execution of arbitrary code.  A
       safer alternative when dealing with untrusted executables is:

           $ objdump -p /path/to/program | grep NEEDED

       Note, however, that this alternative shows only the direct
       dependencies of the executable, while ldd shows the entire
       dependency tree of the executable.
OPTIONS
       --version
              Print the version number of ldd.

       --verbose
       -v     Print all information, including, for example, symbol
              versioning information.

       --unused
       -u     Print unused direct dependencies.  (Since glibc 2.3.4.)

       --data-relocs
       -d     Perform relocations and report any missing objects (ELF
              only).

       --function-relocs
       -r     Perform relocations for both data objects and functions,
              and report any missing objects or functions (ELF only).

       --help Usage information.
BUGS
       ldd does not work on a.out shared libraries.

       ldd does not work with some extremely old a.out programs which
       were built before ldd support was added to the compiler releases.
       If you use ldd on one of these programs, the program will attempt
       to run with argc = 0 and the results will be unpredictable.
SEE ALSO
       pldd(1), sprof(1), ld.so(8), ldconfig(8)
COLOPHON
       This page is part of the man-pages (Linux kernel and C library
       user-space interface documentation) project.  Information about
       the project can be found at 
       https://www.kernel.org/doc/man-pages/.  If you have a bug report
       for this manual page, see
       https://git.kernel.org/pub/scm/docs/man-pages/man-pages.git/tree/CONTRIBUTING.
       This page was obtained from the tarball man-pages-6.9.1.tar.gz
       fetched from
       https://mirrors.edge.kernel.org/pub/linux/docs/man-pages/ on
       2024-06-26.  If you discover any rendering problems in this HTML
       version of the page, or you believe there is a better or more up-
       to-date source for the page, or you have corrections or
       improvements to the information in this COLOPHON (which is not
       part of the original manual page), send a mail to
       man-pages@man7.org

Linux man-pages 6.9.1          2024-05-02                         ldd(1)
\end{lstlisting}
}}
\endinput  %  ==  ==  ==  ==  ==  ==  ==  ==  ==

	\input{./components/man/man-lddconfig}
	\input{./components/man/man-locate}
	\input{./components/man/man-lsof}
	\input{./components/man/man-objdump}
	\input{./components/man/man-readelf}
	\input{./components/man/man-nm}
	\input{./components/man/man-strace}
	\input{./components/man/man-strings}
	
\end{document} 

\tiny
\scriptsize
\footnotesize
\small
\normalsize
\large
\Large
\huge
\Huge