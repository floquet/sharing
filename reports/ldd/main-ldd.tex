\documentclass[10pt, oneside]{article}   	% use "amsart" instead of "article" for AMSLaTeX format
\usepackage{geometry}                		% See geometry.pdf to learn the layout options. There are lots.
\geometry{letterpaper}                   		% ... or a4paper or a5paper or ... 
%\geometry{landscape}                		% Activate for rotated page geometry
%\usepackage[parfill]{parskip}    		% Activate to begin paragraphs with an empty line rather than an indent
\usepackage{graphicx}				% Use pdf, png, jpg, or eps§ with pdflatex; use eps in DVI mode
								% TeX will automatically convert eps --> pdf in pdflatex		
\usepackage{amssymb}
\usepackage{fancyvrb}
\usepackage{hyperref}
\usepackage{overpic}
\usepackage{xcolor}

\newcommand{\escapepercent}{\%}

\usepackage{listings}
	\definecolor{textblue}{rgb}{.2,.2,.7}
	\definecolor{textred}{rgb}{0.54,0,0}
	\definecolor{textgreen}{rgb}{0,0.43,0}

% input{./setup/macros}

% macros to simplfy typing and make the product more reliable
\newcommand{\emailTopa}[0]			{\href{mailto:daniel.topa@hii-tsd.com}{daniel.topa@hii-tsd.com}}

\newcommand{\textt}[1]				{{\footnotesize{\texttt{#1}}}}

\newcommand{\urlMan}[0]				{https://man7.org/linux/man-pages/man1/}

\newcommand{\ldd}[0]				{\textt{ldd}}
\newcommand{\urlLdd}[0]				{\urlMan ldd.1.html}
\newcommand{\refLdd}[0]				{\href{\urlLdd}{\ldd}}

\newcommand{\lddconfig}[0]			{\textt{lddconfig}}
\newcommand{\urlLddconfig}[0]			{\urlMan lddconfig.1.html}
\newcommand{\refLddconfig}[0]		{\href{\urlLddconfig}{\lddconfig}}

\newcommand{\lsof}[0]				{\textt{lsof}}
\newcommand{\urlLsof}[0]				{\urlMan lsof.1.html}
\newcommand{\refLsof}[0]				{\href{\urlLsof}{\lsof}}

\newcommand{\locate}[0]				{\textt{locate}}
\newcommand{\urlLocate}[0]			{\urlMan locate.1.html}
\newcommand{\refLocate}[0]			{\href{\urlLocate}{\locate}}

\newcommand{\nm}[0]				{\textt{nm}}
\newcommand{\urlNm}[0]				{\urlMan nm.1.html}
\newcommand{\refNm}[0]				{\href{\urlNm}{\nm}}

\newcommand{\objdump}[0]			{\textt{objdump}}
\newcommand{\urlObjdump}[0]			{\urlMan objdump.1.html}
\newcommand{\refObjdump}[0]			{\href{\urlObjdump}{\objdump}}

\newcommand{\readelf}[0]				{\textt{readelf}}
\newcommand{\urlReadelf}[0]			{\urlMan readelf.1.html}
\newcommand{\refReadelf}[0]			{\href{\urlReadelf}{\readelf}}

\newcommand{\strace}[0]				{\textt{strace}}
\newcommand{\urlStrace}[0]			{\urlMan strace.1.html}
\newcommand{\refStrace}[0]			{\href{\urlStrace}{\strace}}

\newcommand{\strings}[0]				{\textt{strings}}
\newcommand{\urlStrings}[0]			{\urlMan strings.1.html}
\newcommand{\refStrings}[0]			{\href{\urlStrings}{\strings}}

\newcommand{\elf}[0]				{\href{https://en.wikipedia.org/wiki/Executable_and_Linkable_Format}{ELF}}


\endinput  %  ==  ==  ==  ==  ==  ==  ==  ==  ==


% input{./setup/listing-bash}
% https://tex.stackexchange.com/questions/310335/using-bash-listings-to-bold-variables-and-functions

\usepackage[T1]{fontenc}
\usepackage{
  color,
  beramono,
  listings,
  textcomp
}

\definecolor{lightgray}{RGB}{245,245,245}
\definecolor{darkgray}{RGB}{128,128,128}

\lstset{
  abovecaptionskip={0cm},
  backgroundcolor={\color{lightgray}},
  basicstyle={\small\ttfamily},
  breakatwhitespace=true,
  breaklines=true,
  captionpos=b,
  frame=tb,
  resetmargins=true,
  sensitive=true,
  stepnumber=1,
  tabsize=4,
  upquote=true
}

\AtBeginDocument{\lstdefinelanguage{bash}[]{sh}%
  {morekeywords={alias,bg,bind,builtin,caller,command,compgen,compopt,%
      complete,coproc,curl,declare,disown,dirs,enable,fc,fg,help,%
      history,jobs,let,local,logout,mapfile,printf,pushd,popd,%
      readarray,select,set,suspend,shopt,source,times,type,typeset,%
      ulimit,unalias,wait},%
   otherkeywords={ [, ], [[, ]], \{, \} }%
  }%

\lstdefinelanguage{sh}%
  {morekeywords={awk,break,case,cat,cd,continue,do,done,echo,elif,else,%
      env,esac,eval,exec,exit,export,expr,false,fi,for,function,getopts,%
      hash,history,if,in,kill,login,newgrp,nice,nohup,ps,pwd,read,%
      readonly,return,set,sed,shift,test,then,times,trap,true,type,%
      ulimit,umask,unset,until,wait,while},%
   morecomment=[l]\#,%
   morestring=[d]",%
   alsoletter={*"'0123456789.},%
   alsoother={\{\=\}},%
   literate={{=}{{{=}}}1},%
   literate={\$\{}{{{{\bfseries{}\$\{}}}}2,%
   otherkeywords={ [, ], \{, \} }%
  }[keywords,comments,strings]%
}

\endinput  %  ==  ==  ==  ==  ==  ==  ==  ==  ==



\title{Unix Tools for Probing Executable Files}
\author{Daniel Topa\\HII-TSD\\\href{mailto:daniel.topa@hii-tsd.com}{daniel.topa@hii-tsd.com}}

\begin{document}
\maketitle
\abstract{This article surveys Unix tools for the exploration of executable files, some of which depend upon the application being compiled with debug information. The \textt{man}ual pages are included, making this document usefull in siloed computing networks.}
\tableofcontents

\section{Overview}
Here are several Unix commands for probing executable files. The following section shows sample useage for each command and the final section contains the information from the \textt{man}ual page.
\begin{enumerate}
	\item gdb
	\item \hyperref[sec:ldd]{\ldd}
	\item \hyperref[sec:ldd]{\lddconfig}
	\item \hyperref[sec:locate]{\locate}
	\item \hyperref[sec:objdump]{\objdump}
	\item \hyperref[sec:lsof]{\lsof}
	\item \hyperref[sec:readelf]{\readelf}
	\item \hyperref[sec:nm]{\nm}
	\item \hyperref[sec:strace]{\strace}
	\item \hyperref[sec:strings]{\strings}
\end{enumerate}

The goal is to be able to resolve the workings of an executable file exploiting the \elf \ structure show in figures \ref{fig:elf}. The next figure, \ref{fig:elf-II}, shows the relationship between source files, header files, shared objects, and the executable program.

\begin{figure}[htbp] %  figure placement: here, top, bottom, or page
\centering
	\href{https://camo.githubusercontent.com/00cd4e64df02caf11e9c7c8f67a4d7e9470ea03c244e6d5bce8444a674b9143c/68747470733a2f2f692e696d6775722e636f6d2f4169394f714f422e706e67}{
	\begin{overpic}[ scale = 0.4 ]
		{./local/graphics-local/elf-01}
		%\put(50, 50)	{\colorbox{white}{$a+b$}}
	\end{overpic}
	}
\caption{The structure of a Unix \elf \ file.}
\label{fig:elf}
\end{figure}

\begin{figure}[htbp] %  figure placement: here, top, bottom, or page
\centering
	\href{https://camo.githubusercontent.com/94b1128b885c29e21c64fb3b247d0184c54f4248e4195462bd15671003afc319/68747470733a2f2f692e696d6775722e636f6d2f4c4e6464546d6b2e706e67}{
	\begin{overpic}[ scale = 0.5 ]
		{./local/graphics-local/elf-02}
		%\put(50, 50)	{\colorbox{white}{$a+b$}}
	\end{overpic}
	}
\caption{Connecting source files, object files, libraries, and bindary executables.}
\label{fig:elf-II}
\end{figure}

\section{Command Examples}
		% % % \input{./sections/ssec-ldd}

\subsection{\ldd}
\label{sec:ldd}
The command \refLdd \ prints shared object dependencies, in this example, for the executable \texttt{bash}:
{\footnotesize{
\begin{Verbatim}[commandchars=\\\{\}]
{\color{darkgray}{root@69cb14a32689:/}}# ldd /bin/bash
{\color{darkgray}{	linux-vdso.so.1 (0x00007ffe64317000)}}
{\color{darkgray}{	libtinfo.so.6 }{\color{blue}{=>}}\color{darkgray}{ /lib/x86_64-linux-gnu/libtinfo.so.6 (0x00007f842112d000)}}
{\color{darkgray}{	libc.so.6 }{\color{blue}{=>}}\color{darkgray}{ /lib/x86_64-linux-gnu/libc.so.6 (0x00007f8420f04000)}}
{\color{darkgray}{	/lib64/ld-linux-x86-64.so.2 (0x00007f84212e3000)}}
\end{Verbatim}
}}
\href{https://en.wikipedia.org/wiki/Symbolic_link}{Symbolic link}s (symlinks) are highlighted with blue color.

\endinput  %  ==  ==  ==  ==  ==  ==  ==  ==  ==


\subsection{\ldd}
\label{sec:ldd}
The command \refLdd \ prints shared object dependencies, in this example, for the executable \texttt{bash}:
{\footnotesize{
\begin{Verbatim}[commandchars=\\\{\}]
{\color{darkgray}{root@69cb14a32689:/}}# ldd /bin/bash
{\color{darkgray}{	linux-vdso.so.1 (0x00007ffe64317000)}}
{\color{darkgray}{	libtinfo.so.6 }{\color{blue}{=>}}\color{darkgray}{ /lib/x86_64-linux-gnu/libtinfo.so.6 (0x00007f842112d000)}}
{\color{darkgray}{	libc.so.6 }{\color{blue}{=>}}\color{darkgray}{ /lib/x86_64-linux-gnu/libc.so.6 (0x00007f8420f04000)}}
{\color{darkgray}{	/lib64/ld-linux-x86-64.so.2 (0x00007f84212e3000)}}
\end{Verbatim}
}}
\href{https://en.wikipedia.org/wiki/Symbolic_link}{Symbolic link}s (symlinks) are highlighted with blue color.

\endinput  %  ==  ==  ==  ==  ==  ==  ==  ==  ==


\subsection{\ldd}
\label{sec:ldd}
The command \refLdd \ prints shared object dependencies, in this example, for the executable \texttt{bash}:
{\footnotesize{
\begin{Verbatim}[commandchars=\\\{\}]
{\color{darkgray}{root@69cb14a32689:/}}# ldd /bin/bash
{\color{darkgray}{	linux-vdso.so.1 (0x00007ffe64317000)}}
{\color{darkgray}{	libtinfo.so.6 }{\color{blue}{=>}}\color{darkgray}{ /lib/x86_64-linux-gnu/libtinfo.so.6 (0x00007f842112d000)}}
{\color{darkgray}{	libc.so.6 }{\color{blue}{=>}}\color{darkgray}{ /lib/x86_64-linux-gnu/libc.so.6 (0x00007f8420f04000)}}
{\color{darkgray}{	/lib64/ld-linux-x86-64.so.2 (0x00007f84212e3000)}}
\end{Verbatim}
}}
\href{https://en.wikipedia.org/wiki/Symbolic_link}{Symbolic link}s (symlinks) are highlighted with blue color.

\endinput  %  ==  ==  ==  ==  ==  ==  ==  ==  ==

		% % % \input{./sections/ssec-nm}

\subsection{\nm}
\label{sec:nm}

The \refNm \ command shows dependent shared objects and executables; 

\endinput  %  ==  ==  ==  ==  ==  ==  ==  ==  ==


\subsection{\nm}
\label{sec:nm}

The \refNm \ command shows dependent shared objects and executables; 

\endinput  %  ==  ==  ==  ==  ==  ==  ==  ==  ==


\subsection{\lddconfig}
\label{sec:lddconfig}

Stub for \refLddconfig \  In \textt{/sbin/lddconfig}. Configure dynamic linker run-time bindings.

\endinput  %  ==  ==  ==  ==  ==  ==  ==  ==  ==

		% % % \input{./sections/ssec-locate}

\subsection{\locate}
\label{sec:locate}
The \refLocate \ \href{https://en.wikipedia.org/wiki/Locate_(Unix)}{command} lists files in a prebuilt database of files generated by the \textt{updatedb} command or by a daemon and compressed using incremental encoding.
{\footnotesize{
\begin{Verbatim}[commandchars=\\\{\}]
{\color{darkgray}{dantopa@92bc4c447e32:/}}$ locate libc.so.6
{\color{darkgray}{/usr/lib/x86_64-linux-gnu/libc.so.6}}
{\color{darkgray}{/usr/lib32/libc.so.6}}
\end{Verbatim}
}}

\endinput  %  ==  ==  ==  ==  ==  ==  ==  ==  ==


\subsection{\locate}
\label{sec:locate}
The \refLocate \ \href{https://en.wikipedia.org/wiki/Locate_(Unix)}{command} lists files in a prebuilt database of files generated by the \textt{updatedb} command or by a daemon and compressed using incremental encoding.
{\footnotesize{
\begin{Verbatim}[commandchars=\\\{\}]
{\color{darkgray}{dantopa@92bc4c447e32:/}}$ locate libc.so.6
{\color{darkgray}{/usr/lib/x86_64-linux-gnu/libc.so.6}}
{\color{darkgray}{/usr/lib32/libc.so.6}}
\end{Verbatim}
}}

\endinput  %  ==  ==  ==  ==  ==  ==  ==  ==  ==


\subsection{\locate}
\label{sec:locate}
The \refLocate \ \href{https://en.wikipedia.org/wiki/Locate_(Unix)}{command} lists files in a prebuilt database of files generated by the \textt{updatedb} command or by a daemon and compressed using incremental encoding.
{\footnotesize{
\begin{Verbatim}[commandchars=\\\{\}]
{\color{darkgray}{dantopa@92bc4c447e32:/}}$ locate libc.so.6
{\color{darkgray}{/usr/lib/x86_64-linux-gnu/libc.so.6}}
{\color{darkgray}{/usr/lib32/libc.so.6}}
\end{Verbatim}
}}

\endinput  %  ==  ==  ==  ==  ==  ==  ==  ==  ==

		% % % \input{./sections/ssec-losf}

\subsection{\lsof}
\label{sec:lsof}
This command does an \textt{ls} on open files. The example show how to query both a user and a process id (\textt{pid}).
		\subsubsection{\lsof \ on Process ID}
The \refLsof \ command shows open files, here for the bash process with PID = 10932:
{\footnotesize{
\begin{Verbatim}[commandchars=\\\{\}]
{\color{darkgray}{dantopa@92bc4c447e32:~}}$ ps
{\color{darkgray}{  PID TTY          TIME CMD}}
{\color{darkgray}{10932 pts/1    00:00:00 bash}}
{\color{darkgray}{11152 pts/1    00:00:00 ps}}
{\color{darkgray}{dantopa@92bc4c447e32:~}}$ lsof -p 10932
{\color{darkgray}{COMMAND   PID    USER   FD   TYPE DEVICE SIZE/OFF     NODE NAME}}
{\color{darkgray}{bash    10932 dantopa  cwd    DIR   0,71     4096  6820049 /home/dantopa}}
{\color{darkgray}{bash    10932 dantopa  rtd    DIR   0,71     4096 61653409 /}}
{\color{darkgray}{bash    10932 dantopa  txt    REG   0,71  1396520 62702252 /usr/bin/bash}}
{\color{darkgray}{bash    10932 dantopa  mem    REG  254,1          62702252 /usr/bin/bash (path dev=0,71)}}
{\color{darkgray}{bash    10932 dantopa  mem    REG  254,1          63095938 /usr/lib/x86_64-linux-gnu/libc.so.6 (path dev=0,71)}}
{\color{darkgray}{bash    10932 dantopa  mem    REG  254,1           1190606 /usr/lib/x86_64-linux-gnu/libtinfo.so.6.3 (path dev=0,71)}}
{\color{darkgray}{bash    10932 dantopa  mem    REG  254,1          63095935 /usr/lib/x86_64-linux-gnu/ld-linux-x86-64.so.2 (path dev=0,71)}}
{\color{darkgray}{bash    10932 dantopa    0u   CHR  136,1      0t0        4 /dev/pts/1}}
{\color{darkgray}{bash    10932 dantopa    1u   CHR  136,1      0t0        4 /dev/pts/1}}
{\color{darkgray}{bash    10932 dantopa    2u   CHR  136,1      0t0        4 /dev/pts/1}}
{\color{darkgray}{bash    10932 dantopa  255u   CHR  136,1      0t0        4 /dev/pts/1}}
\end{Verbatim}
}
		\subsubsection{\lsof \ on User}
These are open files for user \textt{dantopa}:
{\footnotesize{
\begin{Verbatim}[commandchars=\\\{\}]
{\color{darkgray}{dantopa@92bc4c447e32:~}}$ lsof -u dantopa
{\color{darkgray}{COMMAND   PID    USER   FD   TYPE DEVICE SIZE/OFF     NODE NAME}}
{\color{darkgray}{bash    10921 dantopa  cwd    DIR   0,71     4096 61653409 /}}
{\color{darkgray}{bash    10921 dantopa  rtd    DIR   0,71     4096 61653409 /}}
{\color{darkgray}{bash    10921 dantopa  txt    REG   0,71  1396520 62702252 /usr/bin/bash}}
{\color{darkgray}{bash    10921 dantopa  mem    REG  254,1          62702252 /usr/bin/bash (path dev=0,71)}}
{\color{darkgray}{bash    10921 dantopa  mem    REG  254,1          63095938 /usr/lib/x86_64-linux-gnu/libc.so.6 (path dev=0,71)}}
{\color{darkgray}{bash    10921 dantopa  mem    REG  254,1           1190606 /usr/lib/x86_64-linux-gnu/libtinfo.so.6.3 (path dev=0,71)}}
{\color{darkgray}{bash    10921 dantopa  mem    REG  254,1          63095935 /usr/lib/x86_64-linux-gnu/ld-linux-x86-64.so.2 (path dev=0,71)}}
{\color{darkgray}{bash    10921 dantopa    0u   CHR  136,0      0t0        3 /dev/pts/0}}
{\color{darkgray}{bash    10921 dantopa    1u   CHR  136,0      0t0        3 /dev/pts/0}}
{\color{darkgray}{bash    10921 dantopa    2u   CHR  136,0      0t0        3 /dev/pts/0}}
{\color{darkgray}{bash    10921 dantopa  255u   CHR  136,0      0t0        3 /dev/pts/0}}
{\color{darkgray}{bash    10932 dantopa  cwd    DIR   0,33      704     1572 /repos/github/vault-fortran/Xmodern-fortran/tau/apex}}
{\color{darkgray}{bash    10932 dantopa  rtd    DIR   0,71     4096 61653409 /}}
{\color{darkgray}{bash    10932 dantopa  txt    REG   0,71  1396520 62702252 /usr/bin/bash}}
{\color{darkgray}{bash    10932 dantopa  mem    REG  254,1          62702252 /usr/bin/bash (path dev=0,71)}}
{\color{darkgray}{bash    10932 dantopa  mem    REG  254,1          63095938 /usr/lib/x86_64-linux-gnu/libc.so.6 (path dev=0,71)}}
{\color{darkgray}{bash    10932 dantopa  mem    REG  254,1           1190606 /usr/lib/x86_64-linux-gnu/libtinfo.so.6.3 (path dev=0,71)}}
{\color{darkgray}{bash    10932 dantopa  mem    REG  254,1          63095935 /usr/lib/x86_64-linux-gnu/ld-linux-x86-64.so.2 (path dev=0,71)}}
{\color{darkgray}{bash    10932 dantopa    0u   CHR  136,1      0t0        4 /dev/pts/1}}
{\color{darkgray}{bash    10932 dantopa    1u   CHR  136,1      0t0        4 /dev/pts/1}}
{\color{darkgray}{bash    10932 dantopa    2u   CHR  136,1      0t0        4 /dev/pts/1}}
{\color{darkgray}{bash    10932 dantopa  255u   CHR  136,1      0t0        4 /dev/pts/1}}
{\color{darkgray}{lsof    11139 dantopa  cwd    DIR   0,33      704     1572 /repos/github/vault-fortran/Xmodern-fortran/tau/apex}}
{\color{darkgray}{lsof    11139 dantopa  rtd    DIR   0,71     4096 61653409 /}}
{\color{darkgray}{lsof    11139 dantopa  txt    REG   0,71   167544   709329 /usr/bin/lsof}}
{\color{darkgray}{lsof    11139 dantopa  mem    REG  254,1            709329 /usr/bin/lsof (path dev=0,71)}}
{\color{darkgray}{lsof    11139 dantopa  mem    REG  254,1          63095951 /usr/lib/x86_64-linux-gnu/libresolv.so.2 (path dev=0,71)}}
{\color{darkgray}{lsof    11139 dantopa  mem    REG  254,1           1190531 /usr/lib/x86_64-linux-gnu/libkeyutils.so.1.9 (path dev=0,71)}}
{\color{darkgray}{lsof    11139 dantopa  mem    REG  254,1          63096020 /usr/lib/x86_64-linux-gnu/libkrb5support.so.0.1 (path dev=0,71)}}
{\color{darkgray}{lsof    11139 dantopa  mem    REG  254,1          63096026 /usr/lib/x86_64-linux-gnu/libcom_err.so.2.1 (path dev=0,71)}}
{\color{darkgray}{lsof    11139 dantopa  mem    REG  254,1          63096018 /usr/lib/x86_64-linux-gnu/libk5crypto.so.3.1 (path dev=0,71)}}
{\color{darkgray}{lsof    11139 dantopa  mem    REG  254,1          63096022 /usr/lib/x86_64-linux-gnu/libkrb5.so.3.3 (path dev=0,71)}}
{\color{darkgray}{lsof    11139 dantopa  mem    REG  254,1           1190578 /usr/lib/x86_64-linux-gnu/libpcre2-8.so.0.10.4 (path dev=0,71)}}
{\color{darkgray}{lsof    11139 dantopa  mem    REG  254,1          63096024 /usr/lib/x86_64-linux-gnu/libgssapi_krb5.so.2.2 (path dev=0,71)}}
{\color{darkgray}{lsof    11139 dantopa  mem    REG  254,1          63095938 /usr/lib/x86_64-linux-gnu/libc.so.6 (path dev=0,71)}}
{\color{darkgray}{lsof    11139 dantopa  mem    REG  254,1           1190588 /usr/lib/x86_64-linux-gnu/libselinux.so.1 (path dev=0,71)}}
{\color{darkgray}{lsof    11139 dantopa  mem    REG  254,1           1190608 /usr/lib/x86_64-linux-gnu/libtirpc.so.3.0.0 (path dev=0,71)}}
{\color{darkgray}{lsof    11139 dantopa  mem    REG  254,1          63095935 /usr/lib/x86_64-linux-gnu/ld-linux-x86-64.so.2 (path dev=0,71)}}
{\color{darkgray}{lsof    11139 dantopa    0u   CHR  136,1      0t0        4 /dev/pts/1}}
{\color{darkgray}{lsof    11139 dantopa    1u   CHR  136,1      0t0        4 /dev/pts/1}}
{\color{darkgray}{lsof    11139 dantopa    2u   CHR  136,1      0t0        4 /dev/pts/1}}
{\color{darkgray}{lsof    11139 dantopa    3r   DIR   0,74        0        1 /proc}}
{\color{darkgray}{lsof    11139 dantopa    4r   DIR   0,74        7   123326 /proc/11139/fd}}
{\color{darkgray}{lsof    11139 dantopa    5w  FIFO   0,11      0t0   123331 pipe}}
{\color{darkgray}{lsof    11139 dantopa    6r  FIFO   0,11      0t0   123332 pipe}}
{\color{darkgray}{lsof    11140 dantopa  cwd    DIR   0,33      704     1572 /repos/github/vault-fortran/Xmodern-fortran/tau/apex}}
{\color{darkgray}{lsof    11140 dantopa  rtd    DIR   0,71     4096 61653409 /}}
{\color{darkgray}{lsof    11140 dantopa  txt    REG   0,71   167544   709329 /usr/bin/lsof}}
{\color{darkgray}{lsof    11140 dantopa  mem    REG  254,1            709329 /usr/bin/lsof (path dev=0,71)}}
{\color{darkgray}{lsof    11140 dantopa  mem    REG  254,1          63095951 /usr/lib/x86_64-linux-gnu/libresolv.so.2 (path dev=0,71)}}
{\color{darkgray}{lsof    11140 dantopa  mem    REG  254,1           1190531 /usr/lib/x86_64-linux-gnu/libkeyutils.so.1.9 (path dev=0,71)}}
{\color{darkgray}{lsof    11140 dantopa  mem    REG  254,1          63096020 /usr/lib/x86_64-linux-gnu/libkrb5support.so.0.1 (path dev=0,71)}}
{\color{darkgray}{lsof    11140 dantopa  mem    REG  254,1          63096026 /usr/lib/x86_64-linux-gnu/libcom_err.so.2.1 (path dev=0,71)}}
{\color{darkgray}{lsof    11140 dantopa  mem    REG  254,1          63096018 /usr/lib/x86_64-linux-gnu/libk5crypto.so.3.1 (path dev=0,71)}}
{\color{darkgray}{lsof    11140 dantopa  mem    REG  254,1          63096022 /usr/lib/x86_64-linux-gnu/libkrb5.so.3.3 (path dev=0,71)}}
{\color{darkgray}{lsof    11140 dantopa  mem    REG  254,1           1190578 /usr/lib/x86_64-linux-gnu/libpcre2-8.so.0.10.4 (path dev=0,71)}}
{\color{darkgray}{lsof    11140 dantopa  mem    REG  254,1          63096024 /usr/lib/x86_64-linux-gnu/libgssapi_krb5.so.2.2 (path dev=0,71)}}
{\color{darkgray}{lsof    11140 dantopa  mem    REG  254,1          63095938 /usr/lib/x86_64-linux-gnu/libc.so.6 (path dev=0,71)}}
{\color{darkgray}{lsof    11140 dantopa  mem    REG  254,1           1190588 /usr/lib/x86_64-linux-gnu/libselinux.so.1 (path dev=0,71)}}
{\color{darkgray}{lsof    11140 dantopa  mem    REG  254,1           1190608 /usr/lib/x86_64-linux-gnu/libtirpc.so.3.0.0 (path dev=0,71)}}
{\color{darkgray}{lsof    11140 dantopa  mem    REG  254,1          63095935 /usr/lib/x86_64-linux-gnu/ld-linux-x86-64.so.2 (path dev=0,71)}}
{\color{darkgray}{lsof    11140 dantopa    4r  FIFO   0,11      0t0   123331 pipe}}
{\color{darkgray}{lsof    11140 dantopa    7w  FIFO   0,11      0t0   123332 pipe}}
\end{Verbatim}
}}
\endinput  %  ==  ==  ==  ==  ==  ==  ==  ==  ==


\subsection{\lsof}
\label{sec:lsof}
This command does an \textt{ls} on open files. The example show how to query both a user and a process id (\textt{pid}).
		\subsubsection{\lsof \ on Process ID}
The \refLsof \ command shows open files, here for the bash process with PID = 10932:
{\footnotesize{
\begin{Verbatim}[commandchars=\\\{\}]
{\color{darkgray}{dantopa@92bc4c447e32:~}}$ ps
{\color{darkgray}{  PID TTY          TIME CMD}}
{\color{darkgray}{10932 pts/1    00:00:00 bash}}
{\color{darkgray}{11152 pts/1    00:00:00 ps}}
{\color{darkgray}{dantopa@92bc4c447e32:~}}$ lsof -p 10932
{\color{darkgray}{COMMAND   PID    USER   FD   TYPE DEVICE SIZE/OFF     NODE NAME}}
{\color{darkgray}{bash    10932 dantopa  cwd    DIR   0,71     4096  6820049 /home/dantopa}}
{\color{darkgray}{bash    10932 dantopa  rtd    DIR   0,71     4096 61653409 /}}
{\color{darkgray}{bash    10932 dantopa  txt    REG   0,71  1396520 62702252 /usr/bin/bash}}
{\color{darkgray}{bash    10932 dantopa  mem    REG  254,1          62702252 /usr/bin/bash (path dev=0,71)}}
{\color{darkgray}{bash    10932 dantopa  mem    REG  254,1          63095938 /usr/lib/x86_64-linux-gnu/libc.so.6 (path dev=0,71)}}
{\color{darkgray}{bash    10932 dantopa  mem    REG  254,1           1190606 /usr/lib/x86_64-linux-gnu/libtinfo.so.6.3 (path dev=0,71)}}
{\color{darkgray}{bash    10932 dantopa  mem    REG  254,1          63095935 /usr/lib/x86_64-linux-gnu/ld-linux-x86-64.so.2 (path dev=0,71)}}
{\color{darkgray}{bash    10932 dantopa    0u   CHR  136,1      0t0        4 /dev/pts/1}}
{\color{darkgray}{bash    10932 dantopa    1u   CHR  136,1      0t0        4 /dev/pts/1}}
{\color{darkgray}{bash    10932 dantopa    2u   CHR  136,1      0t0        4 /dev/pts/1}}
{\color{darkgray}{bash    10932 dantopa  255u   CHR  136,1      0t0        4 /dev/pts/1}}
\end{Verbatim}
}
		\subsubsection{\lsof \ on User}
These are open files for user \textt{dantopa}:
{\footnotesize{
\begin{Verbatim}[commandchars=\\\{\}]
{\color{darkgray}{dantopa@92bc4c447e32:~}}$ lsof -u dantopa
{\color{darkgray}{COMMAND   PID    USER   FD   TYPE DEVICE SIZE/OFF     NODE NAME}}
{\color{darkgray}{bash    10921 dantopa  cwd    DIR   0,71     4096 61653409 /}}
{\color{darkgray}{bash    10921 dantopa  rtd    DIR   0,71     4096 61653409 /}}
{\color{darkgray}{bash    10921 dantopa  txt    REG   0,71  1396520 62702252 /usr/bin/bash}}
{\color{darkgray}{bash    10921 dantopa  mem    REG  254,1          62702252 /usr/bin/bash (path dev=0,71)}}
{\color{darkgray}{bash    10921 dantopa  mem    REG  254,1          63095938 /usr/lib/x86_64-linux-gnu/libc.so.6 (path dev=0,71)}}
{\color{darkgray}{bash    10921 dantopa  mem    REG  254,1           1190606 /usr/lib/x86_64-linux-gnu/libtinfo.so.6.3 (path dev=0,71)}}
{\color{darkgray}{bash    10921 dantopa  mem    REG  254,1          63095935 /usr/lib/x86_64-linux-gnu/ld-linux-x86-64.so.2 (path dev=0,71)}}
{\color{darkgray}{bash    10921 dantopa    0u   CHR  136,0      0t0        3 /dev/pts/0}}
{\color{darkgray}{bash    10921 dantopa    1u   CHR  136,0      0t0        3 /dev/pts/0}}
{\color{darkgray}{bash    10921 dantopa    2u   CHR  136,0      0t0        3 /dev/pts/0}}
{\color{darkgray}{bash    10921 dantopa  255u   CHR  136,0      0t0        3 /dev/pts/0}}
{\color{darkgray}{bash    10932 dantopa  cwd    DIR   0,33      704     1572 /repos/github/vault-fortran/Xmodern-fortran/tau/apex}}
{\color{darkgray}{bash    10932 dantopa  rtd    DIR   0,71     4096 61653409 /}}
{\color{darkgray}{bash    10932 dantopa  txt    REG   0,71  1396520 62702252 /usr/bin/bash}}
{\color{darkgray}{bash    10932 dantopa  mem    REG  254,1          62702252 /usr/bin/bash (path dev=0,71)}}
{\color{darkgray}{bash    10932 dantopa  mem    REG  254,1          63095938 /usr/lib/x86_64-linux-gnu/libc.so.6 (path dev=0,71)}}
{\color{darkgray}{bash    10932 dantopa  mem    REG  254,1           1190606 /usr/lib/x86_64-linux-gnu/libtinfo.so.6.3 (path dev=0,71)}}
{\color{darkgray}{bash    10932 dantopa  mem    REG  254,1          63095935 /usr/lib/x86_64-linux-gnu/ld-linux-x86-64.so.2 (path dev=0,71)}}
{\color{darkgray}{bash    10932 dantopa    0u   CHR  136,1      0t0        4 /dev/pts/1}}
{\color{darkgray}{bash    10932 dantopa    1u   CHR  136,1      0t0        4 /dev/pts/1}}
{\color{darkgray}{bash    10932 dantopa    2u   CHR  136,1      0t0        4 /dev/pts/1}}
{\color{darkgray}{bash    10932 dantopa  255u   CHR  136,1      0t0        4 /dev/pts/1}}
{\color{darkgray}{lsof    11139 dantopa  cwd    DIR   0,33      704     1572 /repos/github/vault-fortran/Xmodern-fortran/tau/apex}}
{\color{darkgray}{lsof    11139 dantopa  rtd    DIR   0,71     4096 61653409 /}}
{\color{darkgray}{lsof    11139 dantopa  txt    REG   0,71   167544   709329 /usr/bin/lsof}}
{\color{darkgray}{lsof    11139 dantopa  mem    REG  254,1            709329 /usr/bin/lsof (path dev=0,71)}}
{\color{darkgray}{lsof    11139 dantopa  mem    REG  254,1          63095951 /usr/lib/x86_64-linux-gnu/libresolv.so.2 (path dev=0,71)}}
{\color{darkgray}{lsof    11139 dantopa  mem    REG  254,1           1190531 /usr/lib/x86_64-linux-gnu/libkeyutils.so.1.9 (path dev=0,71)}}
{\color{darkgray}{lsof    11139 dantopa  mem    REG  254,1          63096020 /usr/lib/x86_64-linux-gnu/libkrb5support.so.0.1 (path dev=0,71)}}
{\color{darkgray}{lsof    11139 dantopa  mem    REG  254,1          63096026 /usr/lib/x86_64-linux-gnu/libcom_err.so.2.1 (path dev=0,71)}}
{\color{darkgray}{lsof    11139 dantopa  mem    REG  254,1          63096018 /usr/lib/x86_64-linux-gnu/libk5crypto.so.3.1 (path dev=0,71)}}
{\color{darkgray}{lsof    11139 dantopa  mem    REG  254,1          63096022 /usr/lib/x86_64-linux-gnu/libkrb5.so.3.3 (path dev=0,71)}}
{\color{darkgray}{lsof    11139 dantopa  mem    REG  254,1           1190578 /usr/lib/x86_64-linux-gnu/libpcre2-8.so.0.10.4 (path dev=0,71)}}
{\color{darkgray}{lsof    11139 dantopa  mem    REG  254,1          63096024 /usr/lib/x86_64-linux-gnu/libgssapi_krb5.so.2.2 (path dev=0,71)}}
{\color{darkgray}{lsof    11139 dantopa  mem    REG  254,1          63095938 /usr/lib/x86_64-linux-gnu/libc.so.6 (path dev=0,71)}}
{\color{darkgray}{lsof    11139 dantopa  mem    REG  254,1           1190588 /usr/lib/x86_64-linux-gnu/libselinux.so.1 (path dev=0,71)}}
{\color{darkgray}{lsof    11139 dantopa  mem    REG  254,1           1190608 /usr/lib/x86_64-linux-gnu/libtirpc.so.3.0.0 (path dev=0,71)}}
{\color{darkgray}{lsof    11139 dantopa  mem    REG  254,1          63095935 /usr/lib/x86_64-linux-gnu/ld-linux-x86-64.so.2 (path dev=0,71)}}
{\color{darkgray}{lsof    11139 dantopa    0u   CHR  136,1      0t0        4 /dev/pts/1}}
{\color{darkgray}{lsof    11139 dantopa    1u   CHR  136,1      0t0        4 /dev/pts/1}}
{\color{darkgray}{lsof    11139 dantopa    2u   CHR  136,1      0t0        4 /dev/pts/1}}
{\color{darkgray}{lsof    11139 dantopa    3r   DIR   0,74        0        1 /proc}}
{\color{darkgray}{lsof    11139 dantopa    4r   DIR   0,74        7   123326 /proc/11139/fd}}
{\color{darkgray}{lsof    11139 dantopa    5w  FIFO   0,11      0t0   123331 pipe}}
{\color{darkgray}{lsof    11139 dantopa    6r  FIFO   0,11      0t0   123332 pipe}}
{\color{darkgray}{lsof    11140 dantopa  cwd    DIR   0,33      704     1572 /repos/github/vault-fortran/Xmodern-fortran/tau/apex}}
{\color{darkgray}{lsof    11140 dantopa  rtd    DIR   0,71     4096 61653409 /}}
{\color{darkgray}{lsof    11140 dantopa  txt    REG   0,71   167544   709329 /usr/bin/lsof}}
{\color{darkgray}{lsof    11140 dantopa  mem    REG  254,1            709329 /usr/bin/lsof (path dev=0,71)}}
{\color{darkgray}{lsof    11140 dantopa  mem    REG  254,1          63095951 /usr/lib/x86_64-linux-gnu/libresolv.so.2 (path dev=0,71)}}
{\color{darkgray}{lsof    11140 dantopa  mem    REG  254,1           1190531 /usr/lib/x86_64-linux-gnu/libkeyutils.so.1.9 (path dev=0,71)}}
{\color{darkgray}{lsof    11140 dantopa  mem    REG  254,1          63096020 /usr/lib/x86_64-linux-gnu/libkrb5support.so.0.1 (path dev=0,71)}}
{\color{darkgray}{lsof    11140 dantopa  mem    REG  254,1          63096026 /usr/lib/x86_64-linux-gnu/libcom_err.so.2.1 (path dev=0,71)}}
{\color{darkgray}{lsof    11140 dantopa  mem    REG  254,1          63096018 /usr/lib/x86_64-linux-gnu/libk5crypto.so.3.1 (path dev=0,71)}}
{\color{darkgray}{lsof    11140 dantopa  mem    REG  254,1          63096022 /usr/lib/x86_64-linux-gnu/libkrb5.so.3.3 (path dev=0,71)}}
{\color{darkgray}{lsof    11140 dantopa  mem    REG  254,1           1190578 /usr/lib/x86_64-linux-gnu/libpcre2-8.so.0.10.4 (path dev=0,71)}}
{\color{darkgray}{lsof    11140 dantopa  mem    REG  254,1          63096024 /usr/lib/x86_64-linux-gnu/libgssapi_krb5.so.2.2 (path dev=0,71)}}
{\color{darkgray}{lsof    11140 dantopa  mem    REG  254,1          63095938 /usr/lib/x86_64-linux-gnu/libc.so.6 (path dev=0,71)}}
{\color{darkgray}{lsof    11140 dantopa  mem    REG  254,1           1190588 /usr/lib/x86_64-linux-gnu/libselinux.so.1 (path dev=0,71)}}
{\color{darkgray}{lsof    11140 dantopa  mem    REG  254,1           1190608 /usr/lib/x86_64-linux-gnu/libtirpc.so.3.0.0 (path dev=0,71)}}
{\color{darkgray}{lsof    11140 dantopa  mem    REG  254,1          63095935 /usr/lib/x86_64-linux-gnu/ld-linux-x86-64.so.2 (path dev=0,71)}}
{\color{darkgray}{lsof    11140 dantopa    4r  FIFO   0,11      0t0   123331 pipe}}
{\color{darkgray}{lsof    11140 dantopa    7w  FIFO   0,11      0t0   123332 pipe}}
\end{Verbatim}
}}
\endinput  %  ==  ==  ==  ==  ==  ==  ==  ==  ==


\subsection{\lsof}
\label{sec:lsof}
This command does an \textt{ls} on open files. The example show how to query both a user and a process id (\textt{pid}).
		\subsubsection{\lsof \ on Process ID}
The \refLsof \ command shows open files, here for the bash process with PID = 10932:
{\footnotesize{
\begin{Verbatim}[commandchars=\\\{\}]
{\color{darkgray}{dantopa@92bc4c447e32:~}}$ ps
{\color{darkgray}{  PID TTY          TIME CMD}}
{\color{darkgray}{10932 pts/1    00:00:00 bash}}
{\color{darkgray}{11152 pts/1    00:00:00 ps}}
{\color{darkgray}{dantopa@92bc4c447e32:~}}$ lsof -p 10932
{\color{darkgray}{COMMAND   PID    USER   FD   TYPE DEVICE SIZE/OFF     NODE NAME}}
{\color{darkgray}{bash    10932 dantopa  cwd    DIR   0,71     4096  6820049 /home/dantopa}}
{\color{darkgray}{bash    10932 dantopa  rtd    DIR   0,71     4096 61653409 /}}
{\color{darkgray}{bash    10932 dantopa  txt    REG   0,71  1396520 62702252 /usr/bin/bash}}
{\color{darkgray}{bash    10932 dantopa  mem    REG  254,1          62702252 /usr/bin/bash (path dev=0,71)}}
{\color{darkgray}{bash    10932 dantopa  mem    REG  254,1          63095938 /usr/lib/x86_64-linux-gnu/libc.so.6 (path dev=0,71)}}
{\color{darkgray}{bash    10932 dantopa  mem    REG  254,1           1190606 /usr/lib/x86_64-linux-gnu/libtinfo.so.6.3 (path dev=0,71)}}
{\color{darkgray}{bash    10932 dantopa  mem    REG  254,1          63095935 /usr/lib/x86_64-linux-gnu/ld-linux-x86-64.so.2 (path dev=0,71)}}
{\color{darkgray}{bash    10932 dantopa    0u   CHR  136,1      0t0        4 /dev/pts/1}}
{\color{darkgray}{bash    10932 dantopa    1u   CHR  136,1      0t0        4 /dev/pts/1}}
{\color{darkgray}{bash    10932 dantopa    2u   CHR  136,1      0t0        4 /dev/pts/1}}
{\color{darkgray}{bash    10932 dantopa  255u   CHR  136,1      0t0        4 /dev/pts/1}}
\end{Verbatim}
}
		\subsubsection{\lsof \ on User}
These are open files for user \textt{dantopa}:
{\footnotesize{
\begin{Verbatim}[commandchars=\\\{\}]
{\color{darkgray}{dantopa@92bc4c447e32:~}}$ lsof -u dantopa
{\color{darkgray}{COMMAND   PID    USER   FD   TYPE DEVICE SIZE/OFF     NODE NAME}}
{\color{darkgray}{bash    10921 dantopa  cwd    DIR   0,71     4096 61653409 /}}
{\color{darkgray}{bash    10921 dantopa  rtd    DIR   0,71     4096 61653409 /}}
{\color{darkgray}{bash    10921 dantopa  txt    REG   0,71  1396520 62702252 /usr/bin/bash}}
{\color{darkgray}{bash    10921 dantopa  mem    REG  254,1          62702252 /usr/bin/bash (path dev=0,71)}}
{\color{darkgray}{bash    10921 dantopa  mem    REG  254,1          63095938 /usr/lib/x86_64-linux-gnu/libc.so.6 (path dev=0,71)}}
{\color{darkgray}{bash    10921 dantopa  mem    REG  254,1           1190606 /usr/lib/x86_64-linux-gnu/libtinfo.so.6.3 (path dev=0,71)}}
{\color{darkgray}{bash    10921 dantopa  mem    REG  254,1          63095935 /usr/lib/x86_64-linux-gnu/ld-linux-x86-64.so.2 (path dev=0,71)}}
{\color{darkgray}{bash    10921 dantopa    0u   CHR  136,0      0t0        3 /dev/pts/0}}
{\color{darkgray}{bash    10921 dantopa    1u   CHR  136,0      0t0        3 /dev/pts/0}}
{\color{darkgray}{bash    10921 dantopa    2u   CHR  136,0      0t0        3 /dev/pts/0}}
{\color{darkgray}{bash    10921 dantopa  255u   CHR  136,0      0t0        3 /dev/pts/0}}
{\color{darkgray}{bash    10932 dantopa  cwd    DIR   0,33      704     1572 /repos/github/vault-fortran/Xmodern-fortran/tau/apex}}
{\color{darkgray}{bash    10932 dantopa  rtd    DIR   0,71     4096 61653409 /}}
{\color{darkgray}{bash    10932 dantopa  txt    REG   0,71  1396520 62702252 /usr/bin/bash}}
{\color{darkgray}{bash    10932 dantopa  mem    REG  254,1          62702252 /usr/bin/bash (path dev=0,71)}}
{\color{darkgray}{bash    10932 dantopa  mem    REG  254,1          63095938 /usr/lib/x86_64-linux-gnu/libc.so.6 (path dev=0,71)}}
{\color{darkgray}{bash    10932 dantopa  mem    REG  254,1           1190606 /usr/lib/x86_64-linux-gnu/libtinfo.so.6.3 (path dev=0,71)}}
{\color{darkgray}{bash    10932 dantopa  mem    REG  254,1          63095935 /usr/lib/x86_64-linux-gnu/ld-linux-x86-64.so.2 (path dev=0,71)}}
{\color{darkgray}{bash    10932 dantopa    0u   CHR  136,1      0t0        4 /dev/pts/1}}
{\color{darkgray}{bash    10932 dantopa    1u   CHR  136,1      0t0        4 /dev/pts/1}}
{\color{darkgray}{bash    10932 dantopa    2u   CHR  136,1      0t0        4 /dev/pts/1}}
{\color{darkgray}{bash    10932 dantopa  255u   CHR  136,1      0t0        4 /dev/pts/1}}
{\color{darkgray}{lsof    11139 dantopa  cwd    DIR   0,33      704     1572 /repos/github/vault-fortran/Xmodern-fortran/tau/apex}}
{\color{darkgray}{lsof    11139 dantopa  rtd    DIR   0,71     4096 61653409 /}}
{\color{darkgray}{lsof    11139 dantopa  txt    REG   0,71   167544   709329 /usr/bin/lsof}}
{\color{darkgray}{lsof    11139 dantopa  mem    REG  254,1            709329 /usr/bin/lsof (path dev=0,71)}}
{\color{darkgray}{lsof    11139 dantopa  mem    REG  254,1          63095951 /usr/lib/x86_64-linux-gnu/libresolv.so.2 (path dev=0,71)}}
{\color{darkgray}{lsof    11139 dantopa  mem    REG  254,1           1190531 /usr/lib/x86_64-linux-gnu/libkeyutils.so.1.9 (path dev=0,71)}}
{\color{darkgray}{lsof    11139 dantopa  mem    REG  254,1          63096020 /usr/lib/x86_64-linux-gnu/libkrb5support.so.0.1 (path dev=0,71)}}
{\color{darkgray}{lsof    11139 dantopa  mem    REG  254,1          63096026 /usr/lib/x86_64-linux-gnu/libcom_err.so.2.1 (path dev=0,71)}}
{\color{darkgray}{lsof    11139 dantopa  mem    REG  254,1          63096018 /usr/lib/x86_64-linux-gnu/libk5crypto.so.3.1 (path dev=0,71)}}
{\color{darkgray}{lsof    11139 dantopa  mem    REG  254,1          63096022 /usr/lib/x86_64-linux-gnu/libkrb5.so.3.3 (path dev=0,71)}}
{\color{darkgray}{lsof    11139 dantopa  mem    REG  254,1           1190578 /usr/lib/x86_64-linux-gnu/libpcre2-8.so.0.10.4 (path dev=0,71)}}
{\color{darkgray}{lsof    11139 dantopa  mem    REG  254,1          63096024 /usr/lib/x86_64-linux-gnu/libgssapi_krb5.so.2.2 (path dev=0,71)}}
{\color{darkgray}{lsof    11139 dantopa  mem    REG  254,1          63095938 /usr/lib/x86_64-linux-gnu/libc.so.6 (path dev=0,71)}}
{\color{darkgray}{lsof    11139 dantopa  mem    REG  254,1           1190588 /usr/lib/x86_64-linux-gnu/libselinux.so.1 (path dev=0,71)}}
{\color{darkgray}{lsof    11139 dantopa  mem    REG  254,1           1190608 /usr/lib/x86_64-linux-gnu/libtirpc.so.3.0.0 (path dev=0,71)}}
{\color{darkgray}{lsof    11139 dantopa  mem    REG  254,1          63095935 /usr/lib/x86_64-linux-gnu/ld-linux-x86-64.so.2 (path dev=0,71)}}
{\color{darkgray}{lsof    11139 dantopa    0u   CHR  136,1      0t0        4 /dev/pts/1}}
{\color{darkgray}{lsof    11139 dantopa    1u   CHR  136,1      0t0        4 /dev/pts/1}}
{\color{darkgray}{lsof    11139 dantopa    2u   CHR  136,1      0t0        4 /dev/pts/1}}
{\color{darkgray}{lsof    11139 dantopa    3r   DIR   0,74        0        1 /proc}}
{\color{darkgray}{lsof    11139 dantopa    4r   DIR   0,74        7   123326 /proc/11139/fd}}
{\color{darkgray}{lsof    11139 dantopa    5w  FIFO   0,11      0t0   123331 pipe}}
{\color{darkgray}{lsof    11139 dantopa    6r  FIFO   0,11      0t0   123332 pipe}}
{\color{darkgray}{lsof    11140 dantopa  cwd    DIR   0,33      704     1572 /repos/github/vault-fortran/Xmodern-fortran/tau/apex}}
{\color{darkgray}{lsof    11140 dantopa  rtd    DIR   0,71     4096 61653409 /}}
{\color{darkgray}{lsof    11140 dantopa  txt    REG   0,71   167544   709329 /usr/bin/lsof}}
{\color{darkgray}{lsof    11140 dantopa  mem    REG  254,1            709329 /usr/bin/lsof (path dev=0,71)}}
{\color{darkgray}{lsof    11140 dantopa  mem    REG  254,1          63095951 /usr/lib/x86_64-linux-gnu/libresolv.so.2 (path dev=0,71)}}
{\color{darkgray}{lsof    11140 dantopa  mem    REG  254,1           1190531 /usr/lib/x86_64-linux-gnu/libkeyutils.so.1.9 (path dev=0,71)}}
{\color{darkgray}{lsof    11140 dantopa  mem    REG  254,1          63096020 /usr/lib/x86_64-linux-gnu/libkrb5support.so.0.1 (path dev=0,71)}}
{\color{darkgray}{lsof    11140 dantopa  mem    REG  254,1          63096026 /usr/lib/x86_64-linux-gnu/libcom_err.so.2.1 (path dev=0,71)}}
{\color{darkgray}{lsof    11140 dantopa  mem    REG  254,1          63096018 /usr/lib/x86_64-linux-gnu/libk5crypto.so.3.1 (path dev=0,71)}}
{\color{darkgray}{lsof    11140 dantopa  mem    REG  254,1          63096022 /usr/lib/x86_64-linux-gnu/libkrb5.so.3.3 (path dev=0,71)}}
{\color{darkgray}{lsof    11140 dantopa  mem    REG  254,1           1190578 /usr/lib/x86_64-linux-gnu/libpcre2-8.so.0.10.4 (path dev=0,71)}}
{\color{darkgray}{lsof    11140 dantopa  mem    REG  254,1          63096024 /usr/lib/x86_64-linux-gnu/libgssapi_krb5.so.2.2 (path dev=0,71)}}
{\color{darkgray}{lsof    11140 dantopa  mem    REG  254,1          63095938 /usr/lib/x86_64-linux-gnu/libc.so.6 (path dev=0,71)}}
{\color{darkgray}{lsof    11140 dantopa  mem    REG  254,1           1190588 /usr/lib/x86_64-linux-gnu/libselinux.so.1 (path dev=0,71)}}
{\color{darkgray}{lsof    11140 dantopa  mem    REG  254,1           1190608 /usr/lib/x86_64-linux-gnu/libtirpc.so.3.0.0 (path dev=0,71)}}
{\color{darkgray}{lsof    11140 dantopa  mem    REG  254,1          63095935 /usr/lib/x86_64-linux-gnu/ld-linux-x86-64.so.2 (path dev=0,71)}}
{\color{darkgray}{lsof    11140 dantopa    4r  FIFO   0,11      0t0   123331 pipe}}
{\color{darkgray}{lsof    11140 dantopa    7w  FIFO   0,11      0t0   123332 pipe}}
\end{Verbatim}
}}
\endinput  %  ==  ==  ==  ==  ==  ==  ==  ==  ==

		% % % \input{./sections/ssec-objdump}

\subsection{\objdump}
\label{sec:objdump}
The \refObjdump \ command shows dependent shared objects, typically libraries. Two versions of the shared library for the GNU standard C library -- one 32 bit, the other 64 bit -- are located.
{\footnotesize{
\begin{Verbatim}[commandchars=\\\{\}]
{\color{darkgray}{dantopa@92bc4c447e32:/}}$ locate libc.so.6
{\color{darkgray}{/usr/lib/x86_64-linux-gnu/libc.so.6}}
{\color{darkgray}{/usr/lib32/libc.so.6}}
\end{Verbatim}
}}

\endinput  %  ==  ==  ==  ==  ==  ==  ==  ==  ==


\subsection{\objdump}
\label{sec:objdump}
The \refObjdump \ command shows dependent shared objects, typically libraries. Two versions of the shared library for the GNU standard C library -- one 32 bit, the other 64 bit -- are located.
{\footnotesize{
\begin{Verbatim}[commandchars=\\\{\}]
{\color{darkgray}{dantopa@92bc4c447e32:/}}$ locate libc.so.6
{\color{darkgray}{/usr/lib/x86_64-linux-gnu/libc.so.6}}
{\color{darkgray}{/usr/lib32/libc.so.6}}
\end{Verbatim}
}}

\endinput  %  ==  ==  ==  ==  ==  ==  ==  ==  ==


\subsection{\objdump}
\label{sec:objdump}
The \refObjdump \ command shows dependent shared objects, typically libraries. Two versions of the shared library for the GNU standard C library -- one 32 bit, the other 64 bit -- are located.
{\footnotesize{
\begin{Verbatim}[commandchars=\\\{\}]
{\color{darkgray}{dantopa@92bc4c447e32:/}}$ locate libc.so.6
{\color{darkgray}{/usr/lib/x86_64-linux-gnu/libc.so.6}}
{\color{darkgray}{/usr/lib32/libc.so.6}}
\end{Verbatim}
}}

\endinput  %  ==  ==  ==  ==  ==  ==  ==  ==  ==

		% % % \input{./sections/ssec-readelf}

\subsection{\readelf}
\label{sec:readelf}
The \refReadelf \ command displays information about \elf \ files, or Executable and Linkable Format files which are a standard file format for executable files, object code, shared libraries, and core dumps.\footnote{For an ELF cheatsheet see \href{https://gist.github.com/x0nu11byt3/bcb35c3de461e5fb66173071a2379779}{https://gist.github.com/x0nu11byt3/bcb35c3de461e5fb66173071a2379779}.} This example lists the header file for the command \textt{bash}.
{\footnotesize{
\begin{Verbatim}[commandchars=\\\{\}]
{\color{darkgray}{dantopa@92bc4c447e32:~}}$ file /bin/bash
{\color{darkgray}{/bin/bash: ELF 64-bit LSB pie executable, x86-64, version 1 (SYSV), dynamically linked, interpreter /lib64/ld-linux-x86-64.so.2,}} 
{\color{darkgray}{BuildID[sha1]=7a6408ba82a2d86dd98f1f75ac8edcb695f6fd60, for GNU/Linux 3.2.0, stripped}}
{\color{darkgray}{dantopa@92bc4c447e32:~}}$ readelf -h /bin/bash
{\color{darkgray}{ELF Header:}}
{\color{darkgray}{  Magic:   7f 45 4c 46 02 01 01 00 00 00 00 00 00 00 00 00}} 
{\color{darkgray}{  Class:                             ELF64}}
{\color{darkgray}{  Data:                              2's complement, little endian}}
{\color{darkgray}{  Version:                           1 (current)}}
{\color{darkgray}{  OS/ABI:                            UNIX - System V}}
{\color{darkgray}{  ABI Version:                       0}}
{\color{darkgray}{  Type:                              DYN (Position-Independent Executable file)}}
{\color{darkgray}{  Machine:                           Advanced Micro Devices X86-64}}
{\color{darkgray}{  Version:                           0x1}}
{\color{darkgray}{  Entry point address:               0x32ef0}}
{\color{darkgray}{  Start of program headers:          64 (bytes into file)}}
{\color{darkgray}{  Start of section headers:          1394600 (bytes into file)}}
{\color{darkgray}{  Flags:                             0x0}}
{\color{darkgray}{  Size of this header:               64 (bytes)}}
{\color{darkgray}{  Size of program headers:           56 (bytes)}}
{\color{darkgray}{  Number of program headers:         13}}
{\color{darkgray}{  Size of section headers:           64 (bytes)}}
{\color{darkgray}{  Number of section headers:         30}}
{\color{darkgray}{  Section header string table index: 29}}
\end{Verbatim}
}}

\endinput  %  ==  ==  ==  ==  ==  ==  ==  ==  ==


\subsection{\readelf}
\label{sec:readelf}
The \refReadelf \ command displays information about \elf \ files, or Executable and Linkable Format files which are a standard file format for executable files, object code, shared libraries, and core dumps.\footnote{For an ELF cheatsheet see \href{https://gist.github.com/x0nu11byt3/bcb35c3de461e5fb66173071a2379779}{https://gist.github.com/x0nu11byt3/bcb35c3de461e5fb66173071a2379779}.} This example lists the header file for the command \textt{bash}.
{\footnotesize{
\begin{Verbatim}[commandchars=\\\{\}]
{\color{darkgray}{dantopa@92bc4c447e32:~}}$ file /bin/bash
{\color{darkgray}{/bin/bash: ELF 64-bit LSB pie executable, x86-64, version 1 (SYSV), dynamically linked, interpreter /lib64/ld-linux-x86-64.so.2,}} 
{\color{darkgray}{BuildID[sha1]=7a6408ba82a2d86dd98f1f75ac8edcb695f6fd60, for GNU/Linux 3.2.0, stripped}}
{\color{darkgray}{dantopa@92bc4c447e32:~}}$ readelf -h /bin/bash
{\color{darkgray}{ELF Header:}}
{\color{darkgray}{  Magic:   7f 45 4c 46 02 01 01 00 00 00 00 00 00 00 00 00}} 
{\color{darkgray}{  Class:                             ELF64}}
{\color{darkgray}{  Data:                              2's complement, little endian}}
{\color{darkgray}{  Version:                           1 (current)}}
{\color{darkgray}{  OS/ABI:                            UNIX - System V}}
{\color{darkgray}{  ABI Version:                       0}}
{\color{darkgray}{  Type:                              DYN (Position-Independent Executable file)}}
{\color{darkgray}{  Machine:                           Advanced Micro Devices X86-64}}
{\color{darkgray}{  Version:                           0x1}}
{\color{darkgray}{  Entry point address:               0x32ef0}}
{\color{darkgray}{  Start of program headers:          64 (bytes into file)}}
{\color{darkgray}{  Start of section headers:          1394600 (bytes into file)}}
{\color{darkgray}{  Flags:                             0x0}}
{\color{darkgray}{  Size of this header:               64 (bytes)}}
{\color{darkgray}{  Size of program headers:           56 (bytes)}}
{\color{darkgray}{  Number of program headers:         13}}
{\color{darkgray}{  Size of section headers:           64 (bytes)}}
{\color{darkgray}{  Number of section headers:         30}}
{\color{darkgray}{  Section header string table index: 29}}
\end{Verbatim}
}}

\endinput  %  ==  ==  ==  ==  ==  ==  ==  ==  ==


\subsection{\readelf}
\label{sec:readelf}
The \refReadelf \ command displays information about \elf \ files, or Executable and Linkable Format files which are a standard file format for executable files, object code, shared libraries, and core dumps.\footnote{For an ELF cheatsheet see \href{https://gist.github.com/x0nu11byt3/bcb35c3de461e5fb66173071a2379779}{https://gist.github.com/x0nu11byt3/bcb35c3de461e5fb66173071a2379779}.} This example lists the header file for the command \textt{bash}.
{\footnotesize{
\begin{Verbatim}[commandchars=\\\{\}]
{\color{darkgray}{dantopa@92bc4c447e32:~}}$ file /bin/bash
{\color{darkgray}{/bin/bash: ELF 64-bit LSB pie executable, x86-64, version 1 (SYSV), dynamically linked, interpreter /lib64/ld-linux-x86-64.so.2,}} 
{\color{darkgray}{BuildID[sha1]=7a6408ba82a2d86dd98f1f75ac8edcb695f6fd60, for GNU/Linux 3.2.0, stripped}}
{\color{darkgray}{dantopa@92bc4c447e32:~}}$ readelf -h /bin/bash
{\color{darkgray}{ELF Header:}}
{\color{darkgray}{  Magic:   7f 45 4c 46 02 01 01 00 00 00 00 00 00 00 00 00}} 
{\color{darkgray}{  Class:                             ELF64}}
{\color{darkgray}{  Data:                              2's complement, little endian}}
{\color{darkgray}{  Version:                           1 (current)}}
{\color{darkgray}{  OS/ABI:                            UNIX - System V}}
{\color{darkgray}{  ABI Version:                       0}}
{\color{darkgray}{  Type:                              DYN (Position-Independent Executable file)}}
{\color{darkgray}{  Machine:                           Advanced Micro Devices X86-64}}
{\color{darkgray}{  Version:                           0x1}}
{\color{darkgray}{  Entry point address:               0x32ef0}}
{\color{darkgray}{  Start of program headers:          64 (bytes into file)}}
{\color{darkgray}{  Start of section headers:          1394600 (bytes into file)}}
{\color{darkgray}{  Flags:                             0x0}}
{\color{darkgray}{  Size of this header:               64 (bytes)}}
{\color{darkgray}{  Size of program headers:           56 (bytes)}}
{\color{darkgray}{  Number of program headers:         13}}
{\color{darkgray}{  Size of section headers:           64 (bytes)}}
{\color{darkgray}{  Number of section headers:         30}}
{\color{darkgray}{  Section header string table index: 29}}
\end{Verbatim}
}}

\endinput  %  ==  ==  ==  ==  ==  ==  ==  ==  ==

		% % % \input{./sections/ssec-nm}

\subsection{\nm}
\label{sec:nm}

The \refNm \ command shows dependent shared objects and executables; 

\endinput  %  ==  ==  ==  ==  ==  ==  ==  ==  ==


\subsection{\nm}
\label{sec:nm}

The \refNm \ command shows dependent shared objects and executables; 

\endinput  %  ==  ==  ==  ==  ==  ==  ==  ==  ==


\subsection{\nm}
\label{sec:nm}

The \refNm \ command shows dependent shared objects and executables; 

\endinput  %  ==  ==  ==  ==  ==  ==  ==  ==  ==

		% % % \input{./sections/ssec-strace}

\subsection{\strace}
\label{sec:strace}
The \strace \ command is very powerful and the following examples.

		\subsubsection{Trace System Calls To A Given Path}
{\footnotesize{
\begin{Verbatim}[commandchars=\\\{\}]
{\color{darkgray}{root@169e8b2c1ae3:/#}} strace -P /etc/ld.so.cache ls /dev/null 
{\color{darkgray}{openat(AT_FDCWD, "/etc/ld.so.cache", O_RDONLY|O_CLOEXEC) = 3}}
{\color{darkgray}{newfstatat(3, "", {st_mode=S_IFREG|0644, st_size=135191, ...}, AT_EMPTY_PATH) = 0}}
{\color{darkgray}{mmap(NULL, 135191, PROT_READ, MAP_PRIVATE, 3, 0) = 0x7f03bba95000}}
{\color{darkgray}{close(3)                                = 0}}
{\color{darkgray}{/dev/null}}
{\color{darkgray}{+++ exited with 0 +++}}
\end{Verbatim}
}}


		\subsubsection{Inventory time, calls, and errors for every system call}
{\footnotesize{
\begin{Verbatim}[commandchars=\\\{\}]
{\color{darkgray}{root@169e8b2c1ae3:/}}# strace -c ls > /dev/null
{\color{darkgray}{\escapepercent time     seconds  usecs/call     calls    errors syscall}}
{\color{darkgray}{------ ----------- ----------- --------- --------- ----------------}}
{\color{darkgray}{ 71.76    0.013546        6773         2           getdents64}}
{\color{darkgray}{  7.85    0.001482         247         6           openat}}
{\color{darkgray}{  4.88    0.000922         922         1           execve}}
{\color{darkgray}{  4.44    0.000839          49        17           mmap}}
{\color{darkgray}{  1.84    0.000347          43         8           close}}
{\color{darkgray}{  1.48    0.000279          39         7           mprotect}}
{\color{darkgray}{  1.40    0.000265          37         7           newfstatat}}
{\color{darkgray}{  1.26    0.000237          47         5           read}}
{\color{darkgray}{  0.94    0.000178          44         4           pread64}}
{\color{darkgray}{  0.77    0.000145          48         3           brk}}
{\color{darkgray}{  0.57    0.000108          36         3         3 ioctl}}
{\color{darkgray}{  0.49    0.000092          46         2         2 statfs}}
{\color{darkgray}{  0.47    0.000088          44         2         2 access}}
{\color{darkgray}{  0.34    0.000065          32         2         1 arch_prctl}}
{\color{darkgray}{  0.34    0.000065          65         1           getrandom}}
{\color{darkgray}{  0.32    0.000061          61         1           munmap}}
{\color{darkgray}{  0.18    0.000034          34         1           rseq}}
{\color{darkgray}{  0.17    0.000032          32         1           set_robust_list}}
{\color{darkgray}{  0.16    0.000031          31         1           write}}
{\color{darkgray}{  0.16    0.000031          31         1           set_tid_address}}
{\color{darkgray}{  0.16    0.000031          31         1           prlimit64}}
{\color{darkgray}{------ ----------- ----------- --------- --------- ----------------}}
{\color{darkgray}{100.00    0.018878         248        76         8 total}}
\end{Verbatim}
}}

		\subsubsection{Identify Information Associated With File Descriptorsl}
{\footnotesize{
\begin{Verbatim}[commandchars=\\\{\}]
{\color{darkgray}{root@169e8b2c1ae3:/}}# strace -yy cat /dev/null
{\color{darkgray}{execve("/usr/bin/cat", ["cat", "/dev/null"], 0x7fffb8b235d0 /* 10 vars */) = 0}}
{\color{darkgray}{brk(NULL)                               = 0x5611c6a38000}}
{\color{darkgray}{arch_prctl(0x3001 /* ARCH_??? */, 0x7ffeede990c0) = -1 EINVAL (Invalid argument)}}
{\color{darkgray}{mmap(NULL, 8192, PROT_READ|PROT_WRITE, MAP_PRIVATE|MAP_ANONYMOUS, -1, 0) = 0x7f5c648b8000}}
{\color{darkgray}{access("/etc/ld.so.preload", R_OK)      = -1 ENOENT (No such file or directory)}}
{\color{darkgray}{openat(AT_FDCWD</>, "/etc/ld.so.cache", O_RDONLY|O_CLOEXEC) = 3</etc/ld.so.cache>}}
{\color{darkgray}{newfstatat(3</etc/ld.so.cache>, "", {st_mode=S_IFREG|0644, st_size=135191, ...}, AT_EMPTY_PATH) = 0}}
{\color{darkgray}{mmap(NULL, 135191, PROT_READ, MAP_PRIVATE, 3</etc/ld.so.cache>, 0) = 0x7f5c64896000}}
{\color{darkgray}{close(3</etc/ld.so.cache>)              = 0}}
{\color{darkgray}{openat(AT_FDCWD</>, "/lib/x86_64-linux-gnu/libc.so.6", O_RDONLY|O_CLOEXEC) = 3</usr/lib/x86_64-linux-gnu/libc.so.6>}}
{\color{darkgray}{read(3</usr/lib/x86_64-linux-gnu/libc.so.6>, "\textbackslash 177ELF\textbackslash2\textbackslash1\textbackslash1\textbackslash3\textbackslash0\textbackslash0\textbackslash0\textbackslash0\textbackslash0\textbackslash0\textbackslash0\textbackslash0\textbackslash3\textbackslash0>\textbackslash0\textbackslash1\textbackslash0\textbackslash0\textbackslash0P\textbackslash237\textbackslash2\textbackslash0\textbackslash0\textbackslash0\textbackslash0\textbackslash0"..., 832) = 832}}
{\color{darkgray}{pread64(3</usr/lib/x86_64-linux-gnu/libc.so.6>, "\textbackslash6\textbackslash0\textbackslash0\textbackslash0\textbackslash4\textbackslash0\textbackslash0\textbackslash0@\textbackslash0\textbackslash0\textbackslash0\textbackslash0\textbackslash0\textbackslash0\textbackslash0@\textbackslash0\textbackslash0\textbackslash0\textbackslash0\textbackslash0\textbackslash0\textbackslash0@\textbackslash0\textbackslash0\textbackslash0\textbackslash0\textbackslash0\textbackslash0\textbackslash0"..., 784, 64) = 784}}
{\color{darkgray}{pread64(3</usr/lib/x86_64-linux-gnu/libc.so.6>, "\textbackslash4\textbackslash0\textbackslash0\textbackslash0 \textbackslash0\textbackslash0\textbackslash0\textbackslash5\textbackslash0\textbackslash0\textbackslash0GNU\textbackslash0\textbackslash2\textbackslash0\textbackslash0\textbackslash300\textbackslash4\textbackslash0\textbackslash0\textbackslash0\textbackslash3\textbackslash0\textbackslash0\textbackslash0\textbackslash0\textbackslash0\textbackslash0\textbackslash0"..., 48, 848) = 48}}
{\color{darkgray}{pread64(3</usr/lib/x86_64-linux-gnu/libc.so.6>, "\textbackslash4\textbackslash0\textbackslash0\textbackslash0\textbackslash24\textbackslash0\textbackslash0\textbackslash0\textbackslash3\textbackslash0\textbackslash0\textbackslash0GNU\textbackslash0I\textbackslash17\textbackslash357\textbackslash204\textbackslash3$\textbackslash f\textbackslash221\textbackslash2039x\textbackslash324\textbackslash224\textbackslash323\textbackslash236S"..., 68, 896) = 68}}
{\color{darkgray}{newfstatat(3</usr/lib/x86_64-linux-gnu/libc.so.6>, "", {st_mode=S_IFREG|0755, st_size=2220400, ...}, AT_EMPTY_PATH) = 0}}
{\color{darkgray}{pread64(3</usr/lib/x86_64-linux-gnu/libc.so.6>, "\textbackslash6\textbackslash0\textbackslash0\textbackslash0\textbackslash4\textbackslash0\textbackslash0\textbackslash0@\textbackslash0\textbackslash0\textbackslash0\textbackslash0\textbackslash0\textbackslash0\textbackslash0@\textbackslash0\textbackslash0\textbackslash0\textbackslash0\textbackslash0\textbackslash0\textbackslash0@\textbackslash0\textbackslash0\textbackslash0\textbackslash0\textbackslash0\textbackslash0\textbackslash0"..., 784, 64) = 784}}
{\color{darkgray}{mmap(NULL, 2264656, PROT_READ, MAP_PRIVATE|MAP_DENYWRITE, 3</usr/lib/x86_64-linux-gnu/libc.so.6>, 0) = 0x7f5c6466d000}}
{\color{darkgray}{mprotect(0x7f5c64695000, 2023424, PROT_NONE) = 0}}
{\color{darkgray}{mmap(0x7f5c64695000, 1658880, PROT_READ|PROT_EXEC, MAP_PRIVATE|MAP_FIXED|MAP_DENYWRITE, 3</usr/lib/x86_64-linux-gnu/libc.so.6>, 0x28000) = 0x7f5c64695000}}
{\color{darkgray}{mmap(0x7f5c6482a000, 360448, PROT_READ, MAP_PRIVATE|MAP_FIXED|MAP_DENYWRITE, 3</usr/lib/x86_64-linux-gnu/libc.so.6>, 0x1bd000) = 0x7f5c6482a000}}
{\color{darkgray}{mmap(0x7f5c64883000, 24576, PROT_READ|PROT_WRITE, MAP_PRIVATE|MAP_FIXED|MAP_DENYWRITE, 3</usr/lib/x86_64-linux-gnu/libc.so.6>, 0x215000) = 0x7f5c64883000}}
{\color{darkgray}{mmap(0x7f5c64889000, 52816, PROT_READ|PROT_WRITE, MAP_PRIVATE|MAP_FIXED|MAP_ANONYMOUS, -1, 0) = 0x7f5c64889000}}
{\color{darkgray}{close(3</usr/lib/x86_64-linux-gnu/libc.so.6>) = 0}}
{\color{darkgray}{mmap(NULL, 12288, PROT_READ|PROT_WRITE, MAP_PRIVATE|MAP_ANONYMOUS, -1, 0) = 0x7f5c6466a000}}
{\color{darkgray}{arch_prctl(ARCH_SET_FS, 0x7f5c6466a740) = 0}}
{\color{darkgray}{set_tid_address(0x7f5c6466aa10)         = 23663}}
{\color{darkgray}{set_robust_list(0x7f5c6466aa20, 24)     = 0}}
{\color{darkgray}{rseq(0x7f5c6466b0e0, 0x20, 0, 0x53053053) = 0}}
{\color{darkgray}{mprotect(0x7f5c64883000, 16384, PROT_READ) = 0}}
{\color{darkgray}{mprotect(0x5611c4bde000, 4096, PROT_READ) = 0}}
{\color{darkgray}{mprotect(0x7f5c648f2000, 8192, PROT_READ) = 0}}
{\color{darkgray}{prlimit64(0, RLIMIT_STACK, NULL, {rlim_cur=8192*1024, rlim_max=RLIM64_INFINITY}) = 0}}
{\color{darkgray}{munmap(0x7f5c64896000, 135191)          = 0}}
{\color{darkgray}{getrandom("\textbackslash\textbackslash x7e\textbackslash x74\textbackslash x62\textbackslash xbc\textbackslash x66\textbackslash x05\textbackslash x81\textbackslash xf8", 8, GRND_NONBLOCK) = 8}}
{\color{darkgray}{brk(NULL)                               = 0x5611c6a38000}}
{\color{darkgray}{brk(0x5611c6a59000)                     = 0x5611c6a59000}}
{\color{darkgray}{newfstatat(1</dev/pts/0<char 136:0>>, "", {st_mode=S_IFCHR|0620, st_rdev=makedev(0x88, 0), ...}, AT_EMPTY_PATH) = 0}}
{\color{darkgray}{openat(AT_FDCWD</>, "/dev/null", O_RDONLY) = 3</dev/null<char 1:3>>}}
{\color{darkgray}{newfstatat(3</dev/null<char 1:3>>, "", {st_mode=S_IFCHR|0666, st_rdev=makedev(0x1, 0x3), ...}, AT_EMPTY_PATH) = 0}}
{\color{darkgray}{fadvise64(3</dev/null<char 1:3>>, 0, 0, POSIX_FADV_SEQUENTIAL) = 0}}
{\color{darkgray}{mmap(NULL, 139264, PROT_READ|PROT_WRITE, MAP_PRIVATE|MAP_ANONYMOUS, -1, 0) = 0x7f5c64896000}}
{\color{darkgray}{read(3</dev/null<char 1:3>>, "", 131072) = 0}}
{\color{darkgray}{munmap(0x7f5c64896000, 139264)          = 0}}
{\color{darkgray}{close(3</dev/null<char 1:3>>)           = 0}}
{\color{darkgray}{close(1</dev/pts/0<char 136:0>>)        = 0}}
{\color{darkgray}{close(2</dev/pts/0<char 136:0>>)        = 0}}
{\color{darkgray}{exit_group(0)                           = ?}}
{\color{darkgray}{+++ exited with 0 +++}}
\end{Verbatim}
}}
\endinput  %  ==  ==  ==  ==  ==  ==  ==  ==  ==

\subsection{\strace}
\label{sec:strace}
The \strace \ command is very powerful and the following examples.

		\subsubsection{Trace System Calls To A Given Path}
{\footnotesize{
\begin{Verbatim}[commandchars=\\\{\}]
{\color{darkgray}{root@169e8b2c1ae3:/#}} strace -P /etc/ld.so.cache ls /dev/null 
{\color{darkgray}{openat(AT_FDCWD, "/etc/ld.so.cache", O_RDONLY|O_CLOEXEC) = 3}}
{\color{darkgray}{newfstatat(3, "", {st_mode=S_IFREG|0644, st_size=135191, ...}, AT_EMPTY_PATH) = 0}}
{\color{darkgray}{mmap(NULL, 135191, PROT_READ, MAP_PRIVATE, 3, 0) = 0x7f03bba95000}}
{\color{darkgray}{close(3)                                = 0}}
{\color{darkgray}{/dev/null}}
{\color{darkgray}{+++ exited with 0 +++}}
\end{Verbatim}
}}


		\subsubsection{Inventory time, calls, and errors for every system call}
{\footnotesize{
\begin{Verbatim}[commandchars=\\\{\}]
{\color{darkgray}{root@169e8b2c1ae3:/}}# strace -c ls > /dev/null
{\color{darkgray}{\escapepercent time     seconds  usecs/call     calls    errors syscall}}
{\color{darkgray}{------ ----------- ----------- --------- --------- ----------------}}
{\color{darkgray}{ 71.76    0.013546        6773         2           getdents64}}
{\color{darkgray}{  7.85    0.001482         247         6           openat}}
{\color{darkgray}{  4.88    0.000922         922         1           execve}}
{\color{darkgray}{  4.44    0.000839          49        17           mmap}}
{\color{darkgray}{  1.84    0.000347          43         8           close}}
{\color{darkgray}{  1.48    0.000279          39         7           mprotect}}
{\color{darkgray}{  1.40    0.000265          37         7           newfstatat}}
{\color{darkgray}{  1.26    0.000237          47         5           read}}
{\color{darkgray}{  0.94    0.000178          44         4           pread64}}
{\color{darkgray}{  0.77    0.000145          48         3           brk}}
{\color{darkgray}{  0.57    0.000108          36         3         3 ioctl}}
{\color{darkgray}{  0.49    0.000092          46         2         2 statfs}}
{\color{darkgray}{  0.47    0.000088          44         2         2 access}}
{\color{darkgray}{  0.34    0.000065          32         2         1 arch_prctl}}
{\color{darkgray}{  0.34    0.000065          65         1           getrandom}}
{\color{darkgray}{  0.32    0.000061          61         1           munmap}}
{\color{darkgray}{  0.18    0.000034          34         1           rseq}}
{\color{darkgray}{  0.17    0.000032          32         1           set_robust_list}}
{\color{darkgray}{  0.16    0.000031          31         1           write}}
{\color{darkgray}{  0.16    0.000031          31         1           set_tid_address}}
{\color{darkgray}{  0.16    0.000031          31         1           prlimit64}}
{\color{darkgray}{------ ----------- ----------- --------- --------- ----------------}}
{\color{darkgray}{100.00    0.018878         248        76         8 total}}
\end{Verbatim}
}}

		\subsubsection{Identify Information Associated With File Descriptorsl}
{\footnotesize{
\begin{Verbatim}[commandchars=\\\{\}]
{\color{darkgray}{root@169e8b2c1ae3:/}}# strace -yy cat /dev/null
{\color{darkgray}{execve("/usr/bin/cat", ["cat", "/dev/null"], 0x7fffb8b235d0 /* 10 vars */) = 0}}
{\color{darkgray}{brk(NULL)                               = 0x5611c6a38000}}
{\color{darkgray}{arch_prctl(0x3001 /* ARCH_??? */, 0x7ffeede990c0) = -1 EINVAL (Invalid argument)}}
{\color{darkgray}{mmap(NULL, 8192, PROT_READ|PROT_WRITE, MAP_PRIVATE|MAP_ANONYMOUS, -1, 0) = 0x7f5c648b8000}}
{\color{darkgray}{access("/etc/ld.so.preload", R_OK)      = -1 ENOENT (No such file or directory)}}
{\color{darkgray}{openat(AT_FDCWD</>, "/etc/ld.so.cache", O_RDONLY|O_CLOEXEC) = 3</etc/ld.so.cache>}}
{\color{darkgray}{newfstatat(3</etc/ld.so.cache>, "", {st_mode=S_IFREG|0644, st_size=135191, ...}, AT_EMPTY_PATH) = 0}}
{\color{darkgray}{mmap(NULL, 135191, PROT_READ, MAP_PRIVATE, 3</etc/ld.so.cache>, 0) = 0x7f5c64896000}}
{\color{darkgray}{close(3</etc/ld.so.cache>)              = 0}}
{\color{darkgray}{openat(AT_FDCWD</>, "/lib/x86_64-linux-gnu/libc.so.6", O_RDONLY|O_CLOEXEC) = 3</usr/lib/x86_64-linux-gnu/libc.so.6>}}
{\color{darkgray}{read(3</usr/lib/x86_64-linux-gnu/libc.so.6>, "\textbackslash 177ELF\textbackslash2\textbackslash1\textbackslash1\textbackslash3\textbackslash0\textbackslash0\textbackslash0\textbackslash0\textbackslash0\textbackslash0\textbackslash0\textbackslash0\textbackslash3\textbackslash0>\textbackslash0\textbackslash1\textbackslash0\textbackslash0\textbackslash0P\textbackslash237\textbackslash2\textbackslash0\textbackslash0\textbackslash0\textbackslash0\textbackslash0"..., 832) = 832}}
{\color{darkgray}{pread64(3</usr/lib/x86_64-linux-gnu/libc.so.6>, "\textbackslash6\textbackslash0\textbackslash0\textbackslash0\textbackslash4\textbackslash0\textbackslash0\textbackslash0@\textbackslash0\textbackslash0\textbackslash0\textbackslash0\textbackslash0\textbackslash0\textbackslash0@\textbackslash0\textbackslash0\textbackslash0\textbackslash0\textbackslash0\textbackslash0\textbackslash0@\textbackslash0\textbackslash0\textbackslash0\textbackslash0\textbackslash0\textbackslash0\textbackslash0"..., 784, 64) = 784}}
{\color{darkgray}{pread64(3</usr/lib/x86_64-linux-gnu/libc.so.6>, "\textbackslash4\textbackslash0\textbackslash0\textbackslash0 \textbackslash0\textbackslash0\textbackslash0\textbackslash5\textbackslash0\textbackslash0\textbackslash0GNU\textbackslash0\textbackslash2\textbackslash0\textbackslash0\textbackslash300\textbackslash4\textbackslash0\textbackslash0\textbackslash0\textbackslash3\textbackslash0\textbackslash0\textbackslash0\textbackslash0\textbackslash0\textbackslash0\textbackslash0"..., 48, 848) = 48}}
{\color{darkgray}{pread64(3</usr/lib/x86_64-linux-gnu/libc.so.6>, "\textbackslash4\textbackslash0\textbackslash0\textbackslash0\textbackslash24\textbackslash0\textbackslash0\textbackslash0\textbackslash3\textbackslash0\textbackslash0\textbackslash0GNU\textbackslash0I\textbackslash17\textbackslash357\textbackslash204\textbackslash3$\textbackslash f\textbackslash221\textbackslash2039x\textbackslash324\textbackslash224\textbackslash323\textbackslash236S"..., 68, 896) = 68}}
{\color{darkgray}{newfstatat(3</usr/lib/x86_64-linux-gnu/libc.so.6>, "", {st_mode=S_IFREG|0755, st_size=2220400, ...}, AT_EMPTY_PATH) = 0}}
{\color{darkgray}{pread64(3</usr/lib/x86_64-linux-gnu/libc.so.6>, "\textbackslash6\textbackslash0\textbackslash0\textbackslash0\textbackslash4\textbackslash0\textbackslash0\textbackslash0@\textbackslash0\textbackslash0\textbackslash0\textbackslash0\textbackslash0\textbackslash0\textbackslash0@\textbackslash0\textbackslash0\textbackslash0\textbackslash0\textbackslash0\textbackslash0\textbackslash0@\textbackslash0\textbackslash0\textbackslash0\textbackslash0\textbackslash0\textbackslash0\textbackslash0"..., 784, 64) = 784}}
{\color{darkgray}{mmap(NULL, 2264656, PROT_READ, MAP_PRIVATE|MAP_DENYWRITE, 3</usr/lib/x86_64-linux-gnu/libc.so.6>, 0) = 0x7f5c6466d000}}
{\color{darkgray}{mprotect(0x7f5c64695000, 2023424, PROT_NONE) = 0}}
{\color{darkgray}{mmap(0x7f5c64695000, 1658880, PROT_READ|PROT_EXEC, MAP_PRIVATE|MAP_FIXED|MAP_DENYWRITE, 3</usr/lib/x86_64-linux-gnu/libc.so.6>, 0x28000) = 0x7f5c64695000}}
{\color{darkgray}{mmap(0x7f5c6482a000, 360448, PROT_READ, MAP_PRIVATE|MAP_FIXED|MAP_DENYWRITE, 3</usr/lib/x86_64-linux-gnu/libc.so.6>, 0x1bd000) = 0x7f5c6482a000}}
{\color{darkgray}{mmap(0x7f5c64883000, 24576, PROT_READ|PROT_WRITE, MAP_PRIVATE|MAP_FIXED|MAP_DENYWRITE, 3</usr/lib/x86_64-linux-gnu/libc.so.6>, 0x215000) = 0x7f5c64883000}}
{\color{darkgray}{mmap(0x7f5c64889000, 52816, PROT_READ|PROT_WRITE, MAP_PRIVATE|MAP_FIXED|MAP_ANONYMOUS, -1, 0) = 0x7f5c64889000}}
{\color{darkgray}{close(3</usr/lib/x86_64-linux-gnu/libc.so.6>) = 0}}
{\color{darkgray}{mmap(NULL, 12288, PROT_READ|PROT_WRITE, MAP_PRIVATE|MAP_ANONYMOUS, -1, 0) = 0x7f5c6466a000}}
{\color{darkgray}{arch_prctl(ARCH_SET_FS, 0x7f5c6466a740) = 0}}
{\color{darkgray}{set_tid_address(0x7f5c6466aa10)         = 23663}}
{\color{darkgray}{set_robust_list(0x7f5c6466aa20, 24)     = 0}}
{\color{darkgray}{rseq(0x7f5c6466b0e0, 0x20, 0, 0x53053053) = 0}}
{\color{darkgray}{mprotect(0x7f5c64883000, 16384, PROT_READ) = 0}}
{\color{darkgray}{mprotect(0x5611c4bde000, 4096, PROT_READ) = 0}}
{\color{darkgray}{mprotect(0x7f5c648f2000, 8192, PROT_READ) = 0}}
{\color{darkgray}{prlimit64(0, RLIMIT_STACK, NULL, {rlim_cur=8192*1024, rlim_max=RLIM64_INFINITY}) = 0}}
{\color{darkgray}{munmap(0x7f5c64896000, 135191)          = 0}}
{\color{darkgray}{getrandom("\textbackslash\textbackslash x7e\textbackslash x74\textbackslash x62\textbackslash xbc\textbackslash x66\textbackslash x05\textbackslash x81\textbackslash xf8", 8, GRND_NONBLOCK) = 8}}
{\color{darkgray}{brk(NULL)                               = 0x5611c6a38000}}
{\color{darkgray}{brk(0x5611c6a59000)                     = 0x5611c6a59000}}
{\color{darkgray}{newfstatat(1</dev/pts/0<char 136:0>>, "", {st_mode=S_IFCHR|0620, st_rdev=makedev(0x88, 0), ...}, AT_EMPTY_PATH) = 0}}
{\color{darkgray}{openat(AT_FDCWD</>, "/dev/null", O_RDONLY) = 3</dev/null<char 1:3>>}}
{\color{darkgray}{newfstatat(3</dev/null<char 1:3>>, "", {st_mode=S_IFCHR|0666, st_rdev=makedev(0x1, 0x3), ...}, AT_EMPTY_PATH) = 0}}
{\color{darkgray}{fadvise64(3</dev/null<char 1:3>>, 0, 0, POSIX_FADV_SEQUENTIAL) = 0}}
{\color{darkgray}{mmap(NULL, 139264, PROT_READ|PROT_WRITE, MAP_PRIVATE|MAP_ANONYMOUS, -1, 0) = 0x7f5c64896000}}
{\color{darkgray}{read(3</dev/null<char 1:3>>, "", 131072) = 0}}
{\color{darkgray}{munmap(0x7f5c64896000, 139264)          = 0}}
{\color{darkgray}{close(3</dev/null<char 1:3>>)           = 0}}
{\color{darkgray}{close(1</dev/pts/0<char 136:0>>)        = 0}}
{\color{darkgray}{close(2</dev/pts/0<char 136:0>>)        = 0}}
{\color{darkgray}{exit_group(0)                           = ?}}
{\color{darkgray}{+++ exited with 0 +++}}
\end{Verbatim}
}}
\endinput  %  ==  ==  ==  ==  ==  ==  ==  ==  ==

\subsection{\strace}
\label{sec:strace}
The \strace \ command is very powerful and the following examples.

		\subsubsection{Trace System Calls To A Given Path}
{\footnotesize{
\begin{Verbatim}[commandchars=\\\{\}]
{\color{darkgray}{root@169e8b2c1ae3:/#}} strace -P /etc/ld.so.cache ls /dev/null 
{\color{darkgray}{openat(AT_FDCWD, "/etc/ld.so.cache", O_RDONLY|O_CLOEXEC) = 3}}
{\color{darkgray}{newfstatat(3, "", {st_mode=S_IFREG|0644, st_size=135191, ...}, AT_EMPTY_PATH) = 0}}
{\color{darkgray}{mmap(NULL, 135191, PROT_READ, MAP_PRIVATE, 3, 0) = 0x7f03bba95000}}
{\color{darkgray}{close(3)                                = 0}}
{\color{darkgray}{/dev/null}}
{\color{darkgray}{+++ exited with 0 +++}}
\end{Verbatim}
}}


		\subsubsection{Inventory time, calls, and errors for every system call}
{\footnotesize{
\begin{Verbatim}[commandchars=\\\{\}]
{\color{darkgray}{root@169e8b2c1ae3:/}}# strace -c ls > /dev/null
{\color{darkgray}{\escapepercent time     seconds  usecs/call     calls    errors syscall}}
{\color{darkgray}{------ ----------- ----------- --------- --------- ----------------}}
{\color{darkgray}{ 71.76    0.013546        6773         2           getdents64}}
{\color{darkgray}{  7.85    0.001482         247         6           openat}}
{\color{darkgray}{  4.88    0.000922         922         1           execve}}
{\color{darkgray}{  4.44    0.000839          49        17           mmap}}
{\color{darkgray}{  1.84    0.000347          43         8           close}}
{\color{darkgray}{  1.48    0.000279          39         7           mprotect}}
{\color{darkgray}{  1.40    0.000265          37         7           newfstatat}}
{\color{darkgray}{  1.26    0.000237          47         5           read}}
{\color{darkgray}{  0.94    0.000178          44         4           pread64}}
{\color{darkgray}{  0.77    0.000145          48         3           brk}}
{\color{darkgray}{  0.57    0.000108          36         3         3 ioctl}}
{\color{darkgray}{  0.49    0.000092          46         2         2 statfs}}
{\color{darkgray}{  0.47    0.000088          44         2         2 access}}
{\color{darkgray}{  0.34    0.000065          32         2         1 arch_prctl}}
{\color{darkgray}{  0.34    0.000065          65         1           getrandom}}
{\color{darkgray}{  0.32    0.000061          61         1           munmap}}
{\color{darkgray}{  0.18    0.000034          34         1           rseq}}
{\color{darkgray}{  0.17    0.000032          32         1           set_robust_list}}
{\color{darkgray}{  0.16    0.000031          31         1           write}}
{\color{darkgray}{  0.16    0.000031          31         1           set_tid_address}}
{\color{darkgray}{  0.16    0.000031          31         1           prlimit64}}
{\color{darkgray}{------ ----------- ----------- --------- --------- ----------------}}
{\color{darkgray}{100.00    0.018878         248        76         8 total}}
\end{Verbatim}
}}

		\subsubsection{Identify Information Associated With File Descriptorsl}
{\footnotesize{
\begin{Verbatim}[commandchars=\\\{\}]
{\color{darkgray}{root@169e8b2c1ae3:/}}# strace -yy cat /dev/null
{\color{darkgray}{execve("/usr/bin/cat", ["cat", "/dev/null"], 0x7fffb8b235d0 /* 10 vars */) = 0}}
{\color{darkgray}{brk(NULL)                               = 0x5611c6a38000}}
{\color{darkgray}{arch_prctl(0x3001 /* ARCH_??? */, 0x7ffeede990c0) = -1 EINVAL (Invalid argument)}}
{\color{darkgray}{mmap(NULL, 8192, PROT_READ|PROT_WRITE, MAP_PRIVATE|MAP_ANONYMOUS, -1, 0) = 0x7f5c648b8000}}
{\color{darkgray}{access("/etc/ld.so.preload", R_OK)      = -1 ENOENT (No such file or directory)}}
{\color{darkgray}{openat(AT_FDCWD</>, "/etc/ld.so.cache", O_RDONLY|O_CLOEXEC) = 3</etc/ld.so.cache>}}
{\color{darkgray}{newfstatat(3</etc/ld.so.cache>, "", {st_mode=S_IFREG|0644, st_size=135191, ...}, AT_EMPTY_PATH) = 0}}
{\color{darkgray}{mmap(NULL, 135191, PROT_READ, MAP_PRIVATE, 3</etc/ld.so.cache>, 0) = 0x7f5c64896000}}
{\color{darkgray}{close(3</etc/ld.so.cache>)              = 0}}
{\color{darkgray}{openat(AT_FDCWD</>, "/lib/x86_64-linux-gnu/libc.so.6", O_RDONLY|O_CLOEXEC) = 3</usr/lib/x86_64-linux-gnu/libc.so.6>}}
{\color{darkgray}{read(3</usr/lib/x86_64-linux-gnu/libc.so.6>, "\textbackslash 177ELF\textbackslash2\textbackslash1\textbackslash1\textbackslash3\textbackslash0\textbackslash0\textbackslash0\textbackslash0\textbackslash0\textbackslash0\textbackslash0\textbackslash0\textbackslash3\textbackslash0>\textbackslash0\textbackslash1\textbackslash0\textbackslash0\textbackslash0P\textbackslash237\textbackslash2\textbackslash0\textbackslash0\textbackslash0\textbackslash0\textbackslash0"..., 832) = 832}}
{\color{darkgray}{pread64(3</usr/lib/x86_64-linux-gnu/libc.so.6>, "\textbackslash6\textbackslash0\textbackslash0\textbackslash0\textbackslash4\textbackslash0\textbackslash0\textbackslash0@\textbackslash0\textbackslash0\textbackslash0\textbackslash0\textbackslash0\textbackslash0\textbackslash0@\textbackslash0\textbackslash0\textbackslash0\textbackslash0\textbackslash0\textbackslash0\textbackslash0@\textbackslash0\textbackslash0\textbackslash0\textbackslash0\textbackslash0\textbackslash0\textbackslash0"..., 784, 64) = 784}}
{\color{darkgray}{pread64(3</usr/lib/x86_64-linux-gnu/libc.so.6>, "\textbackslash4\textbackslash0\textbackslash0\textbackslash0 \textbackslash0\textbackslash0\textbackslash0\textbackslash5\textbackslash0\textbackslash0\textbackslash0GNU\textbackslash0\textbackslash2\textbackslash0\textbackslash0\textbackslash300\textbackslash4\textbackslash0\textbackslash0\textbackslash0\textbackslash3\textbackslash0\textbackslash0\textbackslash0\textbackslash0\textbackslash0\textbackslash0\textbackslash0"..., 48, 848) = 48}}
{\color{darkgray}{pread64(3</usr/lib/x86_64-linux-gnu/libc.so.6>, "\textbackslash4\textbackslash0\textbackslash0\textbackslash0\textbackslash24\textbackslash0\textbackslash0\textbackslash0\textbackslash3\textbackslash0\textbackslash0\textbackslash0GNU\textbackslash0I\textbackslash17\textbackslash357\textbackslash204\textbackslash3$\textbackslash f\textbackslash221\textbackslash2039x\textbackslash324\textbackslash224\textbackslash323\textbackslash236S"..., 68, 896) = 68}}
{\color{darkgray}{newfstatat(3</usr/lib/x86_64-linux-gnu/libc.so.6>, "", {st_mode=S_IFREG|0755, st_size=2220400, ...}, AT_EMPTY_PATH) = 0}}
{\color{darkgray}{pread64(3</usr/lib/x86_64-linux-gnu/libc.so.6>, "\textbackslash6\textbackslash0\textbackslash0\textbackslash0\textbackslash4\textbackslash0\textbackslash0\textbackslash0@\textbackslash0\textbackslash0\textbackslash0\textbackslash0\textbackslash0\textbackslash0\textbackslash0@\textbackslash0\textbackslash0\textbackslash0\textbackslash0\textbackslash0\textbackslash0\textbackslash0@\textbackslash0\textbackslash0\textbackslash0\textbackslash0\textbackslash0\textbackslash0\textbackslash0"..., 784, 64) = 784}}
{\color{darkgray}{mmap(NULL, 2264656, PROT_READ, MAP_PRIVATE|MAP_DENYWRITE, 3</usr/lib/x86_64-linux-gnu/libc.so.6>, 0) = 0x7f5c6466d000}}
{\color{darkgray}{mprotect(0x7f5c64695000, 2023424, PROT_NONE) = 0}}
{\color{darkgray}{mmap(0x7f5c64695000, 1658880, PROT_READ|PROT_EXEC, MAP_PRIVATE|MAP_FIXED|MAP_DENYWRITE, 3</usr/lib/x86_64-linux-gnu/libc.so.6>, 0x28000) = 0x7f5c64695000}}
{\color{darkgray}{mmap(0x7f5c6482a000, 360448, PROT_READ, MAP_PRIVATE|MAP_FIXED|MAP_DENYWRITE, 3</usr/lib/x86_64-linux-gnu/libc.so.6>, 0x1bd000) = 0x7f5c6482a000}}
{\color{darkgray}{mmap(0x7f5c64883000, 24576, PROT_READ|PROT_WRITE, MAP_PRIVATE|MAP_FIXED|MAP_DENYWRITE, 3</usr/lib/x86_64-linux-gnu/libc.so.6>, 0x215000) = 0x7f5c64883000}}
{\color{darkgray}{mmap(0x7f5c64889000, 52816, PROT_READ|PROT_WRITE, MAP_PRIVATE|MAP_FIXED|MAP_ANONYMOUS, -1, 0) = 0x7f5c64889000}}
{\color{darkgray}{close(3</usr/lib/x86_64-linux-gnu/libc.so.6>) = 0}}
{\color{darkgray}{mmap(NULL, 12288, PROT_READ|PROT_WRITE, MAP_PRIVATE|MAP_ANONYMOUS, -1, 0) = 0x7f5c6466a000}}
{\color{darkgray}{arch_prctl(ARCH_SET_FS, 0x7f5c6466a740) = 0}}
{\color{darkgray}{set_tid_address(0x7f5c6466aa10)         = 23663}}
{\color{darkgray}{set_robust_list(0x7f5c6466aa20, 24)     = 0}}
{\color{darkgray}{rseq(0x7f5c6466b0e0, 0x20, 0, 0x53053053) = 0}}
{\color{darkgray}{mprotect(0x7f5c64883000, 16384, PROT_READ) = 0}}
{\color{darkgray}{mprotect(0x5611c4bde000, 4096, PROT_READ) = 0}}
{\color{darkgray}{mprotect(0x7f5c648f2000, 8192, PROT_READ) = 0}}
{\color{darkgray}{prlimit64(0, RLIMIT_STACK, NULL, {rlim_cur=8192*1024, rlim_max=RLIM64_INFINITY}) = 0}}
{\color{darkgray}{munmap(0x7f5c64896000, 135191)          = 0}}
{\color{darkgray}{getrandom("\textbackslash\textbackslash x7e\textbackslash x74\textbackslash x62\textbackslash xbc\textbackslash x66\textbackslash x05\textbackslash x81\textbackslash xf8", 8, GRND_NONBLOCK) = 8}}
{\color{darkgray}{brk(NULL)                               = 0x5611c6a38000}}
{\color{darkgray}{brk(0x5611c6a59000)                     = 0x5611c6a59000}}
{\color{darkgray}{newfstatat(1</dev/pts/0<char 136:0>>, "", {st_mode=S_IFCHR|0620, st_rdev=makedev(0x88, 0), ...}, AT_EMPTY_PATH) = 0}}
{\color{darkgray}{openat(AT_FDCWD</>, "/dev/null", O_RDONLY) = 3</dev/null<char 1:3>>}}
{\color{darkgray}{newfstatat(3</dev/null<char 1:3>>, "", {st_mode=S_IFCHR|0666, st_rdev=makedev(0x1, 0x3), ...}, AT_EMPTY_PATH) = 0}}
{\color{darkgray}{fadvise64(3</dev/null<char 1:3>>, 0, 0, POSIX_FADV_SEQUENTIAL) = 0}}
{\color{darkgray}{mmap(NULL, 139264, PROT_READ|PROT_WRITE, MAP_PRIVATE|MAP_ANONYMOUS, -1, 0) = 0x7f5c64896000}}
{\color{darkgray}{read(3</dev/null<char 1:3>>, "", 131072) = 0}}
{\color{darkgray}{munmap(0x7f5c64896000, 139264)          = 0}}
{\color{darkgray}{close(3</dev/null<char 1:3>>)           = 0}}
{\color{darkgray}{close(1</dev/pts/0<char 136:0>>)        = 0}}
{\color{darkgray}{close(2</dev/pts/0<char 136:0>>)        = 0}}
{\color{darkgray}{exit_group(0)                           = ?}}
{\color{darkgray}{+++ exited with 0 +++}}
\end{Verbatim}
}}
\endinput  %  ==  ==  ==  ==  ==  ==  ==  ==  ==
		% % % \input{./sections/ssec-strings}

\subsection{\strings}
\label{sec:strings}

Print the sequences of printable characters in files using \strings.

		\subsubsection{Applied To Compiled Object File}
{\footnotesize{
\begin{Verbatim}[commandchars=\\\{\}]
{\color{darkgray}{dantopa@Quaxolotl.local:vv}} $ strings -a m-precision-definitions.o
{\color{darkgray}{GNU Fortran2008 12.1.0 -fPIC -feliminate-unused-debug-symbols -mmacosx-version-min=12.0.0 -mtune=core2 -g -Og -fno-realloc-lhs -ffpe-trap=denormal,invalid,zero -fbacktrace -fmax-errors=5 -fcheck=all -fcheck=do -fcheck=pointer -fno-protect-parens -faggressive-function-elimination -finit-derived -fintrinsic-modules-path /opt/local/lib/gcc12/gcc/x86_64-apple-darwin21/12.1.0/finclude m-precision-definitions.f08}}
{\color{darkgray}{/Volumes/T7-Touch/repos/github/f/dtra/vv}}
{\color{darkgray}{aint}}
{\color{darkgray}{integer(kind=4)}}
{\color{darkgray}{ascii}}
{\color{darkgray}{kinda}}
{\color{darkgray}{lint}}
{\color{darkgray}{sint}}
{\color{darkgray}{zint}}
{\color{darkgray}{integer(kind=4)}}
{\color{darkgray}{m-precision-definitions.f08}}
\end{Verbatim}
}}


		\subsubsection{Applied To Object Class}
{\footnotesize{
\begin{Verbatim}[commandchars=\\\{\}]
{\color{darkgray}{dantopa@Quaxolotl.local:bravo }}$ strings m-cl-soln-basic.o
{\color{darkgray}{Error allocating %lu bytes}}
{\color{darkgray}{In file 'm-cl-soln-basic.f08', around line 99}}
{\color{darkgray}{Index '%ld' of dimension 1 of array 'sizes' above upper bound of %ld}}
{\color{darkgray}{At line 98 of file m-cl-soln-basic.f08}}
{\color{darkgray}{Loop iterates infinitely}}
{\color{darkgray}{Index '%ld' of dimension 1 of array 'strides' below lower bound of %ld}}
{\color{darkgray}{Index '%ld' of dimension 1 of array 'strides' above upper bound of %ld}}
{\color{darkgray}{Index '%ld' of dimension 1 of array 'sizes' below lower bound of %ld}}
{\color{darkgray}{Recursive call to nonrecursive procedure 'solnscalingparameter_sub'}}
{\color{darkgray}{At line 69 of file m-cl-soln-basic.f08}}
{\color{darkgray}{m-cl-soln-basic.f08}}
{\color{darkgray}{Array bound mismatch for dimension 1 of array 'u' (%ld/%ld)}}
{\color{darkgray}{At line 85 of file m-cl-soln-basic.f08}}
{\color{darkgray}{Recursive call to nonrecursive procedure 'solncomputesigma_sub'}}
{\color{darkgray}{At line 53 of file m-cl-soln-basic.f08}}
{\color{darkgray}{Array bound mismatch for dimension 1 of array 'me' (%ld/%ld)}}
{\color{darkgray}{At line 60 of file m-cl-soln-basic.f08}}
{\color{darkgray}{Array bound mismatch for dimension 1 of array 'measurements' (%ld/%ld)}}
{\color{darkgray}{At line 61 of file m-cl-soln-basic.f08}}
{\color{darkgray}{Recursive call to nonrecursive procedure 'solnbasic_sub'}}
{\color{darkgray}{At line 33 of file m-cl-soln-basic.f08}}
{\color{darkgray}{Allocatable actual argument 'measurements' is not allocated}}
{\color{darkgray}{At line 38 of file m-cl-soln-basic.f08}}
{\color{darkgray}{Potential error: || u || = }}
{\color{darkgray}{Subroutine: solnScalingParameter_sub}}
{\color{darkgray}{Module:     mClassSolutionBasic}}
{\color{darkgray}{Small size may magnifiy errors in solution parameter a}}
{\color{darkgray}{GNU Fortran2008 13.2.0 -fPIC -feliminate-unused-debug-symbols -mmacosx-version-min=14.0.0 -mtune=core2 -g -Og -fno-realloc-lhs -ffpe-trap=denormal,invalid,zero -fbacktrace -fmax-errors=5 -fcheck=all -fcheck=do -fcheck=pointer -fno-protect-parens -faggressive-function-elimination -finit-derived -fintrinsic-modules-path /opt/local/lib/gcc13/gcc/x86_64-apple-darwin23/13.2.0/finclude}}
{\color{darkgray}{m-cl-soln-basic.f08}}
{\color{darkgray}{/Volumes/T7-Touch/repos/github/f/projects/fireball/bravo}}
{\color{darkgray}{sigma}}
{\color{darkgray}{count}}
{\color{darkgray}{phimin}}
{\color{darkgray}{phi2min}}
{\color{darkgray}{sigmasc}}
{\color{darkgray}{residualerror}}
{\color{darkgray}{upsilon}}
{\color{darkgray}{alloc}}
{\color{darkgray}{real(kind=8)}}
{\color{darkgray}{toolkitallocation}}
{\color{darkgray}{requestedgb}}
{\color{darkgray}{alloc_status}}
{\color{darkgray}{sizeelementbits}}
{\color{darkgray}{numelements}}
{\color{darkgray}{numrows}}
{\color{darkgray}{numcols}}
{\color{darkgray}{alloc_message}}
{\color{darkgray}{mykind}}
{\color{darkgray}{integer(kind=4)}}
{\color{darkgray}{__def_init_mclasssolutionbasic_Solnbasic}}
{\color{darkgray}{__mclasssolutionbasic_MOD___def_init_mclasssolutionbasic_Solnbasic}}
{\color{darkgray}{integer(kind=8)}}
{\color{darkgray}{__vtype_mclasssolutionbasic_Solnbasic}}
{\color{darkgray}{_hash}}
{\color{darkgray}{_size}}
{\color{darkgray}{_extends}}
{\color{darkgray}{_def_init}}
{\color{darkgray}{_copy}}
{\color{darkgray}{_final}}
{\color{darkgray}{_deallocate}}
{\color{darkgray}{solncomputesigma}}
{\color{darkgray}{solnscalingparameter}}
{\color{darkgray}{__class_mclasssolutionbasic_Solnbasic_t}}
{\color{darkgray}{_data}}
{\color{darkgray}{_vptr}}
{\color{darkgray}{data}}
{\color{darkgray}{maxradius}}
{\color{darkgray}{time}}
{\color{darkgray}{radius}}
{\color{darkgray}{alloc}}
{\color{darkgray}{__vtab_mclasssolutionbasic_Solnbasic}}
{\color{darkgray}{__mclasssolutionbasic_MOD___vtab_mclasssolutionbasic_Solnbasic}}
{\color{darkgray}{logical(kind=4)}}
{\color{darkgray}{solnbasic_sub}}
{\color{darkgray}{!#__mclasssolutionbasic_MOD_solnbasic_sub}}
{\color{darkgray}{_descriptor}}
{\color{darkgray}{solncomputesigma_sub}}
{\color{darkgray}{5*__mclasssolutionbasic_MOD_solncomputesigma_sub}}
{\color{darkgray}{solnscalingparameter_sub}}
{\color{darkgray}{E.__mclasssolutionbasic_MOD_solnscalingparameter_sub}}
{\color{darkgray}{normu}}
{\color{darkgray}{__copy_mclasssolutionbasic_Solnbasic}}
{\color{darkgray}{__mclasssolutionbasic_MOD___copy_mclasssolutionbasic_Solnbasic}}
{\color{darkgray}{__final_mclasssolutionbasic_Solnbasic}}
{\color{darkgray}{__mclasssolutionbasic_MOD___final_mclasssolutionbasic_Solnbasic}}
{\color{darkgray}{array}}
{\color{darkgray}{byte_stride}}
{\color{darkgray}{fini_coarray}}
{\color{darkgray}{idx2}}
{\color{darkgray}{ignore}}
{\color{darkgray}{is_contiguous}}
{\color{darkgray}{nelem}}
{\color{darkgray}{offset}}
{\color{darkgray}{ptr2}}
{\color{darkgray}{sizes}}
{\color{darkgray}{strides}}
{\color{darkgray}{__result___final_mclasssoluti}}
l{\color{darkgray}{ogical(kind=1)}}
{\color{darkgray}{?}}
{\color{darkgray}{ 4}}
{\color{darkgray}{__def_init_mclasssolutionbasic_Solnbasic}}
{\color{darkgray}{__vtab_mclasssolutionbasic_Solnbasic}}
{\color{darkgray}{__def_init_mclasssolutionbasic_Solnbasic}}
{\color{darkgray}{__vtab_mclasssolutionbasic_Solnbasic}}
{\color{darkgray}{solnbasic_sub}}
{\color{darkgray}{solncomputesigma_sub}}
{\color{darkgray}{solnscalingparameter_sub}}
{\color{darkgray}{__copy_mclasssolutionbasic_Solnbasic}}
{\color{darkgray}{__final_mclasssolutionbasic_Solnbasic}}
{\color{darkgray}{real(kind=8)}}
{\color{darkgray}{integer(kind=4)}}
{\color{darkgray}{mclasssolutionbasic.toolkitallocation}}
{\color{darkgray}{mclasssolutionbasic.solnbasic}}
{\color{darkgray}{integer(kind=8)}}
{\color{darkgray}{mclasssolutionbasic.__class_mclasssolutionbasic_Solnbasic_t}}
{\color{darkgray}{mclasssolutionbasic.data}}
{\color{darkgray}{mclasssolutionbasic.__vtype_mclasssolutionbasic_Solnbasic}}
{\color{darkgray}{logical(kind=4)}}
{\color{darkgray}{logical(kind=1)}}
{\color{darkgray}{m-cl-soln-basic.f08}}
{\color{darkgray}{measurements}}
{\color{darkgray}{descriptor}}
{\color{darkgray}{solnbasic}}
\end{Verbatim}
}}


\endinput  %  ==  ==  ==  ==  ==  ==  ==  ==  ==


\subsection{\strings}
\label{sec:strings}

Print the sequences of printable characters in files using \strings.

		\subsubsection{Applied To Compiled Object File}
{\footnotesize{
\begin{Verbatim}[commandchars=\\\{\}]
{\color{darkgray}{dantopa@Quaxolotl.local:vv}} $ strings -a m-precision-definitions.o
{\color{darkgray}{GNU Fortran2008 12.1.0 -fPIC -feliminate-unused-debug-symbols -mmacosx-version-min=12.0.0 -mtune=core2 -g -Og -fno-realloc-lhs -ffpe-trap=denormal,invalid,zero -fbacktrace -fmax-errors=5 -fcheck=all -fcheck=do -fcheck=pointer -fno-protect-parens -faggressive-function-elimination -finit-derived -fintrinsic-modules-path /opt/local/lib/gcc12/gcc/x86_64-apple-darwin21/12.1.0/finclude m-precision-definitions.f08}}
{\color{darkgray}{/Volumes/T7-Touch/repos/github/f/dtra/vv}}
{\color{darkgray}{aint}}
{\color{darkgray}{integer(kind=4)}}
{\color{darkgray}{ascii}}
{\color{darkgray}{kinda}}
{\color{darkgray}{lint}}
{\color{darkgray}{sint}}
{\color{darkgray}{zint}}
{\color{darkgray}{integer(kind=4)}}
{\color{darkgray}{m-precision-definitions.f08}}
\end{Verbatim}
}}


		\subsubsection{Applied To Object Class}
{\footnotesize{
\begin{Verbatim}[commandchars=\\\{\}]
{\color{darkgray}{dantopa@Quaxolotl.local:bravo }}$ strings m-cl-soln-basic.o
{\color{darkgray}{Error allocating %lu bytes}}
{\color{darkgray}{In file 'm-cl-soln-basic.f08', around line 99}}
{\color{darkgray}{Index '%ld' of dimension 1 of array 'sizes' above upper bound of %ld}}
{\color{darkgray}{At line 98 of file m-cl-soln-basic.f08}}
{\color{darkgray}{Loop iterates infinitely}}
{\color{darkgray}{Index '%ld' of dimension 1 of array 'strides' below lower bound of %ld}}
{\color{darkgray}{Index '%ld' of dimension 1 of array 'strides' above upper bound of %ld}}
{\color{darkgray}{Index '%ld' of dimension 1 of array 'sizes' below lower bound of %ld}}
{\color{darkgray}{Recursive call to nonrecursive procedure 'solnscalingparameter_sub'}}
{\color{darkgray}{At line 69 of file m-cl-soln-basic.f08}}
{\color{darkgray}{m-cl-soln-basic.f08}}
{\color{darkgray}{Array bound mismatch for dimension 1 of array 'u' (%ld/%ld)}}
{\color{darkgray}{At line 85 of file m-cl-soln-basic.f08}}
{\color{darkgray}{Recursive call to nonrecursive procedure 'solncomputesigma_sub'}}
{\color{darkgray}{At line 53 of file m-cl-soln-basic.f08}}
{\color{darkgray}{Array bound mismatch for dimension 1 of array 'me' (%ld/%ld)}}
{\color{darkgray}{At line 60 of file m-cl-soln-basic.f08}}
{\color{darkgray}{Array bound mismatch for dimension 1 of array 'measurements' (%ld/%ld)}}
{\color{darkgray}{At line 61 of file m-cl-soln-basic.f08}}
{\color{darkgray}{Recursive call to nonrecursive procedure 'solnbasic_sub'}}
{\color{darkgray}{At line 33 of file m-cl-soln-basic.f08}}
{\color{darkgray}{Allocatable actual argument 'measurements' is not allocated}}
{\color{darkgray}{At line 38 of file m-cl-soln-basic.f08}}
{\color{darkgray}{Potential error: || u || = }}
{\color{darkgray}{Subroutine: solnScalingParameter_sub}}
{\color{darkgray}{Module:     mClassSolutionBasic}}
{\color{darkgray}{Small size may magnifiy errors in solution parameter a}}
{\color{darkgray}{GNU Fortran2008 13.2.0 -fPIC -feliminate-unused-debug-symbols -mmacosx-version-min=14.0.0 -mtune=core2 -g -Og -fno-realloc-lhs -ffpe-trap=denormal,invalid,zero -fbacktrace -fmax-errors=5 -fcheck=all -fcheck=do -fcheck=pointer -fno-protect-parens -faggressive-function-elimination -finit-derived -fintrinsic-modules-path /opt/local/lib/gcc13/gcc/x86_64-apple-darwin23/13.2.0/finclude}}
{\color{darkgray}{m-cl-soln-basic.f08}}
{\color{darkgray}{/Volumes/T7-Touch/repos/github/f/projects/fireball/bravo}}
{\color{darkgray}{sigma}}
{\color{darkgray}{count}}
{\color{darkgray}{phimin}}
{\color{darkgray}{phi2min}}
{\color{darkgray}{sigmasc}}
{\color{darkgray}{residualerror}}
{\color{darkgray}{upsilon}}
{\color{darkgray}{alloc}}
{\color{darkgray}{real(kind=8)}}
{\color{darkgray}{toolkitallocation}}
{\color{darkgray}{requestedgb}}
{\color{darkgray}{alloc_status}}
{\color{darkgray}{sizeelementbits}}
{\color{darkgray}{numelements}}
{\color{darkgray}{numrows}}
{\color{darkgray}{numcols}}
{\color{darkgray}{alloc_message}}
{\color{darkgray}{mykind}}
{\color{darkgray}{integer(kind=4)}}
{\color{darkgray}{__def_init_mclasssolutionbasic_Solnbasic}}
{\color{darkgray}{__mclasssolutionbasic_MOD___def_init_mclasssolutionbasic_Solnbasic}}
{\color{darkgray}{integer(kind=8)}}
{\color{darkgray}{__vtype_mclasssolutionbasic_Solnbasic}}
{\color{darkgray}{_hash}}
{\color{darkgray}{_size}}
{\color{darkgray}{_extends}}
{\color{darkgray}{_def_init}}
{\color{darkgray}{_copy}}
{\color{darkgray}{_final}}
{\color{darkgray}{_deallocate}}
{\color{darkgray}{solncomputesigma}}
{\color{darkgray}{solnscalingparameter}}
{\color{darkgray}{__class_mclasssolutionbasic_Solnbasic_t}}
{\color{darkgray}{_data}}
{\color{darkgray}{_vptr}}
{\color{darkgray}{data}}
{\color{darkgray}{maxradius}}
{\color{darkgray}{time}}
{\color{darkgray}{radius}}
{\color{darkgray}{alloc}}
{\color{darkgray}{__vtab_mclasssolutionbasic_Solnbasic}}
{\color{darkgray}{__mclasssolutionbasic_MOD___vtab_mclasssolutionbasic_Solnbasic}}
{\color{darkgray}{logical(kind=4)}}
{\color{darkgray}{solnbasic_sub}}
{\color{darkgray}{!#__mclasssolutionbasic_MOD_solnbasic_sub}}
{\color{darkgray}{_descriptor}}
{\color{darkgray}{solncomputesigma_sub}}
{\color{darkgray}{5*__mclasssolutionbasic_MOD_solncomputesigma_sub}}
{\color{darkgray}{solnscalingparameter_sub}}
{\color{darkgray}{E.__mclasssolutionbasic_MOD_solnscalingparameter_sub}}
{\color{darkgray}{normu}}
{\color{darkgray}{__copy_mclasssolutionbasic_Solnbasic}}
{\color{darkgray}{__mclasssolutionbasic_MOD___copy_mclasssolutionbasic_Solnbasic}}
{\color{darkgray}{__final_mclasssolutionbasic_Solnbasic}}
{\color{darkgray}{__mclasssolutionbasic_MOD___final_mclasssolutionbasic_Solnbasic}}
{\color{darkgray}{array}}
{\color{darkgray}{byte_stride}}
{\color{darkgray}{fini_coarray}}
{\color{darkgray}{idx2}}
{\color{darkgray}{ignore}}
{\color{darkgray}{is_contiguous}}
{\color{darkgray}{nelem}}
{\color{darkgray}{offset}}
{\color{darkgray}{ptr2}}
{\color{darkgray}{sizes}}
{\color{darkgray}{strides}}
{\color{darkgray}{__result___final_mclasssoluti}}
l{\color{darkgray}{ogical(kind=1)}}
{\color{darkgray}{?}}
{\color{darkgray}{ 4}}
{\color{darkgray}{__def_init_mclasssolutionbasic_Solnbasic}}
{\color{darkgray}{__vtab_mclasssolutionbasic_Solnbasic}}
{\color{darkgray}{__def_init_mclasssolutionbasic_Solnbasic}}
{\color{darkgray}{__vtab_mclasssolutionbasic_Solnbasic}}
{\color{darkgray}{solnbasic_sub}}
{\color{darkgray}{solncomputesigma_sub}}
{\color{darkgray}{solnscalingparameter_sub}}
{\color{darkgray}{__copy_mclasssolutionbasic_Solnbasic}}
{\color{darkgray}{__final_mclasssolutionbasic_Solnbasic}}
{\color{darkgray}{real(kind=8)}}
{\color{darkgray}{integer(kind=4)}}
{\color{darkgray}{mclasssolutionbasic.toolkitallocation}}
{\color{darkgray}{mclasssolutionbasic.solnbasic}}
{\color{darkgray}{integer(kind=8)}}
{\color{darkgray}{mclasssolutionbasic.__class_mclasssolutionbasic_Solnbasic_t}}
{\color{darkgray}{mclasssolutionbasic.data}}
{\color{darkgray}{mclasssolutionbasic.__vtype_mclasssolutionbasic_Solnbasic}}
{\color{darkgray}{logical(kind=4)}}
{\color{darkgray}{logical(kind=1)}}
{\color{darkgray}{m-cl-soln-basic.f08}}
{\color{darkgray}{measurements}}
{\color{darkgray}{descriptor}}
{\color{darkgray}{solnbasic}}
\end{Verbatim}
}}


\endinput  %  ==  ==  ==  ==  ==  ==  ==  ==  ==


\subsection{\strings}
\label{sec:strings}

Print the sequences of printable characters in files using \strings.

		\subsubsection{Applied To Compiled Object File}
{\footnotesize{
\begin{Verbatim}[commandchars=\\\{\}]
{\color{darkgray}{dantopa@Quaxolotl.local:vv}} $ strings -a m-precision-definitions.o
{\color{darkgray}{GNU Fortran2008 12.1.0 -fPIC -feliminate-unused-debug-symbols -mmacosx-version-min=12.0.0 -mtune=core2 -g -Og -fno-realloc-lhs -ffpe-trap=denormal,invalid,zero -fbacktrace -fmax-errors=5 -fcheck=all -fcheck=do -fcheck=pointer -fno-protect-parens -faggressive-function-elimination -finit-derived -fintrinsic-modules-path /opt/local/lib/gcc12/gcc/x86_64-apple-darwin21/12.1.0/finclude m-precision-definitions.f08}}
{\color{darkgray}{/Volumes/T7-Touch/repos/github/f/dtra/vv}}
{\color{darkgray}{aint}}
{\color{darkgray}{integer(kind=4)}}
{\color{darkgray}{ascii}}
{\color{darkgray}{kinda}}
{\color{darkgray}{lint}}
{\color{darkgray}{sint}}
{\color{darkgray}{zint}}
{\color{darkgray}{integer(kind=4)}}
{\color{darkgray}{m-precision-definitions.f08}}
\end{Verbatim}
}}


		\subsubsection{Applied To Object Class}
{\footnotesize{
\begin{Verbatim}[commandchars=\\\{\}]
{\color{darkgray}{dantopa@Quaxolotl.local:bravo }}$ strings m-cl-soln-basic.o
{\color{darkgray}{Error allocating %lu bytes}}
{\color{darkgray}{In file 'm-cl-soln-basic.f08', around line 99}}
{\color{darkgray}{Index '%ld' of dimension 1 of array 'sizes' above upper bound of %ld}}
{\color{darkgray}{At line 98 of file m-cl-soln-basic.f08}}
{\color{darkgray}{Loop iterates infinitely}}
{\color{darkgray}{Index '%ld' of dimension 1 of array 'strides' below lower bound of %ld}}
{\color{darkgray}{Index '%ld' of dimension 1 of array 'strides' above upper bound of %ld}}
{\color{darkgray}{Index '%ld' of dimension 1 of array 'sizes' below lower bound of %ld}}
{\color{darkgray}{Recursive call to nonrecursive procedure 'solnscalingparameter_sub'}}
{\color{darkgray}{At line 69 of file m-cl-soln-basic.f08}}
{\color{darkgray}{m-cl-soln-basic.f08}}
{\color{darkgray}{Array bound mismatch for dimension 1 of array 'u' (%ld/%ld)}}
{\color{darkgray}{At line 85 of file m-cl-soln-basic.f08}}
{\color{darkgray}{Recursive call to nonrecursive procedure 'solncomputesigma_sub'}}
{\color{darkgray}{At line 53 of file m-cl-soln-basic.f08}}
{\color{darkgray}{Array bound mismatch for dimension 1 of array 'me' (%ld/%ld)}}
{\color{darkgray}{At line 60 of file m-cl-soln-basic.f08}}
{\color{darkgray}{Array bound mismatch for dimension 1 of array 'measurements' (%ld/%ld)}}
{\color{darkgray}{At line 61 of file m-cl-soln-basic.f08}}
{\color{darkgray}{Recursive call to nonrecursive procedure 'solnbasic_sub'}}
{\color{darkgray}{At line 33 of file m-cl-soln-basic.f08}}
{\color{darkgray}{Allocatable actual argument 'measurements' is not allocated}}
{\color{darkgray}{At line 38 of file m-cl-soln-basic.f08}}
{\color{darkgray}{Potential error: || u || = }}
{\color{darkgray}{Subroutine: solnScalingParameter_sub}}
{\color{darkgray}{Module:     mClassSolutionBasic}}
{\color{darkgray}{Small size may magnifiy errors in solution parameter a}}
{\color{darkgray}{GNU Fortran2008 13.2.0 -fPIC -feliminate-unused-debug-symbols -mmacosx-version-min=14.0.0 -mtune=core2 -g -Og -fno-realloc-lhs -ffpe-trap=denormal,invalid,zero -fbacktrace -fmax-errors=5 -fcheck=all -fcheck=do -fcheck=pointer -fno-protect-parens -faggressive-function-elimination -finit-derived -fintrinsic-modules-path /opt/local/lib/gcc13/gcc/x86_64-apple-darwin23/13.2.0/finclude}}
{\color{darkgray}{m-cl-soln-basic.f08}}
{\color{darkgray}{/Volumes/T7-Touch/repos/github/f/projects/fireball/bravo}}
{\color{darkgray}{sigma}}
{\color{darkgray}{count}}
{\color{darkgray}{phimin}}
{\color{darkgray}{phi2min}}
{\color{darkgray}{sigmasc}}
{\color{darkgray}{residualerror}}
{\color{darkgray}{upsilon}}
{\color{darkgray}{alloc}}
{\color{darkgray}{real(kind=8)}}
{\color{darkgray}{toolkitallocation}}
{\color{darkgray}{requestedgb}}
{\color{darkgray}{alloc_status}}
{\color{darkgray}{sizeelementbits}}
{\color{darkgray}{numelements}}
{\color{darkgray}{numrows}}
{\color{darkgray}{numcols}}
{\color{darkgray}{alloc_message}}
{\color{darkgray}{mykind}}
{\color{darkgray}{integer(kind=4)}}
{\color{darkgray}{__def_init_mclasssolutionbasic_Solnbasic}}
{\color{darkgray}{__mclasssolutionbasic_MOD___def_init_mclasssolutionbasic_Solnbasic}}
{\color{darkgray}{integer(kind=8)}}
{\color{darkgray}{__vtype_mclasssolutionbasic_Solnbasic}}
{\color{darkgray}{_hash}}
{\color{darkgray}{_size}}
{\color{darkgray}{_extends}}
{\color{darkgray}{_def_init}}
{\color{darkgray}{_copy}}
{\color{darkgray}{_final}}
{\color{darkgray}{_deallocate}}
{\color{darkgray}{solncomputesigma}}
{\color{darkgray}{solnscalingparameter}}
{\color{darkgray}{__class_mclasssolutionbasic_Solnbasic_t}}
{\color{darkgray}{_data}}
{\color{darkgray}{_vptr}}
{\color{darkgray}{data}}
{\color{darkgray}{maxradius}}
{\color{darkgray}{time}}
{\color{darkgray}{radius}}
{\color{darkgray}{alloc}}
{\color{darkgray}{__vtab_mclasssolutionbasic_Solnbasic}}
{\color{darkgray}{__mclasssolutionbasic_MOD___vtab_mclasssolutionbasic_Solnbasic}}
{\color{darkgray}{logical(kind=4)}}
{\color{darkgray}{solnbasic_sub}}
{\color{darkgray}{!#__mclasssolutionbasic_MOD_solnbasic_sub}}
{\color{darkgray}{_descriptor}}
{\color{darkgray}{solncomputesigma_sub}}
{\color{darkgray}{5*__mclasssolutionbasic_MOD_solncomputesigma_sub}}
{\color{darkgray}{solnscalingparameter_sub}}
{\color{darkgray}{E.__mclasssolutionbasic_MOD_solnscalingparameter_sub}}
{\color{darkgray}{normu}}
{\color{darkgray}{__copy_mclasssolutionbasic_Solnbasic}}
{\color{darkgray}{__mclasssolutionbasic_MOD___copy_mclasssolutionbasic_Solnbasic}}
{\color{darkgray}{__final_mclasssolutionbasic_Solnbasic}}
{\color{darkgray}{__mclasssolutionbasic_MOD___final_mclasssolutionbasic_Solnbasic}}
{\color{darkgray}{array}}
{\color{darkgray}{byte_stride}}
{\color{darkgray}{fini_coarray}}
{\color{darkgray}{idx2}}
{\color{darkgray}{ignore}}
{\color{darkgray}{is_contiguous}}
{\color{darkgray}{nelem}}
{\color{darkgray}{offset}}
{\color{darkgray}{ptr2}}
{\color{darkgray}{sizes}}
{\color{darkgray}{strides}}
{\color{darkgray}{__result___final_mclasssoluti}}
l{\color{darkgray}{ogical(kind=1)}}
{\color{darkgray}{?}}
{\color{darkgray}{ 4}}
{\color{darkgray}{__def_init_mclasssolutionbasic_Solnbasic}}
{\color{darkgray}{__vtab_mclasssolutionbasic_Solnbasic}}
{\color{darkgray}{__def_init_mclasssolutionbasic_Solnbasic}}
{\color{darkgray}{__vtab_mclasssolutionbasic_Solnbasic}}
{\color{darkgray}{solnbasic_sub}}
{\color{darkgray}{solncomputesigma_sub}}
{\color{darkgray}{solnscalingparameter_sub}}
{\color{darkgray}{__copy_mclasssolutionbasic_Solnbasic}}
{\color{darkgray}{__final_mclasssolutionbasic_Solnbasic}}
{\color{darkgray}{real(kind=8)}}
{\color{darkgray}{integer(kind=4)}}
{\color{darkgray}{mclasssolutionbasic.toolkitallocation}}
{\color{darkgray}{mclasssolutionbasic.solnbasic}}
{\color{darkgray}{integer(kind=8)}}
{\color{darkgray}{mclasssolutionbasic.__class_mclasssolutionbasic_Solnbasic_t}}
{\color{darkgray}{mclasssolutionbasic.data}}
{\color{darkgray}{mclasssolutionbasic.__vtype_mclasssolutionbasic_Solnbasic}}
{\color{darkgray}{logical(kind=4)}}
{\color{darkgray}{logical(kind=1)}}
{\color{darkgray}{m-cl-soln-basic.f08}}
{\color{darkgray}{measurements}}
{\color{darkgray}{descriptor}}
{\color{darkgray}{solnbasic}}
\end{Verbatim}
}}


\endinput  %  ==  ==  ==  ==  ==  ==  ==  ==  ==


\section{\href{\urlMan}{Manual Pages}}
	% % % \input{./components/man/man-ldd}
\subsection{\refLdd: Print Shared Object Dependencies}

{\tiny{
\begin{lstlisting}[language=bash]
NAME
       ldd - print shared object dependencies
SYNOPSIS
       ldd [option]... file...
DESCRIPTION
       ldd prints the shared objects (shared libraries) required by each
       program or shared object specified on the command line.  An
       example of its use and output is the following:

           $ ldd /bin/ls
               linux-vdso.so.1 (0x00007ffcc3563000)
               libselinux.so.1 => /lib64/libselinux.so.1 (0x00007f87e5459000)
               libcap.so.2 => /lib64/libcap.so.2 (0x00007f87e5254000)
               libc.so.6 => /lib64/libc.so.6 (0x00007f87e4e92000)
               libpcre.so.1 => /lib64/libpcre.so.1 (0x00007f87e4c22000)
               libdl.so.2 => /lib64/libdl.so.2 (0x00007f87e4a1e000)
               /lib64/ld-linux-x86-64.so.2 (0x00005574bf12e000)
               libattr.so.1 => /lib64/libattr.so.1 (0x00007f87e4817000)
               libpthread.so.0 => /lib64/libpthread.so.0 (0x00007f87e45fa000)

       In the usual case, ldd invokes the standard dynamic linker (see
       ld.so(8)) with the LD_TRACE_LOADED_OBJECTS environment variable
       set to 1.  This causes the dynamic linker to inspect the
       program's dynamic dependencies, and find (according to the rules
       described in ld.so(8)) and load the objects that satisfy those
       dependencies.  For each dependency, ldd displays the location of
       the matching object and the (hexadecimal) address at which it is
       loaded.  (The linux-vdso and ld-linux shared dependencies are
       special; see vdso(7) and ld.so(8).)

   Security
       Be aware that in some circumstances (e.g., where the program
       specifies an ELF interpreter other than ld-linux.so), some
       versions of ldd may attempt to obtain the dependency information
       by attempting to directly execute the program, which may lead to
       the execution of whatever code is defined in the program's ELF
       interpreter, and perhaps to execution of the program itself.
       (Before glibc 2.27, the upstream ldd implementation did this for
       example, although most distributions provided a modified version
       that did not.)

       Thus, you should never employ ldd on an untrusted executable,
       since this may result in the execution of arbitrary code.  A
       safer alternative when dealing with untrusted executables is:

           $ objdump -p /path/to/program | grep NEEDED

       Note, however, that this alternative shows only the direct
       dependencies of the executable, while ldd shows the entire
       dependency tree of the executable.
OPTIONS
       --version
              Print the version number of ldd.

       --verbose
       -v     Print all information, including, for example, symbol
              versioning information.

       --unused
       -u     Print unused direct dependencies.  (Since glibc 2.3.4.)

       --data-relocs
       -d     Perform relocations and report any missing objects (ELF
              only).

       --function-relocs
       -r     Perform relocations for both data objects and functions,
              and report any missing objects or functions (ELF only).

       --help Usage information.
BUGS
       ldd does not work on a.out shared libraries.

       ldd does not work with some extremely old a.out programs which
       were built before ldd support was added to the compiler releases.
       If you use ldd on one of these programs, the program will attempt
       to run with argc = 0 and the results will be unpredictable.
SEE ALSO
       pldd(1), sprof(1), ld.so(8), ldconfig(8)
COLOPHON
       This page is part of the man-pages (Linux kernel and C library
       user-space interface documentation) project.  Information about
       the project can be found at 
       https://www.kernel.org/doc/man-pages/.  If you have a bug report
       for this manual page, see
       https://git.kernel.org/pub/scm/docs/man-pages/man-pages.git/tree/CONTRIBUTING.
       This page was obtained from the tarball man-pages-6.9.1.tar.gz
       fetched from
       https://mirrors.edge.kernel.org/pub/linux/docs/man-pages/ on
       2024-06-26.  If you discover any rendering problems in this HTML
       version of the page, or you believe there is a better or more up-
       to-date source for the page, or you have corrections or
       improvements to the information in this COLOPHON (which is not
       part of the original manual page), send a mail to
       man-pages@man7.org

Linux man-pages 6.9.1          2024-05-02                         ldd(1)
\end{lstlisting}
}}
\endinput  %  ==  ==  ==  ==  ==  ==  ==  ==  ==

\subsection{\refLdd: Print Shared Object Dependencies}

{\tiny{
\begin{lstlisting}[language=bash]
NAME
       ldd - print shared object dependencies
SYNOPSIS
       ldd [option]... file...
DESCRIPTION
       ldd prints the shared objects (shared libraries) required by each
       program or shared object specified on the command line.  An
       example of its use and output is the following:

           $ ldd /bin/ls
               linux-vdso.so.1 (0x00007ffcc3563000)
               libselinux.so.1 => /lib64/libselinux.so.1 (0x00007f87e5459000)
               libcap.so.2 => /lib64/libcap.so.2 (0x00007f87e5254000)
               libc.so.6 => /lib64/libc.so.6 (0x00007f87e4e92000)
               libpcre.so.1 => /lib64/libpcre.so.1 (0x00007f87e4c22000)
               libdl.so.2 => /lib64/libdl.so.2 (0x00007f87e4a1e000)
               /lib64/ld-linux-x86-64.so.2 (0x00005574bf12e000)
               libattr.so.1 => /lib64/libattr.so.1 (0x00007f87e4817000)
               libpthread.so.0 => /lib64/libpthread.so.0 (0x00007f87e45fa000)

       In the usual case, ldd invokes the standard dynamic linker (see
       ld.so(8)) with the LD_TRACE_LOADED_OBJECTS environment variable
       set to 1.  This causes the dynamic linker to inspect the
       program's dynamic dependencies, and find (according to the rules
       described in ld.so(8)) and load the objects that satisfy those
       dependencies.  For each dependency, ldd displays the location of
       the matching object and the (hexadecimal) address at which it is
       loaded.  (The linux-vdso and ld-linux shared dependencies are
       special; see vdso(7) and ld.so(8).)

   Security
       Be aware that in some circumstances (e.g., where the program
       specifies an ELF interpreter other than ld-linux.so), some
       versions of ldd may attempt to obtain the dependency information
       by attempting to directly execute the program, which may lead to
       the execution of whatever code is defined in the program's ELF
       interpreter, and perhaps to execution of the program itself.
       (Before glibc 2.27, the upstream ldd implementation did this for
       example, although most distributions provided a modified version
       that did not.)

       Thus, you should never employ ldd on an untrusted executable,
       since this may result in the execution of arbitrary code.  A
       safer alternative when dealing with untrusted executables is:

           $ objdump -p /path/to/program | grep NEEDED

       Note, however, that this alternative shows only the direct
       dependencies of the executable, while ldd shows the entire
       dependency tree of the executable.
OPTIONS
       --version
              Print the version number of ldd.

       --verbose
       -v     Print all information, including, for example, symbol
              versioning information.

       --unused
       -u     Print unused direct dependencies.  (Since glibc 2.3.4.)

       --data-relocs
       -d     Perform relocations and report any missing objects (ELF
              only).

       --function-relocs
       -r     Perform relocations for both data objects and functions,
              and report any missing objects or functions (ELF only).

       --help Usage information.
BUGS
       ldd does not work on a.out shared libraries.

       ldd does not work with some extremely old a.out programs which
       were built before ldd support was added to the compiler releases.
       If you use ldd on one of these programs, the program will attempt
       to run with argc = 0 and the results will be unpredictable.
SEE ALSO
       pldd(1), sprof(1), ld.so(8), ldconfig(8)
COLOPHON
       This page is part of the man-pages (Linux kernel and C library
       user-space interface documentation) project.  Information about
       the project can be found at 
       https://www.kernel.org/doc/man-pages/.  If you have a bug report
       for this manual page, see
       https://git.kernel.org/pub/scm/docs/man-pages/man-pages.git/tree/CONTRIBUTING.
       This page was obtained from the tarball man-pages-6.9.1.tar.gz
       fetched from
       https://mirrors.edge.kernel.org/pub/linux/docs/man-pages/ on
       2024-06-26.  If you discover any rendering problems in this HTML
       version of the page, or you believe there is a better or more up-
       to-date source for the page, or you have corrections or
       improvements to the information in this COLOPHON (which is not
       part of the original manual page), send a mail to
       man-pages@man7.org

Linux man-pages 6.9.1          2024-05-02                         ldd(1)
\end{lstlisting}
}}
\endinput  %  ==  ==  ==  ==  ==  ==  ==  ==  ==

\subsection{\refLdd: Print Shared Object Dependencies}

{\tiny{
\begin{lstlisting}[language=bash]
NAME
       ldd - print shared object dependencies
SYNOPSIS
       ldd [option]... file...
DESCRIPTION
       ldd prints the shared objects (shared libraries) required by each
       program or shared object specified on the command line.  An
       example of its use and output is the following:

           $ ldd /bin/ls
               linux-vdso.so.1 (0x00007ffcc3563000)
               libselinux.so.1 => /lib64/libselinux.so.1 (0x00007f87e5459000)
               libcap.so.2 => /lib64/libcap.so.2 (0x00007f87e5254000)
               libc.so.6 => /lib64/libc.so.6 (0x00007f87e4e92000)
               libpcre.so.1 => /lib64/libpcre.so.1 (0x00007f87e4c22000)
               libdl.so.2 => /lib64/libdl.so.2 (0x00007f87e4a1e000)
               /lib64/ld-linux-x86-64.so.2 (0x00005574bf12e000)
               libattr.so.1 => /lib64/libattr.so.1 (0x00007f87e4817000)
               libpthread.so.0 => /lib64/libpthread.so.0 (0x00007f87e45fa000)

       In the usual case, ldd invokes the standard dynamic linker (see
       ld.so(8)) with the LD_TRACE_LOADED_OBJECTS environment variable
       set to 1.  This causes the dynamic linker to inspect the
       program's dynamic dependencies, and find (according to the rules
       described in ld.so(8)) and load the objects that satisfy those
       dependencies.  For each dependency, ldd displays the location of
       the matching object and the (hexadecimal) address at which it is
       loaded.  (The linux-vdso and ld-linux shared dependencies are
       special; see vdso(7) and ld.so(8).)

   Security
       Be aware that in some circumstances (e.g., where the program
       specifies an ELF interpreter other than ld-linux.so), some
       versions of ldd may attempt to obtain the dependency information
       by attempting to directly execute the program, which may lead to
       the execution of whatever code is defined in the program's ELF
       interpreter, and perhaps to execution of the program itself.
       (Before glibc 2.27, the upstream ldd implementation did this for
       example, although most distributions provided a modified version
       that did not.)

       Thus, you should never employ ldd on an untrusted executable,
       since this may result in the execution of arbitrary code.  A
       safer alternative when dealing with untrusted executables is:

           $ objdump -p /path/to/program | grep NEEDED

       Note, however, that this alternative shows only the direct
       dependencies of the executable, while ldd shows the entire
       dependency tree of the executable.
OPTIONS
       --version
              Print the version number of ldd.

       --verbose
       -v     Print all information, including, for example, symbol
              versioning information.

       --unused
       -u     Print unused direct dependencies.  (Since glibc 2.3.4.)

       --data-relocs
       -d     Perform relocations and report any missing objects (ELF
              only).

       --function-relocs
       -r     Perform relocations for both data objects and functions,
              and report any missing objects or functions (ELF only).

       --help Usage information.
BUGS
       ldd does not work on a.out shared libraries.

       ldd does not work with some extremely old a.out programs which
       were built before ldd support was added to the compiler releases.
       If you use ldd on one of these programs, the program will attempt
       to run with argc = 0 and the results will be unpredictable.
SEE ALSO
       pldd(1), sprof(1), ld.so(8), ldconfig(8)
COLOPHON
       This page is part of the man-pages (Linux kernel and C library
       user-space interface documentation) project.  Information about
       the project can be found at 
       https://www.kernel.org/doc/man-pages/.  If you have a bug report
       for this manual page, see
       https://git.kernel.org/pub/scm/docs/man-pages/man-pages.git/tree/CONTRIBUTING.
       This page was obtained from the tarball man-pages-6.9.1.tar.gz
       fetched from
       https://mirrors.edge.kernel.org/pub/linux/docs/man-pages/ on
       2024-06-26.  If you discover any rendering problems in this HTML
       version of the page, or you believe there is a better or more up-
       to-date source for the page, or you have corrections or
       improvements to the information in this COLOPHON (which is not
       part of the original manual page), send a mail to
       man-pages@man7.org

Linux man-pages 6.9.1          2024-05-02                         ldd(1)
\end{lstlisting}
}}
\endinput  %  ==  ==  ==  ==  ==  ==  ==  ==  ==

	% % % \input{./components/man/man-ldd}
\subsection{\refLdd: Print Shared Object Dependencies}

{\tiny{
\begin{lstlisting}[language=bash]
NAME
       ldd - print shared object dependencies
SYNOPSIS
       ldd [option]... file...
DESCRIPTION
       ldd prints the shared objects (shared libraries) required by each
       program or shared object specified on the command line.  An
       example of its use and output is the following:

           $ ldd /bin/ls
               linux-vdso.so.1 (0x00007ffcc3563000)
               libselinux.so.1 => /lib64/libselinux.so.1 (0x00007f87e5459000)
               libcap.so.2 => /lib64/libcap.so.2 (0x00007f87e5254000)
               libc.so.6 => /lib64/libc.so.6 (0x00007f87e4e92000)
               libpcre.so.1 => /lib64/libpcre.so.1 (0x00007f87e4c22000)
               libdl.so.2 => /lib64/libdl.so.2 (0x00007f87e4a1e000)
               /lib64/ld-linux-x86-64.so.2 (0x00005574bf12e000)
               libattr.so.1 => /lib64/libattr.so.1 (0x00007f87e4817000)
               libpthread.so.0 => /lib64/libpthread.so.0 (0x00007f87e45fa000)

       In the usual case, ldd invokes the standard dynamic linker (see
       ld.so(8)) with the LD_TRACE_LOADED_OBJECTS environment variable
       set to 1.  This causes the dynamic linker to inspect the
       program's dynamic dependencies, and find (according to the rules
       described in ld.so(8)) and load the objects that satisfy those
       dependencies.  For each dependency, ldd displays the location of
       the matching object and the (hexadecimal) address at which it is
       loaded.  (The linux-vdso and ld-linux shared dependencies are
       special; see vdso(7) and ld.so(8).)

   Security
       Be aware that in some circumstances (e.g., where the program
       specifies an ELF interpreter other than ld-linux.so), some
       versions of ldd may attempt to obtain the dependency information
       by attempting to directly execute the program, which may lead to
       the execution of whatever code is defined in the program's ELF
       interpreter, and perhaps to execution of the program itself.
       (Before glibc 2.27, the upstream ldd implementation did this for
       example, although most distributions provided a modified version
       that did not.)

       Thus, you should never employ ldd on an untrusted executable,
       since this may result in the execution of arbitrary code.  A
       safer alternative when dealing with untrusted executables is:

           $ objdump -p /path/to/program | grep NEEDED

       Note, however, that this alternative shows only the direct
       dependencies of the executable, while ldd shows the entire
       dependency tree of the executable.
OPTIONS
       --version
              Print the version number of ldd.

       --verbose
       -v     Print all information, including, for example, symbol
              versioning information.

       --unused
       -u     Print unused direct dependencies.  (Since glibc 2.3.4.)

       --data-relocs
       -d     Perform relocations and report any missing objects (ELF
              only).

       --function-relocs
       -r     Perform relocations for both data objects and functions,
              and report any missing objects or functions (ELF only).

       --help Usage information.
BUGS
       ldd does not work on a.out shared libraries.

       ldd does not work with some extremely old a.out programs which
       were built before ldd support was added to the compiler releases.
       If you use ldd on one of these programs, the program will attempt
       to run with argc = 0 and the results will be unpredictable.
SEE ALSO
       pldd(1), sprof(1), ld.so(8), ldconfig(8)
COLOPHON
       This page is part of the man-pages (Linux kernel and C library
       user-space interface documentation) project.  Information about
       the project can be found at 
       https://www.kernel.org/doc/man-pages/.  If you have a bug report
       for this manual page, see
       https://git.kernel.org/pub/scm/docs/man-pages/man-pages.git/tree/CONTRIBUTING.
       This page was obtained from the tarball man-pages-6.9.1.tar.gz
       fetched from
       https://mirrors.edge.kernel.org/pub/linux/docs/man-pages/ on
       2024-06-26.  If you discover any rendering problems in this HTML
       version of the page, or you believe there is a better or more up-
       to-date source for the page, or you have corrections or
       improvements to the information in this COLOPHON (which is not
       part of the original manual page), send a mail to
       man-pages@man7.org

Linux man-pages 6.9.1          2024-05-02                         ldd(1)
\end{lstlisting}
}}
\endinput  %  ==  ==  ==  ==  ==  ==  ==  ==  ==

\subsection{\refLdd: Print Shared Object Dependencies}

{\tiny{
\begin{lstlisting}[language=bash]
NAME
       ldd - print shared object dependencies
SYNOPSIS
       ldd [option]... file...
DESCRIPTION
       ldd prints the shared objects (shared libraries) required by each
       program or shared object specified on the command line.  An
       example of its use and output is the following:

           $ ldd /bin/ls
               linux-vdso.so.1 (0x00007ffcc3563000)
               libselinux.so.1 => /lib64/libselinux.so.1 (0x00007f87e5459000)
               libcap.so.2 => /lib64/libcap.so.2 (0x00007f87e5254000)
               libc.so.6 => /lib64/libc.so.6 (0x00007f87e4e92000)
               libpcre.so.1 => /lib64/libpcre.so.1 (0x00007f87e4c22000)
               libdl.so.2 => /lib64/libdl.so.2 (0x00007f87e4a1e000)
               /lib64/ld-linux-x86-64.so.2 (0x00005574bf12e000)
               libattr.so.1 => /lib64/libattr.so.1 (0x00007f87e4817000)
               libpthread.so.0 => /lib64/libpthread.so.0 (0x00007f87e45fa000)

       In the usual case, ldd invokes the standard dynamic linker (see
       ld.so(8)) with the LD_TRACE_LOADED_OBJECTS environment variable
       set to 1.  This causes the dynamic linker to inspect the
       program's dynamic dependencies, and find (according to the rules
       described in ld.so(8)) and load the objects that satisfy those
       dependencies.  For each dependency, ldd displays the location of
       the matching object and the (hexadecimal) address at which it is
       loaded.  (The linux-vdso and ld-linux shared dependencies are
       special; see vdso(7) and ld.so(8).)

   Security
       Be aware that in some circumstances (e.g., where the program
       specifies an ELF interpreter other than ld-linux.so), some
       versions of ldd may attempt to obtain the dependency information
       by attempting to directly execute the program, which may lead to
       the execution of whatever code is defined in the program's ELF
       interpreter, and perhaps to execution of the program itself.
       (Before glibc 2.27, the upstream ldd implementation did this for
       example, although most distributions provided a modified version
       that did not.)

       Thus, you should never employ ldd on an untrusted executable,
       since this may result in the execution of arbitrary code.  A
       safer alternative when dealing with untrusted executables is:

           $ objdump -p /path/to/program | grep NEEDED

       Note, however, that this alternative shows only the direct
       dependencies of the executable, while ldd shows the entire
       dependency tree of the executable.
OPTIONS
       --version
              Print the version number of ldd.

       --verbose
       -v     Print all information, including, for example, symbol
              versioning information.

       --unused
       -u     Print unused direct dependencies.  (Since glibc 2.3.4.)

       --data-relocs
       -d     Perform relocations and report any missing objects (ELF
              only).

       --function-relocs
       -r     Perform relocations for both data objects and functions,
              and report any missing objects or functions (ELF only).

       --help Usage information.
BUGS
       ldd does not work on a.out shared libraries.

       ldd does not work with some extremely old a.out programs which
       were built before ldd support was added to the compiler releases.
       If you use ldd on one of these programs, the program will attempt
       to run with argc = 0 and the results will be unpredictable.
SEE ALSO
       pldd(1), sprof(1), ld.so(8), ldconfig(8)
COLOPHON
       This page is part of the man-pages (Linux kernel and C library
       user-space interface documentation) project.  Information about
       the project can be found at 
       https://www.kernel.org/doc/man-pages/.  If you have a bug report
       for this manual page, see
       https://git.kernel.org/pub/scm/docs/man-pages/man-pages.git/tree/CONTRIBUTING.
       This page was obtained from the tarball man-pages-6.9.1.tar.gz
       fetched from
       https://mirrors.edge.kernel.org/pub/linux/docs/man-pages/ on
       2024-06-26.  If you discover any rendering problems in this HTML
       version of the page, or you believe there is a better or more up-
       to-date source for the page, or you have corrections or
       improvements to the information in this COLOPHON (which is not
       part of the original manual page), send a mail to
       man-pages@man7.org

Linux man-pages 6.9.1          2024-05-02                         ldd(1)
\end{lstlisting}
}}
\endinput  %  ==  ==  ==  ==  ==  ==  ==  ==  ==

\subsection{\refLddconfig: Configure Dynamic Linker Run-time Bindings}

{\tiny{
\begin{lstlisting}[language=bash]
NAME
       ldconfig - configure dynamic linker run-time bindings
SYNOPSIS
       /sbin/ldconfig [-nNvVX] [-C cache] [-f conf] [-r root]
                      directory ...

       /sbin/ldconfig -l [-v] library ...

       /sbin/ldconfig -p
DESCRIPTION
       ldconfig creates the necessary links and cache to the most recent
       shared libraries found in the directories specified on the
       command line, in the file /etc/ld.so.conf, and in the trusted
       directories, /lib and /usr/lib.  On some 64-bit architectures
       such as x86-64, /lib and /usr/lib are the trusted directories for
       32-bit libraries, while /lib64 and /usr/lib64 are used for 64-bit
       libraries.

       The cache is used by the run-time linker, ld.so or ld-linux.so.
       ldconfig checks the header and filenames of the libraries it
       encounters when determining which versions should have their
       links updated.  ldconfig should normally be run by the superuser
       as it may require write permission on some root owned directories
       and files.

       ldconfig will look only at files that are named lib*.so* (for
       regular shared objects) or ld-*.so* (for the dynamic loader
       itself).  Other files will be ignored.  Also, ldconfig expects a
       certain pattern to how the symbolic links are set up, like this
       example, where the middle file (libfoo.so.1 here) is the SONAME
       for the library:

           libfoo.so -> libfoo.so.1 -> libfoo.so.1.12

       Failure to follow this pattern may result in compatibility issues
       after an upgrade.
OPTIONS
       --format=fmt
       -c fmt (Since glibc 2.2) Use cache format fmt, which is one of
              old, new, or compat.  Since glibc 2.32, the default is
              new.  Before that, it was compat.

       -C cache
              Use cache instead of /etc/ld.so.cache.

       -f conf
              Use conf instead of /etc/ld.so.conf.

       --ignore-aux-cache
       -i     (Since glibc 2.7) Ignore auxiliary cache file.

       -l     (Since glibc 2.2) Interpret each operand as a library name
              and configure its links.  Intended for use only by
              experts.

       -n     Process only the directories specified on the command
              line; don't process the trusted directories, nor those
              specified in /etc/ld.so.conf.  Implies -N.

       -N     Don't rebuild the cache.  Unless -X is also specified,
              links are still updated.

       --print-cache
       -p     Print the lists of directories and candidate libraries
              stored in the current cache.

       -r root
              Change to and use root as the root directory.

       --verbose
       -v     Verbose mode.  Print current version number, the name of
              each directory as it is scanned, and any links that are
              created.  Overrides quiet mode.

       --version
       -V     Print program version.

       -X     Don't update links.  Unless -N is also specified, the
              cache is still rebuilt.
FILES
       /lib/ld.so
              is the run-time linker/loader.
       /etc/ld.so.conf
              contains a list of directories, one per line, in which to
              search for libraries.
       /etc/ld.so.cache
              contains an ordered list of libraries found in the
              directories specified in /etc/ld.so.conf, as well as those
              found in the trusted directories.
SEE ALSO
       ldd(1), ld.so(8)
COLOPHON
       This page is part of the man-pages (Linux kernel and C library
       user-space interface documentation) project.  Information about
       the project can be found at 
       https://www.kernel.org/doc/man-pages/.  If you have a bug report
       for this manual page, see
       https://git.kernel.org/pub/scm/docs/man-pages/man-pages.git/tree/CONTRIBUTING.
       This page was obtained from the tarball man-pages-6.9.1.tar.gz
       fetched from
       https://mirrors.edge.kernel.org/pub/linux/docs/man-pages/ on
       2024-06-26.  If you discover any rendering problems in this HTML
       version of the page, or you believe there is a better or more up-
       to-date source for the page, or you have corrections or
       improvements to the information in this COLOPHON (which is not
       part of the original manual page), send a mail to
       man-pages@man7.org

Linux man-pages 6.9.1          2024-05-02                    ldconfig(8)
\end{lstlisting}
}}
\endinput  %  ==  ==  ==  ==  ==  ==  ==  ==  ==

	% % % \input{./components/man/man-locate}
\subsection{\refLocate: List File in Databases}

{\tiny{
\begin{lstlisting}[language=bash]
NAME
       locate - list files in databases that match a pattern
SYNOPSIS
       locate [-d path | --database=path] [-e | -E | --[non-]existing]
       [-i | --ignore-case] [-0 | --null] [-c | --count] [-w |
       --wholename] [-b | --basename] [-l N | --limit=N] [-S |
       --statistics] [-r | --regex ] [--regextype R] [--max-database-age
       D] [-P | -H | --nofollow] [-L | --follow] [--version] [-A |
       --all] [-p | --print] [--help] pattern...
DESCRIPTION
       This manual page documents the GNU version of locate.  For each
       given pattern, locate searches one or more databases of file
       names and displays the file names that contain the pattern.
       Patterns can contain shell-style metacharacters: `*', `?', and
       `[]'.  The metacharacters do not treat `/' or `.'  specially.
       Therefore, a pattern `foo*bar' can match a file name that
       contains `foo3/bar', and a pattern `*duck*' can match a file name
       that contains `lake/.ducky'.  Patterns that contain
       metacharacters should be quoted to protect them from expansion by
       the shell.

       If a pattern is a plain string - it contains no metacharacters -
       locate displays all file names in the database that contain that
       string anywhere.  If a pattern does contain metacharacters,
       locate only displays file names that match the pattern exactly.
       As a result, patterns that contain metacharacters should usually
       begin with a `*', and will most often end with one as well.  The
       exceptions are patterns that are intended to explicitly match the
       beginning or end of a file name.

       The file name databases contain lists of files that were on the
       system when the databases were last updated.  The system
       administrator can choose the file name of the default database,
       the frequency with which the databases are updated, and the
       directories for which they contain entries; see updatedb(1).

       If locate's output is going to a terminal, unusual characters in
       the output are escaped in the same way as for the -print action
       of the find command.  If the output is not going to a terminal,
       file names are printed exactly as-is.
OPTIONS
       -0, --null
              Use ASCII NUL as a separator, instead of newline.

       -A, --all
              Print only names which match all non-option arguments, not
              those matching one or more non-option arguments.

       -b, --basename
              Results are considered to match if the pattern specified
              matches the final component of the name of a file as
              listed in the database.  This final component is usually
              referred to as the `base name'.

       -c, --count
              Instead of printing the matched filenames, just print the
              total number of matches we found, unless --print (-p) is
              also present.

       -d path, --database=path
              Instead of searching the default file name database,
              search the file name databases in path, which is a colon-
              separated list of database file names.  You can also use
              the environment variable LOCATE_PATH to set the list of
              database files to search.  The option overrides the
              environment variable if both are used.  Empty elements in
              the path are taken to be synonyms for the file name of the
              default database.  A database can be supplied on stdin,
              using `-' as an element of path. If more than one element
              of path is `-', later instances are ignored (and a warning
              message is printed).

              The file name database format changed starting with GNU
              find and locate version 4.0 to allow machines with
              different byte orderings to share the databases.  This
              version of locate can automatically recognize and read
              databases produced for older versions of GNU locate or
              Unix versions of locate or find.  Support for the old
              locate database format will be discontinued in a future
              release.

       -e, --existing
              Only print out such names that currently exist (instead of
              such names that existed when the database was created).
              Note that this may slow down the program a lot, if there
              are many matches in the database.  If you are using this
              option within a program, please note that it is possible
              for the file to be deleted after locate has checked that
              it exists, but before you use it.

       -E, --non-existing
              Only print out such names that currently do not exist
              (instead of such names that existed when the database was
              created).  Note that this may slow down the program a lot,
              if there are many matches in the database.

       --help Print a summary of the options to locate and exit.

       -i, --ignore-case
              Ignore case distinctions in both the pattern and the file
              names.

       -l N, --limit=N
              Limit the number of matches to N.  If a limit is set via
              this option, the number of results printed for the -c
              option will never be larger than this number.

       -L, --follow
              If testing for the existence of files (with the -e or -E
              options), consider broken symbolic links to be non-
              existing.   This is the default.

       --max-database-age D
              Normally, locate will issue a warning message when it
              searches a database which is more than 8 days old.  This
              option changes that value to something other than 8.  The
              effect of specifying a negative value is undefined.

       -m, --mmap
              Accepted but does nothing, for compatibility with BSD
              locate.

       -P, -H, --nofollow
              If testing for the existence of files (with the -e or -E
              options), treat broken symbolic links as if they were
              existing files.  The -H form of this option is provided
              purely for similarity with find; the use of -P is
              recommended over -H.

       -p, --print
              Print search results when they normally would not, because
              of the presence of --statistics (-S) or --count (-c).

       -r, --regex
              The pattern specified on the command line is understood to
              be a regular expression, as opposed to a glob pattern.
              The Regular expressions work in the same was as in emacs
              except for the fact that "." will match a newline.  GNU
              find uses the same regular expressions.  Filenames whose
              full paths match the specified regular expression are
              printed (or, in the case of the -c option, counted).  If
              you wish to anchor your regular expression at the ends of
              the full path name, then as is usual with regular
              expressions, you should use the characters ^ and $ to
              signify this.

       --regextype R
              Use regular expression dialect R.  Supported dialects
              include `findutils-default', `posix-awk', `posix-basic',
              `posix-egrep', `posix-extended', `posix-minimal-basic',
              `awk', `ed', `egrep', `emacs', `gnu-awk', `grep' and
              `sed'.  See the Texinfo documentation for a detailed
              explanation of these dialects.

       -s, --stdio
              Accepted but does nothing, for compatibility with BSD
              locate.

       -S, --statistics
              Print various statistics about each locate database and
              then exit without performing a search, unless non-option
              arguments are given.  For compatibility with BSD, -S is
              accepted as a synonym for --statistics.  However, the
              output of locate -S is different for the GNU and BSD
              implementations of locate.

       --version
              Print the version number of locate and exit.

       -w, --wholename
              Match against the whole name of the file as listed in the
              database.  This is the default.
ENVIRONMENT
       LOCATE_PATH
              Colon-separated list of databases to search.  If the value
              has a leading or trailing colon, or has two colons in a
              row, you may get results that vary between different
              versions of locate.
HISTORY
       The locate program started life as the BSD fast find program,
       contributed to BSD by James A. Woods.  This was described by his
       paper Finding Files Fast which was published in Usenix ;login:,
       Vol 8, No 1, February/March, 1983, pp. 8-10.   When the find
       program began to assume a default -print action if no action was
       specified, this changed the interpretation of find pattern.  The
       BSD developers therefore moved the fast find functionality into
       locate.  The GNU implementation of locate appears to be derived
       from the same code.

       Significant changes to locate in reverse order:
       4.3.7     Byte-order independent support for old database format
       4.3.3     locate -i supports multi-byte characters correctly
                 Introduced --max_db_age
       4.3.2     Support for the slocate database format
       4.2.22    Introduced the --all option
       4.2.15    Introduced the --regex option
       4.2.14    Introduced options -L, -P, -H
       4.2.12    Empty items in LOCATE_PATH now indicate the default database
       4.2.11    Introduced the --statistics option
       4.2.4     Introduced --count and --limit
       4.2.0     Glob characters cause matching against the whole file name
       4.0       Introduced the LOCATE02 database format
       3.7       Locate can search multiple databases
BUGS
       The locate database correctly handles filenames containing
       newlines, but only if the system's sort command has a working -z
       option.  If you suspect that locate may need to return filenames
       containing newlines, consider using its --null option.
REPORTING BUGS
       GNU findutils online help:
       <https://www.gnu.org/software/findutils/#get-help>
       Report any translation bugs to
       <https://translationproject.org/team/>

       Report any other issue via the form at the GNU Savannah bug
       tracker:
              <https://savannah.gnu.org/bugs/?group=findutils>
       General topics about the GNU findutils package are discussed at
       the bug-findutils mailing list:
              <https://lists.gnu.org/mailman/listinfo/bug-findutils>
COPYRIGHT
       Copyright (C) 1994-2024 Free Software Foundation, Inc.  License
       GPLv3+: GNU GPL version 3 or later
       <https://gnu.org/licenses/gpl.html>.
       This is free software: you are free to change and redistribute
       it.  There is NO WARRANTY, to the extent permitted by law.
SEE ALSO
       find(1), updatedb(1), xargs(1), glob(3), locatedb(5)

       Full documentation
       <https://www.gnu.org/software/findutils/locate>
       or available locally via: info locate
COLOPHON
       This page is part of the findutils (find utilities) project.
       Information about the project can be found at 
       http://www.gnu.org/software/findutils/.  If you have a bug
       report for this manual page, see
       https://savannah.gnu.org/bugs/?group=findutils.  This page was
       obtained from the project's upstream Git repository
       git://git.savannah.gnu.org/findutils.git on 2024-06-14.  (At
       that time, the date of the most recent commit that was found in
       the repository was 2024-06-03.)  If you discover any rendering
       problems in this HTML version of the page, or you believe there
       is a better or more up-to-date source for the page, or you have
       corrections or improvements to the information in this COLOPHON
       (which is not part of the original manual page), send a mail to
       man-pages@man7.org

                                                               LOCATE(1)
\end{lstlisting}
}}
\endinput  %  ==  ==  ==  ==  ==  ==  ==  ==  ==

\subsection{\refLocate: List File in Databases}

{\tiny{
\begin{lstlisting}[language=bash]
NAME
       locate - list files in databases that match a pattern
SYNOPSIS
       locate [-d path | --database=path] [-e | -E | --[non-]existing]
       [-i | --ignore-case] [-0 | --null] [-c | --count] [-w |
       --wholename] [-b | --basename] [-l N | --limit=N] [-S |
       --statistics] [-r | --regex ] [--regextype R] [--max-database-age
       D] [-P | -H | --nofollow] [-L | --follow] [--version] [-A |
       --all] [-p | --print] [--help] pattern...
DESCRIPTION
       This manual page documents the GNU version of locate.  For each
       given pattern, locate searches one or more databases of file
       names and displays the file names that contain the pattern.
       Patterns can contain shell-style metacharacters: `*', `?', and
       `[]'.  The metacharacters do not treat `/' or `.'  specially.
       Therefore, a pattern `foo*bar' can match a file name that
       contains `foo3/bar', and a pattern `*duck*' can match a file name
       that contains `lake/.ducky'.  Patterns that contain
       metacharacters should be quoted to protect them from expansion by
       the shell.

       If a pattern is a plain string - it contains no metacharacters -
       locate displays all file names in the database that contain that
       string anywhere.  If a pattern does contain metacharacters,
       locate only displays file names that match the pattern exactly.
       As a result, patterns that contain metacharacters should usually
       begin with a `*', and will most often end with one as well.  The
       exceptions are patterns that are intended to explicitly match the
       beginning or end of a file name.

       The file name databases contain lists of files that were on the
       system when the databases were last updated.  The system
       administrator can choose the file name of the default database,
       the frequency with which the databases are updated, and the
       directories for which they contain entries; see updatedb(1).

       If locate's output is going to a terminal, unusual characters in
       the output are escaped in the same way as for the -print action
       of the find command.  If the output is not going to a terminal,
       file names are printed exactly as-is.
OPTIONS
       -0, --null
              Use ASCII NUL as a separator, instead of newline.

       -A, --all
              Print only names which match all non-option arguments, not
              those matching one or more non-option arguments.

       -b, --basename
              Results are considered to match if the pattern specified
              matches the final component of the name of a file as
              listed in the database.  This final component is usually
              referred to as the `base name'.

       -c, --count
              Instead of printing the matched filenames, just print the
              total number of matches we found, unless --print (-p) is
              also present.

       -d path, --database=path
              Instead of searching the default file name database,
              search the file name databases in path, which is a colon-
              separated list of database file names.  You can also use
              the environment variable LOCATE_PATH to set the list of
              database files to search.  The option overrides the
              environment variable if both are used.  Empty elements in
              the path are taken to be synonyms for the file name of the
              default database.  A database can be supplied on stdin,
              using `-' as an element of path. If more than one element
              of path is `-', later instances are ignored (and a warning
              message is printed).

              The file name database format changed starting with GNU
              find and locate version 4.0 to allow machines with
              different byte orderings to share the databases.  This
              version of locate can automatically recognize and read
              databases produced for older versions of GNU locate or
              Unix versions of locate or find.  Support for the old
              locate database format will be discontinued in a future
              release.

       -e, --existing
              Only print out such names that currently exist (instead of
              such names that existed when the database was created).
              Note that this may slow down the program a lot, if there
              are many matches in the database.  If you are using this
              option within a program, please note that it is possible
              for the file to be deleted after locate has checked that
              it exists, but before you use it.

       -E, --non-existing
              Only print out such names that currently do not exist
              (instead of such names that existed when the database was
              created).  Note that this may slow down the program a lot,
              if there are many matches in the database.

       --help Print a summary of the options to locate and exit.

       -i, --ignore-case
              Ignore case distinctions in both the pattern and the file
              names.

       -l N, --limit=N
              Limit the number of matches to N.  If a limit is set via
              this option, the number of results printed for the -c
              option will never be larger than this number.

       -L, --follow
              If testing for the existence of files (with the -e or -E
              options), consider broken symbolic links to be non-
              existing.   This is the default.

       --max-database-age D
              Normally, locate will issue a warning message when it
              searches a database which is more than 8 days old.  This
              option changes that value to something other than 8.  The
              effect of specifying a negative value is undefined.

       -m, --mmap
              Accepted but does nothing, for compatibility with BSD
              locate.

       -P, -H, --nofollow
              If testing for the existence of files (with the -e or -E
              options), treat broken symbolic links as if they were
              existing files.  The -H form of this option is provided
              purely for similarity with find; the use of -P is
              recommended over -H.

       -p, --print
              Print search results when they normally would not, because
              of the presence of --statistics (-S) or --count (-c).

       -r, --regex
              The pattern specified on the command line is understood to
              be a regular expression, as opposed to a glob pattern.
              The Regular expressions work in the same was as in emacs
              except for the fact that "." will match a newline.  GNU
              find uses the same regular expressions.  Filenames whose
              full paths match the specified regular expression are
              printed (or, in the case of the -c option, counted).  If
              you wish to anchor your regular expression at the ends of
              the full path name, then as is usual with regular
              expressions, you should use the characters ^ and $ to
              signify this.

       --regextype R
              Use regular expression dialect R.  Supported dialects
              include `findutils-default', `posix-awk', `posix-basic',
              `posix-egrep', `posix-extended', `posix-minimal-basic',
              `awk', `ed', `egrep', `emacs', `gnu-awk', `grep' and
              `sed'.  See the Texinfo documentation for a detailed
              explanation of these dialects.

       -s, --stdio
              Accepted but does nothing, for compatibility with BSD
              locate.

       -S, --statistics
              Print various statistics about each locate database and
              then exit without performing a search, unless non-option
              arguments are given.  For compatibility with BSD, -S is
              accepted as a synonym for --statistics.  However, the
              output of locate -S is different for the GNU and BSD
              implementations of locate.

       --version
              Print the version number of locate and exit.

       -w, --wholename
              Match against the whole name of the file as listed in the
              database.  This is the default.
ENVIRONMENT
       LOCATE_PATH
              Colon-separated list of databases to search.  If the value
              has a leading or trailing colon, or has two colons in a
              row, you may get results that vary between different
              versions of locate.
HISTORY
       The locate program started life as the BSD fast find program,
       contributed to BSD by James A. Woods.  This was described by his
       paper Finding Files Fast which was published in Usenix ;login:,
       Vol 8, No 1, February/March, 1983, pp. 8-10.   When the find
       program began to assume a default -print action if no action was
       specified, this changed the interpretation of find pattern.  The
       BSD developers therefore moved the fast find functionality into
       locate.  The GNU implementation of locate appears to be derived
       from the same code.

       Significant changes to locate in reverse order:
       4.3.7     Byte-order independent support for old database format
       4.3.3     locate -i supports multi-byte characters correctly
                 Introduced --max_db_age
       4.3.2     Support for the slocate database format
       4.2.22    Introduced the --all option
       4.2.15    Introduced the --regex option
       4.2.14    Introduced options -L, -P, -H
       4.2.12    Empty items in LOCATE_PATH now indicate the default database
       4.2.11    Introduced the --statistics option
       4.2.4     Introduced --count and --limit
       4.2.0     Glob characters cause matching against the whole file name
       4.0       Introduced the LOCATE02 database format
       3.7       Locate can search multiple databases
BUGS
       The locate database correctly handles filenames containing
       newlines, but only if the system's sort command has a working -z
       option.  If you suspect that locate may need to return filenames
       containing newlines, consider using its --null option.
REPORTING BUGS
       GNU findutils online help:
       <https://www.gnu.org/software/findutils/#get-help>
       Report any translation bugs to
       <https://translationproject.org/team/>

       Report any other issue via the form at the GNU Savannah bug
       tracker:
              <https://savannah.gnu.org/bugs/?group=findutils>
       General topics about the GNU findutils package are discussed at
       the bug-findutils mailing list:
              <https://lists.gnu.org/mailman/listinfo/bug-findutils>
COPYRIGHT
       Copyright (C) 1994-2024 Free Software Foundation, Inc.  License
       GPLv3+: GNU GPL version 3 or later
       <https://gnu.org/licenses/gpl.html>.
       This is free software: you are free to change and redistribute
       it.  There is NO WARRANTY, to the extent permitted by law.
SEE ALSO
       find(1), updatedb(1), xargs(1), glob(3), locatedb(5)

       Full documentation
       <https://www.gnu.org/software/findutils/locate>
       or available locally via: info locate
COLOPHON
       This page is part of the findutils (find utilities) project.
       Information about the project can be found at 
       http://www.gnu.org/software/findutils/.  If you have a bug
       report for this manual page, see
       https://savannah.gnu.org/bugs/?group=findutils.  This page was
       obtained from the project's upstream Git repository
       git://git.savannah.gnu.org/findutils.git on 2024-06-14.  (At
       that time, the date of the most recent commit that was found in
       the repository was 2024-06-03.)  If you discover any rendering
       problems in this HTML version of the page, or you believe there
       is a better or more up-to-date source for the page, or you have
       corrections or improvements to the information in this COLOPHON
       (which is not part of the original manual page), send a mail to
       man-pages@man7.org

                                                               LOCATE(1)
\end{lstlisting}
}}
\endinput  %  ==  ==  ==  ==  ==  ==  ==  ==  ==

\subsection{\refLocate: List File in Databases}

{\tiny{
\begin{lstlisting}[language=bash]
NAME
       locate - list files in databases that match a pattern
SYNOPSIS
       locate [-d path | --database=path] [-e | -E | --[non-]existing]
       [-i | --ignore-case] [-0 | --null] [-c | --count] [-w |
       --wholename] [-b | --basename] [-l N | --limit=N] [-S |
       --statistics] [-r | --regex ] [--regextype R] [--max-database-age
       D] [-P | -H | --nofollow] [-L | --follow] [--version] [-A |
       --all] [-p | --print] [--help] pattern...
DESCRIPTION
       This manual page documents the GNU version of locate.  For each
       given pattern, locate searches one or more databases of file
       names and displays the file names that contain the pattern.
       Patterns can contain shell-style metacharacters: `*', `?', and
       `[]'.  The metacharacters do not treat `/' or `.'  specially.
       Therefore, a pattern `foo*bar' can match a file name that
       contains `foo3/bar', and a pattern `*duck*' can match a file name
       that contains `lake/.ducky'.  Patterns that contain
       metacharacters should be quoted to protect them from expansion by
       the shell.

       If a pattern is a plain string - it contains no metacharacters -
       locate displays all file names in the database that contain that
       string anywhere.  If a pattern does contain metacharacters,
       locate only displays file names that match the pattern exactly.
       As a result, patterns that contain metacharacters should usually
       begin with a `*', and will most often end with one as well.  The
       exceptions are patterns that are intended to explicitly match the
       beginning or end of a file name.

       The file name databases contain lists of files that were on the
       system when the databases were last updated.  The system
       administrator can choose the file name of the default database,
       the frequency with which the databases are updated, and the
       directories for which they contain entries; see updatedb(1).

       If locate's output is going to a terminal, unusual characters in
       the output are escaped in the same way as for the -print action
       of the find command.  If the output is not going to a terminal,
       file names are printed exactly as-is.
OPTIONS
       -0, --null
              Use ASCII NUL as a separator, instead of newline.

       -A, --all
              Print only names which match all non-option arguments, not
              those matching one or more non-option arguments.

       -b, --basename
              Results are considered to match if the pattern specified
              matches the final component of the name of a file as
              listed in the database.  This final component is usually
              referred to as the `base name'.

       -c, --count
              Instead of printing the matched filenames, just print the
              total number of matches we found, unless --print (-p) is
              also present.

       -d path, --database=path
              Instead of searching the default file name database,
              search the file name databases in path, which is a colon-
              separated list of database file names.  You can also use
              the environment variable LOCATE_PATH to set the list of
              database files to search.  The option overrides the
              environment variable if both are used.  Empty elements in
              the path are taken to be synonyms for the file name of the
              default database.  A database can be supplied on stdin,
              using `-' as an element of path. If more than one element
              of path is `-', later instances are ignored (and a warning
              message is printed).

              The file name database format changed starting with GNU
              find and locate version 4.0 to allow machines with
              different byte orderings to share the databases.  This
              version of locate can automatically recognize and read
              databases produced for older versions of GNU locate or
              Unix versions of locate or find.  Support for the old
              locate database format will be discontinued in a future
              release.

       -e, --existing
              Only print out such names that currently exist (instead of
              such names that existed when the database was created).
              Note that this may slow down the program a lot, if there
              are many matches in the database.  If you are using this
              option within a program, please note that it is possible
              for the file to be deleted after locate has checked that
              it exists, but before you use it.

       -E, --non-existing
              Only print out such names that currently do not exist
              (instead of such names that existed when the database was
              created).  Note that this may slow down the program a lot,
              if there are many matches in the database.

       --help Print a summary of the options to locate and exit.

       -i, --ignore-case
              Ignore case distinctions in both the pattern and the file
              names.

       -l N, --limit=N
              Limit the number of matches to N.  If a limit is set via
              this option, the number of results printed for the -c
              option will never be larger than this number.

       -L, --follow
              If testing for the existence of files (with the -e or -E
              options), consider broken symbolic links to be non-
              existing.   This is the default.

       --max-database-age D
              Normally, locate will issue a warning message when it
              searches a database which is more than 8 days old.  This
              option changes that value to something other than 8.  The
              effect of specifying a negative value is undefined.

       -m, --mmap
              Accepted but does nothing, for compatibility with BSD
              locate.

       -P, -H, --nofollow
              If testing for the existence of files (with the -e or -E
              options), treat broken symbolic links as if they were
              existing files.  The -H form of this option is provided
              purely for similarity with find; the use of -P is
              recommended over -H.

       -p, --print
              Print search results when they normally would not, because
              of the presence of --statistics (-S) or --count (-c).

       -r, --regex
              The pattern specified on the command line is understood to
              be a regular expression, as opposed to a glob pattern.
              The Regular expressions work in the same was as in emacs
              except for the fact that "." will match a newline.  GNU
              find uses the same regular expressions.  Filenames whose
              full paths match the specified regular expression are
              printed (or, in the case of the -c option, counted).  If
              you wish to anchor your regular expression at the ends of
              the full path name, then as is usual with regular
              expressions, you should use the characters ^ and $ to
              signify this.

       --regextype R
              Use regular expression dialect R.  Supported dialects
              include `findutils-default', `posix-awk', `posix-basic',
              `posix-egrep', `posix-extended', `posix-minimal-basic',
              `awk', `ed', `egrep', `emacs', `gnu-awk', `grep' and
              `sed'.  See the Texinfo documentation for a detailed
              explanation of these dialects.

       -s, --stdio
              Accepted but does nothing, for compatibility with BSD
              locate.

       -S, --statistics
              Print various statistics about each locate database and
              then exit without performing a search, unless non-option
              arguments are given.  For compatibility with BSD, -S is
              accepted as a synonym for --statistics.  However, the
              output of locate -S is different for the GNU and BSD
              implementations of locate.

       --version
              Print the version number of locate and exit.

       -w, --wholename
              Match against the whole name of the file as listed in the
              database.  This is the default.
ENVIRONMENT
       LOCATE_PATH
              Colon-separated list of databases to search.  If the value
              has a leading or trailing colon, or has two colons in a
              row, you may get results that vary between different
              versions of locate.
HISTORY
       The locate program started life as the BSD fast find program,
       contributed to BSD by James A. Woods.  This was described by his
       paper Finding Files Fast which was published in Usenix ;login:,
       Vol 8, No 1, February/March, 1983, pp. 8-10.   When the find
       program began to assume a default -print action if no action was
       specified, this changed the interpretation of find pattern.  The
       BSD developers therefore moved the fast find functionality into
       locate.  The GNU implementation of locate appears to be derived
       from the same code.

       Significant changes to locate in reverse order:
       4.3.7     Byte-order independent support for old database format
       4.3.3     locate -i supports multi-byte characters correctly
                 Introduced --max_db_age
       4.3.2     Support for the slocate database format
       4.2.22    Introduced the --all option
       4.2.15    Introduced the --regex option
       4.2.14    Introduced options -L, -P, -H
       4.2.12    Empty items in LOCATE_PATH now indicate the default database
       4.2.11    Introduced the --statistics option
       4.2.4     Introduced --count and --limit
       4.2.0     Glob characters cause matching against the whole file name
       4.0       Introduced the LOCATE02 database format
       3.7       Locate can search multiple databases
BUGS
       The locate database correctly handles filenames containing
       newlines, but only if the system's sort command has a working -z
       option.  If you suspect that locate may need to return filenames
       containing newlines, consider using its --null option.
REPORTING BUGS
       GNU findutils online help:
       <https://www.gnu.org/software/findutils/#get-help>
       Report any translation bugs to
       <https://translationproject.org/team/>

       Report any other issue via the form at the GNU Savannah bug
       tracker:
              <https://savannah.gnu.org/bugs/?group=findutils>
       General topics about the GNU findutils package are discussed at
       the bug-findutils mailing list:
              <https://lists.gnu.org/mailman/listinfo/bug-findutils>
COPYRIGHT
       Copyright (C) 1994-2024 Free Software Foundation, Inc.  License
       GPLv3+: GNU GPL version 3 or later
       <https://gnu.org/licenses/gpl.html>.
       This is free software: you are free to change and redistribute
       it.  There is NO WARRANTY, to the extent permitted by law.
SEE ALSO
       find(1), updatedb(1), xargs(1), glob(3), locatedb(5)

       Full documentation
       <https://www.gnu.org/software/findutils/locate>
       or available locally via: info locate
COLOPHON
       This page is part of the findutils (find utilities) project.
       Information about the project can be found at 
       http://www.gnu.org/software/findutils/.  If you have a bug
       report for this manual page, see
       https://savannah.gnu.org/bugs/?group=findutils.  This page was
       obtained from the project's upstream Git repository
       git://git.savannah.gnu.org/findutils.git on 2024-06-14.  (At
       that time, the date of the most recent commit that was found in
       the repository was 2024-06-03.)  If you discover any rendering
       problems in this HTML version of the page, or you believe there
       is a better or more up-to-date source for the page, or you have
       corrections or improvements to the information in this COLOPHON
       (which is not part of the original manual page), send a mail to
       man-pages@man7.org

                                                               LOCATE(1)
\end{lstlisting}
}}
\endinput  %  ==  ==  ==  ==  ==  ==  ==  ==  ==

	% % % \input{./components/man/man-ldd}
\subsection{\refLdd: Print Shared Object Dependencies}

{\tiny{
\begin{lstlisting}[language=bash]
NAME
       ldd - print shared object dependencies
SYNOPSIS
       ldd [option]... file...
DESCRIPTION
       ldd prints the shared objects (shared libraries) required by each
       program or shared object specified on the command line.  An
       example of its use and output is the following:

           $ ldd /bin/ls
               linux-vdso.so.1 (0x00007ffcc3563000)
               libselinux.so.1 => /lib64/libselinux.so.1 (0x00007f87e5459000)
               libcap.so.2 => /lib64/libcap.so.2 (0x00007f87e5254000)
               libc.so.6 => /lib64/libc.so.6 (0x00007f87e4e92000)
               libpcre.so.1 => /lib64/libpcre.so.1 (0x00007f87e4c22000)
               libdl.so.2 => /lib64/libdl.so.2 (0x00007f87e4a1e000)
               /lib64/ld-linux-x86-64.so.2 (0x00005574bf12e000)
               libattr.so.1 => /lib64/libattr.so.1 (0x00007f87e4817000)
               libpthread.so.0 => /lib64/libpthread.so.0 (0x00007f87e45fa000)

       In the usual case, ldd invokes the standard dynamic linker (see
       ld.so(8)) with the LD_TRACE_LOADED_OBJECTS environment variable
       set to 1.  This causes the dynamic linker to inspect the
       program's dynamic dependencies, and find (according to the rules
       described in ld.so(8)) and load the objects that satisfy those
       dependencies.  For each dependency, ldd displays the location of
       the matching object and the (hexadecimal) address at which it is
       loaded.  (The linux-vdso and ld-linux shared dependencies are
       special; see vdso(7) and ld.so(8).)

   Security
       Be aware that in some circumstances (e.g., where the program
       specifies an ELF interpreter other than ld-linux.so), some
       versions of ldd may attempt to obtain the dependency information
       by attempting to directly execute the program, which may lead to
       the execution of whatever code is defined in the program's ELF
       interpreter, and perhaps to execution of the program itself.
       (Before glibc 2.27, the upstream ldd implementation did this for
       example, although most distributions provided a modified version
       that did not.)

       Thus, you should never employ ldd on an untrusted executable,
       since this may result in the execution of arbitrary code.  A
       safer alternative when dealing with untrusted executables is:

           $ objdump -p /path/to/program | grep NEEDED

       Note, however, that this alternative shows only the direct
       dependencies of the executable, while ldd shows the entire
       dependency tree of the executable.
OPTIONS
       --version
              Print the version number of ldd.

       --verbose
       -v     Print all information, including, for example, symbol
              versioning information.

       --unused
       -u     Print unused direct dependencies.  (Since glibc 2.3.4.)

       --data-relocs
       -d     Perform relocations and report any missing objects (ELF
              only).

       --function-relocs
       -r     Perform relocations for both data objects and functions,
              and report any missing objects or functions (ELF only).

       --help Usage information.
BUGS
       ldd does not work on a.out shared libraries.

       ldd does not work with some extremely old a.out programs which
       were built before ldd support was added to the compiler releases.
       If you use ldd on one of these programs, the program will attempt
       to run with argc = 0 and the results will be unpredictable.
SEE ALSO
       pldd(1), sprof(1), ld.so(8), ldconfig(8)
COLOPHON
       This page is part of the man-pages (Linux kernel and C library
       user-space interface documentation) project.  Information about
       the project can be found at 
       https://www.kernel.org/doc/man-pages/.  If you have a bug report
       for this manual page, see
       https://git.kernel.org/pub/scm/docs/man-pages/man-pages.git/tree/CONTRIBUTING.
       This page was obtained from the tarball man-pages-6.9.1.tar.gz
       fetched from
       https://mirrors.edge.kernel.org/pub/linux/docs/man-pages/ on
       2024-06-26.  If you discover any rendering problems in this HTML
       version of the page, or you believe there is a better or more up-
       to-date source for the page, or you have corrections or
       improvements to the information in this COLOPHON (which is not
       part of the original manual page), send a mail to
       man-pages@man7.org

Linux man-pages 6.9.1          2024-05-02                         ldd(1)
\end{lstlisting}
}}
\endinput  %  ==  ==  ==  ==  ==  ==  ==  ==  ==

\subsection{\refLdd: Print Shared Object Dependencies}

{\tiny{
\begin{lstlisting}[language=bash]
NAME
       ldd - print shared object dependencies
SYNOPSIS
       ldd [option]... file...
DESCRIPTION
       ldd prints the shared objects (shared libraries) required by each
       program or shared object specified on the command line.  An
       example of its use and output is the following:

           $ ldd /bin/ls
               linux-vdso.so.1 (0x00007ffcc3563000)
               libselinux.so.1 => /lib64/libselinux.so.1 (0x00007f87e5459000)
               libcap.so.2 => /lib64/libcap.so.2 (0x00007f87e5254000)
               libc.so.6 => /lib64/libc.so.6 (0x00007f87e4e92000)
               libpcre.so.1 => /lib64/libpcre.so.1 (0x00007f87e4c22000)
               libdl.so.2 => /lib64/libdl.so.2 (0x00007f87e4a1e000)
               /lib64/ld-linux-x86-64.so.2 (0x00005574bf12e000)
               libattr.so.1 => /lib64/libattr.so.1 (0x00007f87e4817000)
               libpthread.so.0 => /lib64/libpthread.so.0 (0x00007f87e45fa000)

       In the usual case, ldd invokes the standard dynamic linker (see
       ld.so(8)) with the LD_TRACE_LOADED_OBJECTS environment variable
       set to 1.  This causes the dynamic linker to inspect the
       program's dynamic dependencies, and find (according to the rules
       described in ld.so(8)) and load the objects that satisfy those
       dependencies.  For each dependency, ldd displays the location of
       the matching object and the (hexadecimal) address at which it is
       loaded.  (The linux-vdso and ld-linux shared dependencies are
       special; see vdso(7) and ld.so(8).)

   Security
       Be aware that in some circumstances (e.g., where the program
       specifies an ELF interpreter other than ld-linux.so), some
       versions of ldd may attempt to obtain the dependency information
       by attempting to directly execute the program, which may lead to
       the execution of whatever code is defined in the program's ELF
       interpreter, and perhaps to execution of the program itself.
       (Before glibc 2.27, the upstream ldd implementation did this for
       example, although most distributions provided a modified version
       that did not.)

       Thus, you should never employ ldd on an untrusted executable,
       since this may result in the execution of arbitrary code.  A
       safer alternative when dealing with untrusted executables is:

           $ objdump -p /path/to/program | grep NEEDED

       Note, however, that this alternative shows only the direct
       dependencies of the executable, while ldd shows the entire
       dependency tree of the executable.
OPTIONS
       --version
              Print the version number of ldd.

       --verbose
       -v     Print all information, including, for example, symbol
              versioning information.

       --unused
       -u     Print unused direct dependencies.  (Since glibc 2.3.4.)

       --data-relocs
       -d     Perform relocations and report any missing objects (ELF
              only).

       --function-relocs
       -r     Perform relocations for both data objects and functions,
              and report any missing objects or functions (ELF only).

       --help Usage information.
BUGS
       ldd does not work on a.out shared libraries.

       ldd does not work with some extremely old a.out programs which
       were built before ldd support was added to the compiler releases.
       If you use ldd on one of these programs, the program will attempt
       to run with argc = 0 and the results will be unpredictable.
SEE ALSO
       pldd(1), sprof(1), ld.so(8), ldconfig(8)
COLOPHON
       This page is part of the man-pages (Linux kernel and C library
       user-space interface documentation) project.  Information about
       the project can be found at 
       https://www.kernel.org/doc/man-pages/.  If you have a bug report
       for this manual page, see
       https://git.kernel.org/pub/scm/docs/man-pages/man-pages.git/tree/CONTRIBUTING.
       This page was obtained from the tarball man-pages-6.9.1.tar.gz
       fetched from
       https://mirrors.edge.kernel.org/pub/linux/docs/man-pages/ on
       2024-06-26.  If you discover any rendering problems in this HTML
       version of the page, or you believe there is a better or more up-
       to-date source for the page, or you have corrections or
       improvements to the information in this COLOPHON (which is not
       part of the original manual page), send a mail to
       man-pages@man7.org

Linux man-pages 6.9.1          2024-05-02                         ldd(1)
\end{lstlisting}
}}
\endinput  %  ==  ==  ==  ==  ==  ==  ==  ==  ==

\subsection{\refLsof: Show Open Files}

{\tiny{
\begin{lstlisting}[language=bash]
NAME
       lsof - list open files
SYNOPSIS
       lsof [ -?abChlnNOPRtUvVX ] [ -A A ] [ -c c ] [ +c c ] [ +|-d d ]
       [ +|-D D ] [ +|-e s ] [ +|-E ] [ +|-f [cfgGn] ] [ -F [f] ] [ -g
       [s] ] [ -i [i] ] [ -k k ] [ -K k ] [ +|-L [l] ] [ +|-m m ] [ +|-M
       ] [ -o [o] ] [ -p s ] [ +|-r [t[m<fmt>]] ] [ -s [p:s] ] [ -S [t]
       ] [ -T [t] ] [ -u s ] [ +|-w ] [ -x [fl] ] [ -z [z] ] [ -Z [Z] ]
       [ -- ] [names]
DESCRIPTION
       Lsof revision 4.91 lists on its standard output file information
       about files opened by processes for the following UNIX dialects:

            Apple Darwin 9 and Mac OS X 10.[567]
            FreeBSD 8.[234], 9.0 and 1[012].0 for AMD64-based systems
            Linux 2.1.72 and above for x86-based systems
            Solaris 9, 10 and 11

       (See the DISTRIBUTION section of this manual page for information
       on how to obtain the latest lsof revision.)

       An open file may be a regular file, a directory, a block special
       file, a character special file, an executing text reference, a
       library, a stream or a network file (Internet socket, NFS file or
       UNIX domain socket.)  A specific file or all the files in a file
       system may be selected by path.

       Instead of a formatted display, lsof will produce output that can
       be parsed by other programs.  See the -F, option description, and
       the OUTPUT FOR OTHER PROGRAMS section for more information.

       In addition to producing a single output list, lsof will run in
       repeat mode.  In repeat mode it will produce output, delay, then
       repeat the output operation until stopped with an interrupt or
       quit signal.  See the +|-r [t[m<fmt>]] option description for
       more information.
OPTIONS
       In the absence of any options, lsof lists all open files
       belonging to all active processes.

       If any list request option is specified, other list requests must
       be specifically requested - e.g., if -U is specified for the
       listing of UNIX socket files, NFS files won't be listed unless -N
       is also specified; or if a user list is specified with the -u
       option, UNIX domain socket files, belonging to users not in the
       list, won't be listed unless the -U option is also specified.

       Normally list options that are specifically stated are ORed -
       i.e., specifying the -i option without an address and the -ufoo
       option produces a listing of all network files OR files belonging
       to processes owned by user ``foo''.  The exceptions are:

       1) the `^' (negated) login name or user ID (UID), specified with
          the -u option;

       2) the `^' (negated) process ID (PID), specified with the -p
          option;

       3) the `^' (negated) process group ID (PGID), specified with the
          -g option;

       4) the `^' (negated) command, specified with the -c option;

       5) the (`^') negated TCP or UDP protocol state names, specified
          with the -s [p:s] option.

       Since they represent exclusions, they are applied without ORing
       or ANDing and take effect before any other selection criteria are
       applied.

       The -a option may be used to AND the selections.  For example,
       specifying -a, -U, and -ufoo produces a listing of only UNIX
       socket files that belong to processes owned by user ``foo''.

       Caution: the -a option causes all list selection options to be
       ANDed; it can't be used to cause ANDing of selected pairs of
       selection options by placing it between them, even though its
       placement there is acceptable.  Wherever -a is placed, it causes
       the ANDing of all selection options.

       Items of the same selection set - command names, file
       descriptors, network addresses, process identifiers, user
       identifiers, zone names, security contexts - are joined in a
       single ORed set and applied before the result participates in
       ANDing.  Thus, for example, specifying -i@aaa.bbb, -i@ccc.ddd,
       -a, and -ufff,ggg will select the listing of files that belong to
       either login ``fff'' OR ``ggg'' AND have network connections to
       either host aaa.bbb OR ccc.ddd.

       Options may be grouped together following a single prefix --
       e.g., the option set ``-a -b -C'' may be stated as -abC.
       However, since values are optional following +|-f, -F, -g, -i,
       +|-L, -o, +|-r, -s, -S, -T, -x and -z.  when you have no values
       for them be careful that the following character isn't ambiguous.
       For example, -Fn might represent the -F and -n options, or it
       might represent the n field identifier character following the -F
       option.  When ambiguity is possible, start a new option with a
       `-' character - e.g., ``-F -n''.  If the next option is a file
       name, follow the possibly ambiguous option with ``--'' - e.g.,
       ``-F -- name''.

       Either the `+' or the `-' prefix may be applied to a group of
       options.  Options that don't take on separate meanings for each
       prefix - e.g., -i - may be grouped under either prefix.  Thus,
       for example, ``+M -i'' may be stated as ``+Mi'' and the group
       means the same as the separate options.  Be careful of prefix
       grouping when one or more options in the group does take on
       separate meanings under different prefixes - e.g., +|-M; ``-iM''
       is not the same request as ``-i +M''.  When in doubt, use
       separate options with appropriate prefixes.

       -? -h  These two equivalent options select a usage (help) output
              list.  Lsof displays a shortened form of this output when
              it detects an error in the options supplied to it, after
              it has displayed messages explaining each error.  (Escape
              the `?' character as your shell requires.)

       -a     causes list selection options to be ANDed, as described
              above.

       -A A   is available on systems configured for AFS whose AFS
              kernel code is implemented via dynamic modules.  It allows
              the lsof user to specify A as an alternate name list file
              where the kernel addresses of the dynamic modules might be
              found.  See the lsof FAQ (The FAQ section gives its
              location.)  for more information about dynamic modules,
              their symbols, and how they affect lsof.

       -b     causes lsof to avoid kernel functions that might block -
              lstat(2), readlink(2), and stat(2).

              See the BLOCKS AND TIMEOUTS and AVOIDING KERNEL BLOCKS
              sections for information on using this option.

       -c c   selects the listing of files for processes executing the
              command that begins with the characters of c.  Multiple
              commands may be specified, using multiple -c options.
              They are joined in a single ORed set before participating
              in AND option selection.

              If c begins with a `^', then the following characters
              specify a command name whose processes are to be ignored
              (excluded.)

              If c begins and ends with a slash ('/'), the characters
              between the slashes are interpreted as a regular
              expression.  Shell meta-characters in the regular
              expression must be quoted to prevent their interpretation
              by the shell.  The closing slash may be followed by these
              modifiers:

                   b    the regular expression is a basic one.
                   i    ignore the case of letters.
                   x    the regular expression is an extended one
                        (default).

              See the lsof FAQ (The FAQ section gives its location.)
              for more information on basic and extended regular
              expressions.

              The simple command specification is tested first.  If that
              test fails, the command regular expression is applied.  If
              the simple command test succeeds, the command regular
              expression test isn't made.  This may result in ``no
              command found for regex:'' messages when lsof's -V option
              is specified.

       +c w   defines the maximum number of initial characters of the
              name, supplied by the UNIX dialect, of the UNIX command
              associated with a process to be printed in the COMMAND
              column.  (The lsof default is nine.)

              Note that many UNIX dialects do not supply all command
              name characters to lsof in the files and structures from
              which lsof obtains command name.  Often dialects limit the
              number of characters supplied in those sources.  For
              example, Linux 2.4.27 and Solaris 9 both limit command
              name length to 16 characters.

              If w is zero ('0'), all command characters supplied to
              lsof by the UNIX dialect will be printed.

              If w is less than the length of the column title,
              ``COMMAND'', it will be raised to that length.

       -C     disables the reporting of any path name components from
              the kernel's name cache.  See the KERNEL NAME CACHE
              section for more information.

       +d s   causes lsof to search for all open instances of directory
              s and the files and directories it contains at its top
              level.  +d does NOT descend the directory tree, rooted at
              s.  The +D D option may be used to request a full-descent
              directory tree search, rooted at directory D.

              Processing of the +d option does not follow symbolic links
              within s unless the -x or -x  l option is also specified.
              Nor does it search for open files on file system mount
              points on subdirectories of s unless the -x or -x  f
              option is also specified.

              Note: the authority of the user of this option limits it
              to searching for files that the user has permission to
              examine with the system stat(2) function.

       -d s   specifies a list of file descriptors (FDs) to exclude from
              or include in the output listing.  The file descriptors
              are specified in the comma-separated set s - e.g.,
              ``cwd,1,3'', ``^6,^2''.  (There should be no spaces in the
              set.)

              The list is an exclusion list if all entries of the set
              begin with `^'.  It is an inclusion list if no entry
              begins with `^'.  Mixed lists are not permitted.

              A file descriptor number range may be in the set as long
              as neither member is empty, both members are numbers, and
              the ending member is larger than the starting one - e.g.,
              ``0-7'' or ``3-10''.  Ranges may be specified for
              exclusion if they have the `^' prefix - e.g., ``^0-7''
              excludes all file descriptors 0 through 7.

              Multiple file descriptor numbers are joined in a single
              ORed set before participating in AND option selection.

              When there are exclusion and inclusion members in the set,
              lsof reports them as errors and exits with a non-zero
              return code.

              See the description of File Descriptor (FD) output values
              in the OUTPUT section for more information on file
              descriptor names.

       +D D   causes lsof to search for all open instances of directory
              D and all the files and directories it contains to its
              complete depth.

              Processing of the +D option does not follow symbolic links
              within D unless the -x or -x  l option is also specified.
              Nor does it search for open files on file system mount
              points on subdirectories of D unless the -x or -x  f
              option is also specified.

              Note: the authority of the user of this option limits it
              to searching for files that the user has permission to
              examine with the system stat(2) function.

              Further note: lsof may process this option slowly and
              require a large amount of dynamic memory to do it.  This
              is because it must descend the entire directory tree,
              rooted at D, calling stat(2) for each file and directory,
              building a list of all the files it finds, and searching
              that list for a match with every open file.  When
              directory D is large, these steps can take a long time, so
              use this option prudently.

       -D D   directs lsof's use of the device cache file.  The use of
              this option is sometimes restricted.  See the DEVICE CACHE
              FILE section and the sections that follow it for more
              information on this option.

              -D must be followed by a function letter; the function
              letter may optionally be followed by a path name.  Lsof
              recognizes these function letters:

                   ? - report device cache file paths
                   b - build the device cache file
                   i - ignore the device cache file
                   r - read the device cache file
                   u - read and update the device cache file

              The b, r, and u functions, accompanied by a path name, are
              sometimes restricted.  When these functions are
              restricted, they will not appear in the description of the
              -D option that accompanies -h or -?  option output.  See
              the DEVICE CACHE FILE section and the sections that follow
              it for more information on these functions and when
              they're restricted.

              The ?  function reports the read-only and write paths that
              lsof can use for the device cache file, the names of any
              environment variables whose values lsof will examine when
              forming the device cache file path, and the format for the
              personal device cache file path.  (Escape the `?'
              character as your shell requires.)

              When available, the b, r, and u functions may be followed
              by the device cache file's path.  The standard default is
              .lsof_hostname in the home directory of the real user ID
              that executes lsof, but this could have been changed when
              lsof was configured and compiled.  (The output of the -h
              and -?  options show the current default prefix - e.g.,
              ``.lsof''.)  The suffix, hostname, is the first component
              of the host's name returned by gethostname(2).

              When available, the b function directs lsof to build a new
              device cache file at the default or specified path.

              The i function directs lsof to ignore the default device
              cache file and obtain its information about devices via
              direct calls to the kernel.

              The r function directs lsof to read the device cache at
              the default or specified path, but prevents it from
              creating a new device cache file when none exists or the
              existing one is improperly structured.  The r function,
              when specified without a path name, prevents lsof from
              updating an incorrect or outdated device cache file, or
              creating a new one in its place.  The r function is always
              available when it is specified without a path name
              argument; it may be restricted by the permissions of the
              lsof process.

              When available, the u function directs lsof to read the
              device cache file at the default or specified path, if
              possible, and to rebuild it, if necessary.  This is the
              default device cache file function when no -D option has
              been specified.

       +|-e s exempts the file system whose path name is s from being
              subjected to kernel function calls that might block.  The
              +e option exempts stat(2), lstat(2) and most readlink(2)
              kernel function calls.  The -e option exempts only stat(2)
              and lstat(2) kernel function calls.  Multiple file systems
              may be specified with separate +|-e specifications and
              each may have readlink(2) calls exempted or not.

              This option is currently implemented only for Linux.

              CAUTION: this option can easily be mis-applied to other
              than the file system of interest, because it uses path
              name rather than the more reliable device and inode
              numbers.  (Device and inode numbers are acquired via the
              potentially blocking stat(2) kernel call and are thus not
              available, but see the +|-m m option as a possible
              alternative way to supply device numbers.)  Use this
              option with great care and fully specify the path name of
              the file system to be exempted.

              When open files on exempted file systems are reported, it
              may not be possible to obtain all their information.
              Therefore, some information columns will be blank, the
              characters ``UNKN'' preface the values in the TYPE column,
              and the applicable exemption option is added in
              parentheses to the end of the NAME column.  (Some device
              number information might be made available via the +|-m m
              option.)

       +|-E   +E specifies that Linux pipe, Linux UNIX socket and Linux
              pseudoterminal files should be displayed with endpoint
              information and the files of the endpoints should also be
              displayed.  Note: UNIX socket file endpoint information is
              only available when the compile flags line of -v output
              contains HASUXSOCKEPT, and psudoterminal endpoint
              information is only available when the compile flags line
              contains HASPTYEPT.

              Pipe endpoint information is displayed in the NAME column
              in the form ``PID,cmd,FDmode'', where PID is the endpoint
              process ID; cmd is the endpoint process command; FD is the
              endpoint file's descriptor; and mode is the endpoint
              file's access mode.

              Pseudoterminal endpoint information is displayed in the
              NAME column as ``->/dev/ptsmin PID,cmd,FDmode'' or
              ``PID,cmd,FDmode''.  The first form is for a master
              device; the second, for a slave device.  min is a slave
              device's minor device number; and PID, cmd, FD and mode
              are the same as with pipe endpoint information.  Note:
              psudoterminal endpoint information is only available when
              the compile flags line of -V output contains HASPTYEPT.

              UNIX socket file endpoint information is displayed in the
              NAME column in the form
              ``type=TYPE ->INO=INODE PID,cmd,FDmode'', where TYPE is
              the socket type; INODE is the i-node number of the
              connected socket; and PID, cmd, FD and mode are the same
              as with pipe endpoint information.  Note: UNIX socket file
              endpoint information is available only when the compile
              flags line of -v output contains HASUXSOCKEPT.

              Multiple occurrences of this information can appear in a
              file's NAME column.

              -E specfies that Linux pipe and Linux UNIX socket files
              should be displayed with endpoint information, but not the
              files of the endpoints.

       +|-f [cfgGn]
              f by itself clarifies how path name arguments are to be
              interpreted.  When followed by c, f, g, G, or n in any
              combination it specifies that the listing of kernel file
              structure information is to be enabled (`+') or inhibited
              (`-').

              Normally a path name argument is taken to be a file system
              name if it matches a mounted-on directory name reported by
              mount(8), or if it represents a block device, named in the
              mount output and associated with a mounted directory name.
              When +f is specified, all path name arguments will be
              taken to be file system names, and lsof will complain if
              any are not.  This can be useful, for example, when the
              file system name (mounted-on device) isn't a block device.
              This happens for some CD-ROM file systems.

              When -f is specified by itself, all path name arguments
              will be taken to be simple files.  Thus, for example, the
              ``-f -- /'' arguments direct lsof to search for open files
              with a `/' path name, not all open files in the `/' (root)
              file system.

              Be careful to make sure +f and -f are properly terminated
              and aren't followed by a character (e.g., of the file or
              file system name) that might be taken as a parameter.  For
              example, use ``--'' after +f and -f as in these examples.

                   $ lsof +f -- /file/system/name
                   $ lsof -f -- /file/name

              The listing of information from kernel file structures,
              requested with the +f [cfgGn] option form, is normally
              inhibited, and is not available in whole or part for some
              dialects - e.g., /proc-based Linux kernels below 2.6.22.
              When the prefix to f is a plus sign (`+'), these
              characters request file structure information:

                   c    file structure use count (not Linux)
                   f    file structure address (not Linux)
                   g    file flag abbreviations (Linux 2.6.22 and up)
                   G    file flags in hexadecimal (Linux 2.6.22 and up)
                   n    file structure node address (not Linux)

              When the prefix is minus (`-') the same characters disable
              the listing of the indicated values.

              File structure addresses, use counts, flags, and node
              addresses may be used to detect more readily identical
              files inherited by child processes and identical files in
              use by different processes.  Lsof column output can be
              sorted by output columns holding the values and listed to
              identify identical file use, or lsof field output can be
              parsed by an AWK or Perl post-filter script, or by a C
              program.

       -F f   specifies a character list, f, that selects the fields to
              be output for processing by another program, and the
              character that terminates each output field.  Each field
              to be output is specified with a single character in f.
              The field terminator defaults to NL, but may be changed to
              NUL (000).  See the OUTPUT FOR OTHER PROGRAMS section for
              a description of the field identification characters and
              the field output process.

              When the field selection character list is empty, all
              standard fields are selected (except the raw device field,
              security context and zone field for compatibility reasons)
              and the NL field terminator is used.

              When the field selection character list contains only a
              zero (`0'), all fields are selected (except the raw device
              field for compatibility reasons) and the NUL terminator
              character is used.

              Other combinations of fields and their associated field
              terminator character must be set with explicit entries in
              f, as described in the OUTPUT FOR OTHER PROGRAMS section.

              When a field selection character identifies an item lsof
              does not normally list - e.g., PPID, selected with -R -
              specification of the field character - e.g., ``-FR'' -
              also selects the listing of the item.

              When the field selection character list contains the
              single character `?', lsof will display a help list of the
              field identification characters.  (Escape the `?'
              character as your shell requires.)

       -g [s] excludes or selects the listing of files for the processes
              whose optional process group IDentification (PGID) numbers
              are in the comma-separated set s - e.g., ``123'' or
              ``123,^456''.  (There should be no spaces in the set.)

              PGID numbers that begin with `^' (negation) represent
              exclusions.

              Multiple PGID numbers are joined in a single ORed set
              before participating in AND option selection.  However,
              PGID exclusions are applied without ORing or ANDing and
              take effect before other selection criteria are applied.

              The -g option also enables the output display of PGID
              numbers.  When specified without a PGID set that's all it
              does.

       -i [i] selects the listing of files any of whose Internet address
              matches the address specified in i.  If no address is
              specified, this option selects the listing of all Internet
              and x.25 (HP-UX) network files.

              If -i4 or -i6 is specified with no following address, only
              files of the indicated IP version, IPv4 or IPv6, are
              displayed.  (An IPv6 specification may be used only if the
              dialects supports IPv6, as indicated by ``[46]'' and
              ``IPv[46]'' in lsof's -h or -?  output.)  Sequentially
              specifying -i4, followed by -i6 is the same as specifying
              -i, and vice-versa.  Specifying -i4, or -i6 after -i is
              the same as specifying -i4 or -i6 by itself.

              Multiple addresses (up to a limit of 100) may be specified
              with multiple -i options.  (A port number or service name
              range is counted as one address.)  They are joined in a
              single ORed set before participating in AND option
              selection.

              An Internet address is specified in the form (Items in
              square brackets are optional.):

              [46][protocol][@hostname|hostaddr][:service|port]

              where:
                   46 specifies the IP version, IPv4 or IPv6
                        that applies to the following address.
                        '6' may be be specified only if the UNIX
                        dialect supports IPv6.  If neither '4' nor
                        '6' is specified, the following address
                        applies to all IP versions.
                   protocol is a protocol name - TCP, UDP
                   hostname is an Internet host name.  Unless a
                        specific IP version is specified, open
                        network files associated with host names
                        of all versions will be selected.
                   hostaddr is a numeric Internet IPv4 address in
                        dot form; or an IPv6 numeric address in
                        colon form, enclosed in brackets, if the
                        UNIX dialect supports IPv6.  When an IP
                        version is selected, only its numeric
                        addresses may be specified.
                   service is an /etc/services name - e.g., smtp -
                        or a list of them.
                   port is a port number, or a list of them.

              IPv6 options may be used only if the UNIX dialect supports
              IPv6.  To see if the dialect supports IPv6, run lsof and
              specify the -h or -?  (help) option.  If the displayed
              description of the -i option contains ``[46]'' and
              ``IPv[46]'', IPv6 is supported.

              IPv4 host names and addresses may not be specified if
              network file selection is limited to IPv6 with -i 6.  IPv6
              host names and addresses may not be specified if network
              file selection is limited to IPv4 with -i 4.  When an open
              IPv4 network file's address is mapped in an IPv6 address,
              the open file's type will be IPv6, not IPv4, and its
              display will be selected by '6', not '4'.

              At least one address component - 4, 6, protocol, hostname,
              hostaddr, or service - must be supplied.  The `@'
              character, leading the host specification, is always
              required; as is the `:', leading the port specification.
              Specify either hostname or hostaddr.  Specify either
              service name list or port number list.  If a service name
              list is specified, the protocol may also need to be
              specified if the TCP, UDP and UDPLITE port numbers for the
              service name are different.  Use any case - lower or upper
              - for protocol.

              Service names and port numbers may be combined in a list
              whose entries are separated by commas and whose numeric
              range entries are separated by minus signs.  There may be
              no embedded spaces, and all service names must belong to
              the specified protocol.  Since service names may contain
              embedded minus signs, the starting entry of a range can't
              be a service name; it can be a port number, however.

              Here are some sample addresses:

                   -i6 - IPv6 only
                   TCP:25 - TCP and port 25
                   @1.2.3.4 - Internet IPv4 host address 1.2.3.4
                   @[3ffe:1ebc::1]:1234 - Internet IPv6 host address
                        3ffe:1ebc::1, port 1234
                   UDP:who - UDP who service port
                   TCP@lsof.itap:513 - TCP, port 513 and host name lsof.itap
                   tcp@foo:1-10,smtp,99 - TCP, ports 1 through 10,
                        service name smtp, port 99, host name foo
                   tcp@bar:1-smtp - TCP, ports 1 through smtp, host bar
                   :time - either TCP, UDP or UDPLITE time service port

       -K k   selects the listing of tasks (threads) of processes, on
              dialects where task (thread) reporting is supported.  (If
              help output - i.e., the output of the -h or -?  options -
              shows this option, then task (thread) reporting is
              supported by the dialect.)

              If -K is followed by a value, k, it must be ``i''.  That
              causes lsof to ignore tasks, particularly in the default,
              list-everything case when no other options are specified.

              When -K and -a are both specified on Linux, and the tasks
              of a main process are selected by other options, the main
              process will also be listed as though it were a task, but
              without a task ID.  (See the description of the TID column
              in the OUTPUT section.)

              Where the FreeBSD version supports threads, all threads
              will be listed with their IDs.

              In general threads and tasks inherit the files of the
              caller, but may close some and open others, so lsof always
              reports all the open files of threads and tasks.

       -k k   specifies a kernel name list file, k, in place of /vmunix,
              /mach, etc.  -k is not available under AIX on the IBM
              RISC/System 6000.

       -l     inhibits the conversion of user ID numbers to login names.
              It is also useful when login name lookup is working
              improperly or slowly.

       +|-L [l]
              enables (`+') or disables (`-') the listing of file link
              counts, where they are available - e.g., they aren't
              available for sockets, or most FIFOs and pipes.

              When +L is specified without a following number, all link
              counts will be listed.  When -L is specified (the
              default), no link counts will be listed.

              When +L is followed by a number, only files having a link
              count less than that number will be listed.  (No number
              may follow -L.)  A specification of the form ``+L1'' will
              select open files that have been unlinked.  A
              specification of the form ``+aL1 <file_system>'' will
              select unlinked open files on the specified file system.

              For other link count comparisons, use field output (-F)
              and a post-processing script or program.

       +|-m m specifies an alternate kernel memory file or activates
              mount table supplement processing.

              The option form -m m specifies a kernel memory file, m, in
              place of /dev/kmem or /dev/mem - e.g., a crash dump file.

              The option form +m requests that a mount supplement file
              be written to the standard output file.  All other options
              are silently ignored.

              There will be a line in the mount supplement file for each
              mounted file system, containing the mounted file system
              directory, followed by a single space, followed by the
              device number in hexadecimal "0x" format - e.g.,

                   / 0x801

              Lsof can use the mount supplement file to get device
              numbers for file systems when it can't get them via
              stat(2) or lstat(2).

              The option form +m m identifies m as a mount supplement
              file.

              Note: the +m and +m m options are not available for all
              supported dialects.  Check the output of lsof's -h or -?
              options to see if the +m and +m m options are available.

       +|-M   Enables (+) or disables (-) the reporting of portmapper
              registrations for local TCP, UDP and UDPLITE ports, where
              port mapping is supported.  (See the last paragraph of
              this option description for information about where
              portmapper registration reporting is supported.)

              The default reporting mode is set by the lsof builder with
              the HASPMAPENABLED #define in the dialect's machine.h
              header file; lsof is distributed with the HASPMAPENABLED
              #define deactivated, so portmapper reporting is disabled
              by default and must be requested with +M.  Specifying
              lsof's -h or -?  option will report the default mode.
              Disabling portmapper registration when it is already
              disabled or enabling it when already enabled is
              acceptable.  When portmapper registration reporting is
              enabled, lsof displays the portmapper registration (if
              any) for local TCP, UDP or UDPLITE ports in square
              brackets immediately following the port numbers or service
              names - e.g., ``:1234[name]'' or ``:name[100083]''.  The
              registration information may be a name or number,
              depending on what the registering program supplied to the
              portmapper when it registered the port.

              When portmapper registration reporting is enabled, lsof
              may run a little more slowly or even become blocked when
              access to the portmapper becomes congested or stopped.
              Reverse the reporting mode to determine if portmapper
              registration reporting is slowing or blocking lsof.

              For purposes of portmapper registration reporting lsof
              considers a TCP, UDP or UDPLITE port local if: it is found
              in the local part of its containing kernel structure; or
              if it is located in the foreign part of its containing
              kernel structure and the local and foreign Internet
              addresses are the same; or if it is located in the foreign
              part of its containing kernel structure and the foreign
              Internet address is INADDR_LOOPBACK (127.0.0.1).  This
              rule may make lsof ignore some foreign ports on machines
              with multiple interfaces when the foreign Internet address
              is on a different interface from the local one.

              See the lsof FAQ (The FAQ section gives its location.)
              for further discussion of portmapper registration
              reporting issues.

              Portmapper registration reporting is supported only on
              dialects that have RPC header files.  (Some Linux
              distributions with GlibC 2.14 do not have them.)  When
              portmapper registration reporting is supported, the -h or
              -?  help output will show the +|-M option.

       -n     inhibits the conversion of network numbers to host names
              for network files.  Inhibiting conversion may make lsof
              run faster.  It is also useful when host name lookup is
              not working properly.

       -N     selects the listing of NFS files.

       -o     directs lsof to display file offset at all times.  It
              causes the SIZE/OFF output column title to be changed to
              OFFSET.  Note: on some UNIX dialects lsof can't obtain
              accurate or consistent file offset information from its
              kernel data sources, sometimes just for particular kinds
              of files (e.g., socket files.)  Consult the lsof FAQ (The
              FAQ section gives its location.)  for more information.

              The -o and -s options are mutually exclusive; they can't
              both be specified.  When neither is specified, lsof
              displays whatever value - size or offset - is appropriate
              and available for the type of the file.

       -o o   defines the number of decimal digits (o) to be printed
              after the ``0t'' for a file offset before the form is
              switched to ``0x...''.  An o value of zero (unlimited)
              directs lsof to use the ``0t'' form for all offset output.

              This option does NOT direct lsof to display offset at all
              times; specify -o (without a trailing number) to do that.
              -o o only specifies the number of digits after ``0t'' in
              either mixed size and offset or offset-only output.  Thus,
              for example, to direct lsof to display offset at all times
              with a decimal digit count of 10, use:

                   -o -o 10
              or
                   -oo10

              The default number of digits allowed after ``0t'' is
              normally 8, but may have been changed by the lsof builder.
              Consult the description of the -o o option in the output
              of the -h or -?  option to determine the default that is
              in effect.

       -O     directs lsof to bypass the strategy it uses to avoid being
              blocked by some kernel operations - i.e., doing them in
              forked child processes.  See the BLOCKS AND TIMEOUTS and
              AVOIDING KERNEL BLOCKS sections for more information on
              kernel operations that may block lsof.

              While use of this option will reduce lsof startup
              overhead, it may also cause lsof to hang when the kernel
              doesn't respond to a function.  Use this option
              cautiously.

       -p s   excludes or selects the listing of files for the processes
              whose optional process IDentification (PID) numbers are in
              the comma-separated set s - e.g., ``123'' or ``123,^456''.
              (There should be no spaces in the set.)

              PID numbers that begin with `^' (negation) represent
              exclusions.

              Multiple process ID numbers are joined in a single ORed
              set before participating in AND option selection.
              However, PID exclusions are applied without ORing or
              ANDing and take effect before other selection criteria are
              applied.

       -P     inhibits the conversion of port numbers to port names for
              network files.  Inhibiting the conversion may make lsof
              run a little faster.  It is also useful when port name
              lookup is not working properly.

       +|-r [t[m<fmt>]]
              puts lsof in repeat mode.  There lsof lists open files as
              selected by other options, delays t seconds (default
              fifteen), then repeats the listing, delaying and listing
              repetitively until stopped by a condition defined by the
              prefix to the option.

              If the prefix is a `-', repeat mode is endless.  Lsof must
              be terminated with an interrupt or quit signal.

              If the prefix is `+', repeat mode will end the first cycle
              no open files are listed - and of course when lsof is
              stopped with an interrupt or quit signal.  When repeat
              mode ends because no files are listed, the process exit
              code will be zero if any open files were ever listed; one,
              if none were ever listed.

              Lsof marks the end of each listing: if field output is in
              progress (the -F, option has been specified), the default
              marker is `m'; otherwise the default marker is
              ``========''.  The marker is followed by a NL character.

              The optional "m<fmt>" argument specifies a format for the
              marker line.  The <fmt> characters following `m' are
              interpreted as a format specification to the strftime(3)
              function, when both it and the localtime(3) function are
              available in the dialect's C library.  Consult the
              strftime(3) documentation for what may appear in its
              format specification.  Note that when field output is
              requested with the -F option, <fmt> cannot contain the NL
              format, ``%n''.  Note also that when <fmt> contains spaces
              or other characters that affect the shell's interpretation
              of arguments, <fmt> must be quoted appropriately.

              Repeat mode reduces lsof startup overhead, so it is more
              efficient to use this mode than to call lsof repetitively
              from a shell script, for example.

              To use repeat mode most efficiently, accompany +|-r with
              specification of other lsof selection options, so the
              amount of kernel memory access lsof does will be kept to a
              minimum.  Options that filter at the process level - e.g.,
              -c, -g, -p, -u - are the most efficient selectors.

              Repeat mode is useful when coupled with field output (see
              the -F, option description) and a supervising awk or Perl
              script, or a C program.

       -R     directs lsof to list the Parent Process IDentification
              number in the PPID column.

       -s [p:s]
              s alone directs lsof to display file size at all times.
              It causes the SIZE/OFF output column title to be changed
              to SIZE.  If the file does not have a size, nothing is
              displayed.

              The optional -s p:s form is available only for selected
              dialects, and only when the -h or -?  help output lists
              it.

              When the optional form is available, the s may be followed
              by a protocol name (p), either TCP or UDP, a colon (`:')
              and a comma-separated protocol state name list, the option
              causes open TCP and UDP files to be excluded if their
              state name(s) are in the list (s) preceded by a `^'; or
              included if their name(s) are not preceded by a `^'.

              Dialects that support this option may support only one
              protocol.  When an unsupported protocol is specified, a
              message will be displayed indicating state names for the
              protocol are unavailable.

              When an inclusion list is defined, only network files with
              state names in the list will be present in the lsof
              output.  Thus, specifying one state name means that only
              network files with that lone state name will be listed.

              Case is unimportant in the protocol or state names, but
              there may be no spaces and the colon (`:') separating the
              protocol name (p) and the state name list (s) is required.

              If only TCP and UDP files are to be listed, as controlled
              by the specified exclusions and inclusions, the -i option
              must be specified, too.  If only a single protocol's files
              are to be listed, add its name as an argument to the -i
              option.

              For example, to list only network files with TCP state
              LISTEN, use:

                   -iTCP -sTCP:LISTEN

              Or, for example, to list network files with all UDP states
              except Idle, use:

                   -iUDP -sUDP:Idle

              State names vary with UNIX dialects, so it's not possible
              to provide a complete list.  Some common TCP state names
              are: CLOSED, IDLE, BOUND, LISTEN, ESTABLISHED, SYN_SENT,
              SYN_RCDV, ESTABLISHED, CLOSE_WAIT, FIN_WAIT1, CLOSING,
              LAST_ACK, FIN_WAIT_2, and TIME_WAIT.  Two common UDP state
              names are Unbound and Idle.

              See the lsof FAQ (The FAQ section gives its location.)
              for more information on how to use protocol state
              exclusion and inclusion, including examples.

              The -o (without a following decimal digit count) and -s
              option (without a following protocol and state name list)
              are mutually exclusive; they can't both be specified.
              When neither is specified, lsof displays whatever value -
              size or offset - is appropriate and available for the type
              of file.

              Since some types of files don't have true sizes - sockets,
              FIFOs, pipes, etc. - lsof displays for their sizes the
              content amounts in their associated kernel buffers, if
              possible.

       -S [t] specifies an optional time-out seconds value for kernel
              functions - lstat(2), readlink(2), and stat(2) - that
              might otherwise deadlock.  The minimum for t is two; the
              default, fifteen; when no value is specified, the default
              is used.

              See the BLOCKS AND TIMEOUTS section for more information.

       -T [t] controls the reporting of some TCP/TPI information, also
              reported by netstat(1), following the network addresses.
              In normal output the information appears in parentheses,
              each item except TCP or TPI state name identified by a
              keyword, followed by `=', separated from others by a
              single space:

                   <TCP or TPI state name>
                   QR=<read queue length>
                   QS=<send queue length>
                   SO=<socket options and values>
                   SS=<socket states>
                   TF=<TCP flags and values>
                   WR=<window read length>
                   WW=<window write length>

              Not all values are reported for all UNIX dialects.  Items
              values (when available) are reported after the item name
              and '='.

              When the field output mode is in effect (See OUTPUT FOR
              OTHER PROGRAMS.)  each item appears as a field with a `T'
              leading character.

              -T with no following key characters disables TCP/TPI
              information reporting.

              -T with following characters selects the reporting of
              specific TCP/TPI information:

                   f    selects reporting of socket options,
                        states and values, and TCP flags and
                        values.
                   q    selects queue length reporting.
                   s    selects connection state reporting.
                   w    selects window size reporting.

              Not all selections are enabled for some UNIX dialects.
              State may be selected for all dialects and is reported by
              default.  The -h or -?  help output for the -T option will
              show what selections may be used with the UNIX dialect.

              When -T is used to select information - i.e., it is
              followed by one or more selection characters - the
              displaying of state is disabled by default, and it must be
              explicitly selected again in the characters following -T.
              (In effect, then, the default is equivalent to -Ts.)  For
              example, if queue lengths and state are desired, use -Tqs.

              Socket options, socket states, some socket values, TCP
              flags and one TCP value may be reported (when available in
              the UNIX dialect) in the form of the names that commonly
              appear after SO_, so_, SS_, TCP_  and TF_ in the dialect's
              header files - most often <sys/socket.h>,
              <sys/socketvar.h> and <netinet/tcp_var.h>.  Consult those
              header files for the meaning of the flags, options, states
              and values.

              ``SO='' precedes socket options and values; ``SS='',
              socket states; and ``TF='', TCP flags and values.

              If a flag or option has a value, the value will follow an
              '=' and the name -- e.g., ``SO=LINGER=5'', ``SO=QLIM=5'',
              ``TF=MSS=512''.  The following seven values may be
              reported:

                   Name
                   Reported  Description (Common Symbol)

                   KEEPALIVE keep alive time (SO_KEEPALIVE)
                   LINGER    linger time (SO_LINGER)
                   MSS       maximum segment size (TCP_MAXSEG)
                   PQLEN          partial listen queue connections
                   QLEN      established listen queue connections
                   QLIM      established listen queue limit
                   RCVBUF    receive buffer length (SO_RCVBUF)
                   SNDBUF    send buffer length (SO_SNDBUF)

              Details on what socket options and values, socket states,
              and TCP flags and values may be displayed for particular
              UNIX dialects may be found in the answer to the ``Why
              doesn't lsof report socket options, socket states, and TCP
              flags and values for my dialect?'' and ``Why doesn't lsof
              report the partial listen queue connection count for my
              dialect?''  questions in the lsof FAQ (The FAQ section
              gives its location.)

       -t     specifies that lsof should produce terse output with
              process identifiers only and no header - e.g., so that the
              output may be piped to kill(1).  -t selects the -w option.

       -u s   selects the listing of files for the user whose login
              names or user ID numbers are in the comma-separated set s
              - e.g., ``abe'', or ``548,root''.  (There should be no
              spaces in the set.)

              Multiple login names or user ID numbers are joined in a
              single ORed set before participating in AND option
              selection.

              If a login name or user ID is preceded by a `^', it
              becomes a negation - i.e., files of processes owned by the
              login name or user ID will never be listed.  A negated
              login name or user ID selection is neither ANDed nor ORed
              with other selections; it is applied before all other
              selections and absolutely excludes the listing of the
              files of the process.  For example, to direct lsof to
              exclude the listing of files belonging to root processes,
              specify ``-u^root'' or ``-u^0''.

       -U     selects the listing of UNIX domain socket files.

       -v     selects the listing of lsof version information,
              including: revision number; when the lsof binary was
              constructed; who constructed the binary and where; the
              name of the compiler used to construct the lsof binary;
              the version number of the compiler when readily available;
              the compiler and loader flags used to construct the lsof
              binary; and system information, typically the output of
              uname's -a option.

       -V     directs lsof to indicate the items it was asked to list
              and failed to find - command names, file names, Internet
              addresses or files, login names, NFS files, PIDs, PGIDs,
              and UIDs.

              When other options are ANDed to search options, or
              compile-time options restrict the listing of some files,
              lsof may not report that it failed to find a search item
              when an ANDed option or compile-time option prevents the
              listing of the open file containing the located search
              item.

              For example, ``lsof -V -iTCP@foobar -a -d 999'' may not
              report a failure to locate open files at ``TCP@foobar''
              and may not list any, if none have a file descriptor
              number of 999.  A similar situation arises when
              HASSECURITY and HASNOSOCKSECURITY are defined at compile
              time and they prevent the listing of open files.

       +|-w   Enables (+) or disables (-) the suppression of warning
              messages.

              The lsof builder may choose to have warning messages
              disabled or enabled by default.  The default warning
              message state is indicated in the output of the -h or -?
              option.  Disabling warning messages when they are already
              disabled or enabling them when already enabled is
              acceptable.

              The -t option selects the -w option.

       -x [fl]
              may accompany the +d and +D options to direct their
              processing to cross over symbolic links and|or file system
              mount points encountered when scanning the directory (+d)
              or directory tree (+D).

              If -x is specified by itself without a following
              parameter, cross-over processing of both symbolic links
              and file system mount points is enabled.  Note that when
              -x is specified without a parameter, the next argument
              must begin with '-' or '+'.

              The optional 'f' parameter enables file system mount point
              cross-over processing; 'l', symbolic link cross-over
              processing.

              The -x option may not be supplied without also supplying a
              +d or +D option.

       -X     This is a dialect-specific option.

           AIX:
                This IBM AIX RISC/System 6000 option requests the
                reporting of executed text file and shared library
                references.

                WARNING: because this option uses the kernel readx()
                function, its use on a busy AIX system might cause an
                application process to hang so completely that it can
                neither be killed nor stopped.  I have never seen this
                happen or had a report of its happening, but I think
                there is a remote possibility it could happen.

                By default use of readx() is disabled.  On AIX 5L and
                above lsof may need setuid-root permission to perform
                the actions this option requests.

                The lsof builder may specify that the -X option be
                restricted to processes whose real UID is root.  If that
                has been done, the -X option will not appear in the -h
                or -?  help output unless the real UID of the lsof
                process is root.  The default lsof distribution allows
                any UID to specify -X, so by default it will appear in
                the help output.

                When AIX readx() use is disabled, lsof may not be able
                to report information for all text and loader file
                references, but it may also avoid exacerbating an AIX
                kernel directory search kernel error, known as the Stale
                Segment ID bug.

                The readx() function, used by lsof or any other program
                to access some sections of kernel virtual memory, can
                trigger the Stale Segment ID bug.  It can cause the
                kernel's dir_search() function to believe erroneously
                that part of an in-memory copy of a file system
                directory has been zeroed.  Another application process,
                distinct from lsof, asking the kernel to search the
                directory - e.g., by using open(2) - can cause
                dir_search() to loop forever, thus hanging the
                application process.

                Consult the lsof FAQ (The FAQ section gives its
                location.)  and the 00README file of the lsof
                distribution for a more complete description of the
                Stale Segment ID bug, its APAR, and methods for defining
                readx() use when compiling lsof.

           Linux:
                This Linux option requests that lsof skip the reporting
                of information on all open TCP, UDP and UDPLITE IPv4 and
                IPv6 files.

                This Linux option is most useful when the system has an
                extremely large number of open TCP, UDP and UDPLITE
                files, the processing of whose information in the
                /proc/net/tcp* and /proc/net/udp* files would take lsof
                a long time, and whose reporting is not of interest.

                Use this option with care and only when you are sure
                that the information you want lsof to display isn't
                associated with open TCP, UDP or UDPLITE socket files.

           Solaris 10 and above:
                This Solaris 10 and above option requests the reporting
                of cached paths for files that have been deleted - i.e.,
                removed with rm(1) or unlink(2).

                The cached path is followed by the string `` (deleted)''
                to indicate that the path by which the file was opened
                has been deleted.

                Because intervening changes made to the path - i.e.,
                renames with mv(1) or rename(2) - are not recorded in
                the cached path, what lsof reports is only the path by
                which the file was opened, not its possibly different
                final path.

       -z [z]   specifies how Solaris 10 and higher zone information is
                to be handled.

                Without a following argument - e.g., NO z - the option
                specifies that zone names are to be listed in the ZONE
                output column.

                The -z option may be followed by a zone name, z.  That
                causes lsof to list only open files for processes in
                that zone.  Multiple -z z option and argument pairs may
                be specified to form a list of named zones.  Any open
                file of any process in any of the zones will be listed,
                subject to other conditions specified by other options
                and arguments.

       -Z [Z]   specifies how SELinux security contexts are to be
                handled.  It and 'Z' field output character support are
                inhibited when SELinux is disabled in the running Linux
                kernel.  See OUTPUT FOR OTHER PROGRAMS for more
                information on the 'Z' field output character.

                Without a following argument - e.g., NO Z - the option
                specifies that security contexts are to be listed in the
                SECURITY-CONTEXT output column.

                The -Z option may be followed by a wildcard security
                context name, Z.  That causes lsof to list only open
                files for processes in that security context.  Multiple
                -Z Z option and argument pairs may be specified to form
                a list of security contexts.  Any open file of any
                process in any of the security contexts will be listed,
                subject to other conditions specified by other options
                and arguments.  Note that Z can be A:B:C or *:B:C or
                A:B:* or *:*:C to match against the A:B:C context.

       --       The double minus sign option is a marker that signals
                the end of the keyed options.  It may be used, for
                example, when the first file name begins with a minus
                sign.  It may also be used when the absence of a value
                for the last keyed option must be signified by the
                presence of a minus sign in the following option and
                before the start of the file names.

       names    These are path names of specific files to list.
                Symbolic links are resolved before use.  The first name
                may be separated from the preceding options with the
                ``--'' option.

                If a name is the mounted-on directory of a file system
                or the device of the file system, lsof will list all the
                files open on the file system.  To be considered a file
                system, the name must match a mounted-on directory name
                in mount(8) output, or match the name of a block device
                associated with a mounted-on directory name.  The +|-f
                option may be used to force lsof to consider a name a
                file system identifier (+f) or a simple file (-f).

                If name is a path to a directory that is not the
                mounted-on directory name of a file system, it is
                treated just as a regular file is treated - i.e., its
                listing is restricted to processes that have it open as
                a file or as a process-specific directory, such as the
                root or current working directory.  To request that lsof
                look for open files inside a directory name, use the +d
                s and +D D options.

                If a name is the base name of a family of multiplexed
                files - e.g, AIX's /dev/pt[cs] - lsof will list all the
                associated multiplexed files on the device that are open
                - e.g., /dev/pt[cs]/1, /dev/pt[cs]/2, etc.

                If a name is a UNIX domain socket name, lsof will
                usually search for it by the characters of the name
                alone - exactly as it is specified and is recorded in
                the kernel socket structure.  (See the next paragraph
                for an exception to that rule for Linux.)  Specifying a
                relative path - e.g., ./file - in place of the file's
                absolute path - e.g., /tmp/file - won't work because
                lsof must match the characters you specify with what it
                finds in the kernel UNIX domain socket structures.

                If a name is a Linux UNIX domain socket name, in one
                case lsof is able to search for it by its device and
                inode number, allowing name to be a relative path.  The
                case requires that the absolute path -- i.e., one
                beginning with a slash ('/') be used by the process that
                created the socket, and hence be stored in the
                /proc/net/unix file; and it requires that lsof be able
                to obtain the device and node numbers of both the
                absolute path in /proc/net/unix and name via successful
                stat(2) system calls.  When those conditions are met,
                lsof will be able to search for the UNIX domain socket
                when some path to it is is specified in name.  Thus, for
                example, if the path is /dev/log, and an lsof search is
                initiated when the working directory is /dev, then name
                could be ./log.

                If a name is none of the above, lsof will list any open
                files whose device and inode match that of the specified
                path name.

                If you have also specified the -b option, the only names
                you may safely specify are file systems for which your
                mount table supplies alternate device numbers.  See the
                AVOIDING KERNEL BLOCKS and ALTERNATE DEVICE NUMBERS
                sections for more information.

                Multiple file names are joined in a single ORed set
                before participating in AND option selection.
AFS
       Lsof supports the recognition of AFS files for these dialects
       (and AFS versions):

            AIX 4.1.4 (AFS 3.4a)
            HP-UX 9.0.5 (AFS 3.4a)
            Linux 1.2.13 (AFS 3.3)
            Solaris 2.[56] (AFS 3.4a)

       It may recognize AFS files on other versions of these dialects,
       but has not been tested there.  Depending on how AFS is
       implemented, lsof may recognize AFS files in other dialects, or
       may have difficulties recognizing AFS files in the supported
       dialects.

       Lsof may have trouble identifying all aspects of AFS files in
       supported dialects when AFS kernel support is implemented via
       dynamic modules whose addresses do not appear in the kernel's
       variable name list.  In that case, lsof may have to guess at the
       identity of AFS files, and might not be able to obtain volume
       information from the kernel that is needed for calculating AFS
       volume node numbers.  When lsof can't compute volume node
       numbers, it reports blank in the NODE column.

       The -A A option is available in some dialect implementations of
       lsof for specifying the name list file where dynamic module
       kernel addresses may be found.  When this option is available, it
       will be listed in the lsof help output, presented in response to
       the -h or -?

       See the lsof FAQ (The FAQ section gives its location.)  for more
       information about dynamic modules, their symbols, and how they
       affect lsof options.

       Because AFS path lookups don't seem to participate in the
       kernel's name cache operations, lsof can't identify path name
       components for AFS files.
SECURITY
       Lsof has three features that may cause security concerns.  First,
       its default compilation mode allows anyone to list all open files
       with it.  Second, by default it creates a user-readable and
       user-writable device cache file in the home directory of the real
       user ID that executes lsof.  (The list-all-open-files and device
       cache features may be disabled when lsof is compiled.)  Third,
       its -k and -m options name alternate kernel name list or memory
       files.

       Restricting the listing of all open files is controlled by the
       compile-time HASSECURITY and HASNOSOCKSECURITY options.  When
       HASSECURITY is defined, lsof will allow only the root user to
       list all open files.  The non-root user may list only open files
       of processes with the same user IDentification number as the real
       user ID number of the lsof process (the one that its user logged
       on with).

       However, if HASSECURITY and HASNOSOCKSECURITY are both defined,
       anyone may list open socket files, provided they are selected
       with the -i option.

       When HASSECURITY is not defined, anyone may list all open files.

       Help output, presented in response to the -h or -?  option, gives
       the status of the HASSECURITY and HASNOSOCKSECURITY definitions.

       See the Security section of the 00README file of the lsof
       distribution for information on building lsof with the
       HASSECURITY and HASNOSOCKSECURITY options enabled.

       Creation and use of a user-readable and user-writable device
       cache file is controlled by the compile-time HASDCACHE option.
       See the DEVICE CACHE FILE section and the sections that follow it
       for details on how its path is formed.  For security
       considerations it is important to note that in the default lsof
       distribution, if the real user ID under which lsof is executed is
       root, the device cache file will be written in root's home
       directory - e.g., / or /root.  When HASDCACHE is not defined,
       lsof does not write or attempt to read a device cache file.

       When HASDCACHE is defined, the lsof help output, presented in
       response to the -h, -D?, or -?  options, will provide device
       cache file handling information.  When HASDCACHE is not defined,
       the -h or -?  output will have no -D option description.

       Before you decide to disable the device cache file feature -
       enabling it improves the performance of lsof by reducing the
       startup overhead of examining all the nodes in /dev (or /devices)
       - read the discussion of it in the 00DCACHE file of the lsof
       distribution and the lsof FAQ (The FAQ section gives its
       location.)

       WHEN IN DOUBT, YOU CAN TEMPORARILY DISABLE THE USE OF THE DEVICE
       CACHE FILE WITH THE -Di OPTION.

       When lsof user declares alternate kernel name list or memory
       files with the -k and -m options, lsof checks the user's
       authority to read them with access(2).  This is intended to
       prevent whatever special power lsof's modes might confer on it
       from letting it read files not normally accessible via the
       authority of the real user ID.
OUTPUT
       This section describes the information lsof lists for each open
       file.  See the OUTPUT FOR OTHER PROGRAMS section for additional
       information on output that can be processed by another program.

       Lsof only outputs printable (declared so by isprint(3)) 8 bit
       characters.  Non-printable characters are printed in one of three
       forms: the C ``\[bfrnt]'' form; the control character `^' form
       (e.g., ``^@''); or hexadecimal leading ``\x'' form (e.g.,
       ``\xab'').  Space is non-printable in the COMMAND column
       (``\x20'') and printable elsewhere.

       For some dialects - if HASSETLOCALE is defined in the dialect's
       machine.h header file - lsof will print the extended 8 bit
       characters of a language locale.  The lsof process must be
       supplied a language locale environment variable (e.g., LANG)
       whose value represents a known language locale in which the
       extended characters are considered printable by isprint(3).
       Otherwise lsof considers the extended characters non-printable
       and prints them according to its rules for non-printable
       characters, stated above.  Consult your dialect's setlocale(3)
       man page for the names of other environment variables that may be
       used in place of LANG - e.g., LC_ALL, LC_CTYPE, etc.

       Lsof's language locale support for a dialect also covers wide
       characters - e.g., UTF-8 - when HASSETLOCALE and HASWIDECHAR are
       defined in the dialect's machine.h header file, and when a
       suitable language locale has been defined in the appropriate
       environment variable for the lsof process.  Wide characters are
       printable under those conditions if iswprint(3) reports them to
       be.  If HASSETLOCALE, HASWIDECHAR and a suitable language locale
       aren't defined, or if iswprint(3) reports wide characters that
       aren't printable, lsof considers the wide characters
       non-printable and prints each of their 8 bits according to its
       rules for non-printable characters, stated above.

       Consult the answers to the "Language locale support" questions in
       the lsof FAQ (The FAQ section gives its location.) for more
       information.

       Lsof dynamically sizes the output columns each time it runs,
       guaranteeing that each column is a minimum size.  It also
       guarantees that each column is separated from its predecessor by
       at least one space.

       COMMAND
              contains the first nine characters of the name of the UNIX
              command associated with the process.  If a non-zero w
              value is specified to the +c w option, the column contains
              the first w characters of the name of the UNIX command
              associated with the process up to the limit of characters
              supplied to lsof by the UNIX dialect.  (See the
              description of the +c w command or the lsof FAQ for more
              information.  The FAQ section gives its location.)

              If w is less than the length of the column title,
              ``COMMAND'', it will be raised to that length.

              If a zero w value is specified to the +c w option, the
              column contains all the characters of the name of the UNIX
              command associated with the process.

              All command name characters maintained by the kernel in
              its structures are displayed in field output when the
              command name descriptor (`c') is specified.  See the
              OUTPUT FOR OTHER COMMANDS section for information on
              selecting field output and the associated command name
              descriptor.

       PID    is the Process IDentification number of the process.

       TID    is the task (thread) IDentification number, if task
              (thread) reporting is supported by the dialect and a task
              (thread) is being listed.  (If help output - i.e., the
              output of the -h or -?  options - shows this option, then
              task (thread) reporting is supported by the dialect.)

              A blank TID column in Linux indicates a process - i.e., a
              non-task.

       TASKCMD
              is the task command name.  Generally this will be the same
              as the process named in the COMMAND column, but some task
              implementations (e.g., Linux) permit a task to change its
              command name.

              The TASKCMD column width is subject to the same size
              limitation as the COMMAND column.

       ZONE   is the Solaris 10 and higher zone name.  This column must
              be selected with the -z option.

       SECURITY-CONTEXT
              is the SELinux security context.  This column must be
              selected with the -Z option.  Note that the -Z option is
              inhibited when SELinux is disabled in the running Linux
              kernel.

       PPID   is the Parent Process IDentification number of the
              process.  It is only displayed when the -R option has been
              specified.

       PGID   is the process group IDentification number associated with
              the process.  It is only displayed when the -g option has
              been specified.

       USER   is the user ID number or login name of the user to whom
              the process belongs, usually the same as reported by
              ps(1).  However, on Linux USER is the user ID number or
              login that owns the directory in /proc where lsof finds
              information about the process.  Usually that is the same
              value reported by ps(1), but may differ when the process
              has changed its effective user ID.  (See the -l option
              description for information on when a user ID number or
              login name is displayed.)

       FD     is the File Descriptor number of the file or:

                   cwd  current working directory;
                   Lnn  library references (AIX);
                   err  FD information error (see NAME column);
                   jld  jail directory (FreeBSD);
                   ltx  shared library text (code and data);
                   Mxx  hex memory-mapped type number xx.
                   m86  DOS Merge mapped file;
                   mem  memory-mapped file;
                   mmap memory-mapped device;
                   pd   parent directory;
                   rtd  root directory;
                   tr   kernel trace file (OpenBSD);
                   txt  program text (code and data);
                   v86  VP/ix mapped file;

              FD is followed by one of these characters, describing the
              mode under which the file is open:

                   r for read access;
                   w for write access;
                   u for read and write access;
                   space if mode unknown and no lock
                        character follows;
                   `-' if mode unknown and lock
                        character follows.

              The mode character is followed by one of these lock
              characters, describing the type of lock applied to the
              file:

                   N for a Solaris NFS lock of unknown type;
                   r for read lock on part of the file;
                   R for a read lock on the entire file;
                   w for a write lock on part of the file;
                   W for a write lock on the entire file;
                   u for a read and write lock of any length;
                   U for a lock of unknown type;
                   x for an SCO OpenServer Xenix lock on part      of
              the file;
                   X for an SCO OpenServer Xenix lock on the entire
              file;
                   space if there is no lock.

              See the LOCKS section for more information on the lock
              information character.

              The FD column contents constitutes a single field for
              parsing in post-processing scripts.

       TYPE   is the type of the node associated with the file - e.g.,
              GDIR, GREG, VDIR, VREG, etc.

              or ``IPv4'' for an IPv4 socket;

              or ``IPv6'' for an open IPv6 network file - even if its
              address is IPv4, mapped in an IPv6 address;

              or ``ax25'' for a Linux AX.25 socket;

              or ``inet'' for an Internet domain socket;

              or ``lla'' for a HP-UX link level access file;

              or ``rte'' for an AF_ROUTE socket;

              or ``sock'' for a socket of unknown domain;

              or ``unix'' for a UNIX domain socket;

              or ``x.25'' for an HP-UX x.25 socket;

              or ``BLK'' for a block special file;

              or ``CHR'' for a character special file;

              or ``DEL'' for a Linux map file that has been deleted;

              or ``DIR'' for a directory;

              or ``DOOR'' for a VDOOR file;

              or ``FIFO'' for a FIFO special file;

              or ``KQUEUE'' for a BSD style kernel event queue file;

              or ``LINK'' for a symbolic link file;

              or ``MPB'' for a multiplexed block file;

              or ``MPC'' for a multiplexed character file;

              or ``NOFD'' for a Linux /proc/<PID>/fd directory that
              can't be opened -- the directory path appears in the NAME
              column, followed by an error message;

              or ``PAS'' for a /proc/as file;

              or ``PAXV'' for a /proc/auxv file;

              or ``PCRE'' for a /proc/cred file;

              or ``PCTL'' for a /proc control file;

              or ``PCUR'' for the current /proc process;

              or ``PCWD'' for a /proc current working directory;

              or ``PDIR'' for a /proc directory;

              or ``PETY'' for a /proc executable type (etype);

              or ``PFD'' for a /proc file descriptor;

              or ``PFDR'' for a /proc file descriptor directory;

              or ``PFIL'' for an executable /proc file;

              or ``PFPR'' for a /proc FP register set;

              or ``PGD'' for a /proc/pagedata file;

              or ``PGID'' for a /proc group notifier file;

              or ``PIPE'' for pipes;

              or ``PLC'' for a /proc/lwpctl file;

              or ``PLDR'' for a /proc/lpw directory;

              or ``PLDT'' for a /proc/ldt file;

              or ``PLPI'' for a /proc/lpsinfo file;

              or ``PLST'' for a /proc/lstatus file;

              or ``PLU'' for a /proc/lusage file;

              or ``PLWG'' for a /proc/gwindows file;

              or ``PLWI'' for a /proc/lwpsinfo file;

              or ``PLWS'' for a /proc/lwpstatus file;

              or ``PLWU'' for a /proc/lwpusage file;

              or ``PLWX'' for a /proc/xregs file;

              or ``PMAP'' for a /proc map file (map);

              or ``PMEM'' for a /proc memory image file;

              or ``PNTF'' for a /proc process notifier file;

              or ``POBJ'' for a /proc/object file;

              or ``PODR'' for a /proc/object directory;

              or ``POLP'' for an old format /proc light weight process
              file;

              or ``POPF'' for an old format /proc PID file;

              or ``POPG'' for an old format /proc page data file;

              or ``PORT'' for a SYSV named pipe;

              or ``PREG'' for a /proc register file;

              or ``PRMP'' for a /proc/rmap file;

              or ``PRTD'' for a /proc root directory;

              or ``PSGA'' for a /proc/sigact file;

              or ``PSIN'' for a /proc/psinfo file;

              or ``PSTA'' for a /proc status file;

              or ``PSXSEM'' for a POSIX semaphore file;

              or ``PSXSHM'' for a POSIX shared memory file;

              or ``PTS'' for a /dev/pts file;

              or ``PUSG'' for a /proc/usage file;

              or ``PW'' for a /proc/watch file;

              or ``PXMP'' for a /proc/xmap file;

              or ``REG'' for a regular file;

              or ``SMT'' for a shared memory transport file;

              or ``STSO'' for a stream socket;

              or ``UNNM'' for an unnamed type file;

              or ``XNAM'' for an OpenServer Xenix special file of
              unknown type;

              or ``XSEM'' for an OpenServer Xenix semaphore file;

              or ``XSD'' for an OpenServer Xenix shared data file;

              or the four type number octets if the corresponding name
              isn't known.

       FILE-ADDR
              contains the kernel file structure address when f has been
              specified to +f;

       FCT    contains the file reference count from the kernel file
              structure when c has been specified to +f;

       FILE-FLAG
              when g or G has been specified to +f, this field contains
              the contents of the f_flag[s] member of the kernel file
              structure and the kernel's per-process open file flags (if
              available); `G' causes them to be displayed in
              hexadecimal; `g', as short-hand names; two lists may be
              displayed with entries separated by commas, the lists
              separated by a semicolon (`;'); the first list may contain
              short-hand names for f_flag[s] values from the following
              table:

                   AIO       asynchronous I/O (e.g., FAIO)
                   AP        append
                   ASYN      asynchronous I/O (e.g., FASYNC)
                   BAS       block, test, and set in use
                   BKIU      block if in use
                   BL        use block offsets
                   BSK       block seek
                   CA        copy avoid
                   CIO       concurrent I/O
                   CLON      clone
                   CLRD      CL read
                   CR        create
                   DF        defer
                   DFI       defer IND
                   DFLU      data flush
                   DIR       direct
                   DLY       delay
                   DOCL      do clone
                   DSYN      data-only integrity
                   DTY       must be a directory
                   EVO       event only
                   EX        open for exec
                   EXCL      exclusive open
                   FSYN      synchronous writes
                   GCDF      defer during unp_gc() (AIX)
                   GCMK      mark during unp_gc() (AIX)
                   GTTY      accessed via /dev/tty
                   HUP       HUP in progress
                   KERN      kernel
                   KIOC      kernel-issued ioctl
                   LCK       has lock
                   LG        large file
                   MBLK      stream message block
                   MK        mark
                   MNT       mount
                   MSYN      multiplex synchronization
                   NATM      don't update atime
                   NB        non-blocking I/O
                   NBDR      no BDRM check
                   NBIO      SYSV non-blocking I/O
                   NBF       n-buffering in effect
                   NC        no cache
                   ND        no delay
                   NDSY      no data synchronization
                   NET       network
                   NFLK      don't follow links
                   NMFS      NM file system
                   NOTO      disable background stop
                   NSH       no share
                   NTTY      no controlling TTY
                   OLRM      OLR mirror
                   PAIO      POSIX asynchronous I/O
                   PP        POSIX pipe
                   R         read
                   RC        file and record locking cache
                   REV       revoked
                   RSH       shared read
                   RSYN      read synchronization
                   RW        read and write access
                   SL        shared lock
                   SNAP      cooked snapshot
                   SOCK      socket
                   SQSH      Sequent shared set on open
                   SQSV      Sequent SVM set on open
                   SQR       Sequent set repair on open
                   SQS1      Sequent full shared open
                   SQS2      Sequent partial shared open
                   STPI      stop I/O
                   SWR       synchronous read
                   SYN       file integrity while writing
                   TCPM      avoid TCP collision
                   TR        truncate
                   W         write
                   WKUP      parallel I/O synchronization
                   WTG       parallel I/O synchronization
                   VH        vhangup pending
                   VTXT      virtual text
                   XL        exclusive lock

              this list of names was derived from F* #define's in
              dialect header files <fcntl.h>, <linux</fs.h>,
              <sys/fcntl.c>, <sys/fcntlcom.h>, and <sys/file.h>; see the
              lsof.h header file for a list showing the correspondence
              between the above short-hand names and the header file
              definitions;

              the second list (after the semicolon) may contain
              short-hand names for kernel per-process open file flags
              from this table:

                   ALLC      allocated
                   BR        the file has been read
                   BHUP      activity stopped by SIGHUP
                   BW        the file has been written
                   CLSG      closing
                   CX        close-on-exec (see fcntl(F_SETFD))
                   LCK       lock was applied
                   MP        memory-mapped
                   OPIP      open pending - in progress
                   RSVW      reserved wait
                   SHMT      UF_FSHMAT set (AIX)
                   USE       in use (multi-threaded)

       NODE-ID
              (or INODE-ADDR for some dialects) contains a unique
              identifier for the file node (usually the kernel vnode or
              inode address, but also occasionally a concatenation of
              device and node number) when n has been specified to +f;

       DEVICE contains the device numbers, separated by commas, for a
              character special, block special, regular, directory or
              NFS file;

              or ``memory'' for a memory file system node under Tru64
              UNIX;

              or the address of the private data area of a Solaris
              socket stream;

              or a kernel reference address that identifies the file
              (The kernel reference address may be used for FIFO's, for
              example.);

              or the base address or device name of a Linux AX.25 socket
              device.

              Usually only the lower thirty two bits of Tru64 UNIX
              kernel addresses are displayed.

       SIZE, SIZE/OFF, or OFFSET
              is the size of the file or the file offset in bytes.  A
              value is displayed in this column only if it is available.
              Lsof displays whatever value - size or offset - is
              appropriate for the type of the file and the version of
              lsof.

              On some UNIX dialects lsof can't obtain accurate or
              consistent file offset information from its kernel data
              sources, sometimes just for particular kinds of files
              (e.g., socket files.)  In other cases, files don't have
              true sizes - e.g., sockets, FIFOs, pipes - so lsof
              displays for their sizes the content amounts it finds in
              their kernel buffer descriptors (e.g., socket buffer size
              counts or TCP/IP window sizes.)  Consult the lsof FAQ (The
              FAQ section gives its location.)  for more information.

              The file size is displayed in decimal; the offset is
              normally displayed in decimal with a leading ``0t'' if it
              contains 8 digits or less; in hexadecimal with a leading
              ``0x'' if it is longer than 8 digits.  (Consult the -o o
              option description for information on when 8 might default
              to some other value.)

              Thus the leading ``0t'' and ``0x'' identify an offset when
              the column may contain both a size and an offset (i.e.,
              its title is SIZE/OFF).

              If the -o option is specified, lsof always displays the
              file offset (or nothing if no offset is available) and
              labels the column OFFSET.  The offset always begins with
              ``0t'' or ``0x'' as described above.

              The lsof user can control the switch from ``0t'' to ``0x''
              with the -o o option.  Consult its description for more
              information.

              If the -s option is specified, lsof always displays the
              file size (or nothing if no size is available) and labels
              the column SIZE.  The -o and -s options are mutually
              exclusive; they can't both be specified.

              For files that don't have a fixed size - e.g., don't
              reside on a disk device - lsof will display appropriate
              information about the current size or position of the file
              if it is available in the kernel structures that define
              the file.

       NLINK  contains the file link count when +L has been specified;

       NODE   is the node number of a local file;

              or the inode number of an NFS file in the server host;

              or the Internet protocol type - e.g, ``TCP'';

              or ``STR'' for a stream;

              or ``CCITT'' for an HP-UX x.25 socket;

              or the IRQ or inode number of a Linux AX.25 socket device.

       NAME   is the name of the mount point and file system on which
              the file resides;

              or the name of a file specified in the names option (after
              any symbolic links have been resolved);

              or the name of a character special or block special
              device;

              or the local and remote Internet addresses of a network
              file; the local host name or IP number is followed by a
              colon (':'), the port, ``->'', and the two-part remote
              address; IP addresses may be reported as numbers or names,
              depending on the +|-M, -n, and -P options; colon-separated
              IPv6 numbers are enclosed in square brackets; IPv4
              INADDR_ANY and IPv6 IN6_IS_ADDR_UNSPECIFIED addresses, and
              zero port numbers are represented by an asterisk ('*'); a
              UDP destination address may be followed by the amount of
              time elapsed since the last packet was sent to the
              destination; TCP, UDP and UDPLITE remote addresses may be
              followed by TCP/TPI information in parentheses - state
              (e.g., ``(ESTABLISHED)'', ``(Unbound)''), queue sizes, and
              window sizes (not all dialects) - in a fashion similar to
              what netstat(1) reports; see the -T option description or
              the description of the TCP/TPI field in OUTPUT FOR OTHER
              PROGRAMS for more information on state, queue size, and
              window size;

              or the address or name of a UNIX domain socket, possibly
              including a stream clone device name, a file system
              object's path name, local and foreign kernel addresses,
              socket pair information, and a bound vnode address;

              or the local and remote mount point names of an NFS file;

              or ``STR'', followed by the stream name;

              or a stream character device name, followed by ``->'' and
              the stream name or a list of stream module names,
              separated by ``->'';

              or ``STR:'' followed by the SCO OpenServer stream device
              and module names, separated by ``->'';

              or system directory name, `` -- '', and as many components
              of the path name as lsof can find in the kernel's name
              cache for selected dialects (See the KERNEL NAME CACHE
              section for more information.);

              or ``PIPE->'', followed by a Solaris kernel pipe
              destination address;

              or ``COMMON:'', followed by the vnode device information
              structure's device name, for a Solaris common vnode;

              or the address family, followed by a slash (`/'), followed
              by fourteen comma-separated bytes of a non-Internet raw
              socket address;

              or the HP-UX x.25 local address, followed by the virtual
              connection number (if any), followed by the remote address
              (if any);

              or ``(dead)'' for disassociated Tru64 UNIX files -
              typically terminal files that have been flagged with the
              TIOCNOTTY ioctl and closed by daemons;

              or ``rd=<offset>'' and ``wr=<offset>'' for the values of
              the read and write offsets of a FIFO;

              or ``clone n:/dev/event'' for SCO OpenServer file clones
              of the /dev/event device, where n is the minor device
              number of the file;

              or ``(socketpair: n)'' for a Solaris 2.6, 8, 9  or 10 UNIX
              domain socket, created by the socketpair(3N) network
              function;

              or ``no PCB'' for socket files that do not have a protocol
              block associated with them, optionally followed by ``,
              CANTSENDMORE'' if sending on the socket has been disabled,
              or ``, CANTRCVMORE'' if receiving on the socket has been
              disabled (e.g., by the shutdown(2) function);

              or the local and remote addresses of a Linux IPX socket
              file in the form <net>:[<node>:]<port>, followed in
              parentheses by the transmit and receive queue sizes, and
              the connection state;

              or ``dgram'' or ``stream'' for the type UnixWare 7.1.1 and
              above in-kernel UNIX domain sockets, followed by a colon
              (':') and the local path name when available, followed by
              ``->'' and the remote path name or kernel socket address
              in hexadecimal when available;

              or the association value, association index, endpoint
              value, local address, local port, remote address and
              remote port for Linux SCTP sockets;

              or ``protocol: '' followed by the Linux socket's protocol
              attribute.

       For dialects that support a ``namefs'' file system, allowing one
       file to be attached to another with fattach(3C), lsof will add
       ``(FA:<address1><direction><address2>)'' to the NAME column.
       <address1> and <address2> are hexadecimal vnode addresses.
       <direction> will be ``<-'' if <address2> has been fattach'ed to
       this vnode whose address is <address1>; and ``->'' if <address1>,
       the vnode address of this vnode, has been fattach'ed to
       <address2>.  <address1> may be omitted if it already appears in
       the DEVICE column.

       Lsof may add two parenthetical notes to the NAME column for open
       Solaris 10 files: ``(?)'' if lsof considers the path name of
       questionable accuracy; and ``(deleted)'' if the -X option has
       been specified and lsof detects the open file's path name has
       been deleted.  Consult the lsof FAQ (The FAQ section gives its
       location.)  for more information on these NAME column additions.
LOCKS
       Lsof can't adequately report the wide variety of UNIX dialect
       file locks in a single character.  What it reports in a single
       character is a compromise between the information it finds in the
       kernel and the limitations of the reporting format.

       Moreover, when a process holds several byte level locks on a
       file, lsof only reports the status of the first lock it
       encounters.  If it is a byte level lock, then the lock character
       will be reported in lower case - i.e., `r', `w', or `x' - rather
       than the upper case equivalent reported for a full file lock.

       Generally lsof can only report on locks held by local processes
       on local files.  When a local process sets a lock on a remotely
       mounted (e.g., NFS) file, the remote server host usually records
       the lock state.  One exception is Solaris - at some patch levels
       of 2.3, and in all versions above 2.4, the Solaris kernel records
       information on remote locks in local structures.

       Lsof has trouble reporting locks for some UNIX dialects.  Consult
       the BUGS section of this manual page or the lsof FAQ (The FAQ
       section gives its location.)  for more information.
OUTPUT FOR OTHER PROGRAMS
       When the -F option is specified, lsof produces output that is
       suitable for processing by another program - e.g, an awk or Perl
       script, or a C program.

       Each unit of information is output in a field that is identified
       with a leading character and terminated by a NL (012) (or a NUL
       (000) if the 0 (zero) field identifier character is specified.)
       The data of the field follows immediately after the field
       identification character and extends to the field terminator.

       It is possible to think of field output as process and file sets.
       A process set begins with a field whose identifier is `p' (for
       process IDentifier (PID)).  It extends to the beginning of the
       next PID field or the beginning of the first file set of the
       process, whichever comes first.  Included in the process set are
       fields that identify the command, the process group
       IDentification (PGID) number, the task (thread) ID (TID), and the
       user ID (UID) number or login name.

       A file set begins with a field whose identifier is `f' (for file
       descriptor).  It is followed by lines that describe the file's
       access mode, lock state, type, device, size, offset, inode,
       protocol, name and stream module names.  It extends to the
       beginning of the next file or process set, whichever comes first.

       When the NUL (000) field terminator has been selected with the 0
       (zero) field identifier character, lsof ends each process and
       file set with a NL (012) character.

       Lsof always produces one field, the PID (`p') field.  All other
       fields may be declared optionally in the field identifier
       character list that follows the -F option.  When a field
       selection character identifies an item lsof does not normally
       list - e.g., PPID, selected with -R - specification of the field
       character - e.g., ``-FR'' - also selects the listing of the item.

       It is entirely possible to select a set of fields that cannot
       easily be parsed - e.g., if the field descriptor field is not
       selected, it may be difficult to identify file sets.  To help you
       avoid this difficulty, lsof supports the -F option; it selects
       the output of all fields with NL terminators (the -F0 option pair
       selects the output of all fields with NUL terminators).  For
       compatibility reasons neither -F nor -F0 select the raw device
       field.

       These are the fields that lsof will produce.  The single
       character listed first is the field identifier.

            a    file access mode
            c    process command name (all characters from proc or
                 user structure)
            C    file structure share count
            d    file's device character code
            D    file's major/minor device number (0x<hexadecimal>)
            f    file descriptor (always selected)
            F    file structure address (0x<hexadecimal>)
            G    file flaGs (0x<hexadecimal>; names if +fg follows)
            g    process group ID
            i    file's inode number
            K    tasK ID
            k    link count
            l    file's lock status
            L    process login name
            m    marker between repeated output
            M    the task comMand name
            n    file name, comment, Internet address
            N    node identifier (ox<hexadecimal>
            o    file's offset (decimal)
            p    process ID (always selected)
            P    protocol name
            r    raw device number (0x<hexadecimal>)
            R    parent process ID
            s    file's size (decimal)
            S    file's stream identification
            t    file's type
            T    TCP/TPI information, identified by prefixes (the
                 `=' is part of the prefix):
                     QR=<read queue size>
                     QS=<send queue size>
                     SO=<socket options and values> (not all dialects)
                     SS=<socket states> (not all dialects)
                     ST=<connection state>
                     TF=<TCP flags and values> (not all dialects)
                     WR=<window read size>  (not all dialects)
                     WW=<window write size>  (not all dialects)
                 (TCP/TPI information isn't reported for all supported
                   UNIX dialects. The -h or -? help output for the
                   -T option will show what TCP/TPI reporting can be
                   requested.)
            u    process user ID
            z    Solaris 10 and higher zone name
            Z    SELinux security context (inhibited when SELinux is disabled)
            0    use NUL field terminator character in place of NL
            1-9  dialect-specific field identifiers (The output
                 of -F? identifies the information to be found
                 in dialect-specific fields.)

       You can get on-line help information on these characters and
       their descriptions by specifying the -F?  option pair.  (Escape
       the `?' character as your shell requires.)  Additional
       information on field content can be found in the OUTPUT section.

       As an example, ``-F pcfn'' will select the process ID (`p'),
       command name (`c'), file descriptor (`f') and file name (`n')
       fields with an NL field terminator character; ``-F pcfn0''
       selects the same output with a NUL (000) field terminator
       character.

       Lsof doesn't produce all fields for every process or file set,
       only those that are available.  Some fields are mutually
       exclusive: file device characters and file major/minor device
       numbers; file inode number and protocol name; file name and
       stream identification; file size and offset.  One or the other
       member of these mutually exclusive sets will appear in field
       output, but not both.

       Normally lsof ends each field with a NL (012) character.  The 0
       (zero) field identifier character may be specified to change the
       field terminator character to a NUL (000).  A NUL terminator may
       be easier to process with xargs(1), for example, or with programs
       whose quoting mechanisms may not easily cope with the range of
       characters in the field output.  When the NUL field terminator is
       in use, lsof ends each process and file set with a NL (012).

       Three aids to producing programs that can process lsof field
       output are included in the lsof distribution.  The first is a C
       header file, lsof_fields.h, that contains symbols for the field
       identification characters, indexes for storing them in a table,
       and explanation strings that may be compiled into programs.  Lsof
       uses this header file.

       The second aid is a set of sample scripts that process field
       output, written in awk, Perl 4, and Perl 5.  They're located in
       the scripts subdirectory of the lsof distribution.

       The third aid is the C library used for the lsof test suite.  The
       test suite is written in C and uses field output to validate the
       correct operation of lsof.  The library can be found in the
       tests/LTlib.c file of the lsof distribution.  The library uses
       the first aid, the lsof_fields.h header file.
BLOCKS AND TIMEOUTS
       Lsof can be blocked by some kernel functions that it uses -
       lstat(2), readlink(2), and stat(2).  These functions are stalled
       in the kernel, for example, when the hosts where mounted NFS file
       systems reside become inaccessible.

       Lsof attempts to break these blocks with timers and child
       processes, but the techniques are not wholly reliable.  When lsof
       does manage to break a block, it will report the break with an
       error message.  The messages may be suppressed with the -t and -w
       options.

       The default timeout value may be displayed with the -h or -?
       option, and it may be changed with the -S [t] option.  The
       minimum for t is two seconds, but you should avoid small values,
       since slow system responsiveness can cause short timeouts to
       expire unexpectedly and perhaps stop lsof before it can produce
       any output.

       When lsof has to break a block during its access of mounted file
       system information, it normally continues, although with less
       information available to display about open files.

       Lsof can also be directed to avoid the protection of timers and
       child processes when using the kernel functions that might block
       by specifying the -O option.  While this will allow lsof to start
       up with less overhead, it exposes lsof completely to the kernel
       situations that might block it.  Use this option cautiously.
AVOIDING KERNEL BLOCKS
       You can use the -b option to tell lsof to avoid using kernel
       functions that would block.  Some cautions apply.

       First, using this option usually requires that your system supply
       alternate device numbers in place of the device numbers that lsof
       would normally obtain with the lstat(2) and stat(2) kernel
       functions.  See the ALTERNATE DEVICE NUMBERS section for more
       information on alternate device numbers.

       Second, you can't specify names for lsof to locate unless they're
       file system names.  This is because lsof needs to know the device
       and inode numbers of files listed with names in the lsof options,
       and the -b option prevents lsof from obtaining them.  Moreover,
       since lsof only has device numbers for the file systems that have
       alternates, its ability to locate files on file systems depends
       completely on the availability and accuracy of the alternates.
       If no alternates are available, or if they're incorrect, lsof
       won't be able to locate files on the named file systems.

       Third, if the names of your file system directories that lsof
       obtains from your system's mount table are symbolic links, lsof
       won't be able to resolve the links.  This is because the -b
       option causes lsof to avoid the kernel readlink(2) function it
       uses to resolve symbolic links.

       Finally, using the -b option causes lsof to issue warning
       messages when it needs to use the kernel functions that the -b
       option directs it to avoid.  You can suppress these messages by
       specifying the -w option, but if you do, you won't see the
       alternate device numbers reported in the warning messages.
ALTERNATE DEVICE NUMBERS
       On some dialects, when lsof has to break a block because it can't
       get information about a mounted file system via the lstat(2) and
       stat(2) kernel functions, or because you specified the -b option,
       lsof can obtain some of the information it needs - the device
       number and possibly the file system type - from the system mount
       table.  When that is possible, lsof will report the device number
       it obtained.  (You can suppress the report by specifying the -w
       option.)

       You can assist this process if your mount table is supported with
       an /etc/mtab or /etc/mnttab file that contains an options field
       by adding a ``dev=xxxx'' field for mount points that do not have
       one in their options strings.  Note: you must be able to edit the
       file - i.e., some mount tables like recent Solaris /etc/mnttab or
       Linux /proc/mounts are read-only and can't be modified.

       You may also be able to supply device numbers using the +m and +m
       m options, provided they are supported by your dialect.  Check
       the output of lsof's -h or -?  options to see if the +m and +m m
       options are available.

       The ``xxxx'' portion of the field is the hexadecimal value of the
       file system's device number.  (Consult the st_dev field of the
       output of the lstat(2) and stat(2) functions for the appropriate
       values for your file systems.)  Here's an example from a Sun
       Solaris 2.6 /etc/mnttab for a file system remotely mounted via
       NFS:

            nfs  ignore,noquota,dev=2a40001

       There's an advantage to having ``dev=xxxx'' entries in your mount
       table file, especially for file systems that are mounted from
       remote NFS servers.  When a remote server crashes and you want to
       identify its users by running lsof on one of its clients, lsof
       probably won't be able to get output from the lstat(2) and
       stat(2) functions for the file system.  If it can obtain the file
       system's device number from the mount table, it will be able to
       display the files open on the crashed NFS server.

       Some dialects that do not use an ASCII /etc/mtab or /etc/mnttab
       file for the mount table may still provide an alternative device
       number in their internal mount tables.  This includes AIX, Apple
       Darwin, FreeBSD, NetBSD, OpenBSD, and Tru64 UNIX.  Lsof knows how
       to obtain the alternative device number for these dialects and
       uses it when its attempt to lstat(2) or stat(2) the file system
       is blocked.

       If you're not sure your dialect supplies alternate device numbers
       for file systems from its mount table, use this lsof incantation
       to see if it reports any alternate device numbers:

              lsof -b

       Look for standard error file warning messages that begin
       ``assuming "dev=xxxx" from ...''.
KERNEL NAME CACHE
       Lsof is able to examine the kernel's name cache or use other
       kernel facilities (e.g., the ADVFS 4.x tag_to_path() function
       under Tru64 UNIX) on some dialects for most file system types,
       excluding AFS, and extract recently used path name components
       from it.  (AFS file system path lookups don't use the kernel's
       name cache; some Solaris VxFS file system operations apparently
       don't use it, either.)

       Lsof reports the complete paths it finds in the NAME column.  If
       lsof can't report all components in a path, it reports in the
       NAME column the file system name, followed by a space, two `-'
       characters, another space, and the name components it has
       located, separated by the `/' character.

       When lsof is run in repeat mode - i.e., with the -r option
       specified - the extent to which it can report path name
       components for the same file may vary from cycle to cycle.
       That's because other running processes can cause the kernel to
       remove entries from its name cache and replace them with others.

       Lsof's use of the kernel name cache to identify the paths of
       files can lead it to report incorrect components under some
       circumstances.  This can happen when the kernel name cache uses
       device and node number as a key (e.g., SCO OpenServer) and a key
       on a rapidly changing file system is reused.  If the UNIX
       dialect's kernel doesn't purge the name cache entry for a file
       when it is unlinked, lsof may find a reference to the wrong entry
       in the cache.  The lsof FAQ (The FAQ section gives its location.)
       has more information on this situation.

       Lsof can report path name components for these dialects:

            FreeBSD
            HP-UX
            Linux
            NetBSD
            NEXTSTEP
            OpenBSD
            OPENSTEP
            SCO OpenServer
            SCO|Caldera UnixWare
            Solaris
            Tru64 UNIX

       Lsof can't report path name components for these dialects:

            AIX

       If you want to know why lsof can't report path name components
       for some dialects, see the lsof FAQ (The FAQ section gives its
       location.)
DEVICE CACHE FILE
       Examining all members of the /dev (or /devices) node tree with
       stat(2) functions can be time consuming.  What's more, the
       information that lsof needs - device number, inode number, and
       path - rarely changes.

       Consequently, lsof normally maintains an ASCII text file of
       cached /dev (or /devices) information (exception: the /proc-based
       Linux lsof where it's not needed.)  The local system
       administrator who builds lsof can control the way the device
       cache file path is formed, selecting from these options:

            Path from the -D option;
            Path from an environment variable;
            System-wide path;
            Personal path (the default);
            Personal path, modified by an environment variable.

       Consult the output of the -h, -D? , or -?  help options for the
       current state of device cache support.  The help output lists the
       default read-mode device cache file path that is in effect for
       the current invocation of lsof.  The -D?  option output lists the
       read-only and write device cache file paths, the names of any
       applicable environment variables, and the personal device cache
       path format.

       Lsof can detect that the current device cache file has been
       accidentally or maliciously modified by integrity checks,
       including the computation and verification of a sixteen bit
       Cyclic Redundancy Check (CRC) sum on the file's contents.  When
       lsof senses something wrong with the file, it issues a warning
       and attempts to remove the current cache file and create a new
       copy, but only to a path that the process can legitimately write.

       The path from which a lsof process may attempt to read a device
       cache file may not be the same as the path to which it can
       legitimately write.  Thus when lsof senses that it needs to
       update the device cache file, it may choose a different path for
       writing it from the path from which it read an incorrect or
       outdated version.

       If available, the -Dr option will inhibit the writing of a new
       device cache file.  (It's always available when specified without
       a path name argument.)

       When a new device is added to the system, the device cache file
       may need to be recreated.  Since lsof compares the mtime of the
       device cache file with the mtime and ctime of the /dev (or
       /devices) directory, it usually detects that a new device has
       been added; in that case lsof issues a warning message and
       attempts to rebuild the device cache file.

       Whenever lsof writes a device cache file, it sets its ownership
       to the real UID of the executing process, and its permission
       modes to 0600, this restricting its reading and writing to the
       file's owner.
LSOF PERMISSIONS THAT AFFECT DEVICE CACHE FILE ACCESS
       Two permissions of the lsof executable affect its ability to
       access device cache files.  The permissions are set by the local
       system administrator when lsof is installed.

       The first and rarer permission is setuid-root.  It comes into
       effect when lsof is executed; its effective UID is then root,
       while its real (i.e., that of the logged-on user) UID is not.
       The lsof distribution recommends that versions for these dialects
       run setuid-root.

            HP-UX 11.11 and 11.23
            Linux

       The second and more common permission is setgid.  It comes into
       effect when the effective group IDentification number (GID) of
       the lsof process is set to one that can access kernel memory
       devices - e.g., ``kmem'', ``sys'', or ``system''.

       An lsof process that has setgid permission usually surrenders the
       permission after it has accessed the kernel memory devices.  When
       it does that, lsof can allow more liberal device cache path
       formations.  The lsof distribution recommends that versions for
       these dialects run setgid and be allowed to surrender setgid
       permission.

            AIX 5.[12] and 5.3-ML1
            Apple Darwin 7.x Power Macintosh systems
            FreeBSD 4.x, 4.1x, 5.x and [6789].x for x86-based systems
            FreeBSD 5.x, [6789].x and 1[012].8for Alpha, AMD64 and Sparc64
                based systems
            HP-UX 11.00
            NetBSD 1.[456], 2.x and 3.x for Alpha, x86, and SPARC-based
                systems
            NEXTSTEP 3.[13] for NEXTSTEP architectures
            OpenBSD 2.[89] and 3.[0-9] for x86-based systems
            OPENSTEP 4.x
            SCO OpenServer Release 5.0.6 for x86-based systems
            SCO|Caldera UnixWare 7.1.4 for x86-based systems
            Solaris 2.6, 8, 9 and 10
            Tru64 UNIX 5.1

       (Note: lsof for AIX 5L and above needs setuid-root permission if
       its -X option is used.)

       Lsof for these dialects does not support a device cache, so the
       permissions given to the executable don't apply to the device
       cache file.

            Linux
DEVICE CACHE FILE PATH FROM THE -D OPTION
       The -D option provides limited means for specifying the device
       cache file path.  Its ?  function will report the read-only and
       write device cache file paths that lsof will use.

       When the -D b, r, and u functions are available, you can use them
       to request that the cache file be built in a specific location
       (b[path]); read but not rebuilt (r[path]); or read and rebuilt
       (u[path]).  The b, r, and u functions are restricted under some
       conditions.  They are restricted when the lsof process is
       setuid-root.  The path specified with the r function is always
       read-only, even when it is available.

       The b, r, and u functions are also restricted when the lsof
       process runs setgid and lsof doesn't surrender the setgid
       permission.  (See the LSOF PERMISSIONS THAT AFFECT DEVICE CACHE
       FILE ACCESS section for a list of implementations that normally
       don't surrender their setgid permission.)

       A further -D function, i (for ignore), is always available.

       When available, the b function tells lsof to read device
       information from the kernel with the stat(2) function and build a
       device cache file at the indicated path.

       When available, the r function tells lsof to read the device
       cache file, but not update it.  When a path argument accompanies
       -Dr, it names the device cache file path.  The r function is
       always available when it is specified without a path name
       argument.  If lsof is not running setuid-root and surrenders its
       setgid permission, a path name argument may accompany the r
       function.

       When available, the u function tells lsof to attempt to read and
       use the device cache file.  If it can't read the file, or if it
       finds the contents of the file incorrect or outdated, it will
       read information from the kernel, and attempt to write an updated
       version of the device cache file, but only to a path it considers
       legitimate for the lsof process effective and real UIDs.
DEVICE CACHE PATH FROM AN ENVIRONMENT VARIABLE
       Lsof's second choice for the device cache file is the contents of
       the LSOFDEVCACHE environment variable.  It avoids this choice if
       the lsof process is setuid-root, or the real UID of the process
       is root.

       A further restriction applies to a device cache file path taken
       from the LSOFDEVCACHE environment variable: lsof will not write a
       device cache file to the path if the lsof process doesn't
       surrender its setgid permission.  (See the LSOF PERMISSIONS THAT
       AFFECT DEVICE CACHE FILE ACCESS section for information on
       implementations that don't surrender their setgid permission.)

       The local system administrator can disable the use of the
       LSOFDEVCACHE environment variable or change its name when
       building lsof.  Consult the output of -D?  for the environment
       variable's name.
SYSTEM-WIDE DEVICE CACHE PATH
       The local system administrator may choose to have a system-wide
       device cache file when building lsof.  That file will generally
       be constructed by a special system administration procedure when
       the system is booted or when the contents of /dev or /devices)
       changes.  If defined, it is lsof's third device cache file path
       choice.

       You can tell that a system-wide device cache file is in effect
       for your local installation by examining the lsof help option
       output - i.e., the output from the -h or -?  option.

       Lsof will never write to the system-wide device cache file path
       by default.  It must be explicitly named with a -D function in a
       root-owned procedure.  Once the file has been written, the
       procedure must change its permission modes to 0644 (owner-read
       and owner-write, group-read, and other-read).
PERSONAL DEVICE CACHE PATH (DEFAULT)
       The default device cache file path of the lsof distribution is
       one recorded in the home directory of the real UID that executes
       lsof.  Added to the home directory is a second path component of
       the form .lsof_hostname.

       This is lsof's fourth device cache file path choice, and is
       usually the default.  If a system-wide device cache file path was
       defined when lsof was built, this fourth choice will be applied
       when lsof can't find the system-wide device cache file.  This is
       the only time lsof uses two paths when reading the device cache
       file.

       The hostname part of the second component is the base name of the
       executing host, as returned by gethostname(2).  The base name is
       defined to be the characters preceding the first `.'  in the
       gethostname(2) output, or all the gethostname(2) output if it
       contains no `.'.

       The device cache file belongs to the user ID and is readable and
       writable by the user ID alone - i.e., its modes are 0600.  Each
       distinct real user ID on a given host that executes lsof has a
       distinct device cache file.  The hostname part of the path
       distinguishes device cache files in an NFS-mounted home directory
       into which device cache files are written from several different
       hosts.

       The personal device cache file path formed by this method
       represents a device cache file that lsof will attempt to read,
       and will attempt to write should it not exist or should its
       contents be incorrect or outdated.

       The -Dr option without a path name argument will inhibit the
       writing of a new device cache file.

       The -D?  option will list the format specification for
       constructing the personal device cache file.  The conversions
       used in the format specification are described in the 00DCACHE
       file of the lsof distribution.
MODIFIED PERSONAL DEVICE CACHE PATH
       If this option is defined by the local system administrator when
       lsof is built, the LSOFPERSDCPATH environment variable contents
       may be used to add a component of the personal device cache file
       path.

       The LSOFPERSDCPATH variable contents are inserted in the path at
       the place marked by the local system administrator with the
       ``%p'' conversion in the HASPERSDC format specification of the
       dialect's machine.h header file.  (It's placed right after the
       home directory in the default lsof distribution.)

       Thus, for example, if LSOFPERSDCPATH contains ``LSOF'', the home
       directory is ``/Homes/abe'', the host name is
       ``lsof.itap.purdue.edu'', and the HASPERSDC format is the default
       (``%h/%p.lsof_%L''), the modified personal device cache file path
       is:

            /Homes/abe/LSOF/.lsof_vic

       The LSOFPERSDCPATH environment variable is ignored when the lsof
       process is setuid-root or when the real UID of the process is
       root.

       Lsof will not write to a modified personal device cache file path
       if the lsof process doesn't surrender setgid permission.  (See
       the LSOF PERMISSIONS THAT AFFECT DEVICE CACHE FILE ACCESS section
       for a list of implementations that normally don't surrender their
       setgid permission.)

       If, for example, you want to create a sub-directory of personal
       device cache file paths by using the LSOFPERSDCPATH environment
       variable to name it, and lsof doesn't surrender its setgid
       permission, you will have to allow lsof to create device cache
       files at the standard personal path and move them to your
       subdirectory with shell commands.

       The local system administrator may: disable this option when lsof
       is built; change the name of the environment variable from
       LSOFPERSDCPATH to something else; change the HASPERSDC format to
       include the personal path component in another place; or exclude
       the personal path component entirely.  Consult the output of the
       -D?  option for the environment variable's name and the HASPERSDC
       format specification.
DIAGNOSTICS
       Errors are identified with messages on the standard error file.

       Lsof returns a one (1) if any error was detected, including the
       failure to locate command names, file names, Internet addresses
       or files, login names, NFS files, PIDs, PGIDs, or UIDs it was
       asked to list.  If the -V option is specified, lsof will indicate
       the search items it failed to list.

       It returns a zero (0) if no errors were detected and if it was
       able to list some information about all the specified search
       arguments.

       When lsof cannot open access to /dev (or /devices) or one of its
       subdirectories, or get information on a file in them with
       stat(2), it issues a warning message and continues.  That lsof
       will issue warning messages about inaccessible files in /dev (or
       /devices) is indicated in its help output - requested with the -h
       or >B -?  options -  with the message:

            Inaccessible /dev warnings are enabled.

       The warning message may be suppressed with the -w option.  It may
       also have been suppressed by the system administrator when lsof
       was compiled by the setting of the WARNDEVACCESS definition.  In
       this case, the output from the help options will include the
       message:

            Inaccessible /dev warnings are disabled.

       Inaccessible device warning messages usually disappear after lsof
       has created a working device cache file.
EXAMPLES
       For a more extensive set of examples, documented more fully, see
       the 00QUICKSTART file of the lsof distribution.

       To list all open files, use:

              lsof

       To list all open Internet, x.25 (HP-UX), and UNIX domain files,
       use:

              lsof -i -U

       To list all open IPv4 network files in use by the process whose
       PID is 1234, use:

              lsof -i 4 -a -p 1234

       Presuming the UNIX dialect supports IPv6, to list only open IPv6
       network files, use:

              lsof -i 6

       To list all files using any protocol on ports 513, 514, or 515 of
       host wonderland.cc.purdue.edu, use:

              lsof -i @wonderland.cc.purdue.edu:513-515

       To list all files using any protocol on any port of
       mace.cc.purdue.edu (cc.purdue.edu is the default domain), use:

              lsof -i @mace

       To list all open files for login name ``abe'', or user ID 1234,
       or process 456, or process 123, or process 789, use:

              lsof -p 456,123,789 -u 1234,abe

       To list all open files on device /dev/hd4, use:

              lsof /dev/hd4

       To find the process that has /u/abe/foo open, use:

              lsof /u/abe/foo

       To send a SIGHUP to the processes that have /u/abe/bar open, use:

              kill -HUP `lsof -t /u/abe/bar`

       To find any open file, including an open UNIX domain socket file,
       with the name /dev/log, use:

              lsof /dev/log

       To find processes with open files on the NFS file system named
       /nfs/mount/point whose server is inaccessible, and presuming your
       mount table supplies the device number for /nfs/mount/point, use:

              lsof -b /nfs/mount/point

       To do the preceding search with warning messages suppressed, use:

              lsof -bw /nfs/mount/point

       To ignore the device cache file, use:

              lsof -Di

       To obtain PID and command name field output for each process,
       file descriptor, file device number, and file inode number for
       each file of each process, use:

              lsof -FpcfDi

       To list the files at descriptors 1 and 3 of every process running
       the lsof command for login ID ``abe'' every 10 seconds, use:

              lsof -c lsof -a -d 1 -d 3 -u abe -r10

       To list the current working directory of processes running a
       command that is exactly four characters long and has an 'o' or
       'O' in character three, use this regular expression form of the
       -c c option:

              lsof -c /^..o.$/i -a -d cwd

       To find an IP version 4 socket file by its associated numeric
       dot-form address, use:

              lsof -i@128.210.15.17

       To find an IP version 6 socket file (when the UNIX dialect
       supports IPv6) by its associated numeric colon-form address, use:

              lsof -i@[0:1:2:3:4:5:6:7]

       To find an IP version 6 socket file (when the UNIX dialect
       supports IPv6) by an associated numeric colon-form address that
       has a run of zeroes in it - e.g., the loop-back address - use:

              lsof -i@[::1]

       To obtain a repeat mode marker line that contains the current
       time, use:

              lsof -rm====%T====

       To add spaces to the previous marker line, use:

              lsof -r "m==== %T ===="
BUGS
       Since lsof reads kernel memory in its search for open files,
       rapid changes in kernel memory may produce unpredictable results.

       When a file has multiple record locks, the lock status character
       (following the file descriptor) is derived from a test of the
       first lock structure, not from any combination of the individual
       record locks that might be described by multiple lock structures.

       Lsof can't search for files with restrictive access permissions
       by name unless it is installed with root set-UID permission.
       Otherwise it is limited to searching for files to which its user
       or its set-GID group (if any) has access permission.

       The display of the destination address of a raw socket (e.g., for
       ping) depends on the UNIX operating system.  Some dialects store
       the destination address in the raw socket's protocol control
       block, some do not.

       Lsof can't always represent Solaris device numbers in the same
       way that ls(1) does.  For example, the major and minor device
       numbers that the lstat(2) and stat(2) functions report for the
       directory on which CD-ROM files are mounted (typically /cdrom)
       are not the same as the ones that it reports for the device on
       which CD-ROM files are mounted (typically /dev/sr0).  (Lsof
       reports the directory numbers.)

       The support for /proc file systems is available only for BSD and
       Tru64 UNIX dialects, Linux, and dialects derived from SYSV R4 -
       e.g., FreeBSD, NetBSD, OpenBSD, Solaris, UnixWare.

       Some /proc file items - device number, inode number, and file
       size - are unavailable in some dialects.  Searching for files in
       a /proc file system may require that the full path name be
       specified.

       No text (txt) file descriptors are displayed for Linux processes.
       All entries for files other than the current working directory,
       the root directory, and numerical file descriptors are labeled
       mem descriptors.

       Lsof can't search for Tru64 UNIX named pipes by name, because
       their kernel implementation of lstat(2) returns an improper
       device number for a named pipe.

       Lsof can't report fully or correctly on HP-UX 9.01, 10.20, and
       11.00 locks because of insufficient access to kernel data or
       errors in the kernel data.  See the lsof FAQ (The FAQ section
       gives its location.)  for details.

       The AIX SMT file type is a fabrication.  It's made up for file
       structures whose type (15) isn't defined in the AIX
       /usr/include/sys/file.h header file.  One way to create such file
       structures is to run X clients with the DISPLAY variable set to
       ``:0.0''.

       The +|-f[cfgGn] option is not supported under /proc-based Linux
       lsof, because it doesn't read kernel structures from kernel
       memory.
ENVIRONMENT
       Lsof may access these environment variables.

       LANG   defines a language locale.  See setlocale(3) for the names
              of other variables that can be used in place of LANG -
              e.g., LC_ALL, LC_TYPE, etc.

       LSOFDEVCACHE
              defines the path to a device cache file.  See the DEVICE
              CACHE PATH FROM AN ENVIRONMENT VARIABLE section for more
              information.

       LSOFPERSDCPATH
              defines the middle component of a modified personal device
              cache file path.  See the MODIFIED PERSONAL DEVICE CACHE
              PATH section for more information.
FAQ
       Frequently-asked questions and their answers (an FAQ) are
       available in the 00FAQ file of the lsof distribution.

       That file is also available via anonymous ftp from
       lsof.itap.purdue.edu at pub/tools/unix/lsofFAQ.  The URL is:

              ftp://lsof.itap.purdue.edu/pub/tools/unix/lsof/FAQ
FILES
       /dev/kmem
              kernel virtual memory device

       /dev/mem
              physical memory device

       /dev/swap
              system paging device

       .lsof_hostname
              lsof's device cache file (The suffix, hostname, is the
              first component of the host's name returned by
              gethostname(2).)
AUTHORS
       Lsof was written by Victor A.Abell <abe@purdue.edu> of Purdue
       University.  Many others have contributed to lsof.  They're
       listed in the 00CREDITS file of the lsof distribution.
DISTRIBUTION
       The latest distribution of lsof is available via anonymous ftp
       from the host lsof.itap.purdue.edu.  You'll find the lsof
       distribution in the pub/tools/unix/lsof directory.

       You can also use this URL:

              ftp://lsof.itap.purdue.edu/pub/tools/unix/lsof

       Lsof is also mirrored elsewhere.  When you access
       lsof.itap.purdue.edu and change to its pub/tools/unix/lsof
       directory, you'll be given a list of some mirror sites.  The
       pub/tools/unix/lsof directory also contains a more complete list
       in its mirrors file.  Use mirrors with caution - not all mirrors
       always have the latest lsof revision.

       Some pre-compiled Lsof executables are available on
       lsof.itap.purdue.edu, but their use is discouraged - it's better
       that you build your own from the sources.  If you feel you must
       use a pre-compiled executable, please read the cautions that
       appear in the README files of the pub/tools/unix/lsof/binaries
       subdirectories and in the 00* files of the distribution.

       More information on the lsof distribution can be found in its
       README.lsof_<version> file.  If you intend to get the lsof
       distribution and build it, please read README.lsof_<version> and
       the other 00* files of the distribution before sending questions
       to the author.
SEE ALSO
       Not all the following manual pages may exist in every UNIX
       dialect to which lsof has been ported.

       access(2), awk(1), crash(1), fattach(3C), ff(1), fstat(8),
       fuser(1), gethostname(2), isprint(3), kill(1), localtime(3),
       lstat(2), modload(8), mount(8), netstat(1), ofiles(8L), perl(1),
       ps(1), readlink(2), setlocale(3), stat(2), strftime(3), time(2),
       uname(1).
COLOPHON
       This page is part of the lsof (LiSt Open Files) project.
       Information about the project can be found at 
       http://people.freebsd.org/~abe/.  If you have a bug report for
       this manual page, send it to abe@purdue.edu.  This page was
       obtained from the tarball lsof_4.91_src.tar fetched from
       ftp://ftp.fu-berlin.de/pub/unix/tools/lsof/lsof.tar.gz on
       2024-06-14.  If you discover any rendering problems in this HTML
       version of the page, or you believe there is a better or more up-
       to-date source for the page, or you have corrections or
       improvements to the information in this COLOPHON (which is not
       part of the original manual page), send a mail to
       man-pages@man7.org

                              Revision-4.91                      LSOF(8)\end{lstlisting}
}}
\endinput  %  ==  ==  ==  ==  ==  ==  ==  ==  ==

	% % % \input{./components/man/man-objdump}
\subsection{\refObjdump: Display Information From Object Files}

{\tiny{
\begin{lstlisting}[language=bash]
NAME
       objdump - display information from object files
SYNOPSIS
       objdump [-a|--archive-headers]
               [-b bfdname|--target=bfdname]
               [-C|--demangle[=style] ]
               [-d|--disassemble[=symbol]]
               [-D|--disassemble-all]
               [-z|--disassemble-zeroes]
               [-EB|-EL|--endian={big | little }]
               [-f|--file-headers]
               [-F|--file-offsets]
               [--file-start-context]
               [-g|--debugging]
               [-e|--debugging-tags]
               [-h|--section-headers|--headers]
               [-i|--info]
               [-j section|--section=section]
               [-l|--line-numbers]
               [-S|--source]
               [--source-comment[=text]]
               [-m machine|--architecture=machine]
               [-M options|--disassembler-options=options]
               [-p|--private-headers]
               [-P options|--private=options]
               [-r|--reloc]
               [-R|--dynamic-reloc]
               [-s|--full-contents]
               [-Z|--decompress]
               [-W[lLiaprmfFsoORtUuTgAck]|
                --dwarf[=rawline,=decodedline,=info,=abbrev,=pubnames,=aranges,=macro,=frames,=frames-interp,=str,=str-offsets,=loc,=Ranges,=pubtypes,=trace_info,=trace_abbrev,=trace_aranges,=gdb_index,=addr,=cu_index,=links]]
               [-WK|--dwarf=follow-links]
               [-WN|--dwarf=no-follow-links]
               [-wD|--dwarf=use-debuginfod]
               [-wE|--dwarf=do-not-use-debuginfod]
               [-L|--process-links]
               [--ctf=section]
               [--sframe=section]
               [-G|--stabs]
               [-t|--syms]
               [-T|--dynamic-syms]
               [-x|--all-headers]
               [-w|--wide]
               [--start-address=address]
               [--stop-address=address]
               [--no-addresses]
               [--prefix-addresses]
               [--[no-]show-raw-insn]
               [--adjust-vma=offset]
               [--show-all-symbols]
               [--dwarf-depth=n]
               [--dwarf-start=n]
               [--ctf-parent=section]
               [--no-recurse-limit|--recurse-limit]
               [--special-syms]
               [--prefix=prefix]
               [--prefix-strip=level]
               [--insn-width=width]
               [--visualize-jumps[=color|=extended-color|=off]
               [--disassembler-color=[off|terminal|on|extended]
               [-U method] [--unicode=method]
               [-V|--version]
               [-H|--help]
               objfile...
DESCRIPTION
       objdump displays information about one or more object files.  The
       options control what particular information to display.  This
       information is mostly useful to programmers who are working on
       the compilation tools, as opposed to programmers who just want
       their program to compile and work.

       objfile... are the object files to be examined.  When you specify
       archives, objdump shows information on each of the member object
       files.
OPTIONS
       The long and short forms of options, shown here as alternatives,
       are equivalent.  At least one option from the list
       -a,-d,-D,-e,-f,-g,-G,-h,-H,-p,-P,-r,-R,-s,-S,-t,-T,-V,-x must be
       given.

       -a
       --archive-header
           If any of the objfile files are archives, display the archive
           header information (in a format similar to ls -l).  Besides
           the information you could list with ar tv, objdump -a shows
           the object file format of each archive member.

       --adjust-vma=offset
           When dumping information, first add offset to all the section
           addresses.  This is useful if the section addresses do not
           correspond to the symbol table, which can happen when putting
           sections at particular addresses when using a format which
           can not represent section addresses, such as a.out.

       -b bfdname
       --target=bfdname
           Specify that the object-code format for the object files is
           bfdname.  This option may not be necessary; objdump can
           automatically recognize many formats.

           For example,

                   objdump -b oasys -m vax -h fu.o

           displays summary information from the section headers (-h) of
           fu.o, which is explicitly identified (-m) as a VAX object
           file in the format produced by Oasys compilers.  You can list
           the formats available with the -i option.

       -C
       --demangle[=style]
           Decode (demangle) low-level symbol names into user-level
           names.  Besides removing any initial underscore prepended by
           the system, this makes C++ function names readable.
           Different compilers have different mangling styles. The
           optional demangling style argument can be used to choose an
           appropriate demangling style for your compiler.

       --recurse-limit
       --no-recurse-limit
       --recursion-limit
       --no-recursion-limit
           Enables or disables a limit on the amount of recursion
           performed whilst demangling strings.  Since the name mangling
           formats allow for an infinite level of recursion it is
           possible to create strings whose decoding will exhaust the
           amount of stack space available on the host machine,
           triggering a memory fault.  The limit tries to prevent this
           from happening by restricting recursion to 2048 levels of
           nesting.

           The default is for this limit to be enabled, but disabling it
           may be necessary in order to demangle truly complicated
           names.  Note however that if the recursion limit is disabled
           then stack exhaustion is possible and any bug reports about
           such an event will be rejected.

       -g
       --debugging
           Display debugging information.  This attempts to parse STABS
           debugging format information stored in the file and print it
           out using a C like syntax.  If no STABS debugging was found
           this option falls back on the -W option to print any DWARF
           information in the file.

       -e
       --debugging-tags
           Like -g, but the information is generated in a format
           compatible with ctags tool.

       -d
       --disassemble
       --disassemble=symbol
           Display the assembler mnemonics for the machine instructions
           from the input file.  This option only disassembles those
           sections which are expected to contain instructions.  If the
           optional symbol argument is given, then display the assembler
           mnemonics starting at symbol.  If symbol is a function name
           then disassembly will stop at the end of the function,
           otherwise it will stop when the next symbol is encountered.
           If there are no matches for symbol then nothing will be
           displayed.

           Note if the --dwarf=follow-links option is enabled then any
           symbol tables in linked debug info files will be read in and
           used when disassembling.

       -D
       --disassemble-all
           Like -d, but disassemble the contents of all non-empty non-
           bss sections, not just those expected to contain
           instructions.  -j may be used to select specific sections.

           This option also has a subtle effect on the disassembly of
           instructions in code sections.  When option -d is in effect
           objdump will assume that any symbols present in a code
           section occur on the boundary between instructions and it
           will refuse to disassemble across such a boundary.  When
           option -D is in effect however this assumption is supressed.
           This means that it is possible for the output of -d and -D to
           differ if, for example, data is stored in code sections.

           If the target is an ARM architecture this switch also has the
           effect of forcing the disassembler to decode pieces of data
           found in code sections as if they were instructions.

           Note if the --dwarf=follow-links option is enabled then any
           symbol tables in linked debug info files will be read in and
           used when disassembling.

       --no-addresses
           When disassembling, don't print addresses on each line or for
           symbols and relocation offsets.  In combination with
           --no-show-raw-insn this may be useful for comparing compiler
           output.

       --prefix-addresses
           When disassembling, print the complete address on each line.
           This is the older disassembly format.

       -EB
       -EL
       --endian={big|little}
           Specify the endianness of the object files.  This only
           affects disassembly.  This can be useful when disassembling a
           file format which does not describe endianness information,
           such as S-records.

       -f
       --file-headers
           Display summary information from the overall header of each
           of the objfile files.

       -F
       --file-offsets
           When disassembling sections, whenever a symbol is displayed,
           also display the file offset of the region of data that is
           about to be dumped.  If zeroes are being skipped, then when
           disassembly resumes, tell the user how many zeroes were
           skipped and the file offset of the location from where the
           disassembly resumes.  When dumping sections, display the file
           offset of the location from where the dump starts.

       --file-start-context
           Specify that when displaying interlisted source
           code/disassembly (assumes -S) from a file that has not yet
           been displayed, extend the context to the start of the file.

       -h
       --section-headers
       --headers
           Display summary information from the section headers of the
           object file.

           File segments may be relocated to nonstandard addresses, for
           example by using the -Ttext, -Tdata, or -Tbss options to ld.
           However, some object file formats, such as a.out, do not
           store the starting address of the file segments.  In those
           situations, although ld relocates the sections correctly,
           using objdump -h to list the file section headers cannot show
           the correct addresses.  Instead, it shows the usual
           addresses, which are implicit for the target.

           Note, in some cases it is possible for a section to have both
           the READONLY and the NOREAD attributes set.  In such cases
           the NOREAD attribute takes precedence, but objdump will
           report both since the exact setting of the flag bits might be
           important.

       -H
       --help
           Print a summary of the options to objdump and exit.

       -i
       --info
           Display a list showing all architectures and object formats
           available for specification with -b or -m.

       -j name
       --section=name
           Display information for section name.  This option may be
           specified multiple times.

       -L
       --process-links
           Display the contents of non-debug sections found in separate
           debuginfo files that are linked to the main file.  This
           option automatically implies the -WK option, and only
           sections requested by other command line options will be
           displayed.

       -l
       --line-numbers
           Label the display (using debugging information) with the
           filename and source line numbers corresponding to the object
           code or relocs shown.  Only useful with -d, -D, or -r.

       -m machine
       --architecture=machine
           Specify the architecture to use when disassembling object
           files.  This can be useful when disassembling object files
           which do not describe architecture information, such as
           S-records.  You can list the available architectures with the
           -i option.

           For most architectures it is possible to supply an
           architecture name and a machine name, separated by a colon.
           For example foo:bar would refer to the bar machine type in
           the foo architecture.  This can be helpful if objdump has
           been configured to support multiple architectures.

           If the target is an ARM architecture then this switch has an
           additional effect.  It restricts the disassembly to only
           those instructions supported by the architecture specified by
           machine.  If it is necessary to use this switch because the
           input file does not contain any architecture information, but
           it is also desired to disassemble all the instructions use
           -marm.

       -M options
       --disassembler-options=options
           Pass target specific information to the disassembler.  Only
           supported on some targets.  If it is necessary to specify
           more than one disassembler option then multiple -M options
           can be used or can be placed together into a comma separated
           list.

           For ARC, dsp controls the printing of DSP instructions, spfp
           selects the printing of FPX single precision FP instructions,
           dpfp selects the printing of FPX double precision FP
           instructions, quarkse_em selects the printing of special
           QuarkSE-EM instructions, fpuda selects the printing of double
           precision assist instructions, fpus selects the printing of
           FPU single precision FP instructions, while fpud selects the
           printing of FPU double precision FP instructions.
           Additionally, one can choose to have all the immediates
           printed in hexadecimal using hex.  By default, the short
           immediates are printed using the decimal representation,
           while the long immediate values are printed as hexadecimal.

           cpu=... allows one to enforce a particular ISA when
           disassembling instructions, overriding the -m value or
           whatever is in the ELF file.  This might be useful to select
           ARC EM or HS ISA, because architecture is same for those and
           disassembler relies on private ELF header data to decide if
           code is for EM or HS.  This option might be specified
           multiple times - only the latest value will be used.  Valid
           values are same as for the assembler -mcpu=... option.

           If the target is an ARM architecture then this switch can be
           used to select which register name set is used during
           disassembler.  Specifying -M reg-names-std (the default) will
           select the register names as used in ARM's instruction set
           documentation, but with register 13 called 'sp', register 14
           called 'lr' and register 15 called 'pc'.  Specifying -M reg-
           names-apcs will select the name set used by the ARM Procedure
           Call Standard, whilst specifying -M reg-names-raw will just
           use r followed by the register number.

           There are also two variants on the APCS register naming
           scheme enabled by -M reg-names-atpcs and -M reg-names-
           special-atpcs which use the ARM/Thumb Procedure Call Standard
           naming conventions.  (Either with the normal register names
           or the special register names).

           This option can also be used for ARM architectures to force
           the disassembler to interpret all instructions as Thumb
           instructions by using the switch
           --disassembler-options=force-thumb.  This can be useful when
           attempting to disassemble thumb code produced by other
           compilers.

           For AArch64 targets this switch can be used to set whether
           instructions are disassembled as the most general instruction
           using the -M no-aliases option or whether instruction notes
           should be generated as comments in the disasssembly using -M
           notes.

           For the x86, some of the options duplicate functions of the
           -m switch, but allow finer grained control.

           "x86-64"
           "i386"
           "i8086"
               Select disassembly for the given architecture.

           "intel"
           "att"
               Select between intel syntax mode and AT&T syntax mode.

           "amd64"
           "intel64"
               Select between AMD64 ISA and Intel64 ISA.

           "intel-mnemonic"
           "att-mnemonic"
               Select between intel mnemonic mode and AT&T mnemonic
               mode.  Note: "intel-mnemonic" implies "intel" and
               "att-mnemonic" implies "att".

           "addr64"
           "addr32"
           "addr16"
           "data32"
           "data16"
               Specify the default address size and operand size.  These
               five options will be overridden if "x86-64", "i386" or
               "i8086" appear later in the option string.

           "suffix"
               When in AT&T mode and also for a limited set of
               instructions when in Intel mode, instructs the
               disassembler to print a mnemonic suffix even when the
               suffix could be inferred by the operands or, for certain
               instructions, the execution mode's defaults.

           For PowerPC, the -M argument raw selects disasssembly of
           hardware insns rather than aliases.  For example, you will
           see "rlwinm" rather than "clrlwi", and "addi" rather than
           "li".  All of the -m arguments for gas that select a CPU are
           supported.  These are: 403, 405, 440, 464, 476, 601, 603,
           604, 620, 7400, 7410, 7450, 7455, 750cl, 821, 850, 860, a2,
           booke, booke32, cell, com, e200z2, e200z4, e300, e500,
           e500mc, e500mc64, e500x2, e5500, e6500, efs, power4, power5,
           power6, power7, power8, power9, power10, ppc, ppc32, ppc64,
           ppc64bridge, ppcps, pwr, pwr2, pwr4, pwr5, pwr5x, pwr6, pwr7,
           pwr8, pwr9, pwr10, pwrx, titan, vle, and future.  32 and 64
           modify the default or a prior CPU selection, disabling and
           enabling 64-bit insns respectively.  In addition, altivec,
           any, lsp, htm, vsx, spe and  spe2 add capabilities to a
           previous or later CPU selection.  any will disassemble any
           opcode known to binutils, but in cases where an opcode has
           two different meanings or different arguments, you may not
           see the disassembly you expect.  If you disassemble without
           giving a CPU selection, a default will be chosen from
           information gleaned by BFD from the object files headers, but
           the result again may not be as you expect.

           For MIPS, this option controls the printing of instruction
           mnemonic names and register names in disassembled
           instructions.  Multiple selections from the following may be
           specified as a comma separated string, and invalid options
           are ignored:

           "no-aliases"
               Print the 'raw' instruction mnemonic instead of some
               pseudo instruction mnemonic.  I.e., print 'daddu' or 'or'
               instead of 'move', 'sll' instead of 'nop', etc.

           "msa"
               Disassemble MSA instructions.

           "virt"
               Disassemble the virtualization ASE instructions.

           "xpa"
               Disassemble the eXtended Physical Address (XPA) ASE
               instructions.

           "gpr-names=ABI"
               Print GPR (general-purpose register) names as appropriate
               for the specified ABI.  By default, GPR names are
               selected according to the ABI of the binary being
               disassembled.

           "fpr-names=ABI"
               Print FPR (floating-point register) names as appropriate
               for the specified ABI.  By default, FPR numbers are
               printed rather than names.

           "cp0-names=ARCH"
               Print CP0 (system control coprocessor; coprocessor 0)
               register names as appropriate for the CPU or architecture
               specified by ARCH.  By default, CP0 register names are
               selected according to the architecture and CPU of the
               binary being disassembled.

           "hwr-names=ARCH"
               Print HWR (hardware register, used by the "rdhwr"
               instruction) names as appropriate for the CPU or
               architecture specified by ARCH.  By default, HWR names
               are selected according to the architecture and CPU of the
               binary being disassembled.

           "reg-names=ABI"
               Print GPR and FPR names as appropriate for the selected
               ABI.

           "reg-names=ARCH"
               Print CPU-specific register names (CP0 register and HWR
               names) as appropriate for the selected CPU or
               architecture.

           For any of the options listed above, ABI or ARCH may be
           specified as numeric to have numbers printed rather than
           names, for the selected types of registers.  You can list the
           available values of ABI and ARCH using the --help option.

           For VAX, you can specify function entry addresses with -M
           entry:0xf00ba.  You can use this multiple times to properly
           disassemble VAX binary files that don't contain symbol tables
           (like ROM dumps).  In these cases, the function entry mask
           would otherwise be decoded as VAX instructions, which would
           probably lead the rest of the function being wrongly
           disassembled.

       -p
       --private-headers
           Print information that is specific to the object file format.
           The exact information printed depends upon the object file
           format.  For some object file formats, no additional
           information is printed.

       -P options
       --private=options
           Print information that is specific to the object file format.
           The argument options is a comma separated list that depends
           on the format (the lists of options is displayed with the
           help).

           For XCOFF, the available options are:

           "header"
           "aout"
           "sections"
           "syms"
           "relocs"
           "lineno,"
           "loader"
           "except"
           "typchk"
           "traceback"
           "toc"
           "ldinfo"

           For PE, the available options are:

           "header"
           "sections"

           Not all object formats support this option.  In particular
           the ELF format does not use it.

       -r
       --reloc
           Print the relocation entries of the file.  If used with -d or
           -D, the relocations are printed interspersed with the
           disassembly.

       -R
       --dynamic-reloc
           Print the dynamic relocation entries of the file.  This is
           only meaningful for dynamic objects, such as certain types of
           shared libraries.  As for -r, if used with -d or -D, the
           relocations are printed interspersed with the disassembly.

       -s
       --full-contents
           Display the full contents of sections, often used in
           combination with -j to request specific sections.  By default
           all non-empty non-bss sections are displayed.  By default any
           compressed section will be displayed in its compressed form.
           In order to see the contents in a decompressed form add the
           -Z option to the command line.

       -S
       --source
           Display source code intermixed with disassembly, if possible.
           Implies -d.

       --show-all-symbols
           When disassembling, show all the symbols that match a given
           address, not just the first one.

       --source-comment[=txt]
           Like the -S option, but all source code lines are displayed
           with a prefix of txt.  Typically txt will be a comment string
           which can be used to distinguish the assembler code from the
           source code.  If txt is not provided then a default string of
           "# " (hash followed by a space), will be used.

       --prefix=prefix
           Specify prefix to add to the absolute paths when used with
           -S.

       --prefix-strip=level
           Indicate how many initial directory names to strip off the
           hardwired absolute paths. It has no effect without
           --prefix=prefix.

       --show-raw-insn
           When disassembling instructions, print the instruction in hex
           as well as in symbolic form.  This is the default except when
           --prefix-addresses is used.

       --no-show-raw-insn
           When disassembling instructions, do not print the instruction
           bytes.  This is the default when --prefix-addresses is used.

       --insn-width=width
           Display width bytes on a single line when disassembling
           instructions.

       --visualize-jumps[=color|=extended-color|=off]
           Visualize jumps that stay inside a function by drawing ASCII
           art between the start and target addresses.  The optional
           =color argument adds color to the output using simple
           terminal colors.  Alternatively the =extended-color argument
           will add color using 8bit colors, but these might not work on
           all terminals.

           If it is necessary to disable the visualize-jumps option
           after it has previously been enabled then use
           visualize-jumps=off.

       --disassembler-color=off
       --disassembler-color=terminal
       --disassembler-color=on|color|colour
       --disassembler-color=extened|extended-color|extened-colour
           Enables or disables the use of colored syntax highlighting in
           disassembly output.  The default behaviour is determined via
           a configure time option.  Note, not all architectures support
           colored syntax highlighting, and depending upon the terminal
           used, colored output may not actually be legible.

           The on argument adds colors using simple terminal colors.

           The terminal argument does the same, but only if the output
           device is a terminal.

           The extended-color argument is similar to the on argument,
           but it uses 8-bit colors.  These may not work on all
           terminals.

           The off argument disables colored disassembly.

       -W[lLiaprmfFsoORtUuTgAckK]
       --dwarf[=rawline,=decodedline,=info,=abbrev,=pubnames,=aranges,=macro,=frames,=frames-interp,=str,=str-offsets,=loc,=Ranges,=pubtypes,=trace_info,=trace_abbrev,=trace_aranges,=gdb_index,=addr,=cu_index,=links,=follow-links]
           Displays the contents of the DWARF debug sections in the
           file, if any are present.  Compressed debug sections are
           automatically decompressed (temporarily) before they are
           displayed.  If one or more of the optional letters or words
           follows the switch then only those type(s) of data will be
           dumped.  The letters and words refer to the following
           information:

           "a"
           "=abbrev"
               Displays the contents of the .debug_abbrev section.

           "A"
           "=addr"
               Displays the contents of the .debug_addr section.

           "c"
           "=cu_index"
               Displays the contents of the .debug_cu_index and/or
               .debug_tu_index sections.

           "f"
           "=frames"
               Display the raw contents of a .debug_frame section.

           "F"
           "=frames-interp"
               Display the interpreted contents of a .debug_frame
               section.

           "g"
           "=gdb_index"
               Displays the contents of the .gdb_index and/or
               .debug_names sections.

           "i"
           "=info"
               Displays the contents of the .debug_info section.  Note:
               the output from this option can also be restricted by the
               use of the --dwarf-depth and --dwarf-start options.

           "k"
           "=links"
               Displays the contents of the .gnu_debuglink,
               .gnu_debugaltlink and .debug_sup sections, if any of them
               are present.  Also displays any links to separate dwarf
               object files (dwo), if they are specified by the
               DW_AT_GNU_dwo_name or DW_AT_dwo_name attributes in the
               .debug_info section.

           "K"
           "=follow-links"
               Display the contents of any selected debug sections that
               are found in linked, separate debug info file(s).  This
               can result in multiple versions of the same debug section
               being displayed if it exists in more than one file.

               In addition, when displaying DWARF attributes, if a form
               is found that references the separate debug info file,
               then the referenced contents will also be displayed.

               Note - in some distributions this option is enabled by
               default.  It can be disabled via the N debug option.  The
               default can be chosen when configuring the binutils via
               the --enable-follow-debug-links=yes or
               --enable-follow-debug-links=no options.  If these are not
               used then the default is to enable the following of debug
               links.

               Note - if support for the debuginfod protocol was enabled
               when the binutils were built then this option will also
               include an attempt to contact any debuginfod servers
               mentioned in the DEBUGINFOD_URLS environment variable.
               This could take some time to resolve.  This behaviour can
               be disabled via the =do-not-use-debuginfod debug option.

           "N"
           "=no-follow-links"
               Disables the following of links to separate debug info
               files.

           "D"
           "=use-debuginfod"
               Enables contacting debuginfod servers if there is a need
               to follow debug links.  This is the default behaviour.

           "E"
           "=do-not-use-debuginfod"
               Disables contacting debuginfod servers when there is a
               need to follow debug links.

           "l"
           "=rawline"
               Displays the contents of the .debug_line section in a raw
               format.

           "L"
           "=decodedline"
               Displays the interpreted contents of the .debug_line
               section.

           "m"
           "=macro"
               Displays the contents of the .debug_macro and/or
               .debug_macinfo sections.

           "o"
           "=loc"
               Displays the contents of the .debug_loc and/or
               .debug_loclists sections.

           "O"
           "=str-offsets"
               Displays the contents of the .debug_str_offsets section.

           "p"
           "=pubnames"
               Displays the contents of the .debug_pubnames and/or
               .debug_gnu_pubnames sections.

           "r"
           "=aranges"
               Displays the contents of the .debug_aranges section.

           "R"
           "=Ranges"
               Displays the contents of the .debug_ranges and/or
               .debug_rnglists sections.

           "s"
           "=str"
               Displays the contents of the .debug_str, .debug_line_str
               and/or .debug_str_offsets sections.

           "t"
           "=pubtype"
               Displays the contents of the .debug_pubtypes and/or
               .debug_gnu_pubtypes sections.

           "T"
           "=trace_aranges"
               Displays the contents of the .trace_aranges section.

           "u"
           "=trace_abbrev"
               Displays the contents of the .trace_abbrev section.

           "U"
           "=trace_info"
               Displays the contents of the .trace_info section.

           Note: displaying the contents of .debug_static_funcs,
           .debug_static_vars and debug_weaknames sections is not
           currently supported.

       --dwarf-depth=n
           Limit the dump of the ".debug_info" section to n children.
           This is only useful with --debug-dump=info.  The default is
           to print all DIEs; the special value 0 for n will also have
           this effect.

           With a non-zero value for n, DIEs at or deeper than n levels
           will not be printed.  The range for n is zero-based.

       --dwarf-start=n
           Print only DIEs beginning with the DIE numbered n.  This is
           only useful with --debug-dump=info.

           If specified, this option will suppress printing of any
           header information and all DIEs before the DIE numbered n.
           Only siblings and children of the specified DIE will be
           printed.

           This can be used in conjunction with --dwarf-depth.

       --dwarf-check
           Enable additional checks for consistency of Dwarf
           information.

       --ctf[=section]
           Display the contents of the specified CTF section.  CTF
           sections themselves contain many subsections, all of which
           are displayed in order.

           By default, display the name of the section named .ctf, which
           is the name emitted by ld.

       --ctf-parent=member
           If the CTF section contains ambiguously-defined types, it
           will consist of an archive of many CTF dictionaries, all
           inheriting from one dictionary containing unambiguous types.
           This member is by default named .ctf, like the section
           containing it, but it is possible to change this name using
           the "ctf_link_set_memb_name_changer" function at link time.
           When looking at CTF archives that have been created by a
           linker that uses the name changer to rename the parent
           archive member, --ctf-parent can be used to specify the name
           used for the parent.

       --sframe[=section]
           Display the contents of the specified SFrame section.

           By default, display the name of the section named .sframe,
           which is the name emitted by ld.

       -G
       --stabs
           Display the full contents of any sections requested.  Display
           the contents of the .stab and .stab.index and .stab.excl
           sections from an ELF file.  This is only useful on systems
           (such as Solaris 2.0) in which ".stab" debugging symbol-table
           entries are carried in an ELF section.  In most other file
           formats, debugging symbol-table entries are interleaved with
           linkage symbols, and are visible in the --syms output.

       --start-address=address
           Start displaying data at the specified address.  This affects
           the output of the -d, -r and -s options.

       --stop-address=address
           Stop displaying data at the specified address.  This affects
           the output of the -d, -r and -s options.

       -t
       --syms
           Print the symbol table entries of the file.  This is similar
           to the information provided by the nm program, although the
           display format is different.  The format of the output
           depends upon the format of the file being dumped, but there
           are two main types.  One looks like this:

                   [  4](sec  3)(fl 0x00)(ty   0)(scl   3) (nx 1) 0x00000000 .bss
                   [  6](sec  1)(fl 0x00)(ty   0)(scl   2) (nx 0) 0x00000000 fred

           where the number inside the square brackets is the number of
           the entry in the symbol table, the sec number is the section
           number, the fl value are the symbol's flag bits, the ty
           number is the symbol's type, the scl number is the symbol's
           storage class and the nx value is the number of auxiliary
           entries associated with the symbol.  The last two fields are
           the symbol's value and its name.

           The other common output format, usually seen with ELF based
           files, looks like this:

                   00000000 l    d  .bss   00000000 .bss
                   00000000 g       .text  00000000 fred

           Here the first number is the symbol's value (sometimes
           referred to as its address).  The next field is actually a
           set of characters and spaces indicating the flag bits that
           are set on the symbol.  These characters are described below.
           Next is the section with which the symbol is associated or
           *ABS* if the section is absolute (ie not connected with any
           section), or *UND* if the section is referenced in the file
           being dumped, but not defined there.

           After the section name comes another field, a number, which
           for common symbols is the alignment and for other symbol is
           the size.  Finally the symbol's name is displayed.

           The flag characters are divided into 7 groups as follows:

           "l"
           "g"
           "u"
           "!" The symbol is a local (l), global (g), unique global (u),
               neither global nor local (a space) or both global and
               local (!).  A symbol can be neither local or global for a
               variety of reasons, e.g., because it is used for
               debugging, but it is probably an indication of a bug if
               it is ever both local and global.  Unique global symbols
               are a GNU extension to the standard set of ELF symbol
               bindings.  For such a symbol the dynamic linker will make
               sure that in the entire process there is just one symbol
               with this name and type in use.

           "w" The symbol is weak (w) or strong (a space).

           "C" The symbol denotes a constructor (C) or an ordinary
               symbol (a space).

           "W" The symbol is a warning (W) or a normal symbol (a space).
               A warning symbol's name is a message to be displayed if
               the symbol following the warning symbol is ever
               referenced.

           "I"
           "i" The symbol is an indirect reference to another symbol
               (I), a function to be evaluated during reloc processing
               (i) or a normal symbol (a space).

           "d"
           "D" The symbol is a debugging symbol (d) or a dynamic symbol
               (D) or a normal symbol (a space).

           "F"
           "f"
           "O" The symbol is the name of a function (F) or a file (f) or
               an object (O) or just a normal symbol (a space).

       -T
       --dynamic-syms
           Print the dynamic symbol table entries of the file.  This is
           only meaningful for dynamic objects, such as certain types of
           shared libraries.  This is similar to the information
           provided by the nm program when given the -D (--dynamic)
           option.

           The output format is similar to that produced by the --syms
           option, except that an extra field is inserted before the
           symbol's name, giving the version information associated with
           the symbol.  If the version is the default version to be used
           when resolving unversioned references to the symbol then it's
           displayed as is, otherwise it's put into parentheses.

       --special-syms
           When displaying symbols include those which the target
           considers to be special in some way and which would not
           normally be of interest to the user.

       -U [d|i|l|e|x|h]
       --unicode=[default|invalid|locale|escape|hex|highlight]
           Controls the display of UTF-8 encoded multibyte characters in
           strings.  The default (--unicode=default) is to give them no
           special treatment.  The --unicode=locale option displays the
           sequence in the current locale, which may or may not support
           them.  The options --unicode=hex and --unicode=invalid
           display them as hex byte sequences enclosed by either angle
           brackets or curly braces.

           The --unicode=escape option displays them as escape sequences
           (\uxxxx) and the --unicode=highlight option displays them as
           escape sequences highlighted in red (if supported by the
           output device).  The colouring is intended to draw attention
           to the presence of unicode sequences where they might not be
           expected.

       -V
       --version
           Print the version number of objdump and exit.

       -x
       --all-headers
           Display all available header information, including the
           symbol table and relocation entries.  Using -x is equivalent
           to specifying all of -a -f -h -p -r -t.

       -w
       --wide
           Format some lines for output devices that have more than 80
           columns.  Also do not truncate symbol names when they are
           displayed.

       -z
       --disassemble-zeroes
           Normally the disassembly output will skip blocks of zeroes.
           This option directs the disassembler to disassemble those
           blocks, just like any other data.

       -Z
       --decompress
           The -Z option is meant to be used in conunction with the -s
           option.  It instructs objdump to decompress any compressed
           sections before displaying their contents.

       @file
           Read command-line options from file.  The options read are
           inserted in place of the original @file option.  If file does
           not exist, or cannot be read, then the option will be treated
           literally, and not removed.

           Options in file are separated by whitespace.  A whitespace
           character may be included in an option by surrounding the
           entire option in either single or double quotes.  Any
           character (including a backslash) may be included by
           prefixing the character to be included with a backslash.  The
           file may itself contain additional @file options; any such
           options will be processed recursively.
SEE ALSO
       nm(1), readelf(1), and the Info entries for binutils.
COPYRIGHT
       Copyright (c) 1991-2024 Free Software Foundation, Inc.

       Permission is granted to copy, distribute and/or modify this
       document under the terms of the GNU Free Documentation License,
       Version 1.3 or any later version published by the Free Software
       Foundation; with no Invariant Sections, with no Front-Cover
       Texts, and with no Back-Cover Texts.  A copy of the license is
       included in the section entitled "GNU Free Documentation License".
COLOPHON
       This page is part of the binutils (a collection of tools for
       working with executable binaries) project.  Information about the
       project can be found at http://www.gnu.org/software/binutils/.
       If you have a bug report for this manual page, see
       http://sourceware.org/bugzilla/enter_bug.cgi?product=binutils.
       This page was obtained from the tarball binutils-2.42.tar.gz
       fetched from https://ftp.gnu.org/gnu/binutils/ on 2024-06-14.
       If you discover any rendering problems in this HTML version of
       the page, or you believe there is a better or more up-to-date
       source for the page, or you have corrections or improvements to
       the information in this COLOPHON (which is not part of the
       original manual page), send a mail to man-pages@man7.org

binutils-2.42                  2024-06-14                     OBJDUMP(1)
\end{lstlisting}
}}
\endinput  %  ==  ==  ==  ==  ==  ==  ==  ==  ==

\subsection{\refObjdump: Display Information From Object Files}

{\tiny{
\begin{lstlisting}[language=bash]
NAME
       objdump - display information from object files
SYNOPSIS
       objdump [-a|--archive-headers]
               [-b bfdname|--target=bfdname]
               [-C|--demangle[=style] ]
               [-d|--disassemble[=symbol]]
               [-D|--disassemble-all]
               [-z|--disassemble-zeroes]
               [-EB|-EL|--endian={big | little }]
               [-f|--file-headers]
               [-F|--file-offsets]
               [--file-start-context]
               [-g|--debugging]
               [-e|--debugging-tags]
               [-h|--section-headers|--headers]
               [-i|--info]
               [-j section|--section=section]
               [-l|--line-numbers]
               [-S|--source]
               [--source-comment[=text]]
               [-m machine|--architecture=machine]
               [-M options|--disassembler-options=options]
               [-p|--private-headers]
               [-P options|--private=options]
               [-r|--reloc]
               [-R|--dynamic-reloc]
               [-s|--full-contents]
               [-Z|--decompress]
               [-W[lLiaprmfFsoORtUuTgAck]|
                --dwarf[=rawline,=decodedline,=info,=abbrev,=pubnames,=aranges,=macro,=frames,=frames-interp,=str,=str-offsets,=loc,=Ranges,=pubtypes,=trace_info,=trace_abbrev,=trace_aranges,=gdb_index,=addr,=cu_index,=links]]
               [-WK|--dwarf=follow-links]
               [-WN|--dwarf=no-follow-links]
               [-wD|--dwarf=use-debuginfod]
               [-wE|--dwarf=do-not-use-debuginfod]
               [-L|--process-links]
               [--ctf=section]
               [--sframe=section]
               [-G|--stabs]
               [-t|--syms]
               [-T|--dynamic-syms]
               [-x|--all-headers]
               [-w|--wide]
               [--start-address=address]
               [--stop-address=address]
               [--no-addresses]
               [--prefix-addresses]
               [--[no-]show-raw-insn]
               [--adjust-vma=offset]
               [--show-all-symbols]
               [--dwarf-depth=n]
               [--dwarf-start=n]
               [--ctf-parent=section]
               [--no-recurse-limit|--recurse-limit]
               [--special-syms]
               [--prefix=prefix]
               [--prefix-strip=level]
               [--insn-width=width]
               [--visualize-jumps[=color|=extended-color|=off]
               [--disassembler-color=[off|terminal|on|extended]
               [-U method] [--unicode=method]
               [-V|--version]
               [-H|--help]
               objfile...
DESCRIPTION
       objdump displays information about one or more object files.  The
       options control what particular information to display.  This
       information is mostly useful to programmers who are working on
       the compilation tools, as opposed to programmers who just want
       their program to compile and work.

       objfile... are the object files to be examined.  When you specify
       archives, objdump shows information on each of the member object
       files.
OPTIONS
       The long and short forms of options, shown here as alternatives,
       are equivalent.  At least one option from the list
       -a,-d,-D,-e,-f,-g,-G,-h,-H,-p,-P,-r,-R,-s,-S,-t,-T,-V,-x must be
       given.

       -a
       --archive-header
           If any of the objfile files are archives, display the archive
           header information (in a format similar to ls -l).  Besides
           the information you could list with ar tv, objdump -a shows
           the object file format of each archive member.

       --adjust-vma=offset
           When dumping information, first add offset to all the section
           addresses.  This is useful if the section addresses do not
           correspond to the symbol table, which can happen when putting
           sections at particular addresses when using a format which
           can not represent section addresses, such as a.out.

       -b bfdname
       --target=bfdname
           Specify that the object-code format for the object files is
           bfdname.  This option may not be necessary; objdump can
           automatically recognize many formats.

           For example,

                   objdump -b oasys -m vax -h fu.o

           displays summary information from the section headers (-h) of
           fu.o, which is explicitly identified (-m) as a VAX object
           file in the format produced by Oasys compilers.  You can list
           the formats available with the -i option.

       -C
       --demangle[=style]
           Decode (demangle) low-level symbol names into user-level
           names.  Besides removing any initial underscore prepended by
           the system, this makes C++ function names readable.
           Different compilers have different mangling styles. The
           optional demangling style argument can be used to choose an
           appropriate demangling style for your compiler.

       --recurse-limit
       --no-recurse-limit
       --recursion-limit
       --no-recursion-limit
           Enables or disables a limit on the amount of recursion
           performed whilst demangling strings.  Since the name mangling
           formats allow for an infinite level of recursion it is
           possible to create strings whose decoding will exhaust the
           amount of stack space available on the host machine,
           triggering a memory fault.  The limit tries to prevent this
           from happening by restricting recursion to 2048 levels of
           nesting.

           The default is for this limit to be enabled, but disabling it
           may be necessary in order to demangle truly complicated
           names.  Note however that if the recursion limit is disabled
           then stack exhaustion is possible and any bug reports about
           such an event will be rejected.

       -g
       --debugging
           Display debugging information.  This attempts to parse STABS
           debugging format information stored in the file and print it
           out using a C like syntax.  If no STABS debugging was found
           this option falls back on the -W option to print any DWARF
           information in the file.

       -e
       --debugging-tags
           Like -g, but the information is generated in a format
           compatible with ctags tool.

       -d
       --disassemble
       --disassemble=symbol
           Display the assembler mnemonics for the machine instructions
           from the input file.  This option only disassembles those
           sections which are expected to contain instructions.  If the
           optional symbol argument is given, then display the assembler
           mnemonics starting at symbol.  If symbol is a function name
           then disassembly will stop at the end of the function,
           otherwise it will stop when the next symbol is encountered.
           If there are no matches for symbol then nothing will be
           displayed.

           Note if the --dwarf=follow-links option is enabled then any
           symbol tables in linked debug info files will be read in and
           used when disassembling.

       -D
       --disassemble-all
           Like -d, but disassemble the contents of all non-empty non-
           bss sections, not just those expected to contain
           instructions.  -j may be used to select specific sections.

           This option also has a subtle effect on the disassembly of
           instructions in code sections.  When option -d is in effect
           objdump will assume that any symbols present in a code
           section occur on the boundary between instructions and it
           will refuse to disassemble across such a boundary.  When
           option -D is in effect however this assumption is supressed.
           This means that it is possible for the output of -d and -D to
           differ if, for example, data is stored in code sections.

           If the target is an ARM architecture this switch also has the
           effect of forcing the disassembler to decode pieces of data
           found in code sections as if they were instructions.

           Note if the --dwarf=follow-links option is enabled then any
           symbol tables in linked debug info files will be read in and
           used when disassembling.

       --no-addresses
           When disassembling, don't print addresses on each line or for
           symbols and relocation offsets.  In combination with
           --no-show-raw-insn this may be useful for comparing compiler
           output.

       --prefix-addresses
           When disassembling, print the complete address on each line.
           This is the older disassembly format.

       -EB
       -EL
       --endian={big|little}
           Specify the endianness of the object files.  This only
           affects disassembly.  This can be useful when disassembling a
           file format which does not describe endianness information,
           such as S-records.

       -f
       --file-headers
           Display summary information from the overall header of each
           of the objfile files.

       -F
       --file-offsets
           When disassembling sections, whenever a symbol is displayed,
           also display the file offset of the region of data that is
           about to be dumped.  If zeroes are being skipped, then when
           disassembly resumes, tell the user how many zeroes were
           skipped and the file offset of the location from where the
           disassembly resumes.  When dumping sections, display the file
           offset of the location from where the dump starts.

       --file-start-context
           Specify that when displaying interlisted source
           code/disassembly (assumes -S) from a file that has not yet
           been displayed, extend the context to the start of the file.

       -h
       --section-headers
       --headers
           Display summary information from the section headers of the
           object file.

           File segments may be relocated to nonstandard addresses, for
           example by using the -Ttext, -Tdata, or -Tbss options to ld.
           However, some object file formats, such as a.out, do not
           store the starting address of the file segments.  In those
           situations, although ld relocates the sections correctly,
           using objdump -h to list the file section headers cannot show
           the correct addresses.  Instead, it shows the usual
           addresses, which are implicit for the target.

           Note, in some cases it is possible for a section to have both
           the READONLY and the NOREAD attributes set.  In such cases
           the NOREAD attribute takes precedence, but objdump will
           report both since the exact setting of the flag bits might be
           important.

       -H
       --help
           Print a summary of the options to objdump and exit.

       -i
       --info
           Display a list showing all architectures and object formats
           available for specification with -b or -m.

       -j name
       --section=name
           Display information for section name.  This option may be
           specified multiple times.

       -L
       --process-links
           Display the contents of non-debug sections found in separate
           debuginfo files that are linked to the main file.  This
           option automatically implies the -WK option, and only
           sections requested by other command line options will be
           displayed.

       -l
       --line-numbers
           Label the display (using debugging information) with the
           filename and source line numbers corresponding to the object
           code or relocs shown.  Only useful with -d, -D, or -r.

       -m machine
       --architecture=machine
           Specify the architecture to use when disassembling object
           files.  This can be useful when disassembling object files
           which do not describe architecture information, such as
           S-records.  You can list the available architectures with the
           -i option.

           For most architectures it is possible to supply an
           architecture name and a machine name, separated by a colon.
           For example foo:bar would refer to the bar machine type in
           the foo architecture.  This can be helpful if objdump has
           been configured to support multiple architectures.

           If the target is an ARM architecture then this switch has an
           additional effect.  It restricts the disassembly to only
           those instructions supported by the architecture specified by
           machine.  If it is necessary to use this switch because the
           input file does not contain any architecture information, but
           it is also desired to disassemble all the instructions use
           -marm.

       -M options
       --disassembler-options=options
           Pass target specific information to the disassembler.  Only
           supported on some targets.  If it is necessary to specify
           more than one disassembler option then multiple -M options
           can be used or can be placed together into a comma separated
           list.

           For ARC, dsp controls the printing of DSP instructions, spfp
           selects the printing of FPX single precision FP instructions,
           dpfp selects the printing of FPX double precision FP
           instructions, quarkse_em selects the printing of special
           QuarkSE-EM instructions, fpuda selects the printing of double
           precision assist instructions, fpus selects the printing of
           FPU single precision FP instructions, while fpud selects the
           printing of FPU double precision FP instructions.
           Additionally, one can choose to have all the immediates
           printed in hexadecimal using hex.  By default, the short
           immediates are printed using the decimal representation,
           while the long immediate values are printed as hexadecimal.

           cpu=... allows one to enforce a particular ISA when
           disassembling instructions, overriding the -m value or
           whatever is in the ELF file.  This might be useful to select
           ARC EM or HS ISA, because architecture is same for those and
           disassembler relies on private ELF header data to decide if
           code is for EM or HS.  This option might be specified
           multiple times - only the latest value will be used.  Valid
           values are same as for the assembler -mcpu=... option.

           If the target is an ARM architecture then this switch can be
           used to select which register name set is used during
           disassembler.  Specifying -M reg-names-std (the default) will
           select the register names as used in ARM's instruction set
           documentation, but with register 13 called 'sp', register 14
           called 'lr' and register 15 called 'pc'.  Specifying -M reg-
           names-apcs will select the name set used by the ARM Procedure
           Call Standard, whilst specifying -M reg-names-raw will just
           use r followed by the register number.

           There are also two variants on the APCS register naming
           scheme enabled by -M reg-names-atpcs and -M reg-names-
           special-atpcs which use the ARM/Thumb Procedure Call Standard
           naming conventions.  (Either with the normal register names
           or the special register names).

           This option can also be used for ARM architectures to force
           the disassembler to interpret all instructions as Thumb
           instructions by using the switch
           --disassembler-options=force-thumb.  This can be useful when
           attempting to disassemble thumb code produced by other
           compilers.

           For AArch64 targets this switch can be used to set whether
           instructions are disassembled as the most general instruction
           using the -M no-aliases option or whether instruction notes
           should be generated as comments in the disasssembly using -M
           notes.

           For the x86, some of the options duplicate functions of the
           -m switch, but allow finer grained control.

           "x86-64"
           "i386"
           "i8086"
               Select disassembly for the given architecture.

           "intel"
           "att"
               Select between intel syntax mode and AT&T syntax mode.

           "amd64"
           "intel64"
               Select between AMD64 ISA and Intel64 ISA.

           "intel-mnemonic"
           "att-mnemonic"
               Select between intel mnemonic mode and AT&T mnemonic
               mode.  Note: "intel-mnemonic" implies "intel" and
               "att-mnemonic" implies "att".

           "addr64"
           "addr32"
           "addr16"
           "data32"
           "data16"
               Specify the default address size and operand size.  These
               five options will be overridden if "x86-64", "i386" or
               "i8086" appear later in the option string.

           "suffix"
               When in AT&T mode and also for a limited set of
               instructions when in Intel mode, instructs the
               disassembler to print a mnemonic suffix even when the
               suffix could be inferred by the operands or, for certain
               instructions, the execution mode's defaults.

           For PowerPC, the -M argument raw selects disasssembly of
           hardware insns rather than aliases.  For example, you will
           see "rlwinm" rather than "clrlwi", and "addi" rather than
           "li".  All of the -m arguments for gas that select a CPU are
           supported.  These are: 403, 405, 440, 464, 476, 601, 603,
           604, 620, 7400, 7410, 7450, 7455, 750cl, 821, 850, 860, a2,
           booke, booke32, cell, com, e200z2, e200z4, e300, e500,
           e500mc, e500mc64, e500x2, e5500, e6500, efs, power4, power5,
           power6, power7, power8, power9, power10, ppc, ppc32, ppc64,
           ppc64bridge, ppcps, pwr, pwr2, pwr4, pwr5, pwr5x, pwr6, pwr7,
           pwr8, pwr9, pwr10, pwrx, titan, vle, and future.  32 and 64
           modify the default or a prior CPU selection, disabling and
           enabling 64-bit insns respectively.  In addition, altivec,
           any, lsp, htm, vsx, spe and  spe2 add capabilities to a
           previous or later CPU selection.  any will disassemble any
           opcode known to binutils, but in cases where an opcode has
           two different meanings or different arguments, you may not
           see the disassembly you expect.  If you disassemble without
           giving a CPU selection, a default will be chosen from
           information gleaned by BFD from the object files headers, but
           the result again may not be as you expect.

           For MIPS, this option controls the printing of instruction
           mnemonic names and register names in disassembled
           instructions.  Multiple selections from the following may be
           specified as a comma separated string, and invalid options
           are ignored:

           "no-aliases"
               Print the 'raw' instruction mnemonic instead of some
               pseudo instruction mnemonic.  I.e., print 'daddu' or 'or'
               instead of 'move', 'sll' instead of 'nop', etc.

           "msa"
               Disassemble MSA instructions.

           "virt"
               Disassemble the virtualization ASE instructions.

           "xpa"
               Disassemble the eXtended Physical Address (XPA) ASE
               instructions.

           "gpr-names=ABI"
               Print GPR (general-purpose register) names as appropriate
               for the specified ABI.  By default, GPR names are
               selected according to the ABI of the binary being
               disassembled.

           "fpr-names=ABI"
               Print FPR (floating-point register) names as appropriate
               for the specified ABI.  By default, FPR numbers are
               printed rather than names.

           "cp0-names=ARCH"
               Print CP0 (system control coprocessor; coprocessor 0)
               register names as appropriate for the CPU or architecture
               specified by ARCH.  By default, CP0 register names are
               selected according to the architecture and CPU of the
               binary being disassembled.

           "hwr-names=ARCH"
               Print HWR (hardware register, used by the "rdhwr"
               instruction) names as appropriate for the CPU or
               architecture specified by ARCH.  By default, HWR names
               are selected according to the architecture and CPU of the
               binary being disassembled.

           "reg-names=ABI"
               Print GPR and FPR names as appropriate for the selected
               ABI.

           "reg-names=ARCH"
               Print CPU-specific register names (CP0 register and HWR
               names) as appropriate for the selected CPU or
               architecture.

           For any of the options listed above, ABI or ARCH may be
           specified as numeric to have numbers printed rather than
           names, for the selected types of registers.  You can list the
           available values of ABI and ARCH using the --help option.

           For VAX, you can specify function entry addresses with -M
           entry:0xf00ba.  You can use this multiple times to properly
           disassemble VAX binary files that don't contain symbol tables
           (like ROM dumps).  In these cases, the function entry mask
           would otherwise be decoded as VAX instructions, which would
           probably lead the rest of the function being wrongly
           disassembled.

       -p
       --private-headers
           Print information that is specific to the object file format.
           The exact information printed depends upon the object file
           format.  For some object file formats, no additional
           information is printed.

       -P options
       --private=options
           Print information that is specific to the object file format.
           The argument options is a comma separated list that depends
           on the format (the lists of options is displayed with the
           help).

           For XCOFF, the available options are:

           "header"
           "aout"
           "sections"
           "syms"
           "relocs"
           "lineno,"
           "loader"
           "except"
           "typchk"
           "traceback"
           "toc"
           "ldinfo"

           For PE, the available options are:

           "header"
           "sections"

           Not all object formats support this option.  In particular
           the ELF format does not use it.

       -r
       --reloc
           Print the relocation entries of the file.  If used with -d or
           -D, the relocations are printed interspersed with the
           disassembly.

       -R
       --dynamic-reloc
           Print the dynamic relocation entries of the file.  This is
           only meaningful for dynamic objects, such as certain types of
           shared libraries.  As for -r, if used with -d or -D, the
           relocations are printed interspersed with the disassembly.

       -s
       --full-contents
           Display the full contents of sections, often used in
           combination with -j to request specific sections.  By default
           all non-empty non-bss sections are displayed.  By default any
           compressed section will be displayed in its compressed form.
           In order to see the contents in a decompressed form add the
           -Z option to the command line.

       -S
       --source
           Display source code intermixed with disassembly, if possible.
           Implies -d.

       --show-all-symbols
           When disassembling, show all the symbols that match a given
           address, not just the first one.

       --source-comment[=txt]
           Like the -S option, but all source code lines are displayed
           with a prefix of txt.  Typically txt will be a comment string
           which can be used to distinguish the assembler code from the
           source code.  If txt is not provided then a default string of
           "# " (hash followed by a space), will be used.

       --prefix=prefix
           Specify prefix to add to the absolute paths when used with
           -S.

       --prefix-strip=level
           Indicate how many initial directory names to strip off the
           hardwired absolute paths. It has no effect without
           --prefix=prefix.

       --show-raw-insn
           When disassembling instructions, print the instruction in hex
           as well as in symbolic form.  This is the default except when
           --prefix-addresses is used.

       --no-show-raw-insn
           When disassembling instructions, do not print the instruction
           bytes.  This is the default when --prefix-addresses is used.

       --insn-width=width
           Display width bytes on a single line when disassembling
           instructions.

       --visualize-jumps[=color|=extended-color|=off]
           Visualize jumps that stay inside a function by drawing ASCII
           art between the start and target addresses.  The optional
           =color argument adds color to the output using simple
           terminal colors.  Alternatively the =extended-color argument
           will add color using 8bit colors, but these might not work on
           all terminals.

           If it is necessary to disable the visualize-jumps option
           after it has previously been enabled then use
           visualize-jumps=off.

       --disassembler-color=off
       --disassembler-color=terminal
       --disassembler-color=on|color|colour
       --disassembler-color=extened|extended-color|extened-colour
           Enables or disables the use of colored syntax highlighting in
           disassembly output.  The default behaviour is determined via
           a configure time option.  Note, not all architectures support
           colored syntax highlighting, and depending upon the terminal
           used, colored output may not actually be legible.

           The on argument adds colors using simple terminal colors.

           The terminal argument does the same, but only if the output
           device is a terminal.

           The extended-color argument is similar to the on argument,
           but it uses 8-bit colors.  These may not work on all
           terminals.

           The off argument disables colored disassembly.

       -W[lLiaprmfFsoORtUuTgAckK]
       --dwarf[=rawline,=decodedline,=info,=abbrev,=pubnames,=aranges,=macro,=frames,=frames-interp,=str,=str-offsets,=loc,=Ranges,=pubtypes,=trace_info,=trace_abbrev,=trace_aranges,=gdb_index,=addr,=cu_index,=links,=follow-links]
           Displays the contents of the DWARF debug sections in the
           file, if any are present.  Compressed debug sections are
           automatically decompressed (temporarily) before they are
           displayed.  If one or more of the optional letters or words
           follows the switch then only those type(s) of data will be
           dumped.  The letters and words refer to the following
           information:

           "a"
           "=abbrev"
               Displays the contents of the .debug_abbrev section.

           "A"
           "=addr"
               Displays the contents of the .debug_addr section.

           "c"
           "=cu_index"
               Displays the contents of the .debug_cu_index and/or
               .debug_tu_index sections.

           "f"
           "=frames"
               Display the raw contents of a .debug_frame section.

           "F"
           "=frames-interp"
               Display the interpreted contents of a .debug_frame
               section.

           "g"
           "=gdb_index"
               Displays the contents of the .gdb_index and/or
               .debug_names sections.

           "i"
           "=info"
               Displays the contents of the .debug_info section.  Note:
               the output from this option can also be restricted by the
               use of the --dwarf-depth and --dwarf-start options.

           "k"
           "=links"
               Displays the contents of the .gnu_debuglink,
               .gnu_debugaltlink and .debug_sup sections, if any of them
               are present.  Also displays any links to separate dwarf
               object files (dwo), if they are specified by the
               DW_AT_GNU_dwo_name or DW_AT_dwo_name attributes in the
               .debug_info section.

           "K"
           "=follow-links"
               Display the contents of any selected debug sections that
               are found in linked, separate debug info file(s).  This
               can result in multiple versions of the same debug section
               being displayed if it exists in more than one file.

               In addition, when displaying DWARF attributes, if a form
               is found that references the separate debug info file,
               then the referenced contents will also be displayed.

               Note - in some distributions this option is enabled by
               default.  It can be disabled via the N debug option.  The
               default can be chosen when configuring the binutils via
               the --enable-follow-debug-links=yes or
               --enable-follow-debug-links=no options.  If these are not
               used then the default is to enable the following of debug
               links.

               Note - if support for the debuginfod protocol was enabled
               when the binutils were built then this option will also
               include an attempt to contact any debuginfod servers
               mentioned in the DEBUGINFOD_URLS environment variable.
               This could take some time to resolve.  This behaviour can
               be disabled via the =do-not-use-debuginfod debug option.

           "N"
           "=no-follow-links"
               Disables the following of links to separate debug info
               files.

           "D"
           "=use-debuginfod"
               Enables contacting debuginfod servers if there is a need
               to follow debug links.  This is the default behaviour.

           "E"
           "=do-not-use-debuginfod"
               Disables contacting debuginfod servers when there is a
               need to follow debug links.

           "l"
           "=rawline"
               Displays the contents of the .debug_line section in a raw
               format.

           "L"
           "=decodedline"
               Displays the interpreted contents of the .debug_line
               section.

           "m"
           "=macro"
               Displays the contents of the .debug_macro and/or
               .debug_macinfo sections.

           "o"
           "=loc"
               Displays the contents of the .debug_loc and/or
               .debug_loclists sections.

           "O"
           "=str-offsets"
               Displays the contents of the .debug_str_offsets section.

           "p"
           "=pubnames"
               Displays the contents of the .debug_pubnames and/or
               .debug_gnu_pubnames sections.

           "r"
           "=aranges"
               Displays the contents of the .debug_aranges section.

           "R"
           "=Ranges"
               Displays the contents of the .debug_ranges and/or
               .debug_rnglists sections.

           "s"
           "=str"
               Displays the contents of the .debug_str, .debug_line_str
               and/or .debug_str_offsets sections.

           "t"
           "=pubtype"
               Displays the contents of the .debug_pubtypes and/or
               .debug_gnu_pubtypes sections.

           "T"
           "=trace_aranges"
               Displays the contents of the .trace_aranges section.

           "u"
           "=trace_abbrev"
               Displays the contents of the .trace_abbrev section.

           "U"
           "=trace_info"
               Displays the contents of the .trace_info section.

           Note: displaying the contents of .debug_static_funcs,
           .debug_static_vars and debug_weaknames sections is not
           currently supported.

       --dwarf-depth=n
           Limit the dump of the ".debug_info" section to n children.
           This is only useful with --debug-dump=info.  The default is
           to print all DIEs; the special value 0 for n will also have
           this effect.

           With a non-zero value for n, DIEs at or deeper than n levels
           will not be printed.  The range for n is zero-based.

       --dwarf-start=n
           Print only DIEs beginning with the DIE numbered n.  This is
           only useful with --debug-dump=info.

           If specified, this option will suppress printing of any
           header information and all DIEs before the DIE numbered n.
           Only siblings and children of the specified DIE will be
           printed.

           This can be used in conjunction with --dwarf-depth.

       --dwarf-check
           Enable additional checks for consistency of Dwarf
           information.

       --ctf[=section]
           Display the contents of the specified CTF section.  CTF
           sections themselves contain many subsections, all of which
           are displayed in order.

           By default, display the name of the section named .ctf, which
           is the name emitted by ld.

       --ctf-parent=member
           If the CTF section contains ambiguously-defined types, it
           will consist of an archive of many CTF dictionaries, all
           inheriting from one dictionary containing unambiguous types.
           This member is by default named .ctf, like the section
           containing it, but it is possible to change this name using
           the "ctf_link_set_memb_name_changer" function at link time.
           When looking at CTF archives that have been created by a
           linker that uses the name changer to rename the parent
           archive member, --ctf-parent can be used to specify the name
           used for the parent.

       --sframe[=section]
           Display the contents of the specified SFrame section.

           By default, display the name of the section named .sframe,
           which is the name emitted by ld.

       -G
       --stabs
           Display the full contents of any sections requested.  Display
           the contents of the .stab and .stab.index and .stab.excl
           sections from an ELF file.  This is only useful on systems
           (such as Solaris 2.0) in which ".stab" debugging symbol-table
           entries are carried in an ELF section.  In most other file
           formats, debugging symbol-table entries are interleaved with
           linkage symbols, and are visible in the --syms output.

       --start-address=address
           Start displaying data at the specified address.  This affects
           the output of the -d, -r and -s options.

       --stop-address=address
           Stop displaying data at the specified address.  This affects
           the output of the -d, -r and -s options.

       -t
       --syms
           Print the symbol table entries of the file.  This is similar
           to the information provided by the nm program, although the
           display format is different.  The format of the output
           depends upon the format of the file being dumped, but there
           are two main types.  One looks like this:

                   [  4](sec  3)(fl 0x00)(ty   0)(scl   3) (nx 1) 0x00000000 .bss
                   [  6](sec  1)(fl 0x00)(ty   0)(scl   2) (nx 0) 0x00000000 fred

           where the number inside the square brackets is the number of
           the entry in the symbol table, the sec number is the section
           number, the fl value are the symbol's flag bits, the ty
           number is the symbol's type, the scl number is the symbol's
           storage class and the nx value is the number of auxiliary
           entries associated with the symbol.  The last two fields are
           the symbol's value and its name.

           The other common output format, usually seen with ELF based
           files, looks like this:

                   00000000 l    d  .bss   00000000 .bss
                   00000000 g       .text  00000000 fred

           Here the first number is the symbol's value (sometimes
           referred to as its address).  The next field is actually a
           set of characters and spaces indicating the flag bits that
           are set on the symbol.  These characters are described below.
           Next is the section with which the symbol is associated or
           *ABS* if the section is absolute (ie not connected with any
           section), or *UND* if the section is referenced in the file
           being dumped, but not defined there.

           After the section name comes another field, a number, which
           for common symbols is the alignment and for other symbol is
           the size.  Finally the symbol's name is displayed.

           The flag characters are divided into 7 groups as follows:

           "l"
           "g"
           "u"
           "!" The symbol is a local (l), global (g), unique global (u),
               neither global nor local (a space) or both global and
               local (!).  A symbol can be neither local or global for a
               variety of reasons, e.g., because it is used for
               debugging, but it is probably an indication of a bug if
               it is ever both local and global.  Unique global symbols
               are a GNU extension to the standard set of ELF symbol
               bindings.  For such a symbol the dynamic linker will make
               sure that in the entire process there is just one symbol
               with this name and type in use.

           "w" The symbol is weak (w) or strong (a space).

           "C" The symbol denotes a constructor (C) or an ordinary
               symbol (a space).

           "W" The symbol is a warning (W) or a normal symbol (a space).
               A warning symbol's name is a message to be displayed if
               the symbol following the warning symbol is ever
               referenced.

           "I"
           "i" The symbol is an indirect reference to another symbol
               (I), a function to be evaluated during reloc processing
               (i) or a normal symbol (a space).

           "d"
           "D" The symbol is a debugging symbol (d) or a dynamic symbol
               (D) or a normal symbol (a space).

           "F"
           "f"
           "O" The symbol is the name of a function (F) or a file (f) or
               an object (O) or just a normal symbol (a space).

       -T
       --dynamic-syms
           Print the dynamic symbol table entries of the file.  This is
           only meaningful for dynamic objects, such as certain types of
           shared libraries.  This is similar to the information
           provided by the nm program when given the -D (--dynamic)
           option.

           The output format is similar to that produced by the --syms
           option, except that an extra field is inserted before the
           symbol's name, giving the version information associated with
           the symbol.  If the version is the default version to be used
           when resolving unversioned references to the symbol then it's
           displayed as is, otherwise it's put into parentheses.

       --special-syms
           When displaying symbols include those which the target
           considers to be special in some way and which would not
           normally be of interest to the user.

       -U [d|i|l|e|x|h]
       --unicode=[default|invalid|locale|escape|hex|highlight]
           Controls the display of UTF-8 encoded multibyte characters in
           strings.  The default (--unicode=default) is to give them no
           special treatment.  The --unicode=locale option displays the
           sequence in the current locale, which may or may not support
           them.  The options --unicode=hex and --unicode=invalid
           display them as hex byte sequences enclosed by either angle
           brackets or curly braces.

           The --unicode=escape option displays them as escape sequences
           (\uxxxx) and the --unicode=highlight option displays them as
           escape sequences highlighted in red (if supported by the
           output device).  The colouring is intended to draw attention
           to the presence of unicode sequences where they might not be
           expected.

       -V
       --version
           Print the version number of objdump and exit.

       -x
       --all-headers
           Display all available header information, including the
           symbol table and relocation entries.  Using -x is equivalent
           to specifying all of -a -f -h -p -r -t.

       -w
       --wide
           Format some lines for output devices that have more than 80
           columns.  Also do not truncate symbol names when they are
           displayed.

       -z
       --disassemble-zeroes
           Normally the disassembly output will skip blocks of zeroes.
           This option directs the disassembler to disassemble those
           blocks, just like any other data.

       -Z
       --decompress
           The -Z option is meant to be used in conunction with the -s
           option.  It instructs objdump to decompress any compressed
           sections before displaying their contents.

       @file
           Read command-line options from file.  The options read are
           inserted in place of the original @file option.  If file does
           not exist, or cannot be read, then the option will be treated
           literally, and not removed.

           Options in file are separated by whitespace.  A whitespace
           character may be included in an option by surrounding the
           entire option in either single or double quotes.  Any
           character (including a backslash) may be included by
           prefixing the character to be included with a backslash.  The
           file may itself contain additional @file options; any such
           options will be processed recursively.
SEE ALSO
       nm(1), readelf(1), and the Info entries for binutils.
COPYRIGHT
       Copyright (c) 1991-2024 Free Software Foundation, Inc.

       Permission is granted to copy, distribute and/or modify this
       document under the terms of the GNU Free Documentation License,
       Version 1.3 or any later version published by the Free Software
       Foundation; with no Invariant Sections, with no Front-Cover
       Texts, and with no Back-Cover Texts.  A copy of the license is
       included in the section entitled "GNU Free Documentation License".
COLOPHON
       This page is part of the binutils (a collection of tools for
       working with executable binaries) project.  Information about the
       project can be found at http://www.gnu.org/software/binutils/.
       If you have a bug report for this manual page, see
       http://sourceware.org/bugzilla/enter_bug.cgi?product=binutils.
       This page was obtained from the tarball binutils-2.42.tar.gz
       fetched from https://ftp.gnu.org/gnu/binutils/ on 2024-06-14.
       If you discover any rendering problems in this HTML version of
       the page, or you believe there is a better or more up-to-date
       source for the page, or you have corrections or improvements to
       the information in this COLOPHON (which is not part of the
       original manual page), send a mail to man-pages@man7.org

binutils-2.42                  2024-06-14                     OBJDUMP(1)
\end{lstlisting}
}}
\endinput  %  ==  ==  ==  ==  ==  ==  ==  ==  ==

\subsection{\refObjdump: Display Information From Object Files}

{\tiny{
\begin{lstlisting}[language=bash]
NAME
       objdump - display information from object files
SYNOPSIS
       objdump [-a|--archive-headers]
               [-b bfdname|--target=bfdname]
               [-C|--demangle[=style] ]
               [-d|--disassemble[=symbol]]
               [-D|--disassemble-all]
               [-z|--disassemble-zeroes]
               [-EB|-EL|--endian={big | little }]
               [-f|--file-headers]
               [-F|--file-offsets]
               [--file-start-context]
               [-g|--debugging]
               [-e|--debugging-tags]
               [-h|--section-headers|--headers]
               [-i|--info]
               [-j section|--section=section]
               [-l|--line-numbers]
               [-S|--source]
               [--source-comment[=text]]
               [-m machine|--architecture=machine]
               [-M options|--disassembler-options=options]
               [-p|--private-headers]
               [-P options|--private=options]
               [-r|--reloc]
               [-R|--dynamic-reloc]
               [-s|--full-contents]
               [-Z|--decompress]
               [-W[lLiaprmfFsoORtUuTgAck]|
                --dwarf[=rawline,=decodedline,=info,=abbrev,=pubnames,=aranges,=macro,=frames,=frames-interp,=str,=str-offsets,=loc,=Ranges,=pubtypes,=trace_info,=trace_abbrev,=trace_aranges,=gdb_index,=addr,=cu_index,=links]]
               [-WK|--dwarf=follow-links]
               [-WN|--dwarf=no-follow-links]
               [-wD|--dwarf=use-debuginfod]
               [-wE|--dwarf=do-not-use-debuginfod]
               [-L|--process-links]
               [--ctf=section]
               [--sframe=section]
               [-G|--stabs]
               [-t|--syms]
               [-T|--dynamic-syms]
               [-x|--all-headers]
               [-w|--wide]
               [--start-address=address]
               [--stop-address=address]
               [--no-addresses]
               [--prefix-addresses]
               [--[no-]show-raw-insn]
               [--adjust-vma=offset]
               [--show-all-symbols]
               [--dwarf-depth=n]
               [--dwarf-start=n]
               [--ctf-parent=section]
               [--no-recurse-limit|--recurse-limit]
               [--special-syms]
               [--prefix=prefix]
               [--prefix-strip=level]
               [--insn-width=width]
               [--visualize-jumps[=color|=extended-color|=off]
               [--disassembler-color=[off|terminal|on|extended]
               [-U method] [--unicode=method]
               [-V|--version]
               [-H|--help]
               objfile...
DESCRIPTION
       objdump displays information about one or more object files.  The
       options control what particular information to display.  This
       information is mostly useful to programmers who are working on
       the compilation tools, as opposed to programmers who just want
       their program to compile and work.

       objfile... are the object files to be examined.  When you specify
       archives, objdump shows information on each of the member object
       files.
OPTIONS
       The long and short forms of options, shown here as alternatives,
       are equivalent.  At least one option from the list
       -a,-d,-D,-e,-f,-g,-G,-h,-H,-p,-P,-r,-R,-s,-S,-t,-T,-V,-x must be
       given.

       -a
       --archive-header
           If any of the objfile files are archives, display the archive
           header information (in a format similar to ls -l).  Besides
           the information you could list with ar tv, objdump -a shows
           the object file format of each archive member.

       --adjust-vma=offset
           When dumping information, first add offset to all the section
           addresses.  This is useful if the section addresses do not
           correspond to the symbol table, which can happen when putting
           sections at particular addresses when using a format which
           can not represent section addresses, such as a.out.

       -b bfdname
       --target=bfdname
           Specify that the object-code format for the object files is
           bfdname.  This option may not be necessary; objdump can
           automatically recognize many formats.

           For example,

                   objdump -b oasys -m vax -h fu.o

           displays summary information from the section headers (-h) of
           fu.o, which is explicitly identified (-m) as a VAX object
           file in the format produced by Oasys compilers.  You can list
           the formats available with the -i option.

       -C
       --demangle[=style]
           Decode (demangle) low-level symbol names into user-level
           names.  Besides removing any initial underscore prepended by
           the system, this makes C++ function names readable.
           Different compilers have different mangling styles. The
           optional demangling style argument can be used to choose an
           appropriate demangling style for your compiler.

       --recurse-limit
       --no-recurse-limit
       --recursion-limit
       --no-recursion-limit
           Enables or disables a limit on the amount of recursion
           performed whilst demangling strings.  Since the name mangling
           formats allow for an infinite level of recursion it is
           possible to create strings whose decoding will exhaust the
           amount of stack space available on the host machine,
           triggering a memory fault.  The limit tries to prevent this
           from happening by restricting recursion to 2048 levels of
           nesting.

           The default is for this limit to be enabled, but disabling it
           may be necessary in order to demangle truly complicated
           names.  Note however that if the recursion limit is disabled
           then stack exhaustion is possible and any bug reports about
           such an event will be rejected.

       -g
       --debugging
           Display debugging information.  This attempts to parse STABS
           debugging format information stored in the file and print it
           out using a C like syntax.  If no STABS debugging was found
           this option falls back on the -W option to print any DWARF
           information in the file.

       -e
       --debugging-tags
           Like -g, but the information is generated in a format
           compatible with ctags tool.

       -d
       --disassemble
       --disassemble=symbol
           Display the assembler mnemonics for the machine instructions
           from the input file.  This option only disassembles those
           sections which are expected to contain instructions.  If the
           optional symbol argument is given, then display the assembler
           mnemonics starting at symbol.  If symbol is a function name
           then disassembly will stop at the end of the function,
           otherwise it will stop when the next symbol is encountered.
           If there are no matches for symbol then nothing will be
           displayed.

           Note if the --dwarf=follow-links option is enabled then any
           symbol tables in linked debug info files will be read in and
           used when disassembling.

       -D
       --disassemble-all
           Like -d, but disassemble the contents of all non-empty non-
           bss sections, not just those expected to contain
           instructions.  -j may be used to select specific sections.

           This option also has a subtle effect on the disassembly of
           instructions in code sections.  When option -d is in effect
           objdump will assume that any symbols present in a code
           section occur on the boundary between instructions and it
           will refuse to disassemble across such a boundary.  When
           option -D is in effect however this assumption is supressed.
           This means that it is possible for the output of -d and -D to
           differ if, for example, data is stored in code sections.

           If the target is an ARM architecture this switch also has the
           effect of forcing the disassembler to decode pieces of data
           found in code sections as if they were instructions.

           Note if the --dwarf=follow-links option is enabled then any
           symbol tables in linked debug info files will be read in and
           used when disassembling.

       --no-addresses
           When disassembling, don't print addresses on each line or for
           symbols and relocation offsets.  In combination with
           --no-show-raw-insn this may be useful for comparing compiler
           output.

       --prefix-addresses
           When disassembling, print the complete address on each line.
           This is the older disassembly format.

       -EB
       -EL
       --endian={big|little}
           Specify the endianness of the object files.  This only
           affects disassembly.  This can be useful when disassembling a
           file format which does not describe endianness information,
           such as S-records.

       -f
       --file-headers
           Display summary information from the overall header of each
           of the objfile files.

       -F
       --file-offsets
           When disassembling sections, whenever a symbol is displayed,
           also display the file offset of the region of data that is
           about to be dumped.  If zeroes are being skipped, then when
           disassembly resumes, tell the user how many zeroes were
           skipped and the file offset of the location from where the
           disassembly resumes.  When dumping sections, display the file
           offset of the location from where the dump starts.

       --file-start-context
           Specify that when displaying interlisted source
           code/disassembly (assumes -S) from a file that has not yet
           been displayed, extend the context to the start of the file.

       -h
       --section-headers
       --headers
           Display summary information from the section headers of the
           object file.

           File segments may be relocated to nonstandard addresses, for
           example by using the -Ttext, -Tdata, or -Tbss options to ld.
           However, some object file formats, such as a.out, do not
           store the starting address of the file segments.  In those
           situations, although ld relocates the sections correctly,
           using objdump -h to list the file section headers cannot show
           the correct addresses.  Instead, it shows the usual
           addresses, which are implicit for the target.

           Note, in some cases it is possible for a section to have both
           the READONLY and the NOREAD attributes set.  In such cases
           the NOREAD attribute takes precedence, but objdump will
           report both since the exact setting of the flag bits might be
           important.

       -H
       --help
           Print a summary of the options to objdump and exit.

       -i
       --info
           Display a list showing all architectures and object formats
           available for specification with -b or -m.

       -j name
       --section=name
           Display information for section name.  This option may be
           specified multiple times.

       -L
       --process-links
           Display the contents of non-debug sections found in separate
           debuginfo files that are linked to the main file.  This
           option automatically implies the -WK option, and only
           sections requested by other command line options will be
           displayed.

       -l
       --line-numbers
           Label the display (using debugging information) with the
           filename and source line numbers corresponding to the object
           code or relocs shown.  Only useful with -d, -D, or -r.

       -m machine
       --architecture=machine
           Specify the architecture to use when disassembling object
           files.  This can be useful when disassembling object files
           which do not describe architecture information, such as
           S-records.  You can list the available architectures with the
           -i option.

           For most architectures it is possible to supply an
           architecture name and a machine name, separated by a colon.
           For example foo:bar would refer to the bar machine type in
           the foo architecture.  This can be helpful if objdump has
           been configured to support multiple architectures.

           If the target is an ARM architecture then this switch has an
           additional effect.  It restricts the disassembly to only
           those instructions supported by the architecture specified by
           machine.  If it is necessary to use this switch because the
           input file does not contain any architecture information, but
           it is also desired to disassemble all the instructions use
           -marm.

       -M options
       --disassembler-options=options
           Pass target specific information to the disassembler.  Only
           supported on some targets.  If it is necessary to specify
           more than one disassembler option then multiple -M options
           can be used or can be placed together into a comma separated
           list.

           For ARC, dsp controls the printing of DSP instructions, spfp
           selects the printing of FPX single precision FP instructions,
           dpfp selects the printing of FPX double precision FP
           instructions, quarkse_em selects the printing of special
           QuarkSE-EM instructions, fpuda selects the printing of double
           precision assist instructions, fpus selects the printing of
           FPU single precision FP instructions, while fpud selects the
           printing of FPU double precision FP instructions.
           Additionally, one can choose to have all the immediates
           printed in hexadecimal using hex.  By default, the short
           immediates are printed using the decimal representation,
           while the long immediate values are printed as hexadecimal.

           cpu=... allows one to enforce a particular ISA when
           disassembling instructions, overriding the -m value or
           whatever is in the ELF file.  This might be useful to select
           ARC EM or HS ISA, because architecture is same for those and
           disassembler relies on private ELF header data to decide if
           code is for EM or HS.  This option might be specified
           multiple times - only the latest value will be used.  Valid
           values are same as for the assembler -mcpu=... option.

           If the target is an ARM architecture then this switch can be
           used to select which register name set is used during
           disassembler.  Specifying -M reg-names-std (the default) will
           select the register names as used in ARM's instruction set
           documentation, but with register 13 called 'sp', register 14
           called 'lr' and register 15 called 'pc'.  Specifying -M reg-
           names-apcs will select the name set used by the ARM Procedure
           Call Standard, whilst specifying -M reg-names-raw will just
           use r followed by the register number.

           There are also two variants on the APCS register naming
           scheme enabled by -M reg-names-atpcs and -M reg-names-
           special-atpcs which use the ARM/Thumb Procedure Call Standard
           naming conventions.  (Either with the normal register names
           or the special register names).

           This option can also be used for ARM architectures to force
           the disassembler to interpret all instructions as Thumb
           instructions by using the switch
           --disassembler-options=force-thumb.  This can be useful when
           attempting to disassemble thumb code produced by other
           compilers.

           For AArch64 targets this switch can be used to set whether
           instructions are disassembled as the most general instruction
           using the -M no-aliases option or whether instruction notes
           should be generated as comments in the disasssembly using -M
           notes.

           For the x86, some of the options duplicate functions of the
           -m switch, but allow finer grained control.

           "x86-64"
           "i386"
           "i8086"
               Select disassembly for the given architecture.

           "intel"
           "att"
               Select between intel syntax mode and AT&T syntax mode.

           "amd64"
           "intel64"
               Select between AMD64 ISA and Intel64 ISA.

           "intel-mnemonic"
           "att-mnemonic"
               Select between intel mnemonic mode and AT&T mnemonic
               mode.  Note: "intel-mnemonic" implies "intel" and
               "att-mnemonic" implies "att".

           "addr64"
           "addr32"
           "addr16"
           "data32"
           "data16"
               Specify the default address size and operand size.  These
               five options will be overridden if "x86-64", "i386" or
               "i8086" appear later in the option string.

           "suffix"
               When in AT&T mode and also for a limited set of
               instructions when in Intel mode, instructs the
               disassembler to print a mnemonic suffix even when the
               suffix could be inferred by the operands or, for certain
               instructions, the execution mode's defaults.

           For PowerPC, the -M argument raw selects disasssembly of
           hardware insns rather than aliases.  For example, you will
           see "rlwinm" rather than "clrlwi", and "addi" rather than
           "li".  All of the -m arguments for gas that select a CPU are
           supported.  These are: 403, 405, 440, 464, 476, 601, 603,
           604, 620, 7400, 7410, 7450, 7455, 750cl, 821, 850, 860, a2,
           booke, booke32, cell, com, e200z2, e200z4, e300, e500,
           e500mc, e500mc64, e500x2, e5500, e6500, efs, power4, power5,
           power6, power7, power8, power9, power10, ppc, ppc32, ppc64,
           ppc64bridge, ppcps, pwr, pwr2, pwr4, pwr5, pwr5x, pwr6, pwr7,
           pwr8, pwr9, pwr10, pwrx, titan, vle, and future.  32 and 64
           modify the default or a prior CPU selection, disabling and
           enabling 64-bit insns respectively.  In addition, altivec,
           any, lsp, htm, vsx, spe and  spe2 add capabilities to a
           previous or later CPU selection.  any will disassemble any
           opcode known to binutils, but in cases where an opcode has
           two different meanings or different arguments, you may not
           see the disassembly you expect.  If you disassemble without
           giving a CPU selection, a default will be chosen from
           information gleaned by BFD from the object files headers, but
           the result again may not be as you expect.

           For MIPS, this option controls the printing of instruction
           mnemonic names and register names in disassembled
           instructions.  Multiple selections from the following may be
           specified as a comma separated string, and invalid options
           are ignored:

           "no-aliases"
               Print the 'raw' instruction mnemonic instead of some
               pseudo instruction mnemonic.  I.e., print 'daddu' or 'or'
               instead of 'move', 'sll' instead of 'nop', etc.

           "msa"
               Disassemble MSA instructions.

           "virt"
               Disassemble the virtualization ASE instructions.

           "xpa"
               Disassemble the eXtended Physical Address (XPA) ASE
               instructions.

           "gpr-names=ABI"
               Print GPR (general-purpose register) names as appropriate
               for the specified ABI.  By default, GPR names are
               selected according to the ABI of the binary being
               disassembled.

           "fpr-names=ABI"
               Print FPR (floating-point register) names as appropriate
               for the specified ABI.  By default, FPR numbers are
               printed rather than names.

           "cp0-names=ARCH"
               Print CP0 (system control coprocessor; coprocessor 0)
               register names as appropriate for the CPU or architecture
               specified by ARCH.  By default, CP0 register names are
               selected according to the architecture and CPU of the
               binary being disassembled.

           "hwr-names=ARCH"
               Print HWR (hardware register, used by the "rdhwr"
               instruction) names as appropriate for the CPU or
               architecture specified by ARCH.  By default, HWR names
               are selected according to the architecture and CPU of the
               binary being disassembled.

           "reg-names=ABI"
               Print GPR and FPR names as appropriate for the selected
               ABI.

           "reg-names=ARCH"
               Print CPU-specific register names (CP0 register and HWR
               names) as appropriate for the selected CPU or
               architecture.

           For any of the options listed above, ABI or ARCH may be
           specified as numeric to have numbers printed rather than
           names, for the selected types of registers.  You can list the
           available values of ABI and ARCH using the --help option.

           For VAX, you can specify function entry addresses with -M
           entry:0xf00ba.  You can use this multiple times to properly
           disassemble VAX binary files that don't contain symbol tables
           (like ROM dumps).  In these cases, the function entry mask
           would otherwise be decoded as VAX instructions, which would
           probably lead the rest of the function being wrongly
           disassembled.

       -p
       --private-headers
           Print information that is specific to the object file format.
           The exact information printed depends upon the object file
           format.  For some object file formats, no additional
           information is printed.

       -P options
       --private=options
           Print information that is specific to the object file format.
           The argument options is a comma separated list that depends
           on the format (the lists of options is displayed with the
           help).

           For XCOFF, the available options are:

           "header"
           "aout"
           "sections"
           "syms"
           "relocs"
           "lineno,"
           "loader"
           "except"
           "typchk"
           "traceback"
           "toc"
           "ldinfo"

           For PE, the available options are:

           "header"
           "sections"

           Not all object formats support this option.  In particular
           the ELF format does not use it.

       -r
       --reloc
           Print the relocation entries of the file.  If used with -d or
           -D, the relocations are printed interspersed with the
           disassembly.

       -R
       --dynamic-reloc
           Print the dynamic relocation entries of the file.  This is
           only meaningful for dynamic objects, such as certain types of
           shared libraries.  As for -r, if used with -d or -D, the
           relocations are printed interspersed with the disassembly.

       -s
       --full-contents
           Display the full contents of sections, often used in
           combination with -j to request specific sections.  By default
           all non-empty non-bss sections are displayed.  By default any
           compressed section will be displayed in its compressed form.
           In order to see the contents in a decompressed form add the
           -Z option to the command line.

       -S
       --source
           Display source code intermixed with disassembly, if possible.
           Implies -d.

       --show-all-symbols
           When disassembling, show all the symbols that match a given
           address, not just the first one.

       --source-comment[=txt]
           Like the -S option, but all source code lines are displayed
           with a prefix of txt.  Typically txt will be a comment string
           which can be used to distinguish the assembler code from the
           source code.  If txt is not provided then a default string of
           "# " (hash followed by a space), will be used.

       --prefix=prefix
           Specify prefix to add to the absolute paths when used with
           -S.

       --prefix-strip=level
           Indicate how many initial directory names to strip off the
           hardwired absolute paths. It has no effect without
           --prefix=prefix.

       --show-raw-insn
           When disassembling instructions, print the instruction in hex
           as well as in symbolic form.  This is the default except when
           --prefix-addresses is used.

       --no-show-raw-insn
           When disassembling instructions, do not print the instruction
           bytes.  This is the default when --prefix-addresses is used.

       --insn-width=width
           Display width bytes on a single line when disassembling
           instructions.

       --visualize-jumps[=color|=extended-color|=off]
           Visualize jumps that stay inside a function by drawing ASCII
           art between the start and target addresses.  The optional
           =color argument adds color to the output using simple
           terminal colors.  Alternatively the =extended-color argument
           will add color using 8bit colors, but these might not work on
           all terminals.

           If it is necessary to disable the visualize-jumps option
           after it has previously been enabled then use
           visualize-jumps=off.

       --disassembler-color=off
       --disassembler-color=terminal
       --disassembler-color=on|color|colour
       --disassembler-color=extened|extended-color|extened-colour
           Enables or disables the use of colored syntax highlighting in
           disassembly output.  The default behaviour is determined via
           a configure time option.  Note, not all architectures support
           colored syntax highlighting, and depending upon the terminal
           used, colored output may not actually be legible.

           The on argument adds colors using simple terminal colors.

           The terminal argument does the same, but only if the output
           device is a terminal.

           The extended-color argument is similar to the on argument,
           but it uses 8-bit colors.  These may not work on all
           terminals.

           The off argument disables colored disassembly.

       -W[lLiaprmfFsoORtUuTgAckK]
       --dwarf[=rawline,=decodedline,=info,=abbrev,=pubnames,=aranges,=macro,=frames,=frames-interp,=str,=str-offsets,=loc,=Ranges,=pubtypes,=trace_info,=trace_abbrev,=trace_aranges,=gdb_index,=addr,=cu_index,=links,=follow-links]
           Displays the contents of the DWARF debug sections in the
           file, if any are present.  Compressed debug sections are
           automatically decompressed (temporarily) before they are
           displayed.  If one or more of the optional letters or words
           follows the switch then only those type(s) of data will be
           dumped.  The letters and words refer to the following
           information:

           "a"
           "=abbrev"
               Displays the contents of the .debug_abbrev section.

           "A"
           "=addr"
               Displays the contents of the .debug_addr section.

           "c"
           "=cu_index"
               Displays the contents of the .debug_cu_index and/or
               .debug_tu_index sections.

           "f"
           "=frames"
               Display the raw contents of a .debug_frame section.

           "F"
           "=frames-interp"
               Display the interpreted contents of a .debug_frame
               section.

           "g"
           "=gdb_index"
               Displays the contents of the .gdb_index and/or
               .debug_names sections.

           "i"
           "=info"
               Displays the contents of the .debug_info section.  Note:
               the output from this option can also be restricted by the
               use of the --dwarf-depth and --dwarf-start options.

           "k"
           "=links"
               Displays the contents of the .gnu_debuglink,
               .gnu_debugaltlink and .debug_sup sections, if any of them
               are present.  Also displays any links to separate dwarf
               object files (dwo), if they are specified by the
               DW_AT_GNU_dwo_name or DW_AT_dwo_name attributes in the
               .debug_info section.

           "K"
           "=follow-links"
               Display the contents of any selected debug sections that
               are found in linked, separate debug info file(s).  This
               can result in multiple versions of the same debug section
               being displayed if it exists in more than one file.

               In addition, when displaying DWARF attributes, if a form
               is found that references the separate debug info file,
               then the referenced contents will also be displayed.

               Note - in some distributions this option is enabled by
               default.  It can be disabled via the N debug option.  The
               default can be chosen when configuring the binutils via
               the --enable-follow-debug-links=yes or
               --enable-follow-debug-links=no options.  If these are not
               used then the default is to enable the following of debug
               links.

               Note - if support for the debuginfod protocol was enabled
               when the binutils were built then this option will also
               include an attempt to contact any debuginfod servers
               mentioned in the DEBUGINFOD_URLS environment variable.
               This could take some time to resolve.  This behaviour can
               be disabled via the =do-not-use-debuginfod debug option.

           "N"
           "=no-follow-links"
               Disables the following of links to separate debug info
               files.

           "D"
           "=use-debuginfod"
               Enables contacting debuginfod servers if there is a need
               to follow debug links.  This is the default behaviour.

           "E"
           "=do-not-use-debuginfod"
               Disables contacting debuginfod servers when there is a
               need to follow debug links.

           "l"
           "=rawline"
               Displays the contents of the .debug_line section in a raw
               format.

           "L"
           "=decodedline"
               Displays the interpreted contents of the .debug_line
               section.

           "m"
           "=macro"
               Displays the contents of the .debug_macro and/or
               .debug_macinfo sections.

           "o"
           "=loc"
               Displays the contents of the .debug_loc and/or
               .debug_loclists sections.

           "O"
           "=str-offsets"
               Displays the contents of the .debug_str_offsets section.

           "p"
           "=pubnames"
               Displays the contents of the .debug_pubnames and/or
               .debug_gnu_pubnames sections.

           "r"
           "=aranges"
               Displays the contents of the .debug_aranges section.

           "R"
           "=Ranges"
               Displays the contents of the .debug_ranges and/or
               .debug_rnglists sections.

           "s"
           "=str"
               Displays the contents of the .debug_str, .debug_line_str
               and/or .debug_str_offsets sections.

           "t"
           "=pubtype"
               Displays the contents of the .debug_pubtypes and/or
               .debug_gnu_pubtypes sections.

           "T"
           "=trace_aranges"
               Displays the contents of the .trace_aranges section.

           "u"
           "=trace_abbrev"
               Displays the contents of the .trace_abbrev section.

           "U"
           "=trace_info"
               Displays the contents of the .trace_info section.

           Note: displaying the contents of .debug_static_funcs,
           .debug_static_vars and debug_weaknames sections is not
           currently supported.

       --dwarf-depth=n
           Limit the dump of the ".debug_info" section to n children.
           This is only useful with --debug-dump=info.  The default is
           to print all DIEs; the special value 0 for n will also have
           this effect.

           With a non-zero value for n, DIEs at or deeper than n levels
           will not be printed.  The range for n is zero-based.

       --dwarf-start=n
           Print only DIEs beginning with the DIE numbered n.  This is
           only useful with --debug-dump=info.

           If specified, this option will suppress printing of any
           header information and all DIEs before the DIE numbered n.
           Only siblings and children of the specified DIE will be
           printed.

           This can be used in conjunction with --dwarf-depth.

       --dwarf-check
           Enable additional checks for consistency of Dwarf
           information.

       --ctf[=section]
           Display the contents of the specified CTF section.  CTF
           sections themselves contain many subsections, all of which
           are displayed in order.

           By default, display the name of the section named .ctf, which
           is the name emitted by ld.

       --ctf-parent=member
           If the CTF section contains ambiguously-defined types, it
           will consist of an archive of many CTF dictionaries, all
           inheriting from one dictionary containing unambiguous types.
           This member is by default named .ctf, like the section
           containing it, but it is possible to change this name using
           the "ctf_link_set_memb_name_changer" function at link time.
           When looking at CTF archives that have been created by a
           linker that uses the name changer to rename the parent
           archive member, --ctf-parent can be used to specify the name
           used for the parent.

       --sframe[=section]
           Display the contents of the specified SFrame section.

           By default, display the name of the section named .sframe,
           which is the name emitted by ld.

       -G
       --stabs
           Display the full contents of any sections requested.  Display
           the contents of the .stab and .stab.index and .stab.excl
           sections from an ELF file.  This is only useful on systems
           (such as Solaris 2.0) in which ".stab" debugging symbol-table
           entries are carried in an ELF section.  In most other file
           formats, debugging symbol-table entries are interleaved with
           linkage symbols, and are visible in the --syms output.

       --start-address=address
           Start displaying data at the specified address.  This affects
           the output of the -d, -r and -s options.

       --stop-address=address
           Stop displaying data at the specified address.  This affects
           the output of the -d, -r and -s options.

       -t
       --syms
           Print the symbol table entries of the file.  This is similar
           to the information provided by the nm program, although the
           display format is different.  The format of the output
           depends upon the format of the file being dumped, but there
           are two main types.  One looks like this:

                   [  4](sec  3)(fl 0x00)(ty   0)(scl   3) (nx 1) 0x00000000 .bss
                   [  6](sec  1)(fl 0x00)(ty   0)(scl   2) (nx 0) 0x00000000 fred

           where the number inside the square brackets is the number of
           the entry in the symbol table, the sec number is the section
           number, the fl value are the symbol's flag bits, the ty
           number is the symbol's type, the scl number is the symbol's
           storage class and the nx value is the number of auxiliary
           entries associated with the symbol.  The last two fields are
           the symbol's value and its name.

           The other common output format, usually seen with ELF based
           files, looks like this:

                   00000000 l    d  .bss   00000000 .bss
                   00000000 g       .text  00000000 fred

           Here the first number is the symbol's value (sometimes
           referred to as its address).  The next field is actually a
           set of characters and spaces indicating the flag bits that
           are set on the symbol.  These characters are described below.
           Next is the section with which the symbol is associated or
           *ABS* if the section is absolute (ie not connected with any
           section), or *UND* if the section is referenced in the file
           being dumped, but not defined there.

           After the section name comes another field, a number, which
           for common symbols is the alignment and for other symbol is
           the size.  Finally the symbol's name is displayed.

           The flag characters are divided into 7 groups as follows:

           "l"
           "g"
           "u"
           "!" The symbol is a local (l), global (g), unique global (u),
               neither global nor local (a space) or both global and
               local (!).  A symbol can be neither local or global for a
               variety of reasons, e.g., because it is used for
               debugging, but it is probably an indication of a bug if
               it is ever both local and global.  Unique global symbols
               are a GNU extension to the standard set of ELF symbol
               bindings.  For such a symbol the dynamic linker will make
               sure that in the entire process there is just one symbol
               with this name and type in use.

           "w" The symbol is weak (w) or strong (a space).

           "C" The symbol denotes a constructor (C) or an ordinary
               symbol (a space).

           "W" The symbol is a warning (W) or a normal symbol (a space).
               A warning symbol's name is a message to be displayed if
               the symbol following the warning symbol is ever
               referenced.

           "I"
           "i" The symbol is an indirect reference to another symbol
               (I), a function to be evaluated during reloc processing
               (i) or a normal symbol (a space).

           "d"
           "D" The symbol is a debugging symbol (d) or a dynamic symbol
               (D) or a normal symbol (a space).

           "F"
           "f"
           "O" The symbol is the name of a function (F) or a file (f) or
               an object (O) or just a normal symbol (a space).

       -T
       --dynamic-syms
           Print the dynamic symbol table entries of the file.  This is
           only meaningful for dynamic objects, such as certain types of
           shared libraries.  This is similar to the information
           provided by the nm program when given the -D (--dynamic)
           option.

           The output format is similar to that produced by the --syms
           option, except that an extra field is inserted before the
           symbol's name, giving the version information associated with
           the symbol.  If the version is the default version to be used
           when resolving unversioned references to the symbol then it's
           displayed as is, otherwise it's put into parentheses.

       --special-syms
           When displaying symbols include those which the target
           considers to be special in some way and which would not
           normally be of interest to the user.

       -U [d|i|l|e|x|h]
       --unicode=[default|invalid|locale|escape|hex|highlight]
           Controls the display of UTF-8 encoded multibyte characters in
           strings.  The default (--unicode=default) is to give them no
           special treatment.  The --unicode=locale option displays the
           sequence in the current locale, which may or may not support
           them.  The options --unicode=hex and --unicode=invalid
           display them as hex byte sequences enclosed by either angle
           brackets or curly braces.

           The --unicode=escape option displays them as escape sequences
           (\uxxxx) and the --unicode=highlight option displays them as
           escape sequences highlighted in red (if supported by the
           output device).  The colouring is intended to draw attention
           to the presence of unicode sequences where they might not be
           expected.

       -V
       --version
           Print the version number of objdump and exit.

       -x
       --all-headers
           Display all available header information, including the
           symbol table and relocation entries.  Using -x is equivalent
           to specifying all of -a -f -h -p -r -t.

       -w
       --wide
           Format some lines for output devices that have more than 80
           columns.  Also do not truncate symbol names when they are
           displayed.

       -z
       --disassemble-zeroes
           Normally the disassembly output will skip blocks of zeroes.
           This option directs the disassembler to disassemble those
           blocks, just like any other data.

       -Z
       --decompress
           The -Z option is meant to be used in conunction with the -s
           option.  It instructs objdump to decompress any compressed
           sections before displaying their contents.

       @file
           Read command-line options from file.  The options read are
           inserted in place of the original @file option.  If file does
           not exist, or cannot be read, then the option will be treated
           literally, and not removed.

           Options in file are separated by whitespace.  A whitespace
           character may be included in an option by surrounding the
           entire option in either single or double quotes.  Any
           character (including a backslash) may be included by
           prefixing the character to be included with a backslash.  The
           file may itself contain additional @file options; any such
           options will be processed recursively.
SEE ALSO
       nm(1), readelf(1), and the Info entries for binutils.
COPYRIGHT
       Copyright (c) 1991-2024 Free Software Foundation, Inc.

       Permission is granted to copy, distribute and/or modify this
       document under the terms of the GNU Free Documentation License,
       Version 1.3 or any later version published by the Free Software
       Foundation; with no Invariant Sections, with no Front-Cover
       Texts, and with no Back-Cover Texts.  A copy of the license is
       included in the section entitled "GNU Free Documentation License".
COLOPHON
       This page is part of the binutils (a collection of tools for
       working with executable binaries) project.  Information about the
       project can be found at http://www.gnu.org/software/binutils/.
       If you have a bug report for this manual page, see
       http://sourceware.org/bugzilla/enter_bug.cgi?product=binutils.
       This page was obtained from the tarball binutils-2.42.tar.gz
       fetched from https://ftp.gnu.org/gnu/binutils/ on 2024-06-14.
       If you discover any rendering problems in this HTML version of
       the page, or you believe there is a better or more up-to-date
       source for the page, or you have corrections or improvements to
       the information in this COLOPHON (which is not part of the
       original manual page), send a mail to man-pages@man7.org

binutils-2.42                  2024-06-14                     OBJDUMP(1)
\end{lstlisting}
}}
\endinput  %  ==  ==  ==  ==  ==  ==  ==  ==  ==

	% % % \input{./components/man/man-readelf}
\subsection{\refReadelf: Display Information On \elf \ Files}

{\tiny{
\begin{lstlisting}[language=bash]
NAME
       readelf - display information about ELF files
SYNOPSIS
       readelf [-a|--all]
               [-h|--file-header]
               [-l|--program-headers|--segments]
               [-S|--section-headers|--sections]
               [-g|--section-groups]
               [-t|--section-details]
               [-e|--headers]
               [-s|--syms|--symbols]
               [--dyn-syms|--lto-syms]
               [--sym-base=[0|8|10|16]]
               [--demangle=style|--no-demangle]
               [--quiet]
               [--recurse-limit|--no-recurse-limit]
               [-U method|--unicode=method]
               [-X|--extra-sym-info|--no-extra-sym-info]
               [-n|--notes]
               [-r|--relocs]
               [-u|--unwind]
               [-d|--dynamic]
               [-V|--version-info]
               [-A|--arch-specific]
               [-D|--use-dynamic]
               [-L|--lint|--enable-checks]
               [-x <number or name>|--hex-dump=<number or name>]
               [-p <number or name>|--string-dump=<number or name>]
               [-R <number or name>|--relocated-dump=<number or name>]
               [-z|--decompress]
               [-c|--archive-index]
               [-w[lLiaprmfFsoORtUuTgAck]|
                --debug-dump[=rawline,=decodedline,=info,=abbrev,=pubnames,=aranges,=macro,=frames,=frames-interp,=str,=str-offsets,=loc,=Ranges,=pubtypes,=trace_info,=trace_abbrev,=trace_aranges,=gdb_index,=addr,=cu_index,=links]]
               [-wK|--debug-dump=follow-links]
               [-wN|--debug-dump=no-follow-links]
               [-wD|--debug-dump=use-debuginfod]
               [-wE|--debug-dump=do-not-use-debuginfod]
               [-P|--process-links]
               [--dwarf-depth=n]
               [--dwarf-start=n]
               [--ctf=section]
               [--ctf-parent=section]
               [--ctf-symbols=section]
               [--ctf-strings=section]
               [--sframe=section]
               [-I|--histogram]
               [-v|--version]
               [-W|--wide]
               [-T|--silent-truncation]
               [-H|--help]
               elffile...
DESCRIPTION
       readelf displays information about one or more ELF format object
       files.  The options control what particular information to
       display.

       elffile... are the object files to be examined.  32-bit and
       64-bit ELF files are supported, as are archives containing ELF
       files.

       This program performs a similar function to objdump but it goes
       into more detail and it exists independently of the BFD library,
       so if there is a bug in BFD then readelf will not be affected.
OPTIONS
       The long and short forms of options, shown here as alternatives,
       are equivalent.  At least one option besides -v or -H must be
       given.

       -a
       --all
           Equivalent to specifying --file-header, --program-headers,
           --sections, --symbols, --relocs, --dynamic, --notes,
           --version-info, --arch-specific, --unwind, --section-groups
           and --histogram.

           Note - this option does not enable --use-dynamic itself, so
           if that option is not present on the command line then
           dynamic symbols and dynamic relocs will not be displayed.

       -h
       --file-header
           Displays the information contained in the ELF header at the
           start of the file.

       -l
       --program-headers
       --segments
           Displays the information contained in the file's segment
           headers, if it has any.

       --quiet
           Suppress "no symbols" diagnostic.

       -S
       --sections
       --section-headers
           Displays the information contained in the file's section
           headers, if it has any.

       -g
       --section-groups
           Displays the information contained in the file's section
           groups, if it has any.

       -t
       --section-details
           Displays the detailed section information. Implies -S.

       -s
       --symbols
       --syms
           Displays the entries in symbol table section of the file, if
           it has one.  If a symbol has version information associated
           with it then this is displayed as well.  The version string
           is displayed as a suffix to the symbol name, preceded by an @
           character.  For example foo@VER_1.  If the version is the
           default version to be used when resolving unversioned
           references to the symbol then it is displayed as a suffix
           preceded by two @ characters.  For example foo@@VER_2.

       --dyn-syms
           Displays the entries in dynamic symbol table section of the
           file, if it has one.  The output format is the same as the
           format used by the --syms option.

       --lto-syms
           Displays the contents of any LTO symbol tables in the file.

       --sym-base=[0|8|10|16]
           Forces the size field of the symbol table to use the given
           base.  Any unrecognized options will be treated as 0.
           --sym-base=0 represents the default and legacy behaviour.
           This will output sizes as decimal for numbers less than
           100000.  For sizes 100000 and greater hexadecimal notation
           will be used with a 0x prefix.  --sym-base=8 will give the
           symbol sizes in octal.  --sym-base=10 will always give the
           symbol sizes in decimal.  --sym-base=16 will always give the
           symbol sizes in hexadecimal with a 0x prefix.

       -C
       --demangle[=style]
           Decode (demangle) low-level symbol names into user-level
           names.  This makes C++ function names readable.  Different
           compilers have different mangling styles.  The optional
           demangling style argument can be used to choose an
           appropriate demangling style for your compiler.

       --no-demangle
           Do not demangle low-level symbol names.  This is the default.

       --recurse-limit
       --no-recurse-limit
       --recursion-limit
       --no-recursion-limit
           Enables or disables a limit on the amount of recursion
           performed whilst demangling strings.  Since the name mangling
           formats allow for an infinite level of recursion it is
           possible to create strings whose decoding will exhaust the
           amount of stack space available on the host machine,
           triggering a memory fault.  The limit tries to prevent this
           from happening by restricting recursion to 2048 levels of
           nesting.

           The default is for this limit to be enabled, but disabling it
           may be necessary in order to demangle truly complicated
           names.  Note however that if the recursion limit is disabled
           then stack exhaustion is possible and any bug reports about
           such an event will be rejected.

       -U [d|i|l|e|x|h]
       --unicode=[default|invalid|locale|escape|hex|highlight]
           Controls the display of non-ASCII characters in identifier
           names.  The default (--unicode=locale or --unicode=default)
           is to treat them as multibyte characters and display them in
           the current locale.  All other versions of this option treat
           the bytes as UTF-8 encoded values and attempt to interpret
           them.  If they cannot be interpreted or if the
           --unicode=invalid option is used then they are displayed as a
           sequence of hex bytes, encloses in curly parethesis
           characters.

           Using the --unicode=escape option will display the characters
           as as unicode escape sequences (\uxxxx).  Using the
           --unicode=hex will display the characters as hex byte
           sequences enclosed between angle brackets.

           Using the --unicode=highlight will display the characters as
           unicode escape sequences but it will also highlighted them in
           red, assuming that colouring is supported by the output
           device.  The colouring is intended to draw attention to the
           presence of unicode sequences when they might not be
           expected.

       -X
       --extra-sym-info
           When displaying details of symbols, include extra information
           not normally presented.  Currently this just adds the name of
           the section referenced by the symbol's index field, if there
           is one.  In the future more information may be displayed when
           this option is enabled.

           Enabling this option effectively enables the --wide option as
           well, at least when displaying symbol information.

       --no-extra-sym-info
           Disables the effect of the --extra-sym-info option.  This is
           the default.

       -e
       --headers
           Display all the headers in the file.  Equivalent to -h -l -S.

       -n
       --notes
           Displays the contents of the NOTE segments and/or sections,
           if any.

       -r
       --relocs
           Displays the contents of the file's relocation section, if it
           has one.

       -u
       --unwind
           Displays the contents of the file's unwind section, if it has
           one.  Only the unwind sections for IA64 ELF files, as well as
           ARM unwind tables (".ARM.exidx" / ".ARM.extab") are currently
           supported.  If support is not yet implemented for your
           architecture you could try dumping the contents of the
           .eh_frames section using the --debug-dump=frames or
           --debug-dump=frames-interp options.

       -d
       --dynamic
           Displays the contents of the file's dynamic section, if it
           has one.

       -V
       --version-info
           Displays the contents of the version sections in the file, it
           they exist.

       -A
       --arch-specific
           Displays architecture-specific information in the file, if
           there is any.

       -D
       --use-dynamic
           When displaying symbols, this option makes readelf use the
           symbol hash tables in the file's dynamic section, rather than
           the symbol table sections.

           When displaying relocations, this option makes readelf
           display the dynamic relocations rather than the static
           relocations.

       -L
       --lint
       --enable-checks
           Displays warning messages about possible problems with the
           file(s) being examined.  If used on its own then all of the
           contents of the file(s) will be examined.  If used with one
           of the dumping options then the warning messages will only be
           produced for the things being displayed.

       -x <number or name>
       --hex-dump=<number or name>
           Displays the contents of the indicated section as a
           hexadecimal bytes.  A number identifies a particular section
           by index in the section table; any other string identifies
           all sections with that name in the object file.

       -R <number or name>
       --relocated-dump=<number or name>
           Displays the contents of the indicated section as a
           hexadecimal bytes.  A number identifies a particular section
           by index in the section table; any other string identifies
           all sections with that name in the object file.  The contents
           of the section will be relocated before they are displayed.

       -p <number or name>
       --string-dump=<number or name>
           Displays the contents of the indicated section as printable
           strings.  A number identifies a particular section by index
           in the section table; any other string identifies all
           sections with that name in the object file.

       -z
       --decompress
           Requests that the section(s) being dumped by x, R or p
           options are decompressed before being displayed.  If the
           section(s) are not compressed then they are displayed as is.

       -c
       --archive-index
           Displays the file symbol index information contained in the
           header part of binary archives.  Performs the same function
           as the t command to ar, but without using the BFD library.

       -w[lLiaprmfFsOoRtUuTgAckK]
       --debug-dump[=rawline,=decodedline,=info,=abbrev,=pubnames,=aranges,=macro,=frames,=frames-interp,=str,=str-offsets,=loc,=Ranges,=pubtypes,=trace_info,=trace_abbrev,=trace_aranges,=gdb_index,=addr,=cu_index,=links,=follow-links]
           Displays the contents of the DWARF debug sections in the
           file, if any are present.  Compressed debug sections are
           automatically decompressed (temporarily) before they are
           displayed.  If one or more of the optional letters or words
           follows the switch then only those type(s) of data will be
           dumped.  The letters and words refer to the following
           information:

           "a"
           "=abbrev"
               Displays the contents of the .debug_abbrev section.

           "A"
           "=addr"
               Displays the contents of the .debug_addr section.

           "c"
           "=cu_index"
               Displays the contents of the .debug_cu_index and/or
               .debug_tu_index sections.

           "f"
           "=frames"
               Display the raw contents of a .debug_frame section.

           "F"
           "=frames-interp"
               Display the interpreted contents of a .debug_frame
               section.

           "g"
           "=gdb_index"
               Displays the contents of the .gdb_index and/or
               .debug_names sections.

           "i"
           "=info"
               Displays the contents of the .debug_info section.  Note:
               the output from this option can also be restricted by the
               use of the --dwarf-depth and --dwarf-start options.

           "k"
           "=links"
               Displays the contents of the .gnu_debuglink,
               .gnu_debugaltlink and .debug_sup sections, if any of them
               are present.  Also displays any links to separate dwarf
               object files (dwo), if they are specified by the
               DW_AT_GNU_dwo_name or DW_AT_dwo_name attributes in the
               .debug_info section.

           "K"
           "=follow-links"
               Display the contents of any selected debug sections that
               are found in linked, separate debug info file(s).  This
               can result in multiple versions of the same debug section
               being displayed if it exists in more than one file.

               In addition, when displaying DWARF attributes, if a form
               is found that references the separate debug info file,
               then the referenced contents will also be displayed.

               Note - in some distributions this option is enabled by
               default.  It can be disabled via the N debug option.  The
               default can be chosen when configuring the binutils via
               the --enable-follow-debug-links=yes or
               --enable-follow-debug-links=no options.  If these are not
               used then the default is to enable the following of debug
               links.

               Note - if support for the debuginfod protocol was enabled
               when the binutils were built then this option will also
               include an attempt to contact any debuginfod servers
               mentioned in the DEBUGINFOD_URLS environment variable.
               This could take some time to resolve.  This behaviour can
               be disabled via the =do-not-use-debuginfod debug option.

           "N"
           "=no-follow-links"
               Disables the following of links to separate debug info
               files.

           "D"
           "=use-debuginfod"
               Enables contacting debuginfod servers if there is a need
               to follow debug links.  This is the default behaviour.

           "E"
           "=do-not-use-debuginfod"
               Disables contacting debuginfod servers when there is a
               need to follow debug links.

           "l"
           "=rawline"
               Displays the contents of the .debug_line section in a raw
               format.

           "L"
           "=decodedline"
               Displays the interpreted contents of the .debug_line
               section.

           "m"
           "=macro"
               Displays the contents of the .debug_macro and/or
               .debug_macinfo sections.

           "o"
           "=loc"
               Displays the contents of the .debug_loc and/or
               .debug_loclists sections.

           "O"
           "=str-offsets"
               Displays the contents of the .debug_str_offsets section.

           "p"
           "=pubnames"
               Displays the contents of the .debug_pubnames and/or
               .debug_gnu_pubnames sections.

           "r"
           "=aranges"
               Displays the contents of the .debug_aranges section.

           "R"
           "=Ranges"
               Displays the contents of the .debug_ranges and/or
               .debug_rnglists sections.

           "s"
           "=str"
               Displays the contents of the .debug_str, .debug_line_str
               and/or .debug_str_offsets sections.

           "t"
           "=pubtype"
               Displays the contents of the .debug_pubtypes and/or
               .debug_gnu_pubtypes sections.

           "T"
           "=trace_aranges"
               Displays the contents of the .trace_aranges section.

           "u"
           "=trace_abbrev"
               Displays the contents of the .trace_abbrev section.

           "U"
           "=trace_info"
               Displays the contents of the .trace_info section.

           Note: displaying the contents of .debug_static_funcs,
           .debug_static_vars and debug_weaknames sections is not
           currently supported.

       --dwarf-depth=n
           Limit the dump of the ".debug_info" section to n children.
           This is only useful with --debug-dump=info.  The default is
           to print all DIEs; the special value 0 for n will also have
           this effect.

           With a non-zero value for n, DIEs at or deeper than n levels
           will not be printed.  The range for n is zero-based.

       --dwarf-start=n
           Print only DIEs beginning with the DIE numbered n.  This is
           only useful with --debug-dump=info.

           If specified, this option will suppress printing of any
           header information and all DIEs before the DIE numbered n.
           Only siblings and children of the specified DIE will be
           printed.

           This can be used in conjunction with --dwarf-depth.

       -P
       --process-links
           Display the contents of non-debug sections found in separate
           debuginfo files that are linked to the main file.  This
           option automatically implies the -wK option, and only
           sections requested by other command line options will be
           displayed.

       --ctf[=section]
           Display the contents of the specified CTF section.  CTF
           sections themselves contain many subsections, all of which
           are displayed in order.

           By default, display the name of the section named .ctf, which
           is the name emitted by ld.

       --ctf-parent=member
           If the CTF section contains ambiguously-defined types, it
           will consist of an archive of many CTF dictionaries, all
           inheriting from one dictionary containing unambiguous types.
           This member is by default named .ctf, like the section
           containing it, but it is possible to change this name using
           the "ctf_link_set_memb_name_changer" function at link time.
           When looking at CTF archives that have been created by a
           linker that uses the name changer to rename the parent
           archive member, --ctf-parent can be used to specify the name
           used for the parent.

       --ctf-symbols=section
       --ctf-strings=section
           Specify the name of another section from which the CTF file
           can inherit strings and symbols.  By default, the ".symtab"
           and its linked string table are used.

           If either of --ctf-symbols or --ctf-strings is specified, the
           other must be specified as well.

       -I
       --histogram
           Display a histogram of bucket list lengths when displaying
           the contents of the symbol tables.

       -v
       --version
           Display the version number of readelf.

       -W
       --wide
           Don't break output lines to fit into 80 columns. By default
           readelf breaks section header and segment listing lines for
           64-bit ELF files, so that they fit into 80 columns. This
           option causes readelf to print each section header resp. each
           segment one a single line, which is far more readable on
           terminals wider than 80 columns.

       -T
       --silent-truncation
           Normally when readelf is displaying a symbol name, and it has
           to truncate the name to fit into an 80 column display, it
           will add a suffix of "[...]" to the name.  This command line
           option disables this behaviour, allowing 5 more characters of
           the name to be displayed and restoring the old behaviour of
           readelf (prior to release 2.35).

       -H
       --help
           Display the command-line options understood by readelf.

       @file
           Read command-line options from file.  The options read are
           inserted in place of the original @file option.  If file does
           not exist, or cannot be read, then the option will be treated
           literally, and not removed.

           Options in file are separated by whitespace.  A whitespace
           character may be included in an option by surrounding the
           entire option in either single or double quotes.  Any
           character (including a backslash) may be included by
           prefixing the character to be included with a backslash.  The
           file may itself contain additional @file options; any such
           options will be processed recursively.
SEE ALSO
       objdump(1), and the Info entries for binutils.
COPYRIGHT
       Copyright (c) 1991-2024 Free Software Foundation, Inc.

       Permission is granted to copy, distribute and/or modify this
       document under the terms of the GNU Free Documentation License,
       Version 1.3 or any later version published by the Free Software
       Foundation; with no Invariant Sections, with no Front-Cover
       Texts, and with no Back-Cover Texts.  A copy of the license is
       included in the section entitled "GNU Free Documentation
       License".
COLOPHON
       This page is part of the binutils (a collection of tools for
       working with executable binaries) project.  Information about the
       project can be found at http://www.gnu.org/software/binutils/.
       If you have a bug report for this manual page, see
       http://sourceware.org/bugzilla/enter_bug.cgi?product=binutils.
       This page was obtained from the tarball binutils-2.42.tar.gz
       fetched from https://ftp.gnu.org/gnu/binutils/ on 2024-06-14.
       If you discover any rendering problems in this HTML version of
       the page, or you believe there is a better or more up-to-date
       source for the page, or you have corrections or improvements to
       the information in this COLOPHON (which is not part of the
       original manual page), send a mail to man-pages@man7.org

binutils-2.42                  2024-06-14                     READELF(1)
\end{lstlisting}
}}
\endinput  %  ==  ==  ==  ==  ==  ==  ==  ==  ==

\subsection{\refReadelf: Display Information On \elf \ Files}

{\tiny{
\begin{lstlisting}[language=bash]
NAME
       readelf - display information about ELF files
SYNOPSIS
       readelf [-a|--all]
               [-h|--file-header]
               [-l|--program-headers|--segments]
               [-S|--section-headers|--sections]
               [-g|--section-groups]
               [-t|--section-details]
               [-e|--headers]
               [-s|--syms|--symbols]
               [--dyn-syms|--lto-syms]
               [--sym-base=[0|8|10|16]]
               [--demangle=style|--no-demangle]
               [--quiet]
               [--recurse-limit|--no-recurse-limit]
               [-U method|--unicode=method]
               [-X|--extra-sym-info|--no-extra-sym-info]
               [-n|--notes]
               [-r|--relocs]
               [-u|--unwind]
               [-d|--dynamic]
               [-V|--version-info]
               [-A|--arch-specific]
               [-D|--use-dynamic]
               [-L|--lint|--enable-checks]
               [-x <number or name>|--hex-dump=<number or name>]
               [-p <number or name>|--string-dump=<number or name>]
               [-R <number or name>|--relocated-dump=<number or name>]
               [-z|--decompress]
               [-c|--archive-index]
               [-w[lLiaprmfFsoORtUuTgAck]|
                --debug-dump[=rawline,=decodedline,=info,=abbrev,=pubnames,=aranges,=macro,=frames,=frames-interp,=str,=str-offsets,=loc,=Ranges,=pubtypes,=trace_info,=trace_abbrev,=trace_aranges,=gdb_index,=addr,=cu_index,=links]]
               [-wK|--debug-dump=follow-links]
               [-wN|--debug-dump=no-follow-links]
               [-wD|--debug-dump=use-debuginfod]
               [-wE|--debug-dump=do-not-use-debuginfod]
               [-P|--process-links]
               [--dwarf-depth=n]
               [--dwarf-start=n]
               [--ctf=section]
               [--ctf-parent=section]
               [--ctf-symbols=section]
               [--ctf-strings=section]
               [--sframe=section]
               [-I|--histogram]
               [-v|--version]
               [-W|--wide]
               [-T|--silent-truncation]
               [-H|--help]
               elffile...
DESCRIPTION
       readelf displays information about one or more ELF format object
       files.  The options control what particular information to
       display.

       elffile... are the object files to be examined.  32-bit and
       64-bit ELF files are supported, as are archives containing ELF
       files.

       This program performs a similar function to objdump but it goes
       into more detail and it exists independently of the BFD library,
       so if there is a bug in BFD then readelf will not be affected.
OPTIONS
       The long and short forms of options, shown here as alternatives,
       are equivalent.  At least one option besides -v or -H must be
       given.

       -a
       --all
           Equivalent to specifying --file-header, --program-headers,
           --sections, --symbols, --relocs, --dynamic, --notes,
           --version-info, --arch-specific, --unwind, --section-groups
           and --histogram.

           Note - this option does not enable --use-dynamic itself, so
           if that option is not present on the command line then
           dynamic symbols and dynamic relocs will not be displayed.

       -h
       --file-header
           Displays the information contained in the ELF header at the
           start of the file.

       -l
       --program-headers
       --segments
           Displays the information contained in the file's segment
           headers, if it has any.

       --quiet
           Suppress "no symbols" diagnostic.

       -S
       --sections
       --section-headers
           Displays the information contained in the file's section
           headers, if it has any.

       -g
       --section-groups
           Displays the information contained in the file's section
           groups, if it has any.

       -t
       --section-details
           Displays the detailed section information. Implies -S.

       -s
       --symbols
       --syms
           Displays the entries in symbol table section of the file, if
           it has one.  If a symbol has version information associated
           with it then this is displayed as well.  The version string
           is displayed as a suffix to the symbol name, preceded by an @
           character.  For example foo@VER_1.  If the version is the
           default version to be used when resolving unversioned
           references to the symbol then it is displayed as a suffix
           preceded by two @ characters.  For example foo@@VER_2.

       --dyn-syms
           Displays the entries in dynamic symbol table section of the
           file, if it has one.  The output format is the same as the
           format used by the --syms option.

       --lto-syms
           Displays the contents of any LTO symbol tables in the file.

       --sym-base=[0|8|10|16]
           Forces the size field of the symbol table to use the given
           base.  Any unrecognized options will be treated as 0.
           --sym-base=0 represents the default and legacy behaviour.
           This will output sizes as decimal for numbers less than
           100000.  For sizes 100000 and greater hexadecimal notation
           will be used with a 0x prefix.  --sym-base=8 will give the
           symbol sizes in octal.  --sym-base=10 will always give the
           symbol sizes in decimal.  --sym-base=16 will always give the
           symbol sizes in hexadecimal with a 0x prefix.

       -C
       --demangle[=style]
           Decode (demangle) low-level symbol names into user-level
           names.  This makes C++ function names readable.  Different
           compilers have different mangling styles.  The optional
           demangling style argument can be used to choose an
           appropriate demangling style for your compiler.

       --no-demangle
           Do not demangle low-level symbol names.  This is the default.

       --recurse-limit
       --no-recurse-limit
       --recursion-limit
       --no-recursion-limit
           Enables or disables a limit on the amount of recursion
           performed whilst demangling strings.  Since the name mangling
           formats allow for an infinite level of recursion it is
           possible to create strings whose decoding will exhaust the
           amount of stack space available on the host machine,
           triggering a memory fault.  The limit tries to prevent this
           from happening by restricting recursion to 2048 levels of
           nesting.

           The default is for this limit to be enabled, but disabling it
           may be necessary in order to demangle truly complicated
           names.  Note however that if the recursion limit is disabled
           then stack exhaustion is possible and any bug reports about
           such an event will be rejected.

       -U [d|i|l|e|x|h]
       --unicode=[default|invalid|locale|escape|hex|highlight]
           Controls the display of non-ASCII characters in identifier
           names.  The default (--unicode=locale or --unicode=default)
           is to treat them as multibyte characters and display them in
           the current locale.  All other versions of this option treat
           the bytes as UTF-8 encoded values and attempt to interpret
           them.  If they cannot be interpreted or if the
           --unicode=invalid option is used then they are displayed as a
           sequence of hex bytes, encloses in curly parethesis
           characters.

           Using the --unicode=escape option will display the characters
           as as unicode escape sequences (\uxxxx).  Using the
           --unicode=hex will display the characters as hex byte
           sequences enclosed between angle brackets.

           Using the --unicode=highlight will display the characters as
           unicode escape sequences but it will also highlighted them in
           red, assuming that colouring is supported by the output
           device.  The colouring is intended to draw attention to the
           presence of unicode sequences when they might not be
           expected.

       -X
       --extra-sym-info
           When displaying details of symbols, include extra information
           not normally presented.  Currently this just adds the name of
           the section referenced by the symbol's index field, if there
           is one.  In the future more information may be displayed when
           this option is enabled.

           Enabling this option effectively enables the --wide option as
           well, at least when displaying symbol information.

       --no-extra-sym-info
           Disables the effect of the --extra-sym-info option.  This is
           the default.

       -e
       --headers
           Display all the headers in the file.  Equivalent to -h -l -S.

       -n
       --notes
           Displays the contents of the NOTE segments and/or sections,
           if any.

       -r
       --relocs
           Displays the contents of the file's relocation section, if it
           has one.

       -u
       --unwind
           Displays the contents of the file's unwind section, if it has
           one.  Only the unwind sections for IA64 ELF files, as well as
           ARM unwind tables (".ARM.exidx" / ".ARM.extab") are currently
           supported.  If support is not yet implemented for your
           architecture you could try dumping the contents of the
           .eh_frames section using the --debug-dump=frames or
           --debug-dump=frames-interp options.

       -d
       --dynamic
           Displays the contents of the file's dynamic section, if it
           has one.

       -V
       --version-info
           Displays the contents of the version sections in the file, it
           they exist.

       -A
       --arch-specific
           Displays architecture-specific information in the file, if
           there is any.

       -D
       --use-dynamic
           When displaying symbols, this option makes readelf use the
           symbol hash tables in the file's dynamic section, rather than
           the symbol table sections.

           When displaying relocations, this option makes readelf
           display the dynamic relocations rather than the static
           relocations.

       -L
       --lint
       --enable-checks
           Displays warning messages about possible problems with the
           file(s) being examined.  If used on its own then all of the
           contents of the file(s) will be examined.  If used with one
           of the dumping options then the warning messages will only be
           produced for the things being displayed.

       -x <number or name>
       --hex-dump=<number or name>
           Displays the contents of the indicated section as a
           hexadecimal bytes.  A number identifies a particular section
           by index in the section table; any other string identifies
           all sections with that name in the object file.

       -R <number or name>
       --relocated-dump=<number or name>
           Displays the contents of the indicated section as a
           hexadecimal bytes.  A number identifies a particular section
           by index in the section table; any other string identifies
           all sections with that name in the object file.  The contents
           of the section will be relocated before they are displayed.

       -p <number or name>
       --string-dump=<number or name>
           Displays the contents of the indicated section as printable
           strings.  A number identifies a particular section by index
           in the section table; any other string identifies all
           sections with that name in the object file.

       -z
       --decompress
           Requests that the section(s) being dumped by x, R or p
           options are decompressed before being displayed.  If the
           section(s) are not compressed then they are displayed as is.

       -c
       --archive-index
           Displays the file symbol index information contained in the
           header part of binary archives.  Performs the same function
           as the t command to ar, but without using the BFD library.

       -w[lLiaprmfFsOoRtUuTgAckK]
       --debug-dump[=rawline,=decodedline,=info,=abbrev,=pubnames,=aranges,=macro,=frames,=frames-interp,=str,=str-offsets,=loc,=Ranges,=pubtypes,=trace_info,=trace_abbrev,=trace_aranges,=gdb_index,=addr,=cu_index,=links,=follow-links]
           Displays the contents of the DWARF debug sections in the
           file, if any are present.  Compressed debug sections are
           automatically decompressed (temporarily) before they are
           displayed.  If one or more of the optional letters or words
           follows the switch then only those type(s) of data will be
           dumped.  The letters and words refer to the following
           information:

           "a"
           "=abbrev"
               Displays the contents of the .debug_abbrev section.

           "A"
           "=addr"
               Displays the contents of the .debug_addr section.

           "c"
           "=cu_index"
               Displays the contents of the .debug_cu_index and/or
               .debug_tu_index sections.

           "f"
           "=frames"
               Display the raw contents of a .debug_frame section.

           "F"
           "=frames-interp"
               Display the interpreted contents of a .debug_frame
               section.

           "g"
           "=gdb_index"
               Displays the contents of the .gdb_index and/or
               .debug_names sections.

           "i"
           "=info"
               Displays the contents of the .debug_info section.  Note:
               the output from this option can also be restricted by the
               use of the --dwarf-depth and --dwarf-start options.

           "k"
           "=links"
               Displays the contents of the .gnu_debuglink,
               .gnu_debugaltlink and .debug_sup sections, if any of them
               are present.  Also displays any links to separate dwarf
               object files (dwo), if they are specified by the
               DW_AT_GNU_dwo_name or DW_AT_dwo_name attributes in the
               .debug_info section.

           "K"
           "=follow-links"
               Display the contents of any selected debug sections that
               are found in linked, separate debug info file(s).  This
               can result in multiple versions of the same debug section
               being displayed if it exists in more than one file.

               In addition, when displaying DWARF attributes, if a form
               is found that references the separate debug info file,
               then the referenced contents will also be displayed.

               Note - in some distributions this option is enabled by
               default.  It can be disabled via the N debug option.  The
               default can be chosen when configuring the binutils via
               the --enable-follow-debug-links=yes or
               --enable-follow-debug-links=no options.  If these are not
               used then the default is to enable the following of debug
               links.

               Note - if support for the debuginfod protocol was enabled
               when the binutils were built then this option will also
               include an attempt to contact any debuginfod servers
               mentioned in the DEBUGINFOD_URLS environment variable.
               This could take some time to resolve.  This behaviour can
               be disabled via the =do-not-use-debuginfod debug option.

           "N"
           "=no-follow-links"
               Disables the following of links to separate debug info
               files.

           "D"
           "=use-debuginfod"
               Enables contacting debuginfod servers if there is a need
               to follow debug links.  This is the default behaviour.

           "E"
           "=do-not-use-debuginfod"
               Disables contacting debuginfod servers when there is a
               need to follow debug links.

           "l"
           "=rawline"
               Displays the contents of the .debug_line section in a raw
               format.

           "L"
           "=decodedline"
               Displays the interpreted contents of the .debug_line
               section.

           "m"
           "=macro"
               Displays the contents of the .debug_macro and/or
               .debug_macinfo sections.

           "o"
           "=loc"
               Displays the contents of the .debug_loc and/or
               .debug_loclists sections.

           "O"
           "=str-offsets"
               Displays the contents of the .debug_str_offsets section.

           "p"
           "=pubnames"
               Displays the contents of the .debug_pubnames and/or
               .debug_gnu_pubnames sections.

           "r"
           "=aranges"
               Displays the contents of the .debug_aranges section.

           "R"
           "=Ranges"
               Displays the contents of the .debug_ranges and/or
               .debug_rnglists sections.

           "s"
           "=str"
               Displays the contents of the .debug_str, .debug_line_str
               and/or .debug_str_offsets sections.

           "t"
           "=pubtype"
               Displays the contents of the .debug_pubtypes and/or
               .debug_gnu_pubtypes sections.

           "T"
           "=trace_aranges"
               Displays the contents of the .trace_aranges section.

           "u"
           "=trace_abbrev"
               Displays the contents of the .trace_abbrev section.

           "U"
           "=trace_info"
               Displays the contents of the .trace_info section.

           Note: displaying the contents of .debug_static_funcs,
           .debug_static_vars and debug_weaknames sections is not
           currently supported.

       --dwarf-depth=n
           Limit the dump of the ".debug_info" section to n children.
           This is only useful with --debug-dump=info.  The default is
           to print all DIEs; the special value 0 for n will also have
           this effect.

           With a non-zero value for n, DIEs at or deeper than n levels
           will not be printed.  The range for n is zero-based.

       --dwarf-start=n
           Print only DIEs beginning with the DIE numbered n.  This is
           only useful with --debug-dump=info.

           If specified, this option will suppress printing of any
           header information and all DIEs before the DIE numbered n.
           Only siblings and children of the specified DIE will be
           printed.

           This can be used in conjunction with --dwarf-depth.

       -P
       --process-links
           Display the contents of non-debug sections found in separate
           debuginfo files that are linked to the main file.  This
           option automatically implies the -wK option, and only
           sections requested by other command line options will be
           displayed.

       --ctf[=section]
           Display the contents of the specified CTF section.  CTF
           sections themselves contain many subsections, all of which
           are displayed in order.

           By default, display the name of the section named .ctf, which
           is the name emitted by ld.

       --ctf-parent=member
           If the CTF section contains ambiguously-defined types, it
           will consist of an archive of many CTF dictionaries, all
           inheriting from one dictionary containing unambiguous types.
           This member is by default named .ctf, like the section
           containing it, but it is possible to change this name using
           the "ctf_link_set_memb_name_changer" function at link time.
           When looking at CTF archives that have been created by a
           linker that uses the name changer to rename the parent
           archive member, --ctf-parent can be used to specify the name
           used for the parent.

       --ctf-symbols=section
       --ctf-strings=section
           Specify the name of another section from which the CTF file
           can inherit strings and symbols.  By default, the ".symtab"
           and its linked string table are used.

           If either of --ctf-symbols or --ctf-strings is specified, the
           other must be specified as well.

       -I
       --histogram
           Display a histogram of bucket list lengths when displaying
           the contents of the symbol tables.

       -v
       --version
           Display the version number of readelf.

       -W
       --wide
           Don't break output lines to fit into 80 columns. By default
           readelf breaks section header and segment listing lines for
           64-bit ELF files, so that they fit into 80 columns. This
           option causes readelf to print each section header resp. each
           segment one a single line, which is far more readable on
           terminals wider than 80 columns.

       -T
       --silent-truncation
           Normally when readelf is displaying a symbol name, and it has
           to truncate the name to fit into an 80 column display, it
           will add a suffix of "[...]" to the name.  This command line
           option disables this behaviour, allowing 5 more characters of
           the name to be displayed and restoring the old behaviour of
           readelf (prior to release 2.35).

       -H
       --help
           Display the command-line options understood by readelf.

       @file
           Read command-line options from file.  The options read are
           inserted in place of the original @file option.  If file does
           not exist, or cannot be read, then the option will be treated
           literally, and not removed.

           Options in file are separated by whitespace.  A whitespace
           character may be included in an option by surrounding the
           entire option in either single or double quotes.  Any
           character (including a backslash) may be included by
           prefixing the character to be included with a backslash.  The
           file may itself contain additional @file options; any such
           options will be processed recursively.
SEE ALSO
       objdump(1), and the Info entries for binutils.
COPYRIGHT
       Copyright (c) 1991-2024 Free Software Foundation, Inc.

       Permission is granted to copy, distribute and/or modify this
       document under the terms of the GNU Free Documentation License,
       Version 1.3 or any later version published by the Free Software
       Foundation; with no Invariant Sections, with no Front-Cover
       Texts, and with no Back-Cover Texts.  A copy of the license is
       included in the section entitled "GNU Free Documentation
       License".
COLOPHON
       This page is part of the binutils (a collection of tools for
       working with executable binaries) project.  Information about the
       project can be found at http://www.gnu.org/software/binutils/.
       If you have a bug report for this manual page, see
       http://sourceware.org/bugzilla/enter_bug.cgi?product=binutils.
       This page was obtained from the tarball binutils-2.42.tar.gz
       fetched from https://ftp.gnu.org/gnu/binutils/ on 2024-06-14.
       If you discover any rendering problems in this HTML version of
       the page, or you believe there is a better or more up-to-date
       source for the page, or you have corrections or improvements to
       the information in this COLOPHON (which is not part of the
       original manual page), send a mail to man-pages@man7.org

binutils-2.42                  2024-06-14                     READELF(1)
\end{lstlisting}
}}
\endinput  %  ==  ==  ==  ==  ==  ==  ==  ==  ==

\subsection{\refReadelf: Display Information On \elf \ Files}

{\tiny{
\begin{lstlisting}[language=bash]
NAME
       readelf - display information about ELF files
SYNOPSIS
       readelf [-a|--all]
               [-h|--file-header]
               [-l|--program-headers|--segments]
               [-S|--section-headers|--sections]
               [-g|--section-groups]
               [-t|--section-details]
               [-e|--headers]
               [-s|--syms|--symbols]
               [--dyn-syms|--lto-syms]
               [--sym-base=[0|8|10|16]]
               [--demangle=style|--no-demangle]
               [--quiet]
               [--recurse-limit|--no-recurse-limit]
               [-U method|--unicode=method]
               [-X|--extra-sym-info|--no-extra-sym-info]
               [-n|--notes]
               [-r|--relocs]
               [-u|--unwind]
               [-d|--dynamic]
               [-V|--version-info]
               [-A|--arch-specific]
               [-D|--use-dynamic]
               [-L|--lint|--enable-checks]
               [-x <number or name>|--hex-dump=<number or name>]
               [-p <number or name>|--string-dump=<number or name>]
               [-R <number or name>|--relocated-dump=<number or name>]
               [-z|--decompress]
               [-c|--archive-index]
               [-w[lLiaprmfFsoORtUuTgAck]|
                --debug-dump[=rawline,=decodedline,=info,=abbrev,=pubnames,=aranges,=macro,=frames,=frames-interp,=str,=str-offsets,=loc,=Ranges,=pubtypes,=trace_info,=trace_abbrev,=trace_aranges,=gdb_index,=addr,=cu_index,=links]]
               [-wK|--debug-dump=follow-links]
               [-wN|--debug-dump=no-follow-links]
               [-wD|--debug-dump=use-debuginfod]
               [-wE|--debug-dump=do-not-use-debuginfod]
               [-P|--process-links]
               [--dwarf-depth=n]
               [--dwarf-start=n]
               [--ctf=section]
               [--ctf-parent=section]
               [--ctf-symbols=section]
               [--ctf-strings=section]
               [--sframe=section]
               [-I|--histogram]
               [-v|--version]
               [-W|--wide]
               [-T|--silent-truncation]
               [-H|--help]
               elffile...
DESCRIPTION
       readelf displays information about one or more ELF format object
       files.  The options control what particular information to
       display.

       elffile... are the object files to be examined.  32-bit and
       64-bit ELF files are supported, as are archives containing ELF
       files.

       This program performs a similar function to objdump but it goes
       into more detail and it exists independently of the BFD library,
       so if there is a bug in BFD then readelf will not be affected.
OPTIONS
       The long and short forms of options, shown here as alternatives,
       are equivalent.  At least one option besides -v or -H must be
       given.

       -a
       --all
           Equivalent to specifying --file-header, --program-headers,
           --sections, --symbols, --relocs, --dynamic, --notes,
           --version-info, --arch-specific, --unwind, --section-groups
           and --histogram.

           Note - this option does not enable --use-dynamic itself, so
           if that option is not present on the command line then
           dynamic symbols and dynamic relocs will not be displayed.

       -h
       --file-header
           Displays the information contained in the ELF header at the
           start of the file.

       -l
       --program-headers
       --segments
           Displays the information contained in the file's segment
           headers, if it has any.

       --quiet
           Suppress "no symbols" diagnostic.

       -S
       --sections
       --section-headers
           Displays the information contained in the file's section
           headers, if it has any.

       -g
       --section-groups
           Displays the information contained in the file's section
           groups, if it has any.

       -t
       --section-details
           Displays the detailed section information. Implies -S.

       -s
       --symbols
       --syms
           Displays the entries in symbol table section of the file, if
           it has one.  If a symbol has version information associated
           with it then this is displayed as well.  The version string
           is displayed as a suffix to the symbol name, preceded by an @
           character.  For example foo@VER_1.  If the version is the
           default version to be used when resolving unversioned
           references to the symbol then it is displayed as a suffix
           preceded by two @ characters.  For example foo@@VER_2.

       --dyn-syms
           Displays the entries in dynamic symbol table section of the
           file, if it has one.  The output format is the same as the
           format used by the --syms option.

       --lto-syms
           Displays the contents of any LTO symbol tables in the file.

       --sym-base=[0|8|10|16]
           Forces the size field of the symbol table to use the given
           base.  Any unrecognized options will be treated as 0.
           --sym-base=0 represents the default and legacy behaviour.
           This will output sizes as decimal for numbers less than
           100000.  For sizes 100000 and greater hexadecimal notation
           will be used with a 0x prefix.  --sym-base=8 will give the
           symbol sizes in octal.  --sym-base=10 will always give the
           symbol sizes in decimal.  --sym-base=16 will always give the
           symbol sizes in hexadecimal with a 0x prefix.

       -C
       --demangle[=style]
           Decode (demangle) low-level symbol names into user-level
           names.  This makes C++ function names readable.  Different
           compilers have different mangling styles.  The optional
           demangling style argument can be used to choose an
           appropriate demangling style for your compiler.

       --no-demangle
           Do not demangle low-level symbol names.  This is the default.

       --recurse-limit
       --no-recurse-limit
       --recursion-limit
       --no-recursion-limit
           Enables or disables a limit on the amount of recursion
           performed whilst demangling strings.  Since the name mangling
           formats allow for an infinite level of recursion it is
           possible to create strings whose decoding will exhaust the
           amount of stack space available on the host machine,
           triggering a memory fault.  The limit tries to prevent this
           from happening by restricting recursion to 2048 levels of
           nesting.

           The default is for this limit to be enabled, but disabling it
           may be necessary in order to demangle truly complicated
           names.  Note however that if the recursion limit is disabled
           then stack exhaustion is possible and any bug reports about
           such an event will be rejected.

       -U [d|i|l|e|x|h]
       --unicode=[default|invalid|locale|escape|hex|highlight]
           Controls the display of non-ASCII characters in identifier
           names.  The default (--unicode=locale or --unicode=default)
           is to treat them as multibyte characters and display them in
           the current locale.  All other versions of this option treat
           the bytes as UTF-8 encoded values and attempt to interpret
           them.  If they cannot be interpreted or if the
           --unicode=invalid option is used then they are displayed as a
           sequence of hex bytes, encloses in curly parethesis
           characters.

           Using the --unicode=escape option will display the characters
           as as unicode escape sequences (\uxxxx).  Using the
           --unicode=hex will display the characters as hex byte
           sequences enclosed between angle brackets.

           Using the --unicode=highlight will display the characters as
           unicode escape sequences but it will also highlighted them in
           red, assuming that colouring is supported by the output
           device.  The colouring is intended to draw attention to the
           presence of unicode sequences when they might not be
           expected.

       -X
       --extra-sym-info
           When displaying details of symbols, include extra information
           not normally presented.  Currently this just adds the name of
           the section referenced by the symbol's index field, if there
           is one.  In the future more information may be displayed when
           this option is enabled.

           Enabling this option effectively enables the --wide option as
           well, at least when displaying symbol information.

       --no-extra-sym-info
           Disables the effect of the --extra-sym-info option.  This is
           the default.

       -e
       --headers
           Display all the headers in the file.  Equivalent to -h -l -S.

       -n
       --notes
           Displays the contents of the NOTE segments and/or sections,
           if any.

       -r
       --relocs
           Displays the contents of the file's relocation section, if it
           has one.

       -u
       --unwind
           Displays the contents of the file's unwind section, if it has
           one.  Only the unwind sections for IA64 ELF files, as well as
           ARM unwind tables (".ARM.exidx" / ".ARM.extab") are currently
           supported.  If support is not yet implemented for your
           architecture you could try dumping the contents of the
           .eh_frames section using the --debug-dump=frames or
           --debug-dump=frames-interp options.

       -d
       --dynamic
           Displays the contents of the file's dynamic section, if it
           has one.

       -V
       --version-info
           Displays the contents of the version sections in the file, it
           they exist.

       -A
       --arch-specific
           Displays architecture-specific information in the file, if
           there is any.

       -D
       --use-dynamic
           When displaying symbols, this option makes readelf use the
           symbol hash tables in the file's dynamic section, rather than
           the symbol table sections.

           When displaying relocations, this option makes readelf
           display the dynamic relocations rather than the static
           relocations.

       -L
       --lint
       --enable-checks
           Displays warning messages about possible problems with the
           file(s) being examined.  If used on its own then all of the
           contents of the file(s) will be examined.  If used with one
           of the dumping options then the warning messages will only be
           produced for the things being displayed.

       -x <number or name>
       --hex-dump=<number or name>
           Displays the contents of the indicated section as a
           hexadecimal bytes.  A number identifies a particular section
           by index in the section table; any other string identifies
           all sections with that name in the object file.

       -R <number or name>
       --relocated-dump=<number or name>
           Displays the contents of the indicated section as a
           hexadecimal bytes.  A number identifies a particular section
           by index in the section table; any other string identifies
           all sections with that name in the object file.  The contents
           of the section will be relocated before they are displayed.

       -p <number or name>
       --string-dump=<number or name>
           Displays the contents of the indicated section as printable
           strings.  A number identifies a particular section by index
           in the section table; any other string identifies all
           sections with that name in the object file.

       -z
       --decompress
           Requests that the section(s) being dumped by x, R or p
           options are decompressed before being displayed.  If the
           section(s) are not compressed then they are displayed as is.

       -c
       --archive-index
           Displays the file symbol index information contained in the
           header part of binary archives.  Performs the same function
           as the t command to ar, but without using the BFD library.

       -w[lLiaprmfFsOoRtUuTgAckK]
       --debug-dump[=rawline,=decodedline,=info,=abbrev,=pubnames,=aranges,=macro,=frames,=frames-interp,=str,=str-offsets,=loc,=Ranges,=pubtypes,=trace_info,=trace_abbrev,=trace_aranges,=gdb_index,=addr,=cu_index,=links,=follow-links]
           Displays the contents of the DWARF debug sections in the
           file, if any are present.  Compressed debug sections are
           automatically decompressed (temporarily) before they are
           displayed.  If one or more of the optional letters or words
           follows the switch then only those type(s) of data will be
           dumped.  The letters and words refer to the following
           information:

           "a"
           "=abbrev"
               Displays the contents of the .debug_abbrev section.

           "A"
           "=addr"
               Displays the contents of the .debug_addr section.

           "c"
           "=cu_index"
               Displays the contents of the .debug_cu_index and/or
               .debug_tu_index sections.

           "f"
           "=frames"
               Display the raw contents of a .debug_frame section.

           "F"
           "=frames-interp"
               Display the interpreted contents of a .debug_frame
               section.

           "g"
           "=gdb_index"
               Displays the contents of the .gdb_index and/or
               .debug_names sections.

           "i"
           "=info"
               Displays the contents of the .debug_info section.  Note:
               the output from this option can also be restricted by the
               use of the --dwarf-depth and --dwarf-start options.

           "k"
           "=links"
               Displays the contents of the .gnu_debuglink,
               .gnu_debugaltlink and .debug_sup sections, if any of them
               are present.  Also displays any links to separate dwarf
               object files (dwo), if they are specified by the
               DW_AT_GNU_dwo_name or DW_AT_dwo_name attributes in the
               .debug_info section.

           "K"
           "=follow-links"
               Display the contents of any selected debug sections that
               are found in linked, separate debug info file(s).  This
               can result in multiple versions of the same debug section
               being displayed if it exists in more than one file.

               In addition, when displaying DWARF attributes, if a form
               is found that references the separate debug info file,
               then the referenced contents will also be displayed.

               Note - in some distributions this option is enabled by
               default.  It can be disabled via the N debug option.  The
               default can be chosen when configuring the binutils via
               the --enable-follow-debug-links=yes or
               --enable-follow-debug-links=no options.  If these are not
               used then the default is to enable the following of debug
               links.

               Note - if support for the debuginfod protocol was enabled
               when the binutils were built then this option will also
               include an attempt to contact any debuginfod servers
               mentioned in the DEBUGINFOD_URLS environment variable.
               This could take some time to resolve.  This behaviour can
               be disabled via the =do-not-use-debuginfod debug option.

           "N"
           "=no-follow-links"
               Disables the following of links to separate debug info
               files.

           "D"
           "=use-debuginfod"
               Enables contacting debuginfod servers if there is a need
               to follow debug links.  This is the default behaviour.

           "E"
           "=do-not-use-debuginfod"
               Disables contacting debuginfod servers when there is a
               need to follow debug links.

           "l"
           "=rawline"
               Displays the contents of the .debug_line section in a raw
               format.

           "L"
           "=decodedline"
               Displays the interpreted contents of the .debug_line
               section.

           "m"
           "=macro"
               Displays the contents of the .debug_macro and/or
               .debug_macinfo sections.

           "o"
           "=loc"
               Displays the contents of the .debug_loc and/or
               .debug_loclists sections.

           "O"
           "=str-offsets"
               Displays the contents of the .debug_str_offsets section.

           "p"
           "=pubnames"
               Displays the contents of the .debug_pubnames and/or
               .debug_gnu_pubnames sections.

           "r"
           "=aranges"
               Displays the contents of the .debug_aranges section.

           "R"
           "=Ranges"
               Displays the contents of the .debug_ranges and/or
               .debug_rnglists sections.

           "s"
           "=str"
               Displays the contents of the .debug_str, .debug_line_str
               and/or .debug_str_offsets sections.

           "t"
           "=pubtype"
               Displays the contents of the .debug_pubtypes and/or
               .debug_gnu_pubtypes sections.

           "T"
           "=trace_aranges"
               Displays the contents of the .trace_aranges section.

           "u"
           "=trace_abbrev"
               Displays the contents of the .trace_abbrev section.

           "U"
           "=trace_info"
               Displays the contents of the .trace_info section.

           Note: displaying the contents of .debug_static_funcs,
           .debug_static_vars and debug_weaknames sections is not
           currently supported.

       --dwarf-depth=n
           Limit the dump of the ".debug_info" section to n children.
           This is only useful with --debug-dump=info.  The default is
           to print all DIEs; the special value 0 for n will also have
           this effect.

           With a non-zero value for n, DIEs at or deeper than n levels
           will not be printed.  The range for n is zero-based.

       --dwarf-start=n
           Print only DIEs beginning with the DIE numbered n.  This is
           only useful with --debug-dump=info.

           If specified, this option will suppress printing of any
           header information and all DIEs before the DIE numbered n.
           Only siblings and children of the specified DIE will be
           printed.

           This can be used in conjunction with --dwarf-depth.

       -P
       --process-links
           Display the contents of non-debug sections found in separate
           debuginfo files that are linked to the main file.  This
           option automatically implies the -wK option, and only
           sections requested by other command line options will be
           displayed.

       --ctf[=section]
           Display the contents of the specified CTF section.  CTF
           sections themselves contain many subsections, all of which
           are displayed in order.

           By default, display the name of the section named .ctf, which
           is the name emitted by ld.

       --ctf-parent=member
           If the CTF section contains ambiguously-defined types, it
           will consist of an archive of many CTF dictionaries, all
           inheriting from one dictionary containing unambiguous types.
           This member is by default named .ctf, like the section
           containing it, but it is possible to change this name using
           the "ctf_link_set_memb_name_changer" function at link time.
           When looking at CTF archives that have been created by a
           linker that uses the name changer to rename the parent
           archive member, --ctf-parent can be used to specify the name
           used for the parent.

       --ctf-symbols=section
       --ctf-strings=section
           Specify the name of another section from which the CTF file
           can inherit strings and symbols.  By default, the ".symtab"
           and its linked string table are used.

           If either of --ctf-symbols or --ctf-strings is specified, the
           other must be specified as well.

       -I
       --histogram
           Display a histogram of bucket list lengths when displaying
           the contents of the symbol tables.

       -v
       --version
           Display the version number of readelf.

       -W
       --wide
           Don't break output lines to fit into 80 columns. By default
           readelf breaks section header and segment listing lines for
           64-bit ELF files, so that they fit into 80 columns. This
           option causes readelf to print each section header resp. each
           segment one a single line, which is far more readable on
           terminals wider than 80 columns.

       -T
       --silent-truncation
           Normally when readelf is displaying a symbol name, and it has
           to truncate the name to fit into an 80 column display, it
           will add a suffix of "[...]" to the name.  This command line
           option disables this behaviour, allowing 5 more characters of
           the name to be displayed and restoring the old behaviour of
           readelf (prior to release 2.35).

       -H
       --help
           Display the command-line options understood by readelf.

       @file
           Read command-line options from file.  The options read are
           inserted in place of the original @file option.  If file does
           not exist, or cannot be read, then the option will be treated
           literally, and not removed.

           Options in file are separated by whitespace.  A whitespace
           character may be included in an option by surrounding the
           entire option in either single or double quotes.  Any
           character (including a backslash) may be included by
           prefixing the character to be included with a backslash.  The
           file may itself contain additional @file options; any such
           options will be processed recursively.
SEE ALSO
       objdump(1), and the Info entries for binutils.
COPYRIGHT
       Copyright (c) 1991-2024 Free Software Foundation, Inc.

       Permission is granted to copy, distribute and/or modify this
       document under the terms of the GNU Free Documentation License,
       Version 1.3 or any later version published by the Free Software
       Foundation; with no Invariant Sections, with no Front-Cover
       Texts, and with no Back-Cover Texts.  A copy of the license is
       included in the section entitled "GNU Free Documentation
       License".
COLOPHON
       This page is part of the binutils (a collection of tools for
       working with executable binaries) project.  Information about the
       project can be found at http://www.gnu.org/software/binutils/.
       If you have a bug report for this manual page, see
       http://sourceware.org/bugzilla/enter_bug.cgi?product=binutils.
       This page was obtained from the tarball binutils-2.42.tar.gz
       fetched from https://ftp.gnu.org/gnu/binutils/ on 2024-06-14.
       If you discover any rendering problems in this HTML version of
       the page, or you believe there is a better or more up-to-date
       source for the page, or you have corrections or improvements to
       the information in this COLOPHON (which is not part of the
       original manual page), send a mail to man-pages@man7.org

binutils-2.42                  2024-06-14                     READELF(1)
\end{lstlisting}
}}
\endinput  %  ==  ==  ==  ==  ==  ==  ==  ==  ==

	% % % \input{./components/man/man-nm}
\subsection{\refNm: List Symbols From Object Files}

{\tiny{
\begin{lstlisting}[language=bash]
NAME
       nm - list symbols from object files
SYNOPSIS
       nm [-A|-o|--print-file-name]
          [-a|--debug-syms]
          [-B|--format=bsd]
          [-C|--demangle[=style]]
          [-D|--dynamic]
          [-fformat|--format=format]
          [-g|--extern-only]
          [-h|--help]
          [--ifunc-chars=CHARS]
          [-j|--format=just-symbols]
          [-l|--line-numbers] [--inlines]
          [-n|-v|--numeric-sort]
          [-P|--portability]
          [-p|--no-sort]
          [-r|--reverse-sort]
          [-S|--print-size]
          [-s|--print-armap]
          [-t radix|--radix=radix]
          [-u|--undefined-only]
          [-U|--defined-only]
          [-V|--version]
          [-W|--no-weak]
          [-X 32_64]
          [--no-demangle]
          [--no-recurse-limit|--recurse-limit]]
          [--plugin name]
          [--size-sort]
          [--special-syms]
          [--synthetic]
          [--target=bfdname]
          [--unicode=method]
          [--with-symbol-versions]
          [--without-symbol-versions]
          [objfile...]
DESCRIPTION
       GNU nm lists the symbols from object files objfile....  If no
       object files are listed as arguments, nm assumes the file a.out.

       For each symbol, nm shows:

       *   The symbol value, in the radix selected by options (see
           below), or hexadecimal by default.

       *   The symbol type.  At least the following types are used;
           others are, as well, depending on the object file format.  If
           lowercase, the symbol is usually local; if uppercase, the
           symbol is global (external).  There are however a few
           lowercase symbols that are shown for special global symbols
           ("u", "v" and "w").

           "A" The symbol's value is absolute, and will not be changed
               by further linking.

           "B"
           "b" The symbol is in the BSS data section.  This section
               typically contains zero-initialized or uninitialized
               data, although the exact behavior is system dependent.

           "C"
           "c" The symbol is common.  Common symbols are uninitialized
               data.  When linking, multiple common symbols may appear
               with the same name.  If the symbol is defined anywhere,
               the common symbols are treated as undefined references.
               The lower case c character is used when the symbol is in
               a special section for small commons.

           "D"
           "d" The symbol is in the initialized data section.

           "G"
           "g" The symbol is in an initialized data section for small
               objects.  Some object file formats permit more efficient
               access to small data objects, such as a global int
               variable as opposed to a large global array.

           "i" For PE format files this indicates that the symbol is in
               a section specific to the implementation of DLLs.

               For ELF format files this indicates that the symbol is an
               indirect function.  This is a GNU extension to the
               standard set of ELF symbol types.  It indicates a symbol
               which if referenced by a relocation does not evaluate to
               its address, but instead must be invoked at runtime.  The
               runtime execution will then return the value to be used
               in the relocation.

               Note - the actual symbols display for GNU indirect
               symbols is controlled by the --ifunc-chars command line
               option.  If this option has been provided then the first
               character in the string will be used for global indirect
               function symbols.  If the string contains a second
               character then that will be used for local indirect
               function symbols.

           "I" The symbol is an indirect reference to another symbol.

           "N" The symbol is a debugging symbol.

           "n" The symbol is in a non-data, non-code, non-debug read-
               only section.

           "p" The symbol is in a stack unwind section.

           "R"
           "r" The symbol is in a read only data section.

           "S"
           "s" The symbol is in an uninitialized or zero-initialized
               data section for small objects.

           "T"
           "t" The symbol is in the text (code) section.

           "U" The symbol is undefined.

           "u" The symbol is a unique global symbol.  This is a GNU
               extension to the standard set of ELF symbol bindings.
               For such a symbol the dynamic linker will make sure that
               in the entire process there is just one symbol with this
               name and type in use.

           "V"
           "v" The symbol is a weak object.  When a weak defined symbol
               is linked with a normal defined symbol, the normal
               defined symbol is used with no error.  When a weak
               undefined symbol is linked and the symbol is not defined,
               the value of the weak symbol becomes zero with no error.
               On some systems, uppercase indicates that a default value
               has been specified.

           "W"
           "w" The symbol is a weak symbol that has not been
               specifically tagged as a weak object symbol.  When a weak
               defined symbol is linked with a normal defined symbol,
               the normal defined symbol is used with no error.  When a
               weak undefined symbol is linked and the symbol is not
               defined, the value of the symbol is determined in a
               system-specific manner without error.  On some systems,
               uppercase indicates that a default value has been
               specified.

           "-" The symbol is a stabs symbol in an a.out object file.  In
               this case, the next values printed are the stabs other
               field, the stabs desc field, and the stab type.  Stabs
               symbols are used to hold debugging information.

           "?" The symbol type is unknown, or object file format
               specific.

       *   The symbol name.  If a symbol has version information
           associated with it, then the version information is displayed
           as well.  If the versioned symbol is undefined or hidden from
           linker, the version string is displayed as a suffix to the
           symbol name, preceded by an @ character.  For example
           foo@VER_1.  If the version is the default version to be used
           when resolving unversioned references to the symbol, then it
           is displayed as a suffix preceded by two @ characters.  For
           example foo@@VER_2.
OPTIONS
       The long and short forms of options, shown here as alternatives,
       are equivalent.

       -A
       -o
       --print-file-name
           Precede each symbol by the name of the input file (or archive
           member) in which it was found, rather than identifying the
           input file once only, before all of its symbols.

       -a
       --debug-syms
           Display all symbols, even debugger-only symbols; normally
           these are not listed.

       -B  The same as --format=bsd (for compatibility with the MIPS
           nm).

       -C
       --demangle[=style]
           Decode (demangle) low-level symbol names into user-level
           names.  Besides removing any initial underscore prepended by
           the system, this makes C++ function names readable. Different
           compilers have different mangling styles. The optional
           demangling style argument can be used to choose an
           appropriate demangling style for your compiler.

       --no-demangle
           Do not demangle low-level symbol names.  This is the default.

       --recurse-limit
       --no-recurse-limit
       --recursion-limit
       --no-recursion-limit
           Enables or disables a limit on the amount of recursion
           performed whilst demangling strings.  Since the name mangling
           formats allow for an infinite level of recursion it is
           possible to create strings whose decoding will exhaust the
           amount of stack space available on the host machine,
           triggering a memory fault.  The limit tries to prevent this
           from happening by restricting recursion to 2048 levels of
           nesting.

           The default is for this limit to be enabled, but disabling it
           may be necessary in order to demangle truly complicated
           names.  Note however that if the recursion limit is disabled
           then stack exhaustion is possible and any bug reports about
           such an event will be rejected.

       -D
       --dynamic
           Display the dynamic symbols rather than the normal symbols.
           This is only meaningful for dynamic objects, such as certain
           types of shared libraries.

       -f format
       --format=format
           Use the output format format, which can be "bsd", "sysv",
           "posix" or "just-symbols".  The default is "bsd".  Only the
           first character of format is significant; it can be either
           upper or lower case.

       -g
       --extern-only
           Display only external symbols.

       -h
       --help
           Show a summary of the options to nm and exit.

       --ifunc-chars=CHARS
           When display GNU indirect function symbols nm will default to
           using the "i" character for both local indirect functions and
           global indirect functions.  The --ifunc-chars option allows
           the user to specify a string containing one or two
           characters. The first character will be used for global
           indirect function symbols and the second character, if
           present, will be used for local indirect function symbols.

       j   The same as --format=just-symbols.

       -l
       --line-numbers
           For each symbol, use debugging information to try to find a
           filename and line number.  For a defined symbol, look for the
           line number of the address of the symbol.  For an undefined
           symbol, look for the line number of a relocation entry which
           refers to the symbol.  If line number information can be
           found, print it after the other symbol information.

       --inlines
           When option -l is active, if the address belongs to a
           function that was inlined, then this option causes the source
           information for all enclosing scopes back to the first non-
           inlined function to be printed as well.  For example, if
           "main" inlines "callee1" which inlines "callee2", and address
           is from "callee2", the source information for "callee1" and
           "main" will also be printed.

       -n
       -v
       --numeric-sort
           Sort symbols numerically by their addresses, rather than
           alphabetically by their names.

       -p
       --no-sort
           Do not bother to sort the symbols in any order; print them in
           the order encountered.

       -P
       --portability
           Use the POSIX.2 standard output format instead of the default
           format.  Equivalent to -f posix.

       -r
       --reverse-sort
           Reverse the order of the sort (whether numeric or
           alphabetic); let the last come first.

       -S
       --print-size
           Print both value and size of defined symbols for the "bsd"
           output style.  This option has no effect for object formats
           that do not record symbol sizes, unless --size-sort is also
           used in which case a calculated size is displayed.

       -s
       --print-armap
           When listing symbols from archive members, include the index:
           a mapping (stored in the archive by ar or ranlib) of which
           modules contain definitions for which names.

       -t radix
       --radix=radix
           Use radix as the radix for printing the symbol values.  It
           must be d for decimal, o for octal, or x for hexadecimal.

       -u
       --undefined-only
           Display only undefined symbols (those external to each object
           file).  By default both defined and undefined symbols are
           displayed.

       -U
       --defined-only
           Display only defined symbols for each object file.  By
           default both defined and undefined symbols are displayed.

       -V
       --version
           Show the version number of nm and exit.

       -X  This option is ignored for compatibility with the AIX version
           of nm.  It takes one parameter which must be the string
           32_64.  The default mode of AIX nm corresponds to -X 32,
           which is not supported by GNU nm.

       --plugin name
           Load the plugin called name to add support for extra target
           types.  This option is only available if the toolchain has
           been built with plugin support enabled.

           If --plugin is not provided, but plugin support has been
           enabled then nm iterates over the files in
           ${libdir}/bfd-plugins in alphabetic order and the first
           plugin that claims the object in question is used.

           Please note that this plugin search directory is not the one
           used by ld's -plugin option.  In order to make nm use the
           linker plugin it must be copied into the
           ${libdir}/bfd-plugins directory.  For GCC based compilations
           the linker plugin is called liblto_plugin.so.0.0.0.  For
           Clang based compilations it is called LLVMgold.so.  The GCC
           plugin is always backwards compatible with earlier versions,
           so it is sufficient to just copy the newest one.

       --size-sort
           Sort symbols by size.  For ELF objects symbol sizes are read
           from the ELF, for other object types the symbol sizes are
           computed as the difference between the value of the symbol
           and the value of the symbol with the next higher value.  If
           the "bsd" output format is used the size of the symbol is
           printed, rather than the value, and -S must be used in order
           both size and value to be printed.

           Note - this option does not work if --undefined-only has been
           enabled as undefined symbols have no size.

       --special-syms
           Display symbols which have a target-specific special meaning.
           These symbols are usually used by the target for some special
           processing and are not normally helpful when included in the
           normal symbol lists.  For example for ARM targets this option
           would skip the mapping symbols used to mark transitions
           between ARM code, THUMB code and data.

       --synthetic
           Include synthetic symbols in the output.  These are special
           symbols created by the linker for various purposes.  They are
           not shown by default since they are not part of the binary's
           original source code.

       --unicode=[default|invalid|locale|escape|hex|highlight]
           Controls the display of UTF-8 encoded multibyte characters in
           strings.  The default (--unicode=default) is to give them no
           special treatment.  The --unicode=locale option displays the
           sequence in the current locale, which may or may not support
           them.  The options --unicode=hex and --unicode=invalid
           display them as hex byte sequences enclosed by either angle
           brackets or curly braces.

           The --unicode=escape option displays them as escape sequences
           (\uxxxx) and the --unicode=highlight option displays them as
           escape sequences highlighted in red (if supported by the
           output device).  The colouring is intended to draw attention
           to the presence of unicode sequences where they might not be
           expected.

       -W
       --no-weak
           Do not display weak symbols.

       --with-symbol-versions
       --without-symbol-versions
           Enables or disables the display of symbol version
           information.  The version string is displayed as a suffix to
           the symbol name, preceded by an @ character.  For example
           foo@VER_1.  If the version is the default version to be used
           when resolving unversioned references to the symbol then it
           is displayed as a suffix preceded by two @ characters.  For
           example foo@@VER_2.  By default, symbol version information
           is displayed.

       --target=bfdname
           Specify an object code format other than your system's
           default format.

       @file
           Read command-line options from file.  The options read are
           inserted in place of the original @file option.  If file does
           not exist, or cannot be read, then the option will be treated
           literally, and not removed.

           Options in file are separated by whitespace.  A whitespace
           character may be included in an option by surrounding the
           entire option in either single or double quotes.  Any
           character (including a backslash) may be included by
           prefixing the character to be included with a backslash.  The
           file may itself contain additional @file options; any such
           options will be processed recursively.
SEE ALSO
       ar(1), objdump(1), ranlib(1), and the Info entries for binutils.
COPYRIGHT
       Copyright (c) 1991-2024 Free Software Foundation, Inc.

       Permission is granted to copy, distribute and/or modify this
       document under the terms of the GNU Free Documentation License,
       Version 1.3 or any later version published by the Free Software
       Foundation; with no Invariant Sections, with no Front-Cover
       Texts, and with no Back-Cover Texts.  A copy of the license is
       included in the section entitled "GNU Free Documentation
       License".
COLOPHON
       This page is part of the binutils (a collection of tools for
       working with executable binaries) project.  Information about the
       project can be found at http://www.gnu.org/software/binutils/.
       If you have a bug report for this manual page, see
       http://sourceware.org/bugzilla/enter_bug.cgi?product=binutils.
       This page was obtained from the tarball binutils-2.42.tar.gz
       fetched from https://ftp.gnu.org/gnu/binutils/ on 2024-06-14.
       If you discover any rendering problems in this HTML version of
       the page, or you believe there is a better or more up-to-date
       source for the page, or you have corrections or improvements to
       the information in this COLOPHON (which is not part of the
       original manual page), send a mail to man-pages@man7.org

binutils-2.42                  2024-06-14                          NM(1)\end{lstlisting}
}}
\endinput  %  ==  ==  ==  ==  ==  ==  ==  ==  ==

\subsection{\refNm: List Symbols From Object Files}

{\tiny{
\begin{lstlisting}[language=bash]
NAME
       nm - list symbols from object files
SYNOPSIS
       nm [-A|-o|--print-file-name]
          [-a|--debug-syms]
          [-B|--format=bsd]
          [-C|--demangle[=style]]
          [-D|--dynamic]
          [-fformat|--format=format]
          [-g|--extern-only]
          [-h|--help]
          [--ifunc-chars=CHARS]
          [-j|--format=just-symbols]
          [-l|--line-numbers] [--inlines]
          [-n|-v|--numeric-sort]
          [-P|--portability]
          [-p|--no-sort]
          [-r|--reverse-sort]
          [-S|--print-size]
          [-s|--print-armap]
          [-t radix|--radix=radix]
          [-u|--undefined-only]
          [-U|--defined-only]
          [-V|--version]
          [-W|--no-weak]
          [-X 32_64]
          [--no-demangle]
          [--no-recurse-limit|--recurse-limit]]
          [--plugin name]
          [--size-sort]
          [--special-syms]
          [--synthetic]
          [--target=bfdname]
          [--unicode=method]
          [--with-symbol-versions]
          [--without-symbol-versions]
          [objfile...]
DESCRIPTION
       GNU nm lists the symbols from object files objfile....  If no
       object files are listed as arguments, nm assumes the file a.out.

       For each symbol, nm shows:

       *   The symbol value, in the radix selected by options (see
           below), or hexadecimal by default.

       *   The symbol type.  At least the following types are used;
           others are, as well, depending on the object file format.  If
           lowercase, the symbol is usually local; if uppercase, the
           symbol is global (external).  There are however a few
           lowercase symbols that are shown for special global symbols
           ("u", "v" and "w").

           "A" The symbol's value is absolute, and will not be changed
               by further linking.

           "B"
           "b" The symbol is in the BSS data section.  This section
               typically contains zero-initialized or uninitialized
               data, although the exact behavior is system dependent.

           "C"
           "c" The symbol is common.  Common symbols are uninitialized
               data.  When linking, multiple common symbols may appear
               with the same name.  If the symbol is defined anywhere,
               the common symbols are treated as undefined references.
               The lower case c character is used when the symbol is in
               a special section for small commons.

           "D"
           "d" The symbol is in the initialized data section.

           "G"
           "g" The symbol is in an initialized data section for small
               objects.  Some object file formats permit more efficient
               access to small data objects, such as a global int
               variable as opposed to a large global array.

           "i" For PE format files this indicates that the symbol is in
               a section specific to the implementation of DLLs.

               For ELF format files this indicates that the symbol is an
               indirect function.  This is a GNU extension to the
               standard set of ELF symbol types.  It indicates a symbol
               which if referenced by a relocation does not evaluate to
               its address, but instead must be invoked at runtime.  The
               runtime execution will then return the value to be used
               in the relocation.

               Note - the actual symbols display for GNU indirect
               symbols is controlled by the --ifunc-chars command line
               option.  If this option has been provided then the first
               character in the string will be used for global indirect
               function symbols.  If the string contains a second
               character then that will be used for local indirect
               function symbols.

           "I" The symbol is an indirect reference to another symbol.

           "N" The symbol is a debugging symbol.

           "n" The symbol is in a non-data, non-code, non-debug read-
               only section.

           "p" The symbol is in a stack unwind section.

           "R"
           "r" The symbol is in a read only data section.

           "S"
           "s" The symbol is in an uninitialized or zero-initialized
               data section for small objects.

           "T"
           "t" The symbol is in the text (code) section.

           "U" The symbol is undefined.

           "u" The symbol is a unique global symbol.  This is a GNU
               extension to the standard set of ELF symbol bindings.
               For such a symbol the dynamic linker will make sure that
               in the entire process there is just one symbol with this
               name and type in use.

           "V"
           "v" The symbol is a weak object.  When a weak defined symbol
               is linked with a normal defined symbol, the normal
               defined symbol is used with no error.  When a weak
               undefined symbol is linked and the symbol is not defined,
               the value of the weak symbol becomes zero with no error.
               On some systems, uppercase indicates that a default value
               has been specified.

           "W"
           "w" The symbol is a weak symbol that has not been
               specifically tagged as a weak object symbol.  When a weak
               defined symbol is linked with a normal defined symbol,
               the normal defined symbol is used with no error.  When a
               weak undefined symbol is linked and the symbol is not
               defined, the value of the symbol is determined in a
               system-specific manner without error.  On some systems,
               uppercase indicates that a default value has been
               specified.

           "-" The symbol is a stabs symbol in an a.out object file.  In
               this case, the next values printed are the stabs other
               field, the stabs desc field, and the stab type.  Stabs
               symbols are used to hold debugging information.

           "?" The symbol type is unknown, or object file format
               specific.

       *   The symbol name.  If a symbol has version information
           associated with it, then the version information is displayed
           as well.  If the versioned symbol is undefined or hidden from
           linker, the version string is displayed as a suffix to the
           symbol name, preceded by an @ character.  For example
           foo@VER_1.  If the version is the default version to be used
           when resolving unversioned references to the symbol, then it
           is displayed as a suffix preceded by two @ characters.  For
           example foo@@VER_2.
OPTIONS
       The long and short forms of options, shown here as alternatives,
       are equivalent.

       -A
       -o
       --print-file-name
           Precede each symbol by the name of the input file (or archive
           member) in which it was found, rather than identifying the
           input file once only, before all of its symbols.

       -a
       --debug-syms
           Display all symbols, even debugger-only symbols; normally
           these are not listed.

       -B  The same as --format=bsd (for compatibility with the MIPS
           nm).

       -C
       --demangle[=style]
           Decode (demangle) low-level symbol names into user-level
           names.  Besides removing any initial underscore prepended by
           the system, this makes C++ function names readable. Different
           compilers have different mangling styles. The optional
           demangling style argument can be used to choose an
           appropriate demangling style for your compiler.

       --no-demangle
           Do not demangle low-level symbol names.  This is the default.

       --recurse-limit
       --no-recurse-limit
       --recursion-limit
       --no-recursion-limit
           Enables or disables a limit on the amount of recursion
           performed whilst demangling strings.  Since the name mangling
           formats allow for an infinite level of recursion it is
           possible to create strings whose decoding will exhaust the
           amount of stack space available on the host machine,
           triggering a memory fault.  The limit tries to prevent this
           from happening by restricting recursion to 2048 levels of
           nesting.

           The default is for this limit to be enabled, but disabling it
           may be necessary in order to demangle truly complicated
           names.  Note however that if the recursion limit is disabled
           then stack exhaustion is possible and any bug reports about
           such an event will be rejected.

       -D
       --dynamic
           Display the dynamic symbols rather than the normal symbols.
           This is only meaningful for dynamic objects, such as certain
           types of shared libraries.

       -f format
       --format=format
           Use the output format format, which can be "bsd", "sysv",
           "posix" or "just-symbols".  The default is "bsd".  Only the
           first character of format is significant; it can be either
           upper or lower case.

       -g
       --extern-only
           Display only external symbols.

       -h
       --help
           Show a summary of the options to nm and exit.

       --ifunc-chars=CHARS
           When display GNU indirect function symbols nm will default to
           using the "i" character for both local indirect functions and
           global indirect functions.  The --ifunc-chars option allows
           the user to specify a string containing one or two
           characters. The first character will be used for global
           indirect function symbols and the second character, if
           present, will be used for local indirect function symbols.

       j   The same as --format=just-symbols.

       -l
       --line-numbers
           For each symbol, use debugging information to try to find a
           filename and line number.  For a defined symbol, look for the
           line number of the address of the symbol.  For an undefined
           symbol, look for the line number of a relocation entry which
           refers to the symbol.  If line number information can be
           found, print it after the other symbol information.

       --inlines
           When option -l is active, if the address belongs to a
           function that was inlined, then this option causes the source
           information for all enclosing scopes back to the first non-
           inlined function to be printed as well.  For example, if
           "main" inlines "callee1" which inlines "callee2", and address
           is from "callee2", the source information for "callee1" and
           "main" will also be printed.

       -n
       -v
       --numeric-sort
           Sort symbols numerically by their addresses, rather than
           alphabetically by their names.

       -p
       --no-sort
           Do not bother to sort the symbols in any order; print them in
           the order encountered.

       -P
       --portability
           Use the POSIX.2 standard output format instead of the default
           format.  Equivalent to -f posix.

       -r
       --reverse-sort
           Reverse the order of the sort (whether numeric or
           alphabetic); let the last come first.

       -S
       --print-size
           Print both value and size of defined symbols for the "bsd"
           output style.  This option has no effect for object formats
           that do not record symbol sizes, unless --size-sort is also
           used in which case a calculated size is displayed.

       -s
       --print-armap
           When listing symbols from archive members, include the index:
           a mapping (stored in the archive by ar or ranlib) of which
           modules contain definitions for which names.

       -t radix
       --radix=radix
           Use radix as the radix for printing the symbol values.  It
           must be d for decimal, o for octal, or x for hexadecimal.

       -u
       --undefined-only
           Display only undefined symbols (those external to each object
           file).  By default both defined and undefined symbols are
           displayed.

       -U
       --defined-only
           Display only defined symbols for each object file.  By
           default both defined and undefined symbols are displayed.

       -V
       --version
           Show the version number of nm and exit.

       -X  This option is ignored for compatibility with the AIX version
           of nm.  It takes one parameter which must be the string
           32_64.  The default mode of AIX nm corresponds to -X 32,
           which is not supported by GNU nm.

       --plugin name
           Load the plugin called name to add support for extra target
           types.  This option is only available if the toolchain has
           been built with plugin support enabled.

           If --plugin is not provided, but plugin support has been
           enabled then nm iterates over the files in
           ${libdir}/bfd-plugins in alphabetic order and the first
           plugin that claims the object in question is used.

           Please note that this plugin search directory is not the one
           used by ld's -plugin option.  In order to make nm use the
           linker plugin it must be copied into the
           ${libdir}/bfd-plugins directory.  For GCC based compilations
           the linker plugin is called liblto_plugin.so.0.0.0.  For
           Clang based compilations it is called LLVMgold.so.  The GCC
           plugin is always backwards compatible with earlier versions,
           so it is sufficient to just copy the newest one.

       --size-sort
           Sort symbols by size.  For ELF objects symbol sizes are read
           from the ELF, for other object types the symbol sizes are
           computed as the difference between the value of the symbol
           and the value of the symbol with the next higher value.  If
           the "bsd" output format is used the size of the symbol is
           printed, rather than the value, and -S must be used in order
           both size and value to be printed.

           Note - this option does not work if --undefined-only has been
           enabled as undefined symbols have no size.

       --special-syms
           Display symbols which have a target-specific special meaning.
           These symbols are usually used by the target for some special
           processing and are not normally helpful when included in the
           normal symbol lists.  For example for ARM targets this option
           would skip the mapping symbols used to mark transitions
           between ARM code, THUMB code and data.

       --synthetic
           Include synthetic symbols in the output.  These are special
           symbols created by the linker for various purposes.  They are
           not shown by default since they are not part of the binary's
           original source code.

       --unicode=[default|invalid|locale|escape|hex|highlight]
           Controls the display of UTF-8 encoded multibyte characters in
           strings.  The default (--unicode=default) is to give them no
           special treatment.  The --unicode=locale option displays the
           sequence in the current locale, which may or may not support
           them.  The options --unicode=hex and --unicode=invalid
           display them as hex byte sequences enclosed by either angle
           brackets or curly braces.

           The --unicode=escape option displays them as escape sequences
           (\uxxxx) and the --unicode=highlight option displays them as
           escape sequences highlighted in red (if supported by the
           output device).  The colouring is intended to draw attention
           to the presence of unicode sequences where they might not be
           expected.

       -W
       --no-weak
           Do not display weak symbols.

       --with-symbol-versions
       --without-symbol-versions
           Enables or disables the display of symbol version
           information.  The version string is displayed as a suffix to
           the symbol name, preceded by an @ character.  For example
           foo@VER_1.  If the version is the default version to be used
           when resolving unversioned references to the symbol then it
           is displayed as a suffix preceded by two @ characters.  For
           example foo@@VER_2.  By default, symbol version information
           is displayed.

       --target=bfdname
           Specify an object code format other than your system's
           default format.

       @file
           Read command-line options from file.  The options read are
           inserted in place of the original @file option.  If file does
           not exist, or cannot be read, then the option will be treated
           literally, and not removed.

           Options in file are separated by whitespace.  A whitespace
           character may be included in an option by surrounding the
           entire option in either single or double quotes.  Any
           character (including a backslash) may be included by
           prefixing the character to be included with a backslash.  The
           file may itself contain additional @file options; any such
           options will be processed recursively.
SEE ALSO
       ar(1), objdump(1), ranlib(1), and the Info entries for binutils.
COPYRIGHT
       Copyright (c) 1991-2024 Free Software Foundation, Inc.

       Permission is granted to copy, distribute and/or modify this
       document under the terms of the GNU Free Documentation License,
       Version 1.3 or any later version published by the Free Software
       Foundation; with no Invariant Sections, with no Front-Cover
       Texts, and with no Back-Cover Texts.  A copy of the license is
       included in the section entitled "GNU Free Documentation
       License".
COLOPHON
       This page is part of the binutils (a collection of tools for
       working with executable binaries) project.  Information about the
       project can be found at http://www.gnu.org/software/binutils/.
       If you have a bug report for this manual page, see
       http://sourceware.org/bugzilla/enter_bug.cgi?product=binutils.
       This page was obtained from the tarball binutils-2.42.tar.gz
       fetched from https://ftp.gnu.org/gnu/binutils/ on 2024-06-14.
       If you discover any rendering problems in this HTML version of
       the page, or you believe there is a better or more up-to-date
       source for the page, or you have corrections or improvements to
       the information in this COLOPHON (which is not part of the
       original manual page), send a mail to man-pages@man7.org

binutils-2.42                  2024-06-14                          NM(1)\end{lstlisting}
}}
\endinput  %  ==  ==  ==  ==  ==  ==  ==  ==  ==

\subsection{\refNm: List Symbols From Object Files}

{\tiny{
\begin{lstlisting}[language=bash]
NAME
       nm - list symbols from object files
SYNOPSIS
       nm [-A|-o|--print-file-name]
          [-a|--debug-syms]
          [-B|--format=bsd]
          [-C|--demangle[=style]]
          [-D|--dynamic]
          [-fformat|--format=format]
          [-g|--extern-only]
          [-h|--help]
          [--ifunc-chars=CHARS]
          [-j|--format=just-symbols]
          [-l|--line-numbers] [--inlines]
          [-n|-v|--numeric-sort]
          [-P|--portability]
          [-p|--no-sort]
          [-r|--reverse-sort]
          [-S|--print-size]
          [-s|--print-armap]
          [-t radix|--radix=radix]
          [-u|--undefined-only]
          [-U|--defined-only]
          [-V|--version]
          [-W|--no-weak]
          [-X 32_64]
          [--no-demangle]
          [--no-recurse-limit|--recurse-limit]]
          [--plugin name]
          [--size-sort]
          [--special-syms]
          [--synthetic]
          [--target=bfdname]
          [--unicode=method]
          [--with-symbol-versions]
          [--without-symbol-versions]
          [objfile...]
DESCRIPTION
       GNU nm lists the symbols from object files objfile....  If no
       object files are listed as arguments, nm assumes the file a.out.

       For each symbol, nm shows:

       *   The symbol value, in the radix selected by options (see
           below), or hexadecimal by default.

       *   The symbol type.  At least the following types are used;
           others are, as well, depending on the object file format.  If
           lowercase, the symbol is usually local; if uppercase, the
           symbol is global (external).  There are however a few
           lowercase symbols that are shown for special global symbols
           ("u", "v" and "w").

           "A" The symbol's value is absolute, and will not be changed
               by further linking.

           "B"
           "b" The symbol is in the BSS data section.  This section
               typically contains zero-initialized or uninitialized
               data, although the exact behavior is system dependent.

           "C"
           "c" The symbol is common.  Common symbols are uninitialized
               data.  When linking, multiple common symbols may appear
               with the same name.  If the symbol is defined anywhere,
               the common symbols are treated as undefined references.
               The lower case c character is used when the symbol is in
               a special section for small commons.

           "D"
           "d" The symbol is in the initialized data section.

           "G"
           "g" The symbol is in an initialized data section for small
               objects.  Some object file formats permit more efficient
               access to small data objects, such as a global int
               variable as opposed to a large global array.

           "i" For PE format files this indicates that the symbol is in
               a section specific to the implementation of DLLs.

               For ELF format files this indicates that the symbol is an
               indirect function.  This is a GNU extension to the
               standard set of ELF symbol types.  It indicates a symbol
               which if referenced by a relocation does not evaluate to
               its address, but instead must be invoked at runtime.  The
               runtime execution will then return the value to be used
               in the relocation.

               Note - the actual symbols display for GNU indirect
               symbols is controlled by the --ifunc-chars command line
               option.  If this option has been provided then the first
               character in the string will be used for global indirect
               function symbols.  If the string contains a second
               character then that will be used for local indirect
               function symbols.

           "I" The symbol is an indirect reference to another symbol.

           "N" The symbol is a debugging symbol.

           "n" The symbol is in a non-data, non-code, non-debug read-
               only section.

           "p" The symbol is in a stack unwind section.

           "R"
           "r" The symbol is in a read only data section.

           "S"
           "s" The symbol is in an uninitialized or zero-initialized
               data section for small objects.

           "T"
           "t" The symbol is in the text (code) section.

           "U" The symbol is undefined.

           "u" The symbol is a unique global symbol.  This is a GNU
               extension to the standard set of ELF symbol bindings.
               For such a symbol the dynamic linker will make sure that
               in the entire process there is just one symbol with this
               name and type in use.

           "V"
           "v" The symbol is a weak object.  When a weak defined symbol
               is linked with a normal defined symbol, the normal
               defined symbol is used with no error.  When a weak
               undefined symbol is linked and the symbol is not defined,
               the value of the weak symbol becomes zero with no error.
               On some systems, uppercase indicates that a default value
               has been specified.

           "W"
           "w" The symbol is a weak symbol that has not been
               specifically tagged as a weak object symbol.  When a weak
               defined symbol is linked with a normal defined symbol,
               the normal defined symbol is used with no error.  When a
               weak undefined symbol is linked and the symbol is not
               defined, the value of the symbol is determined in a
               system-specific manner without error.  On some systems,
               uppercase indicates that a default value has been
               specified.

           "-" The symbol is a stabs symbol in an a.out object file.  In
               this case, the next values printed are the stabs other
               field, the stabs desc field, and the stab type.  Stabs
               symbols are used to hold debugging information.

           "?" The symbol type is unknown, or object file format
               specific.

       *   The symbol name.  If a symbol has version information
           associated with it, then the version information is displayed
           as well.  If the versioned symbol is undefined or hidden from
           linker, the version string is displayed as a suffix to the
           symbol name, preceded by an @ character.  For example
           foo@VER_1.  If the version is the default version to be used
           when resolving unversioned references to the symbol, then it
           is displayed as a suffix preceded by two @ characters.  For
           example foo@@VER_2.
OPTIONS
       The long and short forms of options, shown here as alternatives,
       are equivalent.

       -A
       -o
       --print-file-name
           Precede each symbol by the name of the input file (or archive
           member) in which it was found, rather than identifying the
           input file once only, before all of its symbols.

       -a
       --debug-syms
           Display all symbols, even debugger-only symbols; normally
           these are not listed.

       -B  The same as --format=bsd (for compatibility with the MIPS
           nm).

       -C
       --demangle[=style]
           Decode (demangle) low-level symbol names into user-level
           names.  Besides removing any initial underscore prepended by
           the system, this makes C++ function names readable. Different
           compilers have different mangling styles. The optional
           demangling style argument can be used to choose an
           appropriate demangling style for your compiler.

       --no-demangle
           Do not demangle low-level symbol names.  This is the default.

       --recurse-limit
       --no-recurse-limit
       --recursion-limit
       --no-recursion-limit
           Enables or disables a limit on the amount of recursion
           performed whilst demangling strings.  Since the name mangling
           formats allow for an infinite level of recursion it is
           possible to create strings whose decoding will exhaust the
           amount of stack space available on the host machine,
           triggering a memory fault.  The limit tries to prevent this
           from happening by restricting recursion to 2048 levels of
           nesting.

           The default is for this limit to be enabled, but disabling it
           may be necessary in order to demangle truly complicated
           names.  Note however that if the recursion limit is disabled
           then stack exhaustion is possible and any bug reports about
           such an event will be rejected.

       -D
       --dynamic
           Display the dynamic symbols rather than the normal symbols.
           This is only meaningful for dynamic objects, such as certain
           types of shared libraries.

       -f format
       --format=format
           Use the output format format, which can be "bsd", "sysv",
           "posix" or "just-symbols".  The default is "bsd".  Only the
           first character of format is significant; it can be either
           upper or lower case.

       -g
       --extern-only
           Display only external symbols.

       -h
       --help
           Show a summary of the options to nm and exit.

       --ifunc-chars=CHARS
           When display GNU indirect function symbols nm will default to
           using the "i" character for both local indirect functions and
           global indirect functions.  The --ifunc-chars option allows
           the user to specify a string containing one or two
           characters. The first character will be used for global
           indirect function symbols and the second character, if
           present, will be used for local indirect function symbols.

       j   The same as --format=just-symbols.

       -l
       --line-numbers
           For each symbol, use debugging information to try to find a
           filename and line number.  For a defined symbol, look for the
           line number of the address of the symbol.  For an undefined
           symbol, look for the line number of a relocation entry which
           refers to the symbol.  If line number information can be
           found, print it after the other symbol information.

       --inlines
           When option -l is active, if the address belongs to a
           function that was inlined, then this option causes the source
           information for all enclosing scopes back to the first non-
           inlined function to be printed as well.  For example, if
           "main" inlines "callee1" which inlines "callee2", and address
           is from "callee2", the source information for "callee1" and
           "main" will also be printed.

       -n
       -v
       --numeric-sort
           Sort symbols numerically by their addresses, rather than
           alphabetically by their names.

       -p
       --no-sort
           Do not bother to sort the symbols in any order; print them in
           the order encountered.

       -P
       --portability
           Use the POSIX.2 standard output format instead of the default
           format.  Equivalent to -f posix.

       -r
       --reverse-sort
           Reverse the order of the sort (whether numeric or
           alphabetic); let the last come first.

       -S
       --print-size
           Print both value and size of defined symbols for the "bsd"
           output style.  This option has no effect for object formats
           that do not record symbol sizes, unless --size-sort is also
           used in which case a calculated size is displayed.

       -s
       --print-armap
           When listing symbols from archive members, include the index:
           a mapping (stored in the archive by ar or ranlib) of which
           modules contain definitions for which names.

       -t radix
       --radix=radix
           Use radix as the radix for printing the symbol values.  It
           must be d for decimal, o for octal, or x for hexadecimal.

       -u
       --undefined-only
           Display only undefined symbols (those external to each object
           file).  By default both defined and undefined symbols are
           displayed.

       -U
       --defined-only
           Display only defined symbols for each object file.  By
           default both defined and undefined symbols are displayed.

       -V
       --version
           Show the version number of nm and exit.

       -X  This option is ignored for compatibility with the AIX version
           of nm.  It takes one parameter which must be the string
           32_64.  The default mode of AIX nm corresponds to -X 32,
           which is not supported by GNU nm.

       --plugin name
           Load the plugin called name to add support for extra target
           types.  This option is only available if the toolchain has
           been built with plugin support enabled.

           If --plugin is not provided, but plugin support has been
           enabled then nm iterates over the files in
           ${libdir}/bfd-plugins in alphabetic order and the first
           plugin that claims the object in question is used.

           Please note that this plugin search directory is not the one
           used by ld's -plugin option.  In order to make nm use the
           linker plugin it must be copied into the
           ${libdir}/bfd-plugins directory.  For GCC based compilations
           the linker plugin is called liblto_plugin.so.0.0.0.  For
           Clang based compilations it is called LLVMgold.so.  The GCC
           plugin is always backwards compatible with earlier versions,
           so it is sufficient to just copy the newest one.

       --size-sort
           Sort symbols by size.  For ELF objects symbol sizes are read
           from the ELF, for other object types the symbol sizes are
           computed as the difference between the value of the symbol
           and the value of the symbol with the next higher value.  If
           the "bsd" output format is used the size of the symbol is
           printed, rather than the value, and -S must be used in order
           both size and value to be printed.

           Note - this option does not work if --undefined-only has been
           enabled as undefined symbols have no size.

       --special-syms
           Display symbols which have a target-specific special meaning.
           These symbols are usually used by the target for some special
           processing and are not normally helpful when included in the
           normal symbol lists.  For example for ARM targets this option
           would skip the mapping symbols used to mark transitions
           between ARM code, THUMB code and data.

       --synthetic
           Include synthetic symbols in the output.  These are special
           symbols created by the linker for various purposes.  They are
           not shown by default since they are not part of the binary's
           original source code.

       --unicode=[default|invalid|locale|escape|hex|highlight]
           Controls the display of UTF-8 encoded multibyte characters in
           strings.  The default (--unicode=default) is to give them no
           special treatment.  The --unicode=locale option displays the
           sequence in the current locale, which may or may not support
           them.  The options --unicode=hex and --unicode=invalid
           display them as hex byte sequences enclosed by either angle
           brackets or curly braces.

           The --unicode=escape option displays them as escape sequences
           (\uxxxx) and the --unicode=highlight option displays them as
           escape sequences highlighted in red (if supported by the
           output device).  The colouring is intended to draw attention
           to the presence of unicode sequences where they might not be
           expected.

       -W
       --no-weak
           Do not display weak symbols.

       --with-symbol-versions
       --without-symbol-versions
           Enables or disables the display of symbol version
           information.  The version string is displayed as a suffix to
           the symbol name, preceded by an @ character.  For example
           foo@VER_1.  If the version is the default version to be used
           when resolving unversioned references to the symbol then it
           is displayed as a suffix preceded by two @ characters.  For
           example foo@@VER_2.  By default, symbol version information
           is displayed.

       --target=bfdname
           Specify an object code format other than your system's
           default format.

       @file
           Read command-line options from file.  The options read are
           inserted in place of the original @file option.  If file does
           not exist, or cannot be read, then the option will be treated
           literally, and not removed.

           Options in file are separated by whitespace.  A whitespace
           character may be included in an option by surrounding the
           entire option in either single or double quotes.  Any
           character (including a backslash) may be included by
           prefixing the character to be included with a backslash.  The
           file may itself contain additional @file options; any such
           options will be processed recursively.
SEE ALSO
       ar(1), objdump(1), ranlib(1), and the Info entries for binutils.
COPYRIGHT
       Copyright (c) 1991-2024 Free Software Foundation, Inc.

       Permission is granted to copy, distribute and/or modify this
       document under the terms of the GNU Free Documentation License,
       Version 1.3 or any later version published by the Free Software
       Foundation; with no Invariant Sections, with no Front-Cover
       Texts, and with no Back-Cover Texts.  A copy of the license is
       included in the section entitled "GNU Free Documentation
       License".
COLOPHON
       This page is part of the binutils (a collection of tools for
       working with executable binaries) project.  Information about the
       project can be found at http://www.gnu.org/software/binutils/.
       If you have a bug report for this manual page, see
       http://sourceware.org/bugzilla/enter_bug.cgi?product=binutils.
       This page was obtained from the tarball binutils-2.42.tar.gz
       fetched from https://ftp.gnu.org/gnu/binutils/ on 2024-06-14.
       If you discover any rendering problems in this HTML version of
       the page, or you believe there is a better or more up-to-date
       source for the page, or you have corrections or improvements to
       the information in this COLOPHON (which is not part of the
       original manual page), send a mail to man-pages@man7.org

binutils-2.42                  2024-06-14                          NM(1)\end{lstlisting}
}}
\endinput  %  ==  ==  ==  ==  ==  ==  ==  ==  ==

	% % % \input{./components/man/man-strace}
\subsection{\refStrace: Trace System Calls and Signals}

{\tiny{
\begin{lstlisting}[language=bash]
NAME
       strace - trace system calls and signals
SYNOPSIS
       strace [-ACdffhikkqqrtttTvVwxxyyYzZ] [-a column] [-b execve]
              [-e expr]... [-I n] [-o file] [-O overhead] [-p pid]...
              [-P path]... [-s strsize] [-S sortby] [-U columns]
              [-X format] [--seccomp-bpf]
              [--stack-trace-frame-limit=limit] [--syscall-limit=limit]
              [--secontext[=format]] [--tips[=format]] { -p pid | [-DDD]
              [-E var[=val]]... [-u username] command [args] }

       strace -c [-dfwzZ] [-b execve] [-e expr]... [-I n] [-O overhead]
              [-p pid]... [-P path]... [-S sortby] [-U columns]
              [--seccomp-bpf] [--syscall-limit=limit] [--tips[=format]]
              { -p pid | [-DDD] [-E var[=val]]... [-u username] command
              [args] }

       strace --tips[=format]
DESCRIPTION
       In the simplest case strace runs the specified command until it
       exits.  It intercepts and records the system calls which are
       called by a process and the signals which are received by a
       process.  The name of each system call, its arguments and its
       return value are printed on standard error or to the file
       specified with the -o option.

       strace is a useful diagnostic, instructional, and debugging tool.
       System administrators, diagnosticians and trouble-shooters will
       find it invaluable for solving problems with programs for which
       the source is not readily available since they do not need to be
       recompiled in order to trace them.  Students, hackers and the
       overly-curious will find that a great deal can be learned about a
       system and its system calls by tracing even ordinary programs.
       And programmers will find that since system calls and signals are
       events that happen at the user/kernel interface, a close
       examination of this boundary is very useful for bug isolation,
       sanity checking and attempting to capture race conditions.

       Each line in the trace contains the system call name, followed by
       its arguments in parentheses and its return value.  An example
       from stracing the command "cat /dev/null" is:

           open("/dev/null", O_RDONLY) = 3

       Errors (typically a return value of -1) have the errno symbol and
       error string appended.

           open("/foo/bar", O_RDONLY) = -1 ENOENT (No such file or directory)

       Signals are printed as signal symbol and decoded siginfo
       structure.  An excerpt from stracing and interrupting the command
       "sleep 666" is:

           sigsuspend([] <unfinished ...>
           --- SIGINT {si_signo=SIGINT, si_code=SI_USER, si_pid=...} ---
           +++ killed by SIGINT +++

       If a system call is being executed and meanwhile another one is
       being called from a different thread/process then strace will try
       to preserve the order of those events and mark the ongoing call
       as being unfinished.  When the call returns it will be marked as
       resumed.

           [pid 28772] select(4, [3], NULL, NULL, NULL <unfinished ...>
           [pid 28779] clock_gettime(CLOCK_REALTIME, {tv_sec=1130322148, tv_nsec=3977000}) = 0
           [pid 28772] <... select resumed> )      = 1 (in [3])

       Interruption of a (restartable) system call by a signal delivery
       is processed differently as kernel terminates the system call and
       also arranges its immediate reexecution after the signal handler
       completes.

           read(0, 0x7ffff72cf5cf, 1)              = ? ERESTARTSYS (To be restarted)
           --- SIGALRM {si_signo=SIGALRM, si_code=SI_KERNEL} ---
           rt_sigreturn({mask=[]})                 = 0
           read(0, "", 1)                          = 0

       Arguments are printed in symbolic form with passion.  This
       example shows the shell performing ">>xyzzy" output redirection:

           open("xyzzy", O_WRONLY|O_APPEND|O_CREAT, 0666) = 3

       Here, the second and the third argument of open(2) are decoded by
       breaking down the flag argument into its three bitwise-OR
       constituents and printing the mode value in octal by tradition.
       Where the traditional or native usage differs from ANSI or POSIX,
       the latter forms are preferred.  In some cases, strace output is
       proven to be more readable than the source.

       Structure pointers are dereferenced and the members are displayed
       as appropriate.  In most cases, arguments are formatted in the
       most C-like fashion possible.  For example, the essence of the
       command "ls -l /dev/null" is captured as:

           lstat("/dev/null", {st_mode=S_IFCHR|0666, st_rdev=makedev(0x1, 0x3), ...}) = 0

       Notice how the 'struct stat' argument is dereferenced and how
       each member is displayed symbolically.  In particular, observe
       how the st_mode member is carefully decoded into a bitwise-OR of
       symbolic and numeric values.  Also notice in this example that
       the first argument to lstat(2) is an input to the system call and
       the second argument is an output.  Since output arguments are not
       modified if the system call fails, arguments may not always be
       dereferenced.  For example, retrying the "ls -l" example with a
       non-existent file produces the following line:

           lstat("/foo/bar", 0xb004) = -1 ENOENT (No such file or directory)

       In this case the porch light is on but nobody is home.

       Syscalls unknown to strace are printed raw, with the unknown
       system call number printed in hexadecimal form and prefixed with
       "syscall_":

           syscall_0xbad(0x1, 0x2, 0x3, 0x4, 0x5, 0x6) = -1 ENOSYS (Function not implemented)

       Character pointers are dereferenced and printed as C strings.
       Non-printing characters in strings are normally represented by
       ordinary C escape codes.  Only the first strsize (32 by default)
       bytes of strings are printed; longer strings have an ellipsis
       appended following the closing quote.  Here is a line from "ls
       -l" where the getpwuid(3) library routine is reading the password
       file:

           read(3, "root::0:0:System Administrator:/"..., 1024) = 422

       While structures are annotated using curly braces, pointers to
       basic types and arrays are printed using square brackets with
       commas separating the elements.  Here is an example from the
       command id(1) on a system with supplementary group ids:

           getgroups(32, [100, 0]) = 2

       On the other hand, bit-sets are also shown using square brackets,
       but set elements are separated only by a space.  Here is the
       shell, preparing to execute an external command:

           sigprocmask(SIG_BLOCK, [CHLD TTOU], []) = 0

       Here, the second argument is a bit-set of two signals, SIGCHLD
       and SIGTTOU.  In some cases, the bit-set is so full that printing
       out the unset elements is more valuable.  In that case, the bit-
       set is prefixed by a tilde like this:

           sigprocmask(SIG_UNBLOCK, ~[], NULL) = 0

       Here, the second argument represents the full set of all signals.
OPTIONS
   General
       -e expr
              A qualifying expression which modifies which events to
              trace or how to trace them.  The format of the expression
              is:

                             [qualifier=][!]value[,value]...

              where qualifier is one of trace (or t), trace-fds (or
              trace-fd or fd or fds), abbrev (or a), verbose (or v), raw
              (or x), signal (or signals or s), read (or reads or r),
              write (or writes or w), fault, inject, status, quiet (or
              silent or silence or q), secontext, decode-fds (or
              decode-fd), decode-pids (or decode-pid), or kvm, and value
              is a qualifier-dependent symbol or number.  The default
              qualifier is trace.  Using an exclamation mark negates the
              set of values.  For example, -e open means literally
              -e trace=open which in turn means trace only the open
              system call.  By contrast, -e trace=!open means to trace
              every system call except open.  In addition, the special
              values all and none have the obvious meanings.

              Note that some shells use the exclamation point for
              history expansion even inside quoted arguments.  If so,
              you must escape the exclamation point with a backslash.

   Startup
       -E var=val
       --env=var=val
              Run command with var=val in its list of environment
              variables.

       -E var
       --env=var
              Remove var from the inherited list of environment
              variables before passing it on to the command.

       -p pid
       --attach=pid
              Attach to the process with the process ID pid and begin
              tracing.  The trace may be terminated at any time by a
              keyboard interrupt signal (CTRL-C).  strace will respond
              by detaching itself from the traced process(es) leaving it
              (them) to continue running.  Multiple -p options can be
              used to attach to many processes in addition to command
              (which is optional if at least one -p option is given).
              Multiple process IDs, separated by either comma (",''),
              space (" "), tab, or newline character, can be provided as
              an argument to a single -p option, so, for example, -p
              "$(pidof PROG)" and -p "$(pgrep PROG)" syntaxes are
              supported.

       -u username
       --user=username
              Run command with the user ID, group ID, and supplementary
              groups of username.  This option is only useful when
              running as root and enables the correct execution of
              setuid and/or setgid binaries.  Unless this option is used
              setuid and setgid programs are executed without effective
              privileges.
       -u UID:GID
       --user=UID:GID
              Alternative syntax where the program is started with
              exactly the given user and group IDs, and an empty list of
              supplementary groups.  In this case, user and group name
              lookups are not performed.

       --argv0=name
              Set argv[0] of the command being executed to name.  Useful
              for tracing multi-call executables which interpret
              argv[0], such as busybox or kmod.

   Tracing
       -b syscall
       --detach-on=syscall
              If specified syscall is reached, detach from traced
              process.  Currently, only execve(2) syscall is supported.
              This option is useful if you want to trace multi-threaded
              process and therefore require -f, but don't want to trace
              its (potentially very complex) children.

       -D
       --daemonize
       --daemonize=grandchild
              Run tracer process as a grandchild, not as the parent of
              the tracee.  This reduces the visible effect of strace by
              keeping the tracee a direct child of the calling process.

       -DD
       --daemonize=pgroup
       --daemonize=pgrp
              Run tracer process as tracee's grandchild in a separate
              process group.  In addition to reduction of the visible
              effect of strace, it also avoids killing of strace with
              kill(2) issued to the whole process group.

       -DDD
       --daemonize=session
              Run tracer process as tracee's grandchild in a separate
              session ("true daemonisation").  In addition to reduction
              of the visible effect of strace, it also avoids killing of
              strace upon session termination.

       -f
       --follow-forks
              Trace child processes as they are created by currently
              traced processes as a result of the fork(2), vfork(2) and
              clone(2) system calls.  Note that -p PID -f will attach
              all threads of process PID if it is multi-threaded, not
              only thread with thread_id = PID.

       --output-separately
              If the --output=filename option is in effect, each
              processes trace is written to filename.pid where pid is
              the numeric process id of each process.

       -ff
       --follow-forks --output-separately
              Combine the effects of --follow-forks and
              --output-separately options.  This is incompatible with
              -c, since no per-process counts are kept.

              One might want to consider using strace-log-merge(1) to
              obtain a combined strace log view.

       -I interruptible
       --interruptible=interruptible
              When strace can be interrupted by signals (such as
              pressing CTRL-C).

              1, anywhere
                     no signals are blocked;
              2, waiting
                     fatal signals are blocked while decoding syscall
                     (default);
              3, never
                     fatal signals are always blocked (default if -o
                     FILE PROG);
              4, never_tstp
                     fatal signals and SIGTSTP (CTRL-Z) are always
                     blocked (useful to make strace -o FILE PROG not
                     stop on CTRL-Z, default if -D).

       --syscall-limit=limit
              Detach all tracees when limit number of syscalls have been
              captured. Syscalls filtered out via --trace, --trace-path
              or --status options are not considered when keeping track
              of the number of syscalls that are captured.

       --kill-on-exit
              Apply PTRACE_O_EXITKILL ptrace option to all tracee
              processes (which sends a SIGKILL signal to the tracee if
              the tracer exits) and do not detach them on cleanup so
              they will not be left running after the tracer exit.
              --kill-on-exit is not compatible with -p/--attach options.

   Filtering
       -e trace=syscall_set
       -e t=syscall_set
       --trace=syscall_set
              Trace only the specified set of system calls.  syscall_set
              is defined as [!]value[,value], and value can be one of
              the following:

              syscall
                     Trace specific syscall, specified by its name (see
                     syscalls(2) for a reference, but also see NOTES).

              ?value Question mark before the syscall qualification
                     allows suppression of error in case no syscalls
                     matched the qualification provided.

              value@64
                     Limit the syscall specification described by value
                     to 64-bit personality.

              value@32
                     Limit the syscall specification described by value
                     to 32-bit personality.

              value@x32
                     Limit the syscall specification described by value
                     to x32 personality.

              all    Trace all system calls.

              /regex Trace only those system calls that match the regex.
                     You can use POSIX Extended Regular Expression
                     syntax (see regex(7)).

              %file
              file   Trace all system calls which take a file name as an
                     argument.  You can think of this as an abbreviation
                     for -e trace=open,stat,chmod,unlink,...  which is
                     useful to seeing what files the process is
                     referencing.  Furthermore, using the abbreviation
                     will ensure that you don't accidentally forget to
                     include a call like lstat(2) in the list.  Betchya
                     woulda forgot that one.  The syntax without a
                     preceding percent sign ("-e trace=file") is
                     deprecated.

              %process
              process
                     Trace system calls associated with process
                     lifecycle (creation, exec, termination).  The
                     syntax without a preceding percent sign ("-e
                     trace=process") is deprecated.

              %net
              %network
              network
                     Trace all the network related system calls.  The
                     syntax without a preceding percent sign ("-e
                     trace=network") is deprecated.

              %signal
              signal Trace all signal related system calls.  The syntax
                     without a preceding percent sign ("-e
                     trace=signal") is deprecated.

              %ipc
              ipc    Trace all IPC related system calls.  The syntax
                     without a preceding percent sign ("-e trace=ipc")
                     is deprecated.

              %desc
              desc   Trace all file descriptor related system calls.
                     The syntax without a preceding percent sign ("-e
                     trace=desc") is deprecated.

              %memory
              memory Trace all memory mapping related system calls.  The
                     syntax without a preceding percent sign ("-e
                     trace=memory") is deprecated.

              %creds Trace system calls that read or modify user and
                     group identifiers or capability sets.

              %stat  Trace stat syscall variants.

              %lstat Trace lstat syscall variants.

              %fstat Trace fstat, fstatat, and statx syscall variants.

              %%stat Trace syscalls used for requesting file status
                     (stat, lstat, fstat, fstatat, statx, and their
                     variants).

              %statfs
                     Trace statfs, statfs64, statvfs, osf_statfs, and
                     osf_statfs64 system calls.  The same effect can be
                     achieved with -e trace=/^(.*_)?statv?fs regular
                     expression.

              %fstatfs
                     Trace fstatfs, fstatfs64, fstatvfs, osf_fstatfs,
                     and osf_fstatfs64 system calls.  The same effect
                     can be achieved with -e trace=/fstatv?fs regular
                     expression.

              %%statfs
                     Trace syscalls related to file system statistics
                     (statfs-like, fstatfs-like, and ustat).  The same
                     effect can be achieved with
                     -e trace=/statv?fs|fsstat|ustat regular expression.

              %clock Trace system calls that read or modify system
                     clocks.

              %pure  Trace syscalls that always succeed and have no
                     arguments.  Currently, this list includes
                     arc_gettls(2), getdtablesize(2), getegid(2),
                     getegid32(2), geteuid(2), geteuid32(2), getgid(2),
                     getgid32(2), getpagesize(2), getpgrp(2), getpid(2),
                     getppid(2), get_thread_area(2) (on architectures
                     other than x86), gettid(2), get_tls(2), getuid(2),
                     getuid32(2), getxgid(2), getxpid(2), getxuid(2),
                     kern_features(2), and metag_get_tls(2) syscalls.

              The -c option is useful for determining which system calls
              might be useful to trace.  For example,
              trace=open,close,read,write means to only trace those four
              system calls.  Be careful when making inferences about the
              user/kernel boundary if only a subset of system calls are
              being monitored.  The default is trace=all.

       -e trace-fd=set
       -e trace-fds=set
       -e fd=set
       -e fds=set
       --trace-fds=set
              Trace only the syscalls that operate on the specified
              subset of (non-negative) file descriptors.  Note that
              usage of this option also filters out all the syscalls
              that do not operate on file descriptors at all.  Applies
              in (inclusive) disjunction with the --trace-path option.

       -e signal=set
       -e signals=set
       -e s=set
       --signal=set
              Trace only the specified subset of signals.  The default
              is signal=all.  For example, signal=!SIGIO (or signal=!io)
              causes SIGIO signals not to be traced.

       -e status=set
       --status=set
              Print only system calls with the specified return status.
              The default is status=all.  When using the status
              qualifier, because strace waits for system calls to return
              before deciding whether they should be printed or not, the
              traditional order of events may not be preserved anymore.
              If two system calls are executed by concurrent threads,
              strace will first print both the entry and exit of the
              first system call to exit, regardless of their respective
              entry time.  The entry and exit of the second system call
              to exit will be printed afterwards.  Here is an example
              when select(2) is called, but a different thread calls
              clock_gettime(2) before select(2) finishes:

                  [pid 28779] 1130322148.939977 clock_gettime(CLOCK_REALTIME, {1130322148, 939977000}) = 0
                  [pid 28772] 1130322148.438139 select(4, [3], NULL, NULL, NULL) = 1 (in [3])

              set can include the following elements:

              successful
                     Trace system calls that returned without an error
                     code.  The -z option has the effect of
                     status=successful.
              failed Trace system calls that returned with an error
                     code.  The -Z option has the effect of
                     status=failed.
              unfinished
                     Trace system calls that did not return.  This might
                     happen, for example, due to an execve call in a
                     neighbour thread.
              unavailable
                     Trace system calls that returned but strace failed
                     to fetch the error status.
              detached
                     Trace system calls for which strace detached before
                     the return.

       -P path
       --trace-path=path
              Trace only system calls accessing path.  Multiple -P
              options can be used to specify several paths.  Applies in
              (inclusive) disjunction with the --trace-fds option.

       -z
       --successful-only
              Print only syscalls that returned without an error code.

       -Z
       --failed-only
              Print only syscalls that returned with an error code.

   Output format
       -a column
       --columns=column
              Align return values in a specific column (default column
              40).

       -e abbrev=syscall_set
       -e a=syscall_set
       --abbrev=syscall_set
              Abbreviate the output from printing each member of large
              structures.  The syntax of the syscall_set specification
              is the same as in the -e trace option.  The default is
              abbrev=all.  The -v option has the effect of abbrev=none.

       -e verbose=syscall_set
       -e v=syscall_set
       --verbose=syscall_set
              Dereference structures for the specified set of system
              calls.  The syntax of the syscall_set specification is the
              same as in the -e trace option.  The default is
              verbose=all.

       -e raw=syscall_set
       -e x=syscall_set
       --raw=syscall_set
              Print raw, undecoded arguments for the specified set of
              system calls.  The syntax of the syscall_set specification
              is the same as in the -e trace option.  This option has
              the effect of causing all arguments to be printed in
              hexadecimal.  This is mostly useful if you don't trust the
              decoding or you need to know the actual numeric value of
              an argument.  See also -X raw option.

       -e read=set
       -e reads=set
       -e r=set
       --read=set
              Perform a full hexadecimal and ASCII dump of all the data
              read from file descriptors listed in the specified set.
              For example, to see all input activity on file descriptors
              3 and 5 use -e read=3,5.  Note that this is independent
              from the normal tracing of the read(2) system call which
              is controlled by the option -e trace=read.

       -e write=set
       -e writes=set
       -e w=set
       --write=set
              Perform a full hexadecimal and ASCII dump of all the data
              written to file descriptors listed in the specified set.
              For example, to see all output activity on file
              descriptors 3 and 5 use -e write=3,5.  Note that this is
              independent from the normal tracing of the write(2) system
              call which is controlled by the option -e trace=write.

       -e quiet=set
       -e silent=set
       -e silence=set
       -e q=set
       --quiet=set
       --silent=set
       --silence=set
              Suppress various information messages.  The default is
              quiet=none.  set can include the following elements:

              attach Suppress messages about attaching and detaching ("[
                     Process NNNN attached ]", "[ Process NNNN detached
                     ]").
              exit   Suppress messages about process exits ("+++ exited
                     with SSS +++").
              path-resolution
                     Suppress messages about resolution of paths
                     provided via the -P option ("Requested path "..."
                     resolved into "..."").
              personality
                     Suppress messages about process personality changes
                     ("[ Process PID=NNNN runs in PPP mode. ]").
              thread-execve
              superseded
                     Suppress messages about process being superseded by
                     execve(2) in another thread ("+++ superseded by
                     execve in pid NNNN +++").

       -e decode-fds=set
       --decode-fds=set
              Decode various information associated with file
              descriptors.  The default is decode-fds=none.  set can
              include the following elements:

              path     Print file paths.  Also enables printing of
                       tracee's current working directory when AT_FDCWD
                       constant is used.
              socket   Print socket protocol-specific information,
              dev      Print character/block device numbers.
              pidfd    Print PIDs associated with pidfd file
                       descriptors.
              signalfd Print signal masks associated with signalfd file
                       descriptors.

       -e decode-pids=set
       --decode-pids=set
              Decode various information associated with process IDs
              (and also thread IDs, process group IDs, and session IDs).
              The default is decode-pids=none.  set can include the
              following elements:

              comm    Print command names associated with thread or
                      process IDs.
              pidns   Print thread, process, process group, and session
                      IDs in strace's PID namespace if the tracee is in
                      a different PID namespace.

       -e kvm=vcpu
       --kvm=vcpu
              Print the exit reason of kvm vcpu.  Requires Linux kernel
              version 4.16.0 or higher.

       -i
       --instruction-pointer
              Print the instruction pointer at the time of the system
              call.

       -n
       --syscall-number
              Print the syscall number.

       -k
       --stack-trace[=symbol]
              Print the execution stack trace of the traced processes
              after each system call.

       -kk
       --stack-trace=source
              Print the execution stack trace and source code
              information of the traced processes after each system
              call. This option expects the target program is compiled
              with appropriate debug options: "-g" (gcc), or "-g
              -gdwarf-aranges" (clang).

       --stack-trace-frame-limit=limit
              Print no more than this amount of stack trace frames when
              backtracing a system call (the default is 256).  Use this
              option with the --stack-trace (or -k) option.

       -o filename
       --output=filename
              Write the trace output to the file filename rather than to
              stderr.  filename.pid form is used if -ff option is
              supplied.  If the argument begins with '|' or '!', the
              rest of the argument is treated as a command and all
              output is piped to it.  This is convenient for piping the
              debugging output to a program without affecting the
              redirections of executed programs.  The latter is not
              compatible with -ff option currently.

       -A
       --output-append-mode
              Open the file provided in the -o option in append mode.

       -q
       --quiet
       --quiet=attach,personality
              Suppress messages about attaching, detaching, and
              personality changes.  This happens automatically when
              output is redirected to a file and the command is run
              directly instead of attaching.

       -qq
       --quiet=attach,personality,exit
              Suppress messages attaching, detaching, personality
              changes, and about process exit status.

       -qqq
       --quiet=all
              Suppress all suppressible messages (please refer to the -e
              quiet option description for the full list of suppressible
              messages).

       -r
       --relative-timestamps[=precision]
              Print a relative timestamp upon entry to each system call.
              This records the time difference between the beginning of
              successive system calls.  precision can be one of s (for
              seconds), ms (milliseconds), us (microseconds), or ns
              (nanoseconds), and allows setting the precision of time
              value being printed.  Default is us (microseconds).  Note
              that since -r option uses the monotonic clock time for
              measuring time difference and not the wall clock time, its
              measurements can differ from the difference in time
              reported by the -t option.

       -s strsize
       --string-limit=strsize
              Specify the maximum string size to print (the default is
              32).  Note that filenames are not considered strings and
              are always printed in full.

       --absolute-timestamps[=[[format:]format],[[precision:]precision]]
       --timestamps[=[[format:]format],[[precision:]precision]]
              Prefix each line of the trace with the wall clock time in
              the specified format with the specified precision.  format
              can be one of the following:

              none   No time stamp is printed.  Can be used to override
                     the previous setting.
              time   Wall clock time (strftime(3) format string is %T).
              unix   Number of seconds since the epoch (strftime(3)
                     format string is %s).

              precision can be one of s (for seconds), ms
              (milliseconds), us (microseconds), or ns (nanoseconds).
              Default arguments for the option are
              format:time,precision:s.

       -t
       --absolute-timestamps
              Prefix each line of the trace with the wall clock time.

       -tt
       --absolute-timestamps=precision:us
              If given twice, the time printed will include the
              microseconds.

       -ttt
       --absolute-timestamps=format:unix,precision:us
              If given thrice, the time printed will include the
              microseconds and the leading portion will be printed as
              the number of seconds since the epoch.

       -T
       --syscall-times[=precision]
              Show the time spent in system calls.  This records the
              time difference between the beginning and the end of each
              system call.  precision can be one of s (for seconds), ms
              (milliseconds), us (microseconds), or ns (nanoseconds),
              and allows setting the precision of time value being
              printed.  Default is us (microseconds).

       -v
       --no-abbrev
              Print unabbreviated versions of environment, stat,
              termios, etc.  calls.  These structures are very common in
              calls and so the default behavior displays a reasonable
              subset of structure members.  Use this option to get all
              of the gory details.

       --strings-in-hex[=option]
              Control usage of escape sequences with hexadecimal numbers
              in the printed strings.  Normally (when no
              --strings-in-hex or -x option is supplied), escape
              sequences are used to print non-printable and non-ASCII
              characters (that is, characters with a character code less
              than 32 or greater than 127), or to disambiguate the
              output (so, for quotes and other characters that encase
              the printed string, for example, angle brackets, in case
              of file descriptor path output); for the former use case,
              unless it is a white space character that has a symbolic
              escape sequence defined in the C standard (that is, "\textbackslash t"
              for a horizontal tab, "\textbackslash n" for a newline, "\textbackslash v" for a
              vertical tab, "\textbackslash f" for a form feed page break, and "\textbackslash r"
              for a carriage return) are printed using escape sequences
              with numbers that correspond to their byte values, with
              octal number format being the default.  option can be one
              of the following:

              none   Hexadecimal numbers are not used in the output at
                     all.  When there is a need to emit an escape
                     sequence, octal numbers are used.
              non-ascii-chars
                     Hexadecimal numbers are used instead of octal in
                     the escape sequences.
              non-ascii
                     Strings that contain non-ASCII characters are
                     printed using escape sequences with hexadecimal
                     numbers.
              all    All strings are printed using escape sequences with
                     hexadecimal numbers.

              When the option is supplied without an argument, all is
              assumed.

       -x
       --strings-in-hex=non-ascii
              Print all non-ASCII strings in hexadecimal string format.

       -xx
       --strings-in-hex[=all]
              Print all strings in hexadecimal string format.

       -X format
       --const-print-style=format
              Set the format for printing of named constants and flags.
              Supported format values are:

              raw    Raw number output, without decoding.
              abbrev Output a named constant or a set of flags instead
                     of the raw number if they are found.  This is the
                     default strace behaviour.
              verbose
                     Output both the raw value and the decoded string
                     (as a comment).

       -y
       --decode-fds
       --decode-fds=path
              Print paths associated with file descriptor arguments and
              with the AT_FDCWD constant.

       -yy
       --decode-fds=all
              Print all available information associated with file
              descriptors: protocol-specific information associated with
              socket file descriptors, block/character device number
              associated with device file descriptors, and PIDs
              associated with pidfd file descriptors.

       --pidns-translation
       --decode-pids=pidns
              If strace and tracee are in different PID namespaces,
              print PIDs in strace's namespace, too.

       -Y
       --decode-pids=comm
              Print command names for PIDs.

       --secontext[=format]
       -e secontext=format
              When SELinux is available and is not disabled, print in
              square brackets SELinux contexts of processes, files, and
              descriptors.  The format argument is a comma-separated
              list of items being one of the following:

              full              Print the full context (user, role, type
                                level and category).
              mismatch          Also print the context recorded by the
                                SELinux database in case the current
                                context differs.  The latter is printed
                                after two exclamation marks (!!).

              The default value for --secontext is !full,mismatch which
              prints only the type instead of full context and doesn't
              check for context mismatches.

       --always-show-pid
              Show PID prefix also for the process started by strace.
              Implied when -f and -o are both specified.

   Statistics
       -c
       --summary-only
              Count time, calls, and errors for each system call and
              report a summary on program exit, suppressing the regular
              output.  This attempts to show system time (CPU time spent
              running in the kernel) independent of wall clock time.  If
              -c is used with -f, only aggregate totals for all traced
              processes are kept.

       -C
       --summary
              Like -c but also print regular output while processes are
              running.

       -O overhead
       --summary-syscall-overhead=overhead
              Set the overhead for tracing system calls to overhead.
              This is useful for overriding the default heuristic for
              guessing how much time is spent in mere measuring when
              timing system calls using the -c option.  The accuracy of
              the heuristic can be gauged by timing a given program run
              without tracing (using time(1)) and comparing the
              accumulated system call time to the total produced using
              -c.

              The format of overhead specification is described in
              section Time specification format description.

       -S sortby
       --summary-sort-by=sortby
              Sort the output of the histogram printed by the -c option
              by the specified criterion.  Legal values are time (or
              time-percent or time-total or total-time), min-time (or
              shortest or time-min), max-time (or longest or time-max),
              avg-time (or time-avg), calls (or count), errors (or
              error), name (or syscall or syscall-name), and nothing (or
              none); default is time.

       -U columns
       --summary-columns=columns
              Configure a set (and order) of columns being shown in the
              call summary.  The columns argument is a comma-separated
              list with items being one of the following:

              time-percent (or time)
                     Percentage of cumulative time consumed by a
                     specific system call.
              total-time (or time-total)
                     Total system (or wall clock, if -w option is
                     provided) time consumed by a specific system call.
              min-time (or shortest or time-min)
                     Minimum observed call duration.
              max-time (or longest or time-max)
                     Maximum observed call duration.
              avg-time (or time-avg)
                     Average call duration.
              calls (or count)
                     Call count.
              errors (or error)
                     Error count.
              name (or syscall or syscall-name)
                     Syscall name.

              The default value is
              time-percent,total-time,avg-time,calls,errors,name.  If
              the name field is not supplied explicitly, it is added as
              the last column.

       -w
       --summary-wall-clock
              Summarise the time difference between the beginning and
              end of each system call.  The default is to summarise the
              system time.

   Tampering
       -e inject=syscall_set[:error=errno|:retval=value][:signal=sig]
       [:syscall=syscall][:delay_enter=delay][:delay_exit=delay]
       [:poke_enter=@argN=DATAN,@argM=DATAM...]
       [:poke_exit=@argN=DATAN,@argM=DATAM...][:when=expr]
       --inject=syscall_set[:error=errno|:retval=value][:signal=sig]
       [:syscall=syscall][:delay_enter=delay][:delay_exit=delay]
       [:poke_enter=@argN=DATAN,@argM=DATAM...]
       [:poke_exit=@argN=DATAN,@argM=DATAM...][:when=expr]
              Perform   syscall  tampering  for  the  specified  set  of
              syscalls.  The syntax of the syscall_set specification  is
              the same as in the -e trace option.

              At  least  one  of  error,  retval,  signal,  delay_enter,
              delay_exit, poke_enter, or poke_exit  options  has  to  be
              specified.  error and retval are mutually exclusive.

              If  :error=errno  option is specified, a fault is injected
              into a syscall invocation: the syscall number is  replaced
              by  -1  which  corresponds to an invalid syscall (unless a
              syscall is specified with :syscall= option), and the error
              code is specified using a symbolic errno value like ENOSYS
              or a numeric value within 1..4095 range.

              If :retval=value option is specified, success injection is
              performed: the syscall number is replaced  by  -1,  but  a
              bogus success value is returned to the callee.

              If  :signal=sig option is specified with either a symbolic
              value like SIGSEGV or a numeric value  within  1..SIGRTMAX
              range,  that signal is delivered on entering every syscall
              specified by the set.

              If :delay_enter=delay  or  :delay_exit=delay  options  are
              specified,  delay  injection  is  performed: the tracee is
              delayed by time period specified by delay on  entering  or
              exiting  the  syscall,  respectively.  The format of delay
              specification is described in section  Time  specification
              format description.

              If        :poke_enter=@argN=DATAN,@argM=DATAM...        or
              :poke_exit=@argN=DATAN,@argM=DATAM...     options      are
              specified,  tracee's  memory  at  locations, pointed to by
              system call arguments argN and argM (going  from  arg1  to
              arg7) is overwritten by data DATAN and DATAM (specified in
              hexadecimal          format;          for          example
              :poke_enter=@arg1=0000DEAD0000BEEF).  :poke_enter modifies
              memory on syscall enter, and :poke_exit - on exit.

              If :signal=sig option is specified  without  :error=errno,
              :retval=value  or  :delay_{enter,exit}=usecs options, then
              only a signal sig is delivered without a syscall fault  or
              delay     injection.     Conversely,    :error=errno    or
              :retval=value    option    without     :delay_enter=delay,
              :delay_exit=delay  or  :signal=sig options injects a fault
              without delivering a signal or injecting a delay, etc.

              If  :signal=sig  option   is   specified   together   with
              :error=errno  or  :retval=value,  then both injection of a
              fault or success and signal delivery are performed.

              if :syscall=syscall option is specified, the corresponding
              syscall with no side effects is injected  instead  of  -1.
              Currently,  only  "pure"  (see -e trace=%pure description)
              syscalls can be specified there.

              Unless  a  :when=expr  subexpression  is   specified,   an
              injection  is  being  made  into  every invocation of each
              syscall from the set.

              The format of the subexpression is:

                             first[..last][+[step]]

              Number first stands for the first invocation number in the
              range, number last stands for the last  invocation  number
              in  the  range,  and  step stands for the step between two
              consecutive invocations.  The following  combinations  are
              useful:

              first  For every syscall from the set, perform an
                     injection for the syscall invocation number first
                     only.
              first..last
                     For every syscall from the set, perform an
                     injection for the syscall invocation number first
                     and all subsequent invocations until the invocation
                     number last (inclusive).
              first+ For every syscall from the set, perform injections
                     for the syscall invocation number first and all
                     subsequent invocations.
              first..last+
                     For every syscall from the set, perform injections
                     for the syscall invocation number first and all
                     subsequent invocations until the invocation number
                     last (inclusive).
              first+step
                     For every syscall from the set, perform injections
                     for syscall invocations number first, first+step,
                     first+step+step, and so on.
              first..last+step
                     Same as the previous, but consider only syscall
                     invocations with numbers up to last (inclusive).

              For example, to fail each third and subsequent chdir
              syscalls with ENOENT, use
              -e inject=chdir:error=ENOENT:when=3+.

              The valid range for numbers first and step is 1..65535,
              and for number last is 1..65534.

              An injection expression can contain only one error= or
              retval= specification, and only one signal= specification.
              If an injection expression contains multiple when=
              specifications, the last one takes precedence.

              Accounting of syscalls that are subject to injection is
              done per syscall and per tracee.

              Specification of syscall injection can be combined with
              other syscall filtering options, for example, -P
              /dev/urandom -e inject=file:error=ENOENT.

       -e fault=syscall_set[:error=errno][:when=expr]
       --fault=syscall_set[:error=errno][:when=expr]
              Perform syscall fault injection for the specified set of
              syscalls.

              This is equivalent to more generic -e inject= expression
              with default value of errno option set to ENOSYS.

   Miscellaneous
       -d
       --debug
              Show some debugging output of strace itself on the
              standard error.

       -F     This option is deprecated.  It is retained for backward
              compatibility only and may be removed in future releases.
              Usage of multiple instances of -F option is still
              equivalent to a single -f, and it is ignored at all if
              used along with one or more instances of -f option.

       -h
       --help Print the help summary.

       --seccomp-bpf
              Try to enable use of seccomp-bpf (see seccomp(2)) to have
              ptrace(2)-stops only when system calls that are being
              traced occur in the traced processes.

              This option has no effect unless -f/--follow-forks is also
              specified.  --seccomp-bpf is not compatible with
              --syscall-limit and -b/--detach-on options.  It is also
              not applicable to processes attached using -p/--attach
              option.

              An attempt to enable system calls filtering using seccomp-
              bpf may fail for various reasons, e.g. there are too many
              system calls to filter, the seccomp API is not available,
              or strace itself is being traced.  In cases when seccomp-
              bpf filter setup failed, strace proceeds as usual and
              stops traced processes on every system call.

              When --seccomp-bpf is activated and -p/--attach option is
              not used, --kill-on-exit option is activated as well.

              Note that in cases when the tracee has another seccomp
              filter that returns an action value with a precedence
              greater than SECCOMP_RET_TRACE, strace --seccomp-bpf will
              not be notified.  That is, if another seccomp filter, for
              example, disables the syscall or kills the tracee, then
              strace --seccomp-bpf will not be aware of that syscall
              invocation at all.

       --tips[=[[id:]id],[[format:]format]]
              Show strace tips, tricks, and tweaks before exit.  id can
              be a non-negative integer number, which enables printing
              of specific tip, trick, or tweak (these ID are not
              guaranteed to be stable), or random (the default), in
              which case a random tip is printed.  format can be one of
              the following:

              none     No tip is printed.  Can be used to override the
                       previous setting.
              compact  Print the tip just big enough to contain all the
                       text.
              full     Print the tip in its full glory.

              Default is id:random,format:compact.

       -V
       --version
              Print the version number of strace.  Multiple instances of
              the option beyond specific threshold tend to increase
              Strauss awareness.

   Time specification format description
       Time values can be specified as a decimal floating point number
       (in a format accepted by strtod(3)), optionally followed by one
       of the following suffices that specify the unit of time: s
       (seconds), ms (milliseconds), us (microseconds), or ns
       (nanoseconds).  If no suffix is specified, the value is
       interpreted as microseconds.

       The described format is used for -O, -e inject=delay_enter, and
       -e inject=delay_exit options.
DIAGNOSTICS
       When command exits, strace exits with the same exit status.  If
       command is terminated by a signal, strace terminates itself with
       the same signal, so that strace can be used as a wrapper process
       transparent to the invoking parent process.  Note that parent-
       child relationship (signal stop notifications, getppid(2) value,
       etc) between traced process and its parent are not preserved
       unless -D is used.

       When using -p without a command, the exit status of strace is
       zero unless no processes has been attached or there was an
       unexpected error in doing the tracing.
SETUID INSTALLATION
       If strace is installed setuid to root then the invoking user will
       be able to attach to and trace processes owned by any user.  In
       addition setuid and setgid programs will be executed and traced
       with the correct effective privileges.  Since only users trusted
       with full root privileges should be allowed to do these things,
       it only makes sense to install strace as setuid to root when the
       users who can execute it are restricted to those users who have
       this trust.  For example, it makes sense to install a special
       version of strace with mode 'rwsr-xr--', user root and group
       trace, where members of the trace group are trusted users.  If
       you do use this feature, please remember to install a regular
       non-setuid version of strace for ordinary users to use.
MULTIPLE PERSONALITIES SUPPORT
       On some architectures, strace supports decoding of syscalls for
       processes that use different ABI rather than the one strace uses.
       Specifically, in addition to decoding native ABI, strace can
       decode the following ABIs on the following architectures:

       [1]  When strace is built as an x86_64 application
       [2]  When strace is built as an x32 application
       [3]  Big endian only

       This support is optional and relies on ability to generate and
       parse structure definitions during the build time.  Please refer
       to the output of the strace -V command in order to figure out
       what support is available in your strace build ("non-native"
       refers to an ABI that differs from the ABI strace has):

       m32-mpers
              strace can trace and properly decode non-native 32-bit
              binaries.
       no-m32-mpers
              strace can trace, but cannot properly decode non-native
              32-bit binaries.
       mx32-mpers
              strace can trace and properly decode non-native
              32-on-64-bit binaries.
       no-mx32-mpers
              strace can trace, but cannot properly decode non-native
              32-on-64-bit binaries.

       If the output contains neither m32-mpers nor no-m32-mpers, then
       decoding of non-native 32-bit binaries is not implemented at all
       or not applicable.

       Likewise, if the output contains neither mx32-mpers nor no-
       mx32-mpers, then decoding of non-native 32-on-64-bit binaries is
       not implemented at all or not applicable.
NOTES
       It is a pity that so much tracing clutter is produced by systems
       employing shared libraries.

       It is instructive to think about system call inputs and outputs
       as data-flow across the user/kernel boundary.  Because user-space
       and kernel-space are separate and address-protected, it is
       sometimes possible to make deductive inferences about process
       behavior using inputs and outputs as propositions.

       In some cases, a system call will differ from the documented
       behavior or have a different name.  For example, the faccessat(2)
       system call does not have flags argument, and the setrlimit(2)
       library function uses prlimit64(2) system call on modern
       (2.6.38+) kernels.  These discrepancies are normal but
       idiosyncratic characteristics of the system call interface and
       are accounted for by C library wrapper functions.

       Some system calls have different names in different architectures
       and personalities.  In these cases, system call filtering and
       printing uses the names that match corresponding __NR_* kernel
       macros of the tracee's architecture and personality.  There are
       two exceptions from this general rule: arm_fadvise64_64(2) ARM
       syscall and xtensa_fadvise64_64(2) Xtensa syscall are filtered
       and printed as fadvise64_64(2).

       On x32, syscalls that are intended to be used by 64-bit processes
       and not x32 ones (for example, readv(2), that has syscall number
       19 on x86_64, with its x32 counterpart has syscall number 515),
       but called with __X32_SYSCALL_BIT flag being set, are designated
       with #64 suffix.

       On some platforms a process that is attached to with the -p
       option may observe a spurious EINTR return from the current
       system call that is not restartable.  (Ideally, all system calls
       should be restarted on strace attach, making the attach invisible
       to the traced process, but a few system calls aren't.  Arguably,
       every instance of such behavior is a kernel bug.)  This may have
       an unpredictable effect on the process if the process takes no
       action to restart the system call.

       As strace executes the specified command directly and does not
       employ a shell for that, scripts without shebang that usually run
       just fine when invoked by shell fail to execute with ENOEXEC
       error.  It is advisable to manually supply a shell as a command
       with the script as its argument.
BUGS
       Programs that use the setuid bit do not have effective user ID
       privileges while being traced.

       A traced process runs slowly (but check out the --seccomp-bpf
       option).

       Unless --kill-on-exit option is used (or --seccomp-bpf option is
       used in a way that implies --kill-on-exit), traced processes
       which are descended from command may be left running after an
       interrupt signal (CTRL-C).

       By using CLONE_UNTRACED flag of clone system call a tracee can
       break the guarantee that --seccomp-bpf will not leave any
       processes with a seccomp program installed for syscall filtering
       purposes.
HISTORY
       The original strace was written by Paul Kranenburg for SunOS and
       was inspired by its trace utility.  The SunOS version of strace
       was ported to Linux and enhanced by Branko Lankester, who also
       wrote the Linux kernel support.  Even though Paul released strace
       2.5 in 1992, Branko's work was based on Paul's strace 1.5 release
       from 1991.  In 1993, Rick Sladkey merged strace 2.5 for SunOS and
       the second release of strace for Linux, added many of the
       features of truss(1) from SVR4, and produced an strace that
       worked on both platforms.  In 1994 Rick ported strace to SVR4 and
       Solaris and wrote the automatic configuration support.  In 1995
       he ported strace to Irix and became tired of writing about
       himself in the third person.

       Beginning with 1996, strace was maintained by Wichert Akkerman.
       During his tenure, strace development migrated to CVS; ports to
       FreeBSD and many architectures on Linux (including ARM, IA-64,
       MIPS, PA-RISC, PowerPC, s390, SPARC) were introduced.  In 2002,
       the burden of strace maintainership was transferred to Roland
       McGrath.  Since then, strace gained support for several new Linux
       architectures (AMD64, s390x, SuperH), bi-architecture support for
       some of them, and received numerous additions and improvements in
       syscalls decoders on Linux; strace development migrated to Git
       during that period.  Since 2009, strace is actively maintained by
       Dmitry Levin.  strace gained support for AArch64, ARC, AVR32,
       Blackfin, Meta, Nios II, OpenRISC 1000, RISC-V, Tile/TileGx,
       Xtensa architectures since that time.  In 2012, unmaintained and
       apparently broken support for non-Linux operating systems was
       removed.  Also, in 2012 strace gained support for path tracing
       and file descriptor path decoding.  In 2014, support for stack
       trace printing was added.  In 2016, syscall fault injection was
       implemented.

       For the additional information, please refer to the NEWS file and
       strace repository commit log.
REPORTING BUGS
       Problems with strace should be reported to the strace mailing
       list mailto:strace-devel@lists.strace.io.
SEE ALSO
       strace-log-merge(1), ltrace(1), perf-trace(1), trace-cmd(1),
       time(1), ptrace(2), seccomp(2), syscall(2), proc(5), signal(7)

       strace Home Page https://strace.io/
AUTHORS
       The complete list of strace contributors can be found in the
       CREDITS file.
COLOPHON
       This page is part of the strace (system call tracer) project.
       Information about the project can be found at 
       http://strace.io/.  If you have a bug report for this manual
       page, send it to strace-devel@lists.sourceforge.net.  This page
       was obtained from the project's upstream Git repository
       https://github.com/strace/strace.git on 2024-06-14.  (At that
       time, the date of the most recent commit that was found in the
       repository was 2024-06-04.)  If you discover any rendering
       problems in this HTML version of the page, or you believe there
       is a better or more up-to-date source for the page, or you have
       corrections or improvements to the information in this COLOPHON
       (which is not part of the original manual page), send a mail to
       man-pages@man7.org

strace 6.9.0.16.2a4c4          2024-06-04                      STRACE(1)
\end{lstlisting}
}}

\endinput  %  ==  ==  ==  ==  ==  ==  ==  ==  ==
\subsection{\refStrace: Trace System Calls and Signals}

{\tiny{
\begin{lstlisting}[language=bash]
NAME
       strace - trace system calls and signals
SYNOPSIS
       strace [-ACdffhikkqqrtttTvVwxxyyYzZ] [-a column] [-b execve]
              [-e expr]... [-I n] [-o file] [-O overhead] [-p pid]...
              [-P path]... [-s strsize] [-S sortby] [-U columns]
              [-X format] [--seccomp-bpf]
              [--stack-trace-frame-limit=limit] [--syscall-limit=limit]
              [--secontext[=format]] [--tips[=format]] { -p pid | [-DDD]
              [-E var[=val]]... [-u username] command [args] }

       strace -c [-dfwzZ] [-b execve] [-e expr]... [-I n] [-O overhead]
              [-p pid]... [-P path]... [-S sortby] [-U columns]
              [--seccomp-bpf] [--syscall-limit=limit] [--tips[=format]]
              { -p pid | [-DDD] [-E var[=val]]... [-u username] command
              [args] }

       strace --tips[=format]
DESCRIPTION
       In the simplest case strace runs the specified command until it
       exits.  It intercepts and records the system calls which are
       called by a process and the signals which are received by a
       process.  The name of each system call, its arguments and its
       return value are printed on standard error or to the file
       specified with the -o option.

       strace is a useful diagnostic, instructional, and debugging tool.
       System administrators, diagnosticians and trouble-shooters will
       find it invaluable for solving problems with programs for which
       the source is not readily available since they do not need to be
       recompiled in order to trace them.  Students, hackers and the
       overly-curious will find that a great deal can be learned about a
       system and its system calls by tracing even ordinary programs.
       And programmers will find that since system calls and signals are
       events that happen at the user/kernel interface, a close
       examination of this boundary is very useful for bug isolation,
       sanity checking and attempting to capture race conditions.

       Each line in the trace contains the system call name, followed by
       its arguments in parentheses and its return value.  An example
       from stracing the command "cat /dev/null" is:

           open("/dev/null", O_RDONLY) = 3

       Errors (typically a return value of -1) have the errno symbol and
       error string appended.

           open("/foo/bar", O_RDONLY) = -1 ENOENT (No such file or directory)

       Signals are printed as signal symbol and decoded siginfo
       structure.  An excerpt from stracing and interrupting the command
       "sleep 666" is:

           sigsuspend([] <unfinished ...>
           --- SIGINT {si_signo=SIGINT, si_code=SI_USER, si_pid=...} ---
           +++ killed by SIGINT +++

       If a system call is being executed and meanwhile another one is
       being called from a different thread/process then strace will try
       to preserve the order of those events and mark the ongoing call
       as being unfinished.  When the call returns it will be marked as
       resumed.

           [pid 28772] select(4, [3], NULL, NULL, NULL <unfinished ...>
           [pid 28779] clock_gettime(CLOCK_REALTIME, {tv_sec=1130322148, tv_nsec=3977000}) = 0
           [pid 28772] <... select resumed> )      = 1 (in [3])

       Interruption of a (restartable) system call by a signal delivery
       is processed differently as kernel terminates the system call and
       also arranges its immediate reexecution after the signal handler
       completes.

           read(0, 0x7ffff72cf5cf, 1)              = ? ERESTARTSYS (To be restarted)
           --- SIGALRM {si_signo=SIGALRM, si_code=SI_KERNEL} ---
           rt_sigreturn({mask=[]})                 = 0
           read(0, "", 1)                          = 0

       Arguments are printed in symbolic form with passion.  This
       example shows the shell performing ">>xyzzy" output redirection:

           open("xyzzy", O_WRONLY|O_APPEND|O_CREAT, 0666) = 3

       Here, the second and the third argument of open(2) are decoded by
       breaking down the flag argument into its three bitwise-OR
       constituents and printing the mode value in octal by tradition.
       Where the traditional or native usage differs from ANSI or POSIX,
       the latter forms are preferred.  In some cases, strace output is
       proven to be more readable than the source.

       Structure pointers are dereferenced and the members are displayed
       as appropriate.  In most cases, arguments are formatted in the
       most C-like fashion possible.  For example, the essence of the
       command "ls -l /dev/null" is captured as:

           lstat("/dev/null", {st_mode=S_IFCHR|0666, st_rdev=makedev(0x1, 0x3), ...}) = 0

       Notice how the 'struct stat' argument is dereferenced and how
       each member is displayed symbolically.  In particular, observe
       how the st_mode member is carefully decoded into a bitwise-OR of
       symbolic and numeric values.  Also notice in this example that
       the first argument to lstat(2) is an input to the system call and
       the second argument is an output.  Since output arguments are not
       modified if the system call fails, arguments may not always be
       dereferenced.  For example, retrying the "ls -l" example with a
       non-existent file produces the following line:

           lstat("/foo/bar", 0xb004) = -1 ENOENT (No such file or directory)

       In this case the porch light is on but nobody is home.

       Syscalls unknown to strace are printed raw, with the unknown
       system call number printed in hexadecimal form and prefixed with
       "syscall_":

           syscall_0xbad(0x1, 0x2, 0x3, 0x4, 0x5, 0x6) = -1 ENOSYS (Function not implemented)

       Character pointers are dereferenced and printed as C strings.
       Non-printing characters in strings are normally represented by
       ordinary C escape codes.  Only the first strsize (32 by default)
       bytes of strings are printed; longer strings have an ellipsis
       appended following the closing quote.  Here is a line from "ls
       -l" where the getpwuid(3) library routine is reading the password
       file:

           read(3, "root::0:0:System Administrator:/"..., 1024) = 422

       While structures are annotated using curly braces, pointers to
       basic types and arrays are printed using square brackets with
       commas separating the elements.  Here is an example from the
       command id(1) on a system with supplementary group ids:

           getgroups(32, [100, 0]) = 2

       On the other hand, bit-sets are also shown using square brackets,
       but set elements are separated only by a space.  Here is the
       shell, preparing to execute an external command:

           sigprocmask(SIG_BLOCK, [CHLD TTOU], []) = 0

       Here, the second argument is a bit-set of two signals, SIGCHLD
       and SIGTTOU.  In some cases, the bit-set is so full that printing
       out the unset elements is more valuable.  In that case, the bit-
       set is prefixed by a tilde like this:

           sigprocmask(SIG_UNBLOCK, ~[], NULL) = 0

       Here, the second argument represents the full set of all signals.
OPTIONS
   General
       -e expr
              A qualifying expression which modifies which events to
              trace or how to trace them.  The format of the expression
              is:

                             [qualifier=][!]value[,value]...

              where qualifier is one of trace (or t), trace-fds (or
              trace-fd or fd or fds), abbrev (or a), verbose (or v), raw
              (or x), signal (or signals or s), read (or reads or r),
              write (or writes or w), fault, inject, status, quiet (or
              silent or silence or q), secontext, decode-fds (or
              decode-fd), decode-pids (or decode-pid), or kvm, and value
              is a qualifier-dependent symbol or number.  The default
              qualifier is trace.  Using an exclamation mark negates the
              set of values.  For example, -e open means literally
              -e trace=open which in turn means trace only the open
              system call.  By contrast, -e trace=!open means to trace
              every system call except open.  In addition, the special
              values all and none have the obvious meanings.

              Note that some shells use the exclamation point for
              history expansion even inside quoted arguments.  If so,
              you must escape the exclamation point with a backslash.

   Startup
       -E var=val
       --env=var=val
              Run command with var=val in its list of environment
              variables.

       -E var
       --env=var
              Remove var from the inherited list of environment
              variables before passing it on to the command.

       -p pid
       --attach=pid
              Attach to the process with the process ID pid and begin
              tracing.  The trace may be terminated at any time by a
              keyboard interrupt signal (CTRL-C).  strace will respond
              by detaching itself from the traced process(es) leaving it
              (them) to continue running.  Multiple -p options can be
              used to attach to many processes in addition to command
              (which is optional if at least one -p option is given).
              Multiple process IDs, separated by either comma (",''),
              space (" "), tab, or newline character, can be provided as
              an argument to a single -p option, so, for example, -p
              "$(pidof PROG)" and -p "$(pgrep PROG)" syntaxes are
              supported.

       -u username
       --user=username
              Run command with the user ID, group ID, and supplementary
              groups of username.  This option is only useful when
              running as root and enables the correct execution of
              setuid and/or setgid binaries.  Unless this option is used
              setuid and setgid programs are executed without effective
              privileges.
       -u UID:GID
       --user=UID:GID
              Alternative syntax where the program is started with
              exactly the given user and group IDs, and an empty list of
              supplementary groups.  In this case, user and group name
              lookups are not performed.

       --argv0=name
              Set argv[0] of the command being executed to name.  Useful
              for tracing multi-call executables which interpret
              argv[0], such as busybox or kmod.

   Tracing
       -b syscall
       --detach-on=syscall
              If specified syscall is reached, detach from traced
              process.  Currently, only execve(2) syscall is supported.
              This option is useful if you want to trace multi-threaded
              process and therefore require -f, but don't want to trace
              its (potentially very complex) children.

       -D
       --daemonize
       --daemonize=grandchild
              Run tracer process as a grandchild, not as the parent of
              the tracee.  This reduces the visible effect of strace by
              keeping the tracee a direct child of the calling process.

       -DD
       --daemonize=pgroup
       --daemonize=pgrp
              Run tracer process as tracee's grandchild in a separate
              process group.  In addition to reduction of the visible
              effect of strace, it also avoids killing of strace with
              kill(2) issued to the whole process group.

       -DDD
       --daemonize=session
              Run tracer process as tracee's grandchild in a separate
              session ("true daemonisation").  In addition to reduction
              of the visible effect of strace, it also avoids killing of
              strace upon session termination.

       -f
       --follow-forks
              Trace child processes as they are created by currently
              traced processes as a result of the fork(2), vfork(2) and
              clone(2) system calls.  Note that -p PID -f will attach
              all threads of process PID if it is multi-threaded, not
              only thread with thread_id = PID.

       --output-separately
              If the --output=filename option is in effect, each
              processes trace is written to filename.pid where pid is
              the numeric process id of each process.

       -ff
       --follow-forks --output-separately
              Combine the effects of --follow-forks and
              --output-separately options.  This is incompatible with
              -c, since no per-process counts are kept.

              One might want to consider using strace-log-merge(1) to
              obtain a combined strace log view.

       -I interruptible
       --interruptible=interruptible
              When strace can be interrupted by signals (such as
              pressing CTRL-C).

              1, anywhere
                     no signals are blocked;
              2, waiting
                     fatal signals are blocked while decoding syscall
                     (default);
              3, never
                     fatal signals are always blocked (default if -o
                     FILE PROG);
              4, never_tstp
                     fatal signals and SIGTSTP (CTRL-Z) are always
                     blocked (useful to make strace -o FILE PROG not
                     stop on CTRL-Z, default if -D).

       --syscall-limit=limit
              Detach all tracees when limit number of syscalls have been
              captured. Syscalls filtered out via --trace, --trace-path
              or --status options are not considered when keeping track
              of the number of syscalls that are captured.

       --kill-on-exit
              Apply PTRACE_O_EXITKILL ptrace option to all tracee
              processes (which sends a SIGKILL signal to the tracee if
              the tracer exits) and do not detach them on cleanup so
              they will not be left running after the tracer exit.
              --kill-on-exit is not compatible with -p/--attach options.

   Filtering
       -e trace=syscall_set
       -e t=syscall_set
       --trace=syscall_set
              Trace only the specified set of system calls.  syscall_set
              is defined as [!]value[,value], and value can be one of
              the following:

              syscall
                     Trace specific syscall, specified by its name (see
                     syscalls(2) for a reference, but also see NOTES).

              ?value Question mark before the syscall qualification
                     allows suppression of error in case no syscalls
                     matched the qualification provided.

              value@64
                     Limit the syscall specification described by value
                     to 64-bit personality.

              value@32
                     Limit the syscall specification described by value
                     to 32-bit personality.

              value@x32
                     Limit the syscall specification described by value
                     to x32 personality.

              all    Trace all system calls.

              /regex Trace only those system calls that match the regex.
                     You can use POSIX Extended Regular Expression
                     syntax (see regex(7)).

              %file
              file   Trace all system calls which take a file name as an
                     argument.  You can think of this as an abbreviation
                     for -e trace=open,stat,chmod,unlink,...  which is
                     useful to seeing what files the process is
                     referencing.  Furthermore, using the abbreviation
                     will ensure that you don't accidentally forget to
                     include a call like lstat(2) in the list.  Betchya
                     woulda forgot that one.  The syntax without a
                     preceding percent sign ("-e trace=file") is
                     deprecated.

              %process
              process
                     Trace system calls associated with process
                     lifecycle (creation, exec, termination).  The
                     syntax without a preceding percent sign ("-e
                     trace=process") is deprecated.

              %net
              %network
              network
                     Trace all the network related system calls.  The
                     syntax without a preceding percent sign ("-e
                     trace=network") is deprecated.

              %signal
              signal Trace all signal related system calls.  The syntax
                     without a preceding percent sign ("-e
                     trace=signal") is deprecated.

              %ipc
              ipc    Trace all IPC related system calls.  The syntax
                     without a preceding percent sign ("-e trace=ipc")
                     is deprecated.

              %desc
              desc   Trace all file descriptor related system calls.
                     The syntax without a preceding percent sign ("-e
                     trace=desc") is deprecated.

              %memory
              memory Trace all memory mapping related system calls.  The
                     syntax without a preceding percent sign ("-e
                     trace=memory") is deprecated.

              %creds Trace system calls that read or modify user and
                     group identifiers or capability sets.

              %stat  Trace stat syscall variants.

              %lstat Trace lstat syscall variants.

              %fstat Trace fstat, fstatat, and statx syscall variants.

              %%stat Trace syscalls used for requesting file status
                     (stat, lstat, fstat, fstatat, statx, and their
                     variants).

              %statfs
                     Trace statfs, statfs64, statvfs, osf_statfs, and
                     osf_statfs64 system calls.  The same effect can be
                     achieved with -e trace=/^(.*_)?statv?fs regular
                     expression.

              %fstatfs
                     Trace fstatfs, fstatfs64, fstatvfs, osf_fstatfs,
                     and osf_fstatfs64 system calls.  The same effect
                     can be achieved with -e trace=/fstatv?fs regular
                     expression.

              %%statfs
                     Trace syscalls related to file system statistics
                     (statfs-like, fstatfs-like, and ustat).  The same
                     effect can be achieved with
                     -e trace=/statv?fs|fsstat|ustat regular expression.

              %clock Trace system calls that read or modify system
                     clocks.

              %pure  Trace syscalls that always succeed and have no
                     arguments.  Currently, this list includes
                     arc_gettls(2), getdtablesize(2), getegid(2),
                     getegid32(2), geteuid(2), geteuid32(2), getgid(2),
                     getgid32(2), getpagesize(2), getpgrp(2), getpid(2),
                     getppid(2), get_thread_area(2) (on architectures
                     other than x86), gettid(2), get_tls(2), getuid(2),
                     getuid32(2), getxgid(2), getxpid(2), getxuid(2),
                     kern_features(2), and metag_get_tls(2) syscalls.

              The -c option is useful for determining which system calls
              might be useful to trace.  For example,
              trace=open,close,read,write means to only trace those four
              system calls.  Be careful when making inferences about the
              user/kernel boundary if only a subset of system calls are
              being monitored.  The default is trace=all.

       -e trace-fd=set
       -e trace-fds=set
       -e fd=set
       -e fds=set
       --trace-fds=set
              Trace only the syscalls that operate on the specified
              subset of (non-negative) file descriptors.  Note that
              usage of this option also filters out all the syscalls
              that do not operate on file descriptors at all.  Applies
              in (inclusive) disjunction with the --trace-path option.

       -e signal=set
       -e signals=set
       -e s=set
       --signal=set
              Trace only the specified subset of signals.  The default
              is signal=all.  For example, signal=!SIGIO (or signal=!io)
              causes SIGIO signals not to be traced.

       -e status=set
       --status=set
              Print only system calls with the specified return status.
              The default is status=all.  When using the status
              qualifier, because strace waits for system calls to return
              before deciding whether they should be printed or not, the
              traditional order of events may not be preserved anymore.
              If two system calls are executed by concurrent threads,
              strace will first print both the entry and exit of the
              first system call to exit, regardless of their respective
              entry time.  The entry and exit of the second system call
              to exit will be printed afterwards.  Here is an example
              when select(2) is called, but a different thread calls
              clock_gettime(2) before select(2) finishes:

                  [pid 28779] 1130322148.939977 clock_gettime(CLOCK_REALTIME, {1130322148, 939977000}) = 0
                  [pid 28772] 1130322148.438139 select(4, [3], NULL, NULL, NULL) = 1 (in [3])

              set can include the following elements:

              successful
                     Trace system calls that returned without an error
                     code.  The -z option has the effect of
                     status=successful.
              failed Trace system calls that returned with an error
                     code.  The -Z option has the effect of
                     status=failed.
              unfinished
                     Trace system calls that did not return.  This might
                     happen, for example, due to an execve call in a
                     neighbour thread.
              unavailable
                     Trace system calls that returned but strace failed
                     to fetch the error status.
              detached
                     Trace system calls for which strace detached before
                     the return.

       -P path
       --trace-path=path
              Trace only system calls accessing path.  Multiple -P
              options can be used to specify several paths.  Applies in
              (inclusive) disjunction with the --trace-fds option.

       -z
       --successful-only
              Print only syscalls that returned without an error code.

       -Z
       --failed-only
              Print only syscalls that returned with an error code.

   Output format
       -a column
       --columns=column
              Align return values in a specific column (default column
              40).

       -e abbrev=syscall_set
       -e a=syscall_set
       --abbrev=syscall_set
              Abbreviate the output from printing each member of large
              structures.  The syntax of the syscall_set specification
              is the same as in the -e trace option.  The default is
              abbrev=all.  The -v option has the effect of abbrev=none.

       -e verbose=syscall_set
       -e v=syscall_set
       --verbose=syscall_set
              Dereference structures for the specified set of system
              calls.  The syntax of the syscall_set specification is the
              same as in the -e trace option.  The default is
              verbose=all.

       -e raw=syscall_set
       -e x=syscall_set
       --raw=syscall_set
              Print raw, undecoded arguments for the specified set of
              system calls.  The syntax of the syscall_set specification
              is the same as in the -e trace option.  This option has
              the effect of causing all arguments to be printed in
              hexadecimal.  This is mostly useful if you don't trust the
              decoding or you need to know the actual numeric value of
              an argument.  See also -X raw option.

       -e read=set
       -e reads=set
       -e r=set
       --read=set
              Perform a full hexadecimal and ASCII dump of all the data
              read from file descriptors listed in the specified set.
              For example, to see all input activity on file descriptors
              3 and 5 use -e read=3,5.  Note that this is independent
              from the normal tracing of the read(2) system call which
              is controlled by the option -e trace=read.

       -e write=set
       -e writes=set
       -e w=set
       --write=set
              Perform a full hexadecimal and ASCII dump of all the data
              written to file descriptors listed in the specified set.
              For example, to see all output activity on file
              descriptors 3 and 5 use -e write=3,5.  Note that this is
              independent from the normal tracing of the write(2) system
              call which is controlled by the option -e trace=write.

       -e quiet=set
       -e silent=set
       -e silence=set
       -e q=set
       --quiet=set
       --silent=set
       --silence=set
              Suppress various information messages.  The default is
              quiet=none.  set can include the following elements:

              attach Suppress messages about attaching and detaching ("[
                     Process NNNN attached ]", "[ Process NNNN detached
                     ]").
              exit   Suppress messages about process exits ("+++ exited
                     with SSS +++").
              path-resolution
                     Suppress messages about resolution of paths
                     provided via the -P option ("Requested path "..."
                     resolved into "..."").
              personality
                     Suppress messages about process personality changes
                     ("[ Process PID=NNNN runs in PPP mode. ]").
              thread-execve
              superseded
                     Suppress messages about process being superseded by
                     execve(2) in another thread ("+++ superseded by
                     execve in pid NNNN +++").

       -e decode-fds=set
       --decode-fds=set
              Decode various information associated with file
              descriptors.  The default is decode-fds=none.  set can
              include the following elements:

              path     Print file paths.  Also enables printing of
                       tracee's current working directory when AT_FDCWD
                       constant is used.
              socket   Print socket protocol-specific information,
              dev      Print character/block device numbers.
              pidfd    Print PIDs associated with pidfd file
                       descriptors.
              signalfd Print signal masks associated with signalfd file
                       descriptors.

       -e decode-pids=set
       --decode-pids=set
              Decode various information associated with process IDs
              (and also thread IDs, process group IDs, and session IDs).
              The default is decode-pids=none.  set can include the
              following elements:

              comm    Print command names associated with thread or
                      process IDs.
              pidns   Print thread, process, process group, and session
                      IDs in strace's PID namespace if the tracee is in
                      a different PID namespace.

       -e kvm=vcpu
       --kvm=vcpu
              Print the exit reason of kvm vcpu.  Requires Linux kernel
              version 4.16.0 or higher.

       -i
       --instruction-pointer
              Print the instruction pointer at the time of the system
              call.

       -n
       --syscall-number
              Print the syscall number.

       -k
       --stack-trace[=symbol]
              Print the execution stack trace of the traced processes
              after each system call.

       -kk
       --stack-trace=source
              Print the execution stack trace and source code
              information of the traced processes after each system
              call. This option expects the target program is compiled
              with appropriate debug options: "-g" (gcc), or "-g
              -gdwarf-aranges" (clang).

       --stack-trace-frame-limit=limit
              Print no more than this amount of stack trace frames when
              backtracing a system call (the default is 256).  Use this
              option with the --stack-trace (or -k) option.

       -o filename
       --output=filename
              Write the trace output to the file filename rather than to
              stderr.  filename.pid form is used if -ff option is
              supplied.  If the argument begins with '|' or '!', the
              rest of the argument is treated as a command and all
              output is piped to it.  This is convenient for piping the
              debugging output to a program without affecting the
              redirections of executed programs.  The latter is not
              compatible with -ff option currently.

       -A
       --output-append-mode
              Open the file provided in the -o option in append mode.

       -q
       --quiet
       --quiet=attach,personality
              Suppress messages about attaching, detaching, and
              personality changes.  This happens automatically when
              output is redirected to a file and the command is run
              directly instead of attaching.

       -qq
       --quiet=attach,personality,exit
              Suppress messages attaching, detaching, personality
              changes, and about process exit status.

       -qqq
       --quiet=all
              Suppress all suppressible messages (please refer to the -e
              quiet option description for the full list of suppressible
              messages).

       -r
       --relative-timestamps[=precision]
              Print a relative timestamp upon entry to each system call.
              This records the time difference between the beginning of
              successive system calls.  precision can be one of s (for
              seconds), ms (milliseconds), us (microseconds), or ns
              (nanoseconds), and allows setting the precision of time
              value being printed.  Default is us (microseconds).  Note
              that since -r option uses the monotonic clock time for
              measuring time difference and not the wall clock time, its
              measurements can differ from the difference in time
              reported by the -t option.

       -s strsize
       --string-limit=strsize
              Specify the maximum string size to print (the default is
              32).  Note that filenames are not considered strings and
              are always printed in full.

       --absolute-timestamps[=[[format:]format],[[precision:]precision]]
       --timestamps[=[[format:]format],[[precision:]precision]]
              Prefix each line of the trace with the wall clock time in
              the specified format with the specified precision.  format
              can be one of the following:

              none   No time stamp is printed.  Can be used to override
                     the previous setting.
              time   Wall clock time (strftime(3) format string is %T).
              unix   Number of seconds since the epoch (strftime(3)
                     format string is %s).

              precision can be one of s (for seconds), ms
              (milliseconds), us (microseconds), or ns (nanoseconds).
              Default arguments for the option are
              format:time,precision:s.

       -t
       --absolute-timestamps
              Prefix each line of the trace with the wall clock time.

       -tt
       --absolute-timestamps=precision:us
              If given twice, the time printed will include the
              microseconds.

       -ttt
       --absolute-timestamps=format:unix,precision:us
              If given thrice, the time printed will include the
              microseconds and the leading portion will be printed as
              the number of seconds since the epoch.

       -T
       --syscall-times[=precision]
              Show the time spent in system calls.  This records the
              time difference between the beginning and the end of each
              system call.  precision can be one of s (for seconds), ms
              (milliseconds), us (microseconds), or ns (nanoseconds),
              and allows setting the precision of time value being
              printed.  Default is us (microseconds).

       -v
       --no-abbrev
              Print unabbreviated versions of environment, stat,
              termios, etc.  calls.  These structures are very common in
              calls and so the default behavior displays a reasonable
              subset of structure members.  Use this option to get all
              of the gory details.

       --strings-in-hex[=option]
              Control usage of escape sequences with hexadecimal numbers
              in the printed strings.  Normally (when no
              --strings-in-hex or -x option is supplied), escape
              sequences are used to print non-printable and non-ASCII
              characters (that is, characters with a character code less
              than 32 or greater than 127), or to disambiguate the
              output (so, for quotes and other characters that encase
              the printed string, for example, angle brackets, in case
              of file descriptor path output); for the former use case,
              unless it is a white space character that has a symbolic
              escape sequence defined in the C standard (that is, "\textbackslash t"
              for a horizontal tab, "\textbackslash n" for a newline, "\textbackslash v" for a
              vertical tab, "\textbackslash f" for a form feed page break, and "\textbackslash r"
              for a carriage return) are printed using escape sequences
              with numbers that correspond to their byte values, with
              octal number format being the default.  option can be one
              of the following:

              none   Hexadecimal numbers are not used in the output at
                     all.  When there is a need to emit an escape
                     sequence, octal numbers are used.
              non-ascii-chars
                     Hexadecimal numbers are used instead of octal in
                     the escape sequences.
              non-ascii
                     Strings that contain non-ASCII characters are
                     printed using escape sequences with hexadecimal
                     numbers.
              all    All strings are printed using escape sequences with
                     hexadecimal numbers.

              When the option is supplied without an argument, all is
              assumed.

       -x
       --strings-in-hex=non-ascii
              Print all non-ASCII strings in hexadecimal string format.

       -xx
       --strings-in-hex[=all]
              Print all strings in hexadecimal string format.

       -X format
       --const-print-style=format
              Set the format for printing of named constants and flags.
              Supported format values are:

              raw    Raw number output, without decoding.
              abbrev Output a named constant or a set of flags instead
                     of the raw number if they are found.  This is the
                     default strace behaviour.
              verbose
                     Output both the raw value and the decoded string
                     (as a comment).

       -y
       --decode-fds
       --decode-fds=path
              Print paths associated with file descriptor arguments and
              with the AT_FDCWD constant.

       -yy
       --decode-fds=all
              Print all available information associated with file
              descriptors: protocol-specific information associated with
              socket file descriptors, block/character device number
              associated with device file descriptors, and PIDs
              associated with pidfd file descriptors.

       --pidns-translation
       --decode-pids=pidns
              If strace and tracee are in different PID namespaces,
              print PIDs in strace's namespace, too.

       -Y
       --decode-pids=comm
              Print command names for PIDs.

       --secontext[=format]
       -e secontext=format
              When SELinux is available and is not disabled, print in
              square brackets SELinux contexts of processes, files, and
              descriptors.  The format argument is a comma-separated
              list of items being one of the following:

              full              Print the full context (user, role, type
                                level and category).
              mismatch          Also print the context recorded by the
                                SELinux database in case the current
                                context differs.  The latter is printed
                                after two exclamation marks (!!).

              The default value for --secontext is !full,mismatch which
              prints only the type instead of full context and doesn't
              check for context mismatches.

       --always-show-pid
              Show PID prefix also for the process started by strace.
              Implied when -f and -o are both specified.

   Statistics
       -c
       --summary-only
              Count time, calls, and errors for each system call and
              report a summary on program exit, suppressing the regular
              output.  This attempts to show system time (CPU time spent
              running in the kernel) independent of wall clock time.  If
              -c is used with -f, only aggregate totals for all traced
              processes are kept.

       -C
       --summary
              Like -c but also print regular output while processes are
              running.

       -O overhead
       --summary-syscall-overhead=overhead
              Set the overhead for tracing system calls to overhead.
              This is useful for overriding the default heuristic for
              guessing how much time is spent in mere measuring when
              timing system calls using the -c option.  The accuracy of
              the heuristic can be gauged by timing a given program run
              without tracing (using time(1)) and comparing the
              accumulated system call time to the total produced using
              -c.

              The format of overhead specification is described in
              section Time specification format description.

       -S sortby
       --summary-sort-by=sortby
              Sort the output of the histogram printed by the -c option
              by the specified criterion.  Legal values are time (or
              time-percent or time-total or total-time), min-time (or
              shortest or time-min), max-time (or longest or time-max),
              avg-time (or time-avg), calls (or count), errors (or
              error), name (or syscall or syscall-name), and nothing (or
              none); default is time.

       -U columns
       --summary-columns=columns
              Configure a set (and order) of columns being shown in the
              call summary.  The columns argument is a comma-separated
              list with items being one of the following:

              time-percent (or time)
                     Percentage of cumulative time consumed by a
                     specific system call.
              total-time (or time-total)
                     Total system (or wall clock, if -w option is
                     provided) time consumed by a specific system call.
              min-time (or shortest or time-min)
                     Minimum observed call duration.
              max-time (or longest or time-max)
                     Maximum observed call duration.
              avg-time (or time-avg)
                     Average call duration.
              calls (or count)
                     Call count.
              errors (or error)
                     Error count.
              name (or syscall or syscall-name)
                     Syscall name.

              The default value is
              time-percent,total-time,avg-time,calls,errors,name.  If
              the name field is not supplied explicitly, it is added as
              the last column.

       -w
       --summary-wall-clock
              Summarise the time difference between the beginning and
              end of each system call.  The default is to summarise the
              system time.

   Tampering
       -e inject=syscall_set[:error=errno|:retval=value][:signal=sig]
       [:syscall=syscall][:delay_enter=delay][:delay_exit=delay]
       [:poke_enter=@argN=DATAN,@argM=DATAM...]
       [:poke_exit=@argN=DATAN,@argM=DATAM...][:when=expr]
       --inject=syscall_set[:error=errno|:retval=value][:signal=sig]
       [:syscall=syscall][:delay_enter=delay][:delay_exit=delay]
       [:poke_enter=@argN=DATAN,@argM=DATAM...]
       [:poke_exit=@argN=DATAN,@argM=DATAM...][:when=expr]
              Perform   syscall  tampering  for  the  specified  set  of
              syscalls.  The syntax of the syscall_set specification  is
              the same as in the -e trace option.

              At  least  one  of  error,  retval,  signal,  delay_enter,
              delay_exit, poke_enter, or poke_exit  options  has  to  be
              specified.  error and retval are mutually exclusive.

              If  :error=errno  option is specified, a fault is injected
              into a syscall invocation: the syscall number is  replaced
              by  -1  which  corresponds to an invalid syscall (unless a
              syscall is specified with :syscall= option), and the error
              code is specified using a symbolic errno value like ENOSYS
              or a numeric value within 1..4095 range.

              If :retval=value option is specified, success injection is
              performed: the syscall number is replaced  by  -1,  but  a
              bogus success value is returned to the callee.

              If  :signal=sig option is specified with either a symbolic
              value like SIGSEGV or a numeric value  within  1..SIGRTMAX
              range,  that signal is delivered on entering every syscall
              specified by the set.

              If :delay_enter=delay  or  :delay_exit=delay  options  are
              specified,  delay  injection  is  performed: the tracee is
              delayed by time period specified by delay on  entering  or
              exiting  the  syscall,  respectively.  The format of delay
              specification is described in section  Time  specification
              format description.

              If        :poke_enter=@argN=DATAN,@argM=DATAM...        or
              :poke_exit=@argN=DATAN,@argM=DATAM...     options      are
              specified,  tracee's  memory  at  locations, pointed to by
              system call arguments argN and argM (going  from  arg1  to
              arg7) is overwritten by data DATAN and DATAM (specified in
              hexadecimal          format;          for          example
              :poke_enter=@arg1=0000DEAD0000BEEF).  :poke_enter modifies
              memory on syscall enter, and :poke_exit - on exit.

              If :signal=sig option is specified  without  :error=errno,
              :retval=value  or  :delay_{enter,exit}=usecs options, then
              only a signal sig is delivered without a syscall fault  or
              delay     injection.     Conversely,    :error=errno    or
              :retval=value    option    without     :delay_enter=delay,
              :delay_exit=delay  or  :signal=sig options injects a fault
              without delivering a signal or injecting a delay, etc.

              If  :signal=sig  option   is   specified   together   with
              :error=errno  or  :retval=value,  then both injection of a
              fault or success and signal delivery are performed.

              if :syscall=syscall option is specified, the corresponding
              syscall with no side effects is injected  instead  of  -1.
              Currently,  only  "pure"  (see -e trace=%pure description)
              syscalls can be specified there.

              Unless  a  :when=expr  subexpression  is   specified,   an
              injection  is  being  made  into  every invocation of each
              syscall from the set.

              The format of the subexpression is:

                             first[..last][+[step]]

              Number first stands for the first invocation number in the
              range, number last stands for the last  invocation  number
              in  the  range,  and  step stands for the step between two
              consecutive invocations.  The following  combinations  are
              useful:

              first  For every syscall from the set, perform an
                     injection for the syscall invocation number first
                     only.
              first..last
                     For every syscall from the set, perform an
                     injection for the syscall invocation number first
                     and all subsequent invocations until the invocation
                     number last (inclusive).
              first+ For every syscall from the set, perform injections
                     for the syscall invocation number first and all
                     subsequent invocations.
              first..last+
                     For every syscall from the set, perform injections
                     for the syscall invocation number first and all
                     subsequent invocations until the invocation number
                     last (inclusive).
              first+step
                     For every syscall from the set, perform injections
                     for syscall invocations number first, first+step,
                     first+step+step, and so on.
              first..last+step
                     Same as the previous, but consider only syscall
                     invocations with numbers up to last (inclusive).

              For example, to fail each third and subsequent chdir
              syscalls with ENOENT, use
              -e inject=chdir:error=ENOENT:when=3+.

              The valid range for numbers first and step is 1..65535,
              and for number last is 1..65534.

              An injection expression can contain only one error= or
              retval= specification, and only one signal= specification.
              If an injection expression contains multiple when=
              specifications, the last one takes precedence.

              Accounting of syscalls that are subject to injection is
              done per syscall and per tracee.

              Specification of syscall injection can be combined with
              other syscall filtering options, for example, -P
              /dev/urandom -e inject=file:error=ENOENT.

       -e fault=syscall_set[:error=errno][:when=expr]
       --fault=syscall_set[:error=errno][:when=expr]
              Perform syscall fault injection for the specified set of
              syscalls.

              This is equivalent to more generic -e inject= expression
              with default value of errno option set to ENOSYS.

   Miscellaneous
       -d
       --debug
              Show some debugging output of strace itself on the
              standard error.

       -F     This option is deprecated.  It is retained for backward
              compatibility only and may be removed in future releases.
              Usage of multiple instances of -F option is still
              equivalent to a single -f, and it is ignored at all if
              used along with one or more instances of -f option.

       -h
       --help Print the help summary.

       --seccomp-bpf
              Try to enable use of seccomp-bpf (see seccomp(2)) to have
              ptrace(2)-stops only when system calls that are being
              traced occur in the traced processes.

              This option has no effect unless -f/--follow-forks is also
              specified.  --seccomp-bpf is not compatible with
              --syscall-limit and -b/--detach-on options.  It is also
              not applicable to processes attached using -p/--attach
              option.

              An attempt to enable system calls filtering using seccomp-
              bpf may fail for various reasons, e.g. there are too many
              system calls to filter, the seccomp API is not available,
              or strace itself is being traced.  In cases when seccomp-
              bpf filter setup failed, strace proceeds as usual and
              stops traced processes on every system call.

              When --seccomp-bpf is activated and -p/--attach option is
              not used, --kill-on-exit option is activated as well.

              Note that in cases when the tracee has another seccomp
              filter that returns an action value with a precedence
              greater than SECCOMP_RET_TRACE, strace --seccomp-bpf will
              not be notified.  That is, if another seccomp filter, for
              example, disables the syscall or kills the tracee, then
              strace --seccomp-bpf will not be aware of that syscall
              invocation at all.

       --tips[=[[id:]id],[[format:]format]]
              Show strace tips, tricks, and tweaks before exit.  id can
              be a non-negative integer number, which enables printing
              of specific tip, trick, or tweak (these ID are not
              guaranteed to be stable), or random (the default), in
              which case a random tip is printed.  format can be one of
              the following:

              none     No tip is printed.  Can be used to override the
                       previous setting.
              compact  Print the tip just big enough to contain all the
                       text.
              full     Print the tip in its full glory.

              Default is id:random,format:compact.

       -V
       --version
              Print the version number of strace.  Multiple instances of
              the option beyond specific threshold tend to increase
              Strauss awareness.

   Time specification format description
       Time values can be specified as a decimal floating point number
       (in a format accepted by strtod(3)), optionally followed by one
       of the following suffices that specify the unit of time: s
       (seconds), ms (milliseconds), us (microseconds), or ns
       (nanoseconds).  If no suffix is specified, the value is
       interpreted as microseconds.

       The described format is used for -O, -e inject=delay_enter, and
       -e inject=delay_exit options.
DIAGNOSTICS
       When command exits, strace exits with the same exit status.  If
       command is terminated by a signal, strace terminates itself with
       the same signal, so that strace can be used as a wrapper process
       transparent to the invoking parent process.  Note that parent-
       child relationship (signal stop notifications, getppid(2) value,
       etc) between traced process and its parent are not preserved
       unless -D is used.

       When using -p without a command, the exit status of strace is
       zero unless no processes has been attached or there was an
       unexpected error in doing the tracing.
SETUID INSTALLATION
       If strace is installed setuid to root then the invoking user will
       be able to attach to and trace processes owned by any user.  In
       addition setuid and setgid programs will be executed and traced
       with the correct effective privileges.  Since only users trusted
       with full root privileges should be allowed to do these things,
       it only makes sense to install strace as setuid to root when the
       users who can execute it are restricted to those users who have
       this trust.  For example, it makes sense to install a special
       version of strace with mode 'rwsr-xr--', user root and group
       trace, where members of the trace group are trusted users.  If
       you do use this feature, please remember to install a regular
       non-setuid version of strace for ordinary users to use.
MULTIPLE PERSONALITIES SUPPORT
       On some architectures, strace supports decoding of syscalls for
       processes that use different ABI rather than the one strace uses.
       Specifically, in addition to decoding native ABI, strace can
       decode the following ABIs on the following architectures:

       [1]  When strace is built as an x86_64 application
       [2]  When strace is built as an x32 application
       [3]  Big endian only

       This support is optional and relies on ability to generate and
       parse structure definitions during the build time.  Please refer
       to the output of the strace -V command in order to figure out
       what support is available in your strace build ("non-native"
       refers to an ABI that differs from the ABI strace has):

       m32-mpers
              strace can trace and properly decode non-native 32-bit
              binaries.
       no-m32-mpers
              strace can trace, but cannot properly decode non-native
              32-bit binaries.
       mx32-mpers
              strace can trace and properly decode non-native
              32-on-64-bit binaries.
       no-mx32-mpers
              strace can trace, but cannot properly decode non-native
              32-on-64-bit binaries.

       If the output contains neither m32-mpers nor no-m32-mpers, then
       decoding of non-native 32-bit binaries is not implemented at all
       or not applicable.

       Likewise, if the output contains neither mx32-mpers nor no-
       mx32-mpers, then decoding of non-native 32-on-64-bit binaries is
       not implemented at all or not applicable.
NOTES
       It is a pity that so much tracing clutter is produced by systems
       employing shared libraries.

       It is instructive to think about system call inputs and outputs
       as data-flow across the user/kernel boundary.  Because user-space
       and kernel-space are separate and address-protected, it is
       sometimes possible to make deductive inferences about process
       behavior using inputs and outputs as propositions.

       In some cases, a system call will differ from the documented
       behavior or have a different name.  For example, the faccessat(2)
       system call does not have flags argument, and the setrlimit(2)
       library function uses prlimit64(2) system call on modern
       (2.6.38+) kernels.  These discrepancies are normal but
       idiosyncratic characteristics of the system call interface and
       are accounted for by C library wrapper functions.

       Some system calls have different names in different architectures
       and personalities.  In these cases, system call filtering and
       printing uses the names that match corresponding __NR_* kernel
       macros of the tracee's architecture and personality.  There are
       two exceptions from this general rule: arm_fadvise64_64(2) ARM
       syscall and xtensa_fadvise64_64(2) Xtensa syscall are filtered
       and printed as fadvise64_64(2).

       On x32, syscalls that are intended to be used by 64-bit processes
       and not x32 ones (for example, readv(2), that has syscall number
       19 on x86_64, with its x32 counterpart has syscall number 515),
       but called with __X32_SYSCALL_BIT flag being set, are designated
       with #64 suffix.

       On some platforms a process that is attached to with the -p
       option may observe a spurious EINTR return from the current
       system call that is not restartable.  (Ideally, all system calls
       should be restarted on strace attach, making the attach invisible
       to the traced process, but a few system calls aren't.  Arguably,
       every instance of such behavior is a kernel bug.)  This may have
       an unpredictable effect on the process if the process takes no
       action to restart the system call.

       As strace executes the specified command directly and does not
       employ a shell for that, scripts without shebang that usually run
       just fine when invoked by shell fail to execute with ENOEXEC
       error.  It is advisable to manually supply a shell as a command
       with the script as its argument.
BUGS
       Programs that use the setuid bit do not have effective user ID
       privileges while being traced.

       A traced process runs slowly (but check out the --seccomp-bpf
       option).

       Unless --kill-on-exit option is used (or --seccomp-bpf option is
       used in a way that implies --kill-on-exit), traced processes
       which are descended from command may be left running after an
       interrupt signal (CTRL-C).

       By using CLONE_UNTRACED flag of clone system call a tracee can
       break the guarantee that --seccomp-bpf will not leave any
       processes with a seccomp program installed for syscall filtering
       purposes.
HISTORY
       The original strace was written by Paul Kranenburg for SunOS and
       was inspired by its trace utility.  The SunOS version of strace
       was ported to Linux and enhanced by Branko Lankester, who also
       wrote the Linux kernel support.  Even though Paul released strace
       2.5 in 1992, Branko's work was based on Paul's strace 1.5 release
       from 1991.  In 1993, Rick Sladkey merged strace 2.5 for SunOS and
       the second release of strace for Linux, added many of the
       features of truss(1) from SVR4, and produced an strace that
       worked on both platforms.  In 1994 Rick ported strace to SVR4 and
       Solaris and wrote the automatic configuration support.  In 1995
       he ported strace to Irix and became tired of writing about
       himself in the third person.

       Beginning with 1996, strace was maintained by Wichert Akkerman.
       During his tenure, strace development migrated to CVS; ports to
       FreeBSD and many architectures on Linux (including ARM, IA-64,
       MIPS, PA-RISC, PowerPC, s390, SPARC) were introduced.  In 2002,
       the burden of strace maintainership was transferred to Roland
       McGrath.  Since then, strace gained support for several new Linux
       architectures (AMD64, s390x, SuperH), bi-architecture support for
       some of them, and received numerous additions and improvements in
       syscalls decoders on Linux; strace development migrated to Git
       during that period.  Since 2009, strace is actively maintained by
       Dmitry Levin.  strace gained support for AArch64, ARC, AVR32,
       Blackfin, Meta, Nios II, OpenRISC 1000, RISC-V, Tile/TileGx,
       Xtensa architectures since that time.  In 2012, unmaintained and
       apparently broken support for non-Linux operating systems was
       removed.  Also, in 2012 strace gained support for path tracing
       and file descriptor path decoding.  In 2014, support for stack
       trace printing was added.  In 2016, syscall fault injection was
       implemented.

       For the additional information, please refer to the NEWS file and
       strace repository commit log.
REPORTING BUGS
       Problems with strace should be reported to the strace mailing
       list mailto:strace-devel@lists.strace.io.
SEE ALSO
       strace-log-merge(1), ltrace(1), perf-trace(1), trace-cmd(1),
       time(1), ptrace(2), seccomp(2), syscall(2), proc(5), signal(7)

       strace Home Page https://strace.io/
AUTHORS
       The complete list of strace contributors can be found in the
       CREDITS file.
COLOPHON
       This page is part of the strace (system call tracer) project.
       Information about the project can be found at 
       http://strace.io/.  If you have a bug report for this manual
       page, send it to strace-devel@lists.sourceforge.net.  This page
       was obtained from the project's upstream Git repository
       https://github.com/strace/strace.git on 2024-06-14.  (At that
       time, the date of the most recent commit that was found in the
       repository was 2024-06-04.)  If you discover any rendering
       problems in this HTML version of the page, or you believe there
       is a better or more up-to-date source for the page, or you have
       corrections or improvements to the information in this COLOPHON
       (which is not part of the original manual page), send a mail to
       man-pages@man7.org

strace 6.9.0.16.2a4c4          2024-06-04                      STRACE(1)
\end{lstlisting}
}}

\endinput  %  ==  ==  ==  ==  ==  ==  ==  ==  ==
\subsection{\refStrace: Trace System Calls and Signals}

{\tiny{
\begin{lstlisting}[language=bash]
NAME
       strace - trace system calls and signals
SYNOPSIS
       strace [-ACdffhikkqqrtttTvVwxxyyYzZ] [-a column] [-b execve]
              [-e expr]... [-I n] [-o file] [-O overhead] [-p pid]...
              [-P path]... [-s strsize] [-S sortby] [-U columns]
              [-X format] [--seccomp-bpf]
              [--stack-trace-frame-limit=limit] [--syscall-limit=limit]
              [--secontext[=format]] [--tips[=format]] { -p pid | [-DDD]
              [-E var[=val]]... [-u username] command [args] }

       strace -c [-dfwzZ] [-b execve] [-e expr]... [-I n] [-O overhead]
              [-p pid]... [-P path]... [-S sortby] [-U columns]
              [--seccomp-bpf] [--syscall-limit=limit] [--tips[=format]]
              { -p pid | [-DDD] [-E var[=val]]... [-u username] command
              [args] }

       strace --tips[=format]
DESCRIPTION
       In the simplest case strace runs the specified command until it
       exits.  It intercepts and records the system calls which are
       called by a process and the signals which are received by a
       process.  The name of each system call, its arguments and its
       return value are printed on standard error or to the file
       specified with the -o option.

       strace is a useful diagnostic, instructional, and debugging tool.
       System administrators, diagnosticians and trouble-shooters will
       find it invaluable for solving problems with programs for which
       the source is not readily available since they do not need to be
       recompiled in order to trace them.  Students, hackers and the
       overly-curious will find that a great deal can be learned about a
       system and its system calls by tracing even ordinary programs.
       And programmers will find that since system calls and signals are
       events that happen at the user/kernel interface, a close
       examination of this boundary is very useful for bug isolation,
       sanity checking and attempting to capture race conditions.

       Each line in the trace contains the system call name, followed by
       its arguments in parentheses and its return value.  An example
       from stracing the command "cat /dev/null" is:

           open("/dev/null", O_RDONLY) = 3

       Errors (typically a return value of -1) have the errno symbol and
       error string appended.

           open("/foo/bar", O_RDONLY) = -1 ENOENT (No such file or directory)

       Signals are printed as signal symbol and decoded siginfo
       structure.  An excerpt from stracing and interrupting the command
       "sleep 666" is:

           sigsuspend([] <unfinished ...>
           --- SIGINT {si_signo=SIGINT, si_code=SI_USER, si_pid=...} ---
           +++ killed by SIGINT +++

       If a system call is being executed and meanwhile another one is
       being called from a different thread/process then strace will try
       to preserve the order of those events and mark the ongoing call
       as being unfinished.  When the call returns it will be marked as
       resumed.

           [pid 28772] select(4, [3], NULL, NULL, NULL <unfinished ...>
           [pid 28779] clock_gettime(CLOCK_REALTIME, {tv_sec=1130322148, tv_nsec=3977000}) = 0
           [pid 28772] <... select resumed> )      = 1 (in [3])

       Interruption of a (restartable) system call by a signal delivery
       is processed differently as kernel terminates the system call and
       also arranges its immediate reexecution after the signal handler
       completes.

           read(0, 0x7ffff72cf5cf, 1)              = ? ERESTARTSYS (To be restarted)
           --- SIGALRM {si_signo=SIGALRM, si_code=SI_KERNEL} ---
           rt_sigreturn({mask=[]})                 = 0
           read(0, "", 1)                          = 0

       Arguments are printed in symbolic form with passion.  This
       example shows the shell performing ">>xyzzy" output redirection:

           open("xyzzy", O_WRONLY|O_APPEND|O_CREAT, 0666) = 3

       Here, the second and the third argument of open(2) are decoded by
       breaking down the flag argument into its three bitwise-OR
       constituents and printing the mode value in octal by tradition.
       Where the traditional or native usage differs from ANSI or POSIX,
       the latter forms are preferred.  In some cases, strace output is
       proven to be more readable than the source.

       Structure pointers are dereferenced and the members are displayed
       as appropriate.  In most cases, arguments are formatted in the
       most C-like fashion possible.  For example, the essence of the
       command "ls -l /dev/null" is captured as:

           lstat("/dev/null", {st_mode=S_IFCHR|0666, st_rdev=makedev(0x1, 0x3), ...}) = 0

       Notice how the 'struct stat' argument is dereferenced and how
       each member is displayed symbolically.  In particular, observe
       how the st_mode member is carefully decoded into a bitwise-OR of
       symbolic and numeric values.  Also notice in this example that
       the first argument to lstat(2) is an input to the system call and
       the second argument is an output.  Since output arguments are not
       modified if the system call fails, arguments may not always be
       dereferenced.  For example, retrying the "ls -l" example with a
       non-existent file produces the following line:

           lstat("/foo/bar", 0xb004) = -1 ENOENT (No such file or directory)

       In this case the porch light is on but nobody is home.

       Syscalls unknown to strace are printed raw, with the unknown
       system call number printed in hexadecimal form and prefixed with
       "syscall_":

           syscall_0xbad(0x1, 0x2, 0x3, 0x4, 0x5, 0x6) = -1 ENOSYS (Function not implemented)

       Character pointers are dereferenced and printed as C strings.
       Non-printing characters in strings are normally represented by
       ordinary C escape codes.  Only the first strsize (32 by default)
       bytes of strings are printed; longer strings have an ellipsis
       appended following the closing quote.  Here is a line from "ls
       -l" where the getpwuid(3) library routine is reading the password
       file:

           read(3, "root::0:0:System Administrator:/"..., 1024) = 422

       While structures are annotated using curly braces, pointers to
       basic types and arrays are printed using square brackets with
       commas separating the elements.  Here is an example from the
       command id(1) on a system with supplementary group ids:

           getgroups(32, [100, 0]) = 2

       On the other hand, bit-sets are also shown using square brackets,
       but set elements are separated only by a space.  Here is the
       shell, preparing to execute an external command:

           sigprocmask(SIG_BLOCK, [CHLD TTOU], []) = 0

       Here, the second argument is a bit-set of two signals, SIGCHLD
       and SIGTTOU.  In some cases, the bit-set is so full that printing
       out the unset elements is more valuable.  In that case, the bit-
       set is prefixed by a tilde like this:

           sigprocmask(SIG_UNBLOCK, ~[], NULL) = 0

       Here, the second argument represents the full set of all signals.
OPTIONS
   General
       -e expr
              A qualifying expression which modifies which events to
              trace or how to trace them.  The format of the expression
              is:

                             [qualifier=][!]value[,value]...

              where qualifier is one of trace (or t), trace-fds (or
              trace-fd or fd or fds), abbrev (or a), verbose (or v), raw
              (or x), signal (or signals or s), read (or reads or r),
              write (or writes or w), fault, inject, status, quiet (or
              silent or silence or q), secontext, decode-fds (or
              decode-fd), decode-pids (or decode-pid), or kvm, and value
              is a qualifier-dependent symbol or number.  The default
              qualifier is trace.  Using an exclamation mark negates the
              set of values.  For example, -e open means literally
              -e trace=open which in turn means trace only the open
              system call.  By contrast, -e trace=!open means to trace
              every system call except open.  In addition, the special
              values all and none have the obvious meanings.

              Note that some shells use the exclamation point for
              history expansion even inside quoted arguments.  If so,
              you must escape the exclamation point with a backslash.

   Startup
       -E var=val
       --env=var=val
              Run command with var=val in its list of environment
              variables.

       -E var
       --env=var
              Remove var from the inherited list of environment
              variables before passing it on to the command.

       -p pid
       --attach=pid
              Attach to the process with the process ID pid and begin
              tracing.  The trace may be terminated at any time by a
              keyboard interrupt signal (CTRL-C).  strace will respond
              by detaching itself from the traced process(es) leaving it
              (them) to continue running.  Multiple -p options can be
              used to attach to many processes in addition to command
              (which is optional if at least one -p option is given).
              Multiple process IDs, separated by either comma (",''),
              space (" "), tab, or newline character, can be provided as
              an argument to a single -p option, so, for example, -p
              "$(pidof PROG)" and -p "$(pgrep PROG)" syntaxes are
              supported.

       -u username
       --user=username
              Run command with the user ID, group ID, and supplementary
              groups of username.  This option is only useful when
              running as root and enables the correct execution of
              setuid and/or setgid binaries.  Unless this option is used
              setuid and setgid programs are executed without effective
              privileges.
       -u UID:GID
       --user=UID:GID
              Alternative syntax where the program is started with
              exactly the given user and group IDs, and an empty list of
              supplementary groups.  In this case, user and group name
              lookups are not performed.

       --argv0=name
              Set argv[0] of the command being executed to name.  Useful
              for tracing multi-call executables which interpret
              argv[0], such as busybox or kmod.

   Tracing
       -b syscall
       --detach-on=syscall
              If specified syscall is reached, detach from traced
              process.  Currently, only execve(2) syscall is supported.
              This option is useful if you want to trace multi-threaded
              process and therefore require -f, but don't want to trace
              its (potentially very complex) children.

       -D
       --daemonize
       --daemonize=grandchild
              Run tracer process as a grandchild, not as the parent of
              the tracee.  This reduces the visible effect of strace by
              keeping the tracee a direct child of the calling process.

       -DD
       --daemonize=pgroup
       --daemonize=pgrp
              Run tracer process as tracee's grandchild in a separate
              process group.  In addition to reduction of the visible
              effect of strace, it also avoids killing of strace with
              kill(2) issued to the whole process group.

       -DDD
       --daemonize=session
              Run tracer process as tracee's grandchild in a separate
              session ("true daemonisation").  In addition to reduction
              of the visible effect of strace, it also avoids killing of
              strace upon session termination.

       -f
       --follow-forks
              Trace child processes as they are created by currently
              traced processes as a result of the fork(2), vfork(2) and
              clone(2) system calls.  Note that -p PID -f will attach
              all threads of process PID if it is multi-threaded, not
              only thread with thread_id = PID.

       --output-separately
              If the --output=filename option is in effect, each
              processes trace is written to filename.pid where pid is
              the numeric process id of each process.

       -ff
       --follow-forks --output-separately
              Combine the effects of --follow-forks and
              --output-separately options.  This is incompatible with
              -c, since no per-process counts are kept.

              One might want to consider using strace-log-merge(1) to
              obtain a combined strace log view.

       -I interruptible
       --interruptible=interruptible
              When strace can be interrupted by signals (such as
              pressing CTRL-C).

              1, anywhere
                     no signals are blocked;
              2, waiting
                     fatal signals are blocked while decoding syscall
                     (default);
              3, never
                     fatal signals are always blocked (default if -o
                     FILE PROG);
              4, never_tstp
                     fatal signals and SIGTSTP (CTRL-Z) are always
                     blocked (useful to make strace -o FILE PROG not
                     stop on CTRL-Z, default if -D).

       --syscall-limit=limit
              Detach all tracees when limit number of syscalls have been
              captured. Syscalls filtered out via --trace, --trace-path
              or --status options are not considered when keeping track
              of the number of syscalls that are captured.

       --kill-on-exit
              Apply PTRACE_O_EXITKILL ptrace option to all tracee
              processes (which sends a SIGKILL signal to the tracee if
              the tracer exits) and do not detach them on cleanup so
              they will not be left running after the tracer exit.
              --kill-on-exit is not compatible with -p/--attach options.

   Filtering
       -e trace=syscall_set
       -e t=syscall_set
       --trace=syscall_set
              Trace only the specified set of system calls.  syscall_set
              is defined as [!]value[,value], and value can be one of
              the following:

              syscall
                     Trace specific syscall, specified by its name (see
                     syscalls(2) for a reference, but also see NOTES).

              ?value Question mark before the syscall qualification
                     allows suppression of error in case no syscalls
                     matched the qualification provided.

              value@64
                     Limit the syscall specification described by value
                     to 64-bit personality.

              value@32
                     Limit the syscall specification described by value
                     to 32-bit personality.

              value@x32
                     Limit the syscall specification described by value
                     to x32 personality.

              all    Trace all system calls.

              /regex Trace only those system calls that match the regex.
                     You can use POSIX Extended Regular Expression
                     syntax (see regex(7)).

              %file
              file   Trace all system calls which take a file name as an
                     argument.  You can think of this as an abbreviation
                     for -e trace=open,stat,chmod,unlink,...  which is
                     useful to seeing what files the process is
                     referencing.  Furthermore, using the abbreviation
                     will ensure that you don't accidentally forget to
                     include a call like lstat(2) in the list.  Betchya
                     woulda forgot that one.  The syntax without a
                     preceding percent sign ("-e trace=file") is
                     deprecated.

              %process
              process
                     Trace system calls associated with process
                     lifecycle (creation, exec, termination).  The
                     syntax without a preceding percent sign ("-e
                     trace=process") is deprecated.

              %net
              %network
              network
                     Trace all the network related system calls.  The
                     syntax without a preceding percent sign ("-e
                     trace=network") is deprecated.

              %signal
              signal Trace all signal related system calls.  The syntax
                     without a preceding percent sign ("-e
                     trace=signal") is deprecated.

              %ipc
              ipc    Trace all IPC related system calls.  The syntax
                     without a preceding percent sign ("-e trace=ipc")
                     is deprecated.

              %desc
              desc   Trace all file descriptor related system calls.
                     The syntax without a preceding percent sign ("-e
                     trace=desc") is deprecated.

              %memory
              memory Trace all memory mapping related system calls.  The
                     syntax without a preceding percent sign ("-e
                     trace=memory") is deprecated.

              %creds Trace system calls that read or modify user and
                     group identifiers or capability sets.

              %stat  Trace stat syscall variants.

              %lstat Trace lstat syscall variants.

              %fstat Trace fstat, fstatat, and statx syscall variants.

              %%stat Trace syscalls used for requesting file status
                     (stat, lstat, fstat, fstatat, statx, and their
                     variants).

              %statfs
                     Trace statfs, statfs64, statvfs, osf_statfs, and
                     osf_statfs64 system calls.  The same effect can be
                     achieved with -e trace=/^(.*_)?statv?fs regular
                     expression.

              %fstatfs
                     Trace fstatfs, fstatfs64, fstatvfs, osf_fstatfs,
                     and osf_fstatfs64 system calls.  The same effect
                     can be achieved with -e trace=/fstatv?fs regular
                     expression.

              %%statfs
                     Trace syscalls related to file system statistics
                     (statfs-like, fstatfs-like, and ustat).  The same
                     effect can be achieved with
                     -e trace=/statv?fs|fsstat|ustat regular expression.

              %clock Trace system calls that read or modify system
                     clocks.

              %pure  Trace syscalls that always succeed and have no
                     arguments.  Currently, this list includes
                     arc_gettls(2), getdtablesize(2), getegid(2),
                     getegid32(2), geteuid(2), geteuid32(2), getgid(2),
                     getgid32(2), getpagesize(2), getpgrp(2), getpid(2),
                     getppid(2), get_thread_area(2) (on architectures
                     other than x86), gettid(2), get_tls(2), getuid(2),
                     getuid32(2), getxgid(2), getxpid(2), getxuid(2),
                     kern_features(2), and metag_get_tls(2) syscalls.

              The -c option is useful for determining which system calls
              might be useful to trace.  For example,
              trace=open,close,read,write means to only trace those four
              system calls.  Be careful when making inferences about the
              user/kernel boundary if only a subset of system calls are
              being monitored.  The default is trace=all.

       -e trace-fd=set
       -e trace-fds=set
       -e fd=set
       -e fds=set
       --trace-fds=set
              Trace only the syscalls that operate on the specified
              subset of (non-negative) file descriptors.  Note that
              usage of this option also filters out all the syscalls
              that do not operate on file descriptors at all.  Applies
              in (inclusive) disjunction with the --trace-path option.

       -e signal=set
       -e signals=set
       -e s=set
       --signal=set
              Trace only the specified subset of signals.  The default
              is signal=all.  For example, signal=!SIGIO (or signal=!io)
              causes SIGIO signals not to be traced.

       -e status=set
       --status=set
              Print only system calls with the specified return status.
              The default is status=all.  When using the status
              qualifier, because strace waits for system calls to return
              before deciding whether they should be printed or not, the
              traditional order of events may not be preserved anymore.
              If two system calls are executed by concurrent threads,
              strace will first print both the entry and exit of the
              first system call to exit, regardless of their respective
              entry time.  The entry and exit of the second system call
              to exit will be printed afterwards.  Here is an example
              when select(2) is called, but a different thread calls
              clock_gettime(2) before select(2) finishes:

                  [pid 28779] 1130322148.939977 clock_gettime(CLOCK_REALTIME, {1130322148, 939977000}) = 0
                  [pid 28772] 1130322148.438139 select(4, [3], NULL, NULL, NULL) = 1 (in [3])

              set can include the following elements:

              successful
                     Trace system calls that returned without an error
                     code.  The -z option has the effect of
                     status=successful.
              failed Trace system calls that returned with an error
                     code.  The -Z option has the effect of
                     status=failed.
              unfinished
                     Trace system calls that did not return.  This might
                     happen, for example, due to an execve call in a
                     neighbour thread.
              unavailable
                     Trace system calls that returned but strace failed
                     to fetch the error status.
              detached
                     Trace system calls for which strace detached before
                     the return.

       -P path
       --trace-path=path
              Trace only system calls accessing path.  Multiple -P
              options can be used to specify several paths.  Applies in
              (inclusive) disjunction with the --trace-fds option.

       -z
       --successful-only
              Print only syscalls that returned without an error code.

       -Z
       --failed-only
              Print only syscalls that returned with an error code.

   Output format
       -a column
       --columns=column
              Align return values in a specific column (default column
              40).

       -e abbrev=syscall_set
       -e a=syscall_set
       --abbrev=syscall_set
              Abbreviate the output from printing each member of large
              structures.  The syntax of the syscall_set specification
              is the same as in the -e trace option.  The default is
              abbrev=all.  The -v option has the effect of abbrev=none.

       -e verbose=syscall_set
       -e v=syscall_set
       --verbose=syscall_set
              Dereference structures for the specified set of system
              calls.  The syntax of the syscall_set specification is the
              same as in the -e trace option.  The default is
              verbose=all.

       -e raw=syscall_set
       -e x=syscall_set
       --raw=syscall_set
              Print raw, undecoded arguments for the specified set of
              system calls.  The syntax of the syscall_set specification
              is the same as in the -e trace option.  This option has
              the effect of causing all arguments to be printed in
              hexadecimal.  This is mostly useful if you don't trust the
              decoding or you need to know the actual numeric value of
              an argument.  See also -X raw option.

       -e read=set
       -e reads=set
       -e r=set
       --read=set
              Perform a full hexadecimal and ASCII dump of all the data
              read from file descriptors listed in the specified set.
              For example, to see all input activity on file descriptors
              3 and 5 use -e read=3,5.  Note that this is independent
              from the normal tracing of the read(2) system call which
              is controlled by the option -e trace=read.

       -e write=set
       -e writes=set
       -e w=set
       --write=set
              Perform a full hexadecimal and ASCII dump of all the data
              written to file descriptors listed in the specified set.
              For example, to see all output activity on file
              descriptors 3 and 5 use -e write=3,5.  Note that this is
              independent from the normal tracing of the write(2) system
              call which is controlled by the option -e trace=write.

       -e quiet=set
       -e silent=set
       -e silence=set
       -e q=set
       --quiet=set
       --silent=set
       --silence=set
              Suppress various information messages.  The default is
              quiet=none.  set can include the following elements:

              attach Suppress messages about attaching and detaching ("[
                     Process NNNN attached ]", "[ Process NNNN detached
                     ]").
              exit   Suppress messages about process exits ("+++ exited
                     with SSS +++").
              path-resolution
                     Suppress messages about resolution of paths
                     provided via the -P option ("Requested path "..."
                     resolved into "..."").
              personality
                     Suppress messages about process personality changes
                     ("[ Process PID=NNNN runs in PPP mode. ]").
              thread-execve
              superseded
                     Suppress messages about process being superseded by
                     execve(2) in another thread ("+++ superseded by
                     execve in pid NNNN +++").

       -e decode-fds=set
       --decode-fds=set
              Decode various information associated with file
              descriptors.  The default is decode-fds=none.  set can
              include the following elements:

              path     Print file paths.  Also enables printing of
                       tracee's current working directory when AT_FDCWD
                       constant is used.
              socket   Print socket protocol-specific information,
              dev      Print character/block device numbers.
              pidfd    Print PIDs associated with pidfd file
                       descriptors.
              signalfd Print signal masks associated with signalfd file
                       descriptors.

       -e decode-pids=set
       --decode-pids=set
              Decode various information associated with process IDs
              (and also thread IDs, process group IDs, and session IDs).
              The default is decode-pids=none.  set can include the
              following elements:

              comm    Print command names associated with thread or
                      process IDs.
              pidns   Print thread, process, process group, and session
                      IDs in strace's PID namespace if the tracee is in
                      a different PID namespace.

       -e kvm=vcpu
       --kvm=vcpu
              Print the exit reason of kvm vcpu.  Requires Linux kernel
              version 4.16.0 or higher.

       -i
       --instruction-pointer
              Print the instruction pointer at the time of the system
              call.

       -n
       --syscall-number
              Print the syscall number.

       -k
       --stack-trace[=symbol]
              Print the execution stack trace of the traced processes
              after each system call.

       -kk
       --stack-trace=source
              Print the execution stack trace and source code
              information of the traced processes after each system
              call. This option expects the target program is compiled
              with appropriate debug options: "-g" (gcc), or "-g
              -gdwarf-aranges" (clang).

       --stack-trace-frame-limit=limit
              Print no more than this amount of stack trace frames when
              backtracing a system call (the default is 256).  Use this
              option with the --stack-trace (or -k) option.

       -o filename
       --output=filename
              Write the trace output to the file filename rather than to
              stderr.  filename.pid form is used if -ff option is
              supplied.  If the argument begins with '|' or '!', the
              rest of the argument is treated as a command and all
              output is piped to it.  This is convenient for piping the
              debugging output to a program without affecting the
              redirections of executed programs.  The latter is not
              compatible with -ff option currently.

       -A
       --output-append-mode
              Open the file provided in the -o option in append mode.

       -q
       --quiet
       --quiet=attach,personality
              Suppress messages about attaching, detaching, and
              personality changes.  This happens automatically when
              output is redirected to a file and the command is run
              directly instead of attaching.

       -qq
       --quiet=attach,personality,exit
              Suppress messages attaching, detaching, personality
              changes, and about process exit status.

       -qqq
       --quiet=all
              Suppress all suppressible messages (please refer to the -e
              quiet option description for the full list of suppressible
              messages).

       -r
       --relative-timestamps[=precision]
              Print a relative timestamp upon entry to each system call.
              This records the time difference between the beginning of
              successive system calls.  precision can be one of s (for
              seconds), ms (milliseconds), us (microseconds), or ns
              (nanoseconds), and allows setting the precision of time
              value being printed.  Default is us (microseconds).  Note
              that since -r option uses the monotonic clock time for
              measuring time difference and not the wall clock time, its
              measurements can differ from the difference in time
              reported by the -t option.

       -s strsize
       --string-limit=strsize
              Specify the maximum string size to print (the default is
              32).  Note that filenames are not considered strings and
              are always printed in full.

       --absolute-timestamps[=[[format:]format],[[precision:]precision]]
       --timestamps[=[[format:]format],[[precision:]precision]]
              Prefix each line of the trace with the wall clock time in
              the specified format with the specified precision.  format
              can be one of the following:

              none   No time stamp is printed.  Can be used to override
                     the previous setting.
              time   Wall clock time (strftime(3) format string is %T).
              unix   Number of seconds since the epoch (strftime(3)
                     format string is %s).

              precision can be one of s (for seconds), ms
              (milliseconds), us (microseconds), or ns (nanoseconds).
              Default arguments for the option are
              format:time,precision:s.

       -t
       --absolute-timestamps
              Prefix each line of the trace with the wall clock time.

       -tt
       --absolute-timestamps=precision:us
              If given twice, the time printed will include the
              microseconds.

       -ttt
       --absolute-timestamps=format:unix,precision:us
              If given thrice, the time printed will include the
              microseconds and the leading portion will be printed as
              the number of seconds since the epoch.

       -T
       --syscall-times[=precision]
              Show the time spent in system calls.  This records the
              time difference between the beginning and the end of each
              system call.  precision can be one of s (for seconds), ms
              (milliseconds), us (microseconds), or ns (nanoseconds),
              and allows setting the precision of time value being
              printed.  Default is us (microseconds).

       -v
       --no-abbrev
              Print unabbreviated versions of environment, stat,
              termios, etc.  calls.  These structures are very common in
              calls and so the default behavior displays a reasonable
              subset of structure members.  Use this option to get all
              of the gory details.

       --strings-in-hex[=option]
              Control usage of escape sequences with hexadecimal numbers
              in the printed strings.  Normally (when no
              --strings-in-hex or -x option is supplied), escape
              sequences are used to print non-printable and non-ASCII
              characters (that is, characters with a character code less
              than 32 or greater than 127), or to disambiguate the
              output (so, for quotes and other characters that encase
              the printed string, for example, angle brackets, in case
              of file descriptor path output); for the former use case,
              unless it is a white space character that has a symbolic
              escape sequence defined in the C standard (that is, "\textbackslash t"
              for a horizontal tab, "\textbackslash n" for a newline, "\textbackslash v" for a
              vertical tab, "\textbackslash f" for a form feed page break, and "\textbackslash r"
              for a carriage return) are printed using escape sequences
              with numbers that correspond to their byte values, with
              octal number format being the default.  option can be one
              of the following:

              none   Hexadecimal numbers are not used in the output at
                     all.  When there is a need to emit an escape
                     sequence, octal numbers are used.
              non-ascii-chars
                     Hexadecimal numbers are used instead of octal in
                     the escape sequences.
              non-ascii
                     Strings that contain non-ASCII characters are
                     printed using escape sequences with hexadecimal
                     numbers.
              all    All strings are printed using escape sequences with
                     hexadecimal numbers.

              When the option is supplied without an argument, all is
              assumed.

       -x
       --strings-in-hex=non-ascii
              Print all non-ASCII strings in hexadecimal string format.

       -xx
       --strings-in-hex[=all]
              Print all strings in hexadecimal string format.

       -X format
       --const-print-style=format
              Set the format for printing of named constants and flags.
              Supported format values are:

              raw    Raw number output, without decoding.
              abbrev Output a named constant or a set of flags instead
                     of the raw number if they are found.  This is the
                     default strace behaviour.
              verbose
                     Output both the raw value and the decoded string
                     (as a comment).

       -y
       --decode-fds
       --decode-fds=path
              Print paths associated with file descriptor arguments and
              with the AT_FDCWD constant.

       -yy
       --decode-fds=all
              Print all available information associated with file
              descriptors: protocol-specific information associated with
              socket file descriptors, block/character device number
              associated with device file descriptors, and PIDs
              associated with pidfd file descriptors.

       --pidns-translation
       --decode-pids=pidns
              If strace and tracee are in different PID namespaces,
              print PIDs in strace's namespace, too.

       -Y
       --decode-pids=comm
              Print command names for PIDs.

       --secontext[=format]
       -e secontext=format
              When SELinux is available and is not disabled, print in
              square brackets SELinux contexts of processes, files, and
              descriptors.  The format argument is a comma-separated
              list of items being one of the following:

              full              Print the full context (user, role, type
                                level and category).
              mismatch          Also print the context recorded by the
                                SELinux database in case the current
                                context differs.  The latter is printed
                                after two exclamation marks (!!).

              The default value for --secontext is !full,mismatch which
              prints only the type instead of full context and doesn't
              check for context mismatches.

       --always-show-pid
              Show PID prefix also for the process started by strace.
              Implied when -f and -o are both specified.

   Statistics
       -c
       --summary-only
              Count time, calls, and errors for each system call and
              report a summary on program exit, suppressing the regular
              output.  This attempts to show system time (CPU time spent
              running in the kernel) independent of wall clock time.  If
              -c is used with -f, only aggregate totals for all traced
              processes are kept.

       -C
       --summary
              Like -c but also print regular output while processes are
              running.

       -O overhead
       --summary-syscall-overhead=overhead
              Set the overhead for tracing system calls to overhead.
              This is useful for overriding the default heuristic for
              guessing how much time is spent in mere measuring when
              timing system calls using the -c option.  The accuracy of
              the heuristic can be gauged by timing a given program run
              without tracing (using time(1)) and comparing the
              accumulated system call time to the total produced using
              -c.

              The format of overhead specification is described in
              section Time specification format description.

       -S sortby
       --summary-sort-by=sortby
              Sort the output of the histogram printed by the -c option
              by the specified criterion.  Legal values are time (or
              time-percent or time-total or total-time), min-time (or
              shortest or time-min), max-time (or longest or time-max),
              avg-time (or time-avg), calls (or count), errors (or
              error), name (or syscall or syscall-name), and nothing (or
              none); default is time.

       -U columns
       --summary-columns=columns
              Configure a set (and order) of columns being shown in the
              call summary.  The columns argument is a comma-separated
              list with items being one of the following:

              time-percent (or time)
                     Percentage of cumulative time consumed by a
                     specific system call.
              total-time (or time-total)
                     Total system (or wall clock, if -w option is
                     provided) time consumed by a specific system call.
              min-time (or shortest or time-min)
                     Minimum observed call duration.
              max-time (or longest or time-max)
                     Maximum observed call duration.
              avg-time (or time-avg)
                     Average call duration.
              calls (or count)
                     Call count.
              errors (or error)
                     Error count.
              name (or syscall or syscall-name)
                     Syscall name.

              The default value is
              time-percent,total-time,avg-time,calls,errors,name.  If
              the name field is not supplied explicitly, it is added as
              the last column.

       -w
       --summary-wall-clock
              Summarise the time difference between the beginning and
              end of each system call.  The default is to summarise the
              system time.

   Tampering
       -e inject=syscall_set[:error=errno|:retval=value][:signal=sig]
       [:syscall=syscall][:delay_enter=delay][:delay_exit=delay]
       [:poke_enter=@argN=DATAN,@argM=DATAM...]
       [:poke_exit=@argN=DATAN,@argM=DATAM...][:when=expr]
       --inject=syscall_set[:error=errno|:retval=value][:signal=sig]
       [:syscall=syscall][:delay_enter=delay][:delay_exit=delay]
       [:poke_enter=@argN=DATAN,@argM=DATAM...]
       [:poke_exit=@argN=DATAN,@argM=DATAM...][:when=expr]
              Perform   syscall  tampering  for  the  specified  set  of
              syscalls.  The syntax of the syscall_set specification  is
              the same as in the -e trace option.

              At  least  one  of  error,  retval,  signal,  delay_enter,
              delay_exit, poke_enter, or poke_exit  options  has  to  be
              specified.  error and retval are mutually exclusive.

              If  :error=errno  option is specified, a fault is injected
              into a syscall invocation: the syscall number is  replaced
              by  -1  which  corresponds to an invalid syscall (unless a
              syscall is specified with :syscall= option), and the error
              code is specified using a symbolic errno value like ENOSYS
              or a numeric value within 1..4095 range.

              If :retval=value option is specified, success injection is
              performed: the syscall number is replaced  by  -1,  but  a
              bogus success value is returned to the callee.

              If  :signal=sig option is specified with either a symbolic
              value like SIGSEGV or a numeric value  within  1..SIGRTMAX
              range,  that signal is delivered on entering every syscall
              specified by the set.

              If :delay_enter=delay  or  :delay_exit=delay  options  are
              specified,  delay  injection  is  performed: the tracee is
              delayed by time period specified by delay on  entering  or
              exiting  the  syscall,  respectively.  The format of delay
              specification is described in section  Time  specification
              format description.

              If        :poke_enter=@argN=DATAN,@argM=DATAM...        or
              :poke_exit=@argN=DATAN,@argM=DATAM...     options      are
              specified,  tracee's  memory  at  locations, pointed to by
              system call arguments argN and argM (going  from  arg1  to
              arg7) is overwritten by data DATAN and DATAM (specified in
              hexadecimal          format;          for          example
              :poke_enter=@arg1=0000DEAD0000BEEF).  :poke_enter modifies
              memory on syscall enter, and :poke_exit - on exit.

              If :signal=sig option is specified  without  :error=errno,
              :retval=value  or  :delay_{enter,exit}=usecs options, then
              only a signal sig is delivered without a syscall fault  or
              delay     injection.     Conversely,    :error=errno    or
              :retval=value    option    without     :delay_enter=delay,
              :delay_exit=delay  or  :signal=sig options injects a fault
              without delivering a signal or injecting a delay, etc.

              If  :signal=sig  option   is   specified   together   with
              :error=errno  or  :retval=value,  then both injection of a
              fault or success and signal delivery are performed.

              if :syscall=syscall option is specified, the corresponding
              syscall with no side effects is injected  instead  of  -1.
              Currently,  only  "pure"  (see -e trace=%pure description)
              syscalls can be specified there.

              Unless  a  :when=expr  subexpression  is   specified,   an
              injection  is  being  made  into  every invocation of each
              syscall from the set.

              The format of the subexpression is:

                             first[..last][+[step]]

              Number first stands for the first invocation number in the
              range, number last stands for the last  invocation  number
              in  the  range,  and  step stands for the step between two
              consecutive invocations.  The following  combinations  are
              useful:

              first  For every syscall from the set, perform an
                     injection for the syscall invocation number first
                     only.
              first..last
                     For every syscall from the set, perform an
                     injection for the syscall invocation number first
                     and all subsequent invocations until the invocation
                     number last (inclusive).
              first+ For every syscall from the set, perform injections
                     for the syscall invocation number first and all
                     subsequent invocations.
              first..last+
                     For every syscall from the set, perform injections
                     for the syscall invocation number first and all
                     subsequent invocations until the invocation number
                     last (inclusive).
              first+step
                     For every syscall from the set, perform injections
                     for syscall invocations number first, first+step,
                     first+step+step, and so on.
              first..last+step
                     Same as the previous, but consider only syscall
                     invocations with numbers up to last (inclusive).

              For example, to fail each third and subsequent chdir
              syscalls with ENOENT, use
              -e inject=chdir:error=ENOENT:when=3+.

              The valid range for numbers first and step is 1..65535,
              and for number last is 1..65534.

              An injection expression can contain only one error= or
              retval= specification, and only one signal= specification.
              If an injection expression contains multiple when=
              specifications, the last one takes precedence.

              Accounting of syscalls that are subject to injection is
              done per syscall and per tracee.

              Specification of syscall injection can be combined with
              other syscall filtering options, for example, -P
              /dev/urandom -e inject=file:error=ENOENT.

       -e fault=syscall_set[:error=errno][:when=expr]
       --fault=syscall_set[:error=errno][:when=expr]
              Perform syscall fault injection for the specified set of
              syscalls.

              This is equivalent to more generic -e inject= expression
              with default value of errno option set to ENOSYS.

   Miscellaneous
       -d
       --debug
              Show some debugging output of strace itself on the
              standard error.

       -F     This option is deprecated.  It is retained for backward
              compatibility only and may be removed in future releases.
              Usage of multiple instances of -F option is still
              equivalent to a single -f, and it is ignored at all if
              used along with one or more instances of -f option.

       -h
       --help Print the help summary.

       --seccomp-bpf
              Try to enable use of seccomp-bpf (see seccomp(2)) to have
              ptrace(2)-stops only when system calls that are being
              traced occur in the traced processes.

              This option has no effect unless -f/--follow-forks is also
              specified.  --seccomp-bpf is not compatible with
              --syscall-limit and -b/--detach-on options.  It is also
              not applicable to processes attached using -p/--attach
              option.

              An attempt to enable system calls filtering using seccomp-
              bpf may fail for various reasons, e.g. there are too many
              system calls to filter, the seccomp API is not available,
              or strace itself is being traced.  In cases when seccomp-
              bpf filter setup failed, strace proceeds as usual and
              stops traced processes on every system call.

              When --seccomp-bpf is activated and -p/--attach option is
              not used, --kill-on-exit option is activated as well.

              Note that in cases when the tracee has another seccomp
              filter that returns an action value with a precedence
              greater than SECCOMP_RET_TRACE, strace --seccomp-bpf will
              not be notified.  That is, if another seccomp filter, for
              example, disables the syscall or kills the tracee, then
              strace --seccomp-bpf will not be aware of that syscall
              invocation at all.

       --tips[=[[id:]id],[[format:]format]]
              Show strace tips, tricks, and tweaks before exit.  id can
              be a non-negative integer number, which enables printing
              of specific tip, trick, or tweak (these ID are not
              guaranteed to be stable), or random (the default), in
              which case a random tip is printed.  format can be one of
              the following:

              none     No tip is printed.  Can be used to override the
                       previous setting.
              compact  Print the tip just big enough to contain all the
                       text.
              full     Print the tip in its full glory.

              Default is id:random,format:compact.

       -V
       --version
              Print the version number of strace.  Multiple instances of
              the option beyond specific threshold tend to increase
              Strauss awareness.

   Time specification format description
       Time values can be specified as a decimal floating point number
       (in a format accepted by strtod(3)), optionally followed by one
       of the following suffices that specify the unit of time: s
       (seconds), ms (milliseconds), us (microseconds), or ns
       (nanoseconds).  If no suffix is specified, the value is
       interpreted as microseconds.

       The described format is used for -O, -e inject=delay_enter, and
       -e inject=delay_exit options.
DIAGNOSTICS
       When command exits, strace exits with the same exit status.  If
       command is terminated by a signal, strace terminates itself with
       the same signal, so that strace can be used as a wrapper process
       transparent to the invoking parent process.  Note that parent-
       child relationship (signal stop notifications, getppid(2) value,
       etc) between traced process and its parent are not preserved
       unless -D is used.

       When using -p without a command, the exit status of strace is
       zero unless no processes has been attached or there was an
       unexpected error in doing the tracing.
SETUID INSTALLATION
       If strace is installed setuid to root then the invoking user will
       be able to attach to and trace processes owned by any user.  In
       addition setuid and setgid programs will be executed and traced
       with the correct effective privileges.  Since only users trusted
       with full root privileges should be allowed to do these things,
       it only makes sense to install strace as setuid to root when the
       users who can execute it are restricted to those users who have
       this trust.  For example, it makes sense to install a special
       version of strace with mode 'rwsr-xr--', user root and group
       trace, where members of the trace group are trusted users.  If
       you do use this feature, please remember to install a regular
       non-setuid version of strace for ordinary users to use.
MULTIPLE PERSONALITIES SUPPORT
       On some architectures, strace supports decoding of syscalls for
       processes that use different ABI rather than the one strace uses.
       Specifically, in addition to decoding native ABI, strace can
       decode the following ABIs on the following architectures:

       [1]  When strace is built as an x86_64 application
       [2]  When strace is built as an x32 application
       [3]  Big endian only

       This support is optional and relies on ability to generate and
       parse structure definitions during the build time.  Please refer
       to the output of the strace -V command in order to figure out
       what support is available in your strace build ("non-native"
       refers to an ABI that differs from the ABI strace has):

       m32-mpers
              strace can trace and properly decode non-native 32-bit
              binaries.
       no-m32-mpers
              strace can trace, but cannot properly decode non-native
              32-bit binaries.
       mx32-mpers
              strace can trace and properly decode non-native
              32-on-64-bit binaries.
       no-mx32-mpers
              strace can trace, but cannot properly decode non-native
              32-on-64-bit binaries.

       If the output contains neither m32-mpers nor no-m32-mpers, then
       decoding of non-native 32-bit binaries is not implemented at all
       or not applicable.

       Likewise, if the output contains neither mx32-mpers nor no-
       mx32-mpers, then decoding of non-native 32-on-64-bit binaries is
       not implemented at all or not applicable.
NOTES
       It is a pity that so much tracing clutter is produced by systems
       employing shared libraries.

       It is instructive to think about system call inputs and outputs
       as data-flow across the user/kernel boundary.  Because user-space
       and kernel-space are separate and address-protected, it is
       sometimes possible to make deductive inferences about process
       behavior using inputs and outputs as propositions.

       In some cases, a system call will differ from the documented
       behavior or have a different name.  For example, the faccessat(2)
       system call does not have flags argument, and the setrlimit(2)
       library function uses prlimit64(2) system call on modern
       (2.6.38+) kernels.  These discrepancies are normal but
       idiosyncratic characteristics of the system call interface and
       are accounted for by C library wrapper functions.

       Some system calls have different names in different architectures
       and personalities.  In these cases, system call filtering and
       printing uses the names that match corresponding __NR_* kernel
       macros of the tracee's architecture and personality.  There are
       two exceptions from this general rule: arm_fadvise64_64(2) ARM
       syscall and xtensa_fadvise64_64(2) Xtensa syscall are filtered
       and printed as fadvise64_64(2).

       On x32, syscalls that are intended to be used by 64-bit processes
       and not x32 ones (for example, readv(2), that has syscall number
       19 on x86_64, with its x32 counterpart has syscall number 515),
       but called with __X32_SYSCALL_BIT flag being set, are designated
       with #64 suffix.

       On some platforms a process that is attached to with the -p
       option may observe a spurious EINTR return from the current
       system call that is not restartable.  (Ideally, all system calls
       should be restarted on strace attach, making the attach invisible
       to the traced process, but a few system calls aren't.  Arguably,
       every instance of such behavior is a kernel bug.)  This may have
       an unpredictable effect on the process if the process takes no
       action to restart the system call.

       As strace executes the specified command directly and does not
       employ a shell for that, scripts without shebang that usually run
       just fine when invoked by shell fail to execute with ENOEXEC
       error.  It is advisable to manually supply a shell as a command
       with the script as its argument.
BUGS
       Programs that use the setuid bit do not have effective user ID
       privileges while being traced.

       A traced process runs slowly (but check out the --seccomp-bpf
       option).

       Unless --kill-on-exit option is used (or --seccomp-bpf option is
       used in a way that implies --kill-on-exit), traced processes
       which are descended from command may be left running after an
       interrupt signal (CTRL-C).

       By using CLONE_UNTRACED flag of clone system call a tracee can
       break the guarantee that --seccomp-bpf will not leave any
       processes with a seccomp program installed for syscall filtering
       purposes.
HISTORY
       The original strace was written by Paul Kranenburg for SunOS and
       was inspired by its trace utility.  The SunOS version of strace
       was ported to Linux and enhanced by Branko Lankester, who also
       wrote the Linux kernel support.  Even though Paul released strace
       2.5 in 1992, Branko's work was based on Paul's strace 1.5 release
       from 1991.  In 1993, Rick Sladkey merged strace 2.5 for SunOS and
       the second release of strace for Linux, added many of the
       features of truss(1) from SVR4, and produced an strace that
       worked on both platforms.  In 1994 Rick ported strace to SVR4 and
       Solaris and wrote the automatic configuration support.  In 1995
       he ported strace to Irix and became tired of writing about
       himself in the third person.

       Beginning with 1996, strace was maintained by Wichert Akkerman.
       During his tenure, strace development migrated to CVS; ports to
       FreeBSD and many architectures on Linux (including ARM, IA-64,
       MIPS, PA-RISC, PowerPC, s390, SPARC) were introduced.  In 2002,
       the burden of strace maintainership was transferred to Roland
       McGrath.  Since then, strace gained support for several new Linux
       architectures (AMD64, s390x, SuperH), bi-architecture support for
       some of them, and received numerous additions and improvements in
       syscalls decoders on Linux; strace development migrated to Git
       during that period.  Since 2009, strace is actively maintained by
       Dmitry Levin.  strace gained support for AArch64, ARC, AVR32,
       Blackfin, Meta, Nios II, OpenRISC 1000, RISC-V, Tile/TileGx,
       Xtensa architectures since that time.  In 2012, unmaintained and
       apparently broken support for non-Linux operating systems was
       removed.  Also, in 2012 strace gained support for path tracing
       and file descriptor path decoding.  In 2014, support for stack
       trace printing was added.  In 2016, syscall fault injection was
       implemented.

       For the additional information, please refer to the NEWS file and
       strace repository commit log.
REPORTING BUGS
       Problems with strace should be reported to the strace mailing
       list mailto:strace-devel@lists.strace.io.
SEE ALSO
       strace-log-merge(1), ltrace(1), perf-trace(1), trace-cmd(1),
       time(1), ptrace(2), seccomp(2), syscall(2), proc(5), signal(7)

       strace Home Page https://strace.io/
AUTHORS
       The complete list of strace contributors can be found in the
       CREDITS file.
COLOPHON
       This page is part of the strace (system call tracer) project.
       Information about the project can be found at 
       http://strace.io/.  If you have a bug report for this manual
       page, send it to strace-devel@lists.sourceforge.net.  This page
       was obtained from the project's upstream Git repository
       https://github.com/strace/strace.git on 2024-06-14.  (At that
       time, the date of the most recent commit that was found in the
       repository was 2024-06-04.)  If you discover any rendering
       problems in this HTML version of the page, or you believe there
       is a better or more up-to-date source for the page, or you have
       corrections or improvements to the information in this COLOPHON
       (which is not part of the original manual page), send a mail to
       man-pages@man7.org

strace 6.9.0.16.2a4c4          2024-06-04                      STRACE(1)
\end{lstlisting}
}}

\endinput  %  ==  ==  ==  ==  ==  ==  ==  ==  ==
	% % % \input{./components/man/man-strings}
\subsection{\refStrings: Print Sequences Of Printable Characters}

{\tiny{
\begin{lstlisting}[language=bash]
NAME
       strings - print the sequences of printable characters in files
SYNOPSIS
       strings [-afovV] [-min-len]
               [-n min-len] [--bytes=min-len]
               [-t radix] [--radix=radix]
               [-e encoding] [--encoding=encoding]
               [-U method] [--unicode=method]
               [-] [--all] [--print-file-name]
               [-T bfdname] [--target=bfdname]
               [-w] [--include-all-whitespace]
               [-s] [--output-separator sep_string]
               [--help] [--version] file...
DESCRIPTION
       For each file given, GNU strings prints the printable character
       sequences that are at least 4 characters long (or the number
       given with the options below) and are followed by an unprintable
       character.

       Depending upon how the strings program was configured it will
       default to either displaying all the printable sequences that it
       can find in each file, or only those sequences that are in
       loadable, initialized data sections.  If the file type is
       unrecognizable, or if strings is reading from stdin then it will
       always display all of the printable sequences that it can find.

       For backwards compatibility any file that occurs after a command-
       line option of just - will also be scanned in full, regardless of
       the presence of any -d option.

       strings is mainly useful for determining the contents of non-text
       files.
OPTIONS
       -a
       --all
       -   Scan the whole file, regardless of what sections it contains
           or whether those sections are loaded or initialized.
           Normally this is the default behaviour, but strings can be
           configured so that the -d is the default instead.

           The - option is position dependent and forces strings to
           perform full scans of any file that is mentioned after the -
           on the command line, even if the -d option has been
           specified.

       -d
       --data
           Only print strings from initialized, loaded data sections in
           the file.  This may reduce the amount of garbage in the
           output, but it also exposes the strings program to any
           security flaws that may be present in the BFD library used to
           scan and load sections.  Strings can be configured so that
           this option is the default behaviour.  In such cases the -a
           option can be used to avoid using the BFD library and instead
           just print all of the strings found in the file.

       -f
       --print-file-name
           Print the name of the file before each string.

       --help
           Print a summary of the program usage on the standard output
           and exit.

       -min-len
       -n min-len
       --bytes=min-len
           Print sequences of displayable characters that are at least
           min-len characters long.  If not specified a default minimum
           length of 4 is used.  The distinction between displayable and
           non-displayable characters depends upon the setting of the -e
           and -U options.  Sequences are always terminated at control
           characters such as new-line and carriage-return, but not the
           tab character.

       -o  Like -t o.  Some other versions of strings have -o act like
           -t d instead.  Since we can not be compatible with both ways,
           we simply chose one.

       -t radix
       --radix=radix
           Print the offset within the file before each string.  The
           single character argument specifies the radix of the
           offset---o for octal, x for hexadecimal, or d for decimal.

       -e encoding
       --encoding=encoding
           Select the character encoding of the strings that are to be
           found.  Possible values for encoding are: s =
           single-7-bit-byte characters (default), S = single-8-bit-byte
           characters, b = 16-bit bigendian, l = 16-bit littleendian, B
           = 32-bit bigendian, L = 32-bit littleendian.  Useful for
           finding wide character strings. (l and b apply to, for
           example, Unicode UTF-16/UCS-2 encodings).

       -U [d|i|l|e|x|h]
       --unicode=[default|invalid|locale|escape|hex|highlight]
           Controls the display of UTF-8 encoded multibyte characters in
           strings.  The default (--unicode=default) is to give them no
           special treatment, and instead rely upon the setting of the
           --encoding option.  The other values for this option
           automatically enable --encoding=S.

           The --unicode=invalid option treats them as non-graphic
           characters and hence not part of a valid string.  All the
           remaining options treat them as valid string characters.

           The --unicode=locale option displays them in the current
           locale, which may or may not support UTF-8 encoding.  The
           --unicode=hex option displays them as hex byte sequences
           enclosed between <> characters.  The --unicode=escape option
           displays them as escape sequences (\uxxxx) and the
           --unicode=highlight option displays them as escape sequences
           highlighted in red (if supported by the output device).  The
           colouring is intended to draw attention to the presence of
           unicode sequences where they might not be expected.

       -T bfdname
       --target=bfdname
           Specify an object code format other than your system's
           default format.

       -v
       -V
       --version
           Print the program version number on the standard output and
           exit.

       -w
       --include-all-whitespace
           By default tab and space characters are included in the
           strings that are displayed, but other whitespace characters,
           such a newlines and carriage returns, are not.  The -w option
           changes this so that all whitespace characters are considered
           to be part of a string.

       -s
       --output-separator
           By default, output strings are delimited by a new-line. This
           option allows you to supply any string to be used as the
           output record separator.  Useful with
           --include-all-whitespace where strings may contain new-lines
           internally.

       @file
           Read command-line options from file.  The options read are
           inserted in place of the original @file option.  If file does
           not exist, or cannot be read, then the option will be treated
           literally, and not removed.

           Options in file are separated by whitespace.  A whitespace
           character may be included in an option by surrounding the
           entire option in either single or double quotes.  Any
           character (including a backslash) may be included by
           prefixing the character to be included with a backslash.  The
           file may itself contain additional @file options; any such
           options will be processed recursively.
SEE ALSO
       ar(1), nm(1), objdump(1), ranlib(1), readelf(1) and the Info
       entries for binutils.
COPYRIGHT
       Copyright (c) 1991-2024 Free Software Foundation, Inc.

       Permission is granted to copy, distribute and/or modify this
       document under the terms of the GNU Free Documentation License,
       Version 1.3 or any later version published by the Free Software
       Foundation; with no Invariant Sections, with no Front-Cover
       Texts, and with no Back-Cover Texts.  A copy of the license is
       included in the section entitled "GNU Free Documentation
       License".
COLOPHON
       This page is part of the binutils (a collection of tools for
       working with executable binaries) project.  Information about the
       project can be found at http://www.gnu.org/software/binutils/.
       If you have a bug report for this manual page, see
       http://sourceware.org/bugzilla/enter_bug.cgi?product=binutils.
       This page was obtained from the tarball binutils-2.42.tar.gz
       fetched from https://ftp.gnu.org/gnu/binutils/ on 2024-06-14.
       If you discover any rendering problems in this HTML version of
       the page, or you believe there is a better or more up-to-date
       source for the page, or you have corrections or improvements to
       the information in this COLOPHON (which is not part of the
       original manual page), send a mail to man-pages@man7.org

binutils-2.42                  2024-06-14                     STRINGS(1)
\end{lstlisting}
}}
\endinput  %  ==  ==  ==  ==  ==  ==  ==  ==  ==

\subsection{\refStrings: Print Sequences Of Printable Characters}

{\tiny{
\begin{lstlisting}[language=bash]
NAME
       strings - print the sequences of printable characters in files
SYNOPSIS
       strings [-afovV] [-min-len]
               [-n min-len] [--bytes=min-len]
               [-t radix] [--radix=radix]
               [-e encoding] [--encoding=encoding]
               [-U method] [--unicode=method]
               [-] [--all] [--print-file-name]
               [-T bfdname] [--target=bfdname]
               [-w] [--include-all-whitespace]
               [-s] [--output-separator sep_string]
               [--help] [--version] file...
DESCRIPTION
       For each file given, GNU strings prints the printable character
       sequences that are at least 4 characters long (or the number
       given with the options below) and are followed by an unprintable
       character.

       Depending upon how the strings program was configured it will
       default to either displaying all the printable sequences that it
       can find in each file, or only those sequences that are in
       loadable, initialized data sections.  If the file type is
       unrecognizable, or if strings is reading from stdin then it will
       always display all of the printable sequences that it can find.

       For backwards compatibility any file that occurs after a command-
       line option of just - will also be scanned in full, regardless of
       the presence of any -d option.

       strings is mainly useful for determining the contents of non-text
       files.
OPTIONS
       -a
       --all
       -   Scan the whole file, regardless of what sections it contains
           or whether those sections are loaded or initialized.
           Normally this is the default behaviour, but strings can be
           configured so that the -d is the default instead.

           The - option is position dependent and forces strings to
           perform full scans of any file that is mentioned after the -
           on the command line, even if the -d option has been
           specified.

       -d
       --data
           Only print strings from initialized, loaded data sections in
           the file.  This may reduce the amount of garbage in the
           output, but it also exposes the strings program to any
           security flaws that may be present in the BFD library used to
           scan and load sections.  Strings can be configured so that
           this option is the default behaviour.  In such cases the -a
           option can be used to avoid using the BFD library and instead
           just print all of the strings found in the file.

       -f
       --print-file-name
           Print the name of the file before each string.

       --help
           Print a summary of the program usage on the standard output
           and exit.

       -min-len
       -n min-len
       --bytes=min-len
           Print sequences of displayable characters that are at least
           min-len characters long.  If not specified a default minimum
           length of 4 is used.  The distinction between displayable and
           non-displayable characters depends upon the setting of the -e
           and -U options.  Sequences are always terminated at control
           characters such as new-line and carriage-return, but not the
           tab character.

       -o  Like -t o.  Some other versions of strings have -o act like
           -t d instead.  Since we can not be compatible with both ways,
           we simply chose one.

       -t radix
       --radix=radix
           Print the offset within the file before each string.  The
           single character argument specifies the radix of the
           offset---o for octal, x for hexadecimal, or d for decimal.

       -e encoding
       --encoding=encoding
           Select the character encoding of the strings that are to be
           found.  Possible values for encoding are: s =
           single-7-bit-byte characters (default), S = single-8-bit-byte
           characters, b = 16-bit bigendian, l = 16-bit littleendian, B
           = 32-bit bigendian, L = 32-bit littleendian.  Useful for
           finding wide character strings. (l and b apply to, for
           example, Unicode UTF-16/UCS-2 encodings).

       -U [d|i|l|e|x|h]
       --unicode=[default|invalid|locale|escape|hex|highlight]
           Controls the display of UTF-8 encoded multibyte characters in
           strings.  The default (--unicode=default) is to give them no
           special treatment, and instead rely upon the setting of the
           --encoding option.  The other values for this option
           automatically enable --encoding=S.

           The --unicode=invalid option treats them as non-graphic
           characters and hence not part of a valid string.  All the
           remaining options treat them as valid string characters.

           The --unicode=locale option displays them in the current
           locale, which may or may not support UTF-8 encoding.  The
           --unicode=hex option displays them as hex byte sequences
           enclosed between <> characters.  The --unicode=escape option
           displays them as escape sequences (\uxxxx) and the
           --unicode=highlight option displays them as escape sequences
           highlighted in red (if supported by the output device).  The
           colouring is intended to draw attention to the presence of
           unicode sequences where they might not be expected.

       -T bfdname
       --target=bfdname
           Specify an object code format other than your system's
           default format.

       -v
       -V
       --version
           Print the program version number on the standard output and
           exit.

       -w
       --include-all-whitespace
           By default tab and space characters are included in the
           strings that are displayed, but other whitespace characters,
           such a newlines and carriage returns, are not.  The -w option
           changes this so that all whitespace characters are considered
           to be part of a string.

       -s
       --output-separator
           By default, output strings are delimited by a new-line. This
           option allows you to supply any string to be used as the
           output record separator.  Useful with
           --include-all-whitespace where strings may contain new-lines
           internally.

       @file
           Read command-line options from file.  The options read are
           inserted in place of the original @file option.  If file does
           not exist, or cannot be read, then the option will be treated
           literally, and not removed.

           Options in file are separated by whitespace.  A whitespace
           character may be included in an option by surrounding the
           entire option in either single or double quotes.  Any
           character (including a backslash) may be included by
           prefixing the character to be included with a backslash.  The
           file may itself contain additional @file options; any such
           options will be processed recursively.
SEE ALSO
       ar(1), nm(1), objdump(1), ranlib(1), readelf(1) and the Info
       entries for binutils.
COPYRIGHT
       Copyright (c) 1991-2024 Free Software Foundation, Inc.

       Permission is granted to copy, distribute and/or modify this
       document under the terms of the GNU Free Documentation License,
       Version 1.3 or any later version published by the Free Software
       Foundation; with no Invariant Sections, with no Front-Cover
       Texts, and with no Back-Cover Texts.  A copy of the license is
       included in the section entitled "GNU Free Documentation
       License".
COLOPHON
       This page is part of the binutils (a collection of tools for
       working with executable binaries) project.  Information about the
       project can be found at http://www.gnu.org/software/binutils/.
       If you have a bug report for this manual page, see
       http://sourceware.org/bugzilla/enter_bug.cgi?product=binutils.
       This page was obtained from the tarball binutils-2.42.tar.gz
       fetched from https://ftp.gnu.org/gnu/binutils/ on 2024-06-14.
       If you discover any rendering problems in this HTML version of
       the page, or you believe there is a better or more up-to-date
       source for the page, or you have corrections or improvements to
       the information in this COLOPHON (which is not part of the
       original manual page), send a mail to man-pages@man7.org

binutils-2.42                  2024-06-14                     STRINGS(1)
\end{lstlisting}
}}
\endinput  %  ==  ==  ==  ==  ==  ==  ==  ==  ==

\subsection{\refStrings: Print Sequences Of Printable Characters}

{\tiny{
\begin{lstlisting}[language=bash]
NAME
       strings - print the sequences of printable characters in files
SYNOPSIS
       strings [-afovV] [-min-len]
               [-n min-len] [--bytes=min-len]
               [-t radix] [--radix=radix]
               [-e encoding] [--encoding=encoding]
               [-U method] [--unicode=method]
               [-] [--all] [--print-file-name]
               [-T bfdname] [--target=bfdname]
               [-w] [--include-all-whitespace]
               [-s] [--output-separator sep_string]
               [--help] [--version] file...
DESCRIPTION
       For each file given, GNU strings prints the printable character
       sequences that are at least 4 characters long (or the number
       given with the options below) and are followed by an unprintable
       character.

       Depending upon how the strings program was configured it will
       default to either displaying all the printable sequences that it
       can find in each file, or only those sequences that are in
       loadable, initialized data sections.  If the file type is
       unrecognizable, or if strings is reading from stdin then it will
       always display all of the printable sequences that it can find.

       For backwards compatibility any file that occurs after a command-
       line option of just - will also be scanned in full, regardless of
       the presence of any -d option.

       strings is mainly useful for determining the contents of non-text
       files.
OPTIONS
       -a
       --all
       -   Scan the whole file, regardless of what sections it contains
           or whether those sections are loaded or initialized.
           Normally this is the default behaviour, but strings can be
           configured so that the -d is the default instead.

           The - option is position dependent and forces strings to
           perform full scans of any file that is mentioned after the -
           on the command line, even if the -d option has been
           specified.

       -d
       --data
           Only print strings from initialized, loaded data sections in
           the file.  This may reduce the amount of garbage in the
           output, but it also exposes the strings program to any
           security flaws that may be present in the BFD library used to
           scan and load sections.  Strings can be configured so that
           this option is the default behaviour.  In such cases the -a
           option can be used to avoid using the BFD library and instead
           just print all of the strings found in the file.

       -f
       --print-file-name
           Print the name of the file before each string.

       --help
           Print a summary of the program usage on the standard output
           and exit.

       -min-len
       -n min-len
       --bytes=min-len
           Print sequences of displayable characters that are at least
           min-len characters long.  If not specified a default minimum
           length of 4 is used.  The distinction between displayable and
           non-displayable characters depends upon the setting of the -e
           and -U options.  Sequences are always terminated at control
           characters such as new-line and carriage-return, but not the
           tab character.

       -o  Like -t o.  Some other versions of strings have -o act like
           -t d instead.  Since we can not be compatible with both ways,
           we simply chose one.

       -t radix
       --radix=radix
           Print the offset within the file before each string.  The
           single character argument specifies the radix of the
           offset---o for octal, x for hexadecimal, or d for decimal.

       -e encoding
       --encoding=encoding
           Select the character encoding of the strings that are to be
           found.  Possible values for encoding are: s =
           single-7-bit-byte characters (default), S = single-8-bit-byte
           characters, b = 16-bit bigendian, l = 16-bit littleendian, B
           = 32-bit bigendian, L = 32-bit littleendian.  Useful for
           finding wide character strings. (l and b apply to, for
           example, Unicode UTF-16/UCS-2 encodings).

       -U [d|i|l|e|x|h]
       --unicode=[default|invalid|locale|escape|hex|highlight]
           Controls the display of UTF-8 encoded multibyte characters in
           strings.  The default (--unicode=default) is to give them no
           special treatment, and instead rely upon the setting of the
           --encoding option.  The other values for this option
           automatically enable --encoding=S.

           The --unicode=invalid option treats them as non-graphic
           characters and hence not part of a valid string.  All the
           remaining options treat them as valid string characters.

           The --unicode=locale option displays them in the current
           locale, which may or may not support UTF-8 encoding.  The
           --unicode=hex option displays them as hex byte sequences
           enclosed between <> characters.  The --unicode=escape option
           displays them as escape sequences (\uxxxx) and the
           --unicode=highlight option displays them as escape sequences
           highlighted in red (if supported by the output device).  The
           colouring is intended to draw attention to the presence of
           unicode sequences where they might not be expected.

       -T bfdname
       --target=bfdname
           Specify an object code format other than your system's
           default format.

       -v
       -V
       --version
           Print the program version number on the standard output and
           exit.

       -w
       --include-all-whitespace
           By default tab and space characters are included in the
           strings that are displayed, but other whitespace characters,
           such a newlines and carriage returns, are not.  The -w option
           changes this so that all whitespace characters are considered
           to be part of a string.

       -s
       --output-separator
           By default, output strings are delimited by a new-line. This
           option allows you to supply any string to be used as the
           output record separator.  Useful with
           --include-all-whitespace where strings may contain new-lines
           internally.

       @file
           Read command-line options from file.  The options read are
           inserted in place of the original @file option.  If file does
           not exist, or cannot be read, then the option will be treated
           literally, and not removed.

           Options in file are separated by whitespace.  A whitespace
           character may be included in an option by surrounding the
           entire option in either single or double quotes.  Any
           character (including a backslash) may be included by
           prefixing the character to be included with a backslash.  The
           file may itself contain additional @file options; any such
           options will be processed recursively.
SEE ALSO
       ar(1), nm(1), objdump(1), ranlib(1), readelf(1) and the Info
       entries for binutils.
COPYRIGHT
       Copyright (c) 1991-2024 Free Software Foundation, Inc.

       Permission is granted to copy, distribute and/or modify this
       document under the terms of the GNU Free Documentation License,
       Version 1.3 or any later version published by the Free Software
       Foundation; with no Invariant Sections, with no Front-Cover
       Texts, and with no Back-Cover Texts.  A copy of the license is
       included in the section entitled "GNU Free Documentation
       License".
COLOPHON
       This page is part of the binutils (a collection of tools for
       working with executable binaries) project.  Information about the
       project can be found at http://www.gnu.org/software/binutils/.
       If you have a bug report for this manual page, see
       http://sourceware.org/bugzilla/enter_bug.cgi?product=binutils.
       This page was obtained from the tarball binutils-2.42.tar.gz
       fetched from https://ftp.gnu.org/gnu/binutils/ on 2024-06-14.
       If you discover any rendering problems in this HTML version of
       the page, or you believe there is a better or more up-to-date
       source for the page, or you have corrections or improvements to
       the information in this COLOPHON (which is not part of the
       original manual page), send a mail to man-pages@man7.org

binutils-2.42                  2024-06-14                     STRINGS(1)
\end{lstlisting}
}}
\endinput  %  ==  ==  ==  ==  ==  ==  ==  ==  ==

	
\end{document} 

\tiny
\scriptsize
\footnotesize
\small
\normalsize
\large
\Large
\huge
\Huge