\documentclass[10pt, oneside]{article}   	% use "amsart" instead of "article" for AMSLaTeX format
\usepackage{geometry}                		% See geometry.pdf to learn the layout options. There are lots.
\geometry{letterpaper}                   		% ... or a4paper or a5paper or ... 
%\geometry{landscape}                		% Activate for rotated page geometry
%\usepackage[parfill]{parskip}    		% Activate to begin paragraphs with an empty line rather than an indent
\usepackage{graphicx}				% Use pdf, png, jpg, or eps§ with pdflatex; use eps in DVI mode
								% TeX will automatically convert eps --> pdf in pdflatex		
\usepackage{amssymb}
\usepackage[affil-it]{authblk}
\usepackage{fancyvrb}
\usepackage{hyperref}
\usepackage{overpic}
\usepackage{xcolor}
%
\usepackage{listings}
	\definecolor{textblue}{rgb}{.2,.2,.7}
	\definecolor{textred}{rgb}{0.54,0,0}
	\definecolor{textgreen}{rgb}{0,0.43,0}

% personalize environment
\newcommand{\pGlobal}[0]			{../../../global/}
\newcommand{\pGlobalSetup}[0]		{\pGlobal setup-global/}	

\input{\pGlobalSetup setup-global-reports}
\input{\pLocalSetup macros}

\newcommand{\pManPages}[0]		{\pLocal man-pages/}
\newcommand{\escapepercent}[0]	{\%}

\title{Unix Tools for Probing Executable Files}
\author{Daniel Topa\\\href{mailto:daniel.topa@hii-tsd.com}{daniel.topa@hii-tsd.com}}
\affil{\href{https://hii.com/what-we-do/divisions/mission-technologies/}{Mission Technologies}
\\Huntington Ingalls Industries
\\Kirtland AFB, NM}

\begin{document}
\maketitle
\abstract{This article surveys Unix tools for the exploration of executable files, some of which depend upon the application being compiled with debug information. The \textt{man}ual pages are included, making this document useful in siloed computing networks.}
\tableofcontents

\section{Overview}
Unix provides powerful tools for probing executable files. The following section shows sample usage for each command and the final section contains the information from the \textt{man}ual page. The final element is the GNU debugger and not a formal element of Unix.
\begin{enumerate}
	\item \hyperref[sec:ldd]{\ldd}	
	\item \hyperref[sec:ldd]{\lddconfig}
	\item \hyperref[sec:locate]{\locate}
	\item \hyperref[sec:objdump]{\objdump}
	\item \hyperref[sec:lsof]{\lsof}
	\item \hyperref[sec:readelf]{\readelf}
	\item \hyperref[sec:nm]{\nm}
	\item \hyperref[sec:strace]{\strace}
	\item \hyperref[sec:strings]{\strings}
	\item \hyperref[sec:gdb]{\gdb}
\end{enumerate}

The goal is to be able to resolve the workings of an executable file exploiting the \elf \ structure show in figures \ref{fig:elf}. The next figure, \ref{fig:elf-II}, shows the relationship between source files, header files, shared objects, and the executable program.

\begin{figure}[htbp] %  figure placement: here, top, bottom, or page
\centering
	\href{https://camo.githubusercontent.com/00cd4e64df02caf11e9c7c8f67a4d7e9470ea03c244e6d5bce8444a674b9143c/68747470733a2f2f692e696d6775722e636f6d2f4169394f714f422e706e67}{
	\begin{overpic}[ scale = 0.4 ]
		{../local/graphics-local/elf-01}
		%\put(50, 50)	{\colorbox{white}{$a+b$}}
	\end{overpic}
	}
\caption{The structure of a Unix \elf \ file.}
\label{fig:elf}
\end{figure}

\begin{figure}[htbp] %  figure placement: here, top, bottom, or page
\centering
	\href{https://camo.githubusercontent.com/94b1128b885c29e21c64fb3b247d0184c54f4248e4195462bd15671003afc319/68747470733a2f2f692e696d6775722e636f6d2f4c4e6464546d6b2e706e67}{
	\begin{overpic}[ scale = 0.5 ]
		{../local/graphics-local/elf-02}
		%\put(50, 50)	{\colorbox{white}{$a+b$}}
	\end{overpic}
	}
\caption{Connecting source files, object files, libraries, and bindary executables.}
\label{fig:elf-II}
\end{figure}

\section{Command Examples}
		\input{\pSections ssec-ldd}
		\input{\pSections ssec-lddconfig}
		\input{\pSections ssec-locate}
		\input{\pSections ssec-losf}
		\input{\pSections ssec-objdump}
		\input{\pSections ssec-readelf}
		\input{\pSections ssec-nm}
		\input{\pSections ssec-strace}
		\input{\pSections ssec-strings}
		\input{\pSections ssec-gdb}

\section{\href{\urlMan}{Manual Pages}}
		\input{\pManPages man-ldd}
		\input{\pManPages man-lddconfig}
		\input{\pManPages man-locate}
		\input{\pManPages man-lsof}
		\input{\pManPages man-objdump}
		\input{\pManPages man-readelf}
		\input{\pManPages man-nm}
		\input{\pManPages man-strace}
		\input{\pManPages man-strings}
	
\end{document} 

\tiny
\scriptsize
\footnotesize
\small
\normalsize
\large
\Large
\huge
\Huge