% typeset: Pdftex
% Afterwards compile with pdflatex > bibtex > pdflatex > pdflatex.
% in TeXShop preferences, changed edit from bibtex to biber
% beamer likes biber
% latex likes bibtex

\documentclass{article}
\usepackage[utf8]{inputenc}
\usepackage{geometry}
\geometry{a4paper, margin=1in}
\usepackage{catchfile}

% Fetch home directory
\CatchFileDef{\HomePath}{|kpsewhich -var-value=HOME}{}

% Strip trailing spaces from HomePath
\makeatletter
\edef\HomePath{\expandafter\zap@space\HomePath \@empty}
\makeatother

% Define base paths
% relies on symlink  at $HOME, e.g.
% 	GitHub -> /Users/dantopa//repos-xiuhcoatal/github
\newcommand{\pGithub}			{\HomePath/GitHub/}
\newcommand{\pGithubSharing}	{\pGithub/sharing/}
\newcommand{\pGlobal}			{\pGithubSharing/global/}
\newcommand{\pGlobalSetup}		{\pGlobal/setup-global/}

% Load Additional Setup Files
	% \input{\pGlobalSetup setup-global-reports.tex}
% global setup

% main

\input{\pGlobalSetup paths-global.tex}
\input{\pGlobalSetup paths-local.tex}
\input{\pGlobalSetup paths-bitbucket}
\input{\pGlobalSetup usepackages-reports.tex}
\input{\pGlobalSetup hyperlinks-global.tex}
\input{\pGlobalSetup libraries-global.tex}
\input{\pGlobalSetup theorems.tex}
\input{\pGlobalSetup enumeration.tex}
%\input{\pGlobalSetup macros-global.tex}

\endinput  %  -  -  -  -  -  -  -  -  -  -  -  -  -  -  -  -  -  -  -  -

	% \input{\pLocalSetup setup-local}

%\input{\pLocalSetup paths-local}
%\input{\pLocalSetup bibliography-local}
\input{\pLocalSetup hyperlinks-local}
%\input{\pLocalSetup input-libraries-local}

\input{\pLocalSetup macros-local}
% LaTeX macros to present a programming environment
% Looks like a terminal session, colored text on black background
\newcommand{\scru}[1]		{\bl   {\texttt{#1}}}
\newcommand{\scrk}[1]		{\bk   {\texttt{#1}}}
\newcommand{\scrg}[1]		{\gr   {\texttt{#1}}}
\newcommand{\scrr}[1]		{\rd   {\texttt{#1}}}
\newcommand{\scrv}[1]		{\dg   {\texttt{#1}}}
\newcommand{\scrz}[1]		{\mg   {\texttt{#1}}}
\newcommand{\scrc}[1]		{\textcolor{cyan}  {\texttt{#1}}}
\newcommand{\scrp}[1]		{\textcolor{purple}{\texttt{#1}}}
\newcommand{\scry}[1]		{\textcolor{yellow}{\texttt{#1}}}
\newcommand{\scrw}[1]		{\textcolor{white} {\texttt{#1}}}

%
\newcommand{\mySize}[1]		{{\normalsize{#1}}}
\newcommand{\myCanvas}[1]		{{\setbeamercolor{background canvas}{bg=black}{#1}}}
\newcommand{\myRasa}[1]		{\myCanvas{\mySize{#1}}}
\newcommand{\myFrame}[2]		{\myRasa{\begin{frame} \frametitle{#1} #2 \end{frame}}}
\newcommand{\myFoo}[2]			{\myFrame{#1}  \program \ \\[10pt] #2 \eprogram}


% tabs
%\newcommand{\tab}[0]     {\phantom{mm}}
\newcommand{\tab}[0]		{\ \ }
\newcommand{\spacer}[0]		{\ps\text{::}\ps}
\newcommand{\ps}[0]		{\phantom{m}}

% is equal?
\newcommand{\iseq}[0]		{ \overset{?}{=} }


\endinput  %  ==  ==  ==  ==  ==  ==  ==  ==  ==

% https://engineering.purdue.edu/ECN/Support/KB/Docs/LaTeXChangingTheFont
% \tiny
% \scriptsize
% \footnotesize
% \small
% \normalsize
% \large
% \Large
% \LARGE
% \huge
% \Huge% programming enviro

\endinput  %  ==  ==  ==  ==  ==  ==  ==  ==  ==


% Bibliography
	\input{\pGlobalSetup packages-global-bibliography-charlie.tex}
%\bibliography{\pShareBibliographies precise.bib}
%\addbibresource{\pShareBibliographies/precise.bib} % Add another file if needed
\addbibresource{\pShareBibliographies/fortran.bib} % Add another file if needed

% watermark
\usepackage[printwatermark]		{xwatermark}
\newcommand{\WatermarkText}	{DRAFT}
\newcommand{\WatermarkColor}	{red!5}
%\newwatermark[allpages, color=\WatermarkColor, angle=45, scale=3, xpos=0, ypos=0]{\WatermarkText}
	%\newwatermark[ allpages,color=red!5, angle=45, scale=3, xpos=0, ypos=0]{DRAFT}
\sloppy % loosen spacing globally:

% Title and author
\title{SwarmStorm: Coarrays in Satellite Simulations}
\author{Daniel Topa\\\TopaHIIEmail}
\affil{\missiontech}
\date{\today}

\author{Daniel T. and Achates}
\date{\today}

\begin{document}

\maketitle

\tableofcontents

\section*{Abstract}
This document explores the application of Fortran coarrays to simulate satellite engagements. By leveraging coarrays' parallelism and synchronization capabilities, the simulation models packs of satellites as coarray images and individual satellites as coarray indices. The concept of \textbf{causal bubbles} which cluster engagements by causal connection offers a powerful abstraction for reducing interdependencies and enhancing the efficiency of parallel discrete event simulation (PDES) in satellite engagements. Within the causal bubbles, we allow interactions to occur at faster times scales which relieves the totality of interactions from the smallest times scale.. The report outlines the foundational principles, architectural considerations, and implementation details of this experimental approach.

\section{Genesis}

The use of coarrays in satellite simulations provides a powerful paradigm for modeling distributed systems. This section explores the foundational concepts and initial design considerations for simulating satellite engagements using coarrays in Fortran.

\subsection{Concepts and Design Philosophy}

The simulation models satellite engagements, where:
\begin{itemize}
    \item \textbf{Packs of Satellites}: Each pack is represented by an image in the coarray model.
    \item \textbf{Individual Satellites}: Satellites within a pack are indexed as elements in coarrays.
    \item \textbf{Engagement Dynamics}: Packs engage enemy packs, sharing data and state across images via coarray communication.
\end{itemize}

This structure leverages Fortran’s coarray capabilities to parallelize and simplify the simulation:
\begin{itemize}
    \item \textbf{Parallelism}: Each image independently simulates its pack, enabling concurrent computation.
    \item \textbf{Data Sharing}: Coarray variables facilitate efficient inter-image communication.
    \item \textbf{Scalability}: The design can accommodate increasing numbers of satellites and packs.
\end{itemize}

\subsection{Initial Considerations}

The key architectural choices are:
\begin{itemize}
    \item \textbf{Packs as Images}:
        Each image represents a pack of satellites. Internal dynamics are simulated locally, while inter-pack engagements are handled via coarray references.
    \item \textbf{Satellites as Coarray Indices}:
        Satellites within a pack are indexed in coarrays. This simplifies data management and ensures efficient access to satellite states.
    \item \textbf{Synchronization}:
        Use coarray synchronization primitives (\texttt{sync all}, \texttt{sync team}, etc.) to manage interactions between packs.
    \item \textbf{Encapsulation}:
        Encapsulate satellite behavior within types, ensuring clean separation between simulation logic and data structure.
\end{itemize}

\subsection{High-Level Design}

The simulation follows this high-level flow:
\begin{enumerate}
    \item \textbf{Initialization}:
        Distribute satellite data across images and initialize states (e.g., position, velocity, fuel).
    \item \textbf{Engagement Simulation}:
        Simulate pack dynamics within each image and coordinate inter-pack engagements using coarray communication.
    \item \textbf{Result Aggregation}:
        Gather and summarize results for analysis (e.g., pack status, satellite losses).
\end{enumerate}

This design ensures scalability, modularity, and efficient use of Fortran’s coarray features. Future sections will explore specific implementation details and results of these simulations.
	\input{\pSections "sec-fortran.tex"}
%
\section{Causal Bubbles}

The concept of \textbf{causal bubbles} offers a powerful abstraction for reducing interdependencies and enhancing the efficiency of parallel discrete event simulation (PDES) in satellite engagements. This section outlines the principles, benefits, and implementation strategies for leveraging causal bubbles with coarrays.

\subsection{Definition and Advantages}

A causal bubble is an isolated group of events or processes that interact only within the bubble, without affecting or being affected by external processes. This approach has several advantages:
\begin{itemize}
    \item \textbf{Reduced Communication Overhead:} Events in one mission bubble do not impact another, minimizing the need for inter-bubble communication.
    \item \textbf{Independent Processing:} Each mission can proceed independently, leveraging parallelism effectively.
    \item \textbf{Scalability:} The simulation can handle increasing numbers of missions without a proportional increase in complexity.
\end{itemize}

\subsection{Local Timescales}

Each mission operates on a local timescale, which can have finer granularity than the global simulation clock. This allows missions with faster dynamics to resolve events more accurately without slowing down the overall simulation. For example:
\begin{itemize}
    \item The global simulation clock advances in 1-second intervals.
    \item Missions resolve their events at a finer resolution, such as 0.1 seconds.
\end{itemize}

\subsection{Implementation Strategy}

\subsubsection{Using Coarrays}

Coarrays provide a natural way to map missions and their satellites to parallel processes:
\begin{itemize}
    \item Each \textbf{mission bubble} is assigned to a coarray image or a team of images.
    \item \textbf{Synchronization} within a bubble is managed using \texttt{sync team}, while \texttt{sync all} is used only for global reporting or major events.
\end{itemize}

\subsubsection{Example Code}

The following illustrates the implementation of local timescales and event resolution:
\begin{verbatim}
real :: global_time = 0.0   ! Overall simulation clock
real :: mission_time = 0.0  ! Local clock for the mission

do while (global_time < max_simulation_time)
    ! Local events
    mission_time = mission_time + 0.1
    call resolve_local_events(mission_time)

    ! Synchronize with global clock periodically
    if (mod(global_time, 1.0) == 0.0) then
        sync team
    end if

    global_time = global_time + 1.0
end do
\end{verbatim}

\subsection{Benefits of the Approach}

By treating missions as causal bubbles:
\begin{itemize}
    \item \textbf{Parallelism is Maximized:} Independent missions can run simultaneously on separate images.
    \item \textbf{Accuracy is Preserved:} Local timescales allow finer resolution where needed.
    \item \textbf{Synchronization Overhead is Minimized:} Inter-bubble communication is avoided unless absolutely necessary.
\end{itemize}

This design ensures an efficient and scalable simulation framework for modeling satellite engagements.

\printbibliography

\end{document}
