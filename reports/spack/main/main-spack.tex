\documentclass[10pt, oneside]{article}   	% use "amsart" instead of "article" for AMSLaTeX format
\usepackage{geometry}                		% See geometry.pdf to learn the layout options. There are lots.
\geometry{letterpaper}                   		% ... or a4paper or a5paper or ... 
%\geometry{landscape}                		% Activate for rotated page geometry
%\usepackage[parfill]{parskip}    		% Activate to begin paragraphs with an empty line rather than an indent
\usepackage{graphicx}				% Use pdf, png, jpg, or eps§ with pdflatex; use eps in DVI mode
								% TeX will automatically convert eps --> pdf in pdflatex		
\usepackage{amssymb}
\usepackage[affil-it]{authblk}
\usepackage{fancyvrb}
\usepackage{hyperref}
\usepackage{overpic}
\usepackage{xcolor}

\usepackage{listings}
	\definecolor{textblue}{rgb}{.2,.2,.7}
	\definecolor{textred}{rgb}{0.54,0,0}
	\definecolor{textgreen}{rgb}{0,0.43,0}

% personalize environment
\newcommand{\pGlobal}[0]			{../../../global/}
\newcommand{\pGlobalSetup}[0]		{\pGlobal setup-global/}	

\input{\pGlobalSetup setup-global-reports}
\input{\pLocalSetup macros}


\title{Package Management With \texttt{Spack}}
\author{Daniel Topa\\\href{mailto:daniel.topa@hii-tsd.com}{daniel.topa@hii-tsd.com}}
\affil{\href{https://hii.com/what-we-do/divisions/mission-technologies/}{Mission Technologies}
\\Huntington Ingalls Industries
\\Kirtland AFB, NM}
%\date{}							% Activate to display a given date or no date

\begin{document}
\maketitle
\abstract{The package \spack \ is a widely used and modern package management toolset born on the HPC and now exploited for personal computation. By design, \spack \ allows user to unite environments under a compiler with a Python version, an MPI instance and manage the many variants. A simple example is provided which demonstrates how quickly the application can be downloaded and used. We conclude with links to articles and briefings which may be of interest to the new user.}

\section{Prevalence}

\subsection{\spack \ Users, Platforms}
\spack \ is used extensively across HPC platforms and personal computing platforms, many of which are sampled below.

\begin{enumerate}
	\item \href{https://ipv6.rs/tutorial/Windows_11/Spack/}{Windows} \href{https://www.kitware.com/spack-on-windows-a-new-era-in-cross-platform-dependency-management/}{11}, \href{https://ipv6.rs/tutorial/macOS/Spack/}{MacOS}, \href{https://en.wikipedia.org/wiki/ARM_architecture_family}{ARM}, \href{https://en.wikipedia.org/wiki/POWER8}{Power8, Power9}, \href{https://en.wikipedia.org/wiki/X86-64}{x86-64}, \href{https://en.wikipedia.org/wiki/IBM_Blue_Gene}{BlueGene}
	\item \href{https://oidc.hwss.hpc.mil/authentication?response_type=code&client_id=FKs5BqquL44U&scope=hpcmp%20openid%20profile&nonce=N66fd6ed373f44&response_mode=form_post&state=D3EvUDabr2Tu1cJ&redirect_uri=https%3A%2F%2Ftraining.hpc.mil%2Fauth%2Foidc%2F&resource=https%3A%2F%2Fgraph.microsoft.com}{DOD HPCMP}
	\item \href{https://computing.llnl.gov/projects/spack-hpc-package-manager}{Lawrence Livermore National Laboratory} 
	\item \href{https://dyninst.github.io/scalable_tools_workshop/petascale2018/assets/slides/Spack%20Shared%20Environment_ScalableTools2018.pdf}{Los Alamos National Laboratory} 
	\item \href{https://docs.olcf.ornl.gov/software/spack_environments.html}{Oak Ridge National Laboratory}
	\item \href{https://www.alcf.anl.gov/support-center/training/software-deployment-spack}{Argonne} \href{https://docs.alcf.anl.gov/polaris/applications-and-libraries/libraries/spack-pe/}{ National Laboratory}
	\item \href{https://www.intel.com/content/www/us/en/developer/articles/technical/distribution-of-optimized-hpc-binaries-using-spack.html}{Intel}
	\item \href{https://ncar-hpc-docs.readthedocs.io/en/latest/environment-and-software/user-environment/spack/spack/}{NCAR} 
	\item \href{https://github.com/HSF/hep-spack}{CERN}
	\item \href{https://www.hpc.iastate.edu/guides/using-spack-to-build-packages}{Iowa State HPC}
	\item \href{https://chtc.cs.wisc.edu/uw-research-computing/hpc-spack-setup}{University of Wisconsin–Madison}
	\item \href{https://kb.uconn.edu/space/SH/26074480709/Spack+Package+Manager}{UConn Storrs HPC}
	\item \href{https://documentation.its.umich.edu/node/5047}{University of Michigan}
	\item \href{https://hpc.nmsu.edu/discovery/software/spack/}{NM State}
	\item \href{https://lehigh.atlassian.net/wiki/spaces/hpc/pages/45252863/Spack}{Lehigh}
	\item \href{https://aws.amazon.com/blogs/hpc/install-optimized-software-with-spack-configs-for-aws-parallelcluster/}{Amazon Web Services}
	\item \href{https://techcommunity.microsoft.com/t5/azure-high-performance-computing/spack-in-a-multi-user-hpc-environment-on-azure/ba-p/3438261s	}{Azure}
\end{enumerate}

\section{Getting Started}

\subsection{Quick Example: \texttt{hwloc}}
Consider an example build of the package, \texttt{hwloc}.

\begin{quote}
The Hardware Locality (hwloc) software project. The Portable Hardware
Locality (hwloc) software package provides a portable abstraction
(across OS, versions, architectures, ...) of the hierarchical topology
of modern architectures, including NUMA memory nodes, sockets, shared
caches, cores and simultaneous multithreading. It also gathers various
system attributes such as cache and memory information as well as the
locality of I/O devices such as network interfaces, InfiniBand HCAs or
GPUs. It primarily aims at helping applications with gathering
information about modern computing hardware so as to exploit it
accordingly and efficiently.
\end{quote}

\subsubsection{Basic Steps}
\begin{enumerate}
	\item download spack
	\item initialize spack
	\item install hwloc
\end{enumerate}

\subsubsection{Command Line Steps and Result}
\begin{verbatim}
$ git clone https://github.com/spack/spack.git
$ source spack/share/spack/setup-env.sh
$ spack install hwloc
\end{verbatim}

%\begin{lstlisting}[language=bash]
\subsection{Install \spack, build \texttt{hwloc}}
{\scriptsize{
\begin{Verbatim}[commandchars=\\\{\}]
{\color{darkgray}{dantopa@Xiuhcoatl-8.local:example }}$ git clone https://github.com/spack/spack.git
{\color{darkgray}{Cloning into 'spack'...}}
{\color{darkgray}{remote: Enumerating objects: 582107, done.}}
{\color{darkgray}{remote: Counting objects: 100% (1607/1607), done.}}
{\color{darkgray}{remote: Compressing objects: 100% (799/799), done.}}
{\color{darkgray}{remote: Total 582107 (delta 772), reused 1273 (delta 547), pack-reused 580500 (from 1)}}
{\color{darkgray}{Receiving objects: 100% (582107/582107), 197.03 MiB | 35.67 MiB/s, done.}}
{\color{darkgray}{Resolving deltas: 100% (273672/273672), done.}}
{\color{darkgray}{Updating files: 100% (11933/11933), done.}}

{\color{darkgray}{dantopa@Xiuhcoatl-8.local:example }}$ spack install hwloc
{\color{darkgray}{==> Installing gmake-4.4.1-ietaaa3kpwrzml6fhorys6hakqmisyf4 [1/8]}}
{\color{darkgray}{==> No binary for gmake-4.4.1-ietaaa3kpwrzml6fhorys6hakqmisyf4 found: installing from source}}
{\color{darkgray}{==> Fetching https://mirror.spack.io/_source-cache/archive/dd/dd16fb1d67bfab79a72f5e8390735c49e3e8e70b4945a15ab1f81ddb78658fb3.tar.gz}}
{\color{darkgray}{==> No patches needed for gmake}}
{\color{darkgray}{==> gmake: Executing phase: 'install'}}
{\color{darkgray}{==> gmake: Successfully installed }} gmake-4.4.1-ietaaa3kpwrzml6fhorys6hakqmisyf4
{\color{darkgray}{  Stage: 0.65s.  Install: 36.72s.  Post-install: 0.06s.  Total: 37.56s}}
{\color{darkgray}{[+] /Volumes/spacktivity/example/spack/opt/spack/darwin-sonoma-skylake/apple-clang-16.0.0/gmake-4.4.1-ietaaa3kpwrzml6fhorys6hakqmisyf4}}
{\color{darkgray}{==> Installing xz-5.4.6-hjg33x3qi6bqecwmlghxfezuddtwcjhw [2/8]}}
{\color{darkgray}{==> No binary for xz-5.4.6-hjg33x3qi6bqecwmlghxfezuddtwcjhw found: installing from source}}
{\color{darkgray}{==> Fetching https://mirror.spack.io/_source-cache/archive/91/913851b274e8e1d31781ec949f1c23e8dbcf0ecf6e73a2436dc21769dd3e6f49.tar.bz2}}
{\color{darkgray}{==> No patches needed for xz}}
{\color{darkgray}{==> xz: Executing phase: 'autoreconf'}}
{\color{darkgray}{==> xz: Executing phase: 'configure'}}
{\color{darkgray}{==> xz: Executing phase: 'build'}}
{\color{darkgray}{==> xz: Executing phase: 'install'}}
{\color{darkgray}{==> xz: Successfully installed }}xz-5.4.6-hjg33x3qi6bqecwmlghxfezuddtwcjhw
{\color{darkgray}{  Stage: 0.78s.  Autoreconf: 0.00s.  Configure: 28.71s.  Build: 12.71s.  Install: 3.55s.  Post-install: 0.27s.  Total: 46.30s}}
{\color{darkgray}{[+] /Volumes/spacktivity/example/spack/opt/spack/darwin-sonoma-skylake/apple-clang-16.0.0/xz-5.4.6-hjg33x3qi6bqecwmlghxfezuddtwcjhw}}
{\color{darkgray}{==> Installing libiconv-1.17-oo6aigel5hcpcpfcvzlmit5mvbkzrrss [3/8]}}
{\color{darkgray}{==> No binary for libiconv-1.17-oo6aigel5hcpcpfcvzlmit5mvbkzrrss found: installing from source}}
{\color{darkgray}{==> Fetching https://mirror.spack.io/_source-cache/archive/8f/8f74213b56238c85a50a5329f77e06198771e70dd9a739779f4c02f65d971313.tar.gz}}
{\color{darkgray}{==> No patches needed for libiconv}}
{\color{darkgray}{==> libiconv: Executing phase: 'autoreconf'}}
{\color{darkgray}{==> libiconv: Executing phase: 'configure'}}
{\color{darkgray}{==> libiconv: Executing phase: 'build'}}
{\color{darkgray}{==> libiconv: Executing phase: 'install'}}
{\color{darkgray}{==> libiconv: Successfully installed }}libiconv-1.17-oo6aigel5hcpcpfcvzlmit5mvbkzrrss
{\color{darkgray}{  Stage: 0.96s.  Autoreconf: 0.00s.  Configure: 54.41s.  Build: 11.15s.  Install: 1.82s.  Post-install: 0.22s.  Total: 1m 8.92s}}
{\color{darkgray}{[+] /Volumes/spacktivity/example/spack/opt/spack/darwin-sonoma-skylake/apple-clang-16.0.0/libiconv-1.17-oo6aigel5hcpcpfcvzlmit5mvbkzrrss}}
{\color{darkgray}{==> Installing zlib-ng-2.2.1-rjskn465o44z4n6q24dksiby2pd5lpm3 [4/8]}}
{\color{darkgray}{==> No binary for zlib-ng-2.2.1-rjskn465o44z4n6q24dksiby2pd5lpm3 found: installing from source}}
{\color{darkgray}{==> Fetching https://mirror.spack.io/_source-cache/archive/ec/ec6a76169d4214e2e8b737e0850ba4acb806c69eeace6240ed4481b9f5c57cdf.tar.gz}}
{\color{darkgray}{==> No patches needed for zlib-ng}}
{\color{darkgray}{==> zlib-ng: Executing phase: 'autoreconf'}}
{\color{darkgray}{==> zlib-ng: Executing phase: 'configure'}}
{\color{darkgray}{==> zlib-ng: Executing phase: 'build'}}
{\color{darkgray}{==> zlib-ng: Executing phase: 'install'}}
{\color{darkgray}{==> zlib-ng: Successfully installed }} zlib-ng-2.2.1-rjskn465o44z4n6q24dksiby2pd5lpm3
{\color{darkgray}{  Stage: 0.94s.  Autoreconf: 0.00s.  Configure: 10.03s.  Build: 6.46s.  Install: 0.37s.  Post-install: 0.06s.  Total: 18.15s}}
{\color{darkgray}{[+] /Volumes/spacktivity/example/spack/opt/spack/darwin-sonoma-skylake/apple-clang-16.0.0/zlib-ng-2.2.1-rjskn465o44z4n6q24dksiby2pd5lpm3}}
{\color{darkgray}{==> Installing pkgconf-2.2.0-7pmnvez4bcl4ydiuih3syxr4w6jlful6 [5/8]}}
{\color{darkgray}{==> No binary for pkgconf-2.2.0-7pmnvez4bcl4ydiuih3syxr4w6jlful6 found: installing from source}}
{\color{darkgray}{==> Fetching https://mirror.spack.io/_source-cache/archive/b0/b06ff63a83536aa8c2f6422fa80ad45e4833f590266feb14eaddfe1d4c853c69.tar.xz}}
{\color{darkgray}{==> No patches needed for pkgconf}}
{\color{darkgray}{==> pkgconf: Executing phase: 'autoreconf'}}
{\color{darkgray}{==> pkgconf: Executing phase: 'configure'}}
{\color{darkgray}{==> pkgconf: Executing phase: 'build'}}
{\color{darkgray}{==> pkgconf: Executing phase: 'install'}}
{\color{darkgray}{==> pkgconf: Successfully installed }} pkgconf-2.2.0-7pmnvez4bcl4ydiuih3syxr4w6jlful6
{\color{darkgray}{  Stage: 0.73s.  Autoreconf: 0.00s.  Configure: 11.10s.  Build: 2.28s.  Install: 0.64s.  Post-install: 0.06s.  Total: 15.25s}}
{\color{darkgray}{[+] /Volumes/spacktivity/example/spack/opt/spack/darwin-sonoma-skylake/apple-clang-16.0.0/pkgconf-2.2.0-7pmnvez4bcl4ydiuih3syxr4w6jlful6}}
{\color{darkgray}{==> Installing libxml2-2.10.3-as2t7b3gziclpsms3fge2vyyhg7gwl5r [6/8]}}
{\color{darkgray}{==> No binary for libxml2-2.10.3-as2t7b3gziclpsms3fge2vyyhg7gwl5r found: installing from source}}
{\color{darkgray}{==> Fetching https://mirror.spack.io/_source-cache/archive/5d/5d2cc3d78bec3dbe212a9d7fa629ada25a7da928af432c93060ff5c17ee28a9c.tar.xz}}
{\color{darkgray}{==> Fetching https://mirror.spack.io/_source-cache/archive/96/96151685cec997e1f9f3387e3626d61e6284d4d6e66e0e440c209286c03e9cc7.tar.gz}}
{\color{darkgray}{==> Moving resource stage}}
{\color{darkgray}{	source: /var/folders/f2/0qk5gn4x1rlczv63skzbp19h0000gn/T/dantopa/spack-stage/resource-xmlts-as2t7b3gziclpsms3fge2vyyhg7gwl5r/spack-src/}}
{\color{darkgray}{	destination: /var/folders/f2/0qk5gn4x1rlczv63skzbp19h0000gn/T/dantopa/spack-stage/spack-stage-libxml2-2.10.3-as2t7b3gziclpsms3fge2vyyhg7gwl5r/spack-src/xmlconf}}
{\color{darkgray}{==> Ran patch() for libxml2}}
{\color{darkgray}{==> libxml2: Executing phase: 'autoreconf'}}
{\color{darkgray}{==> libxml2: Executing phase: 'configure'}}
{\color{darkgray}{==> libxml2: Executing phase: 'build'}}
{\color{darkgray}{==> libxml2: Executing phase: 'install'}}
{\color{darkgray}{==> libxml2: Successfully installed }}libxml2-2.10.3-as2t7b3gziclpsms3fge2vyyhg7gwl5r
{\color{darkgray}{  Stage: 4.56s.  Autoreconf: 0.00s.  Configure: 18.28s.  Build: 12.18s.  Install: 1.78s.  Post-install: 0.12s.  Total: 37.16s}}
{\color{darkgray}{[+] /Volumes/spacktivity/example/spack/opt/spack/darwin-sonoma-skylake/apple-clang-16.0.0/libxml2-2.10.3-as2t7b3gziclpsms3fge2vyyhg7gwl5r}}
{\color{darkgray}{==> Installing ncurses-6.5-y4puwqifh7lcfoyme4xerqpyhy6wk5dd [7/8]}}
{\color{darkgray}{==> No binary for ncurses-6.5-y4puwqifh7lcfoyme4xerqpyhy6wk5dd found: installing from source}}
{\color{darkgray}{==> Fetching https://mirror.spack.io/_source-cache/archive/13/136d91bc269a9a5785e5f9e980bc76ab57428f604ce3e5a5a90cebc767971cc6.tar.gz}}
{\color{darkgray}{==> Applied patch /Volumes/spacktivity/example/spack/var/spack/repos/builtin/packages/ncurses/rxvt_unicode_6_4.patch}}
{\color{darkgray}{==> ncurses: Executing phase: 'autoreconf'}}
{\color{darkgray}{==> ncurses: Executing phase: 'configure'}}
{\color{darkgray}{==> ncurses: Executing phase: 'build'}}
{\color{darkgray}{==> ncurses: Executing phase: 'install'}}
{\color{darkgray}{==> ncurses: Successfully installed }}ncurses-6.5-y4puwqifh7lcfoyme4xerqpyhy6wk5dd
{\color{darkgray}{  Stage: 0.83s.  Autoreconf: 0.00s.  Configure: 1m 29.88s.  Build: 50.04s.  Install: 19.11s.  Post-install: 2.56s.  Total: 2m 42.67s}}
{\color{darkgray}{[+] /Volumes/spacktivity/example/spack/opt/spack/darwin-sonoma-skylake/apple-clang-16.0.0/ncurses-6.5-y4puwqifh7lcfoyme4xerqpyhy6wk5dd}}
{\color{darkgray}{==> Installing hwloc-2.11.1-mfauw6yq45zhpldzh7ot5ns6tiisx4x2 [8/8]}}
{\color{darkgray}{==> No binary for hwloc-2.11.1-mfauw6yq45zhpldzh7ot5ns6tiisx4x2 found: installing from source}}
{\color{darkgray}{==> Fetching https://mirror.spack.io/_source-cache/archive/9f/9f320925cfd0daeaf3a3d724c93e127ecac63750c623654dca0298504aac4c2c.tar.gz}}
{\color{darkgray}{==> No patches needed for hwloc}}
{\color{darkgray}{==> hwloc: Executing phase: 'autoreconf'}}
{\color{darkgray}{==> hwloc: Executing phase: 'configure'}}
{\color{darkgray}{==> hwloc: Executing phase: 'build'}}
{\color{darkgray}{==> hwloc: Executing phase: 'install'}}
{\color{darkgray}{==> hwloc: Successfully installed hwloc-2.11.1-mfauw6yq45zhpldzh7ot5ns6tiisx4x2}}
{\color{darkgray}{  Stage: 1.23s.  Autoreconf: 0.00s.  Configure: 43.66s.  Build: 7.48s.  Install: 2.53s.  Post-install: 0.26s.  Total: 55.51s}}
{\color{darkgray}{[+] /Volumes/spacktivity/example/spack/opt/spack/darwin-sonoma-skylake/apple-clang-16.0.0/}}hwloc-2.11.1{\color{darkgray}{-mfauw6yq45zhpldzh7ot5ns6tiisx4x2}}
\end{Verbatim}
}}
\begin{figure}[htbp] %  figure placement: here, top, bottom, or page
   \centering
	\href{https://computing.llnl.gov/sites/default/files/ares.00818A.png}{
	\begin{overpic}[ scale = 0.075 ]
		{\pLocalGraphics ares}
		%\put(50, 50)	{\colorbox{white}{$a+b$}}
	\end{overpic}}
\caption{Sample dependency tree managed by \spack.}
\label{fig:ares}
\end{figure}


\subsection{Hardware Locality}
The hardware locality application \texttt{hwloc} provides insight into the hardware configuration of the host machine. An example using \texttt{lstopo} is shown in figure \ref{fig:lstopo}.
\begin{figure}[htbp] %  figure placement: here, top, bottom, or page
   \centering
	\begin{overpic}[ scale = 0.75 ]
		{\pLocalGraphics lstopo-p}
		%\put(50, 50)	{\colorbox{white}{$a+b$}}
	\end{overpic}
\caption{The application \texttt{hwloc} contains the utility \texttt{lstopo} which identifies the hardware configuration.}
\label{fig:lstopo}
\end{figure}

\subsection{How Does \spack \ Work?}
\spack \ is a \emph{download}, not an \emph{installation}. It was created at Livermore to empower scientists to build their own custom software stacks. What started as a tool for people with local admin privileges over their machines is now a recognized tool used by the HPC support staffs world wide.

\spack \ changes how developers interact with their uses. Instead of maintaining pages detailing install instructions for each hardware architecture and software environment, developers now maintain a single \spack \ instance and utilize the issue tracking inherent in \href{https://github.com/spack/spack}{\texttt{GitHub}}.

Whether the build system is autotools, make, cmake, ninja, etc., \spack \ automates the process. A critical property is that \spack \ build package creators use a standardized template which causes uniform performance of the builds. Python scripts interrogate the local hardware and software environments. 

In essence, \spack \ is a database managing dependencies, variants, and locations. Below is a sample tree diagram for a Livermore hydrocode showing the complexity managed by \spack.

\spack \ handles combinatorial complexity. For example, consider 4 compilers: Intel, GCC, PGI, NAG. For each compiler maintain 4 different versions; for example gcc 14.2.0, 13.3.0, 12.4.0, 4.8.5. Provide 4 MPI providers: OpenMPI, Cray-MPICH, MVAPICH, Intel-Parallel studio. Maintain 4 versions of each of those. Maintain 4 Python versions for each packages. This represents $4^{5} = 1024$ instances, handled by \spack. 

\section{Probe commands in \spack}
There are many probe and diagnostic commands which help the builder understand the process and products. Two such commands are shown below.
\subsection{Graph dependencies}

\small{
\begin{lstlisting}[language=bash]
$ spack graph openmpi
kpex76l openmpi@1.10.7%gcc
vuijyrm     hwloc@1.11.13%gcc
vlgsd6a         libxml2@2.10.3%gcc
7ffbqyf             libiconv@1.17%gcc
cejtv5p             pkgconf@1.8.0%gcc
ydjmqn5             xz@5.4.1%gcc
kgdj2w7         ncurses@6.4%gcc
cbup2u4     openssh@9.1p1%gcc
74ofkad         krb5@1.20.1%gcc
gw3muwr             bison@3.8.2%gcc
mbfdcbq                 m4@1.4.19%gcc
ytuafo5                     libsigsegv@2.13%gcc
fx3kvo3             diffutils@3.8%gcc
g7g5rxm             gettext@0.21.1%gcc
pirykzv                 bzip2@1.0.8%gcc
lij4icg                 tar@1.34%gcc
3tfa2za                     pigz@2.7%gcc
hnuj2am                     zstd@1.5.2%gcc
mf4yylc         libedit@3.1-20210216%gcc
pnhvhts         libxcrypt@4.4.33%gcc
cck5u3i             perl@5.34.0%gcc
duhpddy         openssl@1.1.1t%gcc
syyclam             ca-certificates-mozilla@2023-01-10%gcc
ggaig6s         zlib@1.2.13%gcc
lxwy7gr     perl@5.34.0%gcc
kzdyfxk     pkgconf@1.8.0%gcc
\end{lstlisting}
}

\subsection{\texttt{spack info petsc}}
The \texttt{spack} command \texttt{info} presents essential information about each package in the following order. 
\begin{enumerate}
	\item Dependencies
	\item Homepage
	\item Versions
	\item Variants
	\begin{enumerate}
		\item build
		\item link
		\item run
	\end{enumerate}
	\item License
\end{enumerate}

The output starts with a brief description of the package and web site providing more information and a listing of available versions. Next is a list of variants and how to invoke them showing the user how to construct specific versions of the package -- which will all be managed by \texttt{spack}.  Users can easily specify whether to use \texttt{C} or \texttt{C++} for the build, whether to use double or single precision, whether to use \texttt{MPI}\footnote{\spack \ allows users to chose between many flavors of \href{https://www.mpi-forum.org/}{MPI}, e.g. \href{https://www.open-mpi.org/}{OpenMPI}, \href{https://www.mpich.org/}{MPICH}, \href{https://www.intel.com/content/www/us/en/developer/tools/oneapi/mpi-library.html}{Intel}, \href{https://downloads.linux.hpe.com/SDR/project/mpi/}{HPE}, etc.}, whether to use \openmp, and so on. The final sections lists dependencies for building, linking, and running. \spack \ will build these as needed.

	% % % \input{./components/bash/info-petsc}

%\lstdefinelanguage{Bash}%
%  {morekeywords={abstract,break,case,catch,const,continue,do,else,elseif,%
%      end,export,false,for,function,immutable,import,importall,if,in,%
%      macro,module,otherwise,quote,return,switch,true,try,type,typealias,%
%      using,while},%
%   sensitive=true,%
%   alsoother={$},%
%   morecomment=[l]\#,%
%   morecomment=[n]{\#=}{=\#},%
%   morestring=[s]{"}{"},%
%   morestring=[m]{'}{'},%
%}[keywords,comments,strings]%
%
%\lstset{%
%    language         = bash,
%    basicstyle       = \ttfamily,
%%    keywordstyle     = \bfseries\color{blue},
%%    stringstyle      = \color{magenta},
%%    commentstyle     = \color{ForestGreen},
%keywordstyle=\color{textblue},
%commentstyle=\color{textred},
%stringstyle=\color{textgreen},
%    showstringspaces = false,
%}

\small{
\begin{lstlisting}[language=bash]
$ spack info petsc
Package:   petsc

Description:
    PETSc is a suite of data structures and routines for the scalable
    (parallel) solution of scientific applications modeled by partial
    differential equations.

Homepage: https://petsc.org

Preferred version:  
    3.22.0    http://web.cels.anl.gov/projects/petsc/.../petsc-3.22.0.tar.gz

Safe versions:  
    main      [git] https://gitlab.com/petsc/petsc.git on branch main
    3.22.0    http://web.cels.anl.gov/projects/petsc/.../petsc-3.22.0.tar.gz
    3.21.6    http://web.cels.anl.gov/projects/petsc/.../petsc-3.21.6.tar.gz
    
    3.11.1    http://web.cels.anl.gov/projects/petsc/.../petsc-3.11.0.tar.gz

Deprecated versions:  
    None

Variants:
    X [false]                     false, true
        Activate X support
    batch [false]                 false, true
        Enable when mpiexec is not available to run binaries
    build_system [generic]        generic
        Build systems supported by the package
    cgns [false]                  false, true
        Activates support for CGNS (only parallel)
    clanguage [C]                 C, C++
        Specify C (recommended) or C++ to compile PETSc
    complex [false]               false, true
        Build with complex numbers
    cuda [false]                  false, true
        Build with CUDA
    debug [false]                 false, true
        Compile in debug mode
    double [true]                 false, true
        Switches between single and double precision
    exodusii [false]              false, true
        Activates support for ExodusII (only parallel)
    fftw [false]                  false, true
        Activates support for FFTW (only parallel)
    fortran [true]                false, true
        Activates fortran support
    giflib [false]                false, true
        Activates support for GIF
    hdf5 [true]                   false, true
        Activates support for HDF5 (only parallel)
    hpddm [false]                 false, true
        Activates support for HPDDM (only parallel)
    hwloc [false]                 false, true
        Activates support for hwloc
    hypre [true]                  false, true
        Activates support for Hypre (only parallel)
    int64 [false]                 false, true
        Compile with 64bit indices
    jpeg [false]                  false, true
        Activates support for JPEG
    knl [false]                   false, true
        Build for KNL
    kokkos [false]                false, true
        Activates support for kokkos and kokkos-kernels
    libpng [false]                false, true
        Activates support for PNG
    libyaml [false]               false, true
        Activates support for YAML
    memalign [none]               none, 16, 32, 4, 64, 8
        Specify alignment of allocated arrays
    memkind [false]               false, true
        Activates support for Memkind
    metis [true]                  false, true
        Activates support for metis and parmetis
    mkl-pardiso [false]           false, true
        Activates support for MKL Pardiso
    mmg [false]                   false, true
        Activates support for MMG
    moab [false]                  false, true
        Acivates support for MOAB (only parallel)
    mpfr [false]                  false, true
        Activates support for MPFR
    mpi [true]                    false, true
        Activates MPI support
    mumps [false]                 false, true
        Activates support for MUMPS (only parallel)
    openmp [false]                false, true
        Activates support for openmp
    p4est [false]                 false, true
        Activates support for P4Est (only parallel)
    parmmg [false]                false, true
        Activates support for ParMMG (only parallel)
    ptscotch [false]              false, true
        Activates support for PTScotch (only parallel)
    random123 [false]             false, true
        Activates support for Random123
    rocm [false]                  false, true
        Enable ROCm support
    saws [false]                  false, true
        Activates support for Saws
    shared [true]                 false, true
        Enables the build of shared libraries
    strumpack [false]             false, true
        Activates support for Strumpack
    suite-sparse [false]          false, true
        Activates support for SuiteSparse
    sycl [false]                  false, true
        Enable sycl build
    tetgen [false]                false, true
        Activates support for Tetgen
    trilinos [false]              false, true
        Activates support for Trilinos (only parallel)
    valgrind [false]              false, true
        Enable Valgrind Client Request mechanism
    zoltan [false]                false, true
        Activates support for Zoltan

    when +rocm
      amdgpu_target [none]        none, gfx1010, gfx1011, gfx1012, gfx1013, gfx1030, gfx1031, gfx1032, gfx1033, gfx1034, gfx1035, gfx1036, gfx1100, gfx1101, gfx1102, gfx1103, gfx701, gfx801, gfx802, gfx803, gfx900, gfx900:xnack-, gfx902, gfx904, gfx906, gfx906:xnack-, gfx908, gfx908:xnack-, gfx909, gfx90a, gfx90a:xnack+, gfx90a:xnack-, gfx90c, gfx940, gfx941, gfx942
          AMD GPU architecture

    when +cuda
      cuda_arch [none]            none, 10, 11, 12, 13, 20, 21, 30, 32, 35, 37, 50, 52, 53, 60, 61, 62, 70, 72, 75, 80, 86, 87, 89, 90, 90a
          CUDA architecture

    when +fortran
      scalapack [false]           false, true
          Activates support for Scalapack
      superlu-dist [true]         false, true
          Activates support for superlu-dist (only parallel)

Build Dependencies:
    blas  cuda       exodusii  giflib  gmp   hip      hipsolver  hsa-rocr-dev  hypre  kokkos          lapack  libx11   llvm-amdgpu  metis  mmg   mpfr  mumps     p4est            parmetis  python     rocblas    rocprim  rocsolver  rocthrust  scalapack  sowing     suite-sparse  tetgen    valgrind  zoltan
    cgns  diffutils  fftw      gmake   hdf5  hipblas  hipsparse  hwloc         jpeg   kokkos-kernels  libpng  libyaml  memkind      mkl    moab  mpi   netcdf-c  parallel-netcdf  parmmg    random123  rocm-core  rocrand  rocsparse  saws       scotch     strumpack  superlu-dist  trilinos  zlib-api

Link Dependencies:
    blas  cuda      fftw    gmake  hdf5  hipblas    hipsparse     hwloc  jpeg    kokkos-kernels  libpng  libyaml      memkind  mkl  moab  mpi    netcdf-c  parallel-netcdf  parmmg     rocblas    rocprim  rocsolver  rocthrust  scalapack  sowing     suite-sparse  tetgen    valgrind  zoltan
    cgns  exodusii  giflib  gmp    hip   hipsolver  hsa-rocr-dev  hypre  kokkos  lapack          libx11  llvm-amdgpu  metis    mmg  mpfr  mumps  p4est     parmetis         random123  rocm-core  rocrand  rocsparse  saws       scotch     strumpack  superlu-dist  trilinos  zlib-api

Run Dependencies:
    None

Licenses: 
    None
\end{lstlisting}
}
\endinput  %  ==  ==  ==  ==  ==  ==  ==  ==  ==


%\lstdefinelanguage{Bash}%
%  {morekeywords={abstract,break,case,catch,const,continue,do,else,elseif,%
%      end,export,false,for,function,immutable,import,importall,if,in,%
%      macro,module,otherwise,quote,return,switch,true,try,type,typealias,%
%      using,while},%
%   sensitive=true,%
%   alsoother={$},%
%   morecomment=[l]\#,%
%   morecomment=[n]{\#=}{=\#},%
%   morestring=[s]{"}{"},%
%   morestring=[m]{'}{'},%
%}[keywords,comments,strings]%
%
%\lstset{%
%    language         = bash,
%    basicstyle       = \ttfamily,
%%    keywordstyle     = \bfseries\color{blue},
%%    stringstyle      = \color{magenta},
%%    commentstyle     = \color{ForestGreen},
%keywordstyle=\color{textblue},
%commentstyle=\color{textred},
%stringstyle=\color{textgreen},
%    showstringspaces = false,
%}

\small{
\begin{lstlisting}[language=bash]
$ spack info petsc
Package:   petsc

Description:
    PETSc is a suite of data structures and routines for the scalable
    (parallel) solution of scientific applications modeled by partial
    differential equations.

Homepage: https://petsc.org

Preferred version:  
    3.22.0    http://web.cels.anl.gov/projects/petsc/.../petsc-3.22.0.tar.gz

Safe versions:  
    main      [git] https://gitlab.com/petsc/petsc.git on branch main
    3.22.0    http://web.cels.anl.gov/projects/petsc/.../petsc-3.22.0.tar.gz
    3.21.6    http://web.cels.anl.gov/projects/petsc/.../petsc-3.21.6.tar.gz
    
    3.11.1    http://web.cels.anl.gov/projects/petsc/.../petsc-3.11.0.tar.gz

Deprecated versions:  
    None

Variants:
    X [false]                     false, true
        Activate X support
    batch [false]                 false, true
        Enable when mpiexec is not available to run binaries
    build_system [generic]        generic
        Build systems supported by the package
    cgns [false]                  false, true
        Activates support for CGNS (only parallel)
    clanguage [C]                 C, C++
        Specify C (recommended) or C++ to compile PETSc
    complex [false]               false, true
        Build with complex numbers
    cuda [false]                  false, true
        Build with CUDA
    debug [false]                 false, true
        Compile in debug mode
    double [true]                 false, true
        Switches between single and double precision
    exodusii [false]              false, true
        Activates support for ExodusII (only parallel)
    fftw [false]                  false, true
        Activates support for FFTW (only parallel)
    fortran [true]                false, true
        Activates fortran support
    giflib [false]                false, true
        Activates support for GIF
    hdf5 [true]                   false, true
        Activates support for HDF5 (only parallel)
    hpddm [false]                 false, true
        Activates support for HPDDM (only parallel)
    hwloc [false]                 false, true
        Activates support for hwloc
    hypre [true]                  false, true
        Activates support for Hypre (only parallel)
    int64 [false]                 false, true
        Compile with 64bit indices
    jpeg [false]                  false, true
        Activates support for JPEG
    knl [false]                   false, true
        Build for KNL
    kokkos [false]                false, true
        Activates support for kokkos and kokkos-kernels
    libpng [false]                false, true
        Activates support for PNG
    libyaml [false]               false, true
        Activates support for YAML
    memalign [none]               none, 16, 32, 4, 64, 8
        Specify alignment of allocated arrays
    memkind [false]               false, true
        Activates support for Memkind
    metis [true]                  false, true
        Activates support for metis and parmetis
    mkl-pardiso [false]           false, true
        Activates support for MKL Pardiso
    mmg [false]                   false, true
        Activates support for MMG
    moab [false]                  false, true
        Acivates support for MOAB (only parallel)
    mpfr [false]                  false, true
        Activates support for MPFR
    mpi [true]                    false, true
        Activates MPI support
    mumps [false]                 false, true
        Activates support for MUMPS (only parallel)
    openmp [false]                false, true
        Activates support for openmp
    p4est [false]                 false, true
        Activates support for P4Est (only parallel)
    parmmg [false]                false, true
        Activates support for ParMMG (only parallel)
    ptscotch [false]              false, true
        Activates support for PTScotch (only parallel)
    random123 [false]             false, true
        Activates support for Random123
    rocm [false]                  false, true
        Enable ROCm support
    saws [false]                  false, true
        Activates support for Saws
    shared [true]                 false, true
        Enables the build of shared libraries
    strumpack [false]             false, true
        Activates support for Strumpack
    suite-sparse [false]          false, true
        Activates support for SuiteSparse
    sycl [false]                  false, true
        Enable sycl build
    tetgen [false]                false, true
        Activates support for Tetgen
    trilinos [false]              false, true
        Activates support for Trilinos (only parallel)
    valgrind [false]              false, true
        Enable Valgrind Client Request mechanism
    zoltan [false]                false, true
        Activates support for Zoltan

    when +rocm
      amdgpu_target [none]        none, gfx1010, gfx1011, gfx1012, gfx1013, gfx1030, gfx1031, gfx1032, gfx1033, gfx1034, gfx1035, gfx1036, gfx1100, gfx1101, gfx1102, gfx1103, gfx701, gfx801, gfx802, gfx803, gfx900, gfx900:xnack-, gfx902, gfx904, gfx906, gfx906:xnack-, gfx908, gfx908:xnack-, gfx909, gfx90a, gfx90a:xnack+, gfx90a:xnack-, gfx90c, gfx940, gfx941, gfx942
          AMD GPU architecture

    when +cuda
      cuda_arch [none]            none, 10, 11, 12, 13, 20, 21, 30, 32, 35, 37, 50, 52, 53, 60, 61, 62, 70, 72, 75, 80, 86, 87, 89, 90, 90a
          CUDA architecture

    when +fortran
      scalapack [false]           false, true
          Activates support for Scalapack
      superlu-dist [true]         false, true
          Activates support for superlu-dist (only parallel)

Build Dependencies:
    blas  cuda       exodusii  giflib  gmp   hip      hipsolver  hsa-rocr-dev  hypre  kokkos          lapack  libx11   llvm-amdgpu  metis  mmg   mpfr  mumps     p4est            parmetis  python     rocblas    rocprim  rocsolver  rocthrust  scalapack  sowing     suite-sparse  tetgen    valgrind  zoltan
    cgns  diffutils  fftw      gmake   hdf5  hipblas  hipsparse  hwloc         jpeg   kokkos-kernels  libpng  libyaml  memkind      mkl    moab  mpi   netcdf-c  parallel-netcdf  parmmg    random123  rocm-core  rocrand  rocsparse  saws       scotch     strumpack  superlu-dist  trilinos  zlib-api

Link Dependencies:
    blas  cuda      fftw    gmake  hdf5  hipblas    hipsparse     hwloc  jpeg    kokkos-kernels  libpng  libyaml      memkind  mkl  moab  mpi    netcdf-c  parallel-netcdf  parmmg     rocblas    rocprim  rocsolver  rocthrust  scalapack  sowing     suite-sparse  tetgen    valgrind  zoltan
    cgns  exodusii  giflib  gmp    hip   hipsolver  hsa-rocr-dev  hypre  kokkos  lapack          libx11  llvm-amdgpu  metis    mmg  mpfr  mumps  p4est     parmetis         random123  rocm-core  rocrand  rocsparse  saws       scotch     strumpack  superlu-dist  trilinos  zlib-api

Run Dependencies:
    None

Licenses: 
    None
\end{lstlisting}
}
\endinput  %  ==  ==  ==  ==  ==  ==  ==  ==  ==


%\lstdefinelanguage{Bash}%
%  {morekeywords={abstract,break,case,catch,const,continue,do,else,elseif,%
%      end,export,false,for,function,immutable,import,importall,if,in,%
%      macro,module,otherwise,quote,return,switch,true,try,type,typealias,%
%      using,while},%
%   sensitive=true,%
%   alsoother={$},%
%   morecomment=[l]\#,%
%   morecomment=[n]{\#=}{=\#},%
%   morestring=[s]{"}{"},%
%   morestring=[m]{'}{'},%
%}[keywords,comments,strings]%
%
%\lstset{%
%    language         = bash,
%    basicstyle       = \ttfamily,
%%    keywordstyle     = \bfseries\color{blue},
%%    stringstyle      = \color{magenta},
%%    commentstyle     = \color{ForestGreen},
%keywordstyle=\color{textblue},
%commentstyle=\color{textred},
%stringstyle=\color{textgreen},
%    showstringspaces = false,
%}

\small{
\begin{lstlisting}[language=bash]
$ spack info petsc
Package:   petsc

Description:
    PETSc is a suite of data structures and routines for the scalable
    (parallel) solution of scientific applications modeled by partial
    differential equations.

Homepage: https://petsc.org

Preferred version:  
    3.22.0    http://web.cels.anl.gov/projects/petsc/.../petsc-3.22.0.tar.gz

Safe versions:  
    main      [git] https://gitlab.com/petsc/petsc.git on branch main
    3.22.0    http://web.cels.anl.gov/projects/petsc/.../petsc-3.22.0.tar.gz
    3.21.6    http://web.cels.anl.gov/projects/petsc/.../petsc-3.21.6.tar.gz
    
    3.11.1    http://web.cels.anl.gov/projects/petsc/.../petsc-3.11.0.tar.gz

Deprecated versions:  
    None

Variants:
    X [false]                     false, true
        Activate X support
    batch [false]                 false, true
        Enable when mpiexec is not available to run binaries
    build_system [generic]        generic
        Build systems supported by the package
    cgns [false]                  false, true
        Activates support for CGNS (only parallel)
    clanguage [C]                 C, C++
        Specify C (recommended) or C++ to compile PETSc
    complex [false]               false, true
        Build with complex numbers
    cuda [false]                  false, true
        Build with CUDA
    debug [false]                 false, true
        Compile in debug mode
    double [true]                 false, true
        Switches between single and double precision
    exodusii [false]              false, true
        Activates support for ExodusII (only parallel)
    fftw [false]                  false, true
        Activates support for FFTW (only parallel)
    fortran [true]                false, true
        Activates fortran support
    giflib [false]                false, true
        Activates support for GIF
    hdf5 [true]                   false, true
        Activates support for HDF5 (only parallel)
    hpddm [false]                 false, true
        Activates support for HPDDM (only parallel)
    hwloc [false]                 false, true
        Activates support for hwloc
    hypre [true]                  false, true
        Activates support for Hypre (only parallel)
    int64 [false]                 false, true
        Compile with 64bit indices
    jpeg [false]                  false, true
        Activates support for JPEG
    knl [false]                   false, true
        Build for KNL
    kokkos [false]                false, true
        Activates support for kokkos and kokkos-kernels
    libpng [false]                false, true
        Activates support for PNG
    libyaml [false]               false, true
        Activates support for YAML
    memalign [none]               none, 16, 32, 4, 64, 8
        Specify alignment of allocated arrays
    memkind [false]               false, true
        Activates support for Memkind
    metis [true]                  false, true
        Activates support for metis and parmetis
    mkl-pardiso [false]           false, true
        Activates support for MKL Pardiso
    mmg [false]                   false, true
        Activates support for MMG
    moab [false]                  false, true
        Acivates support for MOAB (only parallel)
    mpfr [false]                  false, true
        Activates support for MPFR
    mpi [true]                    false, true
        Activates MPI support
    mumps [false]                 false, true
        Activates support for MUMPS (only parallel)
    openmp [false]                false, true
        Activates support for openmp
    p4est [false]                 false, true
        Activates support for P4Est (only parallel)
    parmmg [false]                false, true
        Activates support for ParMMG (only parallel)
    ptscotch [false]              false, true
        Activates support for PTScotch (only parallel)
    random123 [false]             false, true
        Activates support for Random123
    rocm [false]                  false, true
        Enable ROCm support
    saws [false]                  false, true
        Activates support for Saws
    shared [true]                 false, true
        Enables the build of shared libraries
    strumpack [false]             false, true
        Activates support for Strumpack
    suite-sparse [false]          false, true
        Activates support for SuiteSparse
    sycl [false]                  false, true
        Enable sycl build
    tetgen [false]                false, true
        Activates support for Tetgen
    trilinos [false]              false, true
        Activates support for Trilinos (only parallel)
    valgrind [false]              false, true
        Enable Valgrind Client Request mechanism
    zoltan [false]                false, true
        Activates support for Zoltan

    when +rocm
      amdgpu_target [none]        none, gfx1010, gfx1011, gfx1012, gfx1013, gfx1030, gfx1031, gfx1032, gfx1033, gfx1034, gfx1035, gfx1036, gfx1100, gfx1101, gfx1102, gfx1103, gfx701, gfx801, gfx802, gfx803, gfx900, gfx900:xnack-, gfx902, gfx904, gfx906, gfx906:xnack-, gfx908, gfx908:xnack-, gfx909, gfx90a, gfx90a:xnack+, gfx90a:xnack-, gfx90c, gfx940, gfx941, gfx942
          AMD GPU architecture

    when +cuda
      cuda_arch [none]            none, 10, 11, 12, 13, 20, 21, 30, 32, 35, 37, 50, 52, 53, 60, 61, 62, 70, 72, 75, 80, 86, 87, 89, 90, 90a
          CUDA architecture

    when +fortran
      scalapack [false]           false, true
          Activates support for Scalapack
      superlu-dist [true]         false, true
          Activates support for superlu-dist (only parallel)

Build Dependencies:
    blas  cuda       exodusii  giflib  gmp   hip      hipsolver  hsa-rocr-dev  hypre  kokkos          lapack  libx11   llvm-amdgpu  metis  mmg   mpfr  mumps     p4est            parmetis  python     rocblas    rocprim  rocsolver  rocthrust  scalapack  sowing     suite-sparse  tetgen    valgrind  zoltan
    cgns  diffutils  fftw      gmake   hdf5  hipblas  hipsparse  hwloc         jpeg   kokkos-kernels  libpng  libyaml  memkind      mkl    moab  mpi   netcdf-c  parallel-netcdf  parmmg    random123  rocm-core  rocrand  rocsparse  saws       scotch     strumpack  superlu-dist  trilinos  zlib-api

Link Dependencies:
    blas  cuda      fftw    gmake  hdf5  hipblas    hipsparse     hwloc  jpeg    kokkos-kernels  libpng  libyaml      memkind  mkl  moab  mpi    netcdf-c  parallel-netcdf  parmmg     rocblas    rocprim  rocsolver  rocthrust  scalapack  sowing     suite-sparse  tetgen    valgrind  zoltan
    cgns  exodusii  giflib  gmp    hip   hipsolver  hsa-rocr-dev  hypre  kokkos  lapack          libx11  llvm-amdgpu  metis    mmg  mpfr  mumps  p4est     parmetis         random123  rocm-core  rocrand  rocsparse  saws       scotch     strumpack  superlu-dist  trilinos  zlib-api

Run Dependencies:
    None

Licenses: 
    None
\end{lstlisting}
}
\endinput  %  ==  ==  ==  ==  ==  ==  ==  ==  ==


\section{Learn More About Spack}

\subsection{Spack Awards}
At the latest \href{https://sc23.supercomputing.org/}{2023 International Conference for High Performance Computing, Networking, Storage and Analysis}  \spack \ recognized as the Best High Performance Computing (HPC) Programming Tool or Technology:
\href{https://www.llnl.gov/article/50626/spack-receives-prestigious-hpcwire-award-sc23}{Spack receives prestigious HPCwire award at SC23}


\subsection{Articles}
\begin{enumerate}
	\item \href{https://computing.llnl.gov/projects/spack-hpc-package-manager}{Spack: A Flexible Package Manager for HPC Software} 
	\item \href{https://arxiv.org/pdf/2211.05118}{Mapping Out the HPC Dependency Chaos}
	\item \href{https://arxiv.org/pdf/2405.00016}{HPX with Spack and Singularity Containers: Evaluating Overheads for HPX/Kokkos using an astrophysics application}
\end{enumerate}

\subsection{Spack Documentation}
\begin{enumerate}
	\item \href{https://dl-acm-org.libproxy.unm.edu/doi/10.1145/2807591.2807623}{The Spack package manager: bringing order to HPC software chaos}
	\item \href{https://computing.llnl.gov/projects/spack-hpc-package-manager}{Overview}
	\item \href{https://spack.readthedocs.io/en/latest/getting_started.html}{Getting Started}
	\item \href{https://spack-tutorial.readthedocs.io/en/latest/tutorial_basics.html}{Basic Installation}
	\item \href{https://spack.readthedocs.io/en/latest/basic_usage.html}{Basic Usage}
	\item \href{https://spack-tutorial.readthedocs.io/en/latest/}{Tutorial}
	\item \href{https://spack.readthedocs.io/en/latest/packaging_guide.html}{Packaging Guide}
	\item \href{https://spack.readthedocs.io/en/latest/}{Documentation Home}
	\item \href{https://github.com/spack/spacks}{GitHub Repo}
\end{enumerate}


\end{document} 