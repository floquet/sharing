\documentclass[10pt, oneside]{article}   	% use "amsart" instead of "article" for AMSLaTeX format
\usepackage{geometry}                		% See geometry.pdf to learn the layout options. There are lots.
\geometry{letterpaper}                   		% ... or a4paper or a5paper or ... 
%\geometry{landscape}                		% Activate for rotated page geometry
%\usepackage[parfill]{parskip}    		% Activate to begin paragraphs with an empty line rather than an indent
\usepackage{graphicx}				% Use pdf, png, jpg, or eps§ with pdflatex; use eps in DVI mode
								% TeX will automatically convert eps --> pdf in pdflatex		
\usepackage{amssymb}
\usepackage{hyperref}
\usepackage{xcolor}

\usepackage{listings}
	\definecolor{textblue}{rgb}{.2,.2,.7}
	\definecolor{textred}{rgb}{0.54,0,0}
	\definecolor{textgreen}{rgb}{0,0.43,0}

% input{./setup/macros}

% macros to simplfy typing and make the product more reliable
\newcommand{\emailTopa}[0]			{\href{mailto:daniel.topa@hii-tsd.com}{daniel.topa@hii-tsd.com}}

\newcommand{\textt}[1]				{{\footnotesize{\texttt{#1}}}}

\newcommand{\urlMan}[0]				{https://man7.org/linux/man-pages/man1/}

\newcommand{\ldd}[0]				{\textt{ldd}}
\newcommand{\urlLdd}[0]				{\urlMan ldd.1.html}
\newcommand{\refLdd}[0]				{\href{\urlLdd}{\ldd}}

\newcommand{\lddconfig}[0]			{\textt{lddconfig}}
\newcommand{\urlLddconfig}[0]			{\urlMan lddconfig.1.html}
\newcommand{\refLddconfig}[0]		{\href{\urlLddconfig}{\lddconfig}}

\newcommand{\lsof}[0]				{\textt{lsof}}
\newcommand{\urlLsof}[0]				{\urlMan lsof.1.html}
\newcommand{\refLsof}[0]				{\href{\urlLsof}{\lsof}}

\newcommand{\locate}[0]				{\textt{locate}}
\newcommand{\urlLocate}[0]			{\urlMan locate.1.html}
\newcommand{\refLocate}[0]			{\href{\urlLocate}{\locate}}

\newcommand{\nm}[0]				{\textt{nm}}
\newcommand{\urlNm}[0]				{\urlMan nm.1.html}
\newcommand{\refNm}[0]				{\href{\urlNm}{\nm}}

\newcommand{\objdump}[0]			{\textt{objdump}}
\newcommand{\urlObjdump}[0]			{\urlMan objdump.1.html}
\newcommand{\refObjdump}[0]			{\href{\urlObjdump}{\objdump}}

\newcommand{\readelf}[0]				{\textt{readelf}}
\newcommand{\urlReadelf}[0]			{\urlMan readelf.1.html}
\newcommand{\refReadelf}[0]			{\href{\urlReadelf}{\readelf}}

\newcommand{\strace}[0]				{\textt{strace}}
\newcommand{\urlStrace}[0]			{\urlMan strace.1.html}
\newcommand{\refStrace}[0]			{\href{\urlStrace}{\strace}}

\newcommand{\strings}[0]				{\textt{strings}}
\newcommand{\urlStrings}[0]			{\urlMan strings.1.html}
\newcommand{\refStrings}[0]			{\href{\urlStrings}{\strings}}

\newcommand{\elf}[0]				{\href{https://en.wikipedia.org/wiki/Executable_and_Linkable_Format}{ELF}}


\endinput  %  ==  ==  ==  ==  ==  ==  ==  ==  ==


% input{./setup/listing-bash}
% https://tex.stackexchange.com/questions/310335/using-bash-listings-to-bold-variables-and-functions

\usepackage[T1]{fontenc}
\usepackage{
  color,
  beramono,
  listings,
  textcomp
}

\definecolor{lightgray}{RGB}{245,245,245}
\definecolor{darkgray}{RGB}{128,128,128}

\lstset{
  abovecaptionskip={0cm},
  backgroundcolor={\color{lightgray}},
  basicstyle={\small\ttfamily},
  breakatwhitespace=true,
  breaklines=true,
  captionpos=b,
  frame=tb,
  resetmargins=true,
  sensitive=true,
  stepnumber=1,
  tabsize=4,
  upquote=true
}

\AtBeginDocument{\lstdefinelanguage{bash}[]{sh}%
  {morekeywords={alias,bg,bind,builtin,caller,command,compgen,compopt,%
      complete,coproc,curl,declare,disown,dirs,enable,fc,fg,help,%
      history,jobs,let,local,logout,mapfile,printf,pushd,popd,%
      readarray,select,set,suspend,shopt,source,times,type,typeset,%
      ulimit,unalias,wait},%
   otherkeywords={ [, ], [[, ]], \{, \} }%
  }%

\lstdefinelanguage{sh}%
  {morekeywords={awk,break,case,cat,cd,continue,do,done,echo,elif,else,%
      env,esac,eval,exec,exit,export,expr,false,fi,for,function,getopts,%
      hash,history,if,in,kill,login,newgrp,nice,nohup,ps,pwd,read,%
      readonly,return,set,sed,shift,test,then,times,trap,true,type,%
      ulimit,umask,unset,until,wait,while},%
   morecomment=[l]\#,%
   morestring=[d]",%
   alsoletter={*"'0123456789.},%
   alsoother={\{\=\}},%
   literate={{=}{{{=}}}1},%
   literate={\$\{}{{{{\bfseries{}\$\{}}}}2,%
   otherkeywords={ [, ], \{, \} }%
  }[keywords,comments,strings]%
}

\endinput  %  ==  ==  ==  ==  ==  ==  ==  ==  ==



\title{Package Managment With \texttt{Spack}}
%\author{Daniel Topa\\HII-TSD\\\href{mailto:daniel.topa@hii-tsd.com}{daniel.topa@hii-tsd.com}}
\author{Daniel Topa\\HII-TSD\\\href{mailto:daniel.topa@hii-tsd.com}{daniel.topa@hii-tsd.com}}
%\date{}							% Activate to display a given date or no date

\begin{document}
\maketitle
\abstract{Modern package managament. Unite under a compiler, a Python version, an MPI instance.}

\section{Getting Started}
Consider an example build of a powerful and complicated package, PETSc.

\begin{quote}
PETSc, the Portable, Extensible Toolkit for Scientific Computation,  is for the scalable (parallel) solution of scientific applications modeled by partial differential equations (PDEs). It has bindings for C, Fortran, and Python (via petsc4py). PETSc also contains TAO, the Toolkit for Advanced Optimization, software library. It supports MPI, and GPUs through CUDA, HIP, Kokkos, or OpenCL, as well as hybrid MPI-GPU parallelism; it also supports the NEC-SX Tsubasa Vector Engine
\end{quote}

\begin{enumerate}
	\item download spack
	\item initialize spack
	\item find compilers
	\item install PetSc
\end{enumerate}

\begin{verbatim}
$ git clone https://github.com/spack/spack.git
$ source spack/share/spack/setup-env.sh
$ spack compiler find
$ spack install petsc +fortran % gcc@14.2.0
\end{verbatim}

\subsection{How Does \spack \ Work?}
\spack \ is a \emph{download}, not an {installation}. It was born at Livermore out a desire to liberate the scientists from the HPC bureaucracy and allow them to build the packages and versions needed. It started a tool for people with local admin privledges over their machines, and is now a recognized tool used by the HPC support staff to maintain the shared environments.

\spack \ changes how developers interact with their uses. Instead of maintaining pages detailing install instructions for each hardware architecture and software enviroment, developers now maintain a single \spack \ instance and utilize the issue tracking inherent in \href{https://github.com/spack/spack}{\texttt{GitHub}}.


Build systems

In one sense, \spack \ is a database managing 
\small{
\begin{lstlisting}[language=bash]
kpex76l openmpi@1.10.7%gcc
vuijyrm     hwloc@1.11.13%gcc
vlgsd6a         libxml2@2.10.3%gcc
7ffbqyf             libiconv@1.17%gcc
cejtv5p             pkgconf@1.8.0%gcc
ydjmqn5             xz@5.4.1%gcc
kgdj2w7         ncurses@6.4%gcc
cbup2u4     openssh@9.1p1%gcc
74ofkad         krb5@1.20.1%gcc
gw3muwr             bison@3.8.2%gcc
mbfdcbq                 m4@1.4.19%gcc
ytuafo5                     libsigsegv@2.13%gcc
fx3kvo3             diffutils@3.8%gcc
g7g5rxm             gettext@0.21.1%gcc
pirykzv                 bzip2@1.0.8%gcc
lij4icg                 tar@1.34%gcc
3tfa2za                     pigz@2.7%gcc
hnuj2am                     zstd@1.5.2%gcc
mf4yylc         libedit@3.1-20210216%gcc
pnhvhts         libxcrypt@4.4.33%gcc
cck5u3i             perl@5.34.0%gcc
duhpddy         openssl@1.1.1t%gcc
syyclam             ca-certificates-mozilla@2023-01-10%gcc
ggaig6s         zlib@1.2.13%gcc
lxwy7gr     perl@5.34.0%gcc
kzdyfxk     pkgconf@1.8.0%gcc
\end{lstlisting}
}

\spack \ handles combinatorial complexity. For example, consider 4 compilers: Intel, GCC, PGI, NAG. For each compiler maintain 4 different versions; for example gcc 14.2.0, 13.3.0, 12.4.0, 4.8.5. Provide 4 MPI providers: OpenMPI, Cray-MPICH, MVAPICH, Intel-Parallel studio. Maintain 4 versions of each of those. Maintain 4 Python versions for each packages. This represents $4^{5} = 1024$ instances, handled by \spack. 

\subsection{\spack \ Users, Platforms}

\begin{enumerate}
	\item \href{https://ncar-hpc-docs.readthedocs.io/en/latest/environment-and-software/user-environment/spack/spack/}{NCAR} 
	\item \href{https://www.hpc.iastate.edu/guides/using-spack-to-build-packages}{Iowa State HPC}
	\item \href{https://github.com/HSF/hep-spack}{CERN}
	\item \hre{https://ipv6.rs/tutorial/Windows\_11/Spack/}{Windows 11}
	\item \href{https://ipv6.rs/tutorial/macOS/Spack/}{MacOS}
	\item \href{https://en.wikipedia.org/wiki/ARM_architecture_family}{ARM}
	\item \href{https://en.wikipedia.org/wiki/POWER8}{Power8, Power9}
	\item \href{https://en.wikipedia.org/wiki/X86-64}{x86-64}
	\item \href{https://en.wikipedia.org/wiki/IBM_Blue_Gene}{BlueGene}
\end{enumerate}

\section{Probe commands in \spack}

\subsection{\texttt{spack info petsc}}
The \texttt{spack} command \texttt{info} presents essential information about each package. The output starts with a brief descruption of the package and web site providing more information and a listing of available versions. Next is a list of variants and how to invoke them showing the user how to construct specific versions of the package -- which will all be managed by \texttt{spack}.  Users can easliy specify whether to use \texttt{C} or \texttt{C++} for the build, whether to use double or single precision, whether to use \texttt{MPI}\footnote{\spack \ allows users to chose between many flavors of \texttt{MPI}}, whether to use \openmp, and so on. The final sections lists dependencies for building, linking, and running. \spack \ will build these as needed.

	% % % \input{./components/bash/info-petsc}

%\lstdefinelanguage{Bash}%
%  {morekeywords={abstract,break,case,catch,const,continue,do,else,elseif,%
%      end,export,false,for,function,immutable,import,importall,if,in,%
%      macro,module,otherwise,quote,return,switch,true,try,type,typealias,%
%      using,while},%
%   sensitive=true,%
%   alsoother={$},%
%   morecomment=[l]\#,%
%   morecomment=[n]{\#=}{=\#},%
%   morestring=[s]{"}{"},%
%   morestring=[m]{'}{'},%
%}[keywords,comments,strings]%
%
%\lstset{%
%    language         = bash,
%    basicstyle       = \ttfamily,
%%    keywordstyle     = \bfseries\color{blue},
%%    stringstyle      = \color{magenta},
%%    commentstyle     = \color{ForestGreen},
%keywordstyle=\color{textblue},
%commentstyle=\color{textred},
%stringstyle=\color{textgreen},
%    showstringspaces = false,
%}

\small{
\begin{lstlisting}[language=bash]
$ spack info petsc
Package:   petsc

Description:
    PETSc is a suite of data structures and routines for the scalable
    (parallel) solution of scientific applications modeled by partial
    differential equations.

Homepage: https://petsc.org

Preferred version:  
    3.22.0    http://web.cels.anl.gov/projects/petsc/.../petsc-3.22.0.tar.gz

Safe versions:  
    main      [git] https://gitlab.com/petsc/petsc.git on branch main
    3.22.0    http://web.cels.anl.gov/projects/petsc/.../petsc-3.22.0.tar.gz
    3.21.6    http://web.cels.anl.gov/projects/petsc/.../petsc-3.21.6.tar.gz
    
    3.11.1    http://web.cels.anl.gov/projects/petsc/.../petsc-3.11.0.tar.gz

Deprecated versions:  
    None

Variants:
    X [false]                     false, true
        Activate X support
    batch [false]                 false, true
        Enable when mpiexec is not available to run binaries
    build_system [generic]        generic
        Build systems supported by the package
    cgns [false]                  false, true
        Activates support for CGNS (only parallel)
    clanguage [C]                 C, C++
        Specify C (recommended) or C++ to compile PETSc
    complex [false]               false, true
        Build with complex numbers
    cuda [false]                  false, true
        Build with CUDA
    debug [false]                 false, true
        Compile in debug mode
    double [true]                 false, true
        Switches between single and double precision
    exodusii [false]              false, true
        Activates support for ExodusII (only parallel)
    fftw [false]                  false, true
        Activates support for FFTW (only parallel)
    fortran [true]                false, true
        Activates fortran support
    giflib [false]                false, true
        Activates support for GIF
    hdf5 [true]                   false, true
        Activates support for HDF5 (only parallel)
    hpddm [false]                 false, true
        Activates support for HPDDM (only parallel)
    hwloc [false]                 false, true
        Activates support for hwloc
    hypre [true]                  false, true
        Activates support for Hypre (only parallel)
    int64 [false]                 false, true
        Compile with 64bit indices
    jpeg [false]                  false, true
        Activates support for JPEG
    knl [false]                   false, true
        Build for KNL
    kokkos [false]                false, true
        Activates support for kokkos and kokkos-kernels
    libpng [false]                false, true
        Activates support for PNG
    libyaml [false]               false, true
        Activates support for YAML
    memalign [none]               none, 16, 32, 4, 64, 8
        Specify alignment of allocated arrays
    memkind [false]               false, true
        Activates support for Memkind
    metis [true]                  false, true
        Activates support for metis and parmetis
    mkl-pardiso [false]           false, true
        Activates support for MKL Pardiso
    mmg [false]                   false, true
        Activates support for MMG
    moab [false]                  false, true
        Acivates support for MOAB (only parallel)
    mpfr [false]                  false, true
        Activates support for MPFR
    mpi [true]                    false, true
        Activates MPI support
    mumps [false]                 false, true
        Activates support for MUMPS (only parallel)
    openmp [false]                false, true
        Activates support for openmp
    p4est [false]                 false, true
        Activates support for P4Est (only parallel)
    parmmg [false]                false, true
        Activates support for ParMMG (only parallel)
    ptscotch [false]              false, true
        Activates support for PTScotch (only parallel)
    random123 [false]             false, true
        Activates support for Random123
    rocm [false]                  false, true
        Enable ROCm support
    saws [false]                  false, true
        Activates support for Saws
    shared [true]                 false, true
        Enables the build of shared libraries
    strumpack [false]             false, true
        Activates support for Strumpack
    suite-sparse [false]          false, true
        Activates support for SuiteSparse
    sycl [false]                  false, true
        Enable sycl build
    tetgen [false]                false, true
        Activates support for Tetgen
    trilinos [false]              false, true
        Activates support for Trilinos (only parallel)
    valgrind [false]              false, true
        Enable Valgrind Client Request mechanism
    zoltan [false]                false, true
        Activates support for Zoltan

    when +rocm
      amdgpu_target [none]        none, gfx1010, gfx1011, gfx1012, gfx1013, gfx1030, gfx1031, gfx1032, gfx1033, gfx1034, gfx1035, gfx1036, gfx1100, gfx1101, gfx1102, gfx1103, gfx701, gfx801, gfx802, gfx803, gfx900, gfx900:xnack-, gfx902, gfx904, gfx906, gfx906:xnack-, gfx908, gfx908:xnack-, gfx909, gfx90a, gfx90a:xnack+, gfx90a:xnack-, gfx90c, gfx940, gfx941, gfx942
          AMD GPU architecture

    when +cuda
      cuda_arch [none]            none, 10, 11, 12, 13, 20, 21, 30, 32, 35, 37, 50, 52, 53, 60, 61, 62, 70, 72, 75, 80, 86, 87, 89, 90, 90a
          CUDA architecture

    when +fortran
      scalapack [false]           false, true
          Activates support for Scalapack
      superlu-dist [true]         false, true
          Activates support for superlu-dist (only parallel)

Build Dependencies:
    blas  cuda       exodusii  giflib  gmp   hip      hipsolver  hsa-rocr-dev  hypre  kokkos          lapack  libx11   llvm-amdgpu  metis  mmg   mpfr  mumps     p4est            parmetis  python     rocblas    rocprim  rocsolver  rocthrust  scalapack  sowing     suite-sparse  tetgen    valgrind  zoltan
    cgns  diffutils  fftw      gmake   hdf5  hipblas  hipsparse  hwloc         jpeg   kokkos-kernels  libpng  libyaml  memkind      mkl    moab  mpi   netcdf-c  parallel-netcdf  parmmg    random123  rocm-core  rocrand  rocsparse  saws       scotch     strumpack  superlu-dist  trilinos  zlib-api

Link Dependencies:
    blas  cuda      fftw    gmake  hdf5  hipblas    hipsparse     hwloc  jpeg    kokkos-kernels  libpng  libyaml      memkind  mkl  moab  mpi    netcdf-c  parallel-netcdf  parmmg     rocblas    rocprim  rocsolver  rocthrust  scalapack  sowing     suite-sparse  tetgen    valgrind  zoltan
    cgns  exodusii  giflib  gmp    hip   hipsolver  hsa-rocr-dev  hypre  kokkos  lapack          libx11  llvm-amdgpu  metis    mmg  mpfr  mumps  p4est     parmetis         random123  rocm-core  rocrand  rocsparse  saws       scotch     strumpack  superlu-dist  trilinos  zlib-api

Run Dependencies:
    None

Licenses: 
    None
\end{lstlisting}
}
\endinput  %  ==  ==  ==  ==  ==  ==  ==  ==  ==


%\lstdefinelanguage{Bash}%
%  {morekeywords={abstract,break,case,catch,const,continue,do,else,elseif,%
%      end,export,false,for,function,immutable,import,importall,if,in,%
%      macro,module,otherwise,quote,return,switch,true,try,type,typealias,%
%      using,while},%
%   sensitive=true,%
%   alsoother={$},%
%   morecomment=[l]\#,%
%   morecomment=[n]{\#=}{=\#},%
%   morestring=[s]{"}{"},%
%   morestring=[m]{'}{'},%
%}[keywords,comments,strings]%
%
%\lstset{%
%    language         = bash,
%    basicstyle       = \ttfamily,
%%    keywordstyle     = \bfseries\color{blue},
%%    stringstyle      = \color{magenta},
%%    commentstyle     = \color{ForestGreen},
%keywordstyle=\color{textblue},
%commentstyle=\color{textred},
%stringstyle=\color{textgreen},
%    showstringspaces = false,
%}

\small{
\begin{lstlisting}[language=bash]
$ spack info petsc
Package:   petsc

Description:
    PETSc is a suite of data structures and routines for the scalable
    (parallel) solution of scientific applications modeled by partial
    differential equations.

Homepage: https://petsc.org

Preferred version:  
    3.22.0    http://web.cels.anl.gov/projects/petsc/.../petsc-3.22.0.tar.gz

Safe versions:  
    main      [git] https://gitlab.com/petsc/petsc.git on branch main
    3.22.0    http://web.cels.anl.gov/projects/petsc/.../petsc-3.22.0.tar.gz
    3.21.6    http://web.cels.anl.gov/projects/petsc/.../petsc-3.21.6.tar.gz
    
    3.11.1    http://web.cels.anl.gov/projects/petsc/.../petsc-3.11.0.tar.gz

Deprecated versions:  
    None

Variants:
    X [false]                     false, true
        Activate X support
    batch [false]                 false, true
        Enable when mpiexec is not available to run binaries
    build_system [generic]        generic
        Build systems supported by the package
    cgns [false]                  false, true
        Activates support for CGNS (only parallel)
    clanguage [C]                 C, C++
        Specify C (recommended) or C++ to compile PETSc
    complex [false]               false, true
        Build with complex numbers
    cuda [false]                  false, true
        Build with CUDA
    debug [false]                 false, true
        Compile in debug mode
    double [true]                 false, true
        Switches between single and double precision
    exodusii [false]              false, true
        Activates support for ExodusII (only parallel)
    fftw [false]                  false, true
        Activates support for FFTW (only parallel)
    fortran [true]                false, true
        Activates fortran support
    giflib [false]                false, true
        Activates support for GIF
    hdf5 [true]                   false, true
        Activates support for HDF5 (only parallel)
    hpddm [false]                 false, true
        Activates support for HPDDM (only parallel)
    hwloc [false]                 false, true
        Activates support for hwloc
    hypre [true]                  false, true
        Activates support for Hypre (only parallel)
    int64 [false]                 false, true
        Compile with 64bit indices
    jpeg [false]                  false, true
        Activates support for JPEG
    knl [false]                   false, true
        Build for KNL
    kokkos [false]                false, true
        Activates support for kokkos and kokkos-kernels
    libpng [false]                false, true
        Activates support for PNG
    libyaml [false]               false, true
        Activates support for YAML
    memalign [none]               none, 16, 32, 4, 64, 8
        Specify alignment of allocated arrays
    memkind [false]               false, true
        Activates support for Memkind
    metis [true]                  false, true
        Activates support for metis and parmetis
    mkl-pardiso [false]           false, true
        Activates support for MKL Pardiso
    mmg [false]                   false, true
        Activates support for MMG
    moab [false]                  false, true
        Acivates support for MOAB (only parallel)
    mpfr [false]                  false, true
        Activates support for MPFR
    mpi [true]                    false, true
        Activates MPI support
    mumps [false]                 false, true
        Activates support for MUMPS (only parallel)
    openmp [false]                false, true
        Activates support for openmp
    p4est [false]                 false, true
        Activates support for P4Est (only parallel)
    parmmg [false]                false, true
        Activates support for ParMMG (only parallel)
    ptscotch [false]              false, true
        Activates support for PTScotch (only parallel)
    random123 [false]             false, true
        Activates support for Random123
    rocm [false]                  false, true
        Enable ROCm support
    saws [false]                  false, true
        Activates support for Saws
    shared [true]                 false, true
        Enables the build of shared libraries
    strumpack [false]             false, true
        Activates support for Strumpack
    suite-sparse [false]          false, true
        Activates support for SuiteSparse
    sycl [false]                  false, true
        Enable sycl build
    tetgen [false]                false, true
        Activates support for Tetgen
    trilinos [false]              false, true
        Activates support for Trilinos (only parallel)
    valgrind [false]              false, true
        Enable Valgrind Client Request mechanism
    zoltan [false]                false, true
        Activates support for Zoltan

    when +rocm
      amdgpu_target [none]        none, gfx1010, gfx1011, gfx1012, gfx1013, gfx1030, gfx1031, gfx1032, gfx1033, gfx1034, gfx1035, gfx1036, gfx1100, gfx1101, gfx1102, gfx1103, gfx701, gfx801, gfx802, gfx803, gfx900, gfx900:xnack-, gfx902, gfx904, gfx906, gfx906:xnack-, gfx908, gfx908:xnack-, gfx909, gfx90a, gfx90a:xnack+, gfx90a:xnack-, gfx90c, gfx940, gfx941, gfx942
          AMD GPU architecture

    when +cuda
      cuda_arch [none]            none, 10, 11, 12, 13, 20, 21, 30, 32, 35, 37, 50, 52, 53, 60, 61, 62, 70, 72, 75, 80, 86, 87, 89, 90, 90a
          CUDA architecture

    when +fortran
      scalapack [false]           false, true
          Activates support for Scalapack
      superlu-dist [true]         false, true
          Activates support for superlu-dist (only parallel)

Build Dependencies:
    blas  cuda       exodusii  giflib  gmp   hip      hipsolver  hsa-rocr-dev  hypre  kokkos          lapack  libx11   llvm-amdgpu  metis  mmg   mpfr  mumps     p4est            parmetis  python     rocblas    rocprim  rocsolver  rocthrust  scalapack  sowing     suite-sparse  tetgen    valgrind  zoltan
    cgns  diffutils  fftw      gmake   hdf5  hipblas  hipsparse  hwloc         jpeg   kokkos-kernels  libpng  libyaml  memkind      mkl    moab  mpi   netcdf-c  parallel-netcdf  parmmg    random123  rocm-core  rocrand  rocsparse  saws       scotch     strumpack  superlu-dist  trilinos  zlib-api

Link Dependencies:
    blas  cuda      fftw    gmake  hdf5  hipblas    hipsparse     hwloc  jpeg    kokkos-kernels  libpng  libyaml      memkind  mkl  moab  mpi    netcdf-c  parallel-netcdf  parmmg     rocblas    rocprim  rocsolver  rocthrust  scalapack  sowing     suite-sparse  tetgen    valgrind  zoltan
    cgns  exodusii  giflib  gmp    hip   hipsolver  hsa-rocr-dev  hypre  kokkos  lapack          libx11  llvm-amdgpu  metis    mmg  mpfr  mumps  p4est     parmetis         random123  rocm-core  rocrand  rocsparse  saws       scotch     strumpack  superlu-dist  trilinos  zlib-api

Run Dependencies:
    None

Licenses: 
    None
\end{lstlisting}
}
\endinput  %  ==  ==  ==  ==  ==  ==  ==  ==  ==


%\lstdefinelanguage{Bash}%
%  {morekeywords={abstract,break,case,catch,const,continue,do,else,elseif,%
%      end,export,false,for,function,immutable,import,importall,if,in,%
%      macro,module,otherwise,quote,return,switch,true,try,type,typealias,%
%      using,while},%
%   sensitive=true,%
%   alsoother={$},%
%   morecomment=[l]\#,%
%   morecomment=[n]{\#=}{=\#},%
%   morestring=[s]{"}{"},%
%   morestring=[m]{'}{'},%
%}[keywords,comments,strings]%
%
%\lstset{%
%    language         = bash,
%    basicstyle       = \ttfamily,
%%    keywordstyle     = \bfseries\color{blue},
%%    stringstyle      = \color{magenta},
%%    commentstyle     = \color{ForestGreen},
%keywordstyle=\color{textblue},
%commentstyle=\color{textred},
%stringstyle=\color{textgreen},
%    showstringspaces = false,
%}

\small{
\begin{lstlisting}[language=bash]
$ spack info petsc
Package:   petsc

Description:
    PETSc is a suite of data structures and routines for the scalable
    (parallel) solution of scientific applications modeled by partial
    differential equations.

Homepage: https://petsc.org

Preferred version:  
    3.22.0    http://web.cels.anl.gov/projects/petsc/.../petsc-3.22.0.tar.gz

Safe versions:  
    main      [git] https://gitlab.com/petsc/petsc.git on branch main
    3.22.0    http://web.cels.anl.gov/projects/petsc/.../petsc-3.22.0.tar.gz
    3.21.6    http://web.cels.anl.gov/projects/petsc/.../petsc-3.21.6.tar.gz
    
    3.11.1    http://web.cels.anl.gov/projects/petsc/.../petsc-3.11.0.tar.gz

Deprecated versions:  
    None

Variants:
    X [false]                     false, true
        Activate X support
    batch [false]                 false, true
        Enable when mpiexec is not available to run binaries
    build_system [generic]        generic
        Build systems supported by the package
    cgns [false]                  false, true
        Activates support for CGNS (only parallel)
    clanguage [C]                 C, C++
        Specify C (recommended) or C++ to compile PETSc
    complex [false]               false, true
        Build with complex numbers
    cuda [false]                  false, true
        Build with CUDA
    debug [false]                 false, true
        Compile in debug mode
    double [true]                 false, true
        Switches between single and double precision
    exodusii [false]              false, true
        Activates support for ExodusII (only parallel)
    fftw [false]                  false, true
        Activates support for FFTW (only parallel)
    fortran [true]                false, true
        Activates fortran support
    giflib [false]                false, true
        Activates support for GIF
    hdf5 [true]                   false, true
        Activates support for HDF5 (only parallel)
    hpddm [false]                 false, true
        Activates support for HPDDM (only parallel)
    hwloc [false]                 false, true
        Activates support for hwloc
    hypre [true]                  false, true
        Activates support for Hypre (only parallel)
    int64 [false]                 false, true
        Compile with 64bit indices
    jpeg [false]                  false, true
        Activates support for JPEG
    knl [false]                   false, true
        Build for KNL
    kokkos [false]                false, true
        Activates support for kokkos and kokkos-kernels
    libpng [false]                false, true
        Activates support for PNG
    libyaml [false]               false, true
        Activates support for YAML
    memalign [none]               none, 16, 32, 4, 64, 8
        Specify alignment of allocated arrays
    memkind [false]               false, true
        Activates support for Memkind
    metis [true]                  false, true
        Activates support for metis and parmetis
    mkl-pardiso [false]           false, true
        Activates support for MKL Pardiso
    mmg [false]                   false, true
        Activates support for MMG
    moab [false]                  false, true
        Acivates support for MOAB (only parallel)
    mpfr [false]                  false, true
        Activates support for MPFR
    mpi [true]                    false, true
        Activates MPI support
    mumps [false]                 false, true
        Activates support for MUMPS (only parallel)
    openmp [false]                false, true
        Activates support for openmp
    p4est [false]                 false, true
        Activates support for P4Est (only parallel)
    parmmg [false]                false, true
        Activates support for ParMMG (only parallel)
    ptscotch [false]              false, true
        Activates support for PTScotch (only parallel)
    random123 [false]             false, true
        Activates support for Random123
    rocm [false]                  false, true
        Enable ROCm support
    saws [false]                  false, true
        Activates support for Saws
    shared [true]                 false, true
        Enables the build of shared libraries
    strumpack [false]             false, true
        Activates support for Strumpack
    suite-sparse [false]          false, true
        Activates support for SuiteSparse
    sycl [false]                  false, true
        Enable sycl build
    tetgen [false]                false, true
        Activates support for Tetgen
    trilinos [false]              false, true
        Activates support for Trilinos (only parallel)
    valgrind [false]              false, true
        Enable Valgrind Client Request mechanism
    zoltan [false]                false, true
        Activates support for Zoltan

    when +rocm
      amdgpu_target [none]        none, gfx1010, gfx1011, gfx1012, gfx1013, gfx1030, gfx1031, gfx1032, gfx1033, gfx1034, gfx1035, gfx1036, gfx1100, gfx1101, gfx1102, gfx1103, gfx701, gfx801, gfx802, gfx803, gfx900, gfx900:xnack-, gfx902, gfx904, gfx906, gfx906:xnack-, gfx908, gfx908:xnack-, gfx909, gfx90a, gfx90a:xnack+, gfx90a:xnack-, gfx90c, gfx940, gfx941, gfx942
          AMD GPU architecture

    when +cuda
      cuda_arch [none]            none, 10, 11, 12, 13, 20, 21, 30, 32, 35, 37, 50, 52, 53, 60, 61, 62, 70, 72, 75, 80, 86, 87, 89, 90, 90a
          CUDA architecture

    when +fortran
      scalapack [false]           false, true
          Activates support for Scalapack
      superlu-dist [true]         false, true
          Activates support for superlu-dist (only parallel)

Build Dependencies:
    blas  cuda       exodusii  giflib  gmp   hip      hipsolver  hsa-rocr-dev  hypre  kokkos          lapack  libx11   llvm-amdgpu  metis  mmg   mpfr  mumps     p4est            parmetis  python     rocblas    rocprim  rocsolver  rocthrust  scalapack  sowing     suite-sparse  tetgen    valgrind  zoltan
    cgns  diffutils  fftw      gmake   hdf5  hipblas  hipsparse  hwloc         jpeg   kokkos-kernels  libpng  libyaml  memkind      mkl    moab  mpi   netcdf-c  parallel-netcdf  parmmg    random123  rocm-core  rocrand  rocsparse  saws       scotch     strumpack  superlu-dist  trilinos  zlib-api

Link Dependencies:
    blas  cuda      fftw    gmake  hdf5  hipblas    hipsparse     hwloc  jpeg    kokkos-kernels  libpng  libyaml      memkind  mkl  moab  mpi    netcdf-c  parallel-netcdf  parmmg     rocblas    rocprim  rocsolver  rocthrust  scalapack  sowing     suite-sparse  tetgen    valgrind  zoltan
    cgns  exodusii  giflib  gmp    hip   hipsolver  hsa-rocr-dev  hypre  kokkos  lapack          libx11  llvm-amdgpu  metis    mmg  mpfr  mumps  p4est     parmetis         random123  rocm-core  rocrand  rocsparse  saws       scotch     strumpack  superlu-dist  trilinos  zlib-api

Run Dependencies:
    None

Licenses: 
    None
\end{lstlisting}
}
\endinput  %  ==  ==  ==  ==  ==  ==  ==  ==  ==




\end{document} 