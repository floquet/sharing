% typeset: Pdftex
% Afterwards compile with pdflatex > bibtex > pdflatex > pdflatex.
% in TeXShop preferences, changed edit from bibtex to biber
% beamer likes biber
% latex likes bibtex

\documentclass[10pt, oneside]{article}   	% use "amsart" instead of "article" for AMSLaTeX format
\usepackage{geometry}                		% See geometry.pdf to learn the layout options. There are lots.
\geometry{letterpaper}                   		% ... or a4paper or a5paper or ... 
%\geometry{landscape}                		% Activate for rotated page geometry
\usepackage[parfill]{parskip}    		% Activate to begin paragraphs with an empty line rather than an indent
\usepackage{graphicx}				% Use pdf, png, jpg, or eps§ with pdflatex; use eps in DVI mode
								% TeX will automatically convert eps --> pdf in pdflatex		
\usepackage{amssymb}
\usepackage{fancyvrb}
\usepackage{hyperref}
\usepackage{overpic}
\usepackage{pdfpages}
\usepackage{xcolor}
\usepackage[backend=bibtex,bibencoding=ascii,style=authoryear,sorting=none]{biblatex}
\bibliography{./sections/radar.bib}
%\addbibresource{radar.bib}
%\bibliography{\pSections radar.bib}

% input{./setup/macros}

% macros to simplfy typing and make the product more reliable
\newcommand{\emailTopa}[0]			{\href{mailto:daniel.topa@hii-tsd.com}{daniel.topa@hii-tsd.com}}

\newcommand{\textt}[1]				{{\footnotesize{\texttt{#1}}}}

\newcommand{\urlMan}[0]				{https://man7.org/linux/man-pages/man1/}

\newcommand{\ldd}[0]				{\textt{ldd}}
\newcommand{\urlLdd}[0]				{\urlMan ldd.1.html}
\newcommand{\refLdd}[0]				{\href{\urlLdd}{\ldd}}

\newcommand{\lddconfig}[0]			{\textt{lddconfig}}
\newcommand{\urlLddconfig}[0]			{\urlMan lddconfig.1.html}
\newcommand{\refLddconfig}[0]		{\href{\urlLddconfig}{\lddconfig}}

\newcommand{\lsof}[0]				{\textt{lsof}}
\newcommand{\urlLsof}[0]				{\urlMan lsof.1.html}
\newcommand{\refLsof}[0]				{\href{\urlLsof}{\lsof}}

\newcommand{\locate}[0]				{\textt{locate}}
\newcommand{\urlLocate}[0]			{\urlMan locate.1.html}
\newcommand{\refLocate}[0]			{\href{\urlLocate}{\locate}}

\newcommand{\nm}[0]				{\textt{nm}}
\newcommand{\urlNm}[0]				{\urlMan nm.1.html}
\newcommand{\refNm}[0]				{\href{\urlNm}{\nm}}

\newcommand{\objdump}[0]			{\textt{objdump}}
\newcommand{\urlObjdump}[0]			{\urlMan objdump.1.html}
\newcommand{\refObjdump}[0]			{\href{\urlObjdump}{\objdump}}

\newcommand{\readelf}[0]				{\textt{readelf}}
\newcommand{\urlReadelf}[0]			{\urlMan readelf.1.html}
\newcommand{\refReadelf}[0]			{\href{\urlReadelf}{\readelf}}

\newcommand{\strace}[0]				{\textt{strace}}
\newcommand{\urlStrace}[0]			{\urlMan strace.1.html}
\newcommand{\refStrace}[0]			{\href{\urlStrace}{\strace}}

\newcommand{\strings}[0]				{\textt{strings}}
\newcommand{\urlStrings}[0]			{\urlMan strings.1.html}
\newcommand{\refStrings}[0]			{\href{\urlStrings}{\strings}}

\newcommand{\elf}[0]				{\href{https://en.wikipedia.org/wiki/Executable_and_Linkable_Format}{ELF}}


\endinput  %  ==  ==  ==  ==  ==  ==  ==  ==  ==




\title{Simulation of \\Radar Profiles for Satellites \\Using Mercury Method of Moments}
\author{Daniel Topa\\HII-TSD\\\href{mailto:daniel.topa@hii-tsd.com}{daniel.topa@hii-tsd.com}}

\begin{document}
\maketitle
\abstract{A brief survey of characterizing the three dimensional radar cross section of satellites.}
\tableofcontents

\section{Overview}
\cite{topa20200303}
Working with CAF files, producing output, compressing data.
\cite{topa-4-20-2024}
\cite{topa-4-20-2024}
The goal is to be able to resolve the workings of an executable file exploiting the \elf \ structure show in figures \ref{fig:elf}. The next figure, \ref{fig:elf-II}, shows the relationship between source files, header files, shared objects, and the executable program.

\section{Mercury Method of Moments}
\subsection{Copyright Statement by the Author}
\begin{quotation}
{\footnotesize{
\noindent
==================================================\\
    MERCURY MOM(TM) ( Copyrighted and Patents Issued) \\
	MATRIX COMPRESSION TECHNOLOGIES, LLC \\[10pt]
	For licensing information contact: \\
	John Shaeffer\\
	3278 Hunterdon Way\\
	Marietta, Georgia 30067\\
	770.952.3678 \\	
	Copyright 2006 Matrix Compression Technologies, LLC.\\	[10pt]
	This software was developed under NASA Contracts NAS1-02057, NAS1-02117, 
	NNL08AA00B, and NNL13AA08B, and the U.S. Government retains certain rights.\\[10pt]
	The Government, and others acting on its behalf, retain a paid-up, 
	nonexclusive, irrevocable, worldwide license to reproduce, prepare 
	derivative works, and perform publicly and display publicly (but not to 
	distribute copies to the public) by or on behalf of the Government, without
	any obligation of confidentiality on the part of the U.S. Government. Such
	license extends to use by NASA contractors, and others working under 
	agreements with the U.S. Government; provided that use of the software shall
	not be allowed to any person or entity where such use is not in direct 
	performance of a contract with the United States; and provided that such use
	is not for internal research and development by the contractor or others 
	that is not directly funded by the United States.
==================================================
}}
\end{quotation}

\subsection{Legal Statement}
\includepdf[pages={-}, frame=true, scale=0.8]{./local/pdf/"MERCURY MOM Legal Notice"}

\subsection{Obtaining Software and Documentation}
\begin{figure}[htbp]
\begin{center}
	\includegraphics[ width = 3in ]{./local/graphics-local/"Hg Mom"}
\caption{Contact information to request Mercury MoM Software and Documentations}
\label{fig:kam-miller}
\end{center}
\end{figure}

\subsection{Distribution Contents}
\subsubsection{Executables}
\begin{enumerate}
	\item Linux 64-bit
	\item Windows 64-bit
\end{enumerate}

\subsubsection{Documentation}
The disctirubtion includes four documents in PDF:
\begin{enumerate}
	\item User's Guide
	\item Pill Tutorial
	\item Code Validation Report
	\item Benchmark Tests
\end{enumerate}

\subsection{YouTube Videos}
One can find useful didactic presentations and simulations on YouTube.
\begin{enumerate}
	\item \href{https://www.youtube.com/watch?v=ujyoJSzwmQw}{The Radar cross-section of backscattering objects}
	\item \href{https://www.youtube.com/watch?v=0g5x4pXBid8}{Basic Concepts of Radar Cross Section (RCS)}
	\item \href{https://www.youtube.com/watch?v=mM-QDN68ebc}{Mie scattering}
	\item \href{https://www.youtube.com/watch?v=ayI6W6-ypUM&list=PLzD7pNQo-MGzkBnp1HVTGXaIQzWvkJ0M8}{Mie theory (BME51 Lecture 5)}
	\item \href{https://www.youtube.com/shorts/ggMoo8wH1_o}{Mie Scattering}
\end{enumerate}


\section{Command Examples}
		% % % \input{./sections/ssec-ldd}

\subsection{\ldd}
\label{sec:ldd}
The command \refLdd \ prints shared object dependencies, in this example, for the executable \texttt{bash}:
{\footnotesize{
\begin{Verbatim}[commandchars=\\\{\}]
{\color{darkgray}{root@69cb14a32689:/}}# ldd /bin/bash
{\color{darkgray}{	linux-vdso.so.1 (0x00007ffe64317000)}}
{\color{darkgray}{	libtinfo.so.6 }{\color{blue}{=>}}\color{darkgray}{ /lib/x86_64-linux-gnu/libtinfo.so.6 (0x00007f842112d000)}}
{\color{darkgray}{	libc.so.6 }{\color{blue}{=>}}\color{darkgray}{ /lib/x86_64-linux-gnu/libc.so.6 (0x00007f8420f04000)}}
{\color{darkgray}{	/lib64/ld-linux-x86-64.so.2 (0x00007f84212e3000)}}
\end{Verbatim}
}}
\href{https://en.wikipedia.org/wiki/Symbolic_link}{Symbolic link}s (symlinks) are highlighted with blue color.

\endinput  %  ==  ==  ==  ==  ==  ==  ==  ==  ==


\subsection{\ldd}
\label{sec:ldd}
The command \refLdd \ prints shared object dependencies, in this example, for the executable \texttt{bash}:
{\footnotesize{
\begin{Verbatim}[commandchars=\\\{\}]
{\color{darkgray}{root@69cb14a32689:/}}# ldd /bin/bash
{\color{darkgray}{	linux-vdso.so.1 (0x00007ffe64317000)}}
{\color{darkgray}{	libtinfo.so.6 }{\color{blue}{=>}}\color{darkgray}{ /lib/x86_64-linux-gnu/libtinfo.so.6 (0x00007f842112d000)}}
{\color{darkgray}{	libc.so.6 }{\color{blue}{=>}}\color{darkgray}{ /lib/x86_64-linux-gnu/libc.so.6 (0x00007f8420f04000)}}
{\color{darkgray}{	/lib64/ld-linux-x86-64.so.2 (0x00007f84212e3000)}}
\end{Verbatim}
}}
\href{https://en.wikipedia.org/wiki/Symbolic_link}{Symbolic link}s (symlinks) are highlighted with blue color.

\endinput  %  ==  ==  ==  ==  ==  ==  ==  ==  ==


\subsection{\ldd}
\label{sec:ldd}
The command \refLdd \ prints shared object dependencies, in this example, for the executable \texttt{bash}:
{\footnotesize{
\begin{Verbatim}[commandchars=\\\{\}]
{\color{darkgray}{root@69cb14a32689:/}}# ldd /bin/bash
{\color{darkgray}{	linux-vdso.so.1 (0x00007ffe64317000)}}
{\color{darkgray}{	libtinfo.so.6 }{\color{blue}{=>}}\color{darkgray}{ /lib/x86_64-linux-gnu/libtinfo.so.6 (0x00007f842112d000)}}
{\color{darkgray}{	libc.so.6 }{\color{blue}{=>}}\color{darkgray}{ /lib/x86_64-linux-gnu/libc.so.6 (0x00007f8420f04000)}}
{\color{darkgray}{	/lib64/ld-linux-x86-64.so.2 (0x00007f84212e3000)}}
\end{Verbatim}
}}
\href{https://en.wikipedia.org/wiki/Symbolic_link}{Symbolic link}s (symlinks) are highlighted with blue color.

\endinput  %  ==  ==  ==  ==  ==  ==  ==  ==  ==



\section{Further Reading}
Radar rudiments
\begin{enumerate}
	\item \cite{peebles2007radar}
	\item \cite{Handbook}
	\item \cite{kolosov1987}
\end{enumerate}
Radar cross section
\begin{enumerate}
	\item \cite{yuan2009efficient}
	\item \cite{fuhs1982radar}
	\item \cite{knott2004radar}
	\item \cite{crispin2013methods}
	\item \cite{madheswaran2012estimation}
\end{enumerate}
Method of Moments
\begin{enumerate}
	\item \cite{gibson2021method}
	\item \cite{lu2003comparison}
	\item \cite{yuan2009efficient}
\end{enumerate}
Mercury MoM
\begin{enumerate}
	\item \cite{Topa-2020-07-07}
	\item \cite{lu2003comparison}
	\item \cite{yuan2009efficient}
\end{enumerate}

	\printbibliography
%\nocite
%\printbibliography
%\bibliographystyle{thisjournal}
%\include{bibliography} 
%\bibliographystyle{vancouver}
%\bibliography{MonthlyReferences.bib}
%\bibliography{triangulation.bib}
%\bibliography{triangulation.bib}


\end{document} 

\tiny
\scriptsize
\footnotesize
\small
\normalsize
\large
\Large
\huge
\Huge