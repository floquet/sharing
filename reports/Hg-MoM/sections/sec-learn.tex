% \input{\pSections "sec-learn"}

\section{Additional Information}

\subsection{YouTube Videos}
YouTube offers useful didactic presentations and simulations.
\begin{enumerate}
	\item \href{https://www.youtube.com/watch?v=ujyoJSzwmQw}{The Radar cross-section of backscattering objects}
	\item \href{https://www.youtube.com/watch?v=0g5x4pXBid8}{Basic Concepts of Radar Cross Section (RCS)}
	\item \href{https://www.youtube.com/watch?v=mM-QDN68ebc}{Mie scattering}
	\item \href{https://www.youtube.com/watch?v=ayI6W6-ypUM&list=PLzD7pNQo-MGzkBnp1HVTGXaIQzWvkJ0M8}{Mie theory (BME51 Lecture 5)}
	\item \href{https://www.youtube.com/shorts/ggMoo8wH1_o}{Mie Scattering}
\end{enumerate}

\section{Further Reading}
Radar rudiments
\begin{enumerate}
	\item \cite{peebles2007radar}
	\item \cite{Handbook}
	\item \cite{kolosov1987}
\end{enumerate}
Radar cross section
\begin{enumerate}
	\item \cite{yuan2009efficient}
	\item \cite{fuhs1982radar}
	\item \cite{knott2004radar}
	\item \cite{crispin2013methods}
	\item \cite{madheswaran2012estimation}
\end{enumerate}
Method of Moments
\begin{enumerate}
	\item \cite{harrington1987method}
	\item \cite{gibson2021method}
	\item \cite{lu2003comparison}
	\item \cite{yuan2009efficient}
\end{enumerate}
Mercury MoM
\begin{enumerate}
	\item \cite{Topa-2020-07-07}
	\item \cite{lu2003comparison}
	\item \cite{yuan2009efficient}
\end{enumerate}

\endinput  %  ==  ==  ==  ==  ==  ==  ==  ==  ==