% \input{\pSections "ssec-rcsharvester"}

\subsection{\texttt{rcsharvester.f08}}
\footnotesize{
\texttt{! harvest the electric field values from the ASCII file *.4112.txt mixed text and numeric lines}\\
\texttt{! compute the mean total RCS and write these values}\\
}

\subsubsection{Class Electric Fields: \texttt{m-class-electric-fields.f08}}

\subsubsection{\texttt{m-class-electric-fields.f08}}
The primary output of the simulation are the electric fields. Lines 17-24 define the class; the remainder of the codes is for methods. The input electric field is resolved into two polarization axes: horizontal and vertical. Each of these fields are resolved into horizontal and vertical components creating four complex vectors (line 21) whose length matches the angular sample size.

The class \texttt{m-class-electric-fields.f08} reads the text file and harvests the electric fields eventually passing back a composite value (lines 65-66) for all four components of the scattering return.
		\input{\pLocalMoM/code/"mClassElectricField"}



\subsubsection{Class Data File: \texttt{m-class-data-file.f08}}
		\input{\pLocalMoM/code/"mClassDataFile"}

\endinput  %  ==  ==  ==  ==  ==  ==  ==  ==  ==
