% \input{\pSections "sec-precis"}

\section{Mathematical Prec\'is}

%     %     %     %     %     %     %     %     %
\subsection{Models of Radar Cross Section}

		%
\begin{equation}
	\sigma_{\nu}\paren{\alpha}  \approx \dfrac{a_{0}}{2} + \sum_{k=1}^{\bl{d}}a_{k}\cos k\alpha + b_{k}\sin k\alpha
\label{eq:fourier}
\end{equation}
		%
Amplitudes and Errors for $\nu=3$ MHz and $d=7$:
		%
	\begin{equation*}
		\begin{split}
			\sigma_{3}\paren{\theta} = a_{0} &+  a_{1}\cos \theta  +  a_{2}\cos 2\theta +  a_{3}\cos 3\theta\\
				& +  a_{4}\cos 4\theta +  a_{5}\cos 5\theta+  a_{6}\cos 6\theta +  a_{7}\cos 7\theta
		\end{split}
	\end{equation*}	
		%
	\begin{equation*}
		\begin{split}
			\sigma_{3}\paren{\theta} &= 35.237 \pm 0.012   +  (1.675 \pm 0.018) \cos \theta  +  (-3.434 \pm 0.018) \cos 2\theta  +  (-0.866 \pm 0.018) \cos 3\theta   \\
			&+  (5.386 \pm 0.018) \cos 4\theta  +  (-1.280 \pm 0.018) \cos 5\theta  +  (1.379 \pm 0.018) \cos 6\theta +   (-0.675 \pm 0.018) \cos 7\theta
		\end{split}
	\end{equation*}	
		%
	\input{\pSections "sec-plots"}

%     %     %     %     %     %     %     %     %
\subsection{Running the Code}
%\texttt{./MMoM\textbackslash4.1.12 ${geofilename}}

\begin{verbatim}
	./MMoM_4.1.12 sample.geo
\end{verbatim}

%     %     %     %     %     %     %     %     %
\subsection{Radar}

\autocite{topa20200303}
\parencite{topa20200303}
Working with CAF files, producing output, compressing data.
\cite{topa-4-20-2024}
\textcite{topa-4-20-2024}


%     %     %     %     %     %     %     %     %
\subsection{Process}
\begin{table}[htp]
\begin{center}
\begin{tabular}{lll}
	1 & Create CAD model	& CAD software \\
	2 & Convert CAD to \texttt{*.obj}	& CAD software \\
	3 & Convert \texttt{*.obj} to \texttt{*.facet} & Mathematica, Fortran \\
	4 & Input  properties to \texttt{materials.lib}	& \texttt{VIM} \\
	5 & Set radar frequencies & \texttt{VIM} \\
	6 & \bf{Simulate radar irradiation} & \texttt{Mercury MoM} \\
	7 & Harvest reflection values from output & Mathematica, Fortran, Python \\
	8 & Describe RCS as a series of amplitudes & Not written
\end{tabular}
\end{center}
\caption{Start with a CAD model and construct a Radar Cross Section model}
\label{tab:process}
\end{table}%



\endinput  %  ==  ==  ==  ==  ==  ==  ==  ==  ==
