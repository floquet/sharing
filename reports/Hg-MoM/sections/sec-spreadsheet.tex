% \input{\pSections "sec-spreadsheet"}

\section{Using Python to Create Spreadsheets}

Some users preferred to digest the radar data in spreadsheet form. The Python toolkit \href{https://pypi.org/project/XlsxWriter/}{xlswriter} simplifies moving the data into \texttt{*.xlsx} form.

The spreadsheets eschew row-column notation (e.g. \texttt{A4}) and makes use of variables and named ranges to simplify the interpretation of the results.

%     %     %     %     %     %     %     %     %
\subsection{Inputs}

%     %     %     %     %     %     %     %     %
\subsubsection{Main}
Main module: 
		\input{\pLocalMoM/code/"MoM"}

%     %     %     %     %     %     %     %     %
\subsubsection{Class Test Object}	
Radar return data.
		\input{\pLocalMoM/code/"TestObject"}

%     %     %     %     %     %     %     %     %
\subsubsection{Excel Details}	
Toolkit for writing to spreadsheets.
		\input{\pLocalMoM/code/"tools"}

%     %     %     %     %     %     %     %     %
\subsection{Outputs}
The main module arranges the output data by creating a tab for each separate radar frequency in the range $3-30$ MHz. Other tabs contain aggregate information and diagnostics.
		\input{\pLocalMoM/code/"py-main"}

\endinput  %  ==  ==  ==  ==  ==  ==  ==  ==  ==
