% \input{\pSections "sec-gmat"}

\section{Overview: Modeling Radar Cross Section}

%     %     %     %     %     %     %     %     %
\subsection{Radar}
Wave speed equation
		%
\begin{equation}
	\lambda \nu = c
\label{eq:wave-speed}
\end{equation}
		%

\begin{table}[htp]
\begin{center}
\begin{tabular}{lcc
}
	%
	band & \multicolumn{1}{c}{$\nu$} & \multicolumn{1}{c}{$\lambda$} \\\hline
	%
	HF & $3-30$ MHz & $10-1$ m\\
	UHF & $30-300$ MHz & $0.1 - 0.01$ m\\
	VHF & $300 - 1000$ MHz & $0.01 - 0.03$ m\\
	L & $1 - 2$ GHz & $30 - 15$ mm \\
	S & $2 - 4$ GHz & $15 - 7.5$ mm  \\
	C & $4 - 8$ GHz & $7.5 - 3.7$ mm  \\
	X & $8 - 12$ GHz & $3.7 - 2.5$ mm  \\
	Ku & $12 - 18$ GHz & $2.5 - 1.7$ mm  \\
	K & $18 - 27$ GHz & $1.7 - 1.1$ mm  \\
	Ka & $27 - 40$ GHz & $1.1 - 0.75$ mm  \\
	V & $40 - 75$ GHz & $0.75 - 0.4$ mm  \\
	W & $75 - 110$ GHz & $0.4 - 0.27$ mm  \\
	mm & $110 - 300$ GHz & $0.27 - 0.1$ mm  \\
	%
\end{tabular}
\caption{IEEE Standard Designations for Radar Bands (\cite{bruder2003ieee}).}
\end{center}
\label{tab:wavespeed}
\end{table}%


\begin{enumerate}[label=(\Alph*)]
	\item Build a CAD model of the satellite \texttt(*.cad)
	\item Seal the CAD mesh
	\item Create geometry file \texttt(*.geo)
	\item Irradiate object with Mercury MoM
	\item Harvest backscatter
	\item Construct RCS
	\item Resolve RCS measurements into spherical harmonics
\end{enumerate}

%     %     %     %     %     %     %     %     %
\subsection{About}

\begin{enumerate}[label=(\Alph*)]
	\item Build a CAD model of the satellite \texttt(*.cad)
	\item Seal the CAD mesh
	\item Create geometry file \texttt(*.geo)
	\item Irradiate object with Mercury MoM
	\item Harvest backscatter
	\item Construct RCS
	\item Resolve RCS measurements into spherical harmonics
\end{enumerate}


\endinput  %  ==  ==  ==  ==  ==  ==  ==  ==  ==
