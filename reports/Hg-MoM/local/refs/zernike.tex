\documentclass[12pt]{article}
\usepackage{amsmath, amssymb, amsfonts}
\usepackage{graphicx}
\usepackage{hyperref}
\usepackage{geometry}
\geometry{a4paper, margin=1in}

\title{Zernike Decomposition of the Tophat Function}
\author{}
\date{\today}

\begin{document}
\maketitle

\section{Problem Setup}
The tophat function \( g(r) \) is defined as:
\begin{equation}
g(r) = 
\begin{cases} 
0 & \text{if } 0 \leq \abs{r} < \frac{1}{2}, \\
1 & \text{if } \frac{1}{2} \leq \abs{r} \leq 1, \\
\end{cases}
\end{equation}

The Tophat function, f(r)f(r), is a symmetric piecewise-defined function closely related to the Heaviside step function. It is non-zero within a specific radius and zero outside, representing a 1D analog of a 2D tophat profile.

\begin{center}
\textbf{Achates' idea for improvement: Add a plot of \( g(r) \) generated in Mathematica.}
\end{center}

The goal is to approximate \( g(r) \) using a series of rotationally invariant Zernike polynomials:
\begin{equation}
f(r) = a_{0,0} Z_{0,0}(r) + a_{2,0} Z_{2,0}(r) + a_{4,0} Z_{4,0}(r) + \dots
\end{equation}
where \( Z_{k,0}(r) \) are the radial Zernike polynomials with even indices \( k \), and \( a_{k,0} \) are the amplitudes.

\section{Objective Function}
To find the amplitudes \( a_{k,0} \), we minimize the least-squares difference:
\begin{equation}
\int_{0}^{1} \left( g(r) - f(r) \right)^2 r \, dr.
\end{equation}
Expanding \( f(r) \) in terms of the Zernike polynomials, we have:
\begin{equation}
f(r) = \sum_{k=0, k \text{ even}}^{\infty} a_{k,0} Z_{k,0}(r).
\end{equation}

\begin{center}
\textbf{Achates' idea for improvement: Add an explanation of why Zernike polynomials are suitable for this problem.}
\end{center}

\section{Amplitude Calculation}
The amplitudes \( a_{k,0} \) are computed by:
\begin{equation}
a_{k,0} = \frac{\int_{0}^{1} g(r) Z_{k,0}(r) r \, dr}{\int_{0}^{1} Z_{k,0}(r)^2 r \, dr}.
\end{equation}
This formula minimizes the least-squares error in the approximation.

\begin{center}
\textbf{Achates' idea for improvement: Add a plot illustrating the first few Zernike radial polynomials.}
\end{center}

\section{Future Work}
\begin{itemize}
    \item Extend the decomposition to higher-order terms and evaluate convergence.
    \item Explore 2D Zernike polynomials for functions defined over a disk.
    \item Add numerical results and graphs showing the approximation of \( g(r) \).
\end{itemize}

\section{Conclusion}
The Zernike decomposition provides a systematic way to approximate rotationally symmetric functions like the tophat function. By computing the amplitudes \( a_{k,0} \), we achieve an efficient representation using a compact basis of polynomials.

\begin{center}
\textbf{Achates' idea for improvement: Add numerical results and discuss error analysis.}
\end{center}

\end{document}
