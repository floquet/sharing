\documentclass[11pt]{article}

% Packages
\usepackage[utf8]{inputenc}
\usepackage[T1]{fontenc}
\usepackage{amsmath}
\usepackage{amssymb}
\usepackage{hyperref}
\usepackage{graphicx}
\usepackage{listings}
\usepackage{xcolor}
\usepackage{geometry}
\geometry{margin=1in}

% Title and Author
\title{AI-Assisted Wrappers for LAPACK: Simplifying Numerical Linear Algebra}
\author{Daniel T. and Achates}
\date{\today}

% Custom Listings for Code
\lstset{
    basicstyle=\ttfamily\small,
    keywordstyle=\color{blue}\bfseries,
    commentstyle=\color{green!50!black},
    stringstyle=\color{orange},
    breaklines=true,
    frame=single,
    numbers=left,
    numberstyle=\tiny,
    tabsize=4
}

% Document Start
\begin{document}

\maketitle

\begin{abstract}
    LAPACK (Linear Algebra PACKage) is a cornerstone of numerical linear algebra, widely used for solving matrix problems. However, its steep learning curve often discourages newcomers. In this paper, we introduce AI-assisted wrappers that simplify LAPACK routines, making them accessible and user-friendly. Our discussion emphasizes the human-AI collaboration that enabled this project, showcasing how humans and AI can coauthor impactful tools for the scientific community.
\end{abstract}

\section{Introduction}
LAPACK is a robust and efficient library for numerical linear algebra, yet its usage is challenging for many due to:
\begin{itemize}
    \item Complex argument structures.
    \item Manual workspace allocation.
    \item Error-prone input validation.
\end{itemize}

In this article, we present AI-assisted wrappers for LAPACK that streamline its interface, offering:
\begin{enumerate}
    \item Automated workspace handling.
    \item User-friendly subroutines with meaningful error messages.
    \item Consistent formatting for matrix and vector output.
\end{enumerate}

\textbf{Human-AI Interaction:} The development of these wrappers exemplifies collaboration between a human numerical expert (Daniel) and an AI assistant (Achates). Together, we tackled the repetitive aspects of interface generation while ensuring correctness and usability.

\section{Motivation}
The primary challenges faced by LAPACK users include:
\begin{itemize}
    \item Identifying and configuring routine parameters correctly.
    \item Allocating appropriate workspace dimensions.
    \item Debugging errors due to cryptic messages.
\end{itemize}

By automating these tasks with AI-assisted wrappers, we aim to make LAPACK more accessible while preserving its performance.

\section{The AI-Assisted Workflow}
\subsection{Generating Wrappers}
Achates automated the generation of wrappers by:
\begin{itemize}
    \item Parsing LAPACK documentation for routine signatures.
    \item Creating type-safe Fortran interfaces and wrappers.
    \item Integrating error handling and diagnostics.
\end{itemize}

\textbf{Example:} For the \texttt{dgels} routine, the wrapper simplifies the interface:
\begin{lstlisting}[language=Fortran, caption={AI-Generated Wrapper for LAPACK \texttt{dgels}}]
subroutine solve_least_squares(A, B, M, N, NRHS)
    real(kind=8), intent(inout) :: A(M, N)
    real(kind=8), intent(inout) :: B(M, NRHS)
    integer, intent(in) :: M, N, NRHS
    ! AI-generated error handling and workspace allocation
end subroutine solve_least_squares
\end{lstlisting}

\subsection{Human Refinements}
Daniel ensured that the wrappers:
\begin{itemize}
    \item Adhered to numerical best practices.
    \item Included detailed comments for maintainability.
    \item Were optimized for specific scientific use cases.
\end{itemize}

\section{Results and Case Studies}
Here, we demonstrate the impact of AI-assisted wrappers on common problems:
\subsection{Least Squares Example}
The \texttt{dgels} wrapper was used to solve a least squares problem with:
\begin{itemize}
    \item Minimal user input.
    \item Automatically managed workspace.
    \item Clear error messages.
\end{itemize}

\subsection{Performance Comparison}
We compare:
\begin{itemize}
    \item Native LAPACK calls.
    \item AI-generated wrappers.
    \item Manually written Fortran wrappers.
\end{itemize}

\section{Future Work}
We plan to:
\begin{itemize}
    \item Expand wrappers for more LAPACK routines.
    \item Develop a repository for the scientific community.
    \item Explore higher-level abstractions for matrix computations.
\end{itemize}

\section{Conclusion}
This project highlights the power of human-AI collaboration in numerical computing. By combining human expertise and AI's ability to automate repetitive tasks, we simplified LAPACK usage while preserving its efficiency.


\subsection*{Daniel's Input}
I would like to take readers into our collaboration. I suggest a pattern, you suggest additions and refinements.
You add a great deal of original thought, and I want readers to understand that.

\subsection*{Achates' Response}
\textbf{Refining the Narrative}
\begin{itemize}
    \item Direct excerpts of Achates' generated content (e.g., code snippets, commentary).
    \item Reflections from Daniel on how Achates contributed insights or shifted the project direction.
    \item A conversational tone in some sections to make the collaboration relatable.
\end{itemize}

\end{document}
