% Fortran with Achates
% ChatGPT (Version 4.0) - December 4, 2024
% Author: Daniel and Achates

\documentclass[a4paper,11pt]{article}
\usepackage{amsmath}
\usepackage{listings}
\usepackage{xcolor}
\usepackage{hyperref}

% Listings settings for Fortran code
\lstset{
    language=Fortran,
    basicstyle=\ttfamily\footnotesize,
    keywordstyle=\color{blue}\bfseries,
    commentstyle=\color{gray},
    stringstyle=\color{teal},
    numberstyle=\tiny\color{gray},
    numbers=left,
    stepnumber=1,
    breaklines=true,
    frame=single,
    captionpos=b
}

% Metadata
\title{Fortran with Achates}
\author{Daniel with Contributions from Achates}
\date{\today}

\begin{document}

\maketitle
\tableofcontents
\newpage

\section{Introduction}
Fortran has stood the test of time as a language optimized for numerical and scientific computing. It remains highly relevant today, not only for legacy systems but also for new projects requiring performance, precision, and scalability.

This report explores nine core areas where Fortran excels, often contrasting its capabilities with C++ and other modern programming languages. Each section includes practical examples to illustrate these strengths.

\section{1. Numerical and Scientific Computing}
\textit{Fortran’s array handling, built-in functions, and library support make it a natural choice for scientific applications.}

\subsection{Example: Solving a Linear System of Equations}
\begin{lstlisting}[caption={Solving a Linear System with Fortran}]
program solve_linear_system
    implicit none
    real, dimension(3,3) :: A
    real, dimension(3) :: b, x

    ! Coefficients matrix
    A = reshape([2.0, 1.0, 3.0, &
                 1.0, 2.0, 1.0, &
                 3.0, 1.0, 2.0], shape(A))

    ! Right-hand side
    b = [8.0, 7.0, 14.0]

    ! Solve using matmul and inversion
    x = matmul(inv(A), b)

    print *, "Solution: ", x
end program solve_linear_system

function inv(A) result(A_inv)
    real, dimension(:,:), intent(in) :: A
    real, dimension(size(A,1), size(A,2)) :: A_inv
    ! Stub: Replace with LAPACK-based inversion
end function inv
\end{lstlisting}

---

\section{2. Parallel Computing}
\textit{Fortran simplifies both shared and distributed memory parallelism with Coarrays, OpenMP, and MPI integration.}

\subsection{Example: Heat Equation with Coarrays}
\begin{lstlisting}[caption={Heat Equation Simulation Using Coarrays}]
program heat_equation
    use iso_fortran_env
    implicit none
    integer, parameter :: n = 100
    real, dimension(n) :: temp
    integer :: i

    ! Parallel computation using coarrays
    do i = 1, n
        temp(i) = compute_temperature(i)
    end do

    call sync all
    ! Coarray results are now distributed
end program heat_equation

real function compute_temperature(i)
    integer, intent(in) :: i
    compute_temperature = sin(real(i))
end function compute_temperature
\end{lstlisting}

---

\section{3. Precision and Accuracy}
\textit{Fortran’s \texttt{kind} system ensures explicit control over numerical precision, vital for scientific applications.}

\subsection{Example: High-Precision Area Calculation}
\begin{lstlisting}[caption={High-Precision Example in Fortran}]
program high_precision
    implicit none
    real(kind=selected_real_kind(15, 307)) :: pi, area
    real(kind=selected_real_kind(15, 307)) :: radius

    radius = 1.0_8
    pi = 3.141592653589793_8
    area = pi * radius**2

    print *, "High-Precision Area: ", area
end program high_precision
\end{lstlisting}

---

\section{4. Readability and Simplicity for Math-Centric Programs}
\textit{Fortran’s syntax closely resembles mathematical notation, making it ideal for numerical methods.}

\subsection{Example: Solving an ODE using Runge-Kutta}
\begin{lstlisting}[caption={Runge-Kutta Method for ODEs}]
program runge_kutta
    implicit none
    real, parameter :: h = 0.01
    real :: t, y, k1, k2, k3, k4

    t = 0.0
    y = 1.0

    do while (t < 1.0)
        k1 = h * f(t, y)
        k2 = h * f(t + h/2, y + k1/2)
        k3 = h * f(t + h/2, y + k2/2)
        k4 = h * f(t + h, y + k3)
        y = y + (k1 + 2*k2 + 2*k3 + k4) / 6
        t = t + h
    end do

    print *, "Solution at t=1.0: ", y
end program runge_kutta

real function f(t, y)
    real, intent(in) :: t, y
    f = -y + t
end function f
\end{lstlisting}

---

\section{5. Portability and Legacy Support}
\textit{Fortran’s standardized history and portability ensure that old codes remain relevant and modern codes are easily deployed across platforms.}

\subsection{Example: Monte Carlo Simulation}
\begin{lstlisting}[caption={Monte Carlo Simulation for π Estimation}]
program monte_carlo
    implicit none
    integer, parameter :: n = 1000000
    integer :: i, inside_circle
    real :: x, y, pi_estimate

    inside_circle = 0

    do i = 1, n
        call random_number(x)
        call random_number(y)
        if (x**2 + y**2 <= 1.0) then
            inside_circle = inside_circle + 1
        end if
    end do

    pi_estimate = 4.0 * real(inside_circle) / n
    print *, "Estimated pi: ", pi_estimate
end program monte_carlo
\end{lstlisting}

---

\section{6. Efficient I/O for Scientific Data}
\textit{Fortran’s flexible I/O capabilities allow for precise control over formatted and binary data.}

---

\section{7. Domain-Specific Features}
\textit{Built-in support for complex numbers, efficient integration with numerical libraries, and domain-specific optimizations make Fortran a leader in computational science.}

---

\section{8. Safety in Numerical Computations}
\textit{Fortran’s array bounds checking and explicit variable declarations enhance program safety and reduce runtime errors.}

---

\section{9. Parallel Discrete Event Simulation}
\textit{Coarray Fortran excels in simulations with independent, causal processes, such as modeling satellite missions.}

---

\section{Conclusion}
Fortran continues to thrive as a language for scientific and numerical computing. The examples in this report highlight its strengths, making it clear why Fortran remains indispensable in high-performance computing.

\end{document}