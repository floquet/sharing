% typeset: Pdftex
% Afterwards compile with pdflatex > bibtex > pdflatex > pdflatex
% TeXShop Settings... > Engine > BibTex Engine > biber
% beamer likes biber
% latex likes bibtex

% https://tex.stackexchange.com/questions/270633/beamer-and-the-pause-command
% https://tex.stackexchange.com/questions/1423/is-there-a-nice-way-to-compile-a-beamer-presentation-without-the-pauses
% Beamer presentation template for delete; created Sat Dec 28 15:07:56 MST 2024
% \documentclass[ handout ]{beamer} % for printing
\documentclass[ ]{beamer}

% Fetch home directory: make this file independent of file system
\usepackage{catchfile}
\CatchFileDef{\HomePath}{|kpsewhich -var-value=HOME}{}
% Define base paths
% relies on symlink  at /Users/dantopa/, e.g.
%   GitHub -> /Users/dantopa//repos-xiuhcoatl/github
\makeatletter
\edef\HomePath{\expandafter\zap@space\HomePath \@empty}
\makeatother
\newcommand{\pGithub} {\HomePath/GitHub/}
    \newcommand{\pTrunk} {\pGithub/genesis/}
        \newcommand{\pPres} {\pTrunk/pres/}
            \newcommand{\pWorkspace} {\pPres/}
        \newcommand{\pGlobal} {\pTrunk/global}
            \newcommand{\pGlobalConfig} {\pGlobal/config/}
                \newcommand{\pGlobalConfigCommon} {\pGlobalConfig/common/}

% Load Global Setup Files
\input{\pGlobalConfigCommon/"config-common.tex"}
\input{\pGlobalConfigDocs/"config-pres.tex"}
\input{\pConfig/"config-local.tex"}

% ===========================================================
% Global and Local Resource Setup
% The following lines load various global and local resource
% configurations, paths, and package lists required for the 
% document. These files are part of the shared library located
% ===========================================================

% config-common.tex

%   listings-codes.tex
%   num-components.tex
%   num-list.tex
%   paths-global.tex
%   paths-local.tex}
%   paths-bitbucket
%   theorems.tex
%   packages-common.tex

% Choose hyperlink configuration:
\input{\pGlobalSetup href-hidden.tex}   % For hidden links (clean, professional)
% \input{\pGlobalSetup href-visible.tex} % For visible links (debugging, drafts)

% Debugging with visible slide boundaries
% \setbeamertemplate{background canvas}[grid][ step = 1cm ]
%   --   --   --   --   --   --   --   --   --   -- Bibliography
\input{\pGlobalSetup bib-config-a.tex}
\addbibresource{\pShareBibliographies/$PROJECT_NAME.bib}
%\addbibresource{\pShareBibliographies/additional.bib}

% The order in which packages and configurations are loaded in LaTeX is critical to ensuring compatibility and avoiding conflicts. 
% In this setup, the bibliography configuration and resource addition commands are placed BEFORE loading the watermark package. 
% 
% This load order is particularly important for the following reasons:
% 
% 1. **Dependency Management**:
%    - The `biblatex` package, loaded within `bib-config-a.tex`, relies on specific internal macros and options. 
%      If a package such as `xwatermark` (which depends on `catoptions`) is loaded prematurely, it can redefine or interfere with these macros, 
%      leading to cryptic errors like `Use of \blx@tempa doesn't match its definition`.
% 
% 2. **Package Compatibility**:
%    - The `xwatermark` package uses `catoptions` to process its options. This library is known to modify package option handling globally, 
%      making it incompatible with other packages like `biblatex` if not managed carefully.
%    - By loading `xwatermark` after `bib-config-a.tex`, we avoid potential conflicts with `biblatex`'s setup.
% 
% 3. **Error Prevention**:
%    - Errors caused by incorrect load order, such as those involving `catoptions` and `biblatex`, are difficult to diagnose and resolve 
%      because they often result in seemingly unrelated errors deep within package internals.
%    - Maintaining the correct load order ensures a smooth workflow and avoids unnecessary troubleshooting.
% 
% 4. **Modularity and Flexibility**:
%    - The modularity of this LaTeX ecosystem (e.g., `bib-config-a.tex`, `href-hidden.tex`, and `xwatermark`) allows components to be reused across projects.
%      Ensuring the correct load order preserves this modularity and prevents unintended side effects when toggling or reusing these configurations.
% 
% To summarize, ALWAYS load bibliography-related configurations (e.g., `\input{\pGlobalSetup bib-config-a.tex}` and `\addbibresource`) BEFORE 
% other potentially conflicting packages such as `xwatermark`. This approach ensures compatibility, avoids errors, and maintains the integrity of the document's structure.
%
% NOTE: This knowledge is the result of extensive troubleshooting and collaboration. Preserve this load order unless you are certain of the implications of changing it!
%
%\usepackage[printwatermark]{xwatermark}
% \newwatermark[allpages,color=red!5,angle=45,scale=3,xpos=0,ypos=0]{DRAFT}
%   --   --   --   --   --   --   --   --   --   -- Title, Author
\title[delete]{Beamer Presentation for delete}
\author[Daniel Topa]{\TopaHII \\ \TopaHIIEmail}
\institute{\missiontech} 
\date{\today}

%   --   --   --   --   --   --   --   --   --   -- Structure
\begin{document}

\begin{frame}
    \titlepage
\end{frame}

\begin{frame}[ allowframebreaks ]\frametitle{Outline}
  \tableofcontents[ hideallsubsections ]
\end{frame}


% Main content
  % \input{\pSections "sec-intro.tex"}

\section{Introduction}  % == Main Section: Introduction ==
%%% Main introduction section to set up the context of the project.
Portability	of performance. Can a human do better? References: \cite{numrich2018parallel, ray2019fortran}, (\cite{markus2012modern, clerman2011modern, Hanson2013}) as well as the invaluable language reference guide \cite{metcalf2024modern}.	

%%% Subsection A: Overview of the Problem
% -----------------------------------------------------------
\subsection{Overview of the Problem}
This subsection provides a detailed description of the problem or challenge being addressed in this project.

%%% Subsection B: Objectives
% -----------------------------------------------------------
\subsection{Objectives}
This subsection outlines the main objectives of the project, including key research goals or development targets.

%%% Subsection C: Methodology
% -----------------------------------------------------------
\subsection{Methodology}
This subsection describes the methodology or approach taken to address the problem and achieve the objectives.

\endinput  % == End of Introduction Section ==

  % \input{\pSections "sec-backup"}

\section{Backup Slides}
%     %     %     %     %     %     %     %     %
\subsection{Computational Mechanics}

\begin{frame}\frametitle{\href{https://en.wikipedia.org/wiki/Computational_mechanics}{Professional Societies:} \href{https://en.wikipedia.org/wiki/Computational_mech}{Computational Mechanics}}
\center
\end{frame}


\endinput  %  ==  ==  ==  ==  ==  ==  ==  ==  ==


% Bibliography
\begin{frame}[allowframebreaks]
    \frametitle{Bibliography}
    \printbibliography
\end{frame}

\begin{frame}
    \titlepage
\end{frame}

\end{document}

%\tiny
%\scriptsize
%\footnotesize
%\small
%\normalsize
%\large
%\Large
%\LARGE
%\huge
%\Huge

%\, thin space (normally 1/6 of a quad);
%\> medium space (normally 2/9 of a quad);
%\; thick space (normally 5/18 of a quad);

\begin{frame}\frametitle{Frame Title}
\begin{enumerate}
  \item 
  \item 
  \item 
\end{enumerate}
\end{frame}

\begin{frame}\frametitle{Frame Title}
\begin{equation}
  \begin{array}{ccc} 
      %
      %
      %
  \end{array}
%\label{eq:}
\end{equation}
\end{frame}

\begin{frame}\frametitle{ }
\center
  \href{url}{
  \begin{overpic}[ scale = 1.0 ]
  {\pLocalGraphics graphic-file}
    %\put(-7,-10){Auxiliary text.}
  \end{overpic}}
\end{frame}

\begin{frame}\frametitle{Frame Title}
\begin{table}[htp]
%\caption{default}
\begin{center}
  \begin{tabular}{cc}
    %
    %
    %
  \end{tabular}
\end{center}
%\label{tab:label}
\end{table}
\end{frame}
%         ---         ---         ---         ---         ---   Equations
\begin{equation}
  
\label{eq:}
\end{equation}
%
\begin{equation}
\begin{split}
  & \
\end{split} 
\label{eq:}
\end{equation}
%
\begin{equation*}
  
\end{equation*}
%
\begin{equation}
\begin{array}{ccc}
    %
  & & \
    %
  & & \
    %
\end{array}
\end{equation}
%         ---         ---         ---         ---         ---   Tables
\begin{table}[htp]
\caption{default}
\begin{center}
\begin{tabular}{ccc}
    %
  & & \
    %
  & & \
    %
\end{tabular}
\end{center}
\label{tab:}
\end{table}
%         ---         ---         ---         ---         ---   Lists
\begin{enumerate}
  \item 
  \item
  \item
\end{enumerate}
%
\begin{enumerate}
  \item 
  \begin{enumerate}
    \item 
  \end{enumerate}
  \item
  \item
\end{enumerate}
\endinput  %  ==  ==  ==  ==  ==  ==  ==  ==  ==
