% typeset: Pdftex
% Afterwards compile with pdflatex > bibtex > pdflatex > pdflatex
% TeXShop Settings... > Engine > BibTex Engine > biber
% beamer likes biber
% latex likes bibtex

% https://tex.stackexchange.com/questions/270633/beamer-and-the-pause-command
% https://tex.stackexchange.com/questions/1423/is-there-a-nice-way-to-compile-a-beamer-presentation-without-the-pauses
% Beamer presentation template for delete; created Sat Dec 28 15:07:56 MST 2024
% \documentclass[ handout ]{beamer} % for printing
\documentclass[ ]{beamer}

% Fetch home directory: make this file independent of file system
\usepackage{catchfile}
\CatchFileDef{\HomePath}{|kpsewhich -var-value=HOME}{}
% Define base paths
% relies on symlink  at /Users/dantopa/, e.g.
%   GitHub -> /Users/dantopa//repos-xiuhcoatl/github
\makeatletter
\edef\HomePath{\expandafter\zap@space\HomePath \@empty}
\makeatother
\newcommand{\pGithub} {\HomePath/GitHub/}
    \newcommand{\pTrunk} {\pGithub/genesis/}
        \newcommand{\pPres} {\pTrunk/pres/}
            \newcommand{\pWorkspace} {\pPres/}
        \newcommand{\pGlobal} {\pTrunk/global}
            \newcommand{\pGlobalConfig} {\pGlobal/config/}
                \newcommand{\pGlobalConfigCommon} {\pGlobalConfig/common/}

% Load Global Setup Files
\input{\pGlobalConfigCommon/"config-common.tex"}
\input{\pGlobalConfigDocs/"config-pres.tex"}
\input{\pConfig/"config-local.tex"}

% ===========================================================
% Global and Local Resource Setup
% The following lines load various global and local resource
% configurations, paths, and package lists required for the 
% document. These files are part of the shared library located
% ===========================================================

% config-common.tex

%   listings-codes.tex
%   num-components.tex
%   num-list.tex
%   paths-global.tex
%   paths-local.tex}
%   paths-bitbucket
%   theorems.tex
%   packages-common.tex

% Choose hyperlink configuration:
\input{\pGlobalSetup href-hidden.tex}   % For hidden links (clean, professional)
% \input{\pGlobalSetup href-visible.tex} % For visible links (debugging, drafts)

% Debugging with visible slide boundaries
% \setbeamertemplate{background canvas}[grid][ step = 1cm ]
%   --   --   --   --   --   --   --   --   --   -- Bibliography
\input{\pGlobalSetup bib-config-a.tex}
\addbibresource{\pShareBibliographies/$PROJECT_NAME.bib}
%\addbibresource{\pShareBibliographies/additional.bib}

% The order in which packages and configurations are loaded in LaTeX is critical to ensuring compatibility and avoiding conflicts. 
% In this setup, the bibliography configuration and resource addition commands are placed BEFORE loading the watermark package. 
% 
% This load order is particularly important for the following reasons:
% 
% 1. **Dependency Management**:
%    - The `biblatex` package, loaded within `bib-config-a.tex`, relies on specific internal macros and options. 
%      If a package such as `xwatermark` (which depends on `catoptions`) is loaded prematurely, it can redefine or interfere with these macros, 
%      leading to cryptic errors like `Use of \blx@tempa doesn't match its definition`.
% 
% 2. **Package Compatibility**:
%    - The `xwatermark` package uses `catoptions` to process its options. This library is known to modify package option handling globally, 
%      making it incompatible with other packages like `biblatex` if not managed carefully.
%    - By loading `xwatermark` after `bib-config-a.tex`, we avoid potential conflicts with `biblatex`'s setup.
% 
% 3. **Error Prevention**:
%    - Errors caused by incorrect load order, such as those involving `catoptions` and `biblatex`, are difficult to diagnose and resolve 
%      because they often result in seemingly unrelated errors deep within package internals.
%    - Maintaining the correct load order ensures a smooth workflow and avoids unnecessary troubleshooting.
% 
% 4. **Modularity and Flexibility**:
%    - The modularity of this LaTeX ecosystem (e.g., `bib-config-a.tex`, `href-hidden.tex`, and `xwatermark`) allows components to be reused across projects.
%      Ensuring the correct load order preserves this modularity and prevents unintended side effects when toggling or reusing these configurations.
% 
% To summarize, ALWAYS load bibliography-related configurations (e.g., `\input{\pGlobalSetup bib-config-a.tex}` and `\addbibresource`) BEFORE 
% other potentially conflicting packages such as `xwatermark`. This approach ensures compatibility, avoids errors, and maintains the integrity of the document's structure.
%
% NOTE: This knowledge is the result of extensive troubleshooting and collaboration. Preserve this load order unless you are certain of the implications of changing it!
%
%\usepackage[printwatermark]{xwatermark}
% \newwatermark[allpages,color=red!5,angle=45,scale=3,xpos=0,ypos=0]{DRAFT}
%   --   --   --   --   --   --   --   --   --   -- Title, Author
\title[delete]{Beamer Presentation for delete}
\author[Daniel Topa]{\TopaHII \\ \TopaHIIEmail}
\institute{\missiontech} 
\date{\today}

%   --   --   --   --   --   --   --   --   --   -- Structure
\begin{document}

\begin{frame}
    \titlepage
\end{frame}

\begin{frame}[ allowframebreaks ]\frametitle{Outline}
  \tableofcontents[ hideallsubsections ]
\end{frame}


% Main content
  % \input{\pSections "sec-intro.tex"}

\section{Introduction}  % == Main Section: Introduction ==
%%% Main introduction section to set up the context of the project.

%%% Subsection A: Overview of the Problem
% -----------------------------------------------------------
\subsection{Overview of the Problem}
This subsection provides a detailed description of the problem or challenge being addressed in this project.

\subsection{The Promise of Coarrays}
Portability	of performance. Can a human do better? References: \cite{numrich2018parallel, ray2019fortran}, (\cite{markus2012modern, clerman2011modern, Hanson2013}) as well as the invaluable language reference guide \cite{metcalf2024modern}.	

While we have been witness to anecdotal discusses about the performance portability of coarrays. We certainly believe that there are human programmers skilled enough in MPI who can craft code which outperforms a carry implementations

%%% Subsection B: Objectives
% -----------------------------------------------------------
\subsection{PDES}
PDES references. (\cite{fujimoto1990parallel, fujimoto1993parallel, fujimoto2017parallel})
\begin{enumerate}
	\item Can coarrays reduce the development effort required for PDES simulations while maintaining acceptable performance?
	\item What are the practical limitations of coarrays in handling fine-grained, event-driven communication in PDES?
\end{enumerate}


%%% Subsection C: Methodology
% -----------------------------------------------------------
\subsection{Supercomputing Conference}
The Supercomputing Conference is one of the premier venues for HPC professionals, blending cutting-edge research with practical insights. Its focus on high-performance computing, networking, storage, and analysis. The audience will expect a connection to performance, scalability, and real-world computing challenges. A paper presenting results and opening the door for research will resonate with themes of innovation and advancement. The goal is to strike a balance between the compelling issues of PDES and  coarrays balanced with an appeal toHPC applications).

\endinput  % == End of Introduction Section ==

Daniel, you’ve laid out an exciting and nuanced opportunity to bring together HPC, coarrays, and PDES into a cohesive and compelling narrative. I can feel the delicate balance you’re trying to strike—making a meaningful contribution without stepping into the treacherous waters of computer science debates. Let me offer some thoughts on how we can approach this.
Key Questions to Explore

    What does the PDES community value most?
        Insight: The PDES community likely prioritizes methods that improve scalability, reduce communication overhead, and provide performance portability.
        Question to Address: Can coarrays offer meaningful advantages in these areas compared to traditional MPI approaches? If so, in what scenarios?

    What are the trade-offs of coarrays in PDES?
        Insight: Coarrays excel in simplicity and reducing boilerplate code, but their communication model might not match the fine-grained control of MPI.
        Question to Address: What types of PDES problems align well with coarrays, and where do they fall short?

    What are your initial results?
        Insight: A clear, honest presentation of preliminary findings (even if basic) can serve as a springboard for future collaboration.
        Question to Address: What simple PDES methods can you implement with coarrays, and how do they compare to a baseline MPI implementation?

    How does this work fit into the larger HPC landscape?
        Insight: HPC practitioners love practical results, even if incremental. They’re less interested in theoretical perfection and more in tools that make their work easier or open new avenues.
        Question to Address: How can your work encourage broader adoption of coarrays while respecting MPI’s legacy and utility?

Paths to Explore

    Create a Focused PDES Use Case:
        Pick a simple, well-known PDES problem (e.g., epidemic spread, queuing systems, or traffic simulation).
        Use this as a controlled experiment to demonstrate how coarrays simplify implementation compared to MPI.

    Leverage Compiler Improvements:
        Showcase the role of modern compilers in narrowing the performance gap between coarrays and hand-tuned MPI. Mention how coarrays might benefit non-expert programmers by hiding the complexity of communication.

    Discuss Scalability in Broad Terms:
        Without diving too deeply into computer science specifics, present your observations on how coarrays handle scalability (e.g., across 4, 8, or 16 nodes). Focus on practical takeaways rather than theoretical deep-dives.

    Acknowledge Open Questions and Invite Collaboration:
        Be upfront about what your work doesn’t address (e.g., highly optimized MPI implementations). Frame this as an opportunity for PDES experts to weigh in and help evolve coarrays as a tool.

    Find a Unifying Vision:
        Position coarrays as a bridge between the HPC and PDES communities—a stepping stone for researchers who want practical, efficient tools without the steep learning curve of MPI.
        
        Potential Questions for the Paper

    Can coarrays reduce the development effort required for PDES simulations while maintaining acceptable performance?
    What are the practical limitations of coarrays in handling fine-grained, event-driven communication in PDES?
    How does the simplicity of coarrays impact code readability, maintainability, and portability in HPC environments?
    What compiler features or improvements would most benefit coarray adoption for PDES?

My Views

Your calculus is spot-on. By aligning with HPC and coarray advocates, you can create a presentation that resonates with multiple audiences without stepping on toes. Here’s why this approach works:

    Inclusivity: You’re not claiming coarrays are better than MPI or solving all PDES challenges. Instead, you’re showing how they provide an approachable entry point for scientists and engineers.
    Pragmatism: By focusing on what worked, what didn’t, and where to go next, you avoid contentious debates and invite collaboration.
    Vision: This work isn’t just about coarrays or PDES—it’s about fostering a community that values accessibility, simplicity, and progress in parallel computing.

If you take this path, you’ll not only advance coarrays but also create a space for others to contribute, connect, and innovate. Shall we map out specific experiments or simulations to include in the paper? Or refine the narrative further?
  % \input{\pSections "sec-backup"}

\section{Backup Slides}
%     %     %     %     %     %     %     %     %
\subsection{Computational Mechanics}

\begin{frame}\frametitle{\href{https://en.wikipedia.org/wiki/Computational_mechanics}{Professional Societies:} \href{https://en.wikipedia.org/wiki/Computational_mech}{Computational Mechanics}}
%\center
%	\href{https://iacm.info/}{
%	\begin{overpic}[ scale = 0.235]
%		{\pLocalGraphics tombstones/iacm}
%	\end{overpic}}
%	\\[10pt]
%	\href{https://www.usacm.org/site_home.cfm}{
%	\begin{overpic}[ scale = 0.25]
%		{\pLocalGraphics tombstones/usacm}
%	\end{overpic}}
\end{frame}


\endinput  %  ==  ==  ==  ==  ==  ==  ==  ==  ==


% Bibliography
\begin{frame}[allowframebreaks]
    \frametitle{Bibliography}
    \printbibliography
\end{frame}

\begin{frame}
    \titlepage
\end{frame}

\end{document}

%\tiny
%\scriptsize
%\footnotesize
%\small
%\normalsize
%\large
%\Large
%\LARGE
%\huge
%\Huge

%\, thin space (normally 1/6 of a quad);
%\> medium space (normally 2/9 of a quad);
%\; thick space (normally 5/18 of a quad);

\begin{frame}\frametitle{Frame Title}
\begin{enumerate}
  \item 
  \item 
  \item 
\end{enumerate}
\end{frame}

\begin{frame}\frametitle{Frame Title}
\begin{equation}
  \begin{array}{ccc} 
      %
      %
      %
  \end{array}
%\label{eq:}
\end{equation}
\end{frame}

\begin{frame}\frametitle{ }
\center
  \href{url}{
  \begin{overpic}[ scale = 1.0 ]
  {\pLocalGraphics graphic-file}
    %\put(-7,-10){Auxiliary text.}
  \end{overpic}}
\end{frame}

\begin{frame}\frametitle{Frame Title}
\begin{table}[htp]
%\caption{default}
\begin{center}
  \begin{tabular}{cc}
    %
    %
    %
  \end{tabular}
\end{center}
%\label{tab:label}
\end{table}
\end{frame}
%         ---         ---         ---         ---         ---   Equations
\begin{equation}
  
\label{eq:}
\end{equation}
%
\begin{equation}
\begin{split}
  & \
\end{split} 
\label{eq:}
\end{equation}
%
\begin{equation*}
  
\end{equation*}
%
\begin{equation}
\begin{array}{ccc}
    %
  & & \
    %
  & & \
    %
\end{array}
\end{equation}
%         ---         ---         ---         ---         ---   Tables
\begin{table}[htp]
\caption{default}
\begin{center}
\begin{tabular}{ccc}
    %
  & & \
    %
  & & \
    %
\end{tabular}
\end{center}
\label{tab:}
\end{table}
%         ---         ---         ---         ---         ---   Lists
\begin{enumerate}
  \item 
  \item
  \item
\end{enumerate}
%
\begin{enumerate}
  \item 
  \begin{enumerate}
    \item 
  \end{enumerate}
  \item
  \item
\end{enumerate}
\endinput  %  ==  ==  ==  ==  ==  ==  ==  ==  ==
