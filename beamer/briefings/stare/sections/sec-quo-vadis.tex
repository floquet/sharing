% \input{\pathsections "sec-quo-vadis"}

\section{Quo Vadis}

%     %     %     %     %     %     %     %     %
\begin{frame}\frametitle{What Might We See?}
\begin{enumerate}
	\item Glinting
	\begin{itemize}
		\item Reorientation
		\item Orbit Maneuvers
	\end{itemize}
	\item Venting
	\item Stellar obscuration
	\item Collision
	\item Satellites orbiting satellites
\end{enumerate}
\end{frame}

%     %     %     %     %     %     %     %     %
\subsection{Imaging Techniques, Methods}
\begin{frame}\frametitle{Possible Imaging Techniques, Methods,  \& Issues}
\begin{enumerate}
	\item Optical Phase Imaging
	\item Hyperpsectral Imaging
	\item Multiple Time Scales
	\item Atmospheric propagation models
\end{enumerate}
\end{frame}

%     %     %     %     %     %     %     %     %
\subsection{Optical Layouts}
\begin{frame}\frametitle{We Can Draw Optical Layouts}
\center
	\href{https://stanfordhealthcare.org/medical-treatments/l/laser-vision-correction/procedures/wavefront-lasik.html/presentation-mode/content/shc/en/stanford-health-care-now/videos/laser-vision-correction-wavefront-lasik}{
	\begin{overpic}[ scale = 0.4 ]
		{\pLocalGraphics diagram.pdf}
	\end{overpic}}
\end{frame}

%     %     %     %     %     %     %     %     %
\subsection{NVIDIA}
\begin{frame}\frametitle{NVIDIA RTX}
\begin{enumerate}
	\item \href{https://www.nvidia.com/en-us/design-visualization/technologies/rtx/}{NVIDIA RTX Technology} \mg{(NVIDIA)}
	\item \href{https://developer.nvidia.com/rtx/}{NVIDIA RTX Platform} \mg{(NVIDIA)}
	\item \href{https://developer.nvidia.com/rtx/ray-tracing}{RTX Technology} \mg{(NVIDIA)}
	%
	\begin{enumerate}
		\item \href{https://developer.nvidia.com/rtx/path-tracing/}{RTX Path Tracing}
		\item \href{https://developer.nvidia.com/rtxgi}{RTX Global Illumination}
		\item \href{https://developer.nvidia.com/rtxdi}{RTX Dynamic Illumination}
		\item \href{https://developer.nvidia.com/rtx/dlss}{Deep Learning Super Sampling}
		\item \href{https://developer.nvidia.com/nvidia-rt-denoiser}{Real-Time Denoisers}
	\end{enumerate}
%
	\item \href{https://en.wikipedia.org/wiki/Nvidia_RTX}{Nvidia RTX} \mg{(Wikipedia)}
\end{enumerate}
\end{frame}

\begin{frame}\frametitle{\href{https://www-sciencedirect-com.libproxy.unm.edu/science/article/pii/S0377042714002374}{Image Processing with MPI–CUDA}}
\center
	\href{https://www.sciencedirect.com/science/article/pii/S0377042714002374}{
	\begin{overpic}[ scale = 0.5 ]
		{\pLocalGraphics sw/mpi-cuda}
	\end{overpic}}
\end{frame}

\begin{frame}\frametitle{\href{https://www.cs.kent.edu/~xchang/public_bak/paper/Image\%20Processing\%20with\%20CUDA.pdf}{Image Processing with CUDA}}
\center
	\href{https://www.cs.kent.edu/~xchang/public_bak/paper/Image\%20Processing\%20with\%20CUDA.pdf}{
	\begin{overpic}[ scale = 0.5 ]
		{\pLocalGraphics sw/cuda-thesis}
	\end{overpic}}
\end{frame}

\begin{frame}\frametitle{\href{https://www.nvidia.com/content/nvision2008/tech_presentations/Game_Developer_Track/NVISION08-Image_Processing_and_Video_with_CUDA.pdf}{Image Processing with CUDA}}
\center
	\href{https://www.nvidia.com/content/nvision2008/tech_presentations/Game_Developer_Track/NVISION08-Image_Processing_and_Video_with_CUDA.pdf}{
	\begin{overpic}[ scale = 0.25 ]
		{\pLocalGraphics sw/cuda-ppt}
	\end{overpic}}
\end{frame}

\endinput  %  ==  ==  ==  ==  ==  ==  ==  ==  ==
