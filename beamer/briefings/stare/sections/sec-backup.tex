% \input{\pathsections "sec-backup"}

\section{Backup Slides}
%     %     %     %     %     %     %     %     %
\subsection{Images}

\begin{frame}\frametitle{\href{https://www.science.org/}{Science}  -- \href{https://www.science.org/content/article/satellite-swarms-are-threatening-night-sky-creating-new-zone-environmental-conflict}{The Fault in Our Stars}}
\center
	\href{https://www.science.org/content/article/satellite-swarms-are-threatening-night-sky-creating-new-zone-environmental-conflict}{
	\begin{overpic}[ scale = 0.15]
		{\pLocalGraphics photos/science-03x}
		\put(50,60) {\tiny{What the four largest swarm builders}}
		\put(50,55) {\tiny{have so far submitted to regulators}}
	\end{overpic}}
\end{frame}

\begin{frame}\frametitle{\href{https://www.science.org/}{Science}  -- \href{https://www.science.org/content/article/satellite-swarms-are-threatening-night-sky-creating-new-zone-environmental-conflict}{The Fault in Our Stars}}
\center
	\href{https://www.science.org/content/article/satellite-swarms-are-threatening-night-sky-creating-new-zone-environmental-conflict}{
	\begin{overpic}[ scale = 0.25]
		{\pLocalGraphics photos/science-starlink60}
	\end{overpic}}
	\\[5pt]
	\tiny{The first flotilla of 60 Starlink satellites was released in May 2019 from a SpaceX rocket.}
\end{frame}

%     %     %     %     %     %     %     %     %
\subsection{Mathematica}
\begin{frame}\frametitle{\href{https://reference.wolfram.com/language/guide/ImageProcessing.html}{Mathematica Image Processing}}
\center
\begin{table}[htp]
%\caption{default}
\begin{center}
\begin{tabular}{cccc}
	%
	original & edges & key points & backgound out \\
	%
	\begin{overpic}[ scale = 0.75]
		{\pLocalGraphics mm/i0}
	\end{overpic} &
	%
	\begin{overpic}[ scale = 0.75]
		{\pLocalGraphics mm/i1}
	\end{overpic} &
	%
	\begin{overpic}[ scale = 0.75]
		{\pLocalGraphics mm/i2}
	\end{overpic} &
	%
	\begin{overpic}[ scale = 0.75]
		{\pLocalGraphics mm/i3}
	\end{overpic}
	%
\end{tabular}
\end{center}
\label{tab:mm}
\end{table}%
Not so usefull...
\end{frame}

\begin{frame}\frametitle{\referenceWolfram \ \href{\urlmmSatellite}{Satellite} Tools}
%\begin{frame}\frametitle{\referenceWolfram Groundtracks}
\center
	\href{\urlmmSatellite}{
	\begin{overpic}[ scale = 0.25]
		{\pLocalGraphics mm/iss-dataset}
	\end{overpic}}
	\\[2pt]
	\tiny{First of 63 entries for International Space Station dataset}
\end{frame}

%\begin{frame}\frametitle{\referenceWolfram \ \href{\urlmmSatellite}{Satellite} Tools}
%%\begin{frame}\frametitle{\referenceWolfram Groundtracks}
%\center
%	\href{\urlmmSatellite}{
%	\begin{overpic}[ scale = 0.25]
%		{\pLocalGraphics mm/iss-dataset-01}
%	\end{overpic}}
%	\\[2pt]
%	\tiny{Additional entries for International Space Station dataset}
%\end{frame}
%
%\begin{frame}\frametitle{\referenceWolfram \ \href{\urlmmSatellite}{Satellite} Tools}
%%\begin{frame}\frametitle{\referenceWolfram Groundtracks}
%\center
%	\href{\urlmmSatellite}{
%	\begin{overpic}[ scale = 0.35 ]
%		{\pLocalGraphics mm/iss-track}
%	\end{overpic}}
%	\\[2pt]
%	\tiny{Groundtracks for the International Space Station}
%\end{frame}
%
%\begin{frame}\frametitle{\referenceWolfram \ \href{\urlmmGeoVisibleRegion}{GeoVisibleRegion} Tools}
%%\begin{frame}\frametitle{\referenceWolfram Groundtracks}
%\center
%	\href{\urlmmSatellite}{
%	\begin{overpic}[ scale = 0.275 ]
%		{\pLocalGraphics mm/geographics-02}
%	\end{overpic}}
%	\\[2pt]
%	\tiny{\qquad What's visible from altitude of 10000 m?}
%\end{frame}
%
%\begin{frame}\frametitle{\referenceWolfram \ \href{\urlmmGeoVisibleRegion}{GeoVisibleRegion} Tools}
%%\begin{frame}\frametitle{\referenceWolfram Groundtracks}
%\center
%	\href{\urlmmSatellite}{
%	\begin{overpic}[ scale = 0.3 ]
%		{\pLocalGraphics mm/geographics-01}
%	\end{overpic}}
%	\\[2pt]
%	\tiny{\qquad What's visible from geostationary orbit?}
%\end{frame}

\begin{frame}\frametitle{\referenceWolfram \ \href{\urlmmSatellite}{Satellite} Tools}
%\begin{frame}\frametitle{\referenceWolfram Groundtracks}
\center
	\footnotesize{Groundtracks, International Space Station} \\
	\href{\urlmmSatellite}{
	\begin{overpic}[ scale = 0.5 ]
		{\pLocalGraphics mm/mm-iss-01}
	\end{overpic}}
	\\[10pt]
	\tiny{\mg{\texttt{With[{data = 
   EntityValue[
    Entity["Satellite", "25544"], {"Position", "PositionLine"}]},
 GeoGraphics[{Gray, Thickness[.005], 
   Arrowheads[{{0.05, 0.4}, {0.05, 0.13}}], Arrow @@ data[[2]], Red, 
   PointSize[.01], Point[data[[1]]], Opacity[.1], Black, 
   GeoVisibleRegion[data[[1]]]}, GeoCenter -> data[[1]], 
  GeoRange -> "World"]]]}}}
\end{frame}

\begin{frame}\frametitle{\referenceWolfram \ \href{\urlmmGeoVisibleRegion}{GeoVisibleRegion} Tools}
%\begin{frame}\frametitle{\referenceWolfram Groundtracks}
\center
	\footnotesize{What's visible from 10,000 m over Albuquerque?}\\
	\href{\urlmmSatellite}{
	\begin{overpic}[ scale = 0.25 ]
		{\pLocalGraphics mm/mm-geovis-01}
	\end{overpic}}
	\\[15pt]
	\tiny{\mg{\texttt{GeoGraphics[GeoVisibleRegion[{35, -106, 10000}]]}}}
\end{frame}

\begin{frame}\frametitle{\referenceWolfram \ \href{\urlmmGeoVisibleRegion}{GeoVisibleRegion} Tools}
%\begin{frame}\frametitle{\referenceWolfram Groundtracks}
\center
	\footnotesize{What's visible from geostationary orbit over Albuquerque?}
	\href{\urlmmSatellite}{
	\begin{overpic}[ scale = 0.5 ]
		{\pLocalGraphics mm/mm-geovis-02}
	\end{overpic}}
	\\[15pt]
	\tiny{\mg{\texttt{GeoGraphics[
 GeoVisibleRegion[{35, -106, Quantity[35786, "Kilometers"]}], 
 GeoRange -> "World"]}}}
\end{frame}

%     %     %     %     %     %     %     %     %
\subsection{\href{https://www.mathworks.com/}{MATLAB}}
\begin{frame}\frametitle{\href{https://www.mathworks.com/products/image-processing.html}{MATLAB Image Processing Toolbox: I}}
\center
\begin{table}[htp]
%\caption{default}
\begin{center}
\begin{tabular}{ccc}
		%
	\href{https://www.mathworks.com/help/images/image-analysis.html}{\begin{overpic}[ scale = 0.25]
		{\pLocalGraphics matlab-01}
	\end{overpic}} &
		%
	\href{https://www.mathworks.com/help/images/image-segmentation.html}{\begin{overpic}[ scale = 0.25]
		{\pLocalGraphics matlab-02}
	\end{overpic}} &
		%
	\href{https://www.mathworks.com/help/images/hyperspectral-image-processing.html}{\begin{overpic}[ scale = 0.25]
		{\pLocalGraphics matlab-03}
	\end{overpic}}
		%
\end{tabular}
\end{center}
\label{tab:matlab-1}
\end{table}%
\end{frame}

\begin{frame}\frametitle{\href{https://www.mathworks.com/products/image-processing.html}{MATLAB Image Processing Toolbox: II}}
\center
\begin{table}[htp]
%\caption{default}
\begin{center}
\begin{tabular}{ccc}
		%
	\href{https://www.mathworks.com/help/images/deep-learning.html}{\begin{overpic}[ scale = 0.25]
		{\pLocalGraphics matlab-04}
	\end{overpic}} &
		%
	\href{https://www.mathworks.com/help/images/image-enhancement-and-restoration.html}{\begin{overpic}[ scale = 0.25]
		{\pLocalGraphics matlab-05}
	\end{overpic}} &
		%
	\href{https://www.mathworks.com/help/images/code-generation-and-gpu-support.html}{\begin{overpic}[ scale = 0.25]
		{\pLocalGraphics matlab-06}
	\end{overpic}}
		%
\end{tabular}
\end{center}
\label{tab:matlab-2}
\end{table}%
\end{frame}

\endinput  %  ==  ==  ==  ==  ==  ==  ==  ==  ==
