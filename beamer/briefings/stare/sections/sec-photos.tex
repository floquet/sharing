% \input{\pathsections "sec-photos"}

\section{Open Source Photos}%: Streaks \& Satellites}
\begin{frame}\frametitle{Survey of Open Source Satellite Streak Photos}
\begin{enumerate}
	\item Media
	\item Dedicated Web Sites
	\item Federation of American Scientists
\end{enumerate}
\end{frame}


%     %     %     %     %     %     %     %     %
\subsection{Media}

%     %     %     %     %     %     %     %     %
% Physics Today
\begin{frame}\frametitle{\referencePtoday  -- Starlink Streaks}
\center
	\tiny{
	\href{https://pubs.aip.org/physicstoday/online/27738}{
	\begin{overpic}[ scale = 0.3]
		{\pLocalGraphics photos/pt-lowell}
		%\put(2,105) {\color{blue}{https://www.satobs.org/seesat/Aug-2010/0037.html}}
	\end{overpic}}
	}
	\\[5pt]
	%\pause
Streaks from sunlight reflected off {\color{blue}{Starlink satellites}} dominate this image from the Lowell Observatory.\\The satellites had not yet reached their operational altitude.
\end{frame}

\begin{frame}\frametitle{\href{https://pubs.aip.org/physicstoday}{Physics Today}  -- Crisscross Streaks}
\center
	\href{https://pubs.aip.org/physicstoday/article/75/4/25/2842796/Ballooning-satellite-populations-in-low-Earth?searchresult=1\#81048522}{
	\begin{overpic}[ scale = 0.45]
		{\pLocalGraphics photos/physics-today}
		%\put(2,105) {\color{blue}{https://www.satobs.org/seesat/Aug-2010/0037.html}}
	\end{overpic}}
	\\
	\pause
	\tiny{
{\color{blue}{Satellite}} streaks crisscross a two-hour sequence of observations.\\Near the center is a globular cluster and marked in {\color{red}{red}} is a {\color{red}{comet}}.
}
\end{frame}

%     %     %     %     %     %     %     %     %
% Scientifc American
\begin{frame}\frametitle{\href{https://pubs.aip.org/physicstoday}{Scientific American}  -- \href{https://www.scientificamerican.com/article/can-you-spot-a-satellite/}{Can You Spot a Satellite?}}
\center
	\href{https://www.scientificamerican.com/article/can-you-spot-a-satellite/}{
	\begin{overpic}[ scale = 0.18]
		{\pLocalGraphics photos/sciam-glint}
		\put(-4,1) {\tiny{\colorbox{white}{Sunlight {\color{blue}{glinting}} off a {\color{blue}{satellite's}} solar panels can occasionally create visually stunning ``flares''}}}
	\end{overpic}}
%	\tiny{\color{white}{
%Sunlight \color{}{glinting} off a \color{yellow}{satellite's} solar panels can occasionally create visually stunning "flares," like this one captured in a time-lapse image over %Alberta, Canada in 2017.}.
%}}
\end{frame}

\begin{frame}\frametitle{\href{https://pubs.aip.org/physicstoday}{Scientific American}  -- \href{https://www.scientificamerican.com/article/giant-satellite-outshines-most-stars-in-the-sky/}{Satellite Outshines Sky}}
\center
	\href{https://www.scientificamerican.com/article/giant-satellite-outshines-most-stars-in-the-sky/}{
	\begin{overpic}[ scale = 0.18]
		{\pLocalGraphics photos/sciam-outshine}
		\put(-4,1) {\tiny{\colorbox{white}{At times, the enormous {\color{blue}{BlueWalker 3 telecommunications satellite}} is brighter than some of }}}
		\put(-4,-3) {\tiny{\colorbox{white}{the most iconic stars visible from Earth}}}
	\end{overpic}}
\end{frame}

\begin{frame}\frametitle{\href{https://pubs.aip.org/physicstoday}{Scientific American}  -- \href{https://www.scientificamerican.com/article/euclid-space-telescope-snaps-spectacular-first-images/}{Euclid Space Telescope}}
\center
	\href{https://www.scientificamerican.com/article/euclid-space-telescope-snaps-spectacular-first-images/}{
	\begin{overpic}[ scale = 0.12]
		{\pLocalGraphics photos/sciam-euclid}
	\end{overpic}}
	\\[3pt]
	\tiny{Euclid’s telescope collected light for 100 seconds to enable NISP to create this image.}
\end{frame}

%     %     %     %     %     %     %     %     %
% Scientific American
\begin{frame}\frametitle{\href{https://pubs.aip.org/physicstoday}{Scientific American}  -- \href{https://www.scientificamerican.com/article/spacexs-dark-satellites-are-still-too-bright-for-astronomers/}{Dark Satellites Too Bright}}
\center
	\href{https://www.scientificamerican.com/article/spacexs-dark-satellites-are-still-too-bright-for-astronomers/}{
	\begin{overpic}[ scale = 0.11]
		{\pLocalGraphics photos/sciam-chile}
		\put(-5,-5) {\tiny{Time-lapse image of a \bl{Starlink satellite} cluster at the Cerro Tololo}}
		\put(-5,-10) {\tiny{Inter-American Observatory in Chile in November 2019 on the}}
		\put(-5,-15) {\tiny{Blanco telescope’s wide-field camera}}	
	\end{overpic}}
\end{frame}

%     %     %     %     %     %     %     %     %
% Science
\begin{frame}\frametitle{\href{https://www.science.org/}{Science}  -- \href{https://www.science.org/content/article/satellite-swarms-are-threatening-night-sky-creating-new-zone-environmental-conflict}{The Fault in Our Stars}}
\center
	\href{https://www.science.org/content/article/satellite-swarms-are-threatening-night-sky-creating-new-zone-environmental-conflict}{
	\begin{overpic}[ scale = 0.17]
		{\pLocalGraphics photos/science-02}
	\end{overpic}}
	\\[3pt]
	\tiny{\bl{Starlink satellite} train over Cerro Tololo in Chile}
\end{frame}

\begin{frame}\frametitle{\href{https://www.science.org/}{Science}  -- \href{https://www.science.org/content/article/satellite-swarms-are-threatening-night-sky-creating-new-zone-environmental-conflict}{The Fault in Our Stars}}
\center
	\href{https://www.science.org/content/article/satellite-swarms-are-threatening-night-sky-creating-new-zone-environmental-conflict}{
	\begin{overpic}[ scale = 0.17]
		{\pLocalGraphics photos/science-01}
		\put(-3,-3) {\tiny{The bright tracks of \bl{satellites} in low-Earth orbit mar this 2.5-minute exposure of the double star Albireo}}
	\end{overpic}}
\end{frame}

\begin{frame}\frametitle{\href{https://www.science.org/}{Science}  -- \href{https://www.science.org/content/article/satellite-swarms-are-threatening-night-sky-creating-new-zone-environmental-conflict}{The Fault in Our Stars}}
\center
	\href{https://www.science.org/content/article/satellite-swarms-are-threatening-night-sky-creating-new-zone-environmental-conflict}{
	\begin{overpic}[ scale = 0.14]
		{\pLocalGraphics photos/science-04}
		\put(-5,5) {\colorbox{white}{\tiny{\bl{Starlink satellite} tracks next to comet NEOWISE at Idaho’s Craters of the Moon, a national }}}
		\put(-5,1) {\colorbox{white}{\tiny{monument. Fifty time-lapse shots, each 4 seconds long, were stacked to make the image.}}}
	\end{overpic}}
\end{frame}


\begin{frame}\frametitle{\href{https://www.science.org/}{Science}  -- Additional References}
\begin{enumerate}
	\item \href{https://www.science.org/content/article/astronomers-set-center-counter-threat-satellite-swarms}{Astronomers set up center to counter threat of satellite swarms} \mg{3 Feb 2022}
	\item \href{https://www.science.org/content/article/tens-thousands-communications-satellites-could-spoil-view-giant-sky-telescope}{Tens of thousands of communications satellites could spoil view of giant sky telescope}
	\item \href{https://www.science.org/doi/10.1126/science.370.6514.274}{Satellite swarm threatens radio array}
\end{enumerate}
\end{frame}

%     %     %     %     %     %     %     %     %
% Media
\begin{frame}\frametitle{\href{https://www.duluthnewstribune.com/}{Duluth News Tribune}  }
\begin{table}[htp]
%\caption{default}
\begin{center}
\begin{tabular}{cc}
		%
	\href{\urlDuluth}{
	\begin{overpic}[ scale = 0.075]
		{\pLocalGraphics photos/duluth-01}
	\end{overpic}}
		%
	\qquad
		%
	\href{\urlDuluth}{
	\begin{overpic}[ scale = 0.075]
		{\pLocalGraphics photos/streaks}
	\end{overpic}} \\[10pt]
		%
	\href{\urlDuluth}{
	\begin{overpic}[ scale = 0.075]
		{\pLocalGraphics photos/trail}
	\end{overpic}}
		%
	\qquad
		%
	\href{\urlDuluth}{
	\begin{overpic}[ scale = 0.075]
		{\pLocalGraphics photos/hubble}
	\end{overpic}}
	
\end{tabular}
\end{center}
\label{tab:duluth}
\end{table}%
\end{frame}

\begin{frame}\frametitle{\href{https://www.duluthnewstribune.com/}{Duluth News Tribune}  }
{\tiny{
\begin{table}[htp]
%\caption{default}
\begin{center}
\begin{tabular}{cc}
		%
	\href{\urlDuluth}{
	\begin{overpic}[ scale = 0.075]
		{\pLocalGraphics photos/duluth-01}
		\put(5,10) {\color{white}{Three {\color{yellow}{satellites}} left light trails when they}}
		\put(5,5) {\color{white}{passed through this Hubble photo}}
	\end{overpic}}
		%
	\qquad
		%
	\href{\urlDuluth}{
	\begin{overpic}[ scale = 0.075]
		{\pLocalGraphics photos/streaks}
		\put(5,10) {\color{white}{A train of Starlink {\color{yellow}{satellites}} crosses}}
		\put(5,5) {\color{white}{the sky over Duluth, Minnesota}}
	\end{overpic}} \\[10pt]
		%
	\href{\urlDuluth}{
	\begin{overpic}[ scale = 0.075]
		{\pLocalGraphics photos/trail}
		\put(5,10) {\color{white}{During a time exposure, a passing {\color{yellow}{satellite}} }}
		\put(5,5) {\color{white}{leaves a trail as seen in this Hubble photo}}
	\end{overpic}}
		%
	\qquad
		%
	\href{\urlDuluth}{
	\begin{overpic}[ scale = 0.075]
		{\pLocalGraphics photos/hubble}
		\put(5,10) {\color{white}{A {\color{yellow}{satellite}} located closer to the Hubble}}
		\put(5,5) {\color{white}{makes a wide, bright trail}}
	\end{overpic}}
	
\end{tabular}
\end{center}
\label{tab:duluth}
\end{table}
}}
		%
\end{frame}
% During a time exposure, a passing satellite leaves a trail as seen in this Hubble photo
% A train of Starlink satellites crosses the sky over Duluth, Minnesota
% A satellite located closer to the Hubble makes a wide, bright trail


%     %     %     %     %     %     %     %     %
\subsection{\href{https://www.satobs.org/telescope.html}{Telescopic Satellite Observations}}
\begin{frame}\frametitle{USAF Maui}
	\href{https://www.satobs.org/telescope.html}{
	\begin{overpic}[ scale = 0.075]
		{\pLocalGraphics photos/sts37-usaf-plmtf}
	\end{overpic}}
\end{frame}

\begin{frame}\frametitle{\href{http://www.fas.org/spp/military/program/track/lacr1_2.jpg}{Space System --  Lacrosse 1}}
\center
	\tiny{
	\href{https://spp.fas.org/military/program/track/lacr1_2.jpg}{
	\begin{overpic}[ scale = 0.1]
		{\pLocalGraphics photos/lacross}
		\put(2,105) {\color{blue}{https://www.satobs.org/seesat/Aug-2010/0037.html}}
	\end{overpic}}
	}
\end{frame}

%     %     %     %     %     %     %     %     %
\subsection{\streak}
%\begin{frame}\frametitle{\referenceStreak}
%	\href{https://www.anecdata.org/projects/view/satellite-streak-watcher}{
%	\begin{overpic}[ scale = 0.125]
%		{\pLocalGraphics photos/satellite-streak-watcher}
%	\end{overpic}}
%	\quad
%	\raisebox{1.2cm}{\href{https://www.anecdata.org/projects/view/satellite-streak-watcher}{
%	\begin{overpic}[ scale = 0.35]
%		{\pLocalGraphics photos/anecdata/anec-splash}
%	\end{overpic}}}
%	
%\end{frame}

\begin{frame}\frametitle{\referenceStreak}
\center
	\href{\urlStreak}{
	\begin{overpic}[ scale = 0.5]
		{\pLocalGraphics photos/anecdata/anec-splash}
	\end{overpic}}
\end{frame}

\begin{frame}\frametitle{\referenceStreak: Data Base}
\center
	\href{\urlStreak}{
	\begin{overpic}[ scale = 0.15]
		{\pLocalGraphics anec-ss}
	\end{overpic}}
\end{frame}

\begin{frame}\frametitle{\referenceStreak: Datasheet}
\center
	\href{https://anecdata.org/posts/view/106118}{
	\begin{overpic}[ scale = 0.25]
		{\pLocalGraphics anec-datasheet}
	\end{overpic}}
\end{frame}

\begin{frame}\frametitle{\referenceStreak: Photos}
\begin{table}[htp]
%\caption{default}
\begin{center}
\begin{tabular}{ccc}
		%
	\href{\urlAnec105960}{
	\begin{overpic}[ scale = 0.25]
		{\pLocalGraphics photos/anecdata/anec-105960}
	\end{overpic}} &
		%
	\href{\urlAnec106010}{
	\begin{overpic}[ scale = 0.25]
		{\pLocalGraphics photos/anecdata/anec-106010}
	\end{overpic}} &
		%
	\href{\urlAnec106328}{
	\begin{overpic}[ scale = 0.25]
		{\pLocalGraphics photos/anecdata/anec-106328}
	\end{overpic}} \\[15pt]
		%
	\href{\urlAnec106118}{
	\begin{overpic}[ scale = 0.25]
		{\pLocalGraphics photos/anecdata/anec-106118}
	\end{overpic}} &
		%
	\href{\urlAnec107249}{
	\begin{overpic}[ scale = 0.25]
		{\pLocalGraphics photos/anecdata/anec-107249}
	\end{overpic}} &
		%
	\href{\urlAnec 105783}{
	\begin{overpic}[ scale = 0.25]
		{\pLocalGraphics photos/anecdata/anec-105783}
	\end{overpic}}
		%	
\end{tabular}
\end{center}
\label{tab:anec}
\end{table}%
\end{frame}


\begin{frame}\frametitle{\referenceStreak: Annotations}
\tiny{
\begin{table}[htp]
%\caption{default}
\begin{center}
\begin{tabular}{ccc}
		%
	\href{\urlAnec105960}{
	\begin{overpic}[ scale = 0.25]
		{\pLocalGraphics photos/anecdata/anec-105960}
		\put(5,5) {\color{yellow}{$\emptyset$}}
	\end{overpic}} &
		%
	\href{\urlAnec106010}{
	\begin{overpic}[ scale = 0.25]
		{\pLocalGraphics photos/anecdata/anec-106010}
		\put(5,5) {\color{yellow}{SpaceX Satellites}}
	\end{overpic}} &
		%
	\href{\urlAnec106328}{
	\begin{overpic}[ scale = 0.25]
		{\pLocalGraphics photos/anecdata/anec-106328}
		\put(5,5) {\color{yellow}{Starlink Streak}}
	\end{overpic}} 
	\\[15pt]
		%
	\href{\urlAnec106118}{
	\begin{overpic}[ scale = 0.25]
		{\pLocalGraphics photos/anecdata/anec-106118}
		\put(5,12) {\color{yellow}{well over 20 ... }}
		\put(5,2) {\color{yellow}{satellites}}
	\end{overpic}} &
		%
	\href{\urlAnec107249}{
	\begin{overpic}[ scale = 0.25]
		{\pLocalGraphics photos/anecdata/anec-107249}
		\put(5,22) {\color{yellow}{Starlink satellites}}
		\put(5,12) {\color{white}{moving east to west}}
		\put(5,2) {\color{white}{... about 50 in total}}
	\end{overpic}} &
		%
	\href{\urlAnec 105783}{
	\begin{overpic}[ scale = 0.25]
		{\pLocalGraphics photos/anecdata/anec-105783}
		\put(5,5) {\color{yellow}{Space X launch, 2/1}}
	\end{overpic}}
		%	
\end{tabular}
\end{center}
\label{tab:anec-notes}
\end{table}%
}
\end{frame}


\begin{frame}\frametitle{\referenceStreak \href{https://anecdata.org/user/StenOdenwald/}{ @StenOdenwald}}
\begin{table}[htp]
%\caption{default}
\begin{center}
\begin{tabular}{ccc}
		%
	\href{\urlAnec 269188}{
	\begin{overpic}[ scale = 0.0265]
		{\pLocalGraphics photos/anecdata/anec-269188}
	\end{overpic}} &
		%
	\href{\urlAnec 106328}{
	\begin{overpic}[ scale = 0.25]
		{\pLocalGraphics photos/anecdata/anec-106328}
	\end{overpic}} &
		%
	\href{\urlAnec 107686}{
	\begin{overpic}[ scale = 0.25]
		{\pLocalGraphics photos/anecdata/anec-107686a}
	\end{overpic}} \\[15pt]
		%
	\href{\urlAnec 107686}{
	\begin{overpic}[ scale = 0.25]
		{\pLocalGraphics photos/anecdata/anec-107686b}
	\end{overpic}} &
		%
	\href{\urlAnec 107686}{
	\begin{overpic}[ scale = 0.25]
		{\pLocalGraphics photos/anecdata/anec-107686c}
	\end{overpic}} &
		%
	\href{\urlAnec 107686}{
	\begin{overpic}[ scale = 0.25]
		{\pLocalGraphics photos/anecdata/anec-107686d}
	\end{overpic}}
		%	
\end{tabular}
\end{center}
\label{tab:anec-01}
\end{table}%
\end{frame}

\begin{frame}\frametitle{\referenceStreak \href{https://anecdata.org/user/StenOdenwald/}{ @StenOdenwald}}
\begin{table}[htp]
%\caption{default}
\begin{center}
\begin{tabular}{ccc}
		%
	\href{\urlAnec 105719}{
	\begin{overpic}[ scale = 0.25]
		{\pLocalGraphics photos/anecdata/anec-105719a}
	\end{overpic}} &
		%
	\href{\urlAnec105719}{
	\begin{overpic}[ scale = 0.25]
		{\pLocalGraphics photos/anecdata/anec-105719b}
	\end{overpic}} &
		%
	\href{\urlAnec105719}{
	\begin{overpic}[ scale = 0.25]
		{\pLocalGraphics photos/anecdata/anec-105719c}
	\end{overpic}} \\[15pt]
		%
	\href{\urlAnec105719}{
	\begin{overpic}[ scale = 0.25]
		{\pLocalGraphics photos/anecdata/anec-105719d}
	\end{overpic}} &
		%
	\href{\urlAnec105719}{
	\begin{overpic}[ scale = 0.25]
		{\pLocalGraphics photos/anecdata/anec-105719e}
	\end{overpic}} &
		%
	\href{\urlAnec105719}{
	\begin{overpic}[ scale = 0.25]
		{\pLocalGraphics photos/anecdata/anec-105719f}
	\end{overpic}}
		%	
\end{tabular}
\\
\small{{\color{blue}{StarLink satellites}} this morning between 5:24 and 5:38 am using my Galaxy S9Plus set at 10sec and ISO800}
\end{center}
\label{tab:anec-02}
\end{table}%
\end{frame}



%     %     %     %     %     %     %     %     %
\subsection{Amateur Astronomers}
\begin{frame}\frametitle{\referenceGordon}
\begin{table}[htp]
%\caption{default}
\begin{center}
\begin{tabular}{cc}
	%
	\Gunagulla{
	\begin{overpic}[ scale = 0.216]
		{\pLocalGraphics photos/C2011W3201112293X30s}
	\end{overpic}} &
	%
	\Gunagulla{
	\begin{overpic}[ scale = 0.9]
		{\pLocalGraphics photos/C2011W3-85mm-20111225}
	\end{overpic}}
	%
\end{tabular}
\end{center}
\label{tab:Garradd}
\end{table}%
	% Comet Lovejoy, a meteor, a satellite and the Milky way 29/12/2011
\end{frame}

\begin{frame}\frametitle{\referenceGordon}
{\tiny{
\begin{table}[htp]
%\caption{default}
\begin{center}
\begin{tabular}{cc}
	%
	\Gunagulla{
	\begin{overpic}[ scale = 0.216]
		{\pLocalGraphics photos/C2011W3201112293X30s}
		\put(5,6) {\color{white}{Comet Lovejoy, a meteor, a {\color{yellow}{satellite}} }}
		\put(5, 2) {\color{white}{and the Milky way 29/12/2011}}
	\end{overpic}} &
	%
	\Gunagulla{
	\begin{overpic}[ scale = 0.9]
		{\pLocalGraphics photos/C2011W3-85mm-20111225}
		\put(5,6) {\color{white}{Comet Lovejoy and }}
		\put(5, 2) {\color{white}{tumbling {\color{yellow}{satellite}} trail}}
	\end{overpic}}
	%
\end{tabular}
\end{center}
\label{tab:Garradd}
\end{table}
}}
\end{frame}
% Comet Lovejoy in detail, note the elongated streak above where the nucleus should be, and tumbling satellite trail

\begin{frame}\frametitle{\href{https://en.wikipedia.org/wiki/Gorden_J._Garradd}{Gorden J. Garradd}}
\center
{\tiny{
\begin{table}[htp]
%\caption{default}
\begin{center}
\begin{tabular}{cc}
	%
	\href{https://www.satobs.org/image/centur_l.jpg}{
	\begin{overpic}[ scale = 0.45 ]
		{\pLocalGraphics photos/centur_l}
		\put(0.5, 2) {\color{white}{Cassini/Huygens probe and Titan IV Centaur booster venting after separation}}
	\end{overpic}} &
	%
\end{tabular}
\end{center}
\label{tab:Garradd-01}
\end{table}
%\\
%Cassini/Huygens probe and Titan IV Centaur booster venting after separation
}}
\end{frame}


%     %     %     %     %     %     %     %     %
\subsection{Federation of American Scientists}

\begin{frame}\frametitle{Russian Photos --  Adaptive Optics}
\center
	\href{https://spp.fas.org/military/program/track/rao.pdf}{
	\begin{overpic}[ scale = 0.25]
		{\pLocalGraphics photos/russian-ao}
		\put(102, 55) {\colorbox{white}{\tiny{Open Source}}}
		\put(102, 50) {\colorbox{white}{\tiny{\texttt{https://spp.fas.org}}}}
		\put(-28, 30) {\colorbox{white}{\tiny{Images shown at}}}
		\put(-28, 25) {\colorbox{white}{\tiny{common scale}}}
		\put(-25, 43) {\colorbox{white}{\tiny{One arc second}}}
		\put(-26, 71) {\colorbox{white}{\href{RE: Russian adapive optics pix of satellites}{\tiny{Original images}}}}
	\end{overpic}}
\end{frame}

\endinput  %  ==  ==  ==  ==  ==  ==  ==  ==  ==

Lifestyle Astro Bob
Astro Bob: Satellite streaks spoil Hubble photos
The spectacular increase in the number of satellites in recent years is now affecting Hubble images.
Hubble satellite trails
Three satellites left light trails when they passed through this Hubble photo of distant galaxies.Contributed / NASA, ESA, Kruk et al., Nature 2023
Bob King
By Bob King
March 05, 2023 at 2:52 PM
 Share
 Opinion
If you thought the ever-increasing number of satellites affected your enjoyment of the night sky, it's also afflicting the Hubble and other orbiting observatories. Volleys of satellites, particularly the SpaceX Starlinks , which now number around 3,600, photobomb the images and contaminate valuable scientific data. Ground-based observatories are experiencing the same bombardment, infuriating professional and amateur astronomers alike.

Starlink satellites
A train of Starlink satellites crosses the sky over Duluth, Minnesota on March 4, 2020. Typically, 60 satellites are deployed at a time during each Starlink launch. The first took place in 2019.Contributed / Bob King
While SpaceX has tried to mitigate the problem by adjusting the tilt of the Starlinks so they reflect less sunlight and therefore appear fainter, their sheer number has become overwhelming. For me, they're just a nuisance. When studying a nebula or planet through the telescope, a satellite (or a stream of satellites) invariably meanders across the field of view, ruining both the sight and my concentration. If you're doing research, it's worse.

In a recent paper published in Nature , Sandor Kruk (Max Planck Institute for Extraterrestrial Physics, Germany) and his team discovered a disturbing trend in Hubble photos taken between 2002 and 2021.

Hubble trail
During a time exposure, a passing satellite leaves a trail as seen in this Hubble photo. Some of the streaks can be removed with software. Wider trails from satellites passing closer to the telescope can't. The satellite's brightness can also wash out fainter objects.Contributed / NASA, ESA, Kruk et al., Nature 2023
With the help of the citizen-science group Asteroid Hunters and a pair of data-mining algorithms, they found that satellite trails crossed 2.7\% of the Hubble images (with a typical exposure time of 11 minutes) and that the fraction of photos with trails increases by roughly 50\% over the time frame. By 2021, you had a 6\% chance of running across a streak in a Hubble image. Those numbers are in line with the 40\% uptick in the number of satellites sent into orbit between 2005 and 2021.

I hate to sound like the Grim Reaper, but things are only getting worse. Satellite "constellations" like Starlink and OneWeb , both of which provide broadband internet service, continue to grow. Within a few years, SpaceX hopes to have 42,000 Starlinks in orbit. OneWeb has 542 as of early January 2023 but plans to expand its fleet to 7,000. China aims to launch nearly 13,000 satellites to provide world-wide internet services to compete with Elon Musk.

The Combined Force Space Component Command (CFSCC) at Vandenberg Space Force Base in California currently tracks and catalogs 32,750 objects 19.7 inches (50 cm) or larger in orbit. The European Space Agency's space debris office estimates there are 36,500 objects larger than 3.9 inches (10cm); 1 million objects between 0.4-3.9 inches (1-10cm), and a mind-boggling 130 million objects between 0.04-0.4 inch (1mm to 1cm) circling the planet.


Hubble isn't alone. NASA's Near-Earth Object Widefield Infrared Survey Explorer (NEOWISE) and the European Exoplanet satellite (CHEOPS), which makes precision observations of planets around other stars, are also affected. Not to mention the International Space Station and China's Tiangong Space Station.

Skyrocketing satellite numbers could result in additional space debris and its consequences — collisions with active satellites and space telescopes. Thankfully, the recently launched Webb Space Telescope, located nearly a million miles from Earth, is unaffected by satellites.

Hubble thick trail
A satellite located closer to the Hubble makes a wide, bright trail through a photo. Contributed / NASA, ESA, Kruk et al., Nature 2023
Software can remove some of the narrow streaks, but not the wide ones, which occur when a satellite passes closer to the Hubble and etches a brighter trail. The thick band of light that results makes the photo essentially worthless. Kruk and his colleagues estimate that by the 2030s satellites will mark between 20\% and possibly as high as 50\% of Hubble's images.

Satellite-watching can be fun. I first enjoyed it as a kid. My friends and I would sprawl out on the sidewalk, hands cupped behind our heads, and wait for them to appear. Nowadays, a straight line of 60 Starlinks parading across the sky continues to be a mesmerizing sight. But if we're not careful we may find it harder and harder to escape what we make. Boxed in, we'd lose more than just data.
