% \input{\pathfigures  least-squares-solutions/"fig fredholm angle"}

\renewcommand{\lcosth}[0]		{3.87084}
\renewcommand{\lsinth}[0]		{3.16491}
\renewcommand{\ltic}[0]			{0.15}
\renewcommand{\lsize}[0]			{5.8}
\renewcommand{\lrsize}[0]		{5}

\begin{figure}[t]
%
\begin{center}
{\tiny{
\begin{tikzpicture}[scale=0.75,
	opencircle/.style={fill=white, draw=black, thick},
	closedcircle/.style={fill=black, draw=black, thick},
	myarrow/.style={-Stealth, thick},
	circle/.style={thick}]
	opencircle/.style={fill=white, draw=black, thick},
	closedcircle/.style={fill=black, draw=black, thick},
	myarrow/.style={-Stealth, thick},
	circle/.style={thick}]
		%
		% gray pi shape marking angle
	\fill[gray!20] (0,0) -- (\lrsize,0) arc (0: 39.27: \lrsize) -- (\lcosth,\lsinth) -- cycle;
		% arc for theta_{F}
	\draw [gray, domain=0:39.27] plot ({2*cos(\x)}, {2*sin(\x)});
		% draw and label null space axis
	\draw[myarrow, red]  (0, 0) -> (0, \lsize) node[black, above] {$\rnlla{*}$};
		% draw and label range space axis
	\draw[myarrow, blue] (0, 0) -> (\lsize, 0) node[black, right] {$\brnga{}$};
		% data vector
	\draw[myarrow, black] (0, 0) -> (\lcosth, \lsinth) node[black, above] {$b$}; % 39.27
		% data vector: null space component
	\draw[red] (\ltic, \lsinth) -- (-\ltic, \lsinth) node[left] {$\normt{\datanull}$};
	\draw[dotted, red] (0, \lsinth) -- (\lcosth, \lsinth);
	\draw[myarrow, red]  (\lcosth, 0) -> (\lcosth,\lsinth) node[] {};
	\node [] at ( \lcosth + 1.1,\lsinth + 0.15,  )	{\ \ $\datanull{}$};
		% data vector: range space component
	\draw[blue] (\lcosth,\ltic) -- (\lcosth, -\ltic) node[below] {$\normt{\datarange}$};
	\draw[myarrow, blue]  (0, 0) -> (\lcosth,0) node[] {};
	\node [] at ( \lcosth - 0.45, 	0.3  )	{$\datarange{}$};
		% unit circle
	\draw [black!50,thick,domain=0:90] plot ({5*cos(\x)}, {5*sin(\x)});
		% label  for theta_{F}
	\node [] at ( 1.4, 0.45  )				{$\theta_{F}$};
	\draw[] (2.5, 5.1)			node[] {$b = \dataresolved$};
	\node [] at ( \lsize + 3, \lsize * 0.35  )			{$\begin{array}{lclcl}\normt{\datarange} &=& \normt{b} \cos \theta_{F} \\[5pt] \normt{\datanull} &=& \normt{b} \sin \theta_{F} \end{array}$};
		% perpendicular indicator between subspaces
	\perpso{1/3}{\lcosth}
		%
\end{tikzpicture}}}
\end{center}
\caption
[Geometry of the data vector in least squares]
{Geometry of the data vector in the least squares problem.}
\label{fig:geometry of the data vector}
\end{figure}


\endinput  %  ==  ==  ==  ==  ==  ==  ==  ==  ==

% https://tex.stackexchange.com/questions/340257/how-to-create-radial-shading-on-a-portion-of-a-disc
% How to create radial shading on a portion of a disc?
