% % % % \input{\pLocalComponents figures/fig-data-vector-resolution}

\renewcommand\xlen{10}
\renewcommand\ylen{7}

\begin{figure}[htp]
	\centering
	\scalebox{0.8}{
	\begin{tikzpicture}[>=latex]
		%
		\draw [very thick] [blue!50]	[->]	(0,0) -- (\xlen, 0);
		\draw [very thick] [red!50]		[->]	(0,0) -- (0, \ylen);
		\draw [very thick] [black]		[->]	(0,0) -- (\xresult,\yresult);
		%
		\draw [very thick] [blue]		[->]	(0,0) -- (\xresult, 0);		
		\draw [very thick] [red]		[->]	(\xresult, 0) -- (\xresult,\yresult);
		%
		\draw [black] ( 0, \bx ) -- ( \bx, \bx );
		\draw [black] ( \bx, 0 ) -- ( \bx, \bx );
		% label axes
		\node [] at ( \xlen - 0.25, 0.35 )					{$\brnga{}$};
		\node [] at ( 0.75, \ylen - 0.15 )					{$\rnlla{*}$};
		%
		\node [] at ( \xresult - 0.25, \yresult + 0.25 )		{$b$};
		\node [] at ( \xresult + 0.75,  \yresult - 0.25)		{$\datanull$};
		\node [] at ( \xresult - 0.5,  0 + 0.35 )				{$\datarange$};
		\node [] at ( \xlen / 2 - 2.1, \yresult / 2 + 0.5 )		{$b = \dataresolved$};
		%
	\perpso{1/3}{\xresult}
	\end{tikzpicture}}
%\caption
%[The orthogonal structure of the measurement space.]
%{The orthogonal structure of the measurement space provides an instinctive decomposition for the data vector $b$ in terms of range space and null space components. As such, it provides a natural framework for computations in $\bigL$ and $\littlel$.}
\label{fig:data-vector-resolution}
\end{figure}	

\renewcommand\xlen{10}
\renewcommand\ylen{7}

\begin{figure}[htp]
	\centering
	\scalebox{0.8}{
	\begin{tikzpicture}[>=latex]
		%
		\draw [very thick] [blue!50]	[->]	(0,0) -- (\xlen, 0);
		\draw [very thick] [red!50]		[->]	(0,0) -- (0, \ylen);
		\draw [very thick] [black]		[->]	(0,0) -- (\xresult,\yresult);
		%
		\draw [very thick] [blue]		[->]	(0,0) -- (\xresult, 0);		
		\draw [very thick] [red]		[->]	(\xresult, 0) -- (\xresult,\yresult);
		%
		\draw [black] ( 0, \bx ) -- ( \bx, \bx );
		\draw [black] ( \bx, 0 ) -- ( \bx, \bx );
		% label axes
		\node [] at ( \xlen - 0.25, 0.35 )					{$\brnga{}$};
		\node [] at ( 0.75, \ylen - 0.15 )					{$\rnlla{*}$};
		%
		\node [] at ( \xresult - 0.25, \yresult + 0.25 )		{$b$};
		\node [] at ( \xresult + 0.75,  \yresult - 0.25)		{$\datanull$};
		\node [] at ( \xresult - 0.5,  0 + 0.35 )				{$\datarange$};
		\node [] at ( \xlen / 2 - 2.1, \yresult / 2 + 0.5 )		{$b = \dataresolved$};
		%
	\perpso{1/3}{\xresult}
	\end{tikzpicture}}
%\caption
%[The orthogonal structure of the measurement space.]
%{The orthogonal structure of the measurement space provides an instinctive decomposition for the data vector $b$ in terms of range space and null space components. As such, it provides a natural framework for computations in $\bigL$ and $\littlel$.}
\label{fig:data-vector-resolution}
\end{figure}	

\renewcommand\xlen{10}
\renewcommand\ylen{7}

\begin{figure}[htp]
	\centering
	\scalebox{0.8}{
	\begin{tikzpicture}[>=latex]
		%
		\draw [very thick] [blue!50]	[->]	(0,0) -- (\xlen, 0);
		\draw [very thick] [red!50]		[->]	(0,0) -- (0, \ylen);
		\draw [very thick] [black]		[->]	(0,0) -- (\xresult,\yresult);
		%
		\draw [very thick] [blue]		[->]	(0,0) -- (\xresult, 0);		
		\draw [very thick] [red]		[->]	(\xresult, 0) -- (\xresult,\yresult);
		%
		\draw [black] ( 0, \bx ) -- ( \bx, \bx );
		\draw [black] ( \bx, 0 ) -- ( \bx, \bx );
		% label axes
		\node [] at ( \xlen - 0.25, 0.35 )					{$\brnga{}$};
		\node [] at ( 0.75, \ylen - 0.15 )					{$\rnlla{*}$};
		%
		\node [] at ( \xresult - 0.25, \yresult + 0.25 )		{$b$};
		\node [] at ( \xresult + 0.75,  \yresult - 0.25)		{$\datanull$};
		\node [] at ( \xresult - 0.5,  0 + 0.35 )				{$\datarange$};
		\node [] at ( \xlen / 2 - 2.1, \yresult / 2 + 0.5 )		{$b = \dataresolved$};
		%
	\perpso{1/3}{\xresult}
	\end{tikzpicture}}
%\caption
%[The orthogonal structure of the measurement space.]
%{The orthogonal structure of the measurement space provides an instinctive decomposition for the data vector $b$ in terms of range space and null space components. As such, it provides a natural framework for computations in $\bigL$ and $\littlel$.}
\label{fig:data-vector-resolution}
\end{figure}	

\renewcommand\xlen{10}
\renewcommand\ylen{7}

\begin{figure}[htp]
	\centering
	\scalebox{0.8}{
	\begin{tikzpicture}[>=latex]
		%
		\draw [very thick] [blue!50]	[->]	(0,0) -- (\xlen, 0);
		\draw [very thick] [red!50]		[->]	(0,0) -- (0, \ylen);
		\draw [very thick] [black]		[->]	(0,0) -- (\xresult,\yresult);
		%
		\draw [very thick] [blue]		[->]	(0,0) -- (\xresult, 0);		
		\draw [very thick] [red]		[->]	(\xresult, 0) -- (\xresult,\yresult);
		%
		\draw [black] ( 0, \bx ) -- ( \bx, \bx );
		\draw [black] ( \bx, 0 ) -- ( \bx, \bx );
		% label axes
		\node [] at ( \xlen - 0.25, 0.35 )					{$\brnga{}$};
		\node [] at ( 0.75, \ylen - 0.15 )					{$\rnlla{*}$};
		%
		\node [] at ( \xresult - 0.25, \yresult + 0.25 )		{$b$};
		\node [] at ( \xresult + 0.75,  \yresult - 0.25)		{$\datanull$};
		\node [] at ( \xresult - 0.5,  0 + 0.35 )				{$\datarange$};
		\node [] at ( \xlen / 2 - 2.1, \yresult / 2 + 0.5 )		{$b = \dataresolved$};
		%
	\perpso{1/3}{\xresult}
	\end{tikzpicture}}
%\caption
%[The orthogonal structure of the measurement space.]
%{The orthogonal structure of the measurement space provides an instinctive decomposition for the data vector $b$ in terms of range space and null space components. As such, it provides a natural framework for computations in $\bigL$ and $\littlel$.}
\label{fig:data-vector-resolution}
\end{figure}	