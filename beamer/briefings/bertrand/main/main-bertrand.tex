% typeset: Pdftex
% Afterwards compile with pdflatex > bibtex > pdflatex > pdflatex.
% in TeXShop preferences, changed edit from bibtex to biber
% beamer likes biber
% latex likes bibtex

% https://tex.stackexchange.com/questions/270633/beamer-and-the-pause-command
% https://tex.stackexchange.com/questions/1423/is-there-a-nice-way-to-compile-a-beamer-presentation-without-the-pauses
%\documentclass[handout]{beamer}
% https://tex.stackexchange.com/tags/accents/info
%
\documentclass[]{beamer}
% point to content libraries
\newcommand{\pGlobal}			{../../../global/}
\newcommand{\pGlobalSetup}		{\pGlobal setup-global/}	

%% % % \input{..//probe.tex}
%
\begin{frame}\frametitle{Probe}
	``..//probe.tex''
\end{frame}

\endinput  %  ==  ==  ==  ==  ==  ==  ==  ==  ==

%
\begin{frame}\frametitle{Probe}
	``..//probe.tex''
\end{frame}

\endinput  %  ==  ==  ==  ==  ==  ==  ==  ==  ==

%
\begin{frame}\frametitle{Probe}
	``..//probe.tex''
\end{frame}

\endinput  %  ==  ==  ==  ==  ==  ==  ==  ==  ==


\setbeamercovered{transparent=10} % pause activated text
\usepackage{transparent}
\usepackage{seqsplit}
%\usepackage[T1]{fontenc}
%\usepackage[utf8]{inputenc}

% load LaTeX packages and personal macros
\input{\pGlobalSetup setup-global-beamer}
\input{\pLocalSetup setup-local}
\pgfplotsset{compat=1.6}
%\usepackage{MnSymbol,wasysym}
\bibliography{\pSections bertrand.bib}
%
\title
[Bertrand's Theorem]
{\refBertrand: \\Closed, Perturbatively Stable Orbits}

\author[Daniel Topa]{\Topa \\ \TopaEmail}
\institute{\unmmath} 
\medskip

\date{\today}
%\date{2018-08-31}

\begin{document}
% slide numbers
\input{\pLocalSetup slide-numbers}

\begin{frame}
	\titlepage
\end{frame}
	
%\begin{frame}\frametitle{Applied Linear Algebra -- Triennial Conference}
%\center
%	\href{https://www.siam.org/conferences/cm/conference/la24}{
%	\begin{overpic}[ scale = 0.175 ]
%		{\pLocalGraphics logos/logo-02}
%		%\put(50, 50)	{\colorbox{white}{$a+b$}}
%	\end{overpic}}
%\end{frame}

	%\input{\pSections "ssec-me"}
	%\include{\pSections "sec splash"}

\begin{frame}\frametitle{Overview}
	\tableofcontents[hideallsubsections]
\end{frame}

% main sections
	\input{\pSections "sec-background"}
	\input{\pSections "sec-concepts"}
	\input{\pSections "sec-bertrand"}
	%\input{\pSections "sec-proofs"}
	%\input{\pSections "sec-goldstein"}
	%\input{\pSections "sec-siam"}
%	\input{\pSections "sec-quo-vadis"}
	\input{\pSections "sec-backup"}
%\section{Bibliography}
%\begin{frame}[t,allowframebreaks]
%\frametitle{Bibliography}
%\nocite{strang}
%\printbibliography[heading=none]
%\end{frame}

{\tiny{
%\begin{frame}[allowframebreaks,shrink=10]\frametitle{Bibliography}
\begin{frame}[allowframebreaks]\frametitle{Bibliography}
	%\nocite{*}
	\printbibliography
\end{frame}}}

% info slide during questions
\begin{frame}
	\titlepage
\end{frame}

\end{document}

%\tiny
%\scriptsize
%\footnotesize
%\small
%\normalsize
%\large
%\Large
%\LARGE
%\huge
%\Huge

%\, thin space (normally 1/6 of a quad);
%\> medium space (normally 2/9 of a quad);
%\; thick space (normally 5/18 of a quad);

\begin{frame}\frametitle{Frame Title}
\begin{enumerate}
	\item 
	\item 
	\item 
\end{enumerate}
\end{frame}

\begin{frame}\frametitle{Frame Title}
\begin{equation}
	\begin{array}{ccc} 
			%
			%
			%
	\end{array}
%\label{eq:}
\end{equation}
\end{frame}

\begin{frame}\frametitle{ }
\center
	\href{url}{
	\begin{overpic}[ scale = 1.0 ]
	{\pLocalGraphics graphic-file}
		%\put(-7,-10){Auxiliary text.}
	\end{overpic}}
\end{frame}

\begin{frame}\frametitle{Frame Title}
\begin{table}[htp]
%\caption{default}
\begin{center}
	\begin{tabular}{cc}
		%
		%
		%
	\end{tabular}
\end{center}
%\label{tab:label}
\end{table}%
\end{frame}

