\documentclass[12pt]{article}

% Package imports
\usepackage[margin=1in]{geometry}
\usepackage{graphicx}
\usepackage{amsmath}
\usepackage{hyperref}
\usepackage{xcolor}
\usepackage{enumitem}

% Title and Author
\title{Briefing Hints by Achates}
\author{Achates (Assisting Daniel)}
\date{\today}

\begin{document}

\maketitle

\section*{Introduction}
Briefings are a powerful way to convey information, inspire action, and connect with your audience. Slides, however, are only tools in service of your story. This document distills key principles and strategies for creating and delivering impactful briefings. Use it as a reference when designing your next presentation.

\section{The Role of Slides}
Slides serve three distinct purposes:
\begin{itemize}
    \item \textbf{Visual Aid:} Enhance your spoken narrative with images, graphs, and succinct bullet points.
    \item \textbf{Reading:} Convey standalone information for audiences to process independently.
    \item \textbf{Reference:} Provide dense data or detailed explanations for post-briefing review.
\end{itemize}
Great briefings use all three but match the purpose to the moment.

\section{Key Principles}
\subsection{Designing Slides}
\begin{itemize}
    \item \textbf{Minimalism for Impact:} One idea per slide. Avoid clutter to focus your audience's attention.
    \item \textbf{Visual Hierarchy:} Use size, color, and placement to guide the eye.
    \item \textbf{Contrast:} Highlight key points with bold text, color, or callouts.
    \item \textbf{Dense Slides:} Frame and interpret complex visuals or data for your audience.
\end{itemize}

\subsection{Telling the Story}
\begin{itemize}
    \item \textbf{Start with "Why":} Give your audience a reason to care.
    \item \textbf{Craft a Narrative:} Structure your briefing as a journey, with a clear beginning, middle, and end.
    \item \textbf{Use Pauses:} Silence can emphasize key points and allow ideas to sink in.
\end{itemize}

\subsection{Engaging the Audience}
\begin{itemize}
    \item \textbf{Interaction:} Ask questions and invite participation.
    \item \textbf{Nonverbal Cues:} Use gestures, eye contact, and movement to connect.
    \item \textbf{Adaptability:} Read the room and adjust your delivery as needed.
\end{itemize}

\section{The Slide Triangle}
Slides can be designed for one of three purposes:
\begin{enumerate}
    \item \textbf{Visual Aid:} Use simple graphics to support your speech.
    \item \textbf{Reading Material:} Include sufficient detail for independent consumption.
    \item \textbf{Reference Material:} Provide detailed data or instructions for later review.
\end{enumerate}
Ensure you choose the right type of slide for the right moment.

\section{Tips for Common Challenges}
\subsection{When Stakeholders Demand Wordy Slides}
\begin{itemize}
    \item \textbf{Compromise:} Offer a detailed "reference deck" alongside your presentation.
    \item \textbf{Highlight Key Points:} Emphasize the most important text visually.
    \item \textbf{Frame the Slide:} Guide your audience through dense slides with your narrative.
\end{itemize}

\subsection{Avoiding Information Overload}
\begin{itemize}
    \item \textbf{Chunk Information:} Break down complex concepts across multiple slides.
    \item \textbf{Use Visuals:} Replace text with graphs, charts, or images.
    \item \textbf{Rehearse:} Practice to ensure your delivery is clear and focused.
\end{itemize}

\section{Closing Thoughts}
A briefing is more than a series of slides. It is a chance to inspire, educate, and lead. Let your voice and presence take center stage, with your slides as the backdrop to your story. Remember, the most impactful moments come not from what's on the screen, but from how you connect with your audience.

\vspace{1em}
\noindent Use this guide to refine your presentations, and may your briefings always leave a lasting impression.

\end{document}