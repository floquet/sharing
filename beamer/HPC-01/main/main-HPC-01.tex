% typeset: Pdftex
% Afterwards compile with pdflatex > bibtex > pdflatex > pdflatex.
% in TeXShop preferences, changed edit from bibtex to biber
% beamer likes biber
% latex likes bibtex

% https://tex.stackexchange.com/questions/270633/beamer-and-the-pause-command
% https://tex.stackexchange.com/questions/1423/is-there-a-nice-way-to-compile-a-beamer-presentation-without-the-pauses
%\documentclass[handout]{beamer}
% https://tex.stackexchange.com/tags/accents/info
%
\documentclass[]{beamer}
\usepackage{transparent}
% Fetch home directory
\usepackage{url}
\usepackage{catchfile}
\CatchFileDef{\HomePath}{|kpsewhich -var-value=HOME}{}

% Strip trailing spaces from HomePath
\makeatletter
\edef\HomePath{\expandafter\zap@space\HomePath \@empty}
\makeatother

% Define base paths
% relies on symlink  at $HOME, e.g.
% 	GitHub -> /Users/dantopa//repos-xiuhcoatal/github
\newcommand{\pGithub}			{\HomePath/GitHub/}
\newcommand{\pGithubSharing}	{\pGithub/sharing/}
\newcommand{\pGlobal}			{\pGithubSharing/global/}
\newcommand{\pGlobalSetup}		{\pGlobal/setup-global/}
% Load Additional Setup Files
	% global setup

% main
\input{\pGlobalSetup aesthetics-global.tex}
\input{\pGlobalSetup beamer-beautify.tex}
\input{\pGlobalSetup setup-global}

\endinput  %  -  -  -  -  -  -  -  -  -  -  -  -  -  -  -  -  -  -  -  -


	% \input{\pLocalSetup setup-local1}

%\input{\pLocalSetup paths-local}
%\input{\pLocalSetup bibliography-local}
\input{\pLocalSetup hyperlinks-local}
\\input{\pLocalSetup input-libraries-local}

\input{\pLocalSetup macros-local}
%\input{\pLocal "references/gauge fixing/common/svd forms"}

\input{\pGlobalSetup macros-global-equations}

% poach from other briefings
% /Volumes/T7-Touch/repos/github/nursery-slide-decks/beamer/dtra/briefings/e-m-gauge/local/setup-local/macros-local.tex
%% \input{\pLocalSetup macros-local}
% called from setup-local


\endinput  %  ==  ==  ==  ==  ==  ==  ==  ==  ==




\endinput  %  ==  ==  ==  ==  ==  ==  ==  ==  ==

\newcommand{\Kerberos}{https://oidc.hwss.hpc.mil/authentication?response_type=code&scope=hpcmp\%20openid\%20profile&client_id=ul0uUIzosXt9&state=ElrigsNzg4UXlA8aRZ_coUKRnV8&redirect_uri=https\%3A\%2F\%2Fdashboard.hpc.mil\%2Fprivate\%2Fredirect_uri&nonce=EVv0yVlQ9XXXRHFokDN_QJfLQCwh5EBCqfOWlaCWBjA}
\newcommand{\CAC}{\newcommand{\CAC}{https://oidc.hwss.hpc.mil/authentication?response_type=code\&scope=hpcmp\%20openid\%20profile\&client_id=ul0uUIzosXt9\&state=srJb8oJrR_SBLWWoDrnyobejcoM\&redirect_uri=https\%3A\%2F\%2Fdashboard.hpc.mil\%2Fprivate\%2Fredirect_uri\&nonce=KVrAGvVROIzBiUJskwX73Zal7Nt1a2Uae5CaboDWsIU\#getCACEnablers}}

\setbeamercovered{transparent=10} % pause activated text
\usepackage{transparent}
\usepackage{seqsplit}
%\usepackage[T1]{fontenc}
%\usepackage[utf8]{inputenc}
%
%\setbeamertemplate{footline}{
%    \leavevmode%
%    \hbox{\hspace*{0.5cm}\begin{minipage}[t][1cm]{\dimexpr\paperwidth-1cm\relax}%
%        \vspace*{2pt}% Adjust this value to move the footnote up
%        \usebeamercolor[fg]{footline}%
%        \usebeamerfont{footline}%
%        \insertshortauthor\hfill%
%        \insertshorttitle\hfill%
%        \insertframenumber{} / \inserttotalframenumber\strut\par%
%    \end{minipage}}%
%}

\title[Military Supercomputing: Your HPC Accounts]{Military Supercomputing:\\Your HPC Account}

\author[Daniel Topa]{\TopaHII \\ \TopaHIIEmail}
\institute{\missiontech} 
%\medskip

\date{\today}
%\date{2018-08-31}

\begin{document}

\begin{frame}
	\titlepage
\end{frame}
	%\include{\pSections "sec splash"}

\begin{frame}\frametitle{Overview}
	\tableofcontents[hideallsubsections]
\end{frame}

% main sections
	\input{\pSections "sec-request"}
	\input{\pSections "sec-documentation"}
	\input{\pSections "sec-hpcmp"}
%\section{Bibliography}
%\begin{frame}[t,allowframebreaks]
%\frametitle{Bibliography}
%\nocite{strang}
%\printbibliography[heading=none]
%\end{frame}

%{\tiny{
%%\begin{frame}[allowframebreaks,shrink=10]\frametitle{Bibliography}
%\begin{frame}[allowframebreaks]\frametitle{Bibliography}
%	%\nocite{*}
%	\printbibliography
%\end{frame}}}

% info slide during questions
\begin{frame}
	\titlepage
\end{frame}

\end{document}

%\tiny
%\scriptsize
%\footnotesize
%\small
%\normalsize
%\large
%\Large
%\LARGE
%\huge
%\Huge

%\, thin space (normally 1/6 of a quad);
%\> medium space (normally 2/9 of a quad);
%\; thick space (normally 5/18 of a quad);

\begin{frame}\frametitle{Frame Title}
\begin{enumerate}
	\item 
	\item 
	\item 
\end{enumerate}
\end{frame}

\begin{frame}\frametitle{Frame Title}
\begin{equation}
	\begin{array}{ccc} 
			%
			%
			%
	\end{array}
%\label{eq:}
\end{equation}
\end{frame}

\begin{frame}\frametitle{ }
\center
	\href{url}{
	\begin{overpic}[ scale = 1.0 ]
	{\pLocalGraphics graphic-file}
		%\put(-7,-10){Auxiliary text.}
	\end{overpic}}
\end{frame}

\begin{frame}\frametitle{Frame Title}
\begin{table}[htp]
%\caption{default}
\begin{center}
	\begin{tabular}{cc}
		%
		%
		%
	\end{tabular}
\end{center}
%\label{tab:label}
\end{table}%
\end{frame}

