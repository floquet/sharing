\documentclass[12pt, oneside]{article}

\usepackage{amsmath}
\usepackage{amssymb}
\usepackage{amsthm}
	\newtheorem{thm}{Theorem}[section]
	\newtheorem{lem}[thm]{Lemma}
\usepackage{cancel}
\usepackage{colortbl}
\usepackage{esint}
\usepackage{geometry}
\usepackage{graphicx}
\usepackage[bookmarksnumbered=true]{hyperref}
\hypersetup{
colorlinks = true,
linkcolor = blue,
anchorcolor = blue,
citecolor = blue,
filecolor = blue,
urlcolor = blue}
\usepackage{multirow}
\usepackage{xcolor}
\usepackage[printwatermark]{xwatermark}
	\newwatermark[allpages,color=red!10,angle=45,scale=3,xpos=0,ypos=0]{DRAFT}

\geometry{letterpaper}

\newcommand{\abs}[1]			{\left| #1 \right|}
\newcommand{\brac}[1]			{ \left[  #1 \right] }
\newcommand{\inner}[1]			{ \langle #1 \rangle }
\newcommand{\mat}[2]			{\left[\begin{array} {#1}#2 \end{array}\right]}
\newcommand{\normi}[1]			{\left\lVert #1 \right\rVert_{\infty}}
\newcommand{\normt}[1]			{\left\lVert #1 \right\rVert_{2}}
\newcommand{\normts}[1]			{\normt{ #1 }^{2}}
\newcommand{\paren}[1]			{ \left(  #1 \right) }

\DeclareMathOperator*{\argmax}{arg\,max}
\DeclareMathOperator*{\argmin}{arg\,min}

\newcommand{\pathgraphics}[0]	{../graphics/}
\newcommand{\pathsections}[0]	{../sections/}


\title{Trigonometric Approximation of\\Radar Cross Section:\\A Quick Survey}
\author{\href{mailto:daniel.topa@ertcorp.com}{Daniel Topa}\\\href{https://www.ertcorp.com/}{ERT Corp}}
%\date{}							% Activate to display a given date or no date

\begin{document}
\maketitle

\abstract{The challenge is to expand the capability of the AFCAP dashboard by offering a refined representation of the radar cross section. Currently, the radar cross section is a constant. What is the best way to include a more realistic radar cross section computed from simulation? One avenue discussed herein is to approximate the radar cross section as trigonometric polynomial. A mathematical summary of the approximation process follows, concluding with a numeric example.}
\input{\pathsections "numbering conventions"}

\input{\pathsections "sec foundation"}
\input{\pathsections "sec afcap"}
\input{\pathsections "sec rcs"}
\input{\pathsections "sec fourier"}
\input{\pathsections "sec example"}
 \input{\pathsections "sec validation"}
 \input{\pathsections "sec references"}
\input{\pathsections "sec proofs"}


\end{document}  
