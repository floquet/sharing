% \input{\pathsections "sec validation"}

\section{Validation Exercise}
The previous section presented results for a seventh order decomposition to connect the theory to the computation. The goal of this section is to extend the example and provide data suitable for code validation. That is, to be able to check a computer code to full double precision.

The inputs for the amplitudes are quoted to full double precision in table \ref{tab:double-precision}. 
\begin{table}
	\begin{center}
		\begin{tabular}{llr@{.}l}
			$k$ & $\cos k\alpha$ & \multicolumn{2}{c}{amplitude}\\\hline
			$0$ & $1$             & $35$ & $237\,277\,592\,785\,375$ \\
			$1$ & $\cos \alpha$   & $1$  & $675\,022\,046\,154\,634\,1$ \\
			$2$ & $\cos 2\alpha$  & $-3$ & $434\,292\,043\,875\,646$ \\
			$3$ & $\cos 3\alpha$  & $-0$ & $866\,400\,204\,593\,953\,2$ \\
			$4$ & $\cos 4\alpha$  & $5$  & $368\,077\,161\,998\,904\,5$ \\
			$5$ & $\cos 5\alpha$  & $-1$ & $279\,517\,980\,533\,678\,3$ \\
			$6$ & $\cos 6\alpha$  & $1$  & $378\,826\,401\,472\,061\,7$ \\
			$7$ & $\cos 7\alpha$  & $-0$ & $366\,535\,373\,755\,586\,9$
		\end{tabular}
	\end{center}
\caption{The Fourier amplitudes for the data shown in figure \ref{fig:sample-data} and represented in \eqref{eq:fourier-expansion}.}
\label{tab:double-precision}
\end{table}

Given the amplitudes at full precision, it is straightforward to compute the results for special cases.
$$
\begin{array}{crl}
\alpha & \alpha^{\circ }& \multicolumn{1}{c}{\sigma(\alpha)} \\\hline
 0        & 0^{\circ}   & 37.712 \, 457 \, 599 \, 652 \, 11 \\[1pt]
 \pi / 4  & 45^{\circ}  & 32.311 \, 833 \, 530 \, 507 \, 33 \\[1pt]
 \pi / 2  & 90^{\circ}  & 42.660 \, 820 \, 397 \, 187 \, 86 \\[1pt]
 3\pi / 4 & 135^{\circ} & 27.426 \, 567 \, 331 \, 065 \, 61 \\[1pt]
 \pi      & 180^{\circ} & 39.387 \, 320 \, 625 \, 109 \, 275 \\[1pt]
 3\pi / 4 & 225^{\circ} & 27.426 \, 567 \, 331 \, 065 \, 61 \\[1pt]
 3\pi / 2 & 270^{\circ} & 42.660 \, 820 \, 397 \, 187 \, 86 \\[1pt]
 7\pi / 4 & 315^{\circ} & 32.311 \, 833 \, 530 \, 507 \, 33 \\[1pt]
 2 \pi    & 360^{\circ} & 37.712 \, 457 \, 599 \, 652 \, 11 \\
\end{array}
$$


\endinput  %  ==  ==  ==  ==  ==  ==  ==  ==  ==

