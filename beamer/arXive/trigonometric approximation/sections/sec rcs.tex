% \input{\pathsections "sec rcs"}

\section{Computing the Radar Cross Section}
The computation of the radar cross section is based on a simple scenario guided by Maxwell's equations. A radar illuminates the target with radar energy. The target, acting like an antenna, reradiates the energy and illuminates the radar receiver. The radar cross section is measurement of the difference between energy absorbed and energy radiated.

One form of the \href{https://www.radartutorial.eu/01.basics/Radar\%20Cross\%20Section.en.html}{radar cross section equation} is
\begin{equation}
	\sigma=4\pi r^{2} \frac{S_{r}}{S_{t}}
\end{equation}
where $r$ is the distance to the target, $S_{t}$ is the energy {\it{intercepted}} by the target and $S_{r}$ is the energy {\it{radiated}} by the target. The result has units of area and can be compared to an ideally reflecting sphere.

The areal measure $\sigma$ is computed through numerical simulation by the program Mercury MoM.

\subsection{Maxwell's Equations}
The mathematical foundation for the radar cross section computation is the set of Maxwell's equations.
Start with two conservative fields on a simply connected domain, the electric field $\mathbf{E}\colon\mathbb{R}^{3}\mapsto\mathbb{R}^{3}$, and the magnetic field $\mathbf{B}\colon\mathbb{R}^{3}\mapsto\mathbb{R}^{3}$. We can relax the requirements for the electric current $\mathbf{J}\colon\mathbb{R}^{3}\mapsto\mathbb{R}^{3}$.
The differential form of \href{https://physics.info/maxwell/}{Maxwell's equations} in \href{https://www.nist.gov/pml/weights-and-measures/metric-si/si-units}{SI} units is
\begin{equation}
	\begin{array}{rclcrcl}
			%
		\nabla \cdot \mathbf{E} &= &\frac{\rho}{\epsilon_{0}}
			%
		&&\nabla \times \mathbf{E} &= &- \partial_{t}\mathbf{B} \\ 
			%
		\nabla \cdot \mathbf{B} &= &0
			%
		&&\nabla \times \mathbf{B} &= &\mu_{0} \left( \mathbf{J} + \epsilon_{0} \partial_{t}\mathbf{E} \right) \\ 
			%
	\end{array}
\label{eq:Maxwell-differential}
\end{equation}
%
The fundamental physical constant in SI units are given in table \ref{tab:mu-eps}.
%
\begin{table}
	\begin{center}
		\begin{tabular}{lll}
		%
	$\epsilon_{0}$ 
		& \href{https://en.wikipedia.org/wiki/Vacuum_permittivity}{vacuum electric permittivity} 
		& \href{https://physics.nist.gov/cgi-bin/cuu/Value?ep0}{$8.854 \times 10^{-12} \text{ F m}^{-1}$} \\
		%
	$\mu_{0}$ 
		& \href{https://en.wikipedia.org/wiki/Vacuum_permeability}{vacuum magnetic permeability}
		& \href{https://physics.nist.gov/cgi-bin/cuu/Value?mu0}{$1.257 \times 10^{-7} \text{ N A}^{-2}$}
		%
		\end{tabular}
	\end{center}
\caption{Constants used in Maxwell's equations}
\label{tab:mu-eps}
\end{table}

\subsection{Method of Moments}
There is a rich mathematical toolkit for the solution of Maxwell's equations. Mercury MoM uses the \href{https://en.wikipedia.org/wiki/Computational_electromagnetics#Method_of_moments_element_method}{method of moments} (MoM), also known as the boundary element method (BEM).
\href{https://en.wikipedia.org/wiki/Maxwell\%27s\_equations}{Maxwell's equations in integral form}
\begin{equation}
	\begin{array}{cccccccccc}
			%
		\nabla \cdot \mathbf{E} &= &\displaystyle\frac{\rho}{\epsilon_{0}} 
			&\Rightarrow &\displaystyle{\oint_{d\Omega} \mathbf{E}\cdot d\mathbf{S}} &= &\displaystyle\int_{\Omega} \rho dV \\[10pt]
			%
		\nabla \times \mathbf{E} &= &- \partial_{t}\mathbf{B} 
			&\Rightarrow &\displaystyle\oint_{d\Omega} \mathbf{B}\cdot d\mathbf{S} &= &0 \\[10pt]
			%
		\nabla \cdot \mathbf{B} &= &0
			&\Rightarrow &\displaystyle\oint_{d\Sigma} \mathbf{E}\cdot d\mathbf{l} &= &\displaystyle-\frac{d}{dt} \mathbf{B}\cdot d\mathbf{S} \\[10pt]
			%
		\nabla \times \mathbf{B} &= &\mu_{0} \left( \mathbf{J} + \epsilon_{0} \partial_{t}\mathbf{E} \right) 
			&\Rightarrow & \displaystyle\oint_{d\Sigma} \mathbf{B}\cdot d\mathbf{l} &= &\displaystyle\mu_{0} \int_{\Sigma} \paren{ \mathbf{J}+ \epsilon_{0} \frac{d}{dt}  \mathbf{E} } \cdot d\mathbf{S} \\
			%
	\end{array}
\label{eq:Maxwell-differential}
\end{equation}
The method of moments creates dense matrices which implies quadratic growth in storage and computation time as the problem scales up. 

\subsection{Mercury MoM Example}
The power of Mercury MoM is the ability to simultaneously solve equations with millions of unknowns using a patented low rank approximation method for the linear system.

Internally, the Mercury MoM package resolves the signal emitted from the target into two complex electric fields: $\theta$ for the vertical component and $\phi$ for the horizontal component. As these fields are complex, they each have orthogonal real and imaginary components. 

The prescription for the mean total radar cross section is taken from the Sciacca report as the average of the total field energy
\begin{equation}
	\inner{ \sigma_{T} } = \tfrac{1}{2} \left( \theta^{*}\theta + \theta^{*}\phi + \phi^{*}\theta + \phi^{*}\theta \right),
\end{equation}
and is assigned areal units of squared meters m$^{2}$.


\endinput  %  ==  ==  ==  ==  ==  ==  ==  ==  ==
