% \input{\pathsections "sec afcap"}

\section{RCS and the AFCAP Dashboard}

\subsection{Current and Enhanced Capabilities of the Dashboard}
Currently, the AFCAP Dashboard treats target radar cross section as a constant. If we think of this as the constant term in the Fourier expansion, then we see it is the dominant term and a first step. Subsequent steps would be to improve the expansion and encode specific information about the target interaction with the radar. The enhancement is then extending the radar cross section from a scalar to a function of the yaw and pitch angles $\alpha$ and $\beta$ as summarized in table \ref{tab:current-enhanced}.

\begin{table}
	\begin{center}
		\begin{tabular}{ccc}
			Current RCS && Enhanced RCS \\\hline
			constant && $\sigma \paren{ \alpha, \beta }$ \\
			$\sigma\in\mathbb{R}^{+}$ && $\sigma\colon\mathbb{R}^{2}\mapsto\mathbb{R}$ \\
		\end{tabular}
	\end{center}
\caption{In a nutshell: enhance the AFCAP Dashboard by using a richer depiction of radar cross section.}
\label{tab:current-enhanced}
\end{table}

At this stage of development, the pitch angle will be kept constant at $\beta=\tfrac{\pi}{12}$.

Table \ref{tab:dashboard} details the operational differences between the current RCS procedure and the procedure for an enhanced RCS. Instead of selecting from a set of constant RCS values, the user will select an asset which then has detailed RCS spectrum. To unlock the spectrum, the user need to select the operating frequency of the radar, and the pitch angle (related to the intercept angle).

\begin{table}[htp]
	\begin{center}
		\begin{tabular}{rlrl}
				%
			& current && enhanced \\\hline
				%
			1. & select target type (fly or float) & 1. & select asset (A, B, \dots) \\
				%
			2. & select RCS ($2^{k}$, $k\in\mathbb{Z}^{+}$) & 2. & pick radar frequency in MHz, $3\le \nu \le 30$ \\
				%
				&&3. & pick yaw angle $\alpha\in\brac{-\pi,\pi}$ \\
				%
				&&4. & pitch angle is fixed at $\beta = \pi / 12$ \\
				%
		\end{tabular}
	\end{center}
\caption{AFCAP Dashboard capability for RCS.}
\label{tab:dashboard}
\end{table}%


\subsection{Dashboard details}
The current Dashboard presents a series of options to allow the user to specify a radar cross section. An example is shown in figure \ref{fig:rcs-dashboard}.
\begin{figure}[htbp]
	\begin{center}
		\includegraphics[ scale = 0.35 ]{\pathgraphics dashboard-01}
	\end{center}
\caption{AFCAP Dashboard tool for selecting radar cross section of an airborne target.}
\label{fig:rcs-dashboard}
\end{figure}

The RCS is stored in \texttt{ConfigRegion.xml}, and the example below shows a constant radar cross section of 16 m$^{2}$:\\

\indent\texttt{<Asset>} \\
\indent\texttt{\quad <Label>16sm</Label>} \\
\indent\texttt{\quad <ICONImage>Bald\_Eagle-sm.png</ICONImage>} \\
\indent\texttt{\quad \color{blue}{<crossSection>16</crossSection>}} \\
\indent\texttt{\quad <description>Aircraft</description>} \\
\indent\texttt{\quad <nominalSpeed>400</nominalSpeed>} \\
\indent\texttt{\quad <CIT>2.0</CIT>} \\
\indent\texttt{</Asset>} \\

\noindent The radar cross section then is used to compute the probability of detection.  The beginnings of the calculation look like this:\\

\indent\texttt{from frOPCclass.js} \\
	%
\indent\texttt{function plotProbability (ctx, jsonObj, jsonCoord, {\color{blue}{xSection}}, assetCIT, nomSpeed) \{} \\
	%
\indent\texttt{\quad \quad ...} \\
	%
\indent\texttt{\quad var {\color{purple}{xSecRadius}}   = Math.sqrt({\color{blue}{xSection}}/Math.PI)} \\
	%
\indent\texttt{\quad var sphereArea   = Math.PI * {\color{purple}{xSecRadius}} * {\color{purple}{xSecRadius}};} \\
	%
\indent\texttt{\quad \quad ...} \\
	%
\indent\texttt{\}}\\
	%
%In the past, pick and RCS. Now, pick a device which has a library of RCS functions at each frequency. Pick and angle. Get RCS.
%
%\begin{figure}
%	\begin{center}
%		\includegraphics[ scale = 1.15 ]{\pathgraphics mom-sample} 
%	\end{center}
%\caption{Sample output from Mercury MoM from a 3 MHz simulation showing the mean total radar cross section (purple dots) as a function of the nose angle. The red line represents the mean, the best constant approximation.}
%\label{fig:MOM-mean}
%\end{figure}


\endinput  %  ==  ==  ==  ==  ==  ==  ==  ==  ==
