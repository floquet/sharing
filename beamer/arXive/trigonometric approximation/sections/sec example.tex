% \input{\pathsections "sec example"}

\section{Numeric Example}

\subsection{Data}
A specific example will make the theoretical ideas concrete. The data source, the file \texttt{PTW.4112.txt} is described in more detail in table \ref{tab:PTW.4112.txt}. Note the precise computation of the mean value. The data is plotted in figure \ref{fig:sample-data} and shows the output from Mercury MoM: an RCS value for each degree of yaw, $\alpha$.

\begin{table}
	\begin{center}
		\begin{tabular}{ll}
			inpute file & \texttt{PTW.4112.txt} \\
			mean value & $35.248 \, 773 \, 556 \, 863 \, 84$ \\
			frequency & $\nu = 3$ MHz \\
			wavelength & $\lambda = 100$ m \\
			angular sampling range & $0^{\circ} - 360^{\circ}$ \\
			angular sampling interval & $1^{\circ}$
		\end{tabular}
	\end{center}
\caption{The data file from Mercury MoM used for this example.}
\label{tab:PTW.4112.txt}
\end{table}

\begin{figure}[htbp]
	\begin{center}
		\includegraphics[ scale = 0.65 ]{\pathgraphics sigma(3)+mean}
	\end{center}
\caption{Sample data set for $\nu=3$ MHz. The output from MoM (purple dots) is shown against the mean value (gray line).}
\label{fig:sample-data}
\end{figure}

\subsection{Results}

\subsubsection{Approximation sequence}
For comparison, an a sequence of decompositions was performed to provide a sketch of how the approximation improves.
\begin{table}[htp]
	\begin{center}
		\begin{tabular}{rr@{.}lr@{.}lr@{.}lr@{.}lr@{.}lr@{.}lr@{.}lr@{.}l}
				%
			$d$ & \multicolumn{2}{c}{$a_{0}$} & \multicolumn{2}{c}{$a_{1}$} & \multicolumn{2}{c}{$a_{2}$} & \multicolumn{2}{c}{$a_{3}$} & \multicolumn{2}{c}{$a_{4}$} & \multicolumn{2}{c}{$a_{5}$} & \multicolumn{2}{c}{$a_{6}$} & \multicolumn{2}{c}{$a_{7}$} \\\hline
				%
			0 & 35 & 2488 \\
				%
			1 & 35 & 2533 & 1 & 64293 \\
				%
			2 & 35 & 2626 & 1 & 62429 & $-3$ & 38356  \\
				%
			3 & 35 & 2601 & 1 & 62928 & $-3$ & 38855 & $-0$ & 912138  \\
				%
			4 & 35 & 2455 & 1 & 65863 & $-3$ & 41790 & $-0$ & 882795 & 5 & 38447  \\
				%
			5 & 35 & 2420 & 1 & 66561 & $-3$ & 42488 & $-0$ & 875809 & 5 & 37749 & $-1$ & 28893  \\
				%
			6 & 35 & 2383 & 1 & 67306 & $-3$ & 43233 & $-0$ & 868366 & 5 & 37004 & $-1$ & 28148 & 1 & 38079  \\
				%
			%\rowcolor[gray]{0.95}
			7 & 35 & 2373 & 1 & 67502 & $-3$ & 43429 & $-0$ & 866400 & 5 & 36808 & $-1$ & 27952 & 1 & 37883 & $-0$ & 366535 \\
		\end{tabular}
	\end{center}
\caption{Amplitudes for the approximation sequence shown in table \ref{tab:sequence}. In the ideal case the amplitudes $a$ would have constant values independent of the order. For example, note the variation in the $a_{0}$ term.}
\label{tab:amplitudes}
\end{table}%

\subsubsection{Specific solution}
For a specific solution, the decomposition for $d=7$ was chosen. (This is the final line in table \ref{tab:amplitudes}.)
The most dominant modes are $a_{0}$, the mean, $a_{4}$, corresponding to $\cos 4\alpha$, and, finally, $a_{2}$, corresponding to $\cos 2\alpha$.
\begin{figure}[htbp]
	\begin{center}
		\includegraphics[ scale = 1 ]{\pathgraphics bar-chart-07}
	\end{center}
\caption{The amplitudes for the $d=7$ decomposition, the final line of table \ref{tab:amplitudes}.}
\label{fig:sample-data}
\end{figure}

The functional representation of the RCS in terms of the yaw angle $\alpha$ is
%
\begin{equation}
	\boxed{
	\begin{split}
\sigma(\alpha) \approx &\, 35.2373 + 1.67502 \cos \alpha - 3.43429 \cos {\color{blue}{2}} \alpha \\
		%
	&-0.8664 \cos {\color{blue}{3}} \alpha+5.36808 \cos {\color{blue}{4}} \alpha-1.27952 \cos {\color{blue}{5}} \alpha \\
		%
	& + 1.37883 \cos {\color{blue}{6}} \alpha -0.366535 \cos {\color{blue}{7}} \alpha,
	\end{split}}
\label{eq:fourier-expansion}
\end{equation}
and this is the function to use with the AFCAP Dashboard.

\subsection{Fauxthogonality}
Due to problems with the mesh having a supernumerary point, the orthogonality relationship for the cosine vectors is destroyed, and we must solve a dense linear system. 

\subsubsection{Dense linear system}
Using the MoM data, the product matrix is not diagonal; off--diagonal terms have unit magnitude. Of course, you can't solve this linear system by pretending the off--diagonal terms are zero. It is a dense linear system whose 
\begin{equation}
		%
	\mathbf{A}^{*}\mathbf{A} = 
		%
	\mat{rrrrrrrr}{
		 361 & -1 & 1 & -1 & 1 & -1 & 1 & -1\\
		 -1 & 181 & -1 & 1 & -1 & 1 & -1 & 1\\
		 1 & -1 & 181 & -1 & 1 & -1 & 1 & -1 \\
		 -1 & 1 & -1 & 181 & -1 & 1 & -1 & 1 \\
		 1 & -1 & 1 & -1 & 181 & -1 & 1 & -1 \\
		 -1 & 1 & -1 & 1 & -1 & 181 & -1 & 1 \\
		 1 & -1 & 1 & -1 & 1 & -1 & 181 & -1 \\
		 -1 & 1 & -1 & 1 & -1 & 1 & -1 & 181 }
		%
\end{equation}
This matrix has full rank, and the inverse can be computed directly and the result is detailed in the final line of table \ref{tab:amplitudes}.
\begin{equation}
		%
	\paren{\mathbf{A}^{*}\mathbf{A}}^{-1} = 
		%
	\frac{1}{67500}
	\mat{rrrrrrrr}{
		 187 & 1 & -1 & 1 & -1 & 1 & -1 & 1 \\
		1 & 373 & 2 & -2 & 2 & -2 & 2 & -2 \\
		 -1 & 2 & 373 & 2 & -2 & 2 & -2 & 2 \\
		 1 & -2 & 2 & 373 & 2 & -2 & 2 & -2 \\
		 -1 & 2 & -2 & 2 & 373 & 2 & -2 & 2 \\
		 1 & -2 & 2 & -2 & 2 & 373 & 2 & -2 \\
		 -1 & 2 & -2 & 2 & -2 & 2 & 373 & 2 \\
		 1 & -2 & 2 & -2 & 2 & -2 & 2 & 373 }
		%
\end{equation}

\subsubsection{Decoupled linear system}
If the mesh is corrected, the orthogonality of the vectors will manifest with a decoupled linear system: terms can be computed mode--by--mode.
\begin{equation}
		%
	\mathbf{A}^{*}\mathbf{A} = 
		%
	\mat{cccccccc}{
	 360 & 0 & 0 & 0 & 0 & 0 & 0 & 0 \\
	 0 & 180 & 0 & 0 & 0 & 0 & 0 & 0 \\
	 0 & 0 & 180 & 0 & 0 & 0 & 0 & 0 \\
	 0 & 0 & 0 & 180 & 0 & 0 & 0 & 0 \\
	 0 & 0 & 0 & 0 & 180 & 0 & 0 & 0 \\
	 0 & 0 & 0 & 0 & 0 & 180 & 0 & 0 \\
	 0 & 0 & 0 & 0 & 0 & 0 & 180 & 0 \\
	 0 & 0 & 0 & 0 & 0 & 0 & 0 & 180 }
	 	%
\end{equation}
%
%\begin{equation}
%		%
%	\paren{\mathbf{A}^{*}\mathbf{A}}^{-1} = 
%		%
%	\frac{1}{360}
%	\mat{rrrrrrrr}{
%	 1 & 0 & 0 & 0 & 0 & 0 & 0 & 0 \\
%	 0 & 2 & 0 & 0 & 0 & 0 & 0 & 0 \\
%	 0 & 0 & 2 & 0 & 0 & 0 & 0 & 0 \\
%	 0 & 0 & 0 & 2 & 0 & 0 & 0 & 0 \\
%	 0 & 0 & 0 & 0 & 2 & 0 & 0 & 0 \\
%	 0 & 0 & 0 & 0 & 0 & 2 & 0 & 0 \\
%	 0 & 0 & 0 & 0 & 0 & 0 & 2 & 0 \\
%	 0 & 0 & 0 & 0 & 0 & 0 & 0 & 2 }
%		%
%\end{equation}


\endinput  %  ==  ==  ==  ==  ==  ==  ==  ==  ==
