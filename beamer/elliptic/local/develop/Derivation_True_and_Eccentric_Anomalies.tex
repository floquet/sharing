\documentclass[11pt]{article}
\usepackage{amsmath, amssymb, graphicx}
\usepackage[a4paper,margin=1in]{geometry}
\usepackage{hyperref}

\title{Derivation of Relations Between True and Eccentric Anomalies}
\author{Achates (ChatGPT)}
\date{\today}

\begin{document}
\maketitle

\section*{Introduction}
This document outlines the starting points and key relationships necessary for re-deriving Equation (1) and Equation (3) from the paper, which relate the true anomaly (\(v\)) and eccentric anomaly (\(E\)).

\section*{Key Relationships}
\subsection*{1. Tangent Relation Between True and Eccentric Anomaly}
\begin{equation}
\tan\frac{v}{2} = \sqrt{\frac{1 + e}{1 - e}} \tan\frac{E}{2},
\end{equation}
where \(e\) is the orbital eccentricity.

\subsection*{2. Radius Vector in Terms of \(E\) and \(v\)}
\begin{itemize}
    \item Using the \textbf{eccentric anomaly}:
    \[
    r = a(1 - e \cos E),
    \]
    \item Using the \textbf{true anomaly}:
    \[
    r = \frac{a(1 - e^2)}{1 + e \cos v}.
    \]
\end{itemize}

\subsection*{3. Sine and Cosine Relations for True Anomaly}
From the geometry of the ellipse:
\[
\cos v = \frac{\cos E - e}{1 - e \cos E}, \quad \sin v = \frac{\sqrt{1 - e^2} \sin E}{1 - e \cos E}.
\]

\section*{Equation (1): Relating \(v\) and \(E\)}
\subsection*{Starting Point}
Consider the difference in angles:
\[
\tan\frac{v - E}{2}.
\]
Using the tangent addition formula:
\[
\tan\frac{v - E}{2} = \frac{\sin(v - E)}{1 + \cos(v - E)}.
\]

\subsection*{Expand Sine and Cosine Terms}
Expand \(\sin(v - E)\) and \(\cos(v - E)\) using:
\[
\sin(v - E) = \sin v \cos E - \cos v \sin E,
\]
\[
\cos(v - E) = \cos v \cos E + \sin v \sin E.
\]

\subsection*{Substitute Known Relations}
Substitute the expressions for \(\sin v\), \(\cos v\), \(\sin E\), and \(\cos E\) derived earlier.

\subsection*{Simplify}
Simplify the result to:
\[
\tan\frac{v - E}{2} = \frac{\sin v - \sin E}{2 + \cos v - \cos E}.
\]

\section*{Equation (3): \(v - M\) as a Function of \(E\) or \(v\)}
\subsection*{Starting Point}
From Kepler’s equation:
\[
M = E - e \sin E,
\]
express \(v - M\) as:
\[
v - M = v - (E - e \sin E).
\]

\subsection*{Substitute for \(v\)}
Using the tangent formula:
\[
v = 2 \tan^{-1}\left(\sqrt{\frac{1 + e}{1 - e}} \tan\frac{E}{2}\right),
\]
expand \(v - M\).

\subsection*{Simplify Using Trigonometric Identities}
Simplify terms involving:
\[
\tan^{-1}(x) + \tan^{-1}(y) = \tan^{-1}\left(\frac{x + y}{1 - xy}\right),
\]
and expressions for \(\sin E\), \(\cos E\), \(\sin v\), and \(\cos v\).

\section*{Conclusion}
These steps establish a clear starting point for deriving the formulas. Further simplifications should match the results in the paper.

\end{document}
