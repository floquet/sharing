\documentclass{beamer}
\usepackage{amsmath}
\usepackage{graphicx}
\usetheme{Madrid}

% Title Information
\title{Thoughts on Function Signatures}
\subtitle{Building Understanding from 1D to 3D}
\author{Achates \\ \small{Thoughts Shared with Daniel Topa}}
\date{\today}

\begin{document}

% Slide 1: Establish the Foundation
\begin{frame}{1. Establish the Foundation}
    \begin{itemize}
        \item **Function Signatures** describe how data is modeled and represented.
        \item Foundation: Start with **Fourier (1D)**, extend to **Zernike (2D)**, and then **Spherical Harmonics (3D)**.
    \end{itemize}
    \vspace{0.3cm}
    \begin{center}
        \begin{tabular}{|l|l|l|}
            \hline
            \textbf{Domain} & \textbf{Basis Functions} & \textbf{Example Input} \\
            \hline
            1D              & Sines/Cosines           & \( f(x) \) \\
            2D (disk)       & Zernike Polynomials     & \( f(x, y) \) \\
            3D (sphere)     & Spherical Harmonics     & \( f(\theta, \phi) \) \\
            \hline
        \end{tabular}
    \end{center}
\end{frame}

% Slide 2: Trivial Example - Constant Sphere
\begin{frame}{2. Trivial Example: Constant Sphere}
    \begin{itemize}
        \item Input: \( f(\theta, \phi) = 1 \)
        \item Output: Only \( c_{00} \neq 0 \) in the SH expansion.
    \end{itemize}
    \vspace{0.3cm}
    \begin{center}
        \includegraphics[width=0.4\textwidth]{uniform_sphere.png} % Placeholder image
    \end{center}
    \vspace{0.3cm}
    \begin{tabular}{|c|c|c|}
        \hline
        \( l \) & \( m \) & \( c_{lm} \) \\
        \hline
        0       & 0       & 1.0         \\
        1       & -1, 0, 1 & 0.0        \\
        2       & -2..2   & 0.0         \\
        \hline
    \end{tabular}
\end{frame}

% Slide 3: Perturbed Sphere - Higher Modes
\begin{frame}{3. Perturbed Sphere: Higher Modes}
    \begin{itemize}
        \item A small localized **bump** perturbs the sphere.
        \item Higher \( l \)-modes are excited in the SH expansion.
    \end{itemize}
    \vspace{0.3cm}
    \begin{center}
        \includegraphics[width=0.4\textwidth]{bumped_sphere.png} % Placeholder image
    \end{center}
    \vspace{0.3cm}
    \begin{tabular}{|c|c|c|c|}
        \hline
        \( l \) & \( m \) & \( c_{lm} \) (Uniform) & \( c_{lm} \) (Bumped) \\
        \hline
        0       & 0       & 1.0                  & 0.95                  \\
        1       & -1, 0, 1 & 0.0                 & Small nonzero         \\
        2       & -2..2   & 0.0                  & Larger contributions  \\
        \hline
    \end{tabular}
\end{frame}

% Slide 4: Connecting to Visuals
\begin{frame}{4. Connecting to Visuals}
    \begin{itemize}
        \item Pick an angle \( (\theta, \phi) \), e.g., \( (90^\circ, 0^\circ) \).
        \item Value \( f(\theta, \phi) \) determines the color/height at that point.
        \item **Higher SH coefficients \( c_{lm} \)** add detail to the surface.
    \end{itemize}
    \vspace{0.3cm}
    \begin{center}
        \includegraphics[width=0.4\textwidth]{rcs_visual.png} % Placeholder image
    \end{center}
\end{frame}

% Slide 5: Function Signature
\begin{frame}{5. Function Signature}
    \[
    \sigma(\theta, \phi) \approx \sum_{l=0}^L \sum_{m=-l}^l c_{lm} Y_{lm}(\theta, \phi)
    \]
    \begin{itemize}
        \item **Inputs**: Angles \( (\theta, \phi) \).
        \item **Outputs**: Scalar values \( f(\theta, \phi) \) (RCS, etc.).
        \item **Spherical Harmonics Coefficients \( c_{lm} \)** describe the function.
    \end{itemize}
\end{frame}

\end{document}
