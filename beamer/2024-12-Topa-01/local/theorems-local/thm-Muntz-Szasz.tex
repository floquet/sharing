% \input{\pLocalTheorems "thm-Muntz-Szasz"}

%
%\begin{equation}
%	1, \lst{x^{\lambda}}_{k=1}^{\infty}, \qquad \starto{k}
%\label{eq:full set}
%\end{equation}
%%
%Cull the population. For example, even powers
%%
%\begin{equation*}
%	%\begin{split}
%		a = b
%	%\end{split}
%\end{equation*}
%%
%\begin{equation*}
%	K=\lst{2,4,6,8,\dots}
%\end{equation*}
%%
%\begin{equation}
%	1, \lst{x^{\lambda_{k}}}, \qquad \starto{k}
%\label{eq:culled set}
%\end{equation}
%%
% The “full müntz theorem” revisited
% Erdélyi, Tamás
% Constructive approximation, 21:319–335, 2005
% Theorem 1.A
% https://people.tamu.edu/~terdelyi//papers-online/EC
\begin{theorem}[Spans of $C\brac{0,1}$]

Suppose $\paren{\lambda_{j}}_{j=0}^{\infty}$ is a sequence with $0 = \lambda_{0} <\lambda_{1} < \lambda_{2} < \cdots$. The the space $\lst{x^{\lambda_{0}}, x^{\lambda_{1}},\dots}$ is dense in $C\brac{0,1}$ if an only if 
\begin{equation}
	\sum_{j=1}^{\infty} \frac{1}{j} = \infty
\end{equation} 
Let $\lambda_{k}$ be a sequence of numbers which grows without bound. The set of functions like \eqref{eq:culled set} is a span of $C\brac{0,1}$, that vanish at $x=0$ iff 
%
\begin{equation*}
	\sum_{k\in K} \frac{1}{\lambda_{k}} = \infty
\end{equation*}
%
\label{thm:muntz}
\end{theorem}
%\theorembreak
\begin{proof}
%
\cite[p. 89]{lax2002functional}
%
\end{proof}

Example: fail span
%
\begin{equation*}
	K=\lst{1,2,4,8,16,\dots} = \lst{j^{2n}}_{n=0}^{\infty}
\end{equation*}
%
%
\begin{equation*}
	\sum_{n=0}^{\infty} j^{-2n} = 1 + \frac{1}{4} + \frac{1}{4} + \frac{1}{8} + \dots = 2
\end{equation*}
%
Example: successful span
%
\begin{equation*}
	K=\lst{1,2,4,6,8,\dots} = \lst{2n}_{n=1}^{\infty}
\end{equation*}
%
%
\begin{equation*}
	\sum_{n=1}^{\infty} \frac{1}{2n} = 
	\frac{1}{2} + \frac{1}{4} + \frac{1}{6} + \dots \to \infty
\end{equation*}
Integral test reveals logarithmic divergence.


\endinput %   =   =   =   =   =   =   =   =   =   =   =   =   =   =