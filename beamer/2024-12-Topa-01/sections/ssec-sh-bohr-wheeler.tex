% \input{\pSections "ssec-sh-bohr-wheeler.tex"}

%     %     %     %     %     %     %     %     %
\begin{frame}\frametitle{Comparison of Liquid Drop Models}
\begin{table}[htp]
%\caption{default}
\begin{center}
\begin{tabular}{cc}bb
	Constant Radius & Variable Radius \\\hline
	\ \\
	\includegraphics[ width= 2in ]{\pLocalGraphics liquid-drop-sphere} & 
	\includegraphics[ width= 2in ]{\pLocalGraphics liquid-drop-balls} 
\end{tabular}
\end{center}
%\label{default}
\end{table}%
\end{frame}
%
\begin{frame}\frametitle{Historical Survey}
\center
	\href{\reedBook}{
	\begin{overpic}[ scale = 0.105 ]
	{\pLocalGraphics Manhattan}
	\end{overpic}}
\center
\footnotesize{
Appendix E: Formal Derivation of the \\\bl{Bohr–Wheeler Spontaneous Fission Limit}.} 
\end{frame}
%
\begin{frame}\frametitle{Summary of Liquid Drop Model\jumpLittle}
\begin{enumerate}
	\item Nucleus is liquid drop of \bl{variable radius}
	\item Fluid is incompressible
	\item Volume is conserved
\end{enumerate}
\end{frame}


%%%   %%%   %%%   %%%
\subsection{The Mechanism of Nuclear Fission}
\begin{frame}\frametitle{1939 Paper on \href{https://en.wikipedia.org/wiki/Nuclear\_fission}{Nuclear Fission}}
\center
	\href{\bohrPaper}{
	\begin{overpic}[ scale = 0.8]
	{\pLocalGraphics paper-bohr-wheeler-a}
		%\put(24,43){$Z=92$}
	\put(85.5,28) {\includegraphics[ width = 1cm ]{\pLocalGraphics medal}}
	\end{overpic}}
\end{frame}


\begin{frame}\frametitle{Deformations from Sphericity}
\center
	\href{\bohrPaper}{\begin{overpic}[ scale = 0.9]
	{\pLocalGraphics bw-01a}
	\end{overpic}}
\end{frame}

\begin{frame}\frametitle{Bohr-Wheeler Derivation}
\center
	\href{\bohrPaper}{\begin{overpic}[ scale = 0.75]
	{\pLocalGraphics bw-formula-a}
	\end{overpic}}
\end{frame}

\begin{frame}\frametitle{Express Radius as a Function of Polar Angle}
%
\begin{equation}
	\mg{\xcancel{\bk{r = R_{0}}}}
%\label{eq:name}
\end{equation}
$$ \Downarrow $$
\begin{equation}
	r(\theta) = R_{0}\paren{\bwexpa}
\label{eq:bw-radius}
\end{equation}
\end{frame}

\begin{frame}\frametitle{Express Radius as a Function of Polar Angle\jumpLittle}
%
Adjust $ \azero$, $ \atwo$ so that \rd{volume is conserved}:
\begin{equation}
	r(\theta) = R_{0}\paren{\bwexpa}
\tag{\ref{eq:bw-radius}}
\end{equation}
\end{frame}


\endinput  %  ==  ==  ==  ==  ==  ==  ==  ==  ==
