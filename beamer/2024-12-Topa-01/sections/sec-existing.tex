% \input{\pSections "sec-existing.tex"}

\section{Custom Software Tools}
\begin{frame}\frametitle{Sprawling Toolset: Languages}
\begin{enumerate}
	\item Fortran
	\item Python
	\item Mathematica
	\item Shell scripts
\end{enumerate}
\end{frame}

\begin{frame}\frametitle{Sprawling Toolset: Purposes}
\begin{enumerate}
	\item Automation
	\item Conversions
	\item Data Analysis
	\item Diagnostics
\end{enumerate}
\end{frame}

\begin{frame}\frametitle{Sprawling Toolset: Design}
\begin{enumerate}
	\item Object oriented
	\item Emphasis on error tracking
	\item Some crude
	\item Some refined
\end{enumerate}
\end{frame}

%%%   %%%   %%%   %%%

%%%   %%%   %%%   %%%
\subsection{Python}
\renewcommand{\listingFontSize}{\tiny} 
\begin{frame}[ fragile, allowframebreaks ]{Python Tool for \texttt{*.obj} to \texttt{*.facet}}
    \lstset{style=python} % Use the Python style
    \lstinputlisting{\pLocalCode/Obj2Facet_Python3.py}
\end{frame}

%%%   %%%   %%%   %%%
\subsection{Fortran}
\begin{frame}[ allowframebreaks ]{Major Fortran Tools}
\begin{enumerate}
	\item \texttt{aeneas.f08}
	\item \texttt{createFacetFile.f08}
	\item \texttt{esjufjoll.08}
	\item \texttt{facimusFacet.f08}
	\item \texttt{facet-maker.f08}
	\item \texttt{geo-writer.f08}
	\item \texttt{harvestRCSfromMoM.f08}
	\item \texttt{json-writer.f08}
	\item \texttt{gather.f08}
	\item \texttt{revised-reader.f08}
	\item \texttt{shaeffer.f08}
	\item \texttt{sigma.f08}
\end{enumerate}
\end{frame}

\lstdefinestyle{terminal}{
    backgroundcolor=\color{white}, % White background
    basicstyle=\ttfamily\tiny\color{black}, % Monospace font, black text
    breaklines=true, % Allow line breaks
    frame=single, % Optional: add a border
    numbers=left, % Show line numbers
    numberstyle=\tiny\color{gray}, % Smaller, gray line numbers
}
\begin{frame}[ fragile, allowframebreaks ]{\texttt{esjufjollf.f08} Execution}
\scriptsize
\lstinputlisting[
    style=terminal,
    linerange={72-114},
    breaklines=true,
    caption={Excerpt from \texttt{esjufjoll.f08}},
    label={lst:esjufjoll execution}
]{\pLocalCode/esjufjoll.f08}
\end{frame}

\begin{frame}[ fragile, allowframebreaks ]{\texttt{facet-maker.f08} Execution}
\scriptsize
\lstinputlisting[
    style=terminal,
    linerange={91-101},
    breaklines=true,
    caption={Excerpt from \texttt{facet-maker.f08}},
    label={lst:facet-maker execution}
]{\pLocalCode/facet-maker.f08}
\end{frame}

\begin{frame} [fragile, allowframebreaks ]{\texttt{gather.f08} Execution}
\scriptsize
\begin{lstlisting}[style=terminal, breaklines=true, caption={Excerpt from \texttt{gather.f08}}, label={lst:gather-execution}]
! dantopa:3d/xylorimba % ./gather                                                                                                                             (master)fortran-alpha
! 1. file = PTW-elev-0n179.4112.txt, -179, elevation angle = 0.
! 2. file = PTW-elev-0n178.4112.txt, -178, elevation angle = 0.
! 3. file = PTW-elev-0n177.4112.txt, -177, elevation angle = 0.

! 358. file = PTW-elev-0p178.4112.txt, 178, elevation angle = 0.
! 359. file = PTW-elev-0p179.4112.txt, 179, elevation angle = 0.
! 360. file = PTW-elev-0p180.4112.txt, 180, elevation angle = 0.
!
! completed at 2020-05-12 21:49:23
\end{lstlisting}
\end{frame}

\begin{frame}[fragile, allowframebreaks]{\texttt{revised-reader.f08} Excerpts}
\scriptsize
\lstinputlisting[
    style=terminal,
    linerange={39-45},
    breaklines=true,
    caption={Excerpt from \texttt{revised-reader.f08} (lines 39-45)},
    label={lst:revised-reader-39-45}
]{\pLocalCode/revised-reader.f08}

\lstinputlisting[
    style=terminal,
    linerange={123-127},
    breaklines=true,
    caption={Excerpt from \texttt{revised-reader.f08} (lines 123-127)},
    label={lst:revised-reader-123-127}
]{\pLocalCode/revised-reader.f08}
\end{frame}


\begin{frame}[fragile, allowframebreaks]{\texttt{sigma.f08} Overview}
\scriptsize
\begin{lstlisting}[style=terminal, breaklines=true, caption={Excerpt from \texttt{sigma.f08}}, label={lst:sigma-overview}]
! nb: /Users/dantopa/Mathematica_files/nb/ert/mercury/snake/fortran-01.nb
program rcs

! Read the Mercury Methods of Moments processed into a table of mean total RCS values
! Use the method of least squares to find
!   RCS ( yaw angle )           radar frequency fixed
!   RCS ( radar frequency )     yaw angle fixed

! Daniel Topa, ERT Corp
! COVID-19 Prisoner

! Class structure
!   RCStable: table of mean total RCS ( nu, alpha )
!   LinearSystem: Sytem Matrix A, data vector b
!        flavors: Fourier, monomial
!        tied to RCStable
!   LeastSquaresResults:
!        amplitudes
!        errors
!        residual error vector
!        tied to linear system
\end{lstlisting}
\end{frame}

%%%   %%%   %%%   %%%
\subsection{Mathematica}
% nb: /Users/dantopa/Mathematica_files/nb/projects/hii-tsd/radar/obj/obj-02.nb
\renewcommand{\listingFontSize}{\tiny} 
\begin{frame}[fragile,allowframebreaks]{Mathematica Commands}
    \lstset{style=mathematica} % Use the Mathematica style
    
    \begin{lstlisting}
    (* Generate a mesh for a convex hull of points on a sphere *)
    mesh = ConvexHullMesh[SpherePoints[60]] // Region;

    (* Extract vertex coordinates from the mesh *)
    vlist = MeshCoordinates[mesh];

    (* Label each point in the mesh *)
    labeledPoints = MapIndexed[
        Text[Style[ToString[#2[[1]]], Bold, Black], #1] &, vlist
    ];
    \end{lstlisting}
\end{frame}

\endinput  %  ==  ==  ==  ==  ==  ==  ==  ==  ==