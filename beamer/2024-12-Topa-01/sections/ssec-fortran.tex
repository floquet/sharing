% \input{\pSections/"ssec-fortran.tex"}

%     %     %     %     %     %     %     %     %
\subsection{Fortran Codes}
		%
%\begin{frame}[fragile]{Module \texttt{mOBJreader}: Dependencies}
%    \lstinputlisting[
%        style=fortran,
%        firstline=2,
%        lastline=12,
%        caption={Excerpt from \texttt{m\_obj\_reader.f08}},
%        label={lst:m_obj_reader_excerpt}
%    ]{\pLocalCode/m_obj_reader.f08}
%\end{frame}
%		%
%\begin{frame}[fragile]{Module \texttt{mOBJreader}: Read}
%    \lstinputlisting[
%        style=fortran,
%        firstline=34,
%        lastline=60,
%        caption={Excerpt from \texttt{a\_obj\_reader.f08}},
%        label={lst:a_obj_reader_excerpt}
%    ]{\pLocalCode/a_obj_reader.f08}
%\end{frame}

% Title and Metadata
%\title{Static Code Analysis of \texttt{facimusFacet}}
%\subtitle{A Tool for Mesh Analysis and Mercury MoM Preparation}
%\author{Daniel + Achates \\ (Static Code Function and Method Analysis)}
%\date{\today}

% Slide 2: Purpose
\begin{frame}{Purpose of \texttt{facimusFacet}}
    \begin{itemize}
        \item Converts \bl{\texttt{*.obj}} mesh decompositions into \bl{\texttt{*.facet}} files for Mercury MoM.
        \item Serves as a research tool to understand:
            \begin{itemize}
                \item How some meshes fail during analysis.
                \item Why others succeed.
            \end{itemize}
        \item Highlights object-oriented design and robust error handling.
    \end{itemize}
\end{frame}

% Slide 3: Code Overview
\begin{frame}{Code Overview}
    \begin{itemize}
        \item \bl{Inputs:} OBJ file containing vertex and facet data.
        \item \bl{Outputs:} 
            \begin{itemize}
                \item Face areas, nearest neighbors, mesh statistics.
                \item Several debug and analysis files.
            \end{itemize}
        \item \bl{Execution Tracking:}
            \begin{itemize}
                \item CPU timing, execution date, and compiler version.
            \end{itemize}
    \end{itemize}
\end{frame}

\begin{frame}[fragile]{Class Hierarchy and Relationships 1/2}
\begin{description}[leftmargin=2cm]
    \item[\textbf{Model}] \textit{(Base Object)}
        \begin{itemize}
            \item Contains allocatable list of \textbf{Vertex} objects
            \item Contains allocatable list of \textbf{Face} objects
            \item Contains a \textbf{Census} object
        \end{itemize}
        
    \item[\textbf{Vertex}] \textit{(Class)}  
        \begin{itemize}
            \item Position vectors: $\mathbb{R}^3$ (real)
            \item Color indices: $\mathbb{Z}^3$ (integer)
        \end{itemize}
\end{description}
\end{frame}

\begin{frame}[fragile]{Class Hierarchy and Relationships 2/2}
\begin{description}[leftmargin=2cm]
    \item[\textbf{Face}] \textit{(Class)}  
        \begin{itemize}
            \item Vertex indices: $\mathbb{Z}^3$ (integer)
            \item Area: (real)
        \end{itemize}
        
    \item[\textbf{Census}] \textit{(Class)}  
        \begin{itemize}
            \item Contains a \textbf{Lines} object
        \end{itemize}
        
    \item[\textbf{Lines}] \textit{(Class)}  
        \begin{itemize}
            \item Line numbers for vertices
            \item Line numbers for normals
            \item Line numbers for faces
        \end{itemize}
\end{description}
\end{frame}

% Slide 4: Module Breakdown
\begin{frame}{Module Breakdown}
    The program leverages the following modules:
    \begin{itemize}
        \item \bl{mClassModel:} Core type for vertices, faces, and statistics.
        \item \bl{mAllocations:} Robust memory allocation and hygiene.
        \item \bl{mClassFace:} Handles face-related computations.
        \item \bl{mClassVertex:} Manages vertex-related operations.
        \item \bl{mProcessControl:} Drives the overall process and calls subroutines.
    \end{itemize}
\end{frame}

% Slide 5: Key Features
\begin{frame}{Key Features}
    \begin{itemize}
        \item \bl{Error Handling:} 
        \begin{itemize}
            \item Memory allocation checks.
            \item Clear error messages with detailed context.
        \end{itemize}
        \item \bl{Object-Oriented Design:} 
        \begin{itemize}
		\item \bl{Class over Type:} Encapsulates data \emph{and} methods
            	\item \bl{Methods:} e.g. \texttt{computeFaceArea}, \texttt{findNearestNeighbor}.
        		\item \bl{Operations:} tied directly to Face, Vertex, and Census objects
        		\item \bl{Maintainability:} Keeps mesh logic clean, extensible, and avoids scattered code
    	\end{itemize}
    \end{itemize}
\end{frame}

% Slide 6: Flowchart
%\begin{frame}{Program Flowchart}
%    \begin{center}
%        \begin{tikzpicture}[node distance=2cm, auto]
%            % Nodes
%            \node[draw, rectangle, fill=blue!20] (input) {Input: OBJ file};
%            \node[draw, rectangle, below of=input, fill=green!20] (alloc) {Allocate Memory};
%            \node[draw, rectangle, below of=alloc, fill=yellow!20] (process) {Process Faces and Vertices};
%            \node[draw, rectangle, below of=process, fill=red!20] (error) {Error Handling Checks};
%            \node[draw, rectangle, below of=error, fill=orange!20] (output) {Output: Statistics, Face Areas, Debug Files};
%
%            % Arrows
%            \draw[->] (input) -- (alloc);
%            \draw[->] (alloc) -- (process);
%            \draw[->] (process) -- (error);
%            \draw[->] (error) -- (output);
%        \end{tikzpicture}
%    \end{center}
%\end{frame}

%\begin{frame}{Program Flowchart}
%    \begin{center}
%        \resizebox{0.4\textwidth}{!}{ % Rescale diagram to fit slide
%        \begin{tikzpicture}[node distance=1.5cm, auto] % Reduce spacing slightly
%            % Nodes
%            \node[draw, rectangle, fill=blue!20, font=\bfseries] (input) {Input: OBJ file};
%            \node[draw, rectangle, below of=input, fill=green!20, font=\bfseries] (alloc) {Allocate Memory};
%            \node[draw, rectangle, below of=alloc, fill=yellow!20, font=\bfseries] (process) {Process Faces and Vertices};
%            \node[draw, rectangle, below of=process, fill=red!20, font=\bfseries] (error) {Error Handling Checks};
%            \node[draw, rectangle, below of=error, fill=orange!20] (output) {Output: Statistics, Face Areas, Debug Files};
%            
%            % Arrows
%            \draw[->, thick] (input) -- (alloc);
%            \draw[->, thick] (alloc) -- (process);
%            \draw[->, thick] (process) -- (error);
%            \draw[->, thick] (error) -- (output);
%        \end{tikzpicture}}
%    \end{center}
%\end{frame}
 
 
%\begin{frame}{Object Model Hierarchy}
%    \begin{center}
%    \begin{tikzpicture}[
%        node distance=1.5cm, % Vertical node spacing
%        every node/.style={rectangle, draw, text centered, font=\bfseries, minimum width=2.5cm},
%        edge from parent/.style={draw, ->, thick}
%    ]
%        % Root node
%        \node[fill=blue!20] (cad) {CAD Model};
%
%        % First level
%        \node[below=of cad, fill=green!20] (faces) {Faces};
%        \node[right=of faces, fill=yellow!20] (vertices) {Vertices};
%        \node[left=of faces, fill=red!20] (census) {Census};
%
%        % Second level
%        \node[below=of faces, fill=pink!20] (location) {Location Vectors};
%        \node[below=of vertices, fill=orange!20] (color) {Color Attributes};
%
%        % Connections
%        \draw[->] (cad) -- (faces);
%        \draw[->] (cad) -- (vertices);
%        \draw[->] (cad) -- (census);
%        \draw[->] (faces) -- (location);
%        \draw[->] (vertices) -- (color);
%    \end{tikzpicture}
%    \end{center}
%\end{frame}
 
% Slide 7: Pseudocode
\begin{frame}[fragile]{High-Level Pseudocode}
\begin{algorithm}[H]
    \caption{\textit{facimusFacet Program Overview}}
    \begin{algorithmic}[1]
        \State \textbf{Initialize} CPU timer
        \State \textbf{Call} \textit{main\_process\_sub}:
        \Statex \hspace{2em} \textbf{Parse} input \texttt{OBJ} file
        \Statex \hspace{2em} \textbf{Allocate} memory for vertices and faces
        \Statex \hspace{2em} \textbf{Compute} face areas and vertex statistics
        \Statex \hspace{2em} \textbf{Identify} nearest and farthest neighbors
        \Statex \hspace{2em} \textbf{Write} results to output files
        \State \textbf{Record} CPU time and date
        \State \textbf{Print} compiler version and options
    \end{algorithmic}
\end{algorithm}
\end{frame}

%\begin{frame}{Object Model Hierarchy}
%    \begin{center}
%    \begin{tikzpicture}[scale = 0.8, 
%        node distance=1.5cm and 3.5cm, % Adjust horizontal and vertical spacing
%        every node/.style={rectangle, draw, text centered, font=\bfseries, minimum width=3cm, minimum height=1cm},
%        level 1/.style={sibling distance=4.5cm, level distance=2cm}, % Space out first level
%        level 2/.style={sibling distance=2.5cm, level distance=2cm}  % Space out second level
%    ]
%        % Root node
%        \node[fill=blue!20] (cad) {CAD Model}
%            % First level nodes
%            child { node[fill=red!20] {Census} }
%            child { node[fill=yellow!20] {Faces}
%                % Second level node for Faces
%                child { node[fill=orange!20] {Location Vectors} }
%            }
%            child { node[fill=green!20] {Vertices}
%                % Second level node for Vertices
%                child { node[fill=pink!20] {Color Attributes} }
%            };
%    \end{tikzpicture}
%    \end{center}
%\end{frame}

%\begin{frame}[fragile]{Class Relationships (Pseudocode)}
%\begin{lstlisting}[basicstyle=\ttfamily\small,frame=single]
%class Vertex:
%    has Position: vector
%    has Color: attributes
%    has Indices: references
%
%class Face:
%    has PointNumbers: list
%    has Area: real
%
%class Census:
%    has Lines:
%        - FirstLine: int
%        - LastLine: int
%        - LineCount: int
%    has Indices: references to Vertex, Normals, Faces
%
%class CADModel:
%    contains Vertices: Vertex[]
%    contains Faces: Face[]
%    contains Census: Census
%\end{lstlisting}
%\end{frame}
%
%\begin{frame}{Object Model: Is-A vs Has-A}
%    \begin{center}
%    \begin{tikzpicture}[
%        scale=0.8,
%        node distance=1.5cm and 3.5cm, % Adjust node distances
%        every node/.style={
%            rectangle, draw, text centered, font=\bfseries, minimum width=3cm, minimum height=1cm
%        },
%        level 1/.style={sibling distance=4.5cm, level distance=2cm},
%        level 2/.style={sibling distance=2.5cm, level distance=2cm},
%        dashed-line/.style={dashed, ->, thick} % Style for has-a relationships
%    ]
%        % Root node
%        \node[fill=blue!20] (cad) {CAD Model}
%            child { node[fill=red!20] (census) {Census} }
%            child { node[fill=yellow!20] (faces) {Faces}
%                child { node[fill=orange!20] (location) {Location Vectors} }
%            }
%            child { node[fill=green!20] (vertices) {Vertices}
%                child { node[fill=pink!20] (color) {Color Attributes} }
%            };
%
%        % Dashed lines for "has-a" relationships
%        \draw[dashed-line] (faces) -- (vertices);
%        \draw[dashed-line] (census) -- (vertices);
%    \end{tikzpicture}
%    \end{center}
%\end{frame}

%\begin{frame}{Object Model Hierarchy}
%\begin{center}
%\begin{tikzpicture}
%  \umlclass{Model}{
%    - myVertex : Vertex [ ]
%    \\
%    - myFace : Face [ ]
%    \\
%    - myCensus : Census
%  }{
%    + analyze_lengths()
%    \\
%    + compute_centroid()
%  }
%  
%  \umlclass[y=-4]{Face}{
%    - indexVertex : Integer[ ]
%  }{
%    + computeFaceArea()
%    \\
%    + writeFace()
%  }
%  
%  \umlclass[x=-4, y=-4]{Vertex}{
%    - position : Real[ ]
%    \\
%    - color : Color
%  }{
%    + someVertexMethod()
%  }
%  
%  \umlclass[x=4, y=-4]{Census}{
%    - objPointers : Pointer[ ]
%  }{
%    + managePointers()
%  }
%  
%  % Relationships
%  \umlassoc[mult1=1..*, mult2=1..*]{Model}{Vertex}
%  \umlassoc[mult1=1..*, mult2=1..*]{Model}{Face}
%  \umlaggreg[mult1=1, mult2=1]{Model}{Census}
%  \umlcompo[mult1=1, mult2=3]{Face}{Vertex}
%\end{tikzpicture}
%\end{center}
%\end{frame}

% Slide 8: Summary
\begin{frame}{Summary and Next Steps}
    \begin{itemize}
        \item \bl{Purpose:} \bl{OBJ} to \bl{FACET} conversion for Mercury MoM.
        \item \bl{Features:} 
            \begin{itemize}
                \item Object-oriented programming.
                \item Robust error handling.
                \item Modularity and reusability.
            \end{itemize}
        \item \bl{Next Steps for Rewrite:}
            \begin{itemize}
                \item Focus on maintaining modularity.
                \item Retain robust error handling features.
                \item Highlight mesh success/failure analysis in design.
            \end{itemize}
    \end{itemize}
\end{frame}

% Slide 9: Attribution
\begin{frame}{Attribution}
    \begin{itemize}
        \item Static Code Function and Method Analysis.
        \item Code reviewed, analyzed, and presented with insights.
        \item Collaborative contributions by:  
        \bl{Daniel \& Achates}.
    \end{itemize}
\end{frame}

\begin{frame}\frametitle{Reading \texttt{*.obj} Files}
\begin{enumerate}
	\item Modelo: \href{https://www.modelo.io/damf/article/2024/07/09/1755/understanding-the-obj-format--a-comprehensive-guide?hl=en}{Understanding OBJ Format: A Comprehensive Guide}
	\item Paul Bourke: \href{http://paulbourke.net/dataformats/obj/}{Object Files (.obj)}
	\item Wikipedia: \href{https://en.wikipedia.org/wiki/Wavefront_.obj_file}{Wavefront .obj file}
	\item Library of Congress: \href{https://www.loc.gov/preservation/digital/formats/fdd/fdd000507.shtml}{Wavefront OBJ File Format}
	\item https://www.cs.utah.edu/~boulos/cs3505/obj\_spec.pdf
\end{enumerate}
	
\endinput  %  ==  ==  ==  ==  ==  ==  ==  ==  ==
