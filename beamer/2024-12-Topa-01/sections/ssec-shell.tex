% \input{\pSections/"ssec-shell.tex"}

%     %     %     %     %     %     %     %     %
\subsection{Shell Scripts}

\begin{frame}{Running Mercury MoM}
\begin{itemize}
    \item \bl{Shell Execution Command:}
    \begin{center}
        \texttt{\bl{\$ ./MMoM\_4112 example.geo}}
    \end{center}
    \vspace{0.5cm}

    \item \bl{What This Does:}
    \begin{itemize}
        \item \bl{Input:} Uses geometry file \texttt{example.geo}.
        \item \bl{Output:} Simulation results in \texttt{../outputs/example.out}.
    \end{itemize}

    \vspace{0.5cm}

    \item \bl{Key Insight:}
    \begin{itemize}
        \item \textit{While this execution is simple, real-world use requires automation:}
        \begin{itemize}
            \item Running multiple simulations (sweeps).
            \item Handling large geometries and outputs.
            \item Organizing results systematically.
        \end{itemize}
    \end{itemize}
\end{itemize}
\end{frame}

\begin{frame}{Understanding the Outputs}
\begin{itemize}
    \item \textbf{Runtime Information:}
    \begin{center}
        optional \texttt{\textbf{> logs/panel-elev-0p180.out}}.
    \end{center}
    \vspace{0.3cm}
    \item \textbf{Simulation Results:} (MBytes)
    \begin{center}
        Written to \texttt{\textbf{results/example.4112.txt}}.
    \end{center}
    \vspace{0.3cm}
    \item \textbf{Key Insight:}  
    Simple execution commands are powerful when combined with automation:
    \begin{itemize}
        \item Sweeping parameters (e.g., elevation angles).  
        \item Logging runs systematically.  
        \item Managing and analyzing outputs efficiently.  
    \end{itemize}
\end{itemize}
\end{frame}

\begin{frame}{Workflow Overview}
\begin{itemize}
    \item \bl{Input Files:}
            \item \bl{*.geo}: Points to *.facet, *.geo.
        \begin{itemize}
            \item \bl{*.facet}: Mesh model.
            \item \bl{*.lib}: Material properties.
        \end{itemize}
    \item \bl{Simulation Outputs:}
        \begin{itemize}
            \item \bl{*.4112.txt}: Numerical simulation results.
        \end{itemize}
\end{itemize}
\end{frame}

\begin{frame}{Script Purpose Summary}
\begin{center}
\begin{tabular}{ll}
    \bl{Script}            & \bl{Purpose}    \\ \hline
    \texttt{create-geos.sh}    & Automate batch run \texttt{*.geo} template \\
    \texttt{run-MoM-example.sh}& Run a single simulation \\
    \texttt{run-MoM-timer.sh}  & Measure simulation performance \\
    \texttt{elevation-sweep.sh}& didactic 3D sweep $\paren{\theta, \phi}$  \\
    \texttt{harvest.sh}        & alternative 3D sweep $\paren{\theta, \phi}$  \\
\end{tabular}
\end{center}
\end{frame}
%
%\begin{frame}{Script Purpose Summary}
%\begin{center}
%\begin{tabular}{|l|l|l|}
%    \hline
%    \bl{Script}            & \bl{Purpose}                  & \bl{Outputs}       \\ \hline
%    \texttt{create-geos.sh}    & Generate \texttt{*.geo} files     & \texttt{*.geo}         \\ \hline
%    \texttt{run-MoM-example.sh}& Run a single simulation           & \texttt{*.4112}        \\ \hline
%    \texttt{run-MoM-timer.sh}  & Measure simulation performance    & Timed logs             \\ \hline
%    \texttt{elevation-sweep.sh}& Sweep angles, automate runs       & Multiple \texttt{*.4112} \\ \hline
%    \texttt{harvest.sh}        & Consolidate simulation results    & Structured data        \\ \hline
%\end{tabular}
%\end{center}
%\end{frame}

%\begin{frame}{Workflow Diagram}
%\begin{center}
%\begin{tikzpicture}[
%    node distance=1.5cm,
%    every node/.style={rectangle, draw, text centered, font=\bfseries, minimum width=2.5cm, minimum height=1cm},
%    arrow/.style={thick, ->, >=stealth}
%]
%
%% Nodes
%\node[fill=gray!20] (facet) {\bl{*.facet}};
%\node[fill=gray!20, below of=facet, yshift=-1.2cm] (lib) {\bl{*.lib}};
%\node[fill=blue!20, right=2.5cm of facet] (geo) {\bl{*.geo}};
%\node[fill=red!20, right=2.5cm of geo] (output) {\bl{*.4112}};
%\node[fill=yellow!20, below of=output, yshift=-1.2cm] (harvest) {\bl{Results}};
%
%% Arrows
%\draw[arrow] (facet) -- (geo);
%\draw[arrow] (lib) -- (geo);
%\draw[arrow] (geo) -- (output);
%\draw[arrow] (output) -- (harvest);
%
%% Labels
%\node[above of=geo, yshift=0.5cm] {\texttt{create-geos.sh}};
%\node[above of=output, yshift=0.5cm] {\texttt{run-MoM-example.sh}};
%\node[right of=output, xshift=1.5cm] {\texttt{elevation-sweep.sh}};
%\node[below of=harvest, yshift=-0.5cm] {\texttt{harvest.sh}};
%\end{tikzpicture}
%\end{center}
%\end{frame}

\begin{frame}{Key Insights}
\begin{itemize}
    \item \bl{Automation:} Eliminates manual geometry setup and simulation runs.
    \item \bl{Consistency:} Uniform outputs and structured logs across runs.
    \item \bl{Scalability:} Parameter sweeps enable 3D studies.
    \item \bl{Performance:} \texttt{run-MoM-timer.sh} benchmarks computational efficiency.
    \item \bl{Workflow:}
    \begin{itemize}
        \item \bl{Input:} \texttt{*.geo}, \texttt{*.facet}, \texttt{*.lib}.
        \item \bl{Output:} \texttt{*.4112}.
        \item \bl{Harvest:} Complex backscatter energy.
    \end{itemize}
\end{itemize}
\end{frame}


\endinput  %  ==  ==  ==  ==  ==  ==  ==  ==  ==
